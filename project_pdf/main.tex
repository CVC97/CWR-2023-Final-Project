\documentclass{class}
\usepackage{packages}                           

%**************************Layout**************************************
\pagestyle{fancy}
\fancyhf{}
\fancyhead[L]{\sc Computergestütztes wissenschaftliches Rechnen}   % Obere Leiste
\fancyfoot[C]{\thepage}                                 % Untere Leiste
\setlength{\headheight}{15pt} % mindesetens 13.59999pt
\renewcommand{\headrulewidth}{1pt}
\renewcommand{\footrulewidth}{1pt}

\setlength\parindent{0pt} %noindent

%\numberwithin{equation}{section} %Gleichungen nach Sections numeriert

%**********************************************************************
\begin{document}

\begin{titlepage}
    \centering
    \vspace*{1.5cm}
    \begin{figure}[H]
        \includegraphics[width=1\textwidth]{images/Logo_Uni_Göttingen_2022.png}
    \end{figure}
    \vspace*{2.2cm}
    
    \rule{\textwidth}{1pt}\\[0.5cm]
    {\huge \bfseries
      Project Name}\\
    \rule{\textwidth}{1pt}
    
    \vspace*{2.7cm}
    
    {\large Carlo von Carnap\\
    
    \vspace*{0.7cm}

    Sommersemester 2023}
    
    \vspace*{4.7cm}

    {\large Abschlussprojekt Computergestütztes wissenschaftliches Rechnen, \\
    Salvatore R. Manmana

    \vspace*{0.7cm}
    
    Betreuer: Ruben Haag}

    \end{titlepage}
\newpage

% Die folgenden drei Zeilen beim Drucken.

% \thispagestyle{empty} 
% \mbox{}
% \newpage

\tableofcontents 
\thispagestyle{empty}
\newpage
\setcounter{page}{1}

\input{1_Einleitung.tex}

\input{2_Theorie.tex}

\input{3_Algorithmen.tex}

\input{4_Umsetzung}

\input{5_Auswertung_Diskussion.tex}

%**********************************************************************

\addcontentsline{toc}{section}{Literaturverzeichnis}
\printbibliography[title = {Literatur}]%,nottype=online]

%\addcontentsline{toc}{section}{Onlinequellen}
%\printbibliography[title=Onlinequellen,type=online]

%\newpage

%\addcontentsline{toc}{section}{Abbildungsverzeichnis}
%\listoffigures

%\addcontentsline{toc}{section}{Tabellenverzeichnis}
%\listoftables

\end{document}