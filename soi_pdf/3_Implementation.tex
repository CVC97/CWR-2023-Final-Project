\section{Implementation}

\subsection{Usage of the Libraries \texttt{cvc\_numerics.h} and \texttt{cvc\_rng.h}}

\subsection{Structure and Workflow of the Main Program}

\subsubsection*{Generation of Random Numbers by a Static RNG}

For the generation of pseudorandom numbers the in \prettyref{ssec:methodology_rng_mt19937} described Mersenne\,Twister \texttt{MT19937} has been used. It was implemented statically and can therefore be accessed
globally by the respective functions.

\subsubsection*{Implementation of the Modelling Grid}

The above mentioned grid itself is realized as a $\left(L+2\right)\times \left(L+2\right)$ heap section of integer values, with $L$ being the sidelength of the quadratic grid where the actual spread of the infection takes place.
While this section is technically one-dimensional, it will for simplicity reasons be here refered to as a two-dimensional structure of the given shape. Inside the grid, the following integer values have been used 
to model the different states of the people within the simulation:
\begin{itemize}
    \item \texttt{0}: this person is Susceptible \susceptible{} to the infection
    \item \texttt{1}: the person is Infected \infected{}
    \item \texttt{2}: the person is Recovered \recovered{} and currently not susceptible
    \item \texttt{-1}: the person is Vaccinated \vaccinated{} and does not participate in the spread
\end{itemize}
The grid has been implemented with an edge of ghosts at the top, bottom, left and right border, that are neither infectious nor subject to any updates of the grid --- they will permanently take the value \texttt{0}.

\subsubsection*{Functions acting on the Modelling Grid}

\subsubsection*{Structure of the Main Function and General Workflow}

\subsection{Naming of Variables}