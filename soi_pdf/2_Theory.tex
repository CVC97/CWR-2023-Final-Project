\section{Theory}

\subsection{Averaging within Simulation Runs}

Fundamental to the anaylsis of the simulation results is the rate of infected individuals in the grid at each timestep $\langle I\rangle_t$. It is calculated by summing up the infection status $I_{i,j}$
over the totality of the grid and dividing it by the number of grid nodes $L^2$,
\begin{equation}
    \langle I\rangle_t=\frac{1}{L^2}\sum_{i=1,\,j=1}^{L}I_{i,\,j}.
    \label{eq:theo_inf_rate}
\end{equation}
For that, the infection status is either taking $I_{i,j}=1$ for the person being in state \infected{} or $I_{i,j}=0$ otherwise. This infection rate can now be averaged over all timesteps $t$ to receive
\begin{equation}
    \overline{\langle I\rangle}=\frac{1}{T}\sum_{t=0}^{T}\langle I\rangle_t,
    \label{eq:theo_time_avg_inf_rate}
\end{equation}
the time-averaged infection rate which most of the analysis steps are based on.

\subsection{Averaging across Simulation Runs}

For the evaluation of the time-averaging, the infection rate $\langle I\rangle_t$ was also averaged over a multitude of $N$ samples for each timestep. This mean at a given timestep $t$ is calculated as
\begin{equation}
    \overline{\langle I\rangle_t}=\frac{1}{N}\sum_{n=1}^{N}\langle I\rangle_t
    \label{eq:theo_mean_inf_rate}
\end{equation}
and posesses a standard deviation of
\begin{equation}
    \sigma_{\langle I\rangle_t}=\frac{1}{N}\sum_{n=1}^{N}\left(\langle I\rangle_t-\overline{\langle I\rangle_t}\,\right)^2.
    \label{eq:theo_sigma_inf_rate}
\end{equation}