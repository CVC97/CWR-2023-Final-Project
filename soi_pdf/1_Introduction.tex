\section{Introduction}

% \subsection{Infectious Disease Modelling}

This model uses a quadratic grid of size $L\times L$ to examine the spread of an infectious agent. Each grid node is occupied by an immobile person and can be in one of the states
Susceptible \susceptible{}, Infected \infected{} or Recovered \recovered{}. Later, the forth state Vaccinated \vaccinated{} is introduced, that permanently excludes the respective people from taking any of the three other states. For a person 
in one of the three active states \susceptible{}, \infected{} and \recovered{}, the following three transitions are possible,
\begin{equation*}
    \mathrm{Susceptible}~\susceptible\xlongrightarrow{p_1}\mathrm{Infected}~\infected{}\xlongrightarrow{p_2}\mathrm{Recovered}~\recovered{}\xlongrightarrow{p_3}\mathrm{Susceptible}~\susceptible{}.
\end{equation*}
With that, we have the three transition probabilities $p_1$, $p_2$ and $p_3$:
\begin{itemize}
    \item $p_1$: a susceptible person getting infected by a direct infected neighbor
    \item $p_2$: an infected person turning recovered
    \item $p_3$: a recovered person returning susceptible
\end{itemize}
Vaccinated people \vaccinated{} are set at the beginning of the simulation, with $p_4$ being the probability that a spot is occupied by one. The other three states are initializes with the same likelyhood. Each simulation step does
now