\section{Results and Discussion}

% \subsection{Model for the Spread of Infectious Diseases}

% The model was implemented in the previously described way and operates mainly by using a stochastic updating method for each timestep (i.e. $L^2$ random nodes are selected for actualization). 
% However, it is not expected to see a significant difference between the actualization schemes.

\subsection{Ratio of Infected People in the Grid Averaged over Time}

Aim was now to analyse the infection rate inside the grid averaged over all timesteps $\overline{\langle I\rangle}$ for different combinations of $p_1$, $p_2$ and $p_2$. To achive this, the dependency of the 
turnover probability $p_1\left(\susceptible\rightarrow\infected\,\right)$ on the time-averaged infection rates was simulated for different combinations of $p_2\left(\susceptible\rightarrow\infected\,\right)$
and $p_3\left(\susceptible\rightarrow\infected\,\right)$. The simulation was performed over $T=1000$ simulation steps for each of the grid sizes $L=16$, $L=32$, $L=64$ and $L=128$.
%, the results are shown in \prettyref{fig:Res_Dis_Avg_Inf_over_p1}. 
The latter grid size $L=128$ is the largest one used for any simulation and still takes an acceptable but still non-negligable amount of calculation time.
Since this side length should provide more than fluctuation-resistant results and the calculation time more than doubles for a doubling in $L$, this was the maximum grid size trialed.

From looking at the results shown in \prettyref{fig:Res_Dis_Avg_Inf_over_p1} it immediately becomes visible that the lower grid sizes undergo stronger fluctuations in $\overline{\langle I\rangle}$ than the higher ones, 
with very strong deviations for $L=16$, particularly with $p_2=0.6$ and $p_3=0.3$. It is worth noting that the lower grid sizes only deviate towards smaller time-averaged infection rates and never towards higher ones. 
This can be explained by the underlying reason behind the deviations, which is not in fact primarily attributed to statistical fluctuations --- those are kept to a minimum due to averaging over time --- but
rather a secondary effect of the stochastic methods: the spread came to a halt after the number of infected individuals had fluctuated to zero at some point during the simulation. The occurance of these zeroed runs does
obviously happen more frequently with smaller grids, which can be confirmed using the manual updater at the beginning of the program and operating it with even lower grid sizes such as $L=8$.
Since the final time-averaged infection rate also correlates with the timestep where the spread eventually stopped, and earlier halts are more likely with lower infection rates, stronger outliers follow as more frequent 
for low infection rates as well. Consequently the simulation run with $p_2=0.6$ and $p_3=0.3$ and therefore the lowest $\overline{\langle I\rangle}$, also underwent the biggest changes from $L=16$ towards higher grid sizes.

\begin{figure}[ht]
    \centering
    \resizebox{\textwidth}{!}{%% Creator: Matplotlib, PGF backend
%%
%% To include the figure in your LaTeX document, write
%%   \input{<filename>.pgf}
%%
%% Make sure the required packages are loaded in your preamble
%%   \usepackage{pgf}
%%
%% Also ensure that all the required font packages are loaded; for instance,
%% the lmodern package is sometimes necessary when using math font.
%%   \usepackage{lmodern}
%%
%% Figures using additional raster images can only be included by \input if
%% they are in the same directory as the main LaTeX file. For loading figures
%% from other directories you can use the `import` package
%%   \usepackage{import}
%%
%% and then include the figures with
%%   \import{<path to file>}{<filename>.pgf}
%%
%% Matplotlib used the following preamble
%%   
%%   \usepackage{fontspec}
%%   \setmainfont{DejaVuSerif.ttf}[Path=\detokenize{/home/carlo/.local/lib/python3.10/site-packages/matplotlib/mpl-data/fonts/ttf/}]
%%   \setsansfont{DejaVuSans.ttf}[Path=\detokenize{/home/carlo/.local/lib/python3.10/site-packages/matplotlib/mpl-data/fonts/ttf/}]
%%   \setmonofont{DejaVuSansMono.ttf}[Path=\detokenize{/home/carlo/.local/lib/python3.10/site-packages/matplotlib/mpl-data/fonts/ttf/}]
%%   \makeatletter\@ifpackageloaded{underscore}{}{\usepackage[strings]{underscore}}\makeatother
%%
\begingroup%
\makeatletter%
\begin{pgfpicture}%
\pgfpathrectangle{\pgfpointorigin}{\pgfqpoint{10.427823in}{13.953546in}}%
\pgfusepath{use as bounding box, clip}%
\begin{pgfscope}%
\pgfsetbuttcap%
\pgfsetmiterjoin%
\definecolor{currentfill}{rgb}{1.000000,1.000000,1.000000}%
\pgfsetfillcolor{currentfill}%
\pgfsetlinewidth{0.000000pt}%
\definecolor{currentstroke}{rgb}{1.000000,1.000000,1.000000}%
\pgfsetstrokecolor{currentstroke}%
\pgfsetdash{}{0pt}%
\pgfpathmoveto{\pgfqpoint{0.000000in}{0.000000in}}%
\pgfpathlineto{\pgfqpoint{10.427823in}{0.000000in}}%
\pgfpathlineto{\pgfqpoint{10.427823in}{13.953546in}}%
\pgfpathlineto{\pgfqpoint{0.000000in}{13.953546in}}%
\pgfpathlineto{\pgfqpoint{0.000000in}{0.000000in}}%
\pgfpathclose%
\pgfusepath{fill}%
\end{pgfscope}%
\begin{pgfscope}%
\pgfsetbuttcap%
\pgfsetmiterjoin%
\definecolor{currentfill}{rgb}{1.000000,1.000000,1.000000}%
\pgfsetfillcolor{currentfill}%
\pgfsetlinewidth{0.000000pt}%
\definecolor{currentstroke}{rgb}{0.000000,0.000000,0.000000}%
\pgfsetstrokecolor{currentstroke}%
\pgfsetstrokeopacity{0.000000}%
\pgfsetdash{}{0pt}%
\pgfpathmoveto{\pgfqpoint{0.640323in}{9.767436in}}%
\pgfpathlineto{\pgfqpoint{10.327822in}{9.767436in}}%
\pgfpathlineto{\pgfqpoint{10.327822in}{13.617436in}}%
\pgfpathlineto{\pgfqpoint{0.640323in}{13.617436in}}%
\pgfpathlineto{\pgfqpoint{0.640323in}{9.767436in}}%
\pgfpathclose%
\pgfusepath{fill}%
\end{pgfscope}%
\begin{pgfscope}%
\pgfpathrectangle{\pgfqpoint{0.640323in}{9.767436in}}{\pgfqpoint{9.687500in}{3.850000in}}%
\pgfusepath{clip}%
\pgfsetbuttcap%
\pgfsetroundjoin%
\definecolor{currentfill}{rgb}{0.000000,0.000000,1.000000}%
\pgfsetfillcolor{currentfill}%
\pgfsetfillopacity{0.500000}%
\pgfsetlinewidth{1.003750pt}%
\definecolor{currentstroke}{rgb}{0.000000,0.000000,1.000000}%
\pgfsetstrokecolor{currentstroke}%
\pgfsetstrokeopacity{0.500000}%
\pgfsetdash{}{0pt}%
\pgfsys@defobject{currentmarker}{\pgfqpoint{-0.021960in}{-0.021960in}}{\pgfqpoint{0.021960in}{0.021960in}}{%
\pgfpathmoveto{\pgfqpoint{0.000000in}{-0.021960in}}%
\pgfpathcurveto{\pgfqpoint{0.005824in}{-0.021960in}}{\pgfqpoint{0.011410in}{-0.019646in}}{\pgfqpoint{0.015528in}{-0.015528in}}%
\pgfpathcurveto{\pgfqpoint{0.019646in}{-0.011410in}}{\pgfqpoint{0.021960in}{-0.005824in}}{\pgfqpoint{0.021960in}{0.000000in}}%
\pgfpathcurveto{\pgfqpoint{0.021960in}{0.005824in}}{\pgfqpoint{0.019646in}{0.011410in}}{\pgfqpoint{0.015528in}{0.015528in}}%
\pgfpathcurveto{\pgfqpoint{0.011410in}{0.019646in}}{\pgfqpoint{0.005824in}{0.021960in}}{\pgfqpoint{0.000000in}{0.021960in}}%
\pgfpathcurveto{\pgfqpoint{-0.005824in}{0.021960in}}{\pgfqpoint{-0.011410in}{0.019646in}}{\pgfqpoint{-0.015528in}{0.015528in}}%
\pgfpathcurveto{\pgfqpoint{-0.019646in}{0.011410in}}{\pgfqpoint{-0.021960in}{0.005824in}}{\pgfqpoint{-0.021960in}{0.000000in}}%
\pgfpathcurveto{\pgfqpoint{-0.021960in}{-0.005824in}}{\pgfqpoint{-0.019646in}{-0.011410in}}{\pgfqpoint{-0.015528in}{-0.015528in}}%
\pgfpathcurveto{\pgfqpoint{-0.011410in}{-0.019646in}}{\pgfqpoint{-0.005824in}{-0.021960in}}{\pgfqpoint{0.000000in}{-0.021960in}}%
\pgfpathlineto{\pgfqpoint{0.000000in}{-0.021960in}}%
\pgfpathclose%
\pgfusepath{stroke,fill}%
}%
\begin{pgfscope}%
\pgfsys@transformshift{1.080663in}{9.899391in}%
\pgfsys@useobject{currentmarker}{}%
\end{pgfscope}%
\begin{pgfscope}%
\pgfsys@transformshift{1.260394in}{9.900022in}%
\pgfsys@useobject{currentmarker}{}%
\end{pgfscope}%
\begin{pgfscope}%
\pgfsys@transformshift{1.440125in}{9.902399in}%
\pgfsys@useobject{currentmarker}{}%
\end{pgfscope}%
\begin{pgfscope}%
\pgfsys@transformshift{1.619856in}{9.905553in}%
\pgfsys@useobject{currentmarker}{}%
\end{pgfscope}%
\begin{pgfscope}%
\pgfsys@transformshift{1.799587in}{9.905262in}%
\pgfsys@useobject{currentmarker}{}%
\end{pgfscope}%
\begin{pgfscope}%
\pgfsys@transformshift{1.979318in}{9.913363in}%
\pgfsys@useobject{currentmarker}{}%
\end{pgfscope}%
\begin{pgfscope}%
\pgfsys@transformshift{2.159049in}{9.912830in}%
\pgfsys@useobject{currentmarker}{}%
\end{pgfscope}%
\begin{pgfscope}%
\pgfsys@transformshift{2.338780in}{9.914503in}%
\pgfsys@useobject{currentmarker}{}%
\end{pgfscope}%
\begin{pgfscope}%
\pgfsys@transformshift{2.518511in}{9.923842in}%
\pgfsys@useobject{currentmarker}{}%
\end{pgfscope}%
\begin{pgfscope}%
\pgfsys@transformshift{2.698242in}{9.949142in}%
\pgfsys@useobject{currentmarker}{}%
\end{pgfscope}%
\begin{pgfscope}%
\pgfsys@transformshift{2.877973in}{9.994186in}%
\pgfsys@useobject{currentmarker}{}%
\end{pgfscope}%
\begin{pgfscope}%
\pgfsys@transformshift{3.057704in}{10.385080in}%
\pgfsys@useobject{currentmarker}{}%
\end{pgfscope}%
\begin{pgfscope}%
\pgfsys@transformshift{3.237435in}{10.869287in}%
\pgfsys@useobject{currentmarker}{}%
\end{pgfscope}%
\begin{pgfscope}%
\pgfsys@transformshift{3.417166in}{10.938860in}%
\pgfsys@useobject{currentmarker}{}%
\end{pgfscope}%
\begin{pgfscope}%
\pgfsys@transformshift{3.596897in}{11.149381in}%
\pgfsys@useobject{currentmarker}{}%
\end{pgfscope}%
\begin{pgfscope}%
\pgfsys@transformshift{3.776628in}{11.263968in}%
\pgfsys@useobject{currentmarker}{}%
\end{pgfscope}%
\begin{pgfscope}%
\pgfsys@transformshift{3.956359in}{11.293029in}%
\pgfsys@useobject{currentmarker}{}%
\end{pgfscope}%
\begin{pgfscope}%
\pgfsys@transformshift{4.136090in}{11.472067in}%
\pgfsys@useobject{currentmarker}{}%
\end{pgfscope}%
\begin{pgfscope}%
\pgfsys@transformshift{4.315821in}{11.621192in}%
\pgfsys@useobject{currentmarker}{}%
\end{pgfscope}%
\begin{pgfscope}%
\pgfsys@transformshift{4.495552in}{11.671621in}%
\pgfsys@useobject{currentmarker}{}%
\end{pgfscope}%
\begin{pgfscope}%
\pgfsys@transformshift{4.675283in}{11.706842in}%
\pgfsys@useobject{currentmarker}{}%
\end{pgfscope}%
\begin{pgfscope}%
\pgfsys@transformshift{4.855014in}{11.764064in}%
\pgfsys@useobject{currentmarker}{}%
\end{pgfscope}%
\begin{pgfscope}%
\pgfsys@transformshift{5.034745in}{11.845157in}%
\pgfsys@useobject{currentmarker}{}%
\end{pgfscope}%
\begin{pgfscope}%
\pgfsys@transformshift{5.214476in}{11.890152in}%
\pgfsys@useobject{currentmarker}{}%
\end{pgfscope}%
\begin{pgfscope}%
\pgfsys@transformshift{5.394207in}{11.924113in}%
\pgfsys@useobject{currentmarker}{}%
\end{pgfscope}%
\begin{pgfscope}%
\pgfsys@transformshift{5.573938in}{11.921467in}%
\pgfsys@useobject{currentmarker}{}%
\end{pgfscope}%
\begin{pgfscope}%
\pgfsys@transformshift{5.753669in}{12.042022in}%
\pgfsys@useobject{currentmarker}{}%
\end{pgfscope}%
\begin{pgfscope}%
\pgfsys@transformshift{5.933400in}{12.024703in}%
\pgfsys@useobject{currentmarker}{}%
\end{pgfscope}%
\begin{pgfscope}%
\pgfsys@transformshift{6.113131in}{12.074163in}%
\pgfsys@useobject{currentmarker}{}%
\end{pgfscope}%
\begin{pgfscope}%
\pgfsys@transformshift{6.292862in}{12.109720in}%
\pgfsys@useobject{currentmarker}{}%
\end{pgfscope}%
\begin{pgfscope}%
\pgfsys@transformshift{6.472593in}{12.101331in}%
\pgfsys@useobject{currentmarker}{}%
\end{pgfscope}%
\begin{pgfscope}%
\pgfsys@transformshift{6.652324in}{12.120587in}%
\pgfsys@useobject{currentmarker}{}%
\end{pgfscope}%
\begin{pgfscope}%
\pgfsys@transformshift{6.832055in}{12.167985in}%
\pgfsys@useobject{currentmarker}{}%
\end{pgfscope}%
\begin{pgfscope}%
\pgfsys@transformshift{7.011786in}{12.193190in}%
\pgfsys@useobject{currentmarker}{}%
\end{pgfscope}%
\begin{pgfscope}%
\pgfsys@transformshift{7.191517in}{12.211112in}%
\pgfsys@useobject{currentmarker}{}%
\end{pgfscope}%
\begin{pgfscope}%
\pgfsys@transformshift{7.371248in}{12.236900in}%
\pgfsys@useobject{currentmarker}{}%
\end{pgfscope}%
\begin{pgfscope}%
\pgfsys@transformshift{7.550979in}{12.269427in}%
\pgfsys@useobject{currentmarker}{}%
\end{pgfscope}%
\begin{pgfscope}%
\pgfsys@transformshift{7.730710in}{12.262391in}%
\pgfsys@useobject{currentmarker}{}%
\end{pgfscope}%
\begin{pgfscope}%
\pgfsys@transformshift{7.910441in}{12.257274in}%
\pgfsys@useobject{currentmarker}{}%
\end{pgfscope}%
\begin{pgfscope}%
\pgfsys@transformshift{8.090172in}{12.295333in}%
\pgfsys@useobject{currentmarker}{}%
\end{pgfscope}%
\begin{pgfscope}%
\pgfsys@transformshift{8.269903in}{12.305741in}%
\pgfsys@useobject{currentmarker}{}%
\end{pgfscope}%
\begin{pgfscope}%
\pgfsys@transformshift{8.449634in}{12.347991in}%
\pgfsys@useobject{currentmarker}{}%
\end{pgfscope}%
\begin{pgfscope}%
\pgfsys@transformshift{8.629365in}{12.322861in}%
\pgfsys@useobject{currentmarker}{}%
\end{pgfscope}%
\begin{pgfscope}%
\pgfsys@transformshift{8.809096in}{12.368129in}%
\pgfsys@useobject{currentmarker}{}%
\end{pgfscope}%
\begin{pgfscope}%
\pgfsys@transformshift{8.988827in}{12.380766in}%
\pgfsys@useobject{currentmarker}{}%
\end{pgfscope}%
\begin{pgfscope}%
\pgfsys@transformshift{9.168558in}{12.397284in}%
\pgfsys@useobject{currentmarker}{}%
\end{pgfscope}%
\begin{pgfscope}%
\pgfsys@transformshift{9.348289in}{12.388913in}%
\pgfsys@useobject{currentmarker}{}%
\end{pgfscope}%
\begin{pgfscope}%
\pgfsys@transformshift{9.528020in}{12.391968in}%
\pgfsys@useobject{currentmarker}{}%
\end{pgfscope}%
\begin{pgfscope}%
\pgfsys@transformshift{9.707751in}{12.422924in}%
\pgfsys@useobject{currentmarker}{}%
\end{pgfscope}%
\begin{pgfscope}%
\pgfsys@transformshift{9.887482in}{12.423259in}%
\pgfsys@useobject{currentmarker}{}%
\end{pgfscope}%
\end{pgfscope}%
\begin{pgfscope}%
\pgfpathrectangle{\pgfqpoint{0.640323in}{9.767436in}}{\pgfqpoint{9.687500in}{3.850000in}}%
\pgfusepath{clip}%
\pgfsetbuttcap%
\pgfsetroundjoin%
\definecolor{currentfill}{rgb}{0.980392,0.164706,0.333333}%
\pgfsetfillcolor{currentfill}%
\pgfsetfillopacity{0.500000}%
\pgfsetlinewidth{1.003750pt}%
\definecolor{currentstroke}{rgb}{0.980392,0.164706,0.333333}%
\pgfsetstrokecolor{currentstroke}%
\pgfsetstrokeopacity{0.500000}%
\pgfsetdash{}{0pt}%
\pgfsys@defobject{currentmarker}{\pgfqpoint{-0.021960in}{-0.021960in}}{\pgfqpoint{0.021960in}{0.021960in}}{%
\pgfpathmoveto{\pgfqpoint{0.000000in}{-0.021960in}}%
\pgfpathcurveto{\pgfqpoint{0.005824in}{-0.021960in}}{\pgfqpoint{0.011410in}{-0.019646in}}{\pgfqpoint{0.015528in}{-0.015528in}}%
\pgfpathcurveto{\pgfqpoint{0.019646in}{-0.011410in}}{\pgfqpoint{0.021960in}{-0.005824in}}{\pgfqpoint{0.021960in}{0.000000in}}%
\pgfpathcurveto{\pgfqpoint{0.021960in}{0.005824in}}{\pgfqpoint{0.019646in}{0.011410in}}{\pgfqpoint{0.015528in}{0.015528in}}%
\pgfpathcurveto{\pgfqpoint{0.011410in}{0.019646in}}{\pgfqpoint{0.005824in}{0.021960in}}{\pgfqpoint{0.000000in}{0.021960in}}%
\pgfpathcurveto{\pgfqpoint{-0.005824in}{0.021960in}}{\pgfqpoint{-0.011410in}{0.019646in}}{\pgfqpoint{-0.015528in}{0.015528in}}%
\pgfpathcurveto{\pgfqpoint{-0.019646in}{0.011410in}}{\pgfqpoint{-0.021960in}{0.005824in}}{\pgfqpoint{-0.021960in}{0.000000in}}%
\pgfpathcurveto{\pgfqpoint{-0.021960in}{-0.005824in}}{\pgfqpoint{-0.019646in}{-0.011410in}}{\pgfqpoint{-0.015528in}{-0.015528in}}%
\pgfpathcurveto{\pgfqpoint{-0.011410in}{-0.019646in}}{\pgfqpoint{-0.005824in}{-0.021960in}}{\pgfqpoint{0.000000in}{-0.021960in}}%
\pgfpathlineto{\pgfqpoint{0.000000in}{-0.021960in}}%
\pgfpathclose%
\pgfusepath{stroke,fill}%
}%
\begin{pgfscope}%
\pgfsys@transformshift{1.080663in}{9.899119in}%
\pgfsys@useobject{currentmarker}{}%
\end{pgfscope}%
\begin{pgfscope}%
\pgfsys@transformshift{1.260394in}{9.901562in}%
\pgfsys@useobject{currentmarker}{}%
\end{pgfscope}%
\begin{pgfscope}%
\pgfsys@transformshift{1.440125in}{9.903036in}%
\pgfsys@useobject{currentmarker}{}%
\end{pgfscope}%
\begin{pgfscope}%
\pgfsys@transformshift{1.619856in}{9.903576in}%
\pgfsys@useobject{currentmarker}{}%
\end{pgfscope}%
\begin{pgfscope}%
\pgfsys@transformshift{1.799587in}{9.905353in}%
\pgfsys@useobject{currentmarker}{}%
\end{pgfscope}%
\begin{pgfscope}%
\pgfsys@transformshift{1.979318in}{9.910307in}%
\pgfsys@useobject{currentmarker}{}%
\end{pgfscope}%
\begin{pgfscope}%
\pgfsys@transformshift{2.159049in}{9.910531in}%
\pgfsys@useobject{currentmarker}{}%
\end{pgfscope}%
\begin{pgfscope}%
\pgfsys@transformshift{2.338780in}{9.931271in}%
\pgfsys@useobject{currentmarker}{}%
\end{pgfscope}%
\begin{pgfscope}%
\pgfsys@transformshift{2.518511in}{9.930488in}%
\pgfsys@useobject{currentmarker}{}%
\end{pgfscope}%
\begin{pgfscope}%
\pgfsys@transformshift{2.698242in}{9.961937in}%
\pgfsys@useobject{currentmarker}{}%
\end{pgfscope}%
\begin{pgfscope}%
\pgfsys@transformshift{2.877973in}{10.346506in}%
\pgfsys@useobject{currentmarker}{}%
\end{pgfscope}%
\begin{pgfscope}%
\pgfsys@transformshift{3.057704in}{10.660375in}%
\pgfsys@useobject{currentmarker}{}%
\end{pgfscope}%
\begin{pgfscope}%
\pgfsys@transformshift{3.237435in}{10.928155in}%
\pgfsys@useobject{currentmarker}{}%
\end{pgfscope}%
\begin{pgfscope}%
\pgfsys@transformshift{3.417166in}{11.155131in}%
\pgfsys@useobject{currentmarker}{}%
\end{pgfscope}%
\begin{pgfscope}%
\pgfsys@transformshift{3.596897in}{11.258044in}%
\pgfsys@useobject{currentmarker}{}%
\end{pgfscope}%
\begin{pgfscope}%
\pgfsys@transformshift{3.776628in}{11.367067in}%
\pgfsys@useobject{currentmarker}{}%
\end{pgfscope}%
\begin{pgfscope}%
\pgfsys@transformshift{3.956359in}{11.492161in}%
\pgfsys@useobject{currentmarker}{}%
\end{pgfscope}%
\begin{pgfscope}%
\pgfsys@transformshift{4.136090in}{11.565516in}%
\pgfsys@useobject{currentmarker}{}%
\end{pgfscope}%
\begin{pgfscope}%
\pgfsys@transformshift{4.315821in}{11.655749in}%
\pgfsys@useobject{currentmarker}{}%
\end{pgfscope}%
\begin{pgfscope}%
\pgfsys@transformshift{4.495552in}{11.729986in}%
\pgfsys@useobject{currentmarker}{}%
\end{pgfscope}%
\begin{pgfscope}%
\pgfsys@transformshift{4.675283in}{11.783159in}%
\pgfsys@useobject{currentmarker}{}%
\end{pgfscope}%
\begin{pgfscope}%
\pgfsys@transformshift{4.855014in}{11.813022in}%
\pgfsys@useobject{currentmarker}{}%
\end{pgfscope}%
\begin{pgfscope}%
\pgfsys@transformshift{5.034745in}{11.889425in}%
\pgfsys@useobject{currentmarker}{}%
\end{pgfscope}%
\begin{pgfscope}%
\pgfsys@transformshift{5.214476in}{11.921523in}%
\pgfsys@useobject{currentmarker}{}%
\end{pgfscope}%
\begin{pgfscope}%
\pgfsys@transformshift{5.394207in}{11.974063in}%
\pgfsys@useobject{currentmarker}{}%
\end{pgfscope}%
\begin{pgfscope}%
\pgfsys@transformshift{5.573938in}{12.006136in}%
\pgfsys@useobject{currentmarker}{}%
\end{pgfscope}%
\begin{pgfscope}%
\pgfsys@transformshift{5.753669in}{12.026697in}%
\pgfsys@useobject{currentmarker}{}%
\end{pgfscope}%
\begin{pgfscope}%
\pgfsys@transformshift{5.933400in}{12.072760in}%
\pgfsys@useobject{currentmarker}{}%
\end{pgfscope}%
\begin{pgfscope}%
\pgfsys@transformshift{6.113131in}{12.105355in}%
\pgfsys@useobject{currentmarker}{}%
\end{pgfscope}%
\begin{pgfscope}%
\pgfsys@transformshift{6.292862in}{12.138036in}%
\pgfsys@useobject{currentmarker}{}%
\end{pgfscope}%
\begin{pgfscope}%
\pgfsys@transformshift{6.472593in}{12.162440in}%
\pgfsys@useobject{currentmarker}{}%
\end{pgfscope}%
\begin{pgfscope}%
\pgfsys@transformshift{6.652324in}{12.186956in}%
\pgfsys@useobject{currentmarker}{}%
\end{pgfscope}%
\begin{pgfscope}%
\pgfsys@transformshift{6.832055in}{12.201580in}%
\pgfsys@useobject{currentmarker}{}%
\end{pgfscope}%
\begin{pgfscope}%
\pgfsys@transformshift{7.011786in}{12.224034in}%
\pgfsys@useobject{currentmarker}{}%
\end{pgfscope}%
\begin{pgfscope}%
\pgfsys@transformshift{7.191517in}{12.242893in}%
\pgfsys@useobject{currentmarker}{}%
\end{pgfscope}%
\begin{pgfscope}%
\pgfsys@transformshift{7.371248in}{12.274550in}%
\pgfsys@useobject{currentmarker}{}%
\end{pgfscope}%
\begin{pgfscope}%
\pgfsys@transformshift{7.550979in}{12.296240in}%
\pgfsys@useobject{currentmarker}{}%
\end{pgfscope}%
\begin{pgfscope}%
\pgfsys@transformshift{7.730710in}{12.299488in}%
\pgfsys@useobject{currentmarker}{}%
\end{pgfscope}%
\begin{pgfscope}%
\pgfsys@transformshift{7.910441in}{12.302909in}%
\pgfsys@useobject{currentmarker}{}%
\end{pgfscope}%
\begin{pgfscope}%
\pgfsys@transformshift{8.090172in}{12.331325in}%
\pgfsys@useobject{currentmarker}{}%
\end{pgfscope}%
\begin{pgfscope}%
\pgfsys@transformshift{8.269903in}{12.356039in}%
\pgfsys@useobject{currentmarker}{}%
\end{pgfscope}%
\begin{pgfscope}%
\pgfsys@transformshift{8.449634in}{12.350047in}%
\pgfsys@useobject{currentmarker}{}%
\end{pgfscope}%
\begin{pgfscope}%
\pgfsys@transformshift{8.629365in}{12.374147in}%
\pgfsys@useobject{currentmarker}{}%
\end{pgfscope}%
\begin{pgfscope}%
\pgfsys@transformshift{8.809096in}{12.396669in}%
\pgfsys@useobject{currentmarker}{}%
\end{pgfscope}%
\begin{pgfscope}%
\pgfsys@transformshift{8.988827in}{12.390826in}%
\pgfsys@useobject{currentmarker}{}%
\end{pgfscope}%
\begin{pgfscope}%
\pgfsys@transformshift{9.168558in}{12.424544in}%
\pgfsys@useobject{currentmarker}{}%
\end{pgfscope}%
\begin{pgfscope}%
\pgfsys@transformshift{9.348289in}{12.424190in}%
\pgfsys@useobject{currentmarker}{}%
\end{pgfscope}%
\begin{pgfscope}%
\pgfsys@transformshift{9.528020in}{12.436231in}%
\pgfsys@useobject{currentmarker}{}%
\end{pgfscope}%
\begin{pgfscope}%
\pgfsys@transformshift{9.707751in}{12.444862in}%
\pgfsys@useobject{currentmarker}{}%
\end{pgfscope}%
\begin{pgfscope}%
\pgfsys@transformshift{9.887482in}{12.469304in}%
\pgfsys@useobject{currentmarker}{}%
\end{pgfscope}%
\end{pgfscope}%
\begin{pgfscope}%
\pgfpathrectangle{\pgfqpoint{0.640323in}{9.767436in}}{\pgfqpoint{9.687500in}{3.850000in}}%
\pgfusepath{clip}%
\pgfsetbuttcap%
\pgfsetroundjoin%
\definecolor{currentfill}{rgb}{0.239216,0.478431,0.992157}%
\pgfsetfillcolor{currentfill}%
\pgfsetfillopacity{0.500000}%
\pgfsetlinewidth{1.003750pt}%
\definecolor{currentstroke}{rgb}{0.239216,0.478431,0.992157}%
\pgfsetstrokecolor{currentstroke}%
\pgfsetstrokeopacity{0.500000}%
\pgfsetdash{}{0pt}%
\pgfsys@defobject{currentmarker}{\pgfqpoint{-0.021960in}{-0.021960in}}{\pgfqpoint{0.021960in}{0.021960in}}{%
\pgfpathmoveto{\pgfqpoint{0.000000in}{-0.021960in}}%
\pgfpathcurveto{\pgfqpoint{0.005824in}{-0.021960in}}{\pgfqpoint{0.011410in}{-0.019646in}}{\pgfqpoint{0.015528in}{-0.015528in}}%
\pgfpathcurveto{\pgfqpoint{0.019646in}{-0.011410in}}{\pgfqpoint{0.021960in}{-0.005824in}}{\pgfqpoint{0.021960in}{0.000000in}}%
\pgfpathcurveto{\pgfqpoint{0.021960in}{0.005824in}}{\pgfqpoint{0.019646in}{0.011410in}}{\pgfqpoint{0.015528in}{0.015528in}}%
\pgfpathcurveto{\pgfqpoint{0.011410in}{0.019646in}}{\pgfqpoint{0.005824in}{0.021960in}}{\pgfqpoint{0.000000in}{0.021960in}}%
\pgfpathcurveto{\pgfqpoint{-0.005824in}{0.021960in}}{\pgfqpoint{-0.011410in}{0.019646in}}{\pgfqpoint{-0.015528in}{0.015528in}}%
\pgfpathcurveto{\pgfqpoint{-0.019646in}{0.011410in}}{\pgfqpoint{-0.021960in}{0.005824in}}{\pgfqpoint{-0.021960in}{0.000000in}}%
\pgfpathcurveto{\pgfqpoint{-0.021960in}{-0.005824in}}{\pgfqpoint{-0.019646in}{-0.011410in}}{\pgfqpoint{-0.015528in}{-0.015528in}}%
\pgfpathcurveto{\pgfqpoint{-0.011410in}{-0.019646in}}{\pgfqpoint{-0.005824in}{-0.021960in}}{\pgfqpoint{0.000000in}{-0.021960in}}%
\pgfpathlineto{\pgfqpoint{0.000000in}{-0.021960in}}%
\pgfpathclose%
\pgfusepath{stroke,fill}%
}%
\begin{pgfscope}%
\pgfsys@transformshift{1.080663in}{9.899819in}%
\pgfsys@useobject{currentmarker}{}%
\end{pgfscope}%
\begin{pgfscope}%
\pgfsys@transformshift{1.260394in}{9.900851in}%
\pgfsys@useobject{currentmarker}{}%
\end{pgfscope}%
\begin{pgfscope}%
\pgfsys@transformshift{1.440125in}{9.902135in}%
\pgfsys@useobject{currentmarker}{}%
\end{pgfscope}%
\begin{pgfscope}%
\pgfsys@transformshift{1.619856in}{9.903297in}%
\pgfsys@useobject{currentmarker}{}%
\end{pgfscope}%
\begin{pgfscope}%
\pgfsys@transformshift{1.799587in}{9.905530in}%
\pgfsys@useobject{currentmarker}{}%
\end{pgfscope}%
\begin{pgfscope}%
\pgfsys@transformshift{1.979318in}{9.906911in}%
\pgfsys@useobject{currentmarker}{}%
\end{pgfscope}%
\begin{pgfscope}%
\pgfsys@transformshift{2.159049in}{9.914432in}%
\pgfsys@useobject{currentmarker}{}%
\end{pgfscope}%
\begin{pgfscope}%
\pgfsys@transformshift{2.338780in}{9.921424in}%
\pgfsys@useobject{currentmarker}{}%
\end{pgfscope}%
\begin{pgfscope}%
\pgfsys@transformshift{2.518511in}{9.928775in}%
\pgfsys@useobject{currentmarker}{}%
\end{pgfscope}%
\begin{pgfscope}%
\pgfsys@transformshift{2.698242in}{10.018474in}%
\pgfsys@useobject{currentmarker}{}%
\end{pgfscope}%
\begin{pgfscope}%
\pgfsys@transformshift{2.877973in}{10.446193in}%
\pgfsys@useobject{currentmarker}{}%
\end{pgfscope}%
\begin{pgfscope}%
\pgfsys@transformshift{3.057704in}{10.781520in}%
\pgfsys@useobject{currentmarker}{}%
\end{pgfscope}%
\begin{pgfscope}%
\pgfsys@transformshift{3.237435in}{11.000641in}%
\pgfsys@useobject{currentmarker}{}%
\end{pgfscope}%
\begin{pgfscope}%
\pgfsys@transformshift{3.417166in}{11.181789in}%
\pgfsys@useobject{currentmarker}{}%
\end{pgfscope}%
\begin{pgfscope}%
\pgfsys@transformshift{3.596897in}{11.302996in}%
\pgfsys@useobject{currentmarker}{}%
\end{pgfscope}%
\begin{pgfscope}%
\pgfsys@transformshift{3.776628in}{11.438634in}%
\pgfsys@useobject{currentmarker}{}%
\end{pgfscope}%
\begin{pgfscope}%
\pgfsys@transformshift{3.956359in}{11.531605in}%
\pgfsys@useobject{currentmarker}{}%
\end{pgfscope}%
\begin{pgfscope}%
\pgfsys@transformshift{4.136090in}{11.611542in}%
\pgfsys@useobject{currentmarker}{}%
\end{pgfscope}%
\begin{pgfscope}%
\pgfsys@transformshift{4.315821in}{11.691697in}%
\pgfsys@useobject{currentmarker}{}%
\end{pgfscope}%
\begin{pgfscope}%
\pgfsys@transformshift{4.495552in}{11.753551in}%
\pgfsys@useobject{currentmarker}{}%
\end{pgfscope}%
\begin{pgfscope}%
\pgfsys@transformshift{4.675283in}{11.807035in}%
\pgfsys@useobject{currentmarker}{}%
\end{pgfscope}%
\begin{pgfscope}%
\pgfsys@transformshift{4.855014in}{11.859116in}%
\pgfsys@useobject{currentmarker}{}%
\end{pgfscope}%
\begin{pgfscope}%
\pgfsys@transformshift{5.034745in}{11.908011in}%
\pgfsys@useobject{currentmarker}{}%
\end{pgfscope}%
\begin{pgfscope}%
\pgfsys@transformshift{5.214476in}{11.956993in}%
\pgfsys@useobject{currentmarker}{}%
\end{pgfscope}%
\begin{pgfscope}%
\pgfsys@transformshift{5.394207in}{11.990519in}%
\pgfsys@useobject{currentmarker}{}%
\end{pgfscope}%
\begin{pgfscope}%
\pgfsys@transformshift{5.573938in}{12.034204in}%
\pgfsys@useobject{currentmarker}{}%
\end{pgfscope}%
\begin{pgfscope}%
\pgfsys@transformshift{5.753669in}{12.060341in}%
\pgfsys@useobject{currentmarker}{}%
\end{pgfscope}%
\begin{pgfscope}%
\pgfsys@transformshift{5.933400in}{12.094457in}%
\pgfsys@useobject{currentmarker}{}%
\end{pgfscope}%
\begin{pgfscope}%
\pgfsys@transformshift{6.113131in}{12.126294in}%
\pgfsys@useobject{currentmarker}{}%
\end{pgfscope}%
\begin{pgfscope}%
\pgfsys@transformshift{6.292862in}{12.151679in}%
\pgfsys@useobject{currentmarker}{}%
\end{pgfscope}%
\begin{pgfscope}%
\pgfsys@transformshift{6.472593in}{12.183336in}%
\pgfsys@useobject{currentmarker}{}%
\end{pgfscope}%
\begin{pgfscope}%
\pgfsys@transformshift{6.652324in}{12.204020in}%
\pgfsys@useobject{currentmarker}{}%
\end{pgfscope}%
\begin{pgfscope}%
\pgfsys@transformshift{6.832055in}{12.222357in}%
\pgfsys@useobject{currentmarker}{}%
\end{pgfscope}%
\begin{pgfscope}%
\pgfsys@transformshift{7.011786in}{12.241806in}%
\pgfsys@useobject{currentmarker}{}%
\end{pgfscope}%
\begin{pgfscope}%
\pgfsys@transformshift{7.191517in}{12.267825in}%
\pgfsys@useobject{currentmarker}{}%
\end{pgfscope}%
\begin{pgfscope}%
\pgfsys@transformshift{7.371248in}{12.285156in}%
\pgfsys@useobject{currentmarker}{}%
\end{pgfscope}%
\begin{pgfscope}%
\pgfsys@transformshift{7.550979in}{12.303611in}%
\pgfsys@useobject{currentmarker}{}%
\end{pgfscope}%
\begin{pgfscope}%
\pgfsys@transformshift{7.730710in}{12.314826in}%
\pgfsys@useobject{currentmarker}{}%
\end{pgfscope}%
\begin{pgfscope}%
\pgfsys@transformshift{7.910441in}{12.333802in}%
\pgfsys@useobject{currentmarker}{}%
\end{pgfscope}%
\begin{pgfscope}%
\pgfsys@transformshift{8.090172in}{12.342788in}%
\pgfsys@useobject{currentmarker}{}%
\end{pgfscope}%
\begin{pgfscope}%
\pgfsys@transformshift{8.269903in}{12.363267in}%
\pgfsys@useobject{currentmarker}{}%
\end{pgfscope}%
\begin{pgfscope}%
\pgfsys@transformshift{8.449634in}{12.373190in}%
\pgfsys@useobject{currentmarker}{}%
\end{pgfscope}%
\begin{pgfscope}%
\pgfsys@transformshift{8.629365in}{12.393806in}%
\pgfsys@useobject{currentmarker}{}%
\end{pgfscope}%
\begin{pgfscope}%
\pgfsys@transformshift{8.809096in}{12.406996in}%
\pgfsys@useobject{currentmarker}{}%
\end{pgfscope}%
\begin{pgfscope}%
\pgfsys@transformshift{8.988827in}{12.417900in}%
\pgfsys@useobject{currentmarker}{}%
\end{pgfscope}%
\begin{pgfscope}%
\pgfsys@transformshift{9.168558in}{12.433250in}%
\pgfsys@useobject{currentmarker}{}%
\end{pgfscope}%
\begin{pgfscope}%
\pgfsys@transformshift{9.348289in}{12.442900in}%
\pgfsys@useobject{currentmarker}{}%
\end{pgfscope}%
\begin{pgfscope}%
\pgfsys@transformshift{9.528020in}{12.449694in}%
\pgfsys@useobject{currentmarker}{}%
\end{pgfscope}%
\begin{pgfscope}%
\pgfsys@transformshift{9.707751in}{12.460871in}%
\pgfsys@useobject{currentmarker}{}%
\end{pgfscope}%
\begin{pgfscope}%
\pgfsys@transformshift{9.887482in}{12.477637in}%
\pgfsys@useobject{currentmarker}{}%
\end{pgfscope}%
\end{pgfscope}%
\begin{pgfscope}%
\pgfpathrectangle{\pgfqpoint{0.640323in}{9.767436in}}{\pgfqpoint{9.687500in}{3.850000in}}%
\pgfusepath{clip}%
\pgfsetbuttcap%
\pgfsetroundjoin%
\definecolor{currentfill}{rgb}{0.000000,0.000000,0.000000}%
\pgfsetfillcolor{currentfill}%
\pgfsetfillopacity{0.500000}%
\pgfsetlinewidth{1.003750pt}%
\definecolor{currentstroke}{rgb}{0.000000,0.000000,0.000000}%
\pgfsetstrokecolor{currentstroke}%
\pgfsetstrokeopacity{0.500000}%
\pgfsetdash{}{0pt}%
\pgfsys@defobject{currentmarker}{\pgfqpoint{-0.021960in}{-0.021960in}}{\pgfqpoint{0.021960in}{0.021960in}}{%
\pgfpathmoveto{\pgfqpoint{0.000000in}{-0.021960in}}%
\pgfpathcurveto{\pgfqpoint{0.005824in}{-0.021960in}}{\pgfqpoint{0.011410in}{-0.019646in}}{\pgfqpoint{0.015528in}{-0.015528in}}%
\pgfpathcurveto{\pgfqpoint{0.019646in}{-0.011410in}}{\pgfqpoint{0.021960in}{-0.005824in}}{\pgfqpoint{0.021960in}{0.000000in}}%
\pgfpathcurveto{\pgfqpoint{0.021960in}{0.005824in}}{\pgfqpoint{0.019646in}{0.011410in}}{\pgfqpoint{0.015528in}{0.015528in}}%
\pgfpathcurveto{\pgfqpoint{0.011410in}{0.019646in}}{\pgfqpoint{0.005824in}{0.021960in}}{\pgfqpoint{0.000000in}{0.021960in}}%
\pgfpathcurveto{\pgfqpoint{-0.005824in}{0.021960in}}{\pgfqpoint{-0.011410in}{0.019646in}}{\pgfqpoint{-0.015528in}{0.015528in}}%
\pgfpathcurveto{\pgfqpoint{-0.019646in}{0.011410in}}{\pgfqpoint{-0.021960in}{0.005824in}}{\pgfqpoint{-0.021960in}{0.000000in}}%
\pgfpathcurveto{\pgfqpoint{-0.021960in}{-0.005824in}}{\pgfqpoint{-0.019646in}{-0.011410in}}{\pgfqpoint{-0.015528in}{-0.015528in}}%
\pgfpathcurveto{\pgfqpoint{-0.011410in}{-0.019646in}}{\pgfqpoint{-0.005824in}{-0.021960in}}{\pgfqpoint{0.000000in}{-0.021960in}}%
\pgfpathlineto{\pgfqpoint{0.000000in}{-0.021960in}}%
\pgfpathclose%
\pgfusepath{stroke,fill}%
}%
\begin{pgfscope}%
\pgfsys@transformshift{1.080663in}{9.899636in}%
\pgfsys@useobject{currentmarker}{}%
\end{pgfscope}%
\begin{pgfscope}%
\pgfsys@transformshift{1.260394in}{9.900663in}%
\pgfsys@useobject{currentmarker}{}%
\end{pgfscope}%
\begin{pgfscope}%
\pgfsys@transformshift{1.440125in}{9.901686in}%
\pgfsys@useobject{currentmarker}{}%
\end{pgfscope}%
\begin{pgfscope}%
\pgfsys@transformshift{1.619856in}{9.903556in}%
\pgfsys@useobject{currentmarker}{}%
\end{pgfscope}%
\begin{pgfscope}%
\pgfsys@transformshift{1.799587in}{9.904833in}%
\pgfsys@useobject{currentmarker}{}%
\end{pgfscope}%
\begin{pgfscope}%
\pgfsys@transformshift{1.979318in}{9.908198in}%
\pgfsys@useobject{currentmarker}{}%
\end{pgfscope}%
\begin{pgfscope}%
\pgfsys@transformshift{2.159049in}{9.913532in}%
\pgfsys@useobject{currentmarker}{}%
\end{pgfscope}%
\begin{pgfscope}%
\pgfsys@transformshift{2.338780in}{9.920951in}%
\pgfsys@useobject{currentmarker}{}%
\end{pgfscope}%
\begin{pgfscope}%
\pgfsys@transformshift{2.518511in}{9.939267in}%
\pgfsys@useobject{currentmarker}{}%
\end{pgfscope}%
\begin{pgfscope}%
\pgfsys@transformshift{2.698242in}{10.084847in}%
\pgfsys@useobject{currentmarker}{}%
\end{pgfscope}%
\begin{pgfscope}%
\pgfsys@transformshift{2.877973in}{10.523607in}%
\pgfsys@useobject{currentmarker}{}%
\end{pgfscope}%
\begin{pgfscope}%
\pgfsys@transformshift{3.057704in}{10.826707in}%
\pgfsys@useobject{currentmarker}{}%
\end{pgfscope}%
\begin{pgfscope}%
\pgfsys@transformshift{3.237435in}{11.037321in}%
\pgfsys@useobject{currentmarker}{}%
\end{pgfscope}%
\begin{pgfscope}%
\pgfsys@transformshift{3.417166in}{11.207379in}%
\pgfsys@useobject{currentmarker}{}%
\end{pgfscope}%
\begin{pgfscope}%
\pgfsys@transformshift{3.596897in}{11.347308in}%
\pgfsys@useobject{currentmarker}{}%
\end{pgfscope}%
\begin{pgfscope}%
\pgfsys@transformshift{3.776628in}{11.451761in}%
\pgfsys@useobject{currentmarker}{}%
\end{pgfscope}%
\begin{pgfscope}%
\pgfsys@transformshift{3.956359in}{11.552339in}%
\pgfsys@useobject{currentmarker}{}%
\end{pgfscope}%
\begin{pgfscope}%
\pgfsys@transformshift{4.136090in}{11.631910in}%
\pgfsys@useobject{currentmarker}{}%
\end{pgfscope}%
\begin{pgfscope}%
\pgfsys@transformshift{4.315821in}{11.706060in}%
\pgfsys@useobject{currentmarker}{}%
\end{pgfscope}%
\begin{pgfscope}%
\pgfsys@transformshift{4.495552in}{11.767666in}%
\pgfsys@useobject{currentmarker}{}%
\end{pgfscope}%
\begin{pgfscope}%
\pgfsys@transformshift{4.675283in}{11.823696in}%
\pgfsys@useobject{currentmarker}{}%
\end{pgfscope}%
\begin{pgfscope}%
\pgfsys@transformshift{4.855014in}{11.877820in}%
\pgfsys@useobject{currentmarker}{}%
\end{pgfscope}%
\begin{pgfscope}%
\pgfsys@transformshift{5.034745in}{11.921436in}%
\pgfsys@useobject{currentmarker}{}%
\end{pgfscope}%
\begin{pgfscope}%
\pgfsys@transformshift{5.214476in}{11.968543in}%
\pgfsys@useobject{currentmarker}{}%
\end{pgfscope}%
\begin{pgfscope}%
\pgfsys@transformshift{5.394207in}{12.006813in}%
\pgfsys@useobject{currentmarker}{}%
\end{pgfscope}%
\begin{pgfscope}%
\pgfsys@transformshift{5.573938in}{12.045121in}%
\pgfsys@useobject{currentmarker}{}%
\end{pgfscope}%
\begin{pgfscope}%
\pgfsys@transformshift{5.753669in}{12.074294in}%
\pgfsys@useobject{currentmarker}{}%
\end{pgfscope}%
\begin{pgfscope}%
\pgfsys@transformshift{5.933400in}{12.102281in}%
\pgfsys@useobject{currentmarker}{}%
\end{pgfscope}%
\begin{pgfscope}%
\pgfsys@transformshift{6.113131in}{12.135099in}%
\pgfsys@useobject{currentmarker}{}%
\end{pgfscope}%
\begin{pgfscope}%
\pgfsys@transformshift{6.292862in}{12.156907in}%
\pgfsys@useobject{currentmarker}{}%
\end{pgfscope}%
\begin{pgfscope}%
\pgfsys@transformshift{6.472593in}{12.187496in}%
\pgfsys@useobject{currentmarker}{}%
\end{pgfscope}%
\begin{pgfscope}%
\pgfsys@transformshift{6.652324in}{12.211484in}%
\pgfsys@useobject{currentmarker}{}%
\end{pgfscope}%
\begin{pgfscope}%
\pgfsys@transformshift{6.832055in}{12.234019in}%
\pgfsys@useobject{currentmarker}{}%
\end{pgfscope}%
\begin{pgfscope}%
\pgfsys@transformshift{7.011786in}{12.253480in}%
\pgfsys@useobject{currentmarker}{}%
\end{pgfscope}%
\begin{pgfscope}%
\pgfsys@transformshift{7.191517in}{12.277207in}%
\pgfsys@useobject{currentmarker}{}%
\end{pgfscope}%
\begin{pgfscope}%
\pgfsys@transformshift{7.371248in}{12.291862in}%
\pgfsys@useobject{currentmarker}{}%
\end{pgfscope}%
\begin{pgfscope}%
\pgfsys@transformshift{7.550979in}{12.306946in}%
\pgfsys@useobject{currentmarker}{}%
\end{pgfscope}%
\begin{pgfscope}%
\pgfsys@transformshift{7.730710in}{12.324848in}%
\pgfsys@useobject{currentmarker}{}%
\end{pgfscope}%
\begin{pgfscope}%
\pgfsys@transformshift{7.910441in}{12.338280in}%
\pgfsys@useobject{currentmarker}{}%
\end{pgfscope}%
\begin{pgfscope}%
\pgfsys@transformshift{8.090172in}{12.357163in}%
\pgfsys@useobject{currentmarker}{}%
\end{pgfscope}%
\begin{pgfscope}%
\pgfsys@transformshift{8.269903in}{12.370160in}%
\pgfsys@useobject{currentmarker}{}%
\end{pgfscope}%
\begin{pgfscope}%
\pgfsys@transformshift{8.449634in}{12.383151in}%
\pgfsys@useobject{currentmarker}{}%
\end{pgfscope}%
\begin{pgfscope}%
\pgfsys@transformshift{8.629365in}{12.397725in}%
\pgfsys@useobject{currentmarker}{}%
\end{pgfscope}%
\begin{pgfscope}%
\pgfsys@transformshift{8.809096in}{12.408387in}%
\pgfsys@useobject{currentmarker}{}%
\end{pgfscope}%
\begin{pgfscope}%
\pgfsys@transformshift{8.988827in}{12.418701in}%
\pgfsys@useobject{currentmarker}{}%
\end{pgfscope}%
\begin{pgfscope}%
\pgfsys@transformshift{9.168558in}{12.434908in}%
\pgfsys@useobject{currentmarker}{}%
\end{pgfscope}%
\begin{pgfscope}%
\pgfsys@transformshift{9.348289in}{12.445719in}%
\pgfsys@useobject{currentmarker}{}%
\end{pgfscope}%
\begin{pgfscope}%
\pgfsys@transformshift{9.528020in}{12.454338in}%
\pgfsys@useobject{currentmarker}{}%
\end{pgfscope}%
\begin{pgfscope}%
\pgfsys@transformshift{9.707751in}{12.465168in}%
\pgfsys@useobject{currentmarker}{}%
\end{pgfscope}%
\begin{pgfscope}%
\pgfsys@transformshift{9.887482in}{12.471502in}%
\pgfsys@useobject{currentmarker}{}%
\end{pgfscope}%
\end{pgfscope}%
\begin{pgfscope}%
\pgfpathrectangle{\pgfqpoint{0.640323in}{9.767436in}}{\pgfqpoint{9.687500in}{3.850000in}}%
\pgfusepath{clip}%
\pgfsetrectcap%
\pgfsetroundjoin%
\pgfsetlinewidth{0.803000pt}%
\definecolor{currentstroke}{rgb}{0.690196,0.690196,0.690196}%
\pgfsetstrokecolor{currentstroke}%
\pgfsetdash{}{0pt}%
\pgfpathmoveto{\pgfqpoint{1.080663in}{9.767436in}}%
\pgfpathlineto{\pgfqpoint{1.080663in}{13.617436in}}%
\pgfusepath{stroke}%
\end{pgfscope}%
\begin{pgfscope}%
\pgfsetbuttcap%
\pgfsetroundjoin%
\definecolor{currentfill}{rgb}{0.000000,0.000000,0.000000}%
\pgfsetfillcolor{currentfill}%
\pgfsetlinewidth{0.803000pt}%
\definecolor{currentstroke}{rgb}{0.000000,0.000000,0.000000}%
\pgfsetstrokecolor{currentstroke}%
\pgfsetdash{}{0pt}%
\pgfsys@defobject{currentmarker}{\pgfqpoint{0.000000in}{-0.048611in}}{\pgfqpoint{0.000000in}{0.000000in}}{%
\pgfpathmoveto{\pgfqpoint{0.000000in}{0.000000in}}%
\pgfpathlineto{\pgfqpoint{0.000000in}{-0.048611in}}%
\pgfusepath{stroke,fill}%
}%
\begin{pgfscope}%
\pgfsys@transformshift{1.080663in}{9.767436in}%
\pgfsys@useobject{currentmarker}{}%
\end{pgfscope}%
\end{pgfscope}%
\begin{pgfscope}%
\definecolor{textcolor}{rgb}{0.000000,0.000000,0.000000}%
\pgfsetstrokecolor{textcolor}%
\pgfsetfillcolor{textcolor}%
\pgftext[x=1.080663in,y=9.670214in,,top]{\color{textcolor}\sffamily\fontsize{10.000000}{12.000000}\selectfont 0.0}%
\end{pgfscope}%
\begin{pgfscope}%
\pgfpathrectangle{\pgfqpoint{0.640323in}{9.767436in}}{\pgfqpoint{9.687500in}{3.850000in}}%
\pgfusepath{clip}%
\pgfsetrectcap%
\pgfsetroundjoin%
\pgfsetlinewidth{0.803000pt}%
\definecolor{currentstroke}{rgb}{0.690196,0.690196,0.690196}%
\pgfsetstrokecolor{currentstroke}%
\pgfsetdash{}{0pt}%
\pgfpathmoveto{\pgfqpoint{2.877973in}{9.767436in}}%
\pgfpathlineto{\pgfqpoint{2.877973in}{13.617436in}}%
\pgfusepath{stroke}%
\end{pgfscope}%
\begin{pgfscope}%
\pgfsetbuttcap%
\pgfsetroundjoin%
\definecolor{currentfill}{rgb}{0.000000,0.000000,0.000000}%
\pgfsetfillcolor{currentfill}%
\pgfsetlinewidth{0.803000pt}%
\definecolor{currentstroke}{rgb}{0.000000,0.000000,0.000000}%
\pgfsetstrokecolor{currentstroke}%
\pgfsetdash{}{0pt}%
\pgfsys@defobject{currentmarker}{\pgfqpoint{0.000000in}{-0.048611in}}{\pgfqpoint{0.000000in}{0.000000in}}{%
\pgfpathmoveto{\pgfqpoint{0.000000in}{0.000000in}}%
\pgfpathlineto{\pgfqpoint{0.000000in}{-0.048611in}}%
\pgfusepath{stroke,fill}%
}%
\begin{pgfscope}%
\pgfsys@transformshift{2.877973in}{9.767436in}%
\pgfsys@useobject{currentmarker}{}%
\end{pgfscope}%
\end{pgfscope}%
\begin{pgfscope}%
\definecolor{textcolor}{rgb}{0.000000,0.000000,0.000000}%
\pgfsetstrokecolor{textcolor}%
\pgfsetfillcolor{textcolor}%
\pgftext[x=2.877973in,y=9.670214in,,top]{\color{textcolor}\sffamily\fontsize{10.000000}{12.000000}\selectfont 0.2}%
\end{pgfscope}%
\begin{pgfscope}%
\pgfpathrectangle{\pgfqpoint{0.640323in}{9.767436in}}{\pgfqpoint{9.687500in}{3.850000in}}%
\pgfusepath{clip}%
\pgfsetrectcap%
\pgfsetroundjoin%
\pgfsetlinewidth{0.803000pt}%
\definecolor{currentstroke}{rgb}{0.690196,0.690196,0.690196}%
\pgfsetstrokecolor{currentstroke}%
\pgfsetdash{}{0pt}%
\pgfpathmoveto{\pgfqpoint{4.675283in}{9.767436in}}%
\pgfpathlineto{\pgfqpoint{4.675283in}{13.617436in}}%
\pgfusepath{stroke}%
\end{pgfscope}%
\begin{pgfscope}%
\pgfsetbuttcap%
\pgfsetroundjoin%
\definecolor{currentfill}{rgb}{0.000000,0.000000,0.000000}%
\pgfsetfillcolor{currentfill}%
\pgfsetlinewidth{0.803000pt}%
\definecolor{currentstroke}{rgb}{0.000000,0.000000,0.000000}%
\pgfsetstrokecolor{currentstroke}%
\pgfsetdash{}{0pt}%
\pgfsys@defobject{currentmarker}{\pgfqpoint{0.000000in}{-0.048611in}}{\pgfqpoint{0.000000in}{0.000000in}}{%
\pgfpathmoveto{\pgfqpoint{0.000000in}{0.000000in}}%
\pgfpathlineto{\pgfqpoint{0.000000in}{-0.048611in}}%
\pgfusepath{stroke,fill}%
}%
\begin{pgfscope}%
\pgfsys@transformshift{4.675283in}{9.767436in}%
\pgfsys@useobject{currentmarker}{}%
\end{pgfscope}%
\end{pgfscope}%
\begin{pgfscope}%
\definecolor{textcolor}{rgb}{0.000000,0.000000,0.000000}%
\pgfsetstrokecolor{textcolor}%
\pgfsetfillcolor{textcolor}%
\pgftext[x=4.675283in,y=9.670214in,,top]{\color{textcolor}\sffamily\fontsize{10.000000}{12.000000}\selectfont 0.4}%
\end{pgfscope}%
\begin{pgfscope}%
\pgfpathrectangle{\pgfqpoint{0.640323in}{9.767436in}}{\pgfqpoint{9.687500in}{3.850000in}}%
\pgfusepath{clip}%
\pgfsetrectcap%
\pgfsetroundjoin%
\pgfsetlinewidth{0.803000pt}%
\definecolor{currentstroke}{rgb}{0.690196,0.690196,0.690196}%
\pgfsetstrokecolor{currentstroke}%
\pgfsetdash{}{0pt}%
\pgfpathmoveto{\pgfqpoint{6.472593in}{9.767436in}}%
\pgfpathlineto{\pgfqpoint{6.472593in}{13.617436in}}%
\pgfusepath{stroke}%
\end{pgfscope}%
\begin{pgfscope}%
\pgfsetbuttcap%
\pgfsetroundjoin%
\definecolor{currentfill}{rgb}{0.000000,0.000000,0.000000}%
\pgfsetfillcolor{currentfill}%
\pgfsetlinewidth{0.803000pt}%
\definecolor{currentstroke}{rgb}{0.000000,0.000000,0.000000}%
\pgfsetstrokecolor{currentstroke}%
\pgfsetdash{}{0pt}%
\pgfsys@defobject{currentmarker}{\pgfqpoint{0.000000in}{-0.048611in}}{\pgfqpoint{0.000000in}{0.000000in}}{%
\pgfpathmoveto{\pgfqpoint{0.000000in}{0.000000in}}%
\pgfpathlineto{\pgfqpoint{0.000000in}{-0.048611in}}%
\pgfusepath{stroke,fill}%
}%
\begin{pgfscope}%
\pgfsys@transformshift{6.472593in}{9.767436in}%
\pgfsys@useobject{currentmarker}{}%
\end{pgfscope}%
\end{pgfscope}%
\begin{pgfscope}%
\definecolor{textcolor}{rgb}{0.000000,0.000000,0.000000}%
\pgfsetstrokecolor{textcolor}%
\pgfsetfillcolor{textcolor}%
\pgftext[x=6.472593in,y=9.670214in,,top]{\color{textcolor}\sffamily\fontsize{10.000000}{12.000000}\selectfont 0.6}%
\end{pgfscope}%
\begin{pgfscope}%
\pgfpathrectangle{\pgfqpoint{0.640323in}{9.767436in}}{\pgfqpoint{9.687500in}{3.850000in}}%
\pgfusepath{clip}%
\pgfsetrectcap%
\pgfsetroundjoin%
\pgfsetlinewidth{0.803000pt}%
\definecolor{currentstroke}{rgb}{0.690196,0.690196,0.690196}%
\pgfsetstrokecolor{currentstroke}%
\pgfsetdash{}{0pt}%
\pgfpathmoveto{\pgfqpoint{8.269903in}{9.767436in}}%
\pgfpathlineto{\pgfqpoint{8.269903in}{13.617436in}}%
\pgfusepath{stroke}%
\end{pgfscope}%
\begin{pgfscope}%
\pgfsetbuttcap%
\pgfsetroundjoin%
\definecolor{currentfill}{rgb}{0.000000,0.000000,0.000000}%
\pgfsetfillcolor{currentfill}%
\pgfsetlinewidth{0.803000pt}%
\definecolor{currentstroke}{rgb}{0.000000,0.000000,0.000000}%
\pgfsetstrokecolor{currentstroke}%
\pgfsetdash{}{0pt}%
\pgfsys@defobject{currentmarker}{\pgfqpoint{0.000000in}{-0.048611in}}{\pgfqpoint{0.000000in}{0.000000in}}{%
\pgfpathmoveto{\pgfqpoint{0.000000in}{0.000000in}}%
\pgfpathlineto{\pgfqpoint{0.000000in}{-0.048611in}}%
\pgfusepath{stroke,fill}%
}%
\begin{pgfscope}%
\pgfsys@transformshift{8.269903in}{9.767436in}%
\pgfsys@useobject{currentmarker}{}%
\end{pgfscope}%
\end{pgfscope}%
\begin{pgfscope}%
\definecolor{textcolor}{rgb}{0.000000,0.000000,0.000000}%
\pgfsetstrokecolor{textcolor}%
\pgfsetfillcolor{textcolor}%
\pgftext[x=8.269903in,y=9.670214in,,top]{\color{textcolor}\sffamily\fontsize{10.000000}{12.000000}\selectfont 0.8}%
\end{pgfscope}%
\begin{pgfscope}%
\pgfpathrectangle{\pgfqpoint{0.640323in}{9.767436in}}{\pgfqpoint{9.687500in}{3.850000in}}%
\pgfusepath{clip}%
\pgfsetrectcap%
\pgfsetroundjoin%
\pgfsetlinewidth{0.803000pt}%
\definecolor{currentstroke}{rgb}{0.690196,0.690196,0.690196}%
\pgfsetstrokecolor{currentstroke}%
\pgfsetdash{}{0pt}%
\pgfpathmoveto{\pgfqpoint{10.067213in}{9.767436in}}%
\pgfpathlineto{\pgfqpoint{10.067213in}{13.617436in}}%
\pgfusepath{stroke}%
\end{pgfscope}%
\begin{pgfscope}%
\pgfsetbuttcap%
\pgfsetroundjoin%
\definecolor{currentfill}{rgb}{0.000000,0.000000,0.000000}%
\pgfsetfillcolor{currentfill}%
\pgfsetlinewidth{0.803000pt}%
\definecolor{currentstroke}{rgb}{0.000000,0.000000,0.000000}%
\pgfsetstrokecolor{currentstroke}%
\pgfsetdash{}{0pt}%
\pgfsys@defobject{currentmarker}{\pgfqpoint{0.000000in}{-0.048611in}}{\pgfqpoint{0.000000in}{0.000000in}}{%
\pgfpathmoveto{\pgfqpoint{0.000000in}{0.000000in}}%
\pgfpathlineto{\pgfqpoint{0.000000in}{-0.048611in}}%
\pgfusepath{stroke,fill}%
}%
\begin{pgfscope}%
\pgfsys@transformshift{10.067213in}{9.767436in}%
\pgfsys@useobject{currentmarker}{}%
\end{pgfscope}%
\end{pgfscope}%
\begin{pgfscope}%
\definecolor{textcolor}{rgb}{0.000000,0.000000,0.000000}%
\pgfsetstrokecolor{textcolor}%
\pgfsetfillcolor{textcolor}%
\pgftext[x=10.067213in,y=9.670214in,,top]{\color{textcolor}\sffamily\fontsize{10.000000}{12.000000}\selectfont 1.0}%
\end{pgfscope}%
\begin{pgfscope}%
\pgfpathrectangle{\pgfqpoint{0.640323in}{9.767436in}}{\pgfqpoint{9.687500in}{3.850000in}}%
\pgfusepath{clip}%
\pgfsetrectcap%
\pgfsetroundjoin%
\pgfsetlinewidth{0.803000pt}%
\definecolor{currentstroke}{rgb}{0.600000,0.600000,0.600000}%
\pgfsetstrokecolor{currentstroke}%
\pgfsetstrokeopacity{0.200000}%
\pgfsetdash{}{0pt}%
\pgfpathmoveto{\pgfqpoint{1.529991in}{9.767436in}}%
\pgfpathlineto{\pgfqpoint{1.529991in}{13.617436in}}%
\pgfusepath{stroke}%
\end{pgfscope}%
\begin{pgfscope}%
\pgfsetbuttcap%
\pgfsetroundjoin%
\definecolor{currentfill}{rgb}{0.000000,0.000000,0.000000}%
\pgfsetfillcolor{currentfill}%
\pgfsetlinewidth{0.602250pt}%
\definecolor{currentstroke}{rgb}{0.000000,0.000000,0.000000}%
\pgfsetstrokecolor{currentstroke}%
\pgfsetdash{}{0pt}%
\pgfsys@defobject{currentmarker}{\pgfqpoint{0.000000in}{-0.027778in}}{\pgfqpoint{0.000000in}{0.000000in}}{%
\pgfpathmoveto{\pgfqpoint{0.000000in}{0.000000in}}%
\pgfpathlineto{\pgfqpoint{0.000000in}{-0.027778in}}%
\pgfusepath{stroke,fill}%
}%
\begin{pgfscope}%
\pgfsys@transformshift{1.529991in}{9.767436in}%
\pgfsys@useobject{currentmarker}{}%
\end{pgfscope}%
\end{pgfscope}%
\begin{pgfscope}%
\pgfpathrectangle{\pgfqpoint{0.640323in}{9.767436in}}{\pgfqpoint{9.687500in}{3.850000in}}%
\pgfusepath{clip}%
\pgfsetrectcap%
\pgfsetroundjoin%
\pgfsetlinewidth{0.803000pt}%
\definecolor{currentstroke}{rgb}{0.600000,0.600000,0.600000}%
\pgfsetstrokecolor{currentstroke}%
\pgfsetstrokeopacity{0.200000}%
\pgfsetdash{}{0pt}%
\pgfpathmoveto{\pgfqpoint{1.979318in}{9.767436in}}%
\pgfpathlineto{\pgfqpoint{1.979318in}{13.617436in}}%
\pgfusepath{stroke}%
\end{pgfscope}%
\begin{pgfscope}%
\pgfsetbuttcap%
\pgfsetroundjoin%
\definecolor{currentfill}{rgb}{0.000000,0.000000,0.000000}%
\pgfsetfillcolor{currentfill}%
\pgfsetlinewidth{0.602250pt}%
\definecolor{currentstroke}{rgb}{0.000000,0.000000,0.000000}%
\pgfsetstrokecolor{currentstroke}%
\pgfsetdash{}{0pt}%
\pgfsys@defobject{currentmarker}{\pgfqpoint{0.000000in}{-0.027778in}}{\pgfqpoint{0.000000in}{0.000000in}}{%
\pgfpathmoveto{\pgfqpoint{0.000000in}{0.000000in}}%
\pgfpathlineto{\pgfqpoint{0.000000in}{-0.027778in}}%
\pgfusepath{stroke,fill}%
}%
\begin{pgfscope}%
\pgfsys@transformshift{1.979318in}{9.767436in}%
\pgfsys@useobject{currentmarker}{}%
\end{pgfscope}%
\end{pgfscope}%
\begin{pgfscope}%
\pgfpathrectangle{\pgfqpoint{0.640323in}{9.767436in}}{\pgfqpoint{9.687500in}{3.850000in}}%
\pgfusepath{clip}%
\pgfsetrectcap%
\pgfsetroundjoin%
\pgfsetlinewidth{0.803000pt}%
\definecolor{currentstroke}{rgb}{0.600000,0.600000,0.600000}%
\pgfsetstrokecolor{currentstroke}%
\pgfsetstrokeopacity{0.200000}%
\pgfsetdash{}{0pt}%
\pgfpathmoveto{\pgfqpoint{2.428646in}{9.767436in}}%
\pgfpathlineto{\pgfqpoint{2.428646in}{13.617436in}}%
\pgfusepath{stroke}%
\end{pgfscope}%
\begin{pgfscope}%
\pgfsetbuttcap%
\pgfsetroundjoin%
\definecolor{currentfill}{rgb}{0.000000,0.000000,0.000000}%
\pgfsetfillcolor{currentfill}%
\pgfsetlinewidth{0.602250pt}%
\definecolor{currentstroke}{rgb}{0.000000,0.000000,0.000000}%
\pgfsetstrokecolor{currentstroke}%
\pgfsetdash{}{0pt}%
\pgfsys@defobject{currentmarker}{\pgfqpoint{0.000000in}{-0.027778in}}{\pgfqpoint{0.000000in}{0.000000in}}{%
\pgfpathmoveto{\pgfqpoint{0.000000in}{0.000000in}}%
\pgfpathlineto{\pgfqpoint{0.000000in}{-0.027778in}}%
\pgfusepath{stroke,fill}%
}%
\begin{pgfscope}%
\pgfsys@transformshift{2.428646in}{9.767436in}%
\pgfsys@useobject{currentmarker}{}%
\end{pgfscope}%
\end{pgfscope}%
\begin{pgfscope}%
\pgfpathrectangle{\pgfqpoint{0.640323in}{9.767436in}}{\pgfqpoint{9.687500in}{3.850000in}}%
\pgfusepath{clip}%
\pgfsetrectcap%
\pgfsetroundjoin%
\pgfsetlinewidth{0.803000pt}%
\definecolor{currentstroke}{rgb}{0.600000,0.600000,0.600000}%
\pgfsetstrokecolor{currentstroke}%
\pgfsetstrokeopacity{0.200000}%
\pgfsetdash{}{0pt}%
\pgfpathmoveto{\pgfqpoint{3.327301in}{9.767436in}}%
\pgfpathlineto{\pgfqpoint{3.327301in}{13.617436in}}%
\pgfusepath{stroke}%
\end{pgfscope}%
\begin{pgfscope}%
\pgfsetbuttcap%
\pgfsetroundjoin%
\definecolor{currentfill}{rgb}{0.000000,0.000000,0.000000}%
\pgfsetfillcolor{currentfill}%
\pgfsetlinewidth{0.602250pt}%
\definecolor{currentstroke}{rgb}{0.000000,0.000000,0.000000}%
\pgfsetstrokecolor{currentstroke}%
\pgfsetdash{}{0pt}%
\pgfsys@defobject{currentmarker}{\pgfqpoint{0.000000in}{-0.027778in}}{\pgfqpoint{0.000000in}{0.000000in}}{%
\pgfpathmoveto{\pgfqpoint{0.000000in}{0.000000in}}%
\pgfpathlineto{\pgfqpoint{0.000000in}{-0.027778in}}%
\pgfusepath{stroke,fill}%
}%
\begin{pgfscope}%
\pgfsys@transformshift{3.327301in}{9.767436in}%
\pgfsys@useobject{currentmarker}{}%
\end{pgfscope}%
\end{pgfscope}%
\begin{pgfscope}%
\pgfpathrectangle{\pgfqpoint{0.640323in}{9.767436in}}{\pgfqpoint{9.687500in}{3.850000in}}%
\pgfusepath{clip}%
\pgfsetrectcap%
\pgfsetroundjoin%
\pgfsetlinewidth{0.803000pt}%
\definecolor{currentstroke}{rgb}{0.600000,0.600000,0.600000}%
\pgfsetstrokecolor{currentstroke}%
\pgfsetstrokeopacity{0.200000}%
\pgfsetdash{}{0pt}%
\pgfpathmoveto{\pgfqpoint{3.776628in}{9.767436in}}%
\pgfpathlineto{\pgfqpoint{3.776628in}{13.617436in}}%
\pgfusepath{stroke}%
\end{pgfscope}%
\begin{pgfscope}%
\pgfsetbuttcap%
\pgfsetroundjoin%
\definecolor{currentfill}{rgb}{0.000000,0.000000,0.000000}%
\pgfsetfillcolor{currentfill}%
\pgfsetlinewidth{0.602250pt}%
\definecolor{currentstroke}{rgb}{0.000000,0.000000,0.000000}%
\pgfsetstrokecolor{currentstroke}%
\pgfsetdash{}{0pt}%
\pgfsys@defobject{currentmarker}{\pgfqpoint{0.000000in}{-0.027778in}}{\pgfqpoint{0.000000in}{0.000000in}}{%
\pgfpathmoveto{\pgfqpoint{0.000000in}{0.000000in}}%
\pgfpathlineto{\pgfqpoint{0.000000in}{-0.027778in}}%
\pgfusepath{stroke,fill}%
}%
\begin{pgfscope}%
\pgfsys@transformshift{3.776628in}{9.767436in}%
\pgfsys@useobject{currentmarker}{}%
\end{pgfscope}%
\end{pgfscope}%
\begin{pgfscope}%
\pgfpathrectangle{\pgfqpoint{0.640323in}{9.767436in}}{\pgfqpoint{9.687500in}{3.850000in}}%
\pgfusepath{clip}%
\pgfsetrectcap%
\pgfsetroundjoin%
\pgfsetlinewidth{0.803000pt}%
\definecolor{currentstroke}{rgb}{0.600000,0.600000,0.600000}%
\pgfsetstrokecolor{currentstroke}%
\pgfsetstrokeopacity{0.200000}%
\pgfsetdash{}{0pt}%
\pgfpathmoveto{\pgfqpoint{4.225956in}{9.767436in}}%
\pgfpathlineto{\pgfqpoint{4.225956in}{13.617436in}}%
\pgfusepath{stroke}%
\end{pgfscope}%
\begin{pgfscope}%
\pgfsetbuttcap%
\pgfsetroundjoin%
\definecolor{currentfill}{rgb}{0.000000,0.000000,0.000000}%
\pgfsetfillcolor{currentfill}%
\pgfsetlinewidth{0.602250pt}%
\definecolor{currentstroke}{rgb}{0.000000,0.000000,0.000000}%
\pgfsetstrokecolor{currentstroke}%
\pgfsetdash{}{0pt}%
\pgfsys@defobject{currentmarker}{\pgfqpoint{0.000000in}{-0.027778in}}{\pgfqpoint{0.000000in}{0.000000in}}{%
\pgfpathmoveto{\pgfqpoint{0.000000in}{0.000000in}}%
\pgfpathlineto{\pgfqpoint{0.000000in}{-0.027778in}}%
\pgfusepath{stroke,fill}%
}%
\begin{pgfscope}%
\pgfsys@transformshift{4.225956in}{9.767436in}%
\pgfsys@useobject{currentmarker}{}%
\end{pgfscope}%
\end{pgfscope}%
\begin{pgfscope}%
\pgfpathrectangle{\pgfqpoint{0.640323in}{9.767436in}}{\pgfqpoint{9.687500in}{3.850000in}}%
\pgfusepath{clip}%
\pgfsetrectcap%
\pgfsetroundjoin%
\pgfsetlinewidth{0.803000pt}%
\definecolor{currentstroke}{rgb}{0.600000,0.600000,0.600000}%
\pgfsetstrokecolor{currentstroke}%
\pgfsetstrokeopacity{0.200000}%
\pgfsetdash{}{0pt}%
\pgfpathmoveto{\pgfqpoint{5.124611in}{9.767436in}}%
\pgfpathlineto{\pgfqpoint{5.124611in}{13.617436in}}%
\pgfusepath{stroke}%
\end{pgfscope}%
\begin{pgfscope}%
\pgfsetbuttcap%
\pgfsetroundjoin%
\definecolor{currentfill}{rgb}{0.000000,0.000000,0.000000}%
\pgfsetfillcolor{currentfill}%
\pgfsetlinewidth{0.602250pt}%
\definecolor{currentstroke}{rgb}{0.000000,0.000000,0.000000}%
\pgfsetstrokecolor{currentstroke}%
\pgfsetdash{}{0pt}%
\pgfsys@defobject{currentmarker}{\pgfqpoint{0.000000in}{-0.027778in}}{\pgfqpoint{0.000000in}{0.000000in}}{%
\pgfpathmoveto{\pgfqpoint{0.000000in}{0.000000in}}%
\pgfpathlineto{\pgfqpoint{0.000000in}{-0.027778in}}%
\pgfusepath{stroke,fill}%
}%
\begin{pgfscope}%
\pgfsys@transformshift{5.124611in}{9.767436in}%
\pgfsys@useobject{currentmarker}{}%
\end{pgfscope}%
\end{pgfscope}%
\begin{pgfscope}%
\pgfpathrectangle{\pgfqpoint{0.640323in}{9.767436in}}{\pgfqpoint{9.687500in}{3.850000in}}%
\pgfusepath{clip}%
\pgfsetrectcap%
\pgfsetroundjoin%
\pgfsetlinewidth{0.803000pt}%
\definecolor{currentstroke}{rgb}{0.600000,0.600000,0.600000}%
\pgfsetstrokecolor{currentstroke}%
\pgfsetstrokeopacity{0.200000}%
\pgfsetdash{}{0pt}%
\pgfpathmoveto{\pgfqpoint{5.573938in}{9.767436in}}%
\pgfpathlineto{\pgfqpoint{5.573938in}{13.617436in}}%
\pgfusepath{stroke}%
\end{pgfscope}%
\begin{pgfscope}%
\pgfsetbuttcap%
\pgfsetroundjoin%
\definecolor{currentfill}{rgb}{0.000000,0.000000,0.000000}%
\pgfsetfillcolor{currentfill}%
\pgfsetlinewidth{0.602250pt}%
\definecolor{currentstroke}{rgb}{0.000000,0.000000,0.000000}%
\pgfsetstrokecolor{currentstroke}%
\pgfsetdash{}{0pt}%
\pgfsys@defobject{currentmarker}{\pgfqpoint{0.000000in}{-0.027778in}}{\pgfqpoint{0.000000in}{0.000000in}}{%
\pgfpathmoveto{\pgfqpoint{0.000000in}{0.000000in}}%
\pgfpathlineto{\pgfqpoint{0.000000in}{-0.027778in}}%
\pgfusepath{stroke,fill}%
}%
\begin{pgfscope}%
\pgfsys@transformshift{5.573938in}{9.767436in}%
\pgfsys@useobject{currentmarker}{}%
\end{pgfscope}%
\end{pgfscope}%
\begin{pgfscope}%
\pgfpathrectangle{\pgfqpoint{0.640323in}{9.767436in}}{\pgfqpoint{9.687500in}{3.850000in}}%
\pgfusepath{clip}%
\pgfsetrectcap%
\pgfsetroundjoin%
\pgfsetlinewidth{0.803000pt}%
\definecolor{currentstroke}{rgb}{0.600000,0.600000,0.600000}%
\pgfsetstrokecolor{currentstroke}%
\pgfsetstrokeopacity{0.200000}%
\pgfsetdash{}{0pt}%
\pgfpathmoveto{\pgfqpoint{6.023265in}{9.767436in}}%
\pgfpathlineto{\pgfqpoint{6.023265in}{13.617436in}}%
\pgfusepath{stroke}%
\end{pgfscope}%
\begin{pgfscope}%
\pgfsetbuttcap%
\pgfsetroundjoin%
\definecolor{currentfill}{rgb}{0.000000,0.000000,0.000000}%
\pgfsetfillcolor{currentfill}%
\pgfsetlinewidth{0.602250pt}%
\definecolor{currentstroke}{rgb}{0.000000,0.000000,0.000000}%
\pgfsetstrokecolor{currentstroke}%
\pgfsetdash{}{0pt}%
\pgfsys@defobject{currentmarker}{\pgfqpoint{0.000000in}{-0.027778in}}{\pgfqpoint{0.000000in}{0.000000in}}{%
\pgfpathmoveto{\pgfqpoint{0.000000in}{0.000000in}}%
\pgfpathlineto{\pgfqpoint{0.000000in}{-0.027778in}}%
\pgfusepath{stroke,fill}%
}%
\begin{pgfscope}%
\pgfsys@transformshift{6.023265in}{9.767436in}%
\pgfsys@useobject{currentmarker}{}%
\end{pgfscope}%
\end{pgfscope}%
\begin{pgfscope}%
\pgfpathrectangle{\pgfqpoint{0.640323in}{9.767436in}}{\pgfqpoint{9.687500in}{3.850000in}}%
\pgfusepath{clip}%
\pgfsetrectcap%
\pgfsetroundjoin%
\pgfsetlinewidth{0.803000pt}%
\definecolor{currentstroke}{rgb}{0.600000,0.600000,0.600000}%
\pgfsetstrokecolor{currentstroke}%
\pgfsetstrokeopacity{0.200000}%
\pgfsetdash{}{0pt}%
\pgfpathmoveto{\pgfqpoint{6.921920in}{9.767436in}}%
\pgfpathlineto{\pgfqpoint{6.921920in}{13.617436in}}%
\pgfusepath{stroke}%
\end{pgfscope}%
\begin{pgfscope}%
\pgfsetbuttcap%
\pgfsetroundjoin%
\definecolor{currentfill}{rgb}{0.000000,0.000000,0.000000}%
\pgfsetfillcolor{currentfill}%
\pgfsetlinewidth{0.602250pt}%
\definecolor{currentstroke}{rgb}{0.000000,0.000000,0.000000}%
\pgfsetstrokecolor{currentstroke}%
\pgfsetdash{}{0pt}%
\pgfsys@defobject{currentmarker}{\pgfqpoint{0.000000in}{-0.027778in}}{\pgfqpoint{0.000000in}{0.000000in}}{%
\pgfpathmoveto{\pgfqpoint{0.000000in}{0.000000in}}%
\pgfpathlineto{\pgfqpoint{0.000000in}{-0.027778in}}%
\pgfusepath{stroke,fill}%
}%
\begin{pgfscope}%
\pgfsys@transformshift{6.921920in}{9.767436in}%
\pgfsys@useobject{currentmarker}{}%
\end{pgfscope}%
\end{pgfscope}%
\begin{pgfscope}%
\pgfpathrectangle{\pgfqpoint{0.640323in}{9.767436in}}{\pgfqpoint{9.687500in}{3.850000in}}%
\pgfusepath{clip}%
\pgfsetrectcap%
\pgfsetroundjoin%
\pgfsetlinewidth{0.803000pt}%
\definecolor{currentstroke}{rgb}{0.600000,0.600000,0.600000}%
\pgfsetstrokecolor{currentstroke}%
\pgfsetstrokeopacity{0.200000}%
\pgfsetdash{}{0pt}%
\pgfpathmoveto{\pgfqpoint{7.371248in}{9.767436in}}%
\pgfpathlineto{\pgfqpoint{7.371248in}{13.617436in}}%
\pgfusepath{stroke}%
\end{pgfscope}%
\begin{pgfscope}%
\pgfsetbuttcap%
\pgfsetroundjoin%
\definecolor{currentfill}{rgb}{0.000000,0.000000,0.000000}%
\pgfsetfillcolor{currentfill}%
\pgfsetlinewidth{0.602250pt}%
\definecolor{currentstroke}{rgb}{0.000000,0.000000,0.000000}%
\pgfsetstrokecolor{currentstroke}%
\pgfsetdash{}{0pt}%
\pgfsys@defobject{currentmarker}{\pgfqpoint{0.000000in}{-0.027778in}}{\pgfqpoint{0.000000in}{0.000000in}}{%
\pgfpathmoveto{\pgfqpoint{0.000000in}{0.000000in}}%
\pgfpathlineto{\pgfqpoint{0.000000in}{-0.027778in}}%
\pgfusepath{stroke,fill}%
}%
\begin{pgfscope}%
\pgfsys@transformshift{7.371248in}{9.767436in}%
\pgfsys@useobject{currentmarker}{}%
\end{pgfscope}%
\end{pgfscope}%
\begin{pgfscope}%
\pgfpathrectangle{\pgfqpoint{0.640323in}{9.767436in}}{\pgfqpoint{9.687500in}{3.850000in}}%
\pgfusepath{clip}%
\pgfsetrectcap%
\pgfsetroundjoin%
\pgfsetlinewidth{0.803000pt}%
\definecolor{currentstroke}{rgb}{0.600000,0.600000,0.600000}%
\pgfsetstrokecolor{currentstroke}%
\pgfsetstrokeopacity{0.200000}%
\pgfsetdash{}{0pt}%
\pgfpathmoveto{\pgfqpoint{7.820575in}{9.767436in}}%
\pgfpathlineto{\pgfqpoint{7.820575in}{13.617436in}}%
\pgfusepath{stroke}%
\end{pgfscope}%
\begin{pgfscope}%
\pgfsetbuttcap%
\pgfsetroundjoin%
\definecolor{currentfill}{rgb}{0.000000,0.000000,0.000000}%
\pgfsetfillcolor{currentfill}%
\pgfsetlinewidth{0.602250pt}%
\definecolor{currentstroke}{rgb}{0.000000,0.000000,0.000000}%
\pgfsetstrokecolor{currentstroke}%
\pgfsetdash{}{0pt}%
\pgfsys@defobject{currentmarker}{\pgfqpoint{0.000000in}{-0.027778in}}{\pgfqpoint{0.000000in}{0.000000in}}{%
\pgfpathmoveto{\pgfqpoint{0.000000in}{0.000000in}}%
\pgfpathlineto{\pgfqpoint{0.000000in}{-0.027778in}}%
\pgfusepath{stroke,fill}%
}%
\begin{pgfscope}%
\pgfsys@transformshift{7.820575in}{9.767436in}%
\pgfsys@useobject{currentmarker}{}%
\end{pgfscope}%
\end{pgfscope}%
\begin{pgfscope}%
\pgfpathrectangle{\pgfqpoint{0.640323in}{9.767436in}}{\pgfqpoint{9.687500in}{3.850000in}}%
\pgfusepath{clip}%
\pgfsetrectcap%
\pgfsetroundjoin%
\pgfsetlinewidth{0.803000pt}%
\definecolor{currentstroke}{rgb}{0.600000,0.600000,0.600000}%
\pgfsetstrokecolor{currentstroke}%
\pgfsetstrokeopacity{0.200000}%
\pgfsetdash{}{0pt}%
\pgfpathmoveto{\pgfqpoint{8.719230in}{9.767436in}}%
\pgfpathlineto{\pgfqpoint{8.719230in}{13.617436in}}%
\pgfusepath{stroke}%
\end{pgfscope}%
\begin{pgfscope}%
\pgfsetbuttcap%
\pgfsetroundjoin%
\definecolor{currentfill}{rgb}{0.000000,0.000000,0.000000}%
\pgfsetfillcolor{currentfill}%
\pgfsetlinewidth{0.602250pt}%
\definecolor{currentstroke}{rgb}{0.000000,0.000000,0.000000}%
\pgfsetstrokecolor{currentstroke}%
\pgfsetdash{}{0pt}%
\pgfsys@defobject{currentmarker}{\pgfqpoint{0.000000in}{-0.027778in}}{\pgfqpoint{0.000000in}{0.000000in}}{%
\pgfpathmoveto{\pgfqpoint{0.000000in}{0.000000in}}%
\pgfpathlineto{\pgfqpoint{0.000000in}{-0.027778in}}%
\pgfusepath{stroke,fill}%
}%
\begin{pgfscope}%
\pgfsys@transformshift{8.719230in}{9.767436in}%
\pgfsys@useobject{currentmarker}{}%
\end{pgfscope}%
\end{pgfscope}%
\begin{pgfscope}%
\pgfpathrectangle{\pgfqpoint{0.640323in}{9.767436in}}{\pgfqpoint{9.687500in}{3.850000in}}%
\pgfusepath{clip}%
\pgfsetrectcap%
\pgfsetroundjoin%
\pgfsetlinewidth{0.803000pt}%
\definecolor{currentstroke}{rgb}{0.600000,0.600000,0.600000}%
\pgfsetstrokecolor{currentstroke}%
\pgfsetstrokeopacity{0.200000}%
\pgfsetdash{}{0pt}%
\pgfpathmoveto{\pgfqpoint{9.168558in}{9.767436in}}%
\pgfpathlineto{\pgfqpoint{9.168558in}{13.617436in}}%
\pgfusepath{stroke}%
\end{pgfscope}%
\begin{pgfscope}%
\pgfsetbuttcap%
\pgfsetroundjoin%
\definecolor{currentfill}{rgb}{0.000000,0.000000,0.000000}%
\pgfsetfillcolor{currentfill}%
\pgfsetlinewidth{0.602250pt}%
\definecolor{currentstroke}{rgb}{0.000000,0.000000,0.000000}%
\pgfsetstrokecolor{currentstroke}%
\pgfsetdash{}{0pt}%
\pgfsys@defobject{currentmarker}{\pgfqpoint{0.000000in}{-0.027778in}}{\pgfqpoint{0.000000in}{0.000000in}}{%
\pgfpathmoveto{\pgfqpoint{0.000000in}{0.000000in}}%
\pgfpathlineto{\pgfqpoint{0.000000in}{-0.027778in}}%
\pgfusepath{stroke,fill}%
}%
\begin{pgfscope}%
\pgfsys@transformshift{9.168558in}{9.767436in}%
\pgfsys@useobject{currentmarker}{}%
\end{pgfscope}%
\end{pgfscope}%
\begin{pgfscope}%
\pgfpathrectangle{\pgfqpoint{0.640323in}{9.767436in}}{\pgfqpoint{9.687500in}{3.850000in}}%
\pgfusepath{clip}%
\pgfsetrectcap%
\pgfsetroundjoin%
\pgfsetlinewidth{0.803000pt}%
\definecolor{currentstroke}{rgb}{0.600000,0.600000,0.600000}%
\pgfsetstrokecolor{currentstroke}%
\pgfsetstrokeopacity{0.200000}%
\pgfsetdash{}{0pt}%
\pgfpathmoveto{\pgfqpoint{9.617885in}{9.767436in}}%
\pgfpathlineto{\pgfqpoint{9.617885in}{13.617436in}}%
\pgfusepath{stroke}%
\end{pgfscope}%
\begin{pgfscope}%
\pgfsetbuttcap%
\pgfsetroundjoin%
\definecolor{currentfill}{rgb}{0.000000,0.000000,0.000000}%
\pgfsetfillcolor{currentfill}%
\pgfsetlinewidth{0.602250pt}%
\definecolor{currentstroke}{rgb}{0.000000,0.000000,0.000000}%
\pgfsetstrokecolor{currentstroke}%
\pgfsetdash{}{0pt}%
\pgfsys@defobject{currentmarker}{\pgfqpoint{0.000000in}{-0.027778in}}{\pgfqpoint{0.000000in}{0.000000in}}{%
\pgfpathmoveto{\pgfqpoint{0.000000in}{0.000000in}}%
\pgfpathlineto{\pgfqpoint{0.000000in}{-0.027778in}}%
\pgfusepath{stroke,fill}%
}%
\begin{pgfscope}%
\pgfsys@transformshift{9.617885in}{9.767436in}%
\pgfsys@useobject{currentmarker}{}%
\end{pgfscope}%
\end{pgfscope}%
\begin{pgfscope}%
\definecolor{textcolor}{rgb}{0.000000,0.000000,0.000000}%
\pgfsetstrokecolor{textcolor}%
\pgfsetfillcolor{textcolor}%
\pgftext[x=5.484072in,y=9.480245in,,top]{\color{textcolor}\sffamily\fontsize{10.000000}{12.000000}\selectfont turnover probability \(\displaystyle p_1\,(S\rightarrow I\,)\)}%
\end{pgfscope}%
\begin{pgfscope}%
\pgfpathrectangle{\pgfqpoint{0.640323in}{9.767436in}}{\pgfqpoint{9.687500in}{3.850000in}}%
\pgfusepath{clip}%
\pgfsetrectcap%
\pgfsetroundjoin%
\pgfsetlinewidth{0.803000pt}%
\definecolor{currentstroke}{rgb}{0.690196,0.690196,0.690196}%
\pgfsetstrokecolor{currentstroke}%
\pgfsetdash{}{0pt}%
\pgfpathmoveto{\pgfqpoint{0.640323in}{9.891629in}}%
\pgfpathlineto{\pgfqpoint{10.327822in}{9.891629in}}%
\pgfusepath{stroke}%
\end{pgfscope}%
\begin{pgfscope}%
\pgfsetbuttcap%
\pgfsetroundjoin%
\definecolor{currentfill}{rgb}{0.000000,0.000000,0.000000}%
\pgfsetfillcolor{currentfill}%
\pgfsetlinewidth{0.803000pt}%
\definecolor{currentstroke}{rgb}{0.000000,0.000000,0.000000}%
\pgfsetstrokecolor{currentstroke}%
\pgfsetdash{}{0pt}%
\pgfsys@defobject{currentmarker}{\pgfqpoint{-0.048611in}{0.000000in}}{\pgfqpoint{-0.000000in}{0.000000in}}{%
\pgfpathmoveto{\pgfqpoint{-0.000000in}{0.000000in}}%
\pgfpathlineto{\pgfqpoint{-0.048611in}{0.000000in}}%
\pgfusepath{stroke,fill}%
}%
\begin{pgfscope}%
\pgfsys@transformshift{0.640323in}{9.891629in}%
\pgfsys@useobject{currentmarker}{}%
\end{pgfscope}%
\end{pgfscope}%
\begin{pgfscope}%
\definecolor{textcolor}{rgb}{0.000000,0.000000,0.000000}%
\pgfsetstrokecolor{textcolor}%
\pgfsetfillcolor{textcolor}%
\pgftext[x=0.322221in, y=9.838868in, left, base]{\color{textcolor}\sffamily\fontsize{10.000000}{12.000000}\selectfont 0.0}%
\end{pgfscope}%
\begin{pgfscope}%
\pgfpathrectangle{\pgfqpoint{0.640323in}{9.767436in}}{\pgfqpoint{9.687500in}{3.850000in}}%
\pgfusepath{clip}%
\pgfsetrectcap%
\pgfsetroundjoin%
\pgfsetlinewidth{0.803000pt}%
\definecolor{currentstroke}{rgb}{0.690196,0.690196,0.690196}%
\pgfsetstrokecolor{currentstroke}%
\pgfsetdash{}{0pt}%
\pgfpathmoveto{\pgfqpoint{0.640323in}{10.512597in}}%
\pgfpathlineto{\pgfqpoint{10.327822in}{10.512597in}}%
\pgfusepath{stroke}%
\end{pgfscope}%
\begin{pgfscope}%
\pgfsetbuttcap%
\pgfsetroundjoin%
\definecolor{currentfill}{rgb}{0.000000,0.000000,0.000000}%
\pgfsetfillcolor{currentfill}%
\pgfsetlinewidth{0.803000pt}%
\definecolor{currentstroke}{rgb}{0.000000,0.000000,0.000000}%
\pgfsetstrokecolor{currentstroke}%
\pgfsetdash{}{0pt}%
\pgfsys@defobject{currentmarker}{\pgfqpoint{-0.048611in}{0.000000in}}{\pgfqpoint{-0.000000in}{0.000000in}}{%
\pgfpathmoveto{\pgfqpoint{-0.000000in}{0.000000in}}%
\pgfpathlineto{\pgfqpoint{-0.048611in}{0.000000in}}%
\pgfusepath{stroke,fill}%
}%
\begin{pgfscope}%
\pgfsys@transformshift{0.640323in}{10.512597in}%
\pgfsys@useobject{currentmarker}{}%
\end{pgfscope}%
\end{pgfscope}%
\begin{pgfscope}%
\definecolor{textcolor}{rgb}{0.000000,0.000000,0.000000}%
\pgfsetstrokecolor{textcolor}%
\pgfsetfillcolor{textcolor}%
\pgftext[x=0.322221in, y=10.459836in, left, base]{\color{textcolor}\sffamily\fontsize{10.000000}{12.000000}\selectfont 0.1}%
\end{pgfscope}%
\begin{pgfscope}%
\pgfpathrectangle{\pgfqpoint{0.640323in}{9.767436in}}{\pgfqpoint{9.687500in}{3.850000in}}%
\pgfusepath{clip}%
\pgfsetrectcap%
\pgfsetroundjoin%
\pgfsetlinewidth{0.803000pt}%
\definecolor{currentstroke}{rgb}{0.690196,0.690196,0.690196}%
\pgfsetstrokecolor{currentstroke}%
\pgfsetdash{}{0pt}%
\pgfpathmoveto{\pgfqpoint{0.640323in}{11.133565in}}%
\pgfpathlineto{\pgfqpoint{10.327822in}{11.133565in}}%
\pgfusepath{stroke}%
\end{pgfscope}%
\begin{pgfscope}%
\pgfsetbuttcap%
\pgfsetroundjoin%
\definecolor{currentfill}{rgb}{0.000000,0.000000,0.000000}%
\pgfsetfillcolor{currentfill}%
\pgfsetlinewidth{0.803000pt}%
\definecolor{currentstroke}{rgb}{0.000000,0.000000,0.000000}%
\pgfsetstrokecolor{currentstroke}%
\pgfsetdash{}{0pt}%
\pgfsys@defobject{currentmarker}{\pgfqpoint{-0.048611in}{0.000000in}}{\pgfqpoint{-0.000000in}{0.000000in}}{%
\pgfpathmoveto{\pgfqpoint{-0.000000in}{0.000000in}}%
\pgfpathlineto{\pgfqpoint{-0.048611in}{0.000000in}}%
\pgfusepath{stroke,fill}%
}%
\begin{pgfscope}%
\pgfsys@transformshift{0.640323in}{11.133565in}%
\pgfsys@useobject{currentmarker}{}%
\end{pgfscope}%
\end{pgfscope}%
\begin{pgfscope}%
\definecolor{textcolor}{rgb}{0.000000,0.000000,0.000000}%
\pgfsetstrokecolor{textcolor}%
\pgfsetfillcolor{textcolor}%
\pgftext[x=0.322221in, y=11.080803in, left, base]{\color{textcolor}\sffamily\fontsize{10.000000}{12.000000}\selectfont 0.2}%
\end{pgfscope}%
\begin{pgfscope}%
\pgfpathrectangle{\pgfqpoint{0.640323in}{9.767436in}}{\pgfqpoint{9.687500in}{3.850000in}}%
\pgfusepath{clip}%
\pgfsetrectcap%
\pgfsetroundjoin%
\pgfsetlinewidth{0.803000pt}%
\definecolor{currentstroke}{rgb}{0.690196,0.690196,0.690196}%
\pgfsetstrokecolor{currentstroke}%
\pgfsetdash{}{0pt}%
\pgfpathmoveto{\pgfqpoint{0.640323in}{11.754533in}}%
\pgfpathlineto{\pgfqpoint{10.327822in}{11.754533in}}%
\pgfusepath{stroke}%
\end{pgfscope}%
\begin{pgfscope}%
\pgfsetbuttcap%
\pgfsetroundjoin%
\definecolor{currentfill}{rgb}{0.000000,0.000000,0.000000}%
\pgfsetfillcolor{currentfill}%
\pgfsetlinewidth{0.803000pt}%
\definecolor{currentstroke}{rgb}{0.000000,0.000000,0.000000}%
\pgfsetstrokecolor{currentstroke}%
\pgfsetdash{}{0pt}%
\pgfsys@defobject{currentmarker}{\pgfqpoint{-0.048611in}{0.000000in}}{\pgfqpoint{-0.000000in}{0.000000in}}{%
\pgfpathmoveto{\pgfqpoint{-0.000000in}{0.000000in}}%
\pgfpathlineto{\pgfqpoint{-0.048611in}{0.000000in}}%
\pgfusepath{stroke,fill}%
}%
\begin{pgfscope}%
\pgfsys@transformshift{0.640323in}{11.754533in}%
\pgfsys@useobject{currentmarker}{}%
\end{pgfscope}%
\end{pgfscope}%
\begin{pgfscope}%
\definecolor{textcolor}{rgb}{0.000000,0.000000,0.000000}%
\pgfsetstrokecolor{textcolor}%
\pgfsetfillcolor{textcolor}%
\pgftext[x=0.322221in, y=11.701771in, left, base]{\color{textcolor}\sffamily\fontsize{10.000000}{12.000000}\selectfont 0.3}%
\end{pgfscope}%
\begin{pgfscope}%
\pgfpathrectangle{\pgfqpoint{0.640323in}{9.767436in}}{\pgfqpoint{9.687500in}{3.850000in}}%
\pgfusepath{clip}%
\pgfsetrectcap%
\pgfsetroundjoin%
\pgfsetlinewidth{0.803000pt}%
\definecolor{currentstroke}{rgb}{0.690196,0.690196,0.690196}%
\pgfsetstrokecolor{currentstroke}%
\pgfsetdash{}{0pt}%
\pgfpathmoveto{\pgfqpoint{0.640323in}{12.375500in}}%
\pgfpathlineto{\pgfqpoint{10.327822in}{12.375500in}}%
\pgfusepath{stroke}%
\end{pgfscope}%
\begin{pgfscope}%
\pgfsetbuttcap%
\pgfsetroundjoin%
\definecolor{currentfill}{rgb}{0.000000,0.000000,0.000000}%
\pgfsetfillcolor{currentfill}%
\pgfsetlinewidth{0.803000pt}%
\definecolor{currentstroke}{rgb}{0.000000,0.000000,0.000000}%
\pgfsetstrokecolor{currentstroke}%
\pgfsetdash{}{0pt}%
\pgfsys@defobject{currentmarker}{\pgfqpoint{-0.048611in}{0.000000in}}{\pgfqpoint{-0.000000in}{0.000000in}}{%
\pgfpathmoveto{\pgfqpoint{-0.000000in}{0.000000in}}%
\pgfpathlineto{\pgfqpoint{-0.048611in}{0.000000in}}%
\pgfusepath{stroke,fill}%
}%
\begin{pgfscope}%
\pgfsys@transformshift{0.640323in}{12.375500in}%
\pgfsys@useobject{currentmarker}{}%
\end{pgfscope}%
\end{pgfscope}%
\begin{pgfscope}%
\definecolor{textcolor}{rgb}{0.000000,0.000000,0.000000}%
\pgfsetstrokecolor{textcolor}%
\pgfsetfillcolor{textcolor}%
\pgftext[x=0.322221in, y=12.322739in, left, base]{\color{textcolor}\sffamily\fontsize{10.000000}{12.000000}\selectfont 0.4}%
\end{pgfscope}%
\begin{pgfscope}%
\pgfpathrectangle{\pgfqpoint{0.640323in}{9.767436in}}{\pgfqpoint{9.687500in}{3.850000in}}%
\pgfusepath{clip}%
\pgfsetrectcap%
\pgfsetroundjoin%
\pgfsetlinewidth{0.803000pt}%
\definecolor{currentstroke}{rgb}{0.690196,0.690196,0.690196}%
\pgfsetstrokecolor{currentstroke}%
\pgfsetdash{}{0pt}%
\pgfpathmoveto{\pgfqpoint{0.640323in}{12.996468in}}%
\pgfpathlineto{\pgfqpoint{10.327822in}{12.996468in}}%
\pgfusepath{stroke}%
\end{pgfscope}%
\begin{pgfscope}%
\pgfsetbuttcap%
\pgfsetroundjoin%
\definecolor{currentfill}{rgb}{0.000000,0.000000,0.000000}%
\pgfsetfillcolor{currentfill}%
\pgfsetlinewidth{0.803000pt}%
\definecolor{currentstroke}{rgb}{0.000000,0.000000,0.000000}%
\pgfsetstrokecolor{currentstroke}%
\pgfsetdash{}{0pt}%
\pgfsys@defobject{currentmarker}{\pgfqpoint{-0.048611in}{0.000000in}}{\pgfqpoint{-0.000000in}{0.000000in}}{%
\pgfpathmoveto{\pgfqpoint{-0.000000in}{0.000000in}}%
\pgfpathlineto{\pgfqpoint{-0.048611in}{0.000000in}}%
\pgfusepath{stroke,fill}%
}%
\begin{pgfscope}%
\pgfsys@transformshift{0.640323in}{12.996468in}%
\pgfsys@useobject{currentmarker}{}%
\end{pgfscope}%
\end{pgfscope}%
\begin{pgfscope}%
\definecolor{textcolor}{rgb}{0.000000,0.000000,0.000000}%
\pgfsetstrokecolor{textcolor}%
\pgfsetfillcolor{textcolor}%
\pgftext[x=0.322221in, y=12.943707in, left, base]{\color{textcolor}\sffamily\fontsize{10.000000}{12.000000}\selectfont 0.5}%
\end{pgfscope}%
\begin{pgfscope}%
\pgfpathrectangle{\pgfqpoint{0.640323in}{9.767436in}}{\pgfqpoint{9.687500in}{3.850000in}}%
\pgfusepath{clip}%
\pgfsetrectcap%
\pgfsetroundjoin%
\pgfsetlinewidth{0.803000pt}%
\definecolor{currentstroke}{rgb}{0.690196,0.690196,0.690196}%
\pgfsetstrokecolor{currentstroke}%
\pgfsetdash{}{0pt}%
\pgfpathmoveto{\pgfqpoint{0.640323in}{13.617436in}}%
\pgfpathlineto{\pgfqpoint{10.327822in}{13.617436in}}%
\pgfusepath{stroke}%
\end{pgfscope}%
\begin{pgfscope}%
\pgfsetbuttcap%
\pgfsetroundjoin%
\definecolor{currentfill}{rgb}{0.000000,0.000000,0.000000}%
\pgfsetfillcolor{currentfill}%
\pgfsetlinewidth{0.803000pt}%
\definecolor{currentstroke}{rgb}{0.000000,0.000000,0.000000}%
\pgfsetstrokecolor{currentstroke}%
\pgfsetdash{}{0pt}%
\pgfsys@defobject{currentmarker}{\pgfqpoint{-0.048611in}{0.000000in}}{\pgfqpoint{-0.000000in}{0.000000in}}{%
\pgfpathmoveto{\pgfqpoint{-0.000000in}{0.000000in}}%
\pgfpathlineto{\pgfqpoint{-0.048611in}{0.000000in}}%
\pgfusepath{stroke,fill}%
}%
\begin{pgfscope}%
\pgfsys@transformshift{0.640323in}{13.617436in}%
\pgfsys@useobject{currentmarker}{}%
\end{pgfscope}%
\end{pgfscope}%
\begin{pgfscope}%
\definecolor{textcolor}{rgb}{0.000000,0.000000,0.000000}%
\pgfsetstrokecolor{textcolor}%
\pgfsetfillcolor{textcolor}%
\pgftext[x=0.322221in, y=13.564674in, left, base]{\color{textcolor}\sffamily\fontsize{10.000000}{12.000000}\selectfont 0.6}%
\end{pgfscope}%
\begin{pgfscope}%
\pgfpathrectangle{\pgfqpoint{0.640323in}{9.767436in}}{\pgfqpoint{9.687500in}{3.850000in}}%
\pgfusepath{clip}%
\pgfsetrectcap%
\pgfsetroundjoin%
\pgfsetlinewidth{0.803000pt}%
\definecolor{currentstroke}{rgb}{0.600000,0.600000,0.600000}%
\pgfsetstrokecolor{currentstroke}%
\pgfsetstrokeopacity{0.200000}%
\pgfsetdash{}{0pt}%
\pgfpathmoveto{\pgfqpoint{0.640323in}{10.015823in}}%
\pgfpathlineto{\pgfqpoint{10.327822in}{10.015823in}}%
\pgfusepath{stroke}%
\end{pgfscope}%
\begin{pgfscope}%
\pgfsetbuttcap%
\pgfsetroundjoin%
\definecolor{currentfill}{rgb}{0.000000,0.000000,0.000000}%
\pgfsetfillcolor{currentfill}%
\pgfsetlinewidth{0.602250pt}%
\definecolor{currentstroke}{rgb}{0.000000,0.000000,0.000000}%
\pgfsetstrokecolor{currentstroke}%
\pgfsetdash{}{0pt}%
\pgfsys@defobject{currentmarker}{\pgfqpoint{-0.027778in}{0.000000in}}{\pgfqpoint{-0.000000in}{0.000000in}}{%
\pgfpathmoveto{\pgfqpoint{-0.000000in}{0.000000in}}%
\pgfpathlineto{\pgfqpoint{-0.027778in}{0.000000in}}%
\pgfusepath{stroke,fill}%
}%
\begin{pgfscope}%
\pgfsys@transformshift{0.640323in}{10.015823in}%
\pgfsys@useobject{currentmarker}{}%
\end{pgfscope}%
\end{pgfscope}%
\begin{pgfscope}%
\pgfpathrectangle{\pgfqpoint{0.640323in}{9.767436in}}{\pgfqpoint{9.687500in}{3.850000in}}%
\pgfusepath{clip}%
\pgfsetrectcap%
\pgfsetroundjoin%
\pgfsetlinewidth{0.803000pt}%
\definecolor{currentstroke}{rgb}{0.600000,0.600000,0.600000}%
\pgfsetstrokecolor{currentstroke}%
\pgfsetstrokeopacity{0.200000}%
\pgfsetdash{}{0pt}%
\pgfpathmoveto{\pgfqpoint{0.640323in}{10.140016in}}%
\pgfpathlineto{\pgfqpoint{10.327822in}{10.140016in}}%
\pgfusepath{stroke}%
\end{pgfscope}%
\begin{pgfscope}%
\pgfsetbuttcap%
\pgfsetroundjoin%
\definecolor{currentfill}{rgb}{0.000000,0.000000,0.000000}%
\pgfsetfillcolor{currentfill}%
\pgfsetlinewidth{0.602250pt}%
\definecolor{currentstroke}{rgb}{0.000000,0.000000,0.000000}%
\pgfsetstrokecolor{currentstroke}%
\pgfsetdash{}{0pt}%
\pgfsys@defobject{currentmarker}{\pgfqpoint{-0.027778in}{0.000000in}}{\pgfqpoint{-0.000000in}{0.000000in}}{%
\pgfpathmoveto{\pgfqpoint{-0.000000in}{0.000000in}}%
\pgfpathlineto{\pgfqpoint{-0.027778in}{0.000000in}}%
\pgfusepath{stroke,fill}%
}%
\begin{pgfscope}%
\pgfsys@transformshift{0.640323in}{10.140016in}%
\pgfsys@useobject{currentmarker}{}%
\end{pgfscope}%
\end{pgfscope}%
\begin{pgfscope}%
\pgfpathrectangle{\pgfqpoint{0.640323in}{9.767436in}}{\pgfqpoint{9.687500in}{3.850000in}}%
\pgfusepath{clip}%
\pgfsetrectcap%
\pgfsetroundjoin%
\pgfsetlinewidth{0.803000pt}%
\definecolor{currentstroke}{rgb}{0.600000,0.600000,0.600000}%
\pgfsetstrokecolor{currentstroke}%
\pgfsetstrokeopacity{0.200000}%
\pgfsetdash{}{0pt}%
\pgfpathmoveto{\pgfqpoint{0.640323in}{10.264210in}}%
\pgfpathlineto{\pgfqpoint{10.327822in}{10.264210in}}%
\pgfusepath{stroke}%
\end{pgfscope}%
\begin{pgfscope}%
\pgfsetbuttcap%
\pgfsetroundjoin%
\definecolor{currentfill}{rgb}{0.000000,0.000000,0.000000}%
\pgfsetfillcolor{currentfill}%
\pgfsetlinewidth{0.602250pt}%
\definecolor{currentstroke}{rgb}{0.000000,0.000000,0.000000}%
\pgfsetstrokecolor{currentstroke}%
\pgfsetdash{}{0pt}%
\pgfsys@defobject{currentmarker}{\pgfqpoint{-0.027778in}{0.000000in}}{\pgfqpoint{-0.000000in}{0.000000in}}{%
\pgfpathmoveto{\pgfqpoint{-0.000000in}{0.000000in}}%
\pgfpathlineto{\pgfqpoint{-0.027778in}{0.000000in}}%
\pgfusepath{stroke,fill}%
}%
\begin{pgfscope}%
\pgfsys@transformshift{0.640323in}{10.264210in}%
\pgfsys@useobject{currentmarker}{}%
\end{pgfscope}%
\end{pgfscope}%
\begin{pgfscope}%
\pgfpathrectangle{\pgfqpoint{0.640323in}{9.767436in}}{\pgfqpoint{9.687500in}{3.850000in}}%
\pgfusepath{clip}%
\pgfsetrectcap%
\pgfsetroundjoin%
\pgfsetlinewidth{0.803000pt}%
\definecolor{currentstroke}{rgb}{0.600000,0.600000,0.600000}%
\pgfsetstrokecolor{currentstroke}%
\pgfsetstrokeopacity{0.200000}%
\pgfsetdash{}{0pt}%
\pgfpathmoveto{\pgfqpoint{0.640323in}{10.388404in}}%
\pgfpathlineto{\pgfqpoint{10.327822in}{10.388404in}}%
\pgfusepath{stroke}%
\end{pgfscope}%
\begin{pgfscope}%
\pgfsetbuttcap%
\pgfsetroundjoin%
\definecolor{currentfill}{rgb}{0.000000,0.000000,0.000000}%
\pgfsetfillcolor{currentfill}%
\pgfsetlinewidth{0.602250pt}%
\definecolor{currentstroke}{rgb}{0.000000,0.000000,0.000000}%
\pgfsetstrokecolor{currentstroke}%
\pgfsetdash{}{0pt}%
\pgfsys@defobject{currentmarker}{\pgfqpoint{-0.027778in}{0.000000in}}{\pgfqpoint{-0.000000in}{0.000000in}}{%
\pgfpathmoveto{\pgfqpoint{-0.000000in}{0.000000in}}%
\pgfpathlineto{\pgfqpoint{-0.027778in}{0.000000in}}%
\pgfusepath{stroke,fill}%
}%
\begin{pgfscope}%
\pgfsys@transformshift{0.640323in}{10.388404in}%
\pgfsys@useobject{currentmarker}{}%
\end{pgfscope}%
\end{pgfscope}%
\begin{pgfscope}%
\pgfpathrectangle{\pgfqpoint{0.640323in}{9.767436in}}{\pgfqpoint{9.687500in}{3.850000in}}%
\pgfusepath{clip}%
\pgfsetrectcap%
\pgfsetroundjoin%
\pgfsetlinewidth{0.803000pt}%
\definecolor{currentstroke}{rgb}{0.600000,0.600000,0.600000}%
\pgfsetstrokecolor{currentstroke}%
\pgfsetstrokeopacity{0.200000}%
\pgfsetdash{}{0pt}%
\pgfpathmoveto{\pgfqpoint{0.640323in}{10.636791in}}%
\pgfpathlineto{\pgfqpoint{10.327822in}{10.636791in}}%
\pgfusepath{stroke}%
\end{pgfscope}%
\begin{pgfscope}%
\pgfsetbuttcap%
\pgfsetroundjoin%
\definecolor{currentfill}{rgb}{0.000000,0.000000,0.000000}%
\pgfsetfillcolor{currentfill}%
\pgfsetlinewidth{0.602250pt}%
\definecolor{currentstroke}{rgb}{0.000000,0.000000,0.000000}%
\pgfsetstrokecolor{currentstroke}%
\pgfsetdash{}{0pt}%
\pgfsys@defobject{currentmarker}{\pgfqpoint{-0.027778in}{0.000000in}}{\pgfqpoint{-0.000000in}{0.000000in}}{%
\pgfpathmoveto{\pgfqpoint{-0.000000in}{0.000000in}}%
\pgfpathlineto{\pgfqpoint{-0.027778in}{0.000000in}}%
\pgfusepath{stroke,fill}%
}%
\begin{pgfscope}%
\pgfsys@transformshift{0.640323in}{10.636791in}%
\pgfsys@useobject{currentmarker}{}%
\end{pgfscope}%
\end{pgfscope}%
\begin{pgfscope}%
\pgfpathrectangle{\pgfqpoint{0.640323in}{9.767436in}}{\pgfqpoint{9.687500in}{3.850000in}}%
\pgfusepath{clip}%
\pgfsetrectcap%
\pgfsetroundjoin%
\pgfsetlinewidth{0.803000pt}%
\definecolor{currentstroke}{rgb}{0.600000,0.600000,0.600000}%
\pgfsetstrokecolor{currentstroke}%
\pgfsetstrokeopacity{0.200000}%
\pgfsetdash{}{0pt}%
\pgfpathmoveto{\pgfqpoint{0.640323in}{10.760984in}}%
\pgfpathlineto{\pgfqpoint{10.327822in}{10.760984in}}%
\pgfusepath{stroke}%
\end{pgfscope}%
\begin{pgfscope}%
\pgfsetbuttcap%
\pgfsetroundjoin%
\definecolor{currentfill}{rgb}{0.000000,0.000000,0.000000}%
\pgfsetfillcolor{currentfill}%
\pgfsetlinewidth{0.602250pt}%
\definecolor{currentstroke}{rgb}{0.000000,0.000000,0.000000}%
\pgfsetstrokecolor{currentstroke}%
\pgfsetdash{}{0pt}%
\pgfsys@defobject{currentmarker}{\pgfqpoint{-0.027778in}{0.000000in}}{\pgfqpoint{-0.000000in}{0.000000in}}{%
\pgfpathmoveto{\pgfqpoint{-0.000000in}{0.000000in}}%
\pgfpathlineto{\pgfqpoint{-0.027778in}{0.000000in}}%
\pgfusepath{stroke,fill}%
}%
\begin{pgfscope}%
\pgfsys@transformshift{0.640323in}{10.760984in}%
\pgfsys@useobject{currentmarker}{}%
\end{pgfscope}%
\end{pgfscope}%
\begin{pgfscope}%
\pgfpathrectangle{\pgfqpoint{0.640323in}{9.767436in}}{\pgfqpoint{9.687500in}{3.850000in}}%
\pgfusepath{clip}%
\pgfsetrectcap%
\pgfsetroundjoin%
\pgfsetlinewidth{0.803000pt}%
\definecolor{currentstroke}{rgb}{0.600000,0.600000,0.600000}%
\pgfsetstrokecolor{currentstroke}%
\pgfsetstrokeopacity{0.200000}%
\pgfsetdash{}{0pt}%
\pgfpathmoveto{\pgfqpoint{0.640323in}{10.885178in}}%
\pgfpathlineto{\pgfqpoint{10.327822in}{10.885178in}}%
\pgfusepath{stroke}%
\end{pgfscope}%
\begin{pgfscope}%
\pgfsetbuttcap%
\pgfsetroundjoin%
\definecolor{currentfill}{rgb}{0.000000,0.000000,0.000000}%
\pgfsetfillcolor{currentfill}%
\pgfsetlinewidth{0.602250pt}%
\definecolor{currentstroke}{rgb}{0.000000,0.000000,0.000000}%
\pgfsetstrokecolor{currentstroke}%
\pgfsetdash{}{0pt}%
\pgfsys@defobject{currentmarker}{\pgfqpoint{-0.027778in}{0.000000in}}{\pgfqpoint{-0.000000in}{0.000000in}}{%
\pgfpathmoveto{\pgfqpoint{-0.000000in}{0.000000in}}%
\pgfpathlineto{\pgfqpoint{-0.027778in}{0.000000in}}%
\pgfusepath{stroke,fill}%
}%
\begin{pgfscope}%
\pgfsys@transformshift{0.640323in}{10.885178in}%
\pgfsys@useobject{currentmarker}{}%
\end{pgfscope}%
\end{pgfscope}%
\begin{pgfscope}%
\pgfpathrectangle{\pgfqpoint{0.640323in}{9.767436in}}{\pgfqpoint{9.687500in}{3.850000in}}%
\pgfusepath{clip}%
\pgfsetrectcap%
\pgfsetroundjoin%
\pgfsetlinewidth{0.803000pt}%
\definecolor{currentstroke}{rgb}{0.600000,0.600000,0.600000}%
\pgfsetstrokecolor{currentstroke}%
\pgfsetstrokeopacity{0.200000}%
\pgfsetdash{}{0pt}%
\pgfpathmoveto{\pgfqpoint{0.640323in}{11.009371in}}%
\pgfpathlineto{\pgfqpoint{10.327822in}{11.009371in}}%
\pgfusepath{stroke}%
\end{pgfscope}%
\begin{pgfscope}%
\pgfsetbuttcap%
\pgfsetroundjoin%
\definecolor{currentfill}{rgb}{0.000000,0.000000,0.000000}%
\pgfsetfillcolor{currentfill}%
\pgfsetlinewidth{0.602250pt}%
\definecolor{currentstroke}{rgb}{0.000000,0.000000,0.000000}%
\pgfsetstrokecolor{currentstroke}%
\pgfsetdash{}{0pt}%
\pgfsys@defobject{currentmarker}{\pgfqpoint{-0.027778in}{0.000000in}}{\pgfqpoint{-0.000000in}{0.000000in}}{%
\pgfpathmoveto{\pgfqpoint{-0.000000in}{0.000000in}}%
\pgfpathlineto{\pgfqpoint{-0.027778in}{0.000000in}}%
\pgfusepath{stroke,fill}%
}%
\begin{pgfscope}%
\pgfsys@transformshift{0.640323in}{11.009371in}%
\pgfsys@useobject{currentmarker}{}%
\end{pgfscope}%
\end{pgfscope}%
\begin{pgfscope}%
\pgfpathrectangle{\pgfqpoint{0.640323in}{9.767436in}}{\pgfqpoint{9.687500in}{3.850000in}}%
\pgfusepath{clip}%
\pgfsetrectcap%
\pgfsetroundjoin%
\pgfsetlinewidth{0.803000pt}%
\definecolor{currentstroke}{rgb}{0.600000,0.600000,0.600000}%
\pgfsetstrokecolor{currentstroke}%
\pgfsetstrokeopacity{0.200000}%
\pgfsetdash{}{0pt}%
\pgfpathmoveto{\pgfqpoint{0.640323in}{11.257758in}}%
\pgfpathlineto{\pgfqpoint{10.327822in}{11.257758in}}%
\pgfusepath{stroke}%
\end{pgfscope}%
\begin{pgfscope}%
\pgfsetbuttcap%
\pgfsetroundjoin%
\definecolor{currentfill}{rgb}{0.000000,0.000000,0.000000}%
\pgfsetfillcolor{currentfill}%
\pgfsetlinewidth{0.602250pt}%
\definecolor{currentstroke}{rgb}{0.000000,0.000000,0.000000}%
\pgfsetstrokecolor{currentstroke}%
\pgfsetdash{}{0pt}%
\pgfsys@defobject{currentmarker}{\pgfqpoint{-0.027778in}{0.000000in}}{\pgfqpoint{-0.000000in}{0.000000in}}{%
\pgfpathmoveto{\pgfqpoint{-0.000000in}{0.000000in}}%
\pgfpathlineto{\pgfqpoint{-0.027778in}{0.000000in}}%
\pgfusepath{stroke,fill}%
}%
\begin{pgfscope}%
\pgfsys@transformshift{0.640323in}{11.257758in}%
\pgfsys@useobject{currentmarker}{}%
\end{pgfscope}%
\end{pgfscope}%
\begin{pgfscope}%
\pgfpathrectangle{\pgfqpoint{0.640323in}{9.767436in}}{\pgfqpoint{9.687500in}{3.850000in}}%
\pgfusepath{clip}%
\pgfsetrectcap%
\pgfsetroundjoin%
\pgfsetlinewidth{0.803000pt}%
\definecolor{currentstroke}{rgb}{0.600000,0.600000,0.600000}%
\pgfsetstrokecolor{currentstroke}%
\pgfsetstrokeopacity{0.200000}%
\pgfsetdash{}{0pt}%
\pgfpathmoveto{\pgfqpoint{0.640323in}{11.381952in}}%
\pgfpathlineto{\pgfqpoint{10.327822in}{11.381952in}}%
\pgfusepath{stroke}%
\end{pgfscope}%
\begin{pgfscope}%
\pgfsetbuttcap%
\pgfsetroundjoin%
\definecolor{currentfill}{rgb}{0.000000,0.000000,0.000000}%
\pgfsetfillcolor{currentfill}%
\pgfsetlinewidth{0.602250pt}%
\definecolor{currentstroke}{rgb}{0.000000,0.000000,0.000000}%
\pgfsetstrokecolor{currentstroke}%
\pgfsetdash{}{0pt}%
\pgfsys@defobject{currentmarker}{\pgfqpoint{-0.027778in}{0.000000in}}{\pgfqpoint{-0.000000in}{0.000000in}}{%
\pgfpathmoveto{\pgfqpoint{-0.000000in}{0.000000in}}%
\pgfpathlineto{\pgfqpoint{-0.027778in}{0.000000in}}%
\pgfusepath{stroke,fill}%
}%
\begin{pgfscope}%
\pgfsys@transformshift{0.640323in}{11.381952in}%
\pgfsys@useobject{currentmarker}{}%
\end{pgfscope}%
\end{pgfscope}%
\begin{pgfscope}%
\pgfpathrectangle{\pgfqpoint{0.640323in}{9.767436in}}{\pgfqpoint{9.687500in}{3.850000in}}%
\pgfusepath{clip}%
\pgfsetrectcap%
\pgfsetroundjoin%
\pgfsetlinewidth{0.803000pt}%
\definecolor{currentstroke}{rgb}{0.600000,0.600000,0.600000}%
\pgfsetstrokecolor{currentstroke}%
\pgfsetstrokeopacity{0.200000}%
\pgfsetdash{}{0pt}%
\pgfpathmoveto{\pgfqpoint{0.640323in}{11.506146in}}%
\pgfpathlineto{\pgfqpoint{10.327822in}{11.506146in}}%
\pgfusepath{stroke}%
\end{pgfscope}%
\begin{pgfscope}%
\pgfsetbuttcap%
\pgfsetroundjoin%
\definecolor{currentfill}{rgb}{0.000000,0.000000,0.000000}%
\pgfsetfillcolor{currentfill}%
\pgfsetlinewidth{0.602250pt}%
\definecolor{currentstroke}{rgb}{0.000000,0.000000,0.000000}%
\pgfsetstrokecolor{currentstroke}%
\pgfsetdash{}{0pt}%
\pgfsys@defobject{currentmarker}{\pgfqpoint{-0.027778in}{0.000000in}}{\pgfqpoint{-0.000000in}{0.000000in}}{%
\pgfpathmoveto{\pgfqpoint{-0.000000in}{0.000000in}}%
\pgfpathlineto{\pgfqpoint{-0.027778in}{0.000000in}}%
\pgfusepath{stroke,fill}%
}%
\begin{pgfscope}%
\pgfsys@transformshift{0.640323in}{11.506146in}%
\pgfsys@useobject{currentmarker}{}%
\end{pgfscope}%
\end{pgfscope}%
\begin{pgfscope}%
\pgfpathrectangle{\pgfqpoint{0.640323in}{9.767436in}}{\pgfqpoint{9.687500in}{3.850000in}}%
\pgfusepath{clip}%
\pgfsetrectcap%
\pgfsetroundjoin%
\pgfsetlinewidth{0.803000pt}%
\definecolor{currentstroke}{rgb}{0.600000,0.600000,0.600000}%
\pgfsetstrokecolor{currentstroke}%
\pgfsetstrokeopacity{0.200000}%
\pgfsetdash{}{0pt}%
\pgfpathmoveto{\pgfqpoint{0.640323in}{11.630339in}}%
\pgfpathlineto{\pgfqpoint{10.327822in}{11.630339in}}%
\pgfusepath{stroke}%
\end{pgfscope}%
\begin{pgfscope}%
\pgfsetbuttcap%
\pgfsetroundjoin%
\definecolor{currentfill}{rgb}{0.000000,0.000000,0.000000}%
\pgfsetfillcolor{currentfill}%
\pgfsetlinewidth{0.602250pt}%
\definecolor{currentstroke}{rgb}{0.000000,0.000000,0.000000}%
\pgfsetstrokecolor{currentstroke}%
\pgfsetdash{}{0pt}%
\pgfsys@defobject{currentmarker}{\pgfqpoint{-0.027778in}{0.000000in}}{\pgfqpoint{-0.000000in}{0.000000in}}{%
\pgfpathmoveto{\pgfqpoint{-0.000000in}{0.000000in}}%
\pgfpathlineto{\pgfqpoint{-0.027778in}{0.000000in}}%
\pgfusepath{stroke,fill}%
}%
\begin{pgfscope}%
\pgfsys@transformshift{0.640323in}{11.630339in}%
\pgfsys@useobject{currentmarker}{}%
\end{pgfscope}%
\end{pgfscope}%
\begin{pgfscope}%
\pgfpathrectangle{\pgfqpoint{0.640323in}{9.767436in}}{\pgfqpoint{9.687500in}{3.850000in}}%
\pgfusepath{clip}%
\pgfsetrectcap%
\pgfsetroundjoin%
\pgfsetlinewidth{0.803000pt}%
\definecolor{currentstroke}{rgb}{0.600000,0.600000,0.600000}%
\pgfsetstrokecolor{currentstroke}%
\pgfsetstrokeopacity{0.200000}%
\pgfsetdash{}{0pt}%
\pgfpathmoveto{\pgfqpoint{0.640323in}{11.878726in}}%
\pgfpathlineto{\pgfqpoint{10.327822in}{11.878726in}}%
\pgfusepath{stroke}%
\end{pgfscope}%
\begin{pgfscope}%
\pgfsetbuttcap%
\pgfsetroundjoin%
\definecolor{currentfill}{rgb}{0.000000,0.000000,0.000000}%
\pgfsetfillcolor{currentfill}%
\pgfsetlinewidth{0.602250pt}%
\definecolor{currentstroke}{rgb}{0.000000,0.000000,0.000000}%
\pgfsetstrokecolor{currentstroke}%
\pgfsetdash{}{0pt}%
\pgfsys@defobject{currentmarker}{\pgfqpoint{-0.027778in}{0.000000in}}{\pgfqpoint{-0.000000in}{0.000000in}}{%
\pgfpathmoveto{\pgfqpoint{-0.000000in}{0.000000in}}%
\pgfpathlineto{\pgfqpoint{-0.027778in}{0.000000in}}%
\pgfusepath{stroke,fill}%
}%
\begin{pgfscope}%
\pgfsys@transformshift{0.640323in}{11.878726in}%
\pgfsys@useobject{currentmarker}{}%
\end{pgfscope}%
\end{pgfscope}%
\begin{pgfscope}%
\pgfpathrectangle{\pgfqpoint{0.640323in}{9.767436in}}{\pgfqpoint{9.687500in}{3.850000in}}%
\pgfusepath{clip}%
\pgfsetrectcap%
\pgfsetroundjoin%
\pgfsetlinewidth{0.803000pt}%
\definecolor{currentstroke}{rgb}{0.600000,0.600000,0.600000}%
\pgfsetstrokecolor{currentstroke}%
\pgfsetstrokeopacity{0.200000}%
\pgfsetdash{}{0pt}%
\pgfpathmoveto{\pgfqpoint{0.640323in}{12.002920in}}%
\pgfpathlineto{\pgfqpoint{10.327822in}{12.002920in}}%
\pgfusepath{stroke}%
\end{pgfscope}%
\begin{pgfscope}%
\pgfsetbuttcap%
\pgfsetroundjoin%
\definecolor{currentfill}{rgb}{0.000000,0.000000,0.000000}%
\pgfsetfillcolor{currentfill}%
\pgfsetlinewidth{0.602250pt}%
\definecolor{currentstroke}{rgb}{0.000000,0.000000,0.000000}%
\pgfsetstrokecolor{currentstroke}%
\pgfsetdash{}{0pt}%
\pgfsys@defobject{currentmarker}{\pgfqpoint{-0.027778in}{0.000000in}}{\pgfqpoint{-0.000000in}{0.000000in}}{%
\pgfpathmoveto{\pgfqpoint{-0.000000in}{0.000000in}}%
\pgfpathlineto{\pgfqpoint{-0.027778in}{0.000000in}}%
\pgfusepath{stroke,fill}%
}%
\begin{pgfscope}%
\pgfsys@transformshift{0.640323in}{12.002920in}%
\pgfsys@useobject{currentmarker}{}%
\end{pgfscope}%
\end{pgfscope}%
\begin{pgfscope}%
\pgfpathrectangle{\pgfqpoint{0.640323in}{9.767436in}}{\pgfqpoint{9.687500in}{3.850000in}}%
\pgfusepath{clip}%
\pgfsetrectcap%
\pgfsetroundjoin%
\pgfsetlinewidth{0.803000pt}%
\definecolor{currentstroke}{rgb}{0.600000,0.600000,0.600000}%
\pgfsetstrokecolor{currentstroke}%
\pgfsetstrokeopacity{0.200000}%
\pgfsetdash{}{0pt}%
\pgfpathmoveto{\pgfqpoint{0.640323in}{12.127113in}}%
\pgfpathlineto{\pgfqpoint{10.327822in}{12.127113in}}%
\pgfusepath{stroke}%
\end{pgfscope}%
\begin{pgfscope}%
\pgfsetbuttcap%
\pgfsetroundjoin%
\definecolor{currentfill}{rgb}{0.000000,0.000000,0.000000}%
\pgfsetfillcolor{currentfill}%
\pgfsetlinewidth{0.602250pt}%
\definecolor{currentstroke}{rgb}{0.000000,0.000000,0.000000}%
\pgfsetstrokecolor{currentstroke}%
\pgfsetdash{}{0pt}%
\pgfsys@defobject{currentmarker}{\pgfqpoint{-0.027778in}{0.000000in}}{\pgfqpoint{-0.000000in}{0.000000in}}{%
\pgfpathmoveto{\pgfqpoint{-0.000000in}{0.000000in}}%
\pgfpathlineto{\pgfqpoint{-0.027778in}{0.000000in}}%
\pgfusepath{stroke,fill}%
}%
\begin{pgfscope}%
\pgfsys@transformshift{0.640323in}{12.127113in}%
\pgfsys@useobject{currentmarker}{}%
\end{pgfscope}%
\end{pgfscope}%
\begin{pgfscope}%
\pgfpathrectangle{\pgfqpoint{0.640323in}{9.767436in}}{\pgfqpoint{9.687500in}{3.850000in}}%
\pgfusepath{clip}%
\pgfsetrectcap%
\pgfsetroundjoin%
\pgfsetlinewidth{0.803000pt}%
\definecolor{currentstroke}{rgb}{0.600000,0.600000,0.600000}%
\pgfsetstrokecolor{currentstroke}%
\pgfsetstrokeopacity{0.200000}%
\pgfsetdash{}{0pt}%
\pgfpathmoveto{\pgfqpoint{0.640323in}{12.251307in}}%
\pgfpathlineto{\pgfqpoint{10.327822in}{12.251307in}}%
\pgfusepath{stroke}%
\end{pgfscope}%
\begin{pgfscope}%
\pgfsetbuttcap%
\pgfsetroundjoin%
\definecolor{currentfill}{rgb}{0.000000,0.000000,0.000000}%
\pgfsetfillcolor{currentfill}%
\pgfsetlinewidth{0.602250pt}%
\definecolor{currentstroke}{rgb}{0.000000,0.000000,0.000000}%
\pgfsetstrokecolor{currentstroke}%
\pgfsetdash{}{0pt}%
\pgfsys@defobject{currentmarker}{\pgfqpoint{-0.027778in}{0.000000in}}{\pgfqpoint{-0.000000in}{0.000000in}}{%
\pgfpathmoveto{\pgfqpoint{-0.000000in}{0.000000in}}%
\pgfpathlineto{\pgfqpoint{-0.027778in}{0.000000in}}%
\pgfusepath{stroke,fill}%
}%
\begin{pgfscope}%
\pgfsys@transformshift{0.640323in}{12.251307in}%
\pgfsys@useobject{currentmarker}{}%
\end{pgfscope}%
\end{pgfscope}%
\begin{pgfscope}%
\pgfpathrectangle{\pgfqpoint{0.640323in}{9.767436in}}{\pgfqpoint{9.687500in}{3.850000in}}%
\pgfusepath{clip}%
\pgfsetrectcap%
\pgfsetroundjoin%
\pgfsetlinewidth{0.803000pt}%
\definecolor{currentstroke}{rgb}{0.600000,0.600000,0.600000}%
\pgfsetstrokecolor{currentstroke}%
\pgfsetstrokeopacity{0.200000}%
\pgfsetdash{}{0pt}%
\pgfpathmoveto{\pgfqpoint{0.640323in}{12.499694in}}%
\pgfpathlineto{\pgfqpoint{10.327822in}{12.499694in}}%
\pgfusepath{stroke}%
\end{pgfscope}%
\begin{pgfscope}%
\pgfsetbuttcap%
\pgfsetroundjoin%
\definecolor{currentfill}{rgb}{0.000000,0.000000,0.000000}%
\pgfsetfillcolor{currentfill}%
\pgfsetlinewidth{0.602250pt}%
\definecolor{currentstroke}{rgb}{0.000000,0.000000,0.000000}%
\pgfsetstrokecolor{currentstroke}%
\pgfsetdash{}{0pt}%
\pgfsys@defobject{currentmarker}{\pgfqpoint{-0.027778in}{0.000000in}}{\pgfqpoint{-0.000000in}{0.000000in}}{%
\pgfpathmoveto{\pgfqpoint{-0.000000in}{0.000000in}}%
\pgfpathlineto{\pgfqpoint{-0.027778in}{0.000000in}}%
\pgfusepath{stroke,fill}%
}%
\begin{pgfscope}%
\pgfsys@transformshift{0.640323in}{12.499694in}%
\pgfsys@useobject{currentmarker}{}%
\end{pgfscope}%
\end{pgfscope}%
\begin{pgfscope}%
\pgfpathrectangle{\pgfqpoint{0.640323in}{9.767436in}}{\pgfqpoint{9.687500in}{3.850000in}}%
\pgfusepath{clip}%
\pgfsetrectcap%
\pgfsetroundjoin%
\pgfsetlinewidth{0.803000pt}%
\definecolor{currentstroke}{rgb}{0.600000,0.600000,0.600000}%
\pgfsetstrokecolor{currentstroke}%
\pgfsetstrokeopacity{0.200000}%
\pgfsetdash{}{0pt}%
\pgfpathmoveto{\pgfqpoint{0.640323in}{12.623887in}}%
\pgfpathlineto{\pgfqpoint{10.327822in}{12.623887in}}%
\pgfusepath{stroke}%
\end{pgfscope}%
\begin{pgfscope}%
\pgfsetbuttcap%
\pgfsetroundjoin%
\definecolor{currentfill}{rgb}{0.000000,0.000000,0.000000}%
\pgfsetfillcolor{currentfill}%
\pgfsetlinewidth{0.602250pt}%
\definecolor{currentstroke}{rgb}{0.000000,0.000000,0.000000}%
\pgfsetstrokecolor{currentstroke}%
\pgfsetdash{}{0pt}%
\pgfsys@defobject{currentmarker}{\pgfqpoint{-0.027778in}{0.000000in}}{\pgfqpoint{-0.000000in}{0.000000in}}{%
\pgfpathmoveto{\pgfqpoint{-0.000000in}{0.000000in}}%
\pgfpathlineto{\pgfqpoint{-0.027778in}{0.000000in}}%
\pgfusepath{stroke,fill}%
}%
\begin{pgfscope}%
\pgfsys@transformshift{0.640323in}{12.623887in}%
\pgfsys@useobject{currentmarker}{}%
\end{pgfscope}%
\end{pgfscope}%
\begin{pgfscope}%
\pgfpathrectangle{\pgfqpoint{0.640323in}{9.767436in}}{\pgfqpoint{9.687500in}{3.850000in}}%
\pgfusepath{clip}%
\pgfsetrectcap%
\pgfsetroundjoin%
\pgfsetlinewidth{0.803000pt}%
\definecolor{currentstroke}{rgb}{0.600000,0.600000,0.600000}%
\pgfsetstrokecolor{currentstroke}%
\pgfsetstrokeopacity{0.200000}%
\pgfsetdash{}{0pt}%
\pgfpathmoveto{\pgfqpoint{0.640323in}{12.748081in}}%
\pgfpathlineto{\pgfqpoint{10.327822in}{12.748081in}}%
\pgfusepath{stroke}%
\end{pgfscope}%
\begin{pgfscope}%
\pgfsetbuttcap%
\pgfsetroundjoin%
\definecolor{currentfill}{rgb}{0.000000,0.000000,0.000000}%
\pgfsetfillcolor{currentfill}%
\pgfsetlinewidth{0.602250pt}%
\definecolor{currentstroke}{rgb}{0.000000,0.000000,0.000000}%
\pgfsetstrokecolor{currentstroke}%
\pgfsetdash{}{0pt}%
\pgfsys@defobject{currentmarker}{\pgfqpoint{-0.027778in}{0.000000in}}{\pgfqpoint{-0.000000in}{0.000000in}}{%
\pgfpathmoveto{\pgfqpoint{-0.000000in}{0.000000in}}%
\pgfpathlineto{\pgfqpoint{-0.027778in}{0.000000in}}%
\pgfusepath{stroke,fill}%
}%
\begin{pgfscope}%
\pgfsys@transformshift{0.640323in}{12.748081in}%
\pgfsys@useobject{currentmarker}{}%
\end{pgfscope}%
\end{pgfscope}%
\begin{pgfscope}%
\pgfpathrectangle{\pgfqpoint{0.640323in}{9.767436in}}{\pgfqpoint{9.687500in}{3.850000in}}%
\pgfusepath{clip}%
\pgfsetrectcap%
\pgfsetroundjoin%
\pgfsetlinewidth{0.803000pt}%
\definecolor{currentstroke}{rgb}{0.600000,0.600000,0.600000}%
\pgfsetstrokecolor{currentstroke}%
\pgfsetstrokeopacity{0.200000}%
\pgfsetdash{}{0pt}%
\pgfpathmoveto{\pgfqpoint{0.640323in}{12.872275in}}%
\pgfpathlineto{\pgfqpoint{10.327822in}{12.872275in}}%
\pgfusepath{stroke}%
\end{pgfscope}%
\begin{pgfscope}%
\pgfsetbuttcap%
\pgfsetroundjoin%
\definecolor{currentfill}{rgb}{0.000000,0.000000,0.000000}%
\pgfsetfillcolor{currentfill}%
\pgfsetlinewidth{0.602250pt}%
\definecolor{currentstroke}{rgb}{0.000000,0.000000,0.000000}%
\pgfsetstrokecolor{currentstroke}%
\pgfsetdash{}{0pt}%
\pgfsys@defobject{currentmarker}{\pgfqpoint{-0.027778in}{0.000000in}}{\pgfqpoint{-0.000000in}{0.000000in}}{%
\pgfpathmoveto{\pgfqpoint{-0.000000in}{0.000000in}}%
\pgfpathlineto{\pgfqpoint{-0.027778in}{0.000000in}}%
\pgfusepath{stroke,fill}%
}%
\begin{pgfscope}%
\pgfsys@transformshift{0.640323in}{12.872275in}%
\pgfsys@useobject{currentmarker}{}%
\end{pgfscope}%
\end{pgfscope}%
\begin{pgfscope}%
\pgfpathrectangle{\pgfqpoint{0.640323in}{9.767436in}}{\pgfqpoint{9.687500in}{3.850000in}}%
\pgfusepath{clip}%
\pgfsetrectcap%
\pgfsetroundjoin%
\pgfsetlinewidth{0.803000pt}%
\definecolor{currentstroke}{rgb}{0.600000,0.600000,0.600000}%
\pgfsetstrokecolor{currentstroke}%
\pgfsetstrokeopacity{0.200000}%
\pgfsetdash{}{0pt}%
\pgfpathmoveto{\pgfqpoint{0.640323in}{13.120662in}}%
\pgfpathlineto{\pgfqpoint{10.327822in}{13.120662in}}%
\pgfusepath{stroke}%
\end{pgfscope}%
\begin{pgfscope}%
\pgfsetbuttcap%
\pgfsetroundjoin%
\definecolor{currentfill}{rgb}{0.000000,0.000000,0.000000}%
\pgfsetfillcolor{currentfill}%
\pgfsetlinewidth{0.602250pt}%
\definecolor{currentstroke}{rgb}{0.000000,0.000000,0.000000}%
\pgfsetstrokecolor{currentstroke}%
\pgfsetdash{}{0pt}%
\pgfsys@defobject{currentmarker}{\pgfqpoint{-0.027778in}{0.000000in}}{\pgfqpoint{-0.000000in}{0.000000in}}{%
\pgfpathmoveto{\pgfqpoint{-0.000000in}{0.000000in}}%
\pgfpathlineto{\pgfqpoint{-0.027778in}{0.000000in}}%
\pgfusepath{stroke,fill}%
}%
\begin{pgfscope}%
\pgfsys@transformshift{0.640323in}{13.120662in}%
\pgfsys@useobject{currentmarker}{}%
\end{pgfscope}%
\end{pgfscope}%
\begin{pgfscope}%
\pgfpathrectangle{\pgfqpoint{0.640323in}{9.767436in}}{\pgfqpoint{9.687500in}{3.850000in}}%
\pgfusepath{clip}%
\pgfsetrectcap%
\pgfsetroundjoin%
\pgfsetlinewidth{0.803000pt}%
\definecolor{currentstroke}{rgb}{0.600000,0.600000,0.600000}%
\pgfsetstrokecolor{currentstroke}%
\pgfsetstrokeopacity{0.200000}%
\pgfsetdash{}{0pt}%
\pgfpathmoveto{\pgfqpoint{0.640323in}{13.244855in}}%
\pgfpathlineto{\pgfqpoint{10.327822in}{13.244855in}}%
\pgfusepath{stroke}%
\end{pgfscope}%
\begin{pgfscope}%
\pgfsetbuttcap%
\pgfsetroundjoin%
\definecolor{currentfill}{rgb}{0.000000,0.000000,0.000000}%
\pgfsetfillcolor{currentfill}%
\pgfsetlinewidth{0.602250pt}%
\definecolor{currentstroke}{rgb}{0.000000,0.000000,0.000000}%
\pgfsetstrokecolor{currentstroke}%
\pgfsetdash{}{0pt}%
\pgfsys@defobject{currentmarker}{\pgfqpoint{-0.027778in}{0.000000in}}{\pgfqpoint{-0.000000in}{0.000000in}}{%
\pgfpathmoveto{\pgfqpoint{-0.000000in}{0.000000in}}%
\pgfpathlineto{\pgfqpoint{-0.027778in}{0.000000in}}%
\pgfusepath{stroke,fill}%
}%
\begin{pgfscope}%
\pgfsys@transformshift{0.640323in}{13.244855in}%
\pgfsys@useobject{currentmarker}{}%
\end{pgfscope}%
\end{pgfscope}%
\begin{pgfscope}%
\pgfpathrectangle{\pgfqpoint{0.640323in}{9.767436in}}{\pgfqpoint{9.687500in}{3.850000in}}%
\pgfusepath{clip}%
\pgfsetrectcap%
\pgfsetroundjoin%
\pgfsetlinewidth{0.803000pt}%
\definecolor{currentstroke}{rgb}{0.600000,0.600000,0.600000}%
\pgfsetstrokecolor{currentstroke}%
\pgfsetstrokeopacity{0.200000}%
\pgfsetdash{}{0pt}%
\pgfpathmoveto{\pgfqpoint{0.640323in}{13.369049in}}%
\pgfpathlineto{\pgfqpoint{10.327822in}{13.369049in}}%
\pgfusepath{stroke}%
\end{pgfscope}%
\begin{pgfscope}%
\pgfsetbuttcap%
\pgfsetroundjoin%
\definecolor{currentfill}{rgb}{0.000000,0.000000,0.000000}%
\pgfsetfillcolor{currentfill}%
\pgfsetlinewidth{0.602250pt}%
\definecolor{currentstroke}{rgb}{0.000000,0.000000,0.000000}%
\pgfsetstrokecolor{currentstroke}%
\pgfsetdash{}{0pt}%
\pgfsys@defobject{currentmarker}{\pgfqpoint{-0.027778in}{0.000000in}}{\pgfqpoint{-0.000000in}{0.000000in}}{%
\pgfpathmoveto{\pgfqpoint{-0.000000in}{0.000000in}}%
\pgfpathlineto{\pgfqpoint{-0.027778in}{0.000000in}}%
\pgfusepath{stroke,fill}%
}%
\begin{pgfscope}%
\pgfsys@transformshift{0.640323in}{13.369049in}%
\pgfsys@useobject{currentmarker}{}%
\end{pgfscope}%
\end{pgfscope}%
\begin{pgfscope}%
\pgfpathrectangle{\pgfqpoint{0.640323in}{9.767436in}}{\pgfqpoint{9.687500in}{3.850000in}}%
\pgfusepath{clip}%
\pgfsetrectcap%
\pgfsetroundjoin%
\pgfsetlinewidth{0.803000pt}%
\definecolor{currentstroke}{rgb}{0.600000,0.600000,0.600000}%
\pgfsetstrokecolor{currentstroke}%
\pgfsetstrokeopacity{0.200000}%
\pgfsetdash{}{0pt}%
\pgfpathmoveto{\pgfqpoint{0.640323in}{13.493242in}}%
\pgfpathlineto{\pgfqpoint{10.327822in}{13.493242in}}%
\pgfusepath{stroke}%
\end{pgfscope}%
\begin{pgfscope}%
\pgfsetbuttcap%
\pgfsetroundjoin%
\definecolor{currentfill}{rgb}{0.000000,0.000000,0.000000}%
\pgfsetfillcolor{currentfill}%
\pgfsetlinewidth{0.602250pt}%
\definecolor{currentstroke}{rgb}{0.000000,0.000000,0.000000}%
\pgfsetstrokecolor{currentstroke}%
\pgfsetdash{}{0pt}%
\pgfsys@defobject{currentmarker}{\pgfqpoint{-0.027778in}{0.000000in}}{\pgfqpoint{-0.000000in}{0.000000in}}{%
\pgfpathmoveto{\pgfqpoint{-0.000000in}{0.000000in}}%
\pgfpathlineto{\pgfqpoint{-0.027778in}{0.000000in}}%
\pgfusepath{stroke,fill}%
}%
\begin{pgfscope}%
\pgfsys@transformshift{0.640323in}{13.493242in}%
\pgfsys@useobject{currentmarker}{}%
\end{pgfscope}%
\end{pgfscope}%
\begin{pgfscope}%
\definecolor{textcolor}{rgb}{0.000000,0.000000,0.000000}%
\pgfsetstrokecolor{textcolor}%
\pgfsetfillcolor{textcolor}%
\pgftext[x=0.266665in,y=11.692436in,,bottom,rotate=90.000000]{\color{textcolor}\sffamily\fontsize{10.000000}{12.000000}\selectfont avg. infection rate \(\displaystyle \overline{\langle I\rangle}\)}%
\end{pgfscope}%
\begin{pgfscope}%
\pgfpathrectangle{\pgfqpoint{0.640323in}{9.767436in}}{\pgfqpoint{9.687500in}{3.850000in}}%
\pgfusepath{clip}%
\pgfsetbuttcap%
\pgfsetroundjoin%
\pgfsetlinewidth{1.003750pt}%
\definecolor{currentstroke}{rgb}{0.000000,0.000000,1.000000}%
\pgfsetstrokecolor{currentstroke}%
\pgfsetstrokeopacity{0.500000}%
\pgfsetdash{{3.700000pt}{1.600000pt}}{0.000000pt}%
\pgfpathmoveto{\pgfqpoint{1.080663in}{9.899391in}}%
\pgfpathlineto{\pgfqpoint{1.260394in}{9.900022in}}%
\pgfpathlineto{\pgfqpoint{1.440125in}{9.902399in}}%
\pgfpathlineto{\pgfqpoint{1.619856in}{9.905553in}}%
\pgfpathlineto{\pgfqpoint{1.799587in}{9.905262in}}%
\pgfpathlineto{\pgfqpoint{1.979318in}{9.913363in}}%
\pgfpathlineto{\pgfqpoint{2.159049in}{9.912830in}}%
\pgfpathlineto{\pgfqpoint{2.338780in}{9.914503in}}%
\pgfpathlineto{\pgfqpoint{2.518511in}{9.923842in}}%
\pgfpathlineto{\pgfqpoint{2.698242in}{9.949142in}}%
\pgfpathlineto{\pgfqpoint{2.877973in}{9.994186in}}%
\pgfpathlineto{\pgfqpoint{3.057704in}{10.385080in}}%
\pgfpathlineto{\pgfqpoint{3.237435in}{10.869287in}}%
\pgfpathlineto{\pgfqpoint{3.417166in}{10.938860in}}%
\pgfpathlineto{\pgfqpoint{3.596897in}{11.149381in}}%
\pgfpathlineto{\pgfqpoint{3.776628in}{11.263968in}}%
\pgfpathlineto{\pgfqpoint{3.956359in}{11.293029in}}%
\pgfpathlineto{\pgfqpoint{4.136090in}{11.472067in}}%
\pgfpathlineto{\pgfqpoint{4.315821in}{11.621192in}}%
\pgfpathlineto{\pgfqpoint{4.495552in}{11.671621in}}%
\pgfpathlineto{\pgfqpoint{4.675283in}{11.706842in}}%
\pgfpathlineto{\pgfqpoint{4.855014in}{11.764064in}}%
\pgfpathlineto{\pgfqpoint{5.034745in}{11.845157in}}%
\pgfpathlineto{\pgfqpoint{5.214476in}{11.890152in}}%
\pgfpathlineto{\pgfqpoint{5.394207in}{11.924113in}}%
\pgfpathlineto{\pgfqpoint{5.573938in}{11.921467in}}%
\pgfpathlineto{\pgfqpoint{5.753669in}{12.042022in}}%
\pgfpathlineto{\pgfqpoint{5.933400in}{12.024703in}}%
\pgfpathlineto{\pgfqpoint{6.113131in}{12.074163in}}%
\pgfpathlineto{\pgfqpoint{6.292862in}{12.109720in}}%
\pgfpathlineto{\pgfqpoint{6.472593in}{12.101331in}}%
\pgfpathlineto{\pgfqpoint{6.652324in}{12.120587in}}%
\pgfpathlineto{\pgfqpoint{6.832055in}{12.167985in}}%
\pgfpathlineto{\pgfqpoint{7.011786in}{12.193190in}}%
\pgfpathlineto{\pgfqpoint{7.191517in}{12.211112in}}%
\pgfpathlineto{\pgfqpoint{7.371248in}{12.236900in}}%
\pgfpathlineto{\pgfqpoint{7.550979in}{12.269427in}}%
\pgfpathlineto{\pgfqpoint{7.730710in}{12.262391in}}%
\pgfpathlineto{\pgfqpoint{7.910441in}{12.257274in}}%
\pgfpathlineto{\pgfqpoint{8.090172in}{12.295333in}}%
\pgfpathlineto{\pgfqpoint{8.269903in}{12.305741in}}%
\pgfpathlineto{\pgfqpoint{8.449634in}{12.347991in}}%
\pgfpathlineto{\pgfqpoint{8.629365in}{12.322861in}}%
\pgfpathlineto{\pgfqpoint{8.809096in}{12.368129in}}%
\pgfpathlineto{\pgfqpoint{8.988827in}{12.380766in}}%
\pgfpathlineto{\pgfqpoint{9.168558in}{12.397284in}}%
\pgfpathlineto{\pgfqpoint{9.348289in}{12.388913in}}%
\pgfpathlineto{\pgfqpoint{9.528020in}{12.391968in}}%
\pgfpathlineto{\pgfqpoint{9.707751in}{12.422924in}}%
\pgfpathlineto{\pgfqpoint{9.887482in}{12.423259in}}%
\pgfusepath{stroke}%
\end{pgfscope}%
\begin{pgfscope}%
\pgfpathrectangle{\pgfqpoint{0.640323in}{9.767436in}}{\pgfqpoint{9.687500in}{3.850000in}}%
\pgfusepath{clip}%
\pgfsetbuttcap%
\pgfsetroundjoin%
\pgfsetlinewidth{1.003750pt}%
\definecolor{currentstroke}{rgb}{0.980392,0.164706,0.333333}%
\pgfsetstrokecolor{currentstroke}%
\pgfsetstrokeopacity{0.500000}%
\pgfsetdash{{3.700000pt}{1.600000pt}}{0.000000pt}%
\pgfpathmoveto{\pgfqpoint{1.080663in}{9.899119in}}%
\pgfpathlineto{\pgfqpoint{1.260394in}{9.901562in}}%
\pgfpathlineto{\pgfqpoint{1.440125in}{9.903036in}}%
\pgfpathlineto{\pgfqpoint{1.619856in}{9.903576in}}%
\pgfpathlineto{\pgfqpoint{1.799587in}{9.905353in}}%
\pgfpathlineto{\pgfqpoint{1.979318in}{9.910307in}}%
\pgfpathlineto{\pgfqpoint{2.159049in}{9.910531in}}%
\pgfpathlineto{\pgfqpoint{2.338780in}{9.931271in}}%
\pgfpathlineto{\pgfqpoint{2.518511in}{9.930488in}}%
\pgfpathlineto{\pgfqpoint{2.698242in}{9.961937in}}%
\pgfpathlineto{\pgfqpoint{2.877973in}{10.346506in}}%
\pgfpathlineto{\pgfqpoint{3.057704in}{10.660375in}}%
\pgfpathlineto{\pgfqpoint{3.237435in}{10.928155in}}%
\pgfpathlineto{\pgfqpoint{3.417166in}{11.155131in}}%
\pgfpathlineto{\pgfqpoint{3.596897in}{11.258044in}}%
\pgfpathlineto{\pgfqpoint{3.776628in}{11.367067in}}%
\pgfpathlineto{\pgfqpoint{3.956359in}{11.492161in}}%
\pgfpathlineto{\pgfqpoint{4.136090in}{11.565516in}}%
\pgfpathlineto{\pgfqpoint{4.315821in}{11.655749in}}%
\pgfpathlineto{\pgfqpoint{4.495552in}{11.729986in}}%
\pgfpathlineto{\pgfqpoint{4.675283in}{11.783159in}}%
\pgfpathlineto{\pgfqpoint{4.855014in}{11.813022in}}%
\pgfpathlineto{\pgfqpoint{5.034745in}{11.889425in}}%
\pgfpathlineto{\pgfqpoint{5.214476in}{11.921523in}}%
\pgfpathlineto{\pgfqpoint{5.394207in}{11.974063in}}%
\pgfpathlineto{\pgfqpoint{5.573938in}{12.006136in}}%
\pgfpathlineto{\pgfqpoint{5.753669in}{12.026697in}}%
\pgfpathlineto{\pgfqpoint{5.933400in}{12.072760in}}%
\pgfpathlineto{\pgfqpoint{6.113131in}{12.105355in}}%
\pgfpathlineto{\pgfqpoint{6.292862in}{12.138036in}}%
\pgfpathlineto{\pgfqpoint{6.472593in}{12.162440in}}%
\pgfpathlineto{\pgfqpoint{6.652324in}{12.186956in}}%
\pgfpathlineto{\pgfqpoint{6.832055in}{12.201580in}}%
\pgfpathlineto{\pgfqpoint{7.011786in}{12.224034in}}%
\pgfpathlineto{\pgfqpoint{7.191517in}{12.242893in}}%
\pgfpathlineto{\pgfqpoint{7.371248in}{12.274550in}}%
\pgfpathlineto{\pgfqpoint{7.550979in}{12.296240in}}%
\pgfpathlineto{\pgfqpoint{7.730710in}{12.299488in}}%
\pgfpathlineto{\pgfqpoint{7.910441in}{12.302909in}}%
\pgfpathlineto{\pgfqpoint{8.090172in}{12.331325in}}%
\pgfpathlineto{\pgfqpoint{8.269903in}{12.356039in}}%
\pgfpathlineto{\pgfqpoint{8.449634in}{12.350047in}}%
\pgfpathlineto{\pgfqpoint{8.629365in}{12.374147in}}%
\pgfpathlineto{\pgfqpoint{8.809096in}{12.396669in}}%
\pgfpathlineto{\pgfqpoint{8.988827in}{12.390826in}}%
\pgfpathlineto{\pgfqpoint{9.168558in}{12.424544in}}%
\pgfpathlineto{\pgfqpoint{9.348289in}{12.424190in}}%
\pgfpathlineto{\pgfqpoint{9.528020in}{12.436231in}}%
\pgfpathlineto{\pgfqpoint{9.707751in}{12.444862in}}%
\pgfpathlineto{\pgfqpoint{9.887482in}{12.469304in}}%
\pgfusepath{stroke}%
\end{pgfscope}%
\begin{pgfscope}%
\pgfpathrectangle{\pgfqpoint{0.640323in}{9.767436in}}{\pgfqpoint{9.687500in}{3.850000in}}%
\pgfusepath{clip}%
\pgfsetbuttcap%
\pgfsetroundjoin%
\pgfsetlinewidth{1.003750pt}%
\definecolor{currentstroke}{rgb}{0.239216,0.478431,0.992157}%
\pgfsetstrokecolor{currentstroke}%
\pgfsetstrokeopacity{0.500000}%
\pgfsetdash{{3.700000pt}{1.600000pt}}{0.000000pt}%
\pgfpathmoveto{\pgfqpoint{1.080663in}{9.899819in}}%
\pgfpathlineto{\pgfqpoint{1.260394in}{9.900851in}}%
\pgfpathlineto{\pgfqpoint{1.440125in}{9.902135in}}%
\pgfpathlineto{\pgfqpoint{1.619856in}{9.903297in}}%
\pgfpathlineto{\pgfqpoint{1.799587in}{9.905530in}}%
\pgfpathlineto{\pgfqpoint{1.979318in}{9.906911in}}%
\pgfpathlineto{\pgfqpoint{2.159049in}{9.914432in}}%
\pgfpathlineto{\pgfqpoint{2.338780in}{9.921424in}}%
\pgfpathlineto{\pgfqpoint{2.518511in}{9.928775in}}%
\pgfpathlineto{\pgfqpoint{2.698242in}{10.018474in}}%
\pgfpathlineto{\pgfqpoint{2.877973in}{10.446193in}}%
\pgfpathlineto{\pgfqpoint{3.057704in}{10.781520in}}%
\pgfpathlineto{\pgfqpoint{3.237435in}{11.000641in}}%
\pgfpathlineto{\pgfqpoint{3.417166in}{11.181789in}}%
\pgfpathlineto{\pgfqpoint{3.596897in}{11.302996in}}%
\pgfpathlineto{\pgfqpoint{3.776628in}{11.438634in}}%
\pgfpathlineto{\pgfqpoint{3.956359in}{11.531605in}}%
\pgfpathlineto{\pgfqpoint{4.136090in}{11.611542in}}%
\pgfpathlineto{\pgfqpoint{4.315821in}{11.691697in}}%
\pgfpathlineto{\pgfqpoint{4.495552in}{11.753551in}}%
\pgfpathlineto{\pgfqpoint{4.675283in}{11.807035in}}%
\pgfpathlineto{\pgfqpoint{4.855014in}{11.859116in}}%
\pgfpathlineto{\pgfqpoint{5.034745in}{11.908011in}}%
\pgfpathlineto{\pgfqpoint{5.214476in}{11.956993in}}%
\pgfpathlineto{\pgfqpoint{5.394207in}{11.990519in}}%
\pgfpathlineto{\pgfqpoint{5.573938in}{12.034204in}}%
\pgfpathlineto{\pgfqpoint{5.753669in}{12.060341in}}%
\pgfpathlineto{\pgfqpoint{5.933400in}{12.094457in}}%
\pgfpathlineto{\pgfqpoint{6.113131in}{12.126294in}}%
\pgfpathlineto{\pgfqpoint{6.292862in}{12.151679in}}%
\pgfpathlineto{\pgfqpoint{6.472593in}{12.183336in}}%
\pgfpathlineto{\pgfqpoint{6.652324in}{12.204020in}}%
\pgfpathlineto{\pgfqpoint{6.832055in}{12.222357in}}%
\pgfpathlineto{\pgfqpoint{7.011786in}{12.241806in}}%
\pgfpathlineto{\pgfqpoint{7.191517in}{12.267825in}}%
\pgfpathlineto{\pgfqpoint{7.371248in}{12.285156in}}%
\pgfpathlineto{\pgfqpoint{7.550979in}{12.303611in}}%
\pgfpathlineto{\pgfqpoint{7.730710in}{12.314826in}}%
\pgfpathlineto{\pgfqpoint{7.910441in}{12.333802in}}%
\pgfpathlineto{\pgfqpoint{8.090172in}{12.342788in}}%
\pgfpathlineto{\pgfqpoint{8.269903in}{12.363267in}}%
\pgfpathlineto{\pgfqpoint{8.449634in}{12.373190in}}%
\pgfpathlineto{\pgfqpoint{8.629365in}{12.393806in}}%
\pgfpathlineto{\pgfqpoint{8.809096in}{12.406996in}}%
\pgfpathlineto{\pgfqpoint{8.988827in}{12.417900in}}%
\pgfpathlineto{\pgfqpoint{9.168558in}{12.433250in}}%
\pgfpathlineto{\pgfqpoint{9.348289in}{12.442900in}}%
\pgfpathlineto{\pgfqpoint{9.528020in}{12.449694in}}%
\pgfpathlineto{\pgfqpoint{9.707751in}{12.460871in}}%
\pgfpathlineto{\pgfqpoint{9.887482in}{12.477637in}}%
\pgfusepath{stroke}%
\end{pgfscope}%
\begin{pgfscope}%
\pgfpathrectangle{\pgfqpoint{0.640323in}{9.767436in}}{\pgfqpoint{9.687500in}{3.850000in}}%
\pgfusepath{clip}%
\pgfsetbuttcap%
\pgfsetroundjoin%
\pgfsetlinewidth{1.003750pt}%
\definecolor{currentstroke}{rgb}{0.000000,0.000000,0.000000}%
\pgfsetstrokecolor{currentstroke}%
\pgfsetstrokeopacity{0.500000}%
\pgfsetdash{{3.700000pt}{1.600000pt}}{0.000000pt}%
\pgfpathmoveto{\pgfqpoint{1.080663in}{9.899636in}}%
\pgfpathlineto{\pgfqpoint{1.260394in}{9.900663in}}%
\pgfpathlineto{\pgfqpoint{1.440125in}{9.901686in}}%
\pgfpathlineto{\pgfqpoint{1.619856in}{9.903556in}}%
\pgfpathlineto{\pgfqpoint{1.799587in}{9.904833in}}%
\pgfpathlineto{\pgfqpoint{1.979318in}{9.908198in}}%
\pgfpathlineto{\pgfqpoint{2.159049in}{9.913532in}}%
\pgfpathlineto{\pgfqpoint{2.338780in}{9.920951in}}%
\pgfpathlineto{\pgfqpoint{2.518511in}{9.939267in}}%
\pgfpathlineto{\pgfqpoint{2.698242in}{10.084847in}}%
\pgfpathlineto{\pgfqpoint{2.877973in}{10.523607in}}%
\pgfpathlineto{\pgfqpoint{3.057704in}{10.826707in}}%
\pgfpathlineto{\pgfqpoint{3.237435in}{11.037321in}}%
\pgfpathlineto{\pgfqpoint{3.417166in}{11.207379in}}%
\pgfpathlineto{\pgfqpoint{3.596897in}{11.347308in}}%
\pgfpathlineto{\pgfqpoint{3.776628in}{11.451761in}}%
\pgfpathlineto{\pgfqpoint{3.956359in}{11.552339in}}%
\pgfpathlineto{\pgfqpoint{4.136090in}{11.631910in}}%
\pgfpathlineto{\pgfqpoint{4.315821in}{11.706060in}}%
\pgfpathlineto{\pgfqpoint{4.495552in}{11.767666in}}%
\pgfpathlineto{\pgfqpoint{4.675283in}{11.823696in}}%
\pgfpathlineto{\pgfqpoint{4.855014in}{11.877820in}}%
\pgfpathlineto{\pgfqpoint{5.034745in}{11.921436in}}%
\pgfpathlineto{\pgfqpoint{5.214476in}{11.968543in}}%
\pgfpathlineto{\pgfqpoint{5.394207in}{12.006813in}}%
\pgfpathlineto{\pgfqpoint{5.573938in}{12.045121in}}%
\pgfpathlineto{\pgfqpoint{5.753669in}{12.074294in}}%
\pgfpathlineto{\pgfqpoint{5.933400in}{12.102281in}}%
\pgfpathlineto{\pgfqpoint{6.113131in}{12.135099in}}%
\pgfpathlineto{\pgfqpoint{6.292862in}{12.156907in}}%
\pgfpathlineto{\pgfqpoint{6.472593in}{12.187496in}}%
\pgfpathlineto{\pgfqpoint{6.652324in}{12.211484in}}%
\pgfpathlineto{\pgfqpoint{6.832055in}{12.234019in}}%
\pgfpathlineto{\pgfqpoint{7.011786in}{12.253480in}}%
\pgfpathlineto{\pgfqpoint{7.191517in}{12.277207in}}%
\pgfpathlineto{\pgfqpoint{7.371248in}{12.291862in}}%
\pgfpathlineto{\pgfqpoint{7.550979in}{12.306946in}}%
\pgfpathlineto{\pgfqpoint{7.730710in}{12.324848in}}%
\pgfpathlineto{\pgfqpoint{7.910441in}{12.338280in}}%
\pgfpathlineto{\pgfqpoint{8.090172in}{12.357163in}}%
\pgfpathlineto{\pgfqpoint{8.269903in}{12.370160in}}%
\pgfpathlineto{\pgfqpoint{8.449634in}{12.383151in}}%
\pgfpathlineto{\pgfqpoint{8.629365in}{12.397725in}}%
\pgfpathlineto{\pgfqpoint{8.809096in}{12.408387in}}%
\pgfpathlineto{\pgfqpoint{8.988827in}{12.418701in}}%
\pgfpathlineto{\pgfqpoint{9.168558in}{12.434908in}}%
\pgfpathlineto{\pgfqpoint{9.348289in}{12.445719in}}%
\pgfpathlineto{\pgfqpoint{9.528020in}{12.454338in}}%
\pgfpathlineto{\pgfqpoint{9.707751in}{12.465168in}}%
\pgfpathlineto{\pgfqpoint{9.887482in}{12.471502in}}%
\pgfusepath{stroke}%
\end{pgfscope}%
\begin{pgfscope}%
\pgfsetrectcap%
\pgfsetmiterjoin%
\pgfsetlinewidth{0.803000pt}%
\definecolor{currentstroke}{rgb}{0.000000,0.000000,0.000000}%
\pgfsetstrokecolor{currentstroke}%
\pgfsetdash{}{0pt}%
\pgfpathmoveto{\pgfqpoint{0.640323in}{9.767436in}}%
\pgfpathlineto{\pgfqpoint{0.640323in}{13.617436in}}%
\pgfusepath{stroke}%
\end{pgfscope}%
\begin{pgfscope}%
\pgfsetrectcap%
\pgfsetmiterjoin%
\pgfsetlinewidth{0.803000pt}%
\definecolor{currentstroke}{rgb}{0.000000,0.000000,0.000000}%
\pgfsetstrokecolor{currentstroke}%
\pgfsetdash{}{0pt}%
\pgfpathmoveto{\pgfqpoint{10.327822in}{9.767436in}}%
\pgfpathlineto{\pgfqpoint{10.327822in}{13.617436in}}%
\pgfusepath{stroke}%
\end{pgfscope}%
\begin{pgfscope}%
\pgfsetrectcap%
\pgfsetmiterjoin%
\pgfsetlinewidth{0.803000pt}%
\definecolor{currentstroke}{rgb}{0.000000,0.000000,0.000000}%
\pgfsetstrokecolor{currentstroke}%
\pgfsetdash{}{0pt}%
\pgfpathmoveto{\pgfqpoint{0.640322in}{9.767436in}}%
\pgfpathlineto{\pgfqpoint{10.327823in}{9.767436in}}%
\pgfusepath{stroke}%
\end{pgfscope}%
\begin{pgfscope}%
\pgfsetrectcap%
\pgfsetmiterjoin%
\pgfsetlinewidth{0.803000pt}%
\definecolor{currentstroke}{rgb}{0.000000,0.000000,0.000000}%
\pgfsetstrokecolor{currentstroke}%
\pgfsetdash{}{0pt}%
\pgfpathmoveto{\pgfqpoint{0.640322in}{13.617436in}}%
\pgfpathlineto{\pgfqpoint{10.327823in}{13.617436in}}%
\pgfusepath{stroke}%
\end{pgfscope}%
\begin{pgfscope}%
\definecolor{textcolor}{rgb}{0.000000,0.000000,0.000000}%
\pgfsetstrokecolor{textcolor}%
\pgfsetfillcolor{textcolor}%
\pgftext[x=5.484072in,y=13.700769in,,base]{\color{textcolor}\sffamily\fontsize{12.000000}{14.400000}\selectfont \(\displaystyle \overline{\langle I\rangle}\) over \(\displaystyle p_1\) for \(\displaystyle T=1000\) with \(\displaystyle p_2=0.3\), \(\displaystyle p_3=0.3\)}%
\end{pgfscope}%
\begin{pgfscope}%
\pgfsetbuttcap%
\pgfsetmiterjoin%
\definecolor{currentfill}{rgb}{1.000000,1.000000,1.000000}%
\pgfsetfillcolor{currentfill}%
\pgfsetfillopacity{0.800000}%
\pgfsetlinewidth{1.003750pt}%
\definecolor{currentstroke}{rgb}{0.800000,0.800000,0.800000}%
\pgfsetstrokecolor{currentstroke}%
\pgfsetstrokeopacity{0.800000}%
\pgfsetdash{}{0pt}%
\pgfpathmoveto{\pgfqpoint{0.737545in}{12.690896in}}%
\pgfpathlineto{\pgfqpoint{1.670029in}{12.690896in}}%
\pgfpathquadraticcurveto{\pgfqpoint{1.697806in}{12.690896in}}{\pgfqpoint{1.697806in}{12.718674in}}%
\pgfpathlineto{\pgfqpoint{1.697806in}{13.520214in}}%
\pgfpathquadraticcurveto{\pgfqpoint{1.697806in}{13.547991in}}{\pgfqpoint{1.670029in}{13.547991in}}%
\pgfpathlineto{\pgfqpoint{0.737545in}{13.547991in}}%
\pgfpathquadraticcurveto{\pgfqpoint{0.709767in}{13.547991in}}{\pgfqpoint{0.709767in}{13.520214in}}%
\pgfpathlineto{\pgfqpoint{0.709767in}{12.718674in}}%
\pgfpathquadraticcurveto{\pgfqpoint{0.709767in}{12.690896in}}{\pgfqpoint{0.737545in}{12.690896in}}%
\pgfpathlineto{\pgfqpoint{0.737545in}{12.690896in}}%
\pgfpathclose%
\pgfusepath{stroke,fill}%
\end{pgfscope}%
\begin{pgfscope}%
\pgfsetbuttcap%
\pgfsetroundjoin%
\definecolor{currentfill}{rgb}{0.000000,0.000000,1.000000}%
\pgfsetfillcolor{currentfill}%
\pgfsetfillopacity{0.500000}%
\pgfsetlinewidth{1.003750pt}%
\definecolor{currentstroke}{rgb}{0.000000,0.000000,1.000000}%
\pgfsetstrokecolor{currentstroke}%
\pgfsetstrokeopacity{0.500000}%
\pgfsetdash{}{0pt}%
\pgfsys@defobject{currentmarker}{\pgfqpoint{-0.021960in}{-0.021960in}}{\pgfqpoint{0.021960in}{0.021960in}}{%
\pgfpathmoveto{\pgfqpoint{0.000000in}{-0.021960in}}%
\pgfpathcurveto{\pgfqpoint{0.005824in}{-0.021960in}}{\pgfqpoint{0.011410in}{-0.019646in}}{\pgfqpoint{0.015528in}{-0.015528in}}%
\pgfpathcurveto{\pgfqpoint{0.019646in}{-0.011410in}}{\pgfqpoint{0.021960in}{-0.005824in}}{\pgfqpoint{0.021960in}{0.000000in}}%
\pgfpathcurveto{\pgfqpoint{0.021960in}{0.005824in}}{\pgfqpoint{0.019646in}{0.011410in}}{\pgfqpoint{0.015528in}{0.015528in}}%
\pgfpathcurveto{\pgfqpoint{0.011410in}{0.019646in}}{\pgfqpoint{0.005824in}{0.021960in}}{\pgfqpoint{0.000000in}{0.021960in}}%
\pgfpathcurveto{\pgfqpoint{-0.005824in}{0.021960in}}{\pgfqpoint{-0.011410in}{0.019646in}}{\pgfqpoint{-0.015528in}{0.015528in}}%
\pgfpathcurveto{\pgfqpoint{-0.019646in}{0.011410in}}{\pgfqpoint{-0.021960in}{0.005824in}}{\pgfqpoint{-0.021960in}{0.000000in}}%
\pgfpathcurveto{\pgfqpoint{-0.021960in}{-0.005824in}}{\pgfqpoint{-0.019646in}{-0.011410in}}{\pgfqpoint{-0.015528in}{-0.015528in}}%
\pgfpathcurveto{\pgfqpoint{-0.011410in}{-0.019646in}}{\pgfqpoint{-0.005824in}{-0.021960in}}{\pgfqpoint{0.000000in}{-0.021960in}}%
\pgfpathlineto{\pgfqpoint{0.000000in}{-0.021960in}}%
\pgfpathclose%
\pgfusepath{stroke,fill}%
}%
\begin{pgfscope}%
\pgfsys@transformshift{0.904211in}{13.423371in}%
\pgfsys@useobject{currentmarker}{}%
\end{pgfscope}%
\end{pgfscope}%
\begin{pgfscope}%
\definecolor{textcolor}{rgb}{0.000000,0.000000,0.000000}%
\pgfsetstrokecolor{textcolor}%
\pgfsetfillcolor{textcolor}%
\pgftext[x=1.154211in,y=13.386913in,left,base]{\color{textcolor}\sffamily\fontsize{10.000000}{12.000000}\selectfont \(\displaystyle L=16\)}%
\end{pgfscope}%
\begin{pgfscope}%
\pgfsetbuttcap%
\pgfsetroundjoin%
\definecolor{currentfill}{rgb}{0.980392,0.164706,0.333333}%
\pgfsetfillcolor{currentfill}%
\pgfsetfillopacity{0.500000}%
\pgfsetlinewidth{1.003750pt}%
\definecolor{currentstroke}{rgb}{0.980392,0.164706,0.333333}%
\pgfsetstrokecolor{currentstroke}%
\pgfsetstrokeopacity{0.500000}%
\pgfsetdash{}{0pt}%
\pgfsys@defobject{currentmarker}{\pgfqpoint{-0.021960in}{-0.021960in}}{\pgfqpoint{0.021960in}{0.021960in}}{%
\pgfpathmoveto{\pgfqpoint{0.000000in}{-0.021960in}}%
\pgfpathcurveto{\pgfqpoint{0.005824in}{-0.021960in}}{\pgfqpoint{0.011410in}{-0.019646in}}{\pgfqpoint{0.015528in}{-0.015528in}}%
\pgfpathcurveto{\pgfqpoint{0.019646in}{-0.011410in}}{\pgfqpoint{0.021960in}{-0.005824in}}{\pgfqpoint{0.021960in}{0.000000in}}%
\pgfpathcurveto{\pgfqpoint{0.021960in}{0.005824in}}{\pgfqpoint{0.019646in}{0.011410in}}{\pgfqpoint{0.015528in}{0.015528in}}%
\pgfpathcurveto{\pgfqpoint{0.011410in}{0.019646in}}{\pgfqpoint{0.005824in}{0.021960in}}{\pgfqpoint{0.000000in}{0.021960in}}%
\pgfpathcurveto{\pgfqpoint{-0.005824in}{0.021960in}}{\pgfqpoint{-0.011410in}{0.019646in}}{\pgfqpoint{-0.015528in}{0.015528in}}%
\pgfpathcurveto{\pgfqpoint{-0.019646in}{0.011410in}}{\pgfqpoint{-0.021960in}{0.005824in}}{\pgfqpoint{-0.021960in}{0.000000in}}%
\pgfpathcurveto{\pgfqpoint{-0.021960in}{-0.005824in}}{\pgfqpoint{-0.019646in}{-0.011410in}}{\pgfqpoint{-0.015528in}{-0.015528in}}%
\pgfpathcurveto{\pgfqpoint{-0.011410in}{-0.019646in}}{\pgfqpoint{-0.005824in}{-0.021960in}}{\pgfqpoint{0.000000in}{-0.021960in}}%
\pgfpathlineto{\pgfqpoint{0.000000in}{-0.021960in}}%
\pgfpathclose%
\pgfusepath{stroke,fill}%
}%
\begin{pgfscope}%
\pgfsys@transformshift{0.904211in}{13.219514in}%
\pgfsys@useobject{currentmarker}{}%
\end{pgfscope}%
\end{pgfscope}%
\begin{pgfscope}%
\definecolor{textcolor}{rgb}{0.000000,0.000000,0.000000}%
\pgfsetstrokecolor{textcolor}%
\pgfsetfillcolor{textcolor}%
\pgftext[x=1.154211in,y=13.183056in,left,base]{\color{textcolor}\sffamily\fontsize{10.000000}{12.000000}\selectfont \(\displaystyle L=32\)}%
\end{pgfscope}%
\begin{pgfscope}%
\pgfsetbuttcap%
\pgfsetroundjoin%
\definecolor{currentfill}{rgb}{0.239216,0.478431,0.992157}%
\pgfsetfillcolor{currentfill}%
\pgfsetfillopacity{0.500000}%
\pgfsetlinewidth{1.003750pt}%
\definecolor{currentstroke}{rgb}{0.239216,0.478431,0.992157}%
\pgfsetstrokecolor{currentstroke}%
\pgfsetstrokeopacity{0.500000}%
\pgfsetdash{}{0pt}%
\pgfsys@defobject{currentmarker}{\pgfqpoint{-0.021960in}{-0.021960in}}{\pgfqpoint{0.021960in}{0.021960in}}{%
\pgfpathmoveto{\pgfqpoint{0.000000in}{-0.021960in}}%
\pgfpathcurveto{\pgfqpoint{0.005824in}{-0.021960in}}{\pgfqpoint{0.011410in}{-0.019646in}}{\pgfqpoint{0.015528in}{-0.015528in}}%
\pgfpathcurveto{\pgfqpoint{0.019646in}{-0.011410in}}{\pgfqpoint{0.021960in}{-0.005824in}}{\pgfqpoint{0.021960in}{0.000000in}}%
\pgfpathcurveto{\pgfqpoint{0.021960in}{0.005824in}}{\pgfqpoint{0.019646in}{0.011410in}}{\pgfqpoint{0.015528in}{0.015528in}}%
\pgfpathcurveto{\pgfqpoint{0.011410in}{0.019646in}}{\pgfqpoint{0.005824in}{0.021960in}}{\pgfqpoint{0.000000in}{0.021960in}}%
\pgfpathcurveto{\pgfqpoint{-0.005824in}{0.021960in}}{\pgfqpoint{-0.011410in}{0.019646in}}{\pgfqpoint{-0.015528in}{0.015528in}}%
\pgfpathcurveto{\pgfqpoint{-0.019646in}{0.011410in}}{\pgfqpoint{-0.021960in}{0.005824in}}{\pgfqpoint{-0.021960in}{0.000000in}}%
\pgfpathcurveto{\pgfqpoint{-0.021960in}{-0.005824in}}{\pgfqpoint{-0.019646in}{-0.011410in}}{\pgfqpoint{-0.015528in}{-0.015528in}}%
\pgfpathcurveto{\pgfqpoint{-0.011410in}{-0.019646in}}{\pgfqpoint{-0.005824in}{-0.021960in}}{\pgfqpoint{0.000000in}{-0.021960in}}%
\pgfpathlineto{\pgfqpoint{0.000000in}{-0.021960in}}%
\pgfpathclose%
\pgfusepath{stroke,fill}%
}%
\begin{pgfscope}%
\pgfsys@transformshift{0.904211in}{13.015657in}%
\pgfsys@useobject{currentmarker}{}%
\end{pgfscope}%
\end{pgfscope}%
\begin{pgfscope}%
\definecolor{textcolor}{rgb}{0.000000,0.000000,0.000000}%
\pgfsetstrokecolor{textcolor}%
\pgfsetfillcolor{textcolor}%
\pgftext[x=1.154211in,y=12.979198in,left,base]{\color{textcolor}\sffamily\fontsize{10.000000}{12.000000}\selectfont \(\displaystyle L=64\)}%
\end{pgfscope}%
\begin{pgfscope}%
\pgfsetbuttcap%
\pgfsetroundjoin%
\definecolor{currentfill}{rgb}{0.000000,0.000000,0.000000}%
\pgfsetfillcolor{currentfill}%
\pgfsetfillopacity{0.500000}%
\pgfsetlinewidth{1.003750pt}%
\definecolor{currentstroke}{rgb}{0.000000,0.000000,0.000000}%
\pgfsetstrokecolor{currentstroke}%
\pgfsetstrokeopacity{0.500000}%
\pgfsetdash{}{0pt}%
\pgfsys@defobject{currentmarker}{\pgfqpoint{-0.021960in}{-0.021960in}}{\pgfqpoint{0.021960in}{0.021960in}}{%
\pgfpathmoveto{\pgfqpoint{0.000000in}{-0.021960in}}%
\pgfpathcurveto{\pgfqpoint{0.005824in}{-0.021960in}}{\pgfqpoint{0.011410in}{-0.019646in}}{\pgfqpoint{0.015528in}{-0.015528in}}%
\pgfpathcurveto{\pgfqpoint{0.019646in}{-0.011410in}}{\pgfqpoint{0.021960in}{-0.005824in}}{\pgfqpoint{0.021960in}{0.000000in}}%
\pgfpathcurveto{\pgfqpoint{0.021960in}{0.005824in}}{\pgfqpoint{0.019646in}{0.011410in}}{\pgfqpoint{0.015528in}{0.015528in}}%
\pgfpathcurveto{\pgfqpoint{0.011410in}{0.019646in}}{\pgfqpoint{0.005824in}{0.021960in}}{\pgfqpoint{0.000000in}{0.021960in}}%
\pgfpathcurveto{\pgfqpoint{-0.005824in}{0.021960in}}{\pgfqpoint{-0.011410in}{0.019646in}}{\pgfqpoint{-0.015528in}{0.015528in}}%
\pgfpathcurveto{\pgfqpoint{-0.019646in}{0.011410in}}{\pgfqpoint{-0.021960in}{0.005824in}}{\pgfqpoint{-0.021960in}{0.000000in}}%
\pgfpathcurveto{\pgfqpoint{-0.021960in}{-0.005824in}}{\pgfqpoint{-0.019646in}{-0.011410in}}{\pgfqpoint{-0.015528in}{-0.015528in}}%
\pgfpathcurveto{\pgfqpoint{-0.011410in}{-0.019646in}}{\pgfqpoint{-0.005824in}{-0.021960in}}{\pgfqpoint{0.000000in}{-0.021960in}}%
\pgfpathlineto{\pgfqpoint{0.000000in}{-0.021960in}}%
\pgfpathclose%
\pgfusepath{stroke,fill}%
}%
\begin{pgfscope}%
\pgfsys@transformshift{0.904211in}{12.811799in}%
\pgfsys@useobject{currentmarker}{}%
\end{pgfscope}%
\end{pgfscope}%
\begin{pgfscope}%
\definecolor{textcolor}{rgb}{0.000000,0.000000,0.000000}%
\pgfsetstrokecolor{textcolor}%
\pgfsetfillcolor{textcolor}%
\pgftext[x=1.154211in,y=12.775341in,left,base]{\color{textcolor}\sffamily\fontsize{10.000000}{12.000000}\selectfont \(\displaystyle L=128\)}%
\end{pgfscope}%
\begin{pgfscope}%
\pgfsetbuttcap%
\pgfsetmiterjoin%
\definecolor{currentfill}{rgb}{1.000000,1.000000,1.000000}%
\pgfsetfillcolor{currentfill}%
\pgfsetlinewidth{0.000000pt}%
\definecolor{currentstroke}{rgb}{0.000000,0.000000,0.000000}%
\pgfsetstrokecolor{currentstroke}%
\pgfsetstrokeopacity{0.000000}%
\pgfsetdash{}{0pt}%
\pgfpathmoveto{\pgfqpoint{0.640323in}{5.147436in}}%
\pgfpathlineto{\pgfqpoint{10.327822in}{5.147436in}}%
\pgfpathlineto{\pgfqpoint{10.327822in}{8.997436in}}%
\pgfpathlineto{\pgfqpoint{0.640323in}{8.997436in}}%
\pgfpathlineto{\pgfqpoint{0.640323in}{5.147436in}}%
\pgfpathclose%
\pgfusepath{fill}%
\end{pgfscope}%
\begin{pgfscope}%
\pgfpathrectangle{\pgfqpoint{0.640323in}{5.147436in}}{\pgfqpoint{9.687500in}{3.850000in}}%
\pgfusepath{clip}%
\pgfsetbuttcap%
\pgfsetroundjoin%
\definecolor{currentfill}{rgb}{0.000000,0.000000,1.000000}%
\pgfsetfillcolor{currentfill}%
\pgfsetfillopacity{0.500000}%
\pgfsetlinewidth{1.003750pt}%
\definecolor{currentstroke}{rgb}{0.000000,0.000000,1.000000}%
\pgfsetstrokecolor{currentstroke}%
\pgfsetstrokeopacity{0.500000}%
\pgfsetdash{}{0pt}%
\pgfsys@defobject{currentmarker}{\pgfqpoint{-0.021960in}{-0.021960in}}{\pgfqpoint{0.021960in}{0.021960in}}{%
\pgfpathmoveto{\pgfqpoint{0.000000in}{-0.021960in}}%
\pgfpathcurveto{\pgfqpoint{0.005824in}{-0.021960in}}{\pgfqpoint{0.011410in}{-0.019646in}}{\pgfqpoint{0.015528in}{-0.015528in}}%
\pgfpathcurveto{\pgfqpoint{0.019646in}{-0.011410in}}{\pgfqpoint{0.021960in}{-0.005824in}}{\pgfqpoint{0.021960in}{0.000000in}}%
\pgfpathcurveto{\pgfqpoint{0.021960in}{0.005824in}}{\pgfqpoint{0.019646in}{0.011410in}}{\pgfqpoint{0.015528in}{0.015528in}}%
\pgfpathcurveto{\pgfqpoint{0.011410in}{0.019646in}}{\pgfqpoint{0.005824in}{0.021960in}}{\pgfqpoint{0.000000in}{0.021960in}}%
\pgfpathcurveto{\pgfqpoint{-0.005824in}{0.021960in}}{\pgfqpoint{-0.011410in}{0.019646in}}{\pgfqpoint{-0.015528in}{0.015528in}}%
\pgfpathcurveto{\pgfqpoint{-0.019646in}{0.011410in}}{\pgfqpoint{-0.021960in}{0.005824in}}{\pgfqpoint{-0.021960in}{0.000000in}}%
\pgfpathcurveto{\pgfqpoint{-0.021960in}{-0.005824in}}{\pgfqpoint{-0.019646in}{-0.011410in}}{\pgfqpoint{-0.015528in}{-0.015528in}}%
\pgfpathcurveto{\pgfqpoint{-0.011410in}{-0.019646in}}{\pgfqpoint{-0.005824in}{-0.021960in}}{\pgfqpoint{0.000000in}{-0.021960in}}%
\pgfpathlineto{\pgfqpoint{0.000000in}{-0.021960in}}%
\pgfpathclose%
\pgfusepath{stroke,fill}%
}%
\begin{pgfscope}%
\pgfsys@transformshift{1.080663in}{5.275389in}%
\pgfsys@useobject{currentmarker}{}%
\end{pgfscope}%
\begin{pgfscope}%
\pgfsys@transformshift{1.260394in}{5.275607in}%
\pgfsys@useobject{currentmarker}{}%
\end{pgfscope}%
\begin{pgfscope}%
\pgfsys@transformshift{1.440125in}{5.276505in}%
\pgfsys@useobject{currentmarker}{}%
\end{pgfscope}%
\begin{pgfscope}%
\pgfsys@transformshift{1.619856in}{5.276626in}%
\pgfsys@useobject{currentmarker}{}%
\end{pgfscope}%
\begin{pgfscope}%
\pgfsys@transformshift{1.799587in}{5.277742in}%
\pgfsys@useobject{currentmarker}{}%
\end{pgfscope}%
\begin{pgfscope}%
\pgfsys@transformshift{1.979318in}{5.277378in}%
\pgfsys@useobject{currentmarker}{}%
\end{pgfscope}%
\begin{pgfscope}%
\pgfsys@transformshift{2.159049in}{5.279489in}%
\pgfsys@useobject{currentmarker}{}%
\end{pgfscope}%
\begin{pgfscope}%
\pgfsys@transformshift{2.338780in}{5.278809in}%
\pgfsys@useobject{currentmarker}{}%
\end{pgfscope}%
\begin{pgfscope}%
\pgfsys@transformshift{2.518511in}{5.276917in}%
\pgfsys@useobject{currentmarker}{}%
\end{pgfscope}%
\begin{pgfscope}%
\pgfsys@transformshift{2.698242in}{5.278057in}%
\pgfsys@useobject{currentmarker}{}%
\end{pgfscope}%
\begin{pgfscope}%
\pgfsys@transformshift{2.877973in}{5.277475in}%
\pgfsys@useobject{currentmarker}{}%
\end{pgfscope}%
\begin{pgfscope}%
\pgfsys@transformshift{3.057704in}{5.281162in}%
\pgfsys@useobject{currentmarker}{}%
\end{pgfscope}%
\begin{pgfscope}%
\pgfsys@transformshift{3.237435in}{5.278518in}%
\pgfsys@useobject{currentmarker}{}%
\end{pgfscope}%
\begin{pgfscope}%
\pgfsys@transformshift{3.417166in}{5.280968in}%
\pgfsys@useobject{currentmarker}{}%
\end{pgfscope}%
\begin{pgfscope}%
\pgfsys@transformshift{3.596897in}{5.278785in}%
\pgfsys@useobject{currentmarker}{}%
\end{pgfscope}%
\begin{pgfscope}%
\pgfsys@transformshift{3.776628in}{5.283733in}%
\pgfsys@useobject{currentmarker}{}%
\end{pgfscope}%
\begin{pgfscope}%
\pgfsys@transformshift{3.956359in}{5.283345in}%
\pgfsys@useobject{currentmarker}{}%
\end{pgfscope}%
\begin{pgfscope}%
\pgfsys@transformshift{4.136090in}{5.281235in}%
\pgfsys@useobject{currentmarker}{}%
\end{pgfscope}%
\begin{pgfscope}%
\pgfsys@transformshift{4.315821in}{5.290477in}%
\pgfsys@useobject{currentmarker}{}%
\end{pgfscope}%
\begin{pgfscope}%
\pgfsys@transformshift{4.495552in}{5.279901in}%
\pgfsys@useobject{currentmarker}{}%
\end{pgfscope}%
\begin{pgfscope}%
\pgfsys@transformshift{4.675283in}{5.310513in}%
\pgfsys@useobject{currentmarker}{}%
\end{pgfscope}%
\begin{pgfscope}%
\pgfsys@transformshift{4.855014in}{5.287687in}%
\pgfsys@useobject{currentmarker}{}%
\end{pgfscope}%
\begin{pgfscope}%
\pgfsys@transformshift{5.034745in}{5.279537in}%
\pgfsys@useobject{currentmarker}{}%
\end{pgfscope}%
\begin{pgfscope}%
\pgfsys@transformshift{5.214476in}{5.323684in}%
\pgfsys@useobject{currentmarker}{}%
\end{pgfscope}%
\begin{pgfscope}%
\pgfsys@transformshift{5.394207in}{5.299646in}%
\pgfsys@useobject{currentmarker}{}%
\end{pgfscope}%
\begin{pgfscope}%
\pgfsys@transformshift{5.573938in}{5.309664in}%
\pgfsys@useobject{currentmarker}{}%
\end{pgfscope}%
\begin{pgfscope}%
\pgfsys@transformshift{5.753669in}{5.343817in}%
\pgfsys@useobject{currentmarker}{}%
\end{pgfscope}%
\begin{pgfscope}%
\pgfsys@transformshift{5.933400in}{5.524043in}%
\pgfsys@useobject{currentmarker}{}%
\end{pgfscope}%
\begin{pgfscope}%
\pgfsys@transformshift{6.113131in}{5.310319in}%
\pgfsys@useobject{currentmarker}{}%
\end{pgfscope}%
\begin{pgfscope}%
\pgfsys@transformshift{6.292862in}{5.909891in}%
\pgfsys@useobject{currentmarker}{}%
\end{pgfscope}%
\begin{pgfscope}%
\pgfsys@transformshift{6.472593in}{5.552738in}%
\pgfsys@useobject{currentmarker}{}%
\end{pgfscope}%
\begin{pgfscope}%
\pgfsys@transformshift{6.652324in}{5.463984in}%
\pgfsys@useobject{currentmarker}{}%
\end{pgfscope}%
\begin{pgfscope}%
\pgfsys@transformshift{6.832055in}{5.505948in}%
\pgfsys@useobject{currentmarker}{}%
\end{pgfscope}%
\begin{pgfscope}%
\pgfsys@transformshift{7.011786in}{5.527730in}%
\pgfsys@useobject{currentmarker}{}%
\end{pgfscope}%
\begin{pgfscope}%
\pgfsys@transformshift{7.191517in}{6.071175in}%
\pgfsys@useobject{currentmarker}{}%
\end{pgfscope}%
\begin{pgfscope}%
\pgfsys@transformshift{7.371248in}{5.485451in}%
\pgfsys@useobject{currentmarker}{}%
\end{pgfscope}%
\begin{pgfscope}%
\pgfsys@transformshift{7.550979in}{6.155900in}%
\pgfsys@useobject{currentmarker}{}%
\end{pgfscope}%
\begin{pgfscope}%
\pgfsys@transformshift{7.730710in}{6.217078in}%
\pgfsys@useobject{currentmarker}{}%
\end{pgfscope}%
\begin{pgfscope}%
\pgfsys@transformshift{7.910441in}{6.219506in}%
\pgfsys@useobject{currentmarker}{}%
\end{pgfscope}%
\begin{pgfscope}%
\pgfsys@transformshift{8.090172in}{6.263967in}%
\pgfsys@useobject{currentmarker}{}%
\end{pgfscope}%
\begin{pgfscope}%
\pgfsys@transformshift{8.269903in}{6.319388in}%
\pgfsys@useobject{currentmarker}{}%
\end{pgfscope}%
\begin{pgfscope}%
\pgfsys@transformshift{8.449634in}{6.361304in}%
\pgfsys@useobject{currentmarker}{}%
\end{pgfscope}%
\begin{pgfscope}%
\pgfsys@transformshift{8.629365in}{6.346872in}%
\pgfsys@useobject{currentmarker}{}%
\end{pgfscope}%
\begin{pgfscope}%
\pgfsys@transformshift{8.809096in}{6.368538in}%
\pgfsys@useobject{currentmarker}{}%
\end{pgfscope}%
\begin{pgfscope}%
\pgfsys@transformshift{8.988827in}{6.367929in}%
\pgfsys@useobject{currentmarker}{}%
\end{pgfscope}%
\begin{pgfscope}%
\pgfsys@transformshift{9.168558in}{6.419861in}%
\pgfsys@useobject{currentmarker}{}%
\end{pgfscope}%
\begin{pgfscope}%
\pgfsys@transformshift{9.348289in}{6.462018in}%
\pgfsys@useobject{currentmarker}{}%
\end{pgfscope}%
\begin{pgfscope}%
\pgfsys@transformshift{9.528020in}{6.480647in}%
\pgfsys@useobject{currentmarker}{}%
\end{pgfscope}%
\begin{pgfscope}%
\pgfsys@transformshift{9.707751in}{6.471674in}%
\pgfsys@useobject{currentmarker}{}%
\end{pgfscope}%
\begin{pgfscope}%
\pgfsys@transformshift{9.887482in}{6.488943in}%
\pgfsys@useobject{currentmarker}{}%
\end{pgfscope}%
\end{pgfscope}%
\begin{pgfscope}%
\pgfpathrectangle{\pgfqpoint{0.640323in}{5.147436in}}{\pgfqpoint{9.687500in}{3.850000in}}%
\pgfusepath{clip}%
\pgfsetbuttcap%
\pgfsetroundjoin%
\definecolor{currentfill}{rgb}{0.980392,0.164706,0.333333}%
\pgfsetfillcolor{currentfill}%
\pgfsetfillopacity{0.500000}%
\pgfsetlinewidth{1.003750pt}%
\definecolor{currentstroke}{rgb}{0.980392,0.164706,0.333333}%
\pgfsetstrokecolor{currentstroke}%
\pgfsetstrokeopacity{0.500000}%
\pgfsetdash{}{0pt}%
\pgfsys@defobject{currentmarker}{\pgfqpoint{-0.021960in}{-0.021960in}}{\pgfqpoint{0.021960in}{0.021960in}}{%
\pgfpathmoveto{\pgfqpoint{0.000000in}{-0.021960in}}%
\pgfpathcurveto{\pgfqpoint{0.005824in}{-0.021960in}}{\pgfqpoint{0.011410in}{-0.019646in}}{\pgfqpoint{0.015528in}{-0.015528in}}%
\pgfpathcurveto{\pgfqpoint{0.019646in}{-0.011410in}}{\pgfqpoint{0.021960in}{-0.005824in}}{\pgfqpoint{0.021960in}{0.000000in}}%
\pgfpathcurveto{\pgfqpoint{0.021960in}{0.005824in}}{\pgfqpoint{0.019646in}{0.011410in}}{\pgfqpoint{0.015528in}{0.015528in}}%
\pgfpathcurveto{\pgfqpoint{0.011410in}{0.019646in}}{\pgfqpoint{0.005824in}{0.021960in}}{\pgfqpoint{0.000000in}{0.021960in}}%
\pgfpathcurveto{\pgfqpoint{-0.005824in}{0.021960in}}{\pgfqpoint{-0.011410in}{0.019646in}}{\pgfqpoint{-0.015528in}{0.015528in}}%
\pgfpathcurveto{\pgfqpoint{-0.019646in}{0.011410in}}{\pgfqpoint{-0.021960in}{0.005824in}}{\pgfqpoint{-0.021960in}{0.000000in}}%
\pgfpathcurveto{\pgfqpoint{-0.021960in}{-0.005824in}}{\pgfqpoint{-0.019646in}{-0.011410in}}{\pgfqpoint{-0.015528in}{-0.015528in}}%
\pgfpathcurveto{\pgfqpoint{-0.011410in}{-0.019646in}}{\pgfqpoint{-0.005824in}{-0.021960in}}{\pgfqpoint{0.000000in}{-0.021960in}}%
\pgfpathlineto{\pgfqpoint{0.000000in}{-0.021960in}}%
\pgfpathclose%
\pgfusepath{stroke,fill}%
}%
\begin{pgfscope}%
\pgfsys@transformshift{1.080663in}{5.276705in}%
\pgfsys@useobject{currentmarker}{}%
\end{pgfscope}%
\begin{pgfscope}%
\pgfsys@transformshift{1.260394in}{5.276469in}%
\pgfsys@useobject{currentmarker}{}%
\end{pgfscope}%
\begin{pgfscope}%
\pgfsys@transformshift{1.440125in}{5.276523in}%
\pgfsys@useobject{currentmarker}{}%
\end{pgfscope}%
\begin{pgfscope}%
\pgfsys@transformshift{1.619856in}{5.276893in}%
\pgfsys@useobject{currentmarker}{}%
\end{pgfscope}%
\begin{pgfscope}%
\pgfsys@transformshift{1.799587in}{5.277433in}%
\pgfsys@useobject{currentmarker}{}%
\end{pgfscope}%
\begin{pgfscope}%
\pgfsys@transformshift{1.979318in}{5.277633in}%
\pgfsys@useobject{currentmarker}{}%
\end{pgfscope}%
\begin{pgfscope}%
\pgfsys@transformshift{2.159049in}{5.277524in}%
\pgfsys@useobject{currentmarker}{}%
\end{pgfscope}%
\begin{pgfscope}%
\pgfsys@transformshift{2.338780in}{5.277809in}%
\pgfsys@useobject{currentmarker}{}%
\end{pgfscope}%
\begin{pgfscope}%
\pgfsys@transformshift{2.518511in}{5.277851in}%
\pgfsys@useobject{currentmarker}{}%
\end{pgfscope}%
\begin{pgfscope}%
\pgfsys@transformshift{2.698242in}{5.278730in}%
\pgfsys@useobject{currentmarker}{}%
\end{pgfscope}%
\begin{pgfscope}%
\pgfsys@transformshift{2.877973in}{5.278785in}%
\pgfsys@useobject{currentmarker}{}%
\end{pgfscope}%
\begin{pgfscope}%
\pgfsys@transformshift{3.057704in}{5.278542in}%
\pgfsys@useobject{currentmarker}{}%
\end{pgfscope}%
\begin{pgfscope}%
\pgfsys@transformshift{3.237435in}{5.280980in}%
\pgfsys@useobject{currentmarker}{}%
\end{pgfscope}%
\begin{pgfscope}%
\pgfsys@transformshift{3.417166in}{5.281532in}%
\pgfsys@useobject{currentmarker}{}%
\end{pgfscope}%
\begin{pgfscope}%
\pgfsys@transformshift{3.596897in}{5.279246in}%
\pgfsys@useobject{currentmarker}{}%
\end{pgfscope}%
\begin{pgfscope}%
\pgfsys@transformshift{3.776628in}{5.284643in}%
\pgfsys@useobject{currentmarker}{}%
\end{pgfscope}%
\begin{pgfscope}%
\pgfsys@transformshift{3.956359in}{5.280544in}%
\pgfsys@useobject{currentmarker}{}%
\end{pgfscope}%
\begin{pgfscope}%
\pgfsys@transformshift{4.136090in}{5.287057in}%
\pgfsys@useobject{currentmarker}{}%
\end{pgfscope}%
\begin{pgfscope}%
\pgfsys@transformshift{4.315821in}{5.283927in}%
\pgfsys@useobject{currentmarker}{}%
\end{pgfscope}%
\begin{pgfscope}%
\pgfsys@transformshift{4.495552in}{5.285977in}%
\pgfsys@useobject{currentmarker}{}%
\end{pgfscope}%
\begin{pgfscope}%
\pgfsys@transformshift{4.675283in}{5.298706in}%
\pgfsys@useobject{currentmarker}{}%
\end{pgfscope}%
\begin{pgfscope}%
\pgfsys@transformshift{4.855014in}{5.304655in}%
\pgfsys@useobject{currentmarker}{}%
\end{pgfscope}%
\begin{pgfscope}%
\pgfsys@transformshift{5.034745in}{5.314369in}%
\pgfsys@useobject{currentmarker}{}%
\end{pgfscope}%
\begin{pgfscope}%
\pgfsys@transformshift{5.214476in}{5.376624in}%
\pgfsys@useobject{currentmarker}{}%
\end{pgfscope}%
\begin{pgfscope}%
\pgfsys@transformshift{5.394207in}{5.585012in}%
\pgfsys@useobject{currentmarker}{}%
\end{pgfscope}%
\begin{pgfscope}%
\pgfsys@transformshift{5.573938in}{5.684949in}%
\pgfsys@useobject{currentmarker}{}%
\end{pgfscope}%
\begin{pgfscope}%
\pgfsys@transformshift{5.753669in}{5.807293in}%
\pgfsys@useobject{currentmarker}{}%
\end{pgfscope}%
\begin{pgfscope}%
\pgfsys@transformshift{5.933400in}{5.842271in}%
\pgfsys@useobject{currentmarker}{}%
\end{pgfscope}%
\begin{pgfscope}%
\pgfsys@transformshift{6.113131in}{5.905811in}%
\pgfsys@useobject{currentmarker}{}%
\end{pgfscope}%
\begin{pgfscope}%
\pgfsys@transformshift{6.292862in}{6.021479in}%
\pgfsys@useobject{currentmarker}{}%
\end{pgfscope}%
\begin{pgfscope}%
\pgfsys@transformshift{6.472593in}{6.038841in}%
\pgfsys@useobject{currentmarker}{}%
\end{pgfscope}%
\begin{pgfscope}%
\pgfsys@transformshift{6.652324in}{6.079912in}%
\pgfsys@useobject{currentmarker}{}%
\end{pgfscope}%
\begin{pgfscope}%
\pgfsys@transformshift{6.832055in}{6.124516in}%
\pgfsys@useobject{currentmarker}{}%
\end{pgfscope}%
\begin{pgfscope}%
\pgfsys@transformshift{7.011786in}{6.162743in}%
\pgfsys@useobject{currentmarker}{}%
\end{pgfscope}%
\begin{pgfscope}%
\pgfsys@transformshift{7.191517in}{6.221120in}%
\pgfsys@useobject{currentmarker}{}%
\end{pgfscope}%
\begin{pgfscope}%
\pgfsys@transformshift{7.371248in}{6.276051in}%
\pgfsys@useobject{currentmarker}{}%
\end{pgfscope}%
\begin{pgfscope}%
\pgfsys@transformshift{7.550979in}{6.268382in}%
\pgfsys@useobject{currentmarker}{}%
\end{pgfscope}%
\begin{pgfscope}%
\pgfsys@transformshift{7.730710in}{6.312328in}%
\pgfsys@useobject{currentmarker}{}%
\end{pgfscope}%
\begin{pgfscope}%
\pgfsys@transformshift{7.910441in}{6.350170in}%
\pgfsys@useobject{currentmarker}{}%
\end{pgfscope}%
\begin{pgfscope}%
\pgfsys@transformshift{8.090172in}{6.373388in}%
\pgfsys@useobject{currentmarker}{}%
\end{pgfscope}%
\begin{pgfscope}%
\pgfsys@transformshift{8.269903in}{6.400598in}%
\pgfsys@useobject{currentmarker}{}%
\end{pgfscope}%
\begin{pgfscope}%
\pgfsys@transformshift{8.449634in}{6.409627in}%
\pgfsys@useobject{currentmarker}{}%
\end{pgfscope}%
\begin{pgfscope}%
\pgfsys@transformshift{8.629365in}{6.434385in}%
\pgfsys@useobject{currentmarker}{}%
\end{pgfscope}%
\begin{pgfscope}%
\pgfsys@transformshift{8.809096in}{6.445445in}%
\pgfsys@useobject{currentmarker}{}%
\end{pgfscope}%
\begin{pgfscope}%
\pgfsys@transformshift{8.988827in}{6.475251in}%
\pgfsys@useobject{currentmarker}{}%
\end{pgfscope}%
\begin{pgfscope}%
\pgfsys@transformshift{9.168558in}{6.490061in}%
\pgfsys@useobject{currentmarker}{}%
\end{pgfscope}%
\begin{pgfscope}%
\pgfsys@transformshift{9.348289in}{6.506995in}%
\pgfsys@useobject{currentmarker}{}%
\end{pgfscope}%
\begin{pgfscope}%
\pgfsys@transformshift{9.528020in}{6.520184in}%
\pgfsys@useobject{currentmarker}{}%
\end{pgfscope}%
\begin{pgfscope}%
\pgfsys@transformshift{9.707751in}{6.544303in}%
\pgfsys@useobject{currentmarker}{}%
\end{pgfscope}%
\begin{pgfscope}%
\pgfsys@transformshift{9.887482in}{6.554959in}%
\pgfsys@useobject{currentmarker}{}%
\end{pgfscope}%
\end{pgfscope}%
\begin{pgfscope}%
\pgfpathrectangle{\pgfqpoint{0.640323in}{5.147436in}}{\pgfqpoint{9.687500in}{3.850000in}}%
\pgfusepath{clip}%
\pgfsetbuttcap%
\pgfsetroundjoin%
\definecolor{currentfill}{rgb}{0.239216,0.478431,0.992157}%
\pgfsetfillcolor{currentfill}%
\pgfsetfillopacity{0.500000}%
\pgfsetlinewidth{1.003750pt}%
\definecolor{currentstroke}{rgb}{0.239216,0.478431,0.992157}%
\pgfsetstrokecolor{currentstroke}%
\pgfsetstrokeopacity{0.500000}%
\pgfsetdash{}{0pt}%
\pgfsys@defobject{currentmarker}{\pgfqpoint{-0.021960in}{-0.021960in}}{\pgfqpoint{0.021960in}{0.021960in}}{%
\pgfpathmoveto{\pgfqpoint{0.000000in}{-0.021960in}}%
\pgfpathcurveto{\pgfqpoint{0.005824in}{-0.021960in}}{\pgfqpoint{0.011410in}{-0.019646in}}{\pgfqpoint{0.015528in}{-0.015528in}}%
\pgfpathcurveto{\pgfqpoint{0.019646in}{-0.011410in}}{\pgfqpoint{0.021960in}{-0.005824in}}{\pgfqpoint{0.021960in}{0.000000in}}%
\pgfpathcurveto{\pgfqpoint{0.021960in}{0.005824in}}{\pgfqpoint{0.019646in}{0.011410in}}{\pgfqpoint{0.015528in}{0.015528in}}%
\pgfpathcurveto{\pgfqpoint{0.011410in}{0.019646in}}{\pgfqpoint{0.005824in}{0.021960in}}{\pgfqpoint{0.000000in}{0.021960in}}%
\pgfpathcurveto{\pgfqpoint{-0.005824in}{0.021960in}}{\pgfqpoint{-0.011410in}{0.019646in}}{\pgfqpoint{-0.015528in}{0.015528in}}%
\pgfpathcurveto{\pgfqpoint{-0.019646in}{0.011410in}}{\pgfqpoint{-0.021960in}{0.005824in}}{\pgfqpoint{-0.021960in}{0.000000in}}%
\pgfpathcurveto{\pgfqpoint{-0.021960in}{-0.005824in}}{\pgfqpoint{-0.019646in}{-0.011410in}}{\pgfqpoint{-0.015528in}{-0.015528in}}%
\pgfpathcurveto{\pgfqpoint{-0.011410in}{-0.019646in}}{\pgfqpoint{-0.005824in}{-0.021960in}}{\pgfqpoint{0.000000in}{-0.021960in}}%
\pgfpathlineto{\pgfqpoint{0.000000in}{-0.021960in}}%
\pgfpathclose%
\pgfusepath{stroke,fill}%
}%
\begin{pgfscope}%
\pgfsys@transformshift{1.080663in}{5.276250in}%
\pgfsys@useobject{currentmarker}{}%
\end{pgfscope}%
\begin{pgfscope}%
\pgfsys@transformshift{1.260394in}{5.276256in}%
\pgfsys@useobject{currentmarker}{}%
\end{pgfscope}%
\begin{pgfscope}%
\pgfsys@transformshift{1.440125in}{5.276743in}%
\pgfsys@useobject{currentmarker}{}%
\end{pgfscope}%
\begin{pgfscope}%
\pgfsys@transformshift{1.619856in}{5.276764in}%
\pgfsys@useobject{currentmarker}{}%
\end{pgfscope}%
\begin{pgfscope}%
\pgfsys@transformshift{1.799587in}{5.277131in}%
\pgfsys@useobject{currentmarker}{}%
\end{pgfscope}%
\begin{pgfscope}%
\pgfsys@transformshift{1.979318in}{5.277202in}%
\pgfsys@useobject{currentmarker}{}%
\end{pgfscope}%
\begin{pgfscope}%
\pgfsys@transformshift{2.159049in}{5.277747in}%
\pgfsys@useobject{currentmarker}{}%
\end{pgfscope}%
\begin{pgfscope}%
\pgfsys@transformshift{2.338780in}{5.277872in}%
\pgfsys@useobject{currentmarker}{}%
\end{pgfscope}%
\begin{pgfscope}%
\pgfsys@transformshift{2.518511in}{5.278391in}%
\pgfsys@useobject{currentmarker}{}%
\end{pgfscope}%
\begin{pgfscope}%
\pgfsys@transformshift{2.698242in}{5.278438in}%
\pgfsys@useobject{currentmarker}{}%
\end{pgfscope}%
\begin{pgfscope}%
\pgfsys@transformshift{2.877973in}{5.279081in}%
\pgfsys@useobject{currentmarker}{}%
\end{pgfscope}%
\begin{pgfscope}%
\pgfsys@transformshift{3.057704in}{5.279279in}%
\pgfsys@useobject{currentmarker}{}%
\end{pgfscope}%
\begin{pgfscope}%
\pgfsys@transformshift{3.237435in}{5.279899in}%
\pgfsys@useobject{currentmarker}{}%
\end{pgfscope}%
\begin{pgfscope}%
\pgfsys@transformshift{3.417166in}{5.281779in}%
\pgfsys@useobject{currentmarker}{}%
\end{pgfscope}%
\begin{pgfscope}%
\pgfsys@transformshift{3.596897in}{5.282287in}%
\pgfsys@useobject{currentmarker}{}%
\end{pgfscope}%
\begin{pgfscope}%
\pgfsys@transformshift{3.776628in}{5.282116in}%
\pgfsys@useobject{currentmarker}{}%
\end{pgfscope}%
\begin{pgfscope}%
\pgfsys@transformshift{3.956359in}{5.283697in}%
\pgfsys@useobject{currentmarker}{}%
\end{pgfscope}%
\begin{pgfscope}%
\pgfsys@transformshift{4.136090in}{5.284625in}%
\pgfsys@useobject{currentmarker}{}%
\end{pgfscope}%
\begin{pgfscope}%
\pgfsys@transformshift{4.315821in}{5.287903in}%
\pgfsys@useobject{currentmarker}{}%
\end{pgfscope}%
\begin{pgfscope}%
\pgfsys@transformshift{4.495552in}{5.292931in}%
\pgfsys@useobject{currentmarker}{}%
\end{pgfscope}%
\begin{pgfscope}%
\pgfsys@transformshift{4.675283in}{5.300599in}%
\pgfsys@useobject{currentmarker}{}%
\end{pgfscope}%
\begin{pgfscope}%
\pgfsys@transformshift{4.855014in}{5.322623in}%
\pgfsys@useobject{currentmarker}{}%
\end{pgfscope}%
\begin{pgfscope}%
\pgfsys@transformshift{5.034745in}{5.420359in}%
\pgfsys@useobject{currentmarker}{}%
\end{pgfscope}%
\begin{pgfscope}%
\pgfsys@transformshift{5.214476in}{5.518179in}%
\pgfsys@useobject{currentmarker}{}%
\end{pgfscope}%
\begin{pgfscope}%
\pgfsys@transformshift{5.394207in}{5.674770in}%
\pgfsys@useobject{currentmarker}{}%
\end{pgfscope}%
\begin{pgfscope}%
\pgfsys@transformshift{5.573938in}{5.769451in}%
\pgfsys@useobject{currentmarker}{}%
\end{pgfscope}%
\begin{pgfscope}%
\pgfsys@transformshift{5.753669in}{5.871621in}%
\pgfsys@useobject{currentmarker}{}%
\end{pgfscope}%
\begin{pgfscope}%
\pgfsys@transformshift{5.933400in}{5.959183in}%
\pgfsys@useobject{currentmarker}{}%
\end{pgfscope}%
\begin{pgfscope}%
\pgfsys@transformshift{6.113131in}{6.000292in}%
\pgfsys@useobject{currentmarker}{}%
\end{pgfscope}%
\begin{pgfscope}%
\pgfsys@transformshift{6.292862in}{6.066928in}%
\pgfsys@useobject{currentmarker}{}%
\end{pgfscope}%
\begin{pgfscope}%
\pgfsys@transformshift{6.472593in}{6.103111in}%
\pgfsys@useobject{currentmarker}{}%
\end{pgfscope}%
\begin{pgfscope}%
\pgfsys@transformshift{6.652324in}{6.141369in}%
\pgfsys@useobject{currentmarker}{}%
\end{pgfscope}%
\begin{pgfscope}%
\pgfsys@transformshift{6.832055in}{6.191748in}%
\pgfsys@useobject{currentmarker}{}%
\end{pgfscope}%
\begin{pgfscope}%
\pgfsys@transformshift{7.011786in}{6.245009in}%
\pgfsys@useobject{currentmarker}{}%
\end{pgfscope}%
\begin{pgfscope}%
\pgfsys@transformshift{7.191517in}{6.266047in}%
\pgfsys@useobject{currentmarker}{}%
\end{pgfscope}%
\begin{pgfscope}%
\pgfsys@transformshift{7.371248in}{6.301194in}%
\pgfsys@useobject{currentmarker}{}%
\end{pgfscope}%
\begin{pgfscope}%
\pgfsys@transformshift{7.550979in}{6.332093in}%
\pgfsys@useobject{currentmarker}{}%
\end{pgfscope}%
\begin{pgfscope}%
\pgfsys@transformshift{7.730710in}{6.362396in}%
\pgfsys@useobject{currentmarker}{}%
\end{pgfscope}%
\begin{pgfscope}%
\pgfsys@transformshift{7.910441in}{6.373996in}%
\pgfsys@useobject{currentmarker}{}%
\end{pgfscope}%
\begin{pgfscope}%
\pgfsys@transformshift{8.090172in}{6.410559in}%
\pgfsys@useobject{currentmarker}{}%
\end{pgfscope}%
\begin{pgfscope}%
\pgfsys@transformshift{8.269903in}{6.434006in}%
\pgfsys@useobject{currentmarker}{}%
\end{pgfscope}%
\begin{pgfscope}%
\pgfsys@transformshift{8.449634in}{6.460640in}%
\pgfsys@useobject{currentmarker}{}%
\end{pgfscope}%
\begin{pgfscope}%
\pgfsys@transformshift{8.629365in}{6.481896in}%
\pgfsys@useobject{currentmarker}{}%
\end{pgfscope}%
\begin{pgfscope}%
\pgfsys@transformshift{8.809096in}{6.502561in}%
\pgfsys@useobject{currentmarker}{}%
\end{pgfscope}%
\begin{pgfscope}%
\pgfsys@transformshift{8.988827in}{6.510205in}%
\pgfsys@useobject{currentmarker}{}%
\end{pgfscope}%
\begin{pgfscope}%
\pgfsys@transformshift{9.168558in}{6.531380in}%
\pgfsys@useobject{currentmarker}{}%
\end{pgfscope}%
\begin{pgfscope}%
\pgfsys@transformshift{9.348289in}{6.541744in}%
\pgfsys@useobject{currentmarker}{}%
\end{pgfscope}%
\begin{pgfscope}%
\pgfsys@transformshift{9.528020in}{6.555753in}%
\pgfsys@useobject{currentmarker}{}%
\end{pgfscope}%
\begin{pgfscope}%
\pgfsys@transformshift{9.707751in}{6.575891in}%
\pgfsys@useobject{currentmarker}{}%
\end{pgfscope}%
\begin{pgfscope}%
\pgfsys@transformshift{9.887482in}{6.591403in}%
\pgfsys@useobject{currentmarker}{}%
\end{pgfscope}%
\end{pgfscope}%
\begin{pgfscope}%
\pgfpathrectangle{\pgfqpoint{0.640323in}{5.147436in}}{\pgfqpoint{9.687500in}{3.850000in}}%
\pgfusepath{clip}%
\pgfsetbuttcap%
\pgfsetroundjoin%
\definecolor{currentfill}{rgb}{0.000000,0.000000,0.000000}%
\pgfsetfillcolor{currentfill}%
\pgfsetfillopacity{0.500000}%
\pgfsetlinewidth{1.003750pt}%
\definecolor{currentstroke}{rgb}{0.000000,0.000000,0.000000}%
\pgfsetstrokecolor{currentstroke}%
\pgfsetstrokeopacity{0.500000}%
\pgfsetdash{}{0pt}%
\pgfsys@defobject{currentmarker}{\pgfqpoint{-0.021960in}{-0.021960in}}{\pgfqpoint{0.021960in}{0.021960in}}{%
\pgfpathmoveto{\pgfqpoint{0.000000in}{-0.021960in}}%
\pgfpathcurveto{\pgfqpoint{0.005824in}{-0.021960in}}{\pgfqpoint{0.011410in}{-0.019646in}}{\pgfqpoint{0.015528in}{-0.015528in}}%
\pgfpathcurveto{\pgfqpoint{0.019646in}{-0.011410in}}{\pgfqpoint{0.021960in}{-0.005824in}}{\pgfqpoint{0.021960in}{0.000000in}}%
\pgfpathcurveto{\pgfqpoint{0.021960in}{0.005824in}}{\pgfqpoint{0.019646in}{0.011410in}}{\pgfqpoint{0.015528in}{0.015528in}}%
\pgfpathcurveto{\pgfqpoint{0.011410in}{0.019646in}}{\pgfqpoint{0.005824in}{0.021960in}}{\pgfqpoint{0.000000in}{0.021960in}}%
\pgfpathcurveto{\pgfqpoint{-0.005824in}{0.021960in}}{\pgfqpoint{-0.011410in}{0.019646in}}{\pgfqpoint{-0.015528in}{0.015528in}}%
\pgfpathcurveto{\pgfqpoint{-0.019646in}{0.011410in}}{\pgfqpoint{-0.021960in}{0.005824in}}{\pgfqpoint{-0.021960in}{0.000000in}}%
\pgfpathcurveto{\pgfqpoint{-0.021960in}{-0.005824in}}{\pgfqpoint{-0.019646in}{-0.011410in}}{\pgfqpoint{-0.015528in}{-0.015528in}}%
\pgfpathcurveto{\pgfqpoint{-0.011410in}{-0.019646in}}{\pgfqpoint{-0.005824in}{-0.021960in}}{\pgfqpoint{0.000000in}{-0.021960in}}%
\pgfpathlineto{\pgfqpoint{0.000000in}{-0.021960in}}%
\pgfpathclose%
\pgfusepath{stroke,fill}%
}%
\begin{pgfscope}%
\pgfsys@transformshift{1.080663in}{5.276317in}%
\pgfsys@useobject{currentmarker}{}%
\end{pgfscope}%
\begin{pgfscope}%
\pgfsys@transformshift{1.260394in}{5.276409in}%
\pgfsys@useobject{currentmarker}{}%
\end{pgfscope}%
\begin{pgfscope}%
\pgfsys@transformshift{1.440125in}{5.276751in}%
\pgfsys@useobject{currentmarker}{}%
\end{pgfscope}%
\begin{pgfscope}%
\pgfsys@transformshift{1.619856in}{5.276889in}%
\pgfsys@useobject{currentmarker}{}%
\end{pgfscope}%
\begin{pgfscope}%
\pgfsys@transformshift{1.799587in}{5.277083in}%
\pgfsys@useobject{currentmarker}{}%
\end{pgfscope}%
\begin{pgfscope}%
\pgfsys@transformshift{1.979318in}{5.277261in}%
\pgfsys@useobject{currentmarker}{}%
\end{pgfscope}%
\begin{pgfscope}%
\pgfsys@transformshift{2.159049in}{5.277601in}%
\pgfsys@useobject{currentmarker}{}%
\end{pgfscope}%
\begin{pgfscope}%
\pgfsys@transformshift{2.338780in}{5.277905in}%
\pgfsys@useobject{currentmarker}{}%
\end{pgfscope}%
\begin{pgfscope}%
\pgfsys@transformshift{2.518511in}{5.278229in}%
\pgfsys@useobject{currentmarker}{}%
\end{pgfscope}%
\begin{pgfscope}%
\pgfsys@transformshift{2.698242in}{5.278620in}%
\pgfsys@useobject{currentmarker}{}%
\end{pgfscope}%
\begin{pgfscope}%
\pgfsys@transformshift{2.877973in}{5.278832in}%
\pgfsys@useobject{currentmarker}{}%
\end{pgfscope}%
\begin{pgfscope}%
\pgfsys@transformshift{3.057704in}{5.279708in}%
\pgfsys@useobject{currentmarker}{}%
\end{pgfscope}%
\begin{pgfscope}%
\pgfsys@transformshift{3.237435in}{5.280171in}%
\pgfsys@useobject{currentmarker}{}%
\end{pgfscope}%
\begin{pgfscope}%
\pgfsys@transformshift{3.417166in}{5.280771in}%
\pgfsys@useobject{currentmarker}{}%
\end{pgfscope}%
\begin{pgfscope}%
\pgfsys@transformshift{3.596897in}{5.282038in}%
\pgfsys@useobject{currentmarker}{}%
\end{pgfscope}%
\begin{pgfscope}%
\pgfsys@transformshift{3.776628in}{5.282917in}%
\pgfsys@useobject{currentmarker}{}%
\end{pgfscope}%
\begin{pgfscope}%
\pgfsys@transformshift{3.956359in}{5.283727in}%
\pgfsys@useobject{currentmarker}{}%
\end{pgfscope}%
\begin{pgfscope}%
\pgfsys@transformshift{4.136090in}{5.285470in}%
\pgfsys@useobject{currentmarker}{}%
\end{pgfscope}%
\begin{pgfscope}%
\pgfsys@transformshift{4.315821in}{5.288475in}%
\pgfsys@useobject{currentmarker}{}%
\end{pgfscope}%
\begin{pgfscope}%
\pgfsys@transformshift{4.495552in}{5.291315in}%
\pgfsys@useobject{currentmarker}{}%
\end{pgfscope}%
\begin{pgfscope}%
\pgfsys@transformshift{4.675283in}{5.296769in}%
\pgfsys@useobject{currentmarker}{}%
\end{pgfscope}%
\begin{pgfscope}%
\pgfsys@transformshift{4.855014in}{5.323733in}%
\pgfsys@useobject{currentmarker}{}%
\end{pgfscope}%
\begin{pgfscope}%
\pgfsys@transformshift{5.034745in}{5.428259in}%
\pgfsys@useobject{currentmarker}{}%
\end{pgfscope}%
\begin{pgfscope}%
\pgfsys@transformshift{5.214476in}{5.593213in}%
\pgfsys@useobject{currentmarker}{}%
\end{pgfscope}%
\begin{pgfscope}%
\pgfsys@transformshift{5.394207in}{5.708419in}%
\pgfsys@useobject{currentmarker}{}%
\end{pgfscope}%
\begin{pgfscope}%
\pgfsys@transformshift{5.573938in}{5.825315in}%
\pgfsys@useobject{currentmarker}{}%
\end{pgfscope}%
\begin{pgfscope}%
\pgfsys@transformshift{5.753669in}{5.898707in}%
\pgfsys@useobject{currentmarker}{}%
\end{pgfscope}%
\begin{pgfscope}%
\pgfsys@transformshift{5.933400in}{5.968647in}%
\pgfsys@useobject{currentmarker}{}%
\end{pgfscope}%
\begin{pgfscope}%
\pgfsys@transformshift{6.113131in}{6.037630in}%
\pgfsys@useobject{currentmarker}{}%
\end{pgfscope}%
\begin{pgfscope}%
\pgfsys@transformshift{6.292862in}{6.083787in}%
\pgfsys@useobject{currentmarker}{}%
\end{pgfscope}%
\begin{pgfscope}%
\pgfsys@transformshift{6.472593in}{6.138016in}%
\pgfsys@useobject{currentmarker}{}%
\end{pgfscope}%
\begin{pgfscope}%
\pgfsys@transformshift{6.652324in}{6.177137in}%
\pgfsys@useobject{currentmarker}{}%
\end{pgfscope}%
\begin{pgfscope}%
\pgfsys@transformshift{6.832055in}{6.222561in}%
\pgfsys@useobject{currentmarker}{}%
\end{pgfscope}%
\begin{pgfscope}%
\pgfsys@transformshift{7.011786in}{6.256372in}%
\pgfsys@useobject{currentmarker}{}%
\end{pgfscope}%
\begin{pgfscope}%
\pgfsys@transformshift{7.191517in}{6.291171in}%
\pgfsys@useobject{currentmarker}{}%
\end{pgfscope}%
\begin{pgfscope}%
\pgfsys@transformshift{7.371248in}{6.327728in}%
\pgfsys@useobject{currentmarker}{}%
\end{pgfscope}%
\begin{pgfscope}%
\pgfsys@transformshift{7.550979in}{6.351095in}%
\pgfsys@useobject{currentmarker}{}%
\end{pgfscope}%
\begin{pgfscope}%
\pgfsys@transformshift{7.730710in}{6.377424in}%
\pgfsys@useobject{currentmarker}{}%
\end{pgfscope}%
\begin{pgfscope}%
\pgfsys@transformshift{7.910441in}{6.405759in}%
\pgfsys@useobject{currentmarker}{}%
\end{pgfscope}%
\begin{pgfscope}%
\pgfsys@transformshift{8.090172in}{6.427561in}%
\pgfsys@useobject{currentmarker}{}%
\end{pgfscope}%
\begin{pgfscope}%
\pgfsys@transformshift{8.269903in}{6.450990in}%
\pgfsys@useobject{currentmarker}{}%
\end{pgfscope}%
\begin{pgfscope}%
\pgfsys@transformshift{8.449634in}{6.472084in}%
\pgfsys@useobject{currentmarker}{}%
\end{pgfscope}%
\begin{pgfscope}%
\pgfsys@transformshift{8.629365in}{6.491545in}%
\pgfsys@useobject{currentmarker}{}%
\end{pgfscope}%
\begin{pgfscope}%
\pgfsys@transformshift{8.809096in}{6.511776in}%
\pgfsys@useobject{currentmarker}{}%
\end{pgfscope}%
\begin{pgfscope}%
\pgfsys@transformshift{8.988827in}{6.530474in}%
\pgfsys@useobject{currentmarker}{}%
\end{pgfscope}%
\begin{pgfscope}%
\pgfsys@transformshift{9.168558in}{6.545656in}%
\pgfsys@useobject{currentmarker}{}%
\end{pgfscope}%
\begin{pgfscope}%
\pgfsys@transformshift{9.348289in}{6.564708in}%
\pgfsys@useobject{currentmarker}{}%
\end{pgfscope}%
\begin{pgfscope}%
\pgfsys@transformshift{9.528020in}{6.576997in}%
\pgfsys@useobject{currentmarker}{}%
\end{pgfscope}%
\begin{pgfscope}%
\pgfsys@transformshift{9.707751in}{6.591416in}%
\pgfsys@useobject{currentmarker}{}%
\end{pgfscope}%
\begin{pgfscope}%
\pgfsys@transformshift{9.887482in}{6.607548in}%
\pgfsys@useobject{currentmarker}{}%
\end{pgfscope}%
\end{pgfscope}%
\begin{pgfscope}%
\pgfpathrectangle{\pgfqpoint{0.640323in}{5.147436in}}{\pgfqpoint{9.687500in}{3.850000in}}%
\pgfusepath{clip}%
\pgfsetrectcap%
\pgfsetroundjoin%
\pgfsetlinewidth{0.803000pt}%
\definecolor{currentstroke}{rgb}{0.690196,0.690196,0.690196}%
\pgfsetstrokecolor{currentstroke}%
\pgfsetdash{}{0pt}%
\pgfpathmoveto{\pgfqpoint{1.080663in}{5.147436in}}%
\pgfpathlineto{\pgfqpoint{1.080663in}{8.997436in}}%
\pgfusepath{stroke}%
\end{pgfscope}%
\begin{pgfscope}%
\pgfsetbuttcap%
\pgfsetroundjoin%
\definecolor{currentfill}{rgb}{0.000000,0.000000,0.000000}%
\pgfsetfillcolor{currentfill}%
\pgfsetlinewidth{0.803000pt}%
\definecolor{currentstroke}{rgb}{0.000000,0.000000,0.000000}%
\pgfsetstrokecolor{currentstroke}%
\pgfsetdash{}{0pt}%
\pgfsys@defobject{currentmarker}{\pgfqpoint{0.000000in}{-0.048611in}}{\pgfqpoint{0.000000in}{0.000000in}}{%
\pgfpathmoveto{\pgfqpoint{0.000000in}{0.000000in}}%
\pgfpathlineto{\pgfqpoint{0.000000in}{-0.048611in}}%
\pgfusepath{stroke,fill}%
}%
\begin{pgfscope}%
\pgfsys@transformshift{1.080663in}{5.147436in}%
\pgfsys@useobject{currentmarker}{}%
\end{pgfscope}%
\end{pgfscope}%
\begin{pgfscope}%
\definecolor{textcolor}{rgb}{0.000000,0.000000,0.000000}%
\pgfsetstrokecolor{textcolor}%
\pgfsetfillcolor{textcolor}%
\pgftext[x=1.080663in,y=5.050214in,,top]{\color{textcolor}\sffamily\fontsize{10.000000}{12.000000}\selectfont 0.0}%
\end{pgfscope}%
\begin{pgfscope}%
\pgfpathrectangle{\pgfqpoint{0.640323in}{5.147436in}}{\pgfqpoint{9.687500in}{3.850000in}}%
\pgfusepath{clip}%
\pgfsetrectcap%
\pgfsetroundjoin%
\pgfsetlinewidth{0.803000pt}%
\definecolor{currentstroke}{rgb}{0.690196,0.690196,0.690196}%
\pgfsetstrokecolor{currentstroke}%
\pgfsetdash{}{0pt}%
\pgfpathmoveto{\pgfqpoint{2.877973in}{5.147436in}}%
\pgfpathlineto{\pgfqpoint{2.877973in}{8.997436in}}%
\pgfusepath{stroke}%
\end{pgfscope}%
\begin{pgfscope}%
\pgfsetbuttcap%
\pgfsetroundjoin%
\definecolor{currentfill}{rgb}{0.000000,0.000000,0.000000}%
\pgfsetfillcolor{currentfill}%
\pgfsetlinewidth{0.803000pt}%
\definecolor{currentstroke}{rgb}{0.000000,0.000000,0.000000}%
\pgfsetstrokecolor{currentstroke}%
\pgfsetdash{}{0pt}%
\pgfsys@defobject{currentmarker}{\pgfqpoint{0.000000in}{-0.048611in}}{\pgfqpoint{0.000000in}{0.000000in}}{%
\pgfpathmoveto{\pgfqpoint{0.000000in}{0.000000in}}%
\pgfpathlineto{\pgfqpoint{0.000000in}{-0.048611in}}%
\pgfusepath{stroke,fill}%
}%
\begin{pgfscope}%
\pgfsys@transformshift{2.877973in}{5.147436in}%
\pgfsys@useobject{currentmarker}{}%
\end{pgfscope}%
\end{pgfscope}%
\begin{pgfscope}%
\definecolor{textcolor}{rgb}{0.000000,0.000000,0.000000}%
\pgfsetstrokecolor{textcolor}%
\pgfsetfillcolor{textcolor}%
\pgftext[x=2.877973in,y=5.050214in,,top]{\color{textcolor}\sffamily\fontsize{10.000000}{12.000000}\selectfont 0.2}%
\end{pgfscope}%
\begin{pgfscope}%
\pgfpathrectangle{\pgfqpoint{0.640323in}{5.147436in}}{\pgfqpoint{9.687500in}{3.850000in}}%
\pgfusepath{clip}%
\pgfsetrectcap%
\pgfsetroundjoin%
\pgfsetlinewidth{0.803000pt}%
\definecolor{currentstroke}{rgb}{0.690196,0.690196,0.690196}%
\pgfsetstrokecolor{currentstroke}%
\pgfsetdash{}{0pt}%
\pgfpathmoveto{\pgfqpoint{4.675283in}{5.147436in}}%
\pgfpathlineto{\pgfqpoint{4.675283in}{8.997436in}}%
\pgfusepath{stroke}%
\end{pgfscope}%
\begin{pgfscope}%
\pgfsetbuttcap%
\pgfsetroundjoin%
\definecolor{currentfill}{rgb}{0.000000,0.000000,0.000000}%
\pgfsetfillcolor{currentfill}%
\pgfsetlinewidth{0.803000pt}%
\definecolor{currentstroke}{rgb}{0.000000,0.000000,0.000000}%
\pgfsetstrokecolor{currentstroke}%
\pgfsetdash{}{0pt}%
\pgfsys@defobject{currentmarker}{\pgfqpoint{0.000000in}{-0.048611in}}{\pgfqpoint{0.000000in}{0.000000in}}{%
\pgfpathmoveto{\pgfqpoint{0.000000in}{0.000000in}}%
\pgfpathlineto{\pgfqpoint{0.000000in}{-0.048611in}}%
\pgfusepath{stroke,fill}%
}%
\begin{pgfscope}%
\pgfsys@transformshift{4.675283in}{5.147436in}%
\pgfsys@useobject{currentmarker}{}%
\end{pgfscope}%
\end{pgfscope}%
\begin{pgfscope}%
\definecolor{textcolor}{rgb}{0.000000,0.000000,0.000000}%
\pgfsetstrokecolor{textcolor}%
\pgfsetfillcolor{textcolor}%
\pgftext[x=4.675283in,y=5.050214in,,top]{\color{textcolor}\sffamily\fontsize{10.000000}{12.000000}\selectfont 0.4}%
\end{pgfscope}%
\begin{pgfscope}%
\pgfpathrectangle{\pgfqpoint{0.640323in}{5.147436in}}{\pgfqpoint{9.687500in}{3.850000in}}%
\pgfusepath{clip}%
\pgfsetrectcap%
\pgfsetroundjoin%
\pgfsetlinewidth{0.803000pt}%
\definecolor{currentstroke}{rgb}{0.690196,0.690196,0.690196}%
\pgfsetstrokecolor{currentstroke}%
\pgfsetdash{}{0pt}%
\pgfpathmoveto{\pgfqpoint{6.472593in}{5.147436in}}%
\pgfpathlineto{\pgfqpoint{6.472593in}{8.997436in}}%
\pgfusepath{stroke}%
\end{pgfscope}%
\begin{pgfscope}%
\pgfsetbuttcap%
\pgfsetroundjoin%
\definecolor{currentfill}{rgb}{0.000000,0.000000,0.000000}%
\pgfsetfillcolor{currentfill}%
\pgfsetlinewidth{0.803000pt}%
\definecolor{currentstroke}{rgb}{0.000000,0.000000,0.000000}%
\pgfsetstrokecolor{currentstroke}%
\pgfsetdash{}{0pt}%
\pgfsys@defobject{currentmarker}{\pgfqpoint{0.000000in}{-0.048611in}}{\pgfqpoint{0.000000in}{0.000000in}}{%
\pgfpathmoveto{\pgfqpoint{0.000000in}{0.000000in}}%
\pgfpathlineto{\pgfqpoint{0.000000in}{-0.048611in}}%
\pgfusepath{stroke,fill}%
}%
\begin{pgfscope}%
\pgfsys@transformshift{6.472593in}{5.147436in}%
\pgfsys@useobject{currentmarker}{}%
\end{pgfscope}%
\end{pgfscope}%
\begin{pgfscope}%
\definecolor{textcolor}{rgb}{0.000000,0.000000,0.000000}%
\pgfsetstrokecolor{textcolor}%
\pgfsetfillcolor{textcolor}%
\pgftext[x=6.472593in,y=5.050214in,,top]{\color{textcolor}\sffamily\fontsize{10.000000}{12.000000}\selectfont 0.6}%
\end{pgfscope}%
\begin{pgfscope}%
\pgfpathrectangle{\pgfqpoint{0.640323in}{5.147436in}}{\pgfqpoint{9.687500in}{3.850000in}}%
\pgfusepath{clip}%
\pgfsetrectcap%
\pgfsetroundjoin%
\pgfsetlinewidth{0.803000pt}%
\definecolor{currentstroke}{rgb}{0.690196,0.690196,0.690196}%
\pgfsetstrokecolor{currentstroke}%
\pgfsetdash{}{0pt}%
\pgfpathmoveto{\pgfqpoint{8.269903in}{5.147436in}}%
\pgfpathlineto{\pgfqpoint{8.269903in}{8.997436in}}%
\pgfusepath{stroke}%
\end{pgfscope}%
\begin{pgfscope}%
\pgfsetbuttcap%
\pgfsetroundjoin%
\definecolor{currentfill}{rgb}{0.000000,0.000000,0.000000}%
\pgfsetfillcolor{currentfill}%
\pgfsetlinewidth{0.803000pt}%
\definecolor{currentstroke}{rgb}{0.000000,0.000000,0.000000}%
\pgfsetstrokecolor{currentstroke}%
\pgfsetdash{}{0pt}%
\pgfsys@defobject{currentmarker}{\pgfqpoint{0.000000in}{-0.048611in}}{\pgfqpoint{0.000000in}{0.000000in}}{%
\pgfpathmoveto{\pgfqpoint{0.000000in}{0.000000in}}%
\pgfpathlineto{\pgfqpoint{0.000000in}{-0.048611in}}%
\pgfusepath{stroke,fill}%
}%
\begin{pgfscope}%
\pgfsys@transformshift{8.269903in}{5.147436in}%
\pgfsys@useobject{currentmarker}{}%
\end{pgfscope}%
\end{pgfscope}%
\begin{pgfscope}%
\definecolor{textcolor}{rgb}{0.000000,0.000000,0.000000}%
\pgfsetstrokecolor{textcolor}%
\pgfsetfillcolor{textcolor}%
\pgftext[x=8.269903in,y=5.050214in,,top]{\color{textcolor}\sffamily\fontsize{10.000000}{12.000000}\selectfont 0.8}%
\end{pgfscope}%
\begin{pgfscope}%
\pgfpathrectangle{\pgfqpoint{0.640323in}{5.147436in}}{\pgfqpoint{9.687500in}{3.850000in}}%
\pgfusepath{clip}%
\pgfsetrectcap%
\pgfsetroundjoin%
\pgfsetlinewidth{0.803000pt}%
\definecolor{currentstroke}{rgb}{0.690196,0.690196,0.690196}%
\pgfsetstrokecolor{currentstroke}%
\pgfsetdash{}{0pt}%
\pgfpathmoveto{\pgfqpoint{10.067213in}{5.147436in}}%
\pgfpathlineto{\pgfqpoint{10.067213in}{8.997436in}}%
\pgfusepath{stroke}%
\end{pgfscope}%
\begin{pgfscope}%
\pgfsetbuttcap%
\pgfsetroundjoin%
\definecolor{currentfill}{rgb}{0.000000,0.000000,0.000000}%
\pgfsetfillcolor{currentfill}%
\pgfsetlinewidth{0.803000pt}%
\definecolor{currentstroke}{rgb}{0.000000,0.000000,0.000000}%
\pgfsetstrokecolor{currentstroke}%
\pgfsetdash{}{0pt}%
\pgfsys@defobject{currentmarker}{\pgfqpoint{0.000000in}{-0.048611in}}{\pgfqpoint{0.000000in}{0.000000in}}{%
\pgfpathmoveto{\pgfqpoint{0.000000in}{0.000000in}}%
\pgfpathlineto{\pgfqpoint{0.000000in}{-0.048611in}}%
\pgfusepath{stroke,fill}%
}%
\begin{pgfscope}%
\pgfsys@transformshift{10.067213in}{5.147436in}%
\pgfsys@useobject{currentmarker}{}%
\end{pgfscope}%
\end{pgfscope}%
\begin{pgfscope}%
\definecolor{textcolor}{rgb}{0.000000,0.000000,0.000000}%
\pgfsetstrokecolor{textcolor}%
\pgfsetfillcolor{textcolor}%
\pgftext[x=10.067213in,y=5.050214in,,top]{\color{textcolor}\sffamily\fontsize{10.000000}{12.000000}\selectfont 1.0}%
\end{pgfscope}%
\begin{pgfscope}%
\pgfpathrectangle{\pgfqpoint{0.640323in}{5.147436in}}{\pgfqpoint{9.687500in}{3.850000in}}%
\pgfusepath{clip}%
\pgfsetrectcap%
\pgfsetroundjoin%
\pgfsetlinewidth{0.803000pt}%
\definecolor{currentstroke}{rgb}{0.600000,0.600000,0.600000}%
\pgfsetstrokecolor{currentstroke}%
\pgfsetstrokeopacity{0.200000}%
\pgfsetdash{}{0pt}%
\pgfpathmoveto{\pgfqpoint{1.529991in}{5.147436in}}%
\pgfpathlineto{\pgfqpoint{1.529991in}{8.997436in}}%
\pgfusepath{stroke}%
\end{pgfscope}%
\begin{pgfscope}%
\pgfsetbuttcap%
\pgfsetroundjoin%
\definecolor{currentfill}{rgb}{0.000000,0.000000,0.000000}%
\pgfsetfillcolor{currentfill}%
\pgfsetlinewidth{0.602250pt}%
\definecolor{currentstroke}{rgb}{0.000000,0.000000,0.000000}%
\pgfsetstrokecolor{currentstroke}%
\pgfsetdash{}{0pt}%
\pgfsys@defobject{currentmarker}{\pgfqpoint{0.000000in}{-0.027778in}}{\pgfqpoint{0.000000in}{0.000000in}}{%
\pgfpathmoveto{\pgfqpoint{0.000000in}{0.000000in}}%
\pgfpathlineto{\pgfqpoint{0.000000in}{-0.027778in}}%
\pgfusepath{stroke,fill}%
}%
\begin{pgfscope}%
\pgfsys@transformshift{1.529991in}{5.147436in}%
\pgfsys@useobject{currentmarker}{}%
\end{pgfscope}%
\end{pgfscope}%
\begin{pgfscope}%
\pgfpathrectangle{\pgfqpoint{0.640323in}{5.147436in}}{\pgfqpoint{9.687500in}{3.850000in}}%
\pgfusepath{clip}%
\pgfsetrectcap%
\pgfsetroundjoin%
\pgfsetlinewidth{0.803000pt}%
\definecolor{currentstroke}{rgb}{0.600000,0.600000,0.600000}%
\pgfsetstrokecolor{currentstroke}%
\pgfsetstrokeopacity{0.200000}%
\pgfsetdash{}{0pt}%
\pgfpathmoveto{\pgfqpoint{1.979318in}{5.147436in}}%
\pgfpathlineto{\pgfqpoint{1.979318in}{8.997436in}}%
\pgfusepath{stroke}%
\end{pgfscope}%
\begin{pgfscope}%
\pgfsetbuttcap%
\pgfsetroundjoin%
\definecolor{currentfill}{rgb}{0.000000,0.000000,0.000000}%
\pgfsetfillcolor{currentfill}%
\pgfsetlinewidth{0.602250pt}%
\definecolor{currentstroke}{rgb}{0.000000,0.000000,0.000000}%
\pgfsetstrokecolor{currentstroke}%
\pgfsetdash{}{0pt}%
\pgfsys@defobject{currentmarker}{\pgfqpoint{0.000000in}{-0.027778in}}{\pgfqpoint{0.000000in}{0.000000in}}{%
\pgfpathmoveto{\pgfqpoint{0.000000in}{0.000000in}}%
\pgfpathlineto{\pgfqpoint{0.000000in}{-0.027778in}}%
\pgfusepath{stroke,fill}%
}%
\begin{pgfscope}%
\pgfsys@transformshift{1.979318in}{5.147436in}%
\pgfsys@useobject{currentmarker}{}%
\end{pgfscope}%
\end{pgfscope}%
\begin{pgfscope}%
\pgfpathrectangle{\pgfqpoint{0.640323in}{5.147436in}}{\pgfqpoint{9.687500in}{3.850000in}}%
\pgfusepath{clip}%
\pgfsetrectcap%
\pgfsetroundjoin%
\pgfsetlinewidth{0.803000pt}%
\definecolor{currentstroke}{rgb}{0.600000,0.600000,0.600000}%
\pgfsetstrokecolor{currentstroke}%
\pgfsetstrokeopacity{0.200000}%
\pgfsetdash{}{0pt}%
\pgfpathmoveto{\pgfqpoint{2.428646in}{5.147436in}}%
\pgfpathlineto{\pgfqpoint{2.428646in}{8.997436in}}%
\pgfusepath{stroke}%
\end{pgfscope}%
\begin{pgfscope}%
\pgfsetbuttcap%
\pgfsetroundjoin%
\definecolor{currentfill}{rgb}{0.000000,0.000000,0.000000}%
\pgfsetfillcolor{currentfill}%
\pgfsetlinewidth{0.602250pt}%
\definecolor{currentstroke}{rgb}{0.000000,0.000000,0.000000}%
\pgfsetstrokecolor{currentstroke}%
\pgfsetdash{}{0pt}%
\pgfsys@defobject{currentmarker}{\pgfqpoint{0.000000in}{-0.027778in}}{\pgfqpoint{0.000000in}{0.000000in}}{%
\pgfpathmoveto{\pgfqpoint{0.000000in}{0.000000in}}%
\pgfpathlineto{\pgfqpoint{0.000000in}{-0.027778in}}%
\pgfusepath{stroke,fill}%
}%
\begin{pgfscope}%
\pgfsys@transformshift{2.428646in}{5.147436in}%
\pgfsys@useobject{currentmarker}{}%
\end{pgfscope}%
\end{pgfscope}%
\begin{pgfscope}%
\pgfpathrectangle{\pgfqpoint{0.640323in}{5.147436in}}{\pgfqpoint{9.687500in}{3.850000in}}%
\pgfusepath{clip}%
\pgfsetrectcap%
\pgfsetroundjoin%
\pgfsetlinewidth{0.803000pt}%
\definecolor{currentstroke}{rgb}{0.600000,0.600000,0.600000}%
\pgfsetstrokecolor{currentstroke}%
\pgfsetstrokeopacity{0.200000}%
\pgfsetdash{}{0pt}%
\pgfpathmoveto{\pgfqpoint{3.327301in}{5.147436in}}%
\pgfpathlineto{\pgfqpoint{3.327301in}{8.997436in}}%
\pgfusepath{stroke}%
\end{pgfscope}%
\begin{pgfscope}%
\pgfsetbuttcap%
\pgfsetroundjoin%
\definecolor{currentfill}{rgb}{0.000000,0.000000,0.000000}%
\pgfsetfillcolor{currentfill}%
\pgfsetlinewidth{0.602250pt}%
\definecolor{currentstroke}{rgb}{0.000000,0.000000,0.000000}%
\pgfsetstrokecolor{currentstroke}%
\pgfsetdash{}{0pt}%
\pgfsys@defobject{currentmarker}{\pgfqpoint{0.000000in}{-0.027778in}}{\pgfqpoint{0.000000in}{0.000000in}}{%
\pgfpathmoveto{\pgfqpoint{0.000000in}{0.000000in}}%
\pgfpathlineto{\pgfqpoint{0.000000in}{-0.027778in}}%
\pgfusepath{stroke,fill}%
}%
\begin{pgfscope}%
\pgfsys@transformshift{3.327301in}{5.147436in}%
\pgfsys@useobject{currentmarker}{}%
\end{pgfscope}%
\end{pgfscope}%
\begin{pgfscope}%
\pgfpathrectangle{\pgfqpoint{0.640323in}{5.147436in}}{\pgfqpoint{9.687500in}{3.850000in}}%
\pgfusepath{clip}%
\pgfsetrectcap%
\pgfsetroundjoin%
\pgfsetlinewidth{0.803000pt}%
\definecolor{currentstroke}{rgb}{0.600000,0.600000,0.600000}%
\pgfsetstrokecolor{currentstroke}%
\pgfsetstrokeopacity{0.200000}%
\pgfsetdash{}{0pt}%
\pgfpathmoveto{\pgfqpoint{3.776628in}{5.147436in}}%
\pgfpathlineto{\pgfqpoint{3.776628in}{8.997436in}}%
\pgfusepath{stroke}%
\end{pgfscope}%
\begin{pgfscope}%
\pgfsetbuttcap%
\pgfsetroundjoin%
\definecolor{currentfill}{rgb}{0.000000,0.000000,0.000000}%
\pgfsetfillcolor{currentfill}%
\pgfsetlinewidth{0.602250pt}%
\definecolor{currentstroke}{rgb}{0.000000,0.000000,0.000000}%
\pgfsetstrokecolor{currentstroke}%
\pgfsetdash{}{0pt}%
\pgfsys@defobject{currentmarker}{\pgfqpoint{0.000000in}{-0.027778in}}{\pgfqpoint{0.000000in}{0.000000in}}{%
\pgfpathmoveto{\pgfqpoint{0.000000in}{0.000000in}}%
\pgfpathlineto{\pgfqpoint{0.000000in}{-0.027778in}}%
\pgfusepath{stroke,fill}%
}%
\begin{pgfscope}%
\pgfsys@transformshift{3.776628in}{5.147436in}%
\pgfsys@useobject{currentmarker}{}%
\end{pgfscope}%
\end{pgfscope}%
\begin{pgfscope}%
\pgfpathrectangle{\pgfqpoint{0.640323in}{5.147436in}}{\pgfqpoint{9.687500in}{3.850000in}}%
\pgfusepath{clip}%
\pgfsetrectcap%
\pgfsetroundjoin%
\pgfsetlinewidth{0.803000pt}%
\definecolor{currentstroke}{rgb}{0.600000,0.600000,0.600000}%
\pgfsetstrokecolor{currentstroke}%
\pgfsetstrokeopacity{0.200000}%
\pgfsetdash{}{0pt}%
\pgfpathmoveto{\pgfqpoint{4.225956in}{5.147436in}}%
\pgfpathlineto{\pgfqpoint{4.225956in}{8.997436in}}%
\pgfusepath{stroke}%
\end{pgfscope}%
\begin{pgfscope}%
\pgfsetbuttcap%
\pgfsetroundjoin%
\definecolor{currentfill}{rgb}{0.000000,0.000000,0.000000}%
\pgfsetfillcolor{currentfill}%
\pgfsetlinewidth{0.602250pt}%
\definecolor{currentstroke}{rgb}{0.000000,0.000000,0.000000}%
\pgfsetstrokecolor{currentstroke}%
\pgfsetdash{}{0pt}%
\pgfsys@defobject{currentmarker}{\pgfqpoint{0.000000in}{-0.027778in}}{\pgfqpoint{0.000000in}{0.000000in}}{%
\pgfpathmoveto{\pgfqpoint{0.000000in}{0.000000in}}%
\pgfpathlineto{\pgfqpoint{0.000000in}{-0.027778in}}%
\pgfusepath{stroke,fill}%
}%
\begin{pgfscope}%
\pgfsys@transformshift{4.225956in}{5.147436in}%
\pgfsys@useobject{currentmarker}{}%
\end{pgfscope}%
\end{pgfscope}%
\begin{pgfscope}%
\pgfpathrectangle{\pgfqpoint{0.640323in}{5.147436in}}{\pgfqpoint{9.687500in}{3.850000in}}%
\pgfusepath{clip}%
\pgfsetrectcap%
\pgfsetroundjoin%
\pgfsetlinewidth{0.803000pt}%
\definecolor{currentstroke}{rgb}{0.600000,0.600000,0.600000}%
\pgfsetstrokecolor{currentstroke}%
\pgfsetstrokeopacity{0.200000}%
\pgfsetdash{}{0pt}%
\pgfpathmoveto{\pgfqpoint{5.124611in}{5.147436in}}%
\pgfpathlineto{\pgfqpoint{5.124611in}{8.997436in}}%
\pgfusepath{stroke}%
\end{pgfscope}%
\begin{pgfscope}%
\pgfsetbuttcap%
\pgfsetroundjoin%
\definecolor{currentfill}{rgb}{0.000000,0.000000,0.000000}%
\pgfsetfillcolor{currentfill}%
\pgfsetlinewidth{0.602250pt}%
\definecolor{currentstroke}{rgb}{0.000000,0.000000,0.000000}%
\pgfsetstrokecolor{currentstroke}%
\pgfsetdash{}{0pt}%
\pgfsys@defobject{currentmarker}{\pgfqpoint{0.000000in}{-0.027778in}}{\pgfqpoint{0.000000in}{0.000000in}}{%
\pgfpathmoveto{\pgfqpoint{0.000000in}{0.000000in}}%
\pgfpathlineto{\pgfqpoint{0.000000in}{-0.027778in}}%
\pgfusepath{stroke,fill}%
}%
\begin{pgfscope}%
\pgfsys@transformshift{5.124611in}{5.147436in}%
\pgfsys@useobject{currentmarker}{}%
\end{pgfscope}%
\end{pgfscope}%
\begin{pgfscope}%
\pgfpathrectangle{\pgfqpoint{0.640323in}{5.147436in}}{\pgfqpoint{9.687500in}{3.850000in}}%
\pgfusepath{clip}%
\pgfsetrectcap%
\pgfsetroundjoin%
\pgfsetlinewidth{0.803000pt}%
\definecolor{currentstroke}{rgb}{0.600000,0.600000,0.600000}%
\pgfsetstrokecolor{currentstroke}%
\pgfsetstrokeopacity{0.200000}%
\pgfsetdash{}{0pt}%
\pgfpathmoveto{\pgfqpoint{5.573938in}{5.147436in}}%
\pgfpathlineto{\pgfqpoint{5.573938in}{8.997436in}}%
\pgfusepath{stroke}%
\end{pgfscope}%
\begin{pgfscope}%
\pgfsetbuttcap%
\pgfsetroundjoin%
\definecolor{currentfill}{rgb}{0.000000,0.000000,0.000000}%
\pgfsetfillcolor{currentfill}%
\pgfsetlinewidth{0.602250pt}%
\definecolor{currentstroke}{rgb}{0.000000,0.000000,0.000000}%
\pgfsetstrokecolor{currentstroke}%
\pgfsetdash{}{0pt}%
\pgfsys@defobject{currentmarker}{\pgfqpoint{0.000000in}{-0.027778in}}{\pgfqpoint{0.000000in}{0.000000in}}{%
\pgfpathmoveto{\pgfqpoint{0.000000in}{0.000000in}}%
\pgfpathlineto{\pgfqpoint{0.000000in}{-0.027778in}}%
\pgfusepath{stroke,fill}%
}%
\begin{pgfscope}%
\pgfsys@transformshift{5.573938in}{5.147436in}%
\pgfsys@useobject{currentmarker}{}%
\end{pgfscope}%
\end{pgfscope}%
\begin{pgfscope}%
\pgfpathrectangle{\pgfqpoint{0.640323in}{5.147436in}}{\pgfqpoint{9.687500in}{3.850000in}}%
\pgfusepath{clip}%
\pgfsetrectcap%
\pgfsetroundjoin%
\pgfsetlinewidth{0.803000pt}%
\definecolor{currentstroke}{rgb}{0.600000,0.600000,0.600000}%
\pgfsetstrokecolor{currentstroke}%
\pgfsetstrokeopacity{0.200000}%
\pgfsetdash{}{0pt}%
\pgfpathmoveto{\pgfqpoint{6.023265in}{5.147436in}}%
\pgfpathlineto{\pgfqpoint{6.023265in}{8.997436in}}%
\pgfusepath{stroke}%
\end{pgfscope}%
\begin{pgfscope}%
\pgfsetbuttcap%
\pgfsetroundjoin%
\definecolor{currentfill}{rgb}{0.000000,0.000000,0.000000}%
\pgfsetfillcolor{currentfill}%
\pgfsetlinewidth{0.602250pt}%
\definecolor{currentstroke}{rgb}{0.000000,0.000000,0.000000}%
\pgfsetstrokecolor{currentstroke}%
\pgfsetdash{}{0pt}%
\pgfsys@defobject{currentmarker}{\pgfqpoint{0.000000in}{-0.027778in}}{\pgfqpoint{0.000000in}{0.000000in}}{%
\pgfpathmoveto{\pgfqpoint{0.000000in}{0.000000in}}%
\pgfpathlineto{\pgfqpoint{0.000000in}{-0.027778in}}%
\pgfusepath{stroke,fill}%
}%
\begin{pgfscope}%
\pgfsys@transformshift{6.023265in}{5.147436in}%
\pgfsys@useobject{currentmarker}{}%
\end{pgfscope}%
\end{pgfscope}%
\begin{pgfscope}%
\pgfpathrectangle{\pgfqpoint{0.640323in}{5.147436in}}{\pgfqpoint{9.687500in}{3.850000in}}%
\pgfusepath{clip}%
\pgfsetrectcap%
\pgfsetroundjoin%
\pgfsetlinewidth{0.803000pt}%
\definecolor{currentstroke}{rgb}{0.600000,0.600000,0.600000}%
\pgfsetstrokecolor{currentstroke}%
\pgfsetstrokeopacity{0.200000}%
\pgfsetdash{}{0pt}%
\pgfpathmoveto{\pgfqpoint{6.921920in}{5.147436in}}%
\pgfpathlineto{\pgfqpoint{6.921920in}{8.997436in}}%
\pgfusepath{stroke}%
\end{pgfscope}%
\begin{pgfscope}%
\pgfsetbuttcap%
\pgfsetroundjoin%
\definecolor{currentfill}{rgb}{0.000000,0.000000,0.000000}%
\pgfsetfillcolor{currentfill}%
\pgfsetlinewidth{0.602250pt}%
\definecolor{currentstroke}{rgb}{0.000000,0.000000,0.000000}%
\pgfsetstrokecolor{currentstroke}%
\pgfsetdash{}{0pt}%
\pgfsys@defobject{currentmarker}{\pgfqpoint{0.000000in}{-0.027778in}}{\pgfqpoint{0.000000in}{0.000000in}}{%
\pgfpathmoveto{\pgfqpoint{0.000000in}{0.000000in}}%
\pgfpathlineto{\pgfqpoint{0.000000in}{-0.027778in}}%
\pgfusepath{stroke,fill}%
}%
\begin{pgfscope}%
\pgfsys@transformshift{6.921920in}{5.147436in}%
\pgfsys@useobject{currentmarker}{}%
\end{pgfscope}%
\end{pgfscope}%
\begin{pgfscope}%
\pgfpathrectangle{\pgfqpoint{0.640323in}{5.147436in}}{\pgfqpoint{9.687500in}{3.850000in}}%
\pgfusepath{clip}%
\pgfsetrectcap%
\pgfsetroundjoin%
\pgfsetlinewidth{0.803000pt}%
\definecolor{currentstroke}{rgb}{0.600000,0.600000,0.600000}%
\pgfsetstrokecolor{currentstroke}%
\pgfsetstrokeopacity{0.200000}%
\pgfsetdash{}{0pt}%
\pgfpathmoveto{\pgfqpoint{7.371248in}{5.147436in}}%
\pgfpathlineto{\pgfqpoint{7.371248in}{8.997436in}}%
\pgfusepath{stroke}%
\end{pgfscope}%
\begin{pgfscope}%
\pgfsetbuttcap%
\pgfsetroundjoin%
\definecolor{currentfill}{rgb}{0.000000,0.000000,0.000000}%
\pgfsetfillcolor{currentfill}%
\pgfsetlinewidth{0.602250pt}%
\definecolor{currentstroke}{rgb}{0.000000,0.000000,0.000000}%
\pgfsetstrokecolor{currentstroke}%
\pgfsetdash{}{0pt}%
\pgfsys@defobject{currentmarker}{\pgfqpoint{0.000000in}{-0.027778in}}{\pgfqpoint{0.000000in}{0.000000in}}{%
\pgfpathmoveto{\pgfqpoint{0.000000in}{0.000000in}}%
\pgfpathlineto{\pgfqpoint{0.000000in}{-0.027778in}}%
\pgfusepath{stroke,fill}%
}%
\begin{pgfscope}%
\pgfsys@transformshift{7.371248in}{5.147436in}%
\pgfsys@useobject{currentmarker}{}%
\end{pgfscope}%
\end{pgfscope}%
\begin{pgfscope}%
\pgfpathrectangle{\pgfqpoint{0.640323in}{5.147436in}}{\pgfqpoint{9.687500in}{3.850000in}}%
\pgfusepath{clip}%
\pgfsetrectcap%
\pgfsetroundjoin%
\pgfsetlinewidth{0.803000pt}%
\definecolor{currentstroke}{rgb}{0.600000,0.600000,0.600000}%
\pgfsetstrokecolor{currentstroke}%
\pgfsetstrokeopacity{0.200000}%
\pgfsetdash{}{0pt}%
\pgfpathmoveto{\pgfqpoint{7.820575in}{5.147436in}}%
\pgfpathlineto{\pgfqpoint{7.820575in}{8.997436in}}%
\pgfusepath{stroke}%
\end{pgfscope}%
\begin{pgfscope}%
\pgfsetbuttcap%
\pgfsetroundjoin%
\definecolor{currentfill}{rgb}{0.000000,0.000000,0.000000}%
\pgfsetfillcolor{currentfill}%
\pgfsetlinewidth{0.602250pt}%
\definecolor{currentstroke}{rgb}{0.000000,0.000000,0.000000}%
\pgfsetstrokecolor{currentstroke}%
\pgfsetdash{}{0pt}%
\pgfsys@defobject{currentmarker}{\pgfqpoint{0.000000in}{-0.027778in}}{\pgfqpoint{0.000000in}{0.000000in}}{%
\pgfpathmoveto{\pgfqpoint{0.000000in}{0.000000in}}%
\pgfpathlineto{\pgfqpoint{0.000000in}{-0.027778in}}%
\pgfusepath{stroke,fill}%
}%
\begin{pgfscope}%
\pgfsys@transformshift{7.820575in}{5.147436in}%
\pgfsys@useobject{currentmarker}{}%
\end{pgfscope}%
\end{pgfscope}%
\begin{pgfscope}%
\pgfpathrectangle{\pgfqpoint{0.640323in}{5.147436in}}{\pgfqpoint{9.687500in}{3.850000in}}%
\pgfusepath{clip}%
\pgfsetrectcap%
\pgfsetroundjoin%
\pgfsetlinewidth{0.803000pt}%
\definecolor{currentstroke}{rgb}{0.600000,0.600000,0.600000}%
\pgfsetstrokecolor{currentstroke}%
\pgfsetstrokeopacity{0.200000}%
\pgfsetdash{}{0pt}%
\pgfpathmoveto{\pgfqpoint{8.719230in}{5.147436in}}%
\pgfpathlineto{\pgfqpoint{8.719230in}{8.997436in}}%
\pgfusepath{stroke}%
\end{pgfscope}%
\begin{pgfscope}%
\pgfsetbuttcap%
\pgfsetroundjoin%
\definecolor{currentfill}{rgb}{0.000000,0.000000,0.000000}%
\pgfsetfillcolor{currentfill}%
\pgfsetlinewidth{0.602250pt}%
\definecolor{currentstroke}{rgb}{0.000000,0.000000,0.000000}%
\pgfsetstrokecolor{currentstroke}%
\pgfsetdash{}{0pt}%
\pgfsys@defobject{currentmarker}{\pgfqpoint{0.000000in}{-0.027778in}}{\pgfqpoint{0.000000in}{0.000000in}}{%
\pgfpathmoveto{\pgfqpoint{0.000000in}{0.000000in}}%
\pgfpathlineto{\pgfqpoint{0.000000in}{-0.027778in}}%
\pgfusepath{stroke,fill}%
}%
\begin{pgfscope}%
\pgfsys@transformshift{8.719230in}{5.147436in}%
\pgfsys@useobject{currentmarker}{}%
\end{pgfscope}%
\end{pgfscope}%
\begin{pgfscope}%
\pgfpathrectangle{\pgfqpoint{0.640323in}{5.147436in}}{\pgfqpoint{9.687500in}{3.850000in}}%
\pgfusepath{clip}%
\pgfsetrectcap%
\pgfsetroundjoin%
\pgfsetlinewidth{0.803000pt}%
\definecolor{currentstroke}{rgb}{0.600000,0.600000,0.600000}%
\pgfsetstrokecolor{currentstroke}%
\pgfsetstrokeopacity{0.200000}%
\pgfsetdash{}{0pt}%
\pgfpathmoveto{\pgfqpoint{9.168558in}{5.147436in}}%
\pgfpathlineto{\pgfqpoint{9.168558in}{8.997436in}}%
\pgfusepath{stroke}%
\end{pgfscope}%
\begin{pgfscope}%
\pgfsetbuttcap%
\pgfsetroundjoin%
\definecolor{currentfill}{rgb}{0.000000,0.000000,0.000000}%
\pgfsetfillcolor{currentfill}%
\pgfsetlinewidth{0.602250pt}%
\definecolor{currentstroke}{rgb}{0.000000,0.000000,0.000000}%
\pgfsetstrokecolor{currentstroke}%
\pgfsetdash{}{0pt}%
\pgfsys@defobject{currentmarker}{\pgfqpoint{0.000000in}{-0.027778in}}{\pgfqpoint{0.000000in}{0.000000in}}{%
\pgfpathmoveto{\pgfqpoint{0.000000in}{0.000000in}}%
\pgfpathlineto{\pgfqpoint{0.000000in}{-0.027778in}}%
\pgfusepath{stroke,fill}%
}%
\begin{pgfscope}%
\pgfsys@transformshift{9.168558in}{5.147436in}%
\pgfsys@useobject{currentmarker}{}%
\end{pgfscope}%
\end{pgfscope}%
\begin{pgfscope}%
\pgfpathrectangle{\pgfqpoint{0.640323in}{5.147436in}}{\pgfqpoint{9.687500in}{3.850000in}}%
\pgfusepath{clip}%
\pgfsetrectcap%
\pgfsetroundjoin%
\pgfsetlinewidth{0.803000pt}%
\definecolor{currentstroke}{rgb}{0.600000,0.600000,0.600000}%
\pgfsetstrokecolor{currentstroke}%
\pgfsetstrokeopacity{0.200000}%
\pgfsetdash{}{0pt}%
\pgfpathmoveto{\pgfqpoint{9.617885in}{5.147436in}}%
\pgfpathlineto{\pgfqpoint{9.617885in}{8.997436in}}%
\pgfusepath{stroke}%
\end{pgfscope}%
\begin{pgfscope}%
\pgfsetbuttcap%
\pgfsetroundjoin%
\definecolor{currentfill}{rgb}{0.000000,0.000000,0.000000}%
\pgfsetfillcolor{currentfill}%
\pgfsetlinewidth{0.602250pt}%
\definecolor{currentstroke}{rgb}{0.000000,0.000000,0.000000}%
\pgfsetstrokecolor{currentstroke}%
\pgfsetdash{}{0pt}%
\pgfsys@defobject{currentmarker}{\pgfqpoint{0.000000in}{-0.027778in}}{\pgfqpoint{0.000000in}{0.000000in}}{%
\pgfpathmoveto{\pgfqpoint{0.000000in}{0.000000in}}%
\pgfpathlineto{\pgfqpoint{0.000000in}{-0.027778in}}%
\pgfusepath{stroke,fill}%
}%
\begin{pgfscope}%
\pgfsys@transformshift{9.617885in}{5.147436in}%
\pgfsys@useobject{currentmarker}{}%
\end{pgfscope}%
\end{pgfscope}%
\begin{pgfscope}%
\definecolor{textcolor}{rgb}{0.000000,0.000000,0.000000}%
\pgfsetstrokecolor{textcolor}%
\pgfsetfillcolor{textcolor}%
\pgftext[x=5.484072in,y=4.860245in,,top]{\color{textcolor}\sffamily\fontsize{10.000000}{12.000000}\selectfont turnover probability \(\displaystyle p_1\,(S\rightarrow I\,)\)}%
\end{pgfscope}%
\begin{pgfscope}%
\pgfpathrectangle{\pgfqpoint{0.640323in}{5.147436in}}{\pgfqpoint{9.687500in}{3.850000in}}%
\pgfusepath{clip}%
\pgfsetrectcap%
\pgfsetroundjoin%
\pgfsetlinewidth{0.803000pt}%
\definecolor{currentstroke}{rgb}{0.690196,0.690196,0.690196}%
\pgfsetstrokecolor{currentstroke}%
\pgfsetdash{}{0pt}%
\pgfpathmoveto{\pgfqpoint{0.640323in}{5.271629in}}%
\pgfpathlineto{\pgfqpoint{10.327822in}{5.271629in}}%
\pgfusepath{stroke}%
\end{pgfscope}%
\begin{pgfscope}%
\pgfsetbuttcap%
\pgfsetroundjoin%
\definecolor{currentfill}{rgb}{0.000000,0.000000,0.000000}%
\pgfsetfillcolor{currentfill}%
\pgfsetlinewidth{0.803000pt}%
\definecolor{currentstroke}{rgb}{0.000000,0.000000,0.000000}%
\pgfsetstrokecolor{currentstroke}%
\pgfsetdash{}{0pt}%
\pgfsys@defobject{currentmarker}{\pgfqpoint{-0.048611in}{0.000000in}}{\pgfqpoint{-0.000000in}{0.000000in}}{%
\pgfpathmoveto{\pgfqpoint{-0.000000in}{0.000000in}}%
\pgfpathlineto{\pgfqpoint{-0.048611in}{0.000000in}}%
\pgfusepath{stroke,fill}%
}%
\begin{pgfscope}%
\pgfsys@transformshift{0.640323in}{5.271629in}%
\pgfsys@useobject{currentmarker}{}%
\end{pgfscope}%
\end{pgfscope}%
\begin{pgfscope}%
\definecolor{textcolor}{rgb}{0.000000,0.000000,0.000000}%
\pgfsetstrokecolor{textcolor}%
\pgfsetfillcolor{textcolor}%
\pgftext[x=0.322221in, y=5.218868in, left, base]{\color{textcolor}\sffamily\fontsize{10.000000}{12.000000}\selectfont 0.0}%
\end{pgfscope}%
\begin{pgfscope}%
\pgfpathrectangle{\pgfqpoint{0.640323in}{5.147436in}}{\pgfqpoint{9.687500in}{3.850000in}}%
\pgfusepath{clip}%
\pgfsetrectcap%
\pgfsetroundjoin%
\pgfsetlinewidth{0.803000pt}%
\definecolor{currentstroke}{rgb}{0.690196,0.690196,0.690196}%
\pgfsetstrokecolor{currentstroke}%
\pgfsetdash{}{0pt}%
\pgfpathmoveto{\pgfqpoint{0.640323in}{5.892597in}}%
\pgfpathlineto{\pgfqpoint{10.327822in}{5.892597in}}%
\pgfusepath{stroke}%
\end{pgfscope}%
\begin{pgfscope}%
\pgfsetbuttcap%
\pgfsetroundjoin%
\definecolor{currentfill}{rgb}{0.000000,0.000000,0.000000}%
\pgfsetfillcolor{currentfill}%
\pgfsetlinewidth{0.803000pt}%
\definecolor{currentstroke}{rgb}{0.000000,0.000000,0.000000}%
\pgfsetstrokecolor{currentstroke}%
\pgfsetdash{}{0pt}%
\pgfsys@defobject{currentmarker}{\pgfqpoint{-0.048611in}{0.000000in}}{\pgfqpoint{-0.000000in}{0.000000in}}{%
\pgfpathmoveto{\pgfqpoint{-0.000000in}{0.000000in}}%
\pgfpathlineto{\pgfqpoint{-0.048611in}{0.000000in}}%
\pgfusepath{stroke,fill}%
}%
\begin{pgfscope}%
\pgfsys@transformshift{0.640323in}{5.892597in}%
\pgfsys@useobject{currentmarker}{}%
\end{pgfscope}%
\end{pgfscope}%
\begin{pgfscope}%
\definecolor{textcolor}{rgb}{0.000000,0.000000,0.000000}%
\pgfsetstrokecolor{textcolor}%
\pgfsetfillcolor{textcolor}%
\pgftext[x=0.322221in, y=5.839836in, left, base]{\color{textcolor}\sffamily\fontsize{10.000000}{12.000000}\selectfont 0.1}%
\end{pgfscope}%
\begin{pgfscope}%
\pgfpathrectangle{\pgfqpoint{0.640323in}{5.147436in}}{\pgfqpoint{9.687500in}{3.850000in}}%
\pgfusepath{clip}%
\pgfsetrectcap%
\pgfsetroundjoin%
\pgfsetlinewidth{0.803000pt}%
\definecolor{currentstroke}{rgb}{0.690196,0.690196,0.690196}%
\pgfsetstrokecolor{currentstroke}%
\pgfsetdash{}{0pt}%
\pgfpathmoveto{\pgfqpoint{0.640323in}{6.513565in}}%
\pgfpathlineto{\pgfqpoint{10.327822in}{6.513565in}}%
\pgfusepath{stroke}%
\end{pgfscope}%
\begin{pgfscope}%
\pgfsetbuttcap%
\pgfsetroundjoin%
\definecolor{currentfill}{rgb}{0.000000,0.000000,0.000000}%
\pgfsetfillcolor{currentfill}%
\pgfsetlinewidth{0.803000pt}%
\definecolor{currentstroke}{rgb}{0.000000,0.000000,0.000000}%
\pgfsetstrokecolor{currentstroke}%
\pgfsetdash{}{0pt}%
\pgfsys@defobject{currentmarker}{\pgfqpoint{-0.048611in}{0.000000in}}{\pgfqpoint{-0.000000in}{0.000000in}}{%
\pgfpathmoveto{\pgfqpoint{-0.000000in}{0.000000in}}%
\pgfpathlineto{\pgfqpoint{-0.048611in}{0.000000in}}%
\pgfusepath{stroke,fill}%
}%
\begin{pgfscope}%
\pgfsys@transformshift{0.640323in}{6.513565in}%
\pgfsys@useobject{currentmarker}{}%
\end{pgfscope}%
\end{pgfscope}%
\begin{pgfscope}%
\definecolor{textcolor}{rgb}{0.000000,0.000000,0.000000}%
\pgfsetstrokecolor{textcolor}%
\pgfsetfillcolor{textcolor}%
\pgftext[x=0.322221in, y=6.460803in, left, base]{\color{textcolor}\sffamily\fontsize{10.000000}{12.000000}\selectfont 0.2}%
\end{pgfscope}%
\begin{pgfscope}%
\pgfpathrectangle{\pgfqpoint{0.640323in}{5.147436in}}{\pgfqpoint{9.687500in}{3.850000in}}%
\pgfusepath{clip}%
\pgfsetrectcap%
\pgfsetroundjoin%
\pgfsetlinewidth{0.803000pt}%
\definecolor{currentstroke}{rgb}{0.690196,0.690196,0.690196}%
\pgfsetstrokecolor{currentstroke}%
\pgfsetdash{}{0pt}%
\pgfpathmoveto{\pgfqpoint{0.640323in}{7.134533in}}%
\pgfpathlineto{\pgfqpoint{10.327822in}{7.134533in}}%
\pgfusepath{stroke}%
\end{pgfscope}%
\begin{pgfscope}%
\pgfsetbuttcap%
\pgfsetroundjoin%
\definecolor{currentfill}{rgb}{0.000000,0.000000,0.000000}%
\pgfsetfillcolor{currentfill}%
\pgfsetlinewidth{0.803000pt}%
\definecolor{currentstroke}{rgb}{0.000000,0.000000,0.000000}%
\pgfsetstrokecolor{currentstroke}%
\pgfsetdash{}{0pt}%
\pgfsys@defobject{currentmarker}{\pgfqpoint{-0.048611in}{0.000000in}}{\pgfqpoint{-0.000000in}{0.000000in}}{%
\pgfpathmoveto{\pgfqpoint{-0.000000in}{0.000000in}}%
\pgfpathlineto{\pgfqpoint{-0.048611in}{0.000000in}}%
\pgfusepath{stroke,fill}%
}%
\begin{pgfscope}%
\pgfsys@transformshift{0.640323in}{7.134533in}%
\pgfsys@useobject{currentmarker}{}%
\end{pgfscope}%
\end{pgfscope}%
\begin{pgfscope}%
\definecolor{textcolor}{rgb}{0.000000,0.000000,0.000000}%
\pgfsetstrokecolor{textcolor}%
\pgfsetfillcolor{textcolor}%
\pgftext[x=0.322221in, y=7.081771in, left, base]{\color{textcolor}\sffamily\fontsize{10.000000}{12.000000}\selectfont 0.3}%
\end{pgfscope}%
\begin{pgfscope}%
\pgfpathrectangle{\pgfqpoint{0.640323in}{5.147436in}}{\pgfqpoint{9.687500in}{3.850000in}}%
\pgfusepath{clip}%
\pgfsetrectcap%
\pgfsetroundjoin%
\pgfsetlinewidth{0.803000pt}%
\definecolor{currentstroke}{rgb}{0.690196,0.690196,0.690196}%
\pgfsetstrokecolor{currentstroke}%
\pgfsetdash{}{0pt}%
\pgfpathmoveto{\pgfqpoint{0.640323in}{7.755500in}}%
\pgfpathlineto{\pgfqpoint{10.327822in}{7.755500in}}%
\pgfusepath{stroke}%
\end{pgfscope}%
\begin{pgfscope}%
\pgfsetbuttcap%
\pgfsetroundjoin%
\definecolor{currentfill}{rgb}{0.000000,0.000000,0.000000}%
\pgfsetfillcolor{currentfill}%
\pgfsetlinewidth{0.803000pt}%
\definecolor{currentstroke}{rgb}{0.000000,0.000000,0.000000}%
\pgfsetstrokecolor{currentstroke}%
\pgfsetdash{}{0pt}%
\pgfsys@defobject{currentmarker}{\pgfqpoint{-0.048611in}{0.000000in}}{\pgfqpoint{-0.000000in}{0.000000in}}{%
\pgfpathmoveto{\pgfqpoint{-0.000000in}{0.000000in}}%
\pgfpathlineto{\pgfqpoint{-0.048611in}{0.000000in}}%
\pgfusepath{stroke,fill}%
}%
\begin{pgfscope}%
\pgfsys@transformshift{0.640323in}{7.755500in}%
\pgfsys@useobject{currentmarker}{}%
\end{pgfscope}%
\end{pgfscope}%
\begin{pgfscope}%
\definecolor{textcolor}{rgb}{0.000000,0.000000,0.000000}%
\pgfsetstrokecolor{textcolor}%
\pgfsetfillcolor{textcolor}%
\pgftext[x=0.322221in, y=7.702739in, left, base]{\color{textcolor}\sffamily\fontsize{10.000000}{12.000000}\selectfont 0.4}%
\end{pgfscope}%
\begin{pgfscope}%
\pgfpathrectangle{\pgfqpoint{0.640323in}{5.147436in}}{\pgfqpoint{9.687500in}{3.850000in}}%
\pgfusepath{clip}%
\pgfsetrectcap%
\pgfsetroundjoin%
\pgfsetlinewidth{0.803000pt}%
\definecolor{currentstroke}{rgb}{0.690196,0.690196,0.690196}%
\pgfsetstrokecolor{currentstroke}%
\pgfsetdash{}{0pt}%
\pgfpathmoveto{\pgfqpoint{0.640323in}{8.376468in}}%
\pgfpathlineto{\pgfqpoint{10.327822in}{8.376468in}}%
\pgfusepath{stroke}%
\end{pgfscope}%
\begin{pgfscope}%
\pgfsetbuttcap%
\pgfsetroundjoin%
\definecolor{currentfill}{rgb}{0.000000,0.000000,0.000000}%
\pgfsetfillcolor{currentfill}%
\pgfsetlinewidth{0.803000pt}%
\definecolor{currentstroke}{rgb}{0.000000,0.000000,0.000000}%
\pgfsetstrokecolor{currentstroke}%
\pgfsetdash{}{0pt}%
\pgfsys@defobject{currentmarker}{\pgfqpoint{-0.048611in}{0.000000in}}{\pgfqpoint{-0.000000in}{0.000000in}}{%
\pgfpathmoveto{\pgfqpoint{-0.000000in}{0.000000in}}%
\pgfpathlineto{\pgfqpoint{-0.048611in}{0.000000in}}%
\pgfusepath{stroke,fill}%
}%
\begin{pgfscope}%
\pgfsys@transformshift{0.640323in}{8.376468in}%
\pgfsys@useobject{currentmarker}{}%
\end{pgfscope}%
\end{pgfscope}%
\begin{pgfscope}%
\definecolor{textcolor}{rgb}{0.000000,0.000000,0.000000}%
\pgfsetstrokecolor{textcolor}%
\pgfsetfillcolor{textcolor}%
\pgftext[x=0.322221in, y=8.323707in, left, base]{\color{textcolor}\sffamily\fontsize{10.000000}{12.000000}\selectfont 0.5}%
\end{pgfscope}%
\begin{pgfscope}%
\pgfpathrectangle{\pgfqpoint{0.640323in}{5.147436in}}{\pgfqpoint{9.687500in}{3.850000in}}%
\pgfusepath{clip}%
\pgfsetrectcap%
\pgfsetroundjoin%
\pgfsetlinewidth{0.803000pt}%
\definecolor{currentstroke}{rgb}{0.690196,0.690196,0.690196}%
\pgfsetstrokecolor{currentstroke}%
\pgfsetdash{}{0pt}%
\pgfpathmoveto{\pgfqpoint{0.640323in}{8.997436in}}%
\pgfpathlineto{\pgfqpoint{10.327822in}{8.997436in}}%
\pgfusepath{stroke}%
\end{pgfscope}%
\begin{pgfscope}%
\pgfsetbuttcap%
\pgfsetroundjoin%
\definecolor{currentfill}{rgb}{0.000000,0.000000,0.000000}%
\pgfsetfillcolor{currentfill}%
\pgfsetlinewidth{0.803000pt}%
\definecolor{currentstroke}{rgb}{0.000000,0.000000,0.000000}%
\pgfsetstrokecolor{currentstroke}%
\pgfsetdash{}{0pt}%
\pgfsys@defobject{currentmarker}{\pgfqpoint{-0.048611in}{0.000000in}}{\pgfqpoint{-0.000000in}{0.000000in}}{%
\pgfpathmoveto{\pgfqpoint{-0.000000in}{0.000000in}}%
\pgfpathlineto{\pgfqpoint{-0.048611in}{0.000000in}}%
\pgfusepath{stroke,fill}%
}%
\begin{pgfscope}%
\pgfsys@transformshift{0.640323in}{8.997436in}%
\pgfsys@useobject{currentmarker}{}%
\end{pgfscope}%
\end{pgfscope}%
\begin{pgfscope}%
\definecolor{textcolor}{rgb}{0.000000,0.000000,0.000000}%
\pgfsetstrokecolor{textcolor}%
\pgfsetfillcolor{textcolor}%
\pgftext[x=0.322221in, y=8.944674in, left, base]{\color{textcolor}\sffamily\fontsize{10.000000}{12.000000}\selectfont 0.6}%
\end{pgfscope}%
\begin{pgfscope}%
\pgfpathrectangle{\pgfqpoint{0.640323in}{5.147436in}}{\pgfqpoint{9.687500in}{3.850000in}}%
\pgfusepath{clip}%
\pgfsetrectcap%
\pgfsetroundjoin%
\pgfsetlinewidth{0.803000pt}%
\definecolor{currentstroke}{rgb}{0.600000,0.600000,0.600000}%
\pgfsetstrokecolor{currentstroke}%
\pgfsetstrokeopacity{0.200000}%
\pgfsetdash{}{0pt}%
\pgfpathmoveto{\pgfqpoint{0.640323in}{5.395823in}}%
\pgfpathlineto{\pgfqpoint{10.327822in}{5.395823in}}%
\pgfusepath{stroke}%
\end{pgfscope}%
\begin{pgfscope}%
\pgfsetbuttcap%
\pgfsetroundjoin%
\definecolor{currentfill}{rgb}{0.000000,0.000000,0.000000}%
\pgfsetfillcolor{currentfill}%
\pgfsetlinewidth{0.602250pt}%
\definecolor{currentstroke}{rgb}{0.000000,0.000000,0.000000}%
\pgfsetstrokecolor{currentstroke}%
\pgfsetdash{}{0pt}%
\pgfsys@defobject{currentmarker}{\pgfqpoint{-0.027778in}{0.000000in}}{\pgfqpoint{-0.000000in}{0.000000in}}{%
\pgfpathmoveto{\pgfqpoint{-0.000000in}{0.000000in}}%
\pgfpathlineto{\pgfqpoint{-0.027778in}{0.000000in}}%
\pgfusepath{stroke,fill}%
}%
\begin{pgfscope}%
\pgfsys@transformshift{0.640323in}{5.395823in}%
\pgfsys@useobject{currentmarker}{}%
\end{pgfscope}%
\end{pgfscope}%
\begin{pgfscope}%
\pgfpathrectangle{\pgfqpoint{0.640323in}{5.147436in}}{\pgfqpoint{9.687500in}{3.850000in}}%
\pgfusepath{clip}%
\pgfsetrectcap%
\pgfsetroundjoin%
\pgfsetlinewidth{0.803000pt}%
\definecolor{currentstroke}{rgb}{0.600000,0.600000,0.600000}%
\pgfsetstrokecolor{currentstroke}%
\pgfsetstrokeopacity{0.200000}%
\pgfsetdash{}{0pt}%
\pgfpathmoveto{\pgfqpoint{0.640323in}{5.520016in}}%
\pgfpathlineto{\pgfqpoint{10.327822in}{5.520016in}}%
\pgfusepath{stroke}%
\end{pgfscope}%
\begin{pgfscope}%
\pgfsetbuttcap%
\pgfsetroundjoin%
\definecolor{currentfill}{rgb}{0.000000,0.000000,0.000000}%
\pgfsetfillcolor{currentfill}%
\pgfsetlinewidth{0.602250pt}%
\definecolor{currentstroke}{rgb}{0.000000,0.000000,0.000000}%
\pgfsetstrokecolor{currentstroke}%
\pgfsetdash{}{0pt}%
\pgfsys@defobject{currentmarker}{\pgfqpoint{-0.027778in}{0.000000in}}{\pgfqpoint{-0.000000in}{0.000000in}}{%
\pgfpathmoveto{\pgfqpoint{-0.000000in}{0.000000in}}%
\pgfpathlineto{\pgfqpoint{-0.027778in}{0.000000in}}%
\pgfusepath{stroke,fill}%
}%
\begin{pgfscope}%
\pgfsys@transformshift{0.640323in}{5.520016in}%
\pgfsys@useobject{currentmarker}{}%
\end{pgfscope}%
\end{pgfscope}%
\begin{pgfscope}%
\pgfpathrectangle{\pgfqpoint{0.640323in}{5.147436in}}{\pgfqpoint{9.687500in}{3.850000in}}%
\pgfusepath{clip}%
\pgfsetrectcap%
\pgfsetroundjoin%
\pgfsetlinewidth{0.803000pt}%
\definecolor{currentstroke}{rgb}{0.600000,0.600000,0.600000}%
\pgfsetstrokecolor{currentstroke}%
\pgfsetstrokeopacity{0.200000}%
\pgfsetdash{}{0pt}%
\pgfpathmoveto{\pgfqpoint{0.640323in}{5.644210in}}%
\pgfpathlineto{\pgfqpoint{10.327822in}{5.644210in}}%
\pgfusepath{stroke}%
\end{pgfscope}%
\begin{pgfscope}%
\pgfsetbuttcap%
\pgfsetroundjoin%
\definecolor{currentfill}{rgb}{0.000000,0.000000,0.000000}%
\pgfsetfillcolor{currentfill}%
\pgfsetlinewidth{0.602250pt}%
\definecolor{currentstroke}{rgb}{0.000000,0.000000,0.000000}%
\pgfsetstrokecolor{currentstroke}%
\pgfsetdash{}{0pt}%
\pgfsys@defobject{currentmarker}{\pgfqpoint{-0.027778in}{0.000000in}}{\pgfqpoint{-0.000000in}{0.000000in}}{%
\pgfpathmoveto{\pgfqpoint{-0.000000in}{0.000000in}}%
\pgfpathlineto{\pgfqpoint{-0.027778in}{0.000000in}}%
\pgfusepath{stroke,fill}%
}%
\begin{pgfscope}%
\pgfsys@transformshift{0.640323in}{5.644210in}%
\pgfsys@useobject{currentmarker}{}%
\end{pgfscope}%
\end{pgfscope}%
\begin{pgfscope}%
\pgfpathrectangle{\pgfqpoint{0.640323in}{5.147436in}}{\pgfqpoint{9.687500in}{3.850000in}}%
\pgfusepath{clip}%
\pgfsetrectcap%
\pgfsetroundjoin%
\pgfsetlinewidth{0.803000pt}%
\definecolor{currentstroke}{rgb}{0.600000,0.600000,0.600000}%
\pgfsetstrokecolor{currentstroke}%
\pgfsetstrokeopacity{0.200000}%
\pgfsetdash{}{0pt}%
\pgfpathmoveto{\pgfqpoint{0.640323in}{5.768404in}}%
\pgfpathlineto{\pgfqpoint{10.327822in}{5.768404in}}%
\pgfusepath{stroke}%
\end{pgfscope}%
\begin{pgfscope}%
\pgfsetbuttcap%
\pgfsetroundjoin%
\definecolor{currentfill}{rgb}{0.000000,0.000000,0.000000}%
\pgfsetfillcolor{currentfill}%
\pgfsetlinewidth{0.602250pt}%
\definecolor{currentstroke}{rgb}{0.000000,0.000000,0.000000}%
\pgfsetstrokecolor{currentstroke}%
\pgfsetdash{}{0pt}%
\pgfsys@defobject{currentmarker}{\pgfqpoint{-0.027778in}{0.000000in}}{\pgfqpoint{-0.000000in}{0.000000in}}{%
\pgfpathmoveto{\pgfqpoint{-0.000000in}{0.000000in}}%
\pgfpathlineto{\pgfqpoint{-0.027778in}{0.000000in}}%
\pgfusepath{stroke,fill}%
}%
\begin{pgfscope}%
\pgfsys@transformshift{0.640323in}{5.768404in}%
\pgfsys@useobject{currentmarker}{}%
\end{pgfscope}%
\end{pgfscope}%
\begin{pgfscope}%
\pgfpathrectangle{\pgfqpoint{0.640323in}{5.147436in}}{\pgfqpoint{9.687500in}{3.850000in}}%
\pgfusepath{clip}%
\pgfsetrectcap%
\pgfsetroundjoin%
\pgfsetlinewidth{0.803000pt}%
\definecolor{currentstroke}{rgb}{0.600000,0.600000,0.600000}%
\pgfsetstrokecolor{currentstroke}%
\pgfsetstrokeopacity{0.200000}%
\pgfsetdash{}{0pt}%
\pgfpathmoveto{\pgfqpoint{0.640323in}{6.016791in}}%
\pgfpathlineto{\pgfqpoint{10.327822in}{6.016791in}}%
\pgfusepath{stroke}%
\end{pgfscope}%
\begin{pgfscope}%
\pgfsetbuttcap%
\pgfsetroundjoin%
\definecolor{currentfill}{rgb}{0.000000,0.000000,0.000000}%
\pgfsetfillcolor{currentfill}%
\pgfsetlinewidth{0.602250pt}%
\definecolor{currentstroke}{rgb}{0.000000,0.000000,0.000000}%
\pgfsetstrokecolor{currentstroke}%
\pgfsetdash{}{0pt}%
\pgfsys@defobject{currentmarker}{\pgfqpoint{-0.027778in}{0.000000in}}{\pgfqpoint{-0.000000in}{0.000000in}}{%
\pgfpathmoveto{\pgfqpoint{-0.000000in}{0.000000in}}%
\pgfpathlineto{\pgfqpoint{-0.027778in}{0.000000in}}%
\pgfusepath{stroke,fill}%
}%
\begin{pgfscope}%
\pgfsys@transformshift{0.640323in}{6.016791in}%
\pgfsys@useobject{currentmarker}{}%
\end{pgfscope}%
\end{pgfscope}%
\begin{pgfscope}%
\pgfpathrectangle{\pgfqpoint{0.640323in}{5.147436in}}{\pgfqpoint{9.687500in}{3.850000in}}%
\pgfusepath{clip}%
\pgfsetrectcap%
\pgfsetroundjoin%
\pgfsetlinewidth{0.803000pt}%
\definecolor{currentstroke}{rgb}{0.600000,0.600000,0.600000}%
\pgfsetstrokecolor{currentstroke}%
\pgfsetstrokeopacity{0.200000}%
\pgfsetdash{}{0pt}%
\pgfpathmoveto{\pgfqpoint{0.640323in}{6.140984in}}%
\pgfpathlineto{\pgfqpoint{10.327822in}{6.140984in}}%
\pgfusepath{stroke}%
\end{pgfscope}%
\begin{pgfscope}%
\pgfsetbuttcap%
\pgfsetroundjoin%
\definecolor{currentfill}{rgb}{0.000000,0.000000,0.000000}%
\pgfsetfillcolor{currentfill}%
\pgfsetlinewidth{0.602250pt}%
\definecolor{currentstroke}{rgb}{0.000000,0.000000,0.000000}%
\pgfsetstrokecolor{currentstroke}%
\pgfsetdash{}{0pt}%
\pgfsys@defobject{currentmarker}{\pgfqpoint{-0.027778in}{0.000000in}}{\pgfqpoint{-0.000000in}{0.000000in}}{%
\pgfpathmoveto{\pgfqpoint{-0.000000in}{0.000000in}}%
\pgfpathlineto{\pgfqpoint{-0.027778in}{0.000000in}}%
\pgfusepath{stroke,fill}%
}%
\begin{pgfscope}%
\pgfsys@transformshift{0.640323in}{6.140984in}%
\pgfsys@useobject{currentmarker}{}%
\end{pgfscope}%
\end{pgfscope}%
\begin{pgfscope}%
\pgfpathrectangle{\pgfqpoint{0.640323in}{5.147436in}}{\pgfqpoint{9.687500in}{3.850000in}}%
\pgfusepath{clip}%
\pgfsetrectcap%
\pgfsetroundjoin%
\pgfsetlinewidth{0.803000pt}%
\definecolor{currentstroke}{rgb}{0.600000,0.600000,0.600000}%
\pgfsetstrokecolor{currentstroke}%
\pgfsetstrokeopacity{0.200000}%
\pgfsetdash{}{0pt}%
\pgfpathmoveto{\pgfqpoint{0.640323in}{6.265178in}}%
\pgfpathlineto{\pgfqpoint{10.327822in}{6.265178in}}%
\pgfusepath{stroke}%
\end{pgfscope}%
\begin{pgfscope}%
\pgfsetbuttcap%
\pgfsetroundjoin%
\definecolor{currentfill}{rgb}{0.000000,0.000000,0.000000}%
\pgfsetfillcolor{currentfill}%
\pgfsetlinewidth{0.602250pt}%
\definecolor{currentstroke}{rgb}{0.000000,0.000000,0.000000}%
\pgfsetstrokecolor{currentstroke}%
\pgfsetdash{}{0pt}%
\pgfsys@defobject{currentmarker}{\pgfqpoint{-0.027778in}{0.000000in}}{\pgfqpoint{-0.000000in}{0.000000in}}{%
\pgfpathmoveto{\pgfqpoint{-0.000000in}{0.000000in}}%
\pgfpathlineto{\pgfqpoint{-0.027778in}{0.000000in}}%
\pgfusepath{stroke,fill}%
}%
\begin{pgfscope}%
\pgfsys@transformshift{0.640323in}{6.265178in}%
\pgfsys@useobject{currentmarker}{}%
\end{pgfscope}%
\end{pgfscope}%
\begin{pgfscope}%
\pgfpathrectangle{\pgfqpoint{0.640323in}{5.147436in}}{\pgfqpoint{9.687500in}{3.850000in}}%
\pgfusepath{clip}%
\pgfsetrectcap%
\pgfsetroundjoin%
\pgfsetlinewidth{0.803000pt}%
\definecolor{currentstroke}{rgb}{0.600000,0.600000,0.600000}%
\pgfsetstrokecolor{currentstroke}%
\pgfsetstrokeopacity{0.200000}%
\pgfsetdash{}{0pt}%
\pgfpathmoveto{\pgfqpoint{0.640323in}{6.389371in}}%
\pgfpathlineto{\pgfqpoint{10.327822in}{6.389371in}}%
\pgfusepath{stroke}%
\end{pgfscope}%
\begin{pgfscope}%
\pgfsetbuttcap%
\pgfsetroundjoin%
\definecolor{currentfill}{rgb}{0.000000,0.000000,0.000000}%
\pgfsetfillcolor{currentfill}%
\pgfsetlinewidth{0.602250pt}%
\definecolor{currentstroke}{rgb}{0.000000,0.000000,0.000000}%
\pgfsetstrokecolor{currentstroke}%
\pgfsetdash{}{0pt}%
\pgfsys@defobject{currentmarker}{\pgfqpoint{-0.027778in}{0.000000in}}{\pgfqpoint{-0.000000in}{0.000000in}}{%
\pgfpathmoveto{\pgfqpoint{-0.000000in}{0.000000in}}%
\pgfpathlineto{\pgfqpoint{-0.027778in}{0.000000in}}%
\pgfusepath{stroke,fill}%
}%
\begin{pgfscope}%
\pgfsys@transformshift{0.640323in}{6.389371in}%
\pgfsys@useobject{currentmarker}{}%
\end{pgfscope}%
\end{pgfscope}%
\begin{pgfscope}%
\pgfpathrectangle{\pgfqpoint{0.640323in}{5.147436in}}{\pgfqpoint{9.687500in}{3.850000in}}%
\pgfusepath{clip}%
\pgfsetrectcap%
\pgfsetroundjoin%
\pgfsetlinewidth{0.803000pt}%
\definecolor{currentstroke}{rgb}{0.600000,0.600000,0.600000}%
\pgfsetstrokecolor{currentstroke}%
\pgfsetstrokeopacity{0.200000}%
\pgfsetdash{}{0pt}%
\pgfpathmoveto{\pgfqpoint{0.640323in}{6.637758in}}%
\pgfpathlineto{\pgfqpoint{10.327822in}{6.637758in}}%
\pgfusepath{stroke}%
\end{pgfscope}%
\begin{pgfscope}%
\pgfsetbuttcap%
\pgfsetroundjoin%
\definecolor{currentfill}{rgb}{0.000000,0.000000,0.000000}%
\pgfsetfillcolor{currentfill}%
\pgfsetlinewidth{0.602250pt}%
\definecolor{currentstroke}{rgb}{0.000000,0.000000,0.000000}%
\pgfsetstrokecolor{currentstroke}%
\pgfsetdash{}{0pt}%
\pgfsys@defobject{currentmarker}{\pgfqpoint{-0.027778in}{0.000000in}}{\pgfqpoint{-0.000000in}{0.000000in}}{%
\pgfpathmoveto{\pgfqpoint{-0.000000in}{0.000000in}}%
\pgfpathlineto{\pgfqpoint{-0.027778in}{0.000000in}}%
\pgfusepath{stroke,fill}%
}%
\begin{pgfscope}%
\pgfsys@transformshift{0.640323in}{6.637758in}%
\pgfsys@useobject{currentmarker}{}%
\end{pgfscope}%
\end{pgfscope}%
\begin{pgfscope}%
\pgfpathrectangle{\pgfqpoint{0.640323in}{5.147436in}}{\pgfqpoint{9.687500in}{3.850000in}}%
\pgfusepath{clip}%
\pgfsetrectcap%
\pgfsetroundjoin%
\pgfsetlinewidth{0.803000pt}%
\definecolor{currentstroke}{rgb}{0.600000,0.600000,0.600000}%
\pgfsetstrokecolor{currentstroke}%
\pgfsetstrokeopacity{0.200000}%
\pgfsetdash{}{0pt}%
\pgfpathmoveto{\pgfqpoint{0.640323in}{6.761952in}}%
\pgfpathlineto{\pgfqpoint{10.327822in}{6.761952in}}%
\pgfusepath{stroke}%
\end{pgfscope}%
\begin{pgfscope}%
\pgfsetbuttcap%
\pgfsetroundjoin%
\definecolor{currentfill}{rgb}{0.000000,0.000000,0.000000}%
\pgfsetfillcolor{currentfill}%
\pgfsetlinewidth{0.602250pt}%
\definecolor{currentstroke}{rgb}{0.000000,0.000000,0.000000}%
\pgfsetstrokecolor{currentstroke}%
\pgfsetdash{}{0pt}%
\pgfsys@defobject{currentmarker}{\pgfqpoint{-0.027778in}{0.000000in}}{\pgfqpoint{-0.000000in}{0.000000in}}{%
\pgfpathmoveto{\pgfqpoint{-0.000000in}{0.000000in}}%
\pgfpathlineto{\pgfqpoint{-0.027778in}{0.000000in}}%
\pgfusepath{stroke,fill}%
}%
\begin{pgfscope}%
\pgfsys@transformshift{0.640323in}{6.761952in}%
\pgfsys@useobject{currentmarker}{}%
\end{pgfscope}%
\end{pgfscope}%
\begin{pgfscope}%
\pgfpathrectangle{\pgfqpoint{0.640323in}{5.147436in}}{\pgfqpoint{9.687500in}{3.850000in}}%
\pgfusepath{clip}%
\pgfsetrectcap%
\pgfsetroundjoin%
\pgfsetlinewidth{0.803000pt}%
\definecolor{currentstroke}{rgb}{0.600000,0.600000,0.600000}%
\pgfsetstrokecolor{currentstroke}%
\pgfsetstrokeopacity{0.200000}%
\pgfsetdash{}{0pt}%
\pgfpathmoveto{\pgfqpoint{0.640323in}{6.886146in}}%
\pgfpathlineto{\pgfqpoint{10.327822in}{6.886146in}}%
\pgfusepath{stroke}%
\end{pgfscope}%
\begin{pgfscope}%
\pgfsetbuttcap%
\pgfsetroundjoin%
\definecolor{currentfill}{rgb}{0.000000,0.000000,0.000000}%
\pgfsetfillcolor{currentfill}%
\pgfsetlinewidth{0.602250pt}%
\definecolor{currentstroke}{rgb}{0.000000,0.000000,0.000000}%
\pgfsetstrokecolor{currentstroke}%
\pgfsetdash{}{0pt}%
\pgfsys@defobject{currentmarker}{\pgfqpoint{-0.027778in}{0.000000in}}{\pgfqpoint{-0.000000in}{0.000000in}}{%
\pgfpathmoveto{\pgfqpoint{-0.000000in}{0.000000in}}%
\pgfpathlineto{\pgfqpoint{-0.027778in}{0.000000in}}%
\pgfusepath{stroke,fill}%
}%
\begin{pgfscope}%
\pgfsys@transformshift{0.640323in}{6.886146in}%
\pgfsys@useobject{currentmarker}{}%
\end{pgfscope}%
\end{pgfscope}%
\begin{pgfscope}%
\pgfpathrectangle{\pgfqpoint{0.640323in}{5.147436in}}{\pgfqpoint{9.687500in}{3.850000in}}%
\pgfusepath{clip}%
\pgfsetrectcap%
\pgfsetroundjoin%
\pgfsetlinewidth{0.803000pt}%
\definecolor{currentstroke}{rgb}{0.600000,0.600000,0.600000}%
\pgfsetstrokecolor{currentstroke}%
\pgfsetstrokeopacity{0.200000}%
\pgfsetdash{}{0pt}%
\pgfpathmoveto{\pgfqpoint{0.640323in}{7.010339in}}%
\pgfpathlineto{\pgfqpoint{10.327822in}{7.010339in}}%
\pgfusepath{stroke}%
\end{pgfscope}%
\begin{pgfscope}%
\pgfsetbuttcap%
\pgfsetroundjoin%
\definecolor{currentfill}{rgb}{0.000000,0.000000,0.000000}%
\pgfsetfillcolor{currentfill}%
\pgfsetlinewidth{0.602250pt}%
\definecolor{currentstroke}{rgb}{0.000000,0.000000,0.000000}%
\pgfsetstrokecolor{currentstroke}%
\pgfsetdash{}{0pt}%
\pgfsys@defobject{currentmarker}{\pgfqpoint{-0.027778in}{0.000000in}}{\pgfqpoint{-0.000000in}{0.000000in}}{%
\pgfpathmoveto{\pgfqpoint{-0.000000in}{0.000000in}}%
\pgfpathlineto{\pgfqpoint{-0.027778in}{0.000000in}}%
\pgfusepath{stroke,fill}%
}%
\begin{pgfscope}%
\pgfsys@transformshift{0.640323in}{7.010339in}%
\pgfsys@useobject{currentmarker}{}%
\end{pgfscope}%
\end{pgfscope}%
\begin{pgfscope}%
\pgfpathrectangle{\pgfqpoint{0.640323in}{5.147436in}}{\pgfqpoint{9.687500in}{3.850000in}}%
\pgfusepath{clip}%
\pgfsetrectcap%
\pgfsetroundjoin%
\pgfsetlinewidth{0.803000pt}%
\definecolor{currentstroke}{rgb}{0.600000,0.600000,0.600000}%
\pgfsetstrokecolor{currentstroke}%
\pgfsetstrokeopacity{0.200000}%
\pgfsetdash{}{0pt}%
\pgfpathmoveto{\pgfqpoint{0.640323in}{7.258726in}}%
\pgfpathlineto{\pgfqpoint{10.327822in}{7.258726in}}%
\pgfusepath{stroke}%
\end{pgfscope}%
\begin{pgfscope}%
\pgfsetbuttcap%
\pgfsetroundjoin%
\definecolor{currentfill}{rgb}{0.000000,0.000000,0.000000}%
\pgfsetfillcolor{currentfill}%
\pgfsetlinewidth{0.602250pt}%
\definecolor{currentstroke}{rgb}{0.000000,0.000000,0.000000}%
\pgfsetstrokecolor{currentstroke}%
\pgfsetdash{}{0pt}%
\pgfsys@defobject{currentmarker}{\pgfqpoint{-0.027778in}{0.000000in}}{\pgfqpoint{-0.000000in}{0.000000in}}{%
\pgfpathmoveto{\pgfqpoint{-0.000000in}{0.000000in}}%
\pgfpathlineto{\pgfqpoint{-0.027778in}{0.000000in}}%
\pgfusepath{stroke,fill}%
}%
\begin{pgfscope}%
\pgfsys@transformshift{0.640323in}{7.258726in}%
\pgfsys@useobject{currentmarker}{}%
\end{pgfscope}%
\end{pgfscope}%
\begin{pgfscope}%
\pgfpathrectangle{\pgfqpoint{0.640323in}{5.147436in}}{\pgfqpoint{9.687500in}{3.850000in}}%
\pgfusepath{clip}%
\pgfsetrectcap%
\pgfsetroundjoin%
\pgfsetlinewidth{0.803000pt}%
\definecolor{currentstroke}{rgb}{0.600000,0.600000,0.600000}%
\pgfsetstrokecolor{currentstroke}%
\pgfsetstrokeopacity{0.200000}%
\pgfsetdash{}{0pt}%
\pgfpathmoveto{\pgfqpoint{0.640323in}{7.382920in}}%
\pgfpathlineto{\pgfqpoint{10.327822in}{7.382920in}}%
\pgfusepath{stroke}%
\end{pgfscope}%
\begin{pgfscope}%
\pgfsetbuttcap%
\pgfsetroundjoin%
\definecolor{currentfill}{rgb}{0.000000,0.000000,0.000000}%
\pgfsetfillcolor{currentfill}%
\pgfsetlinewidth{0.602250pt}%
\definecolor{currentstroke}{rgb}{0.000000,0.000000,0.000000}%
\pgfsetstrokecolor{currentstroke}%
\pgfsetdash{}{0pt}%
\pgfsys@defobject{currentmarker}{\pgfqpoint{-0.027778in}{0.000000in}}{\pgfqpoint{-0.000000in}{0.000000in}}{%
\pgfpathmoveto{\pgfqpoint{-0.000000in}{0.000000in}}%
\pgfpathlineto{\pgfqpoint{-0.027778in}{0.000000in}}%
\pgfusepath{stroke,fill}%
}%
\begin{pgfscope}%
\pgfsys@transformshift{0.640323in}{7.382920in}%
\pgfsys@useobject{currentmarker}{}%
\end{pgfscope}%
\end{pgfscope}%
\begin{pgfscope}%
\pgfpathrectangle{\pgfqpoint{0.640323in}{5.147436in}}{\pgfqpoint{9.687500in}{3.850000in}}%
\pgfusepath{clip}%
\pgfsetrectcap%
\pgfsetroundjoin%
\pgfsetlinewidth{0.803000pt}%
\definecolor{currentstroke}{rgb}{0.600000,0.600000,0.600000}%
\pgfsetstrokecolor{currentstroke}%
\pgfsetstrokeopacity{0.200000}%
\pgfsetdash{}{0pt}%
\pgfpathmoveto{\pgfqpoint{0.640323in}{7.507113in}}%
\pgfpathlineto{\pgfqpoint{10.327822in}{7.507113in}}%
\pgfusepath{stroke}%
\end{pgfscope}%
\begin{pgfscope}%
\pgfsetbuttcap%
\pgfsetroundjoin%
\definecolor{currentfill}{rgb}{0.000000,0.000000,0.000000}%
\pgfsetfillcolor{currentfill}%
\pgfsetlinewidth{0.602250pt}%
\definecolor{currentstroke}{rgb}{0.000000,0.000000,0.000000}%
\pgfsetstrokecolor{currentstroke}%
\pgfsetdash{}{0pt}%
\pgfsys@defobject{currentmarker}{\pgfqpoint{-0.027778in}{0.000000in}}{\pgfqpoint{-0.000000in}{0.000000in}}{%
\pgfpathmoveto{\pgfqpoint{-0.000000in}{0.000000in}}%
\pgfpathlineto{\pgfqpoint{-0.027778in}{0.000000in}}%
\pgfusepath{stroke,fill}%
}%
\begin{pgfscope}%
\pgfsys@transformshift{0.640323in}{7.507113in}%
\pgfsys@useobject{currentmarker}{}%
\end{pgfscope}%
\end{pgfscope}%
\begin{pgfscope}%
\pgfpathrectangle{\pgfqpoint{0.640323in}{5.147436in}}{\pgfqpoint{9.687500in}{3.850000in}}%
\pgfusepath{clip}%
\pgfsetrectcap%
\pgfsetroundjoin%
\pgfsetlinewidth{0.803000pt}%
\definecolor{currentstroke}{rgb}{0.600000,0.600000,0.600000}%
\pgfsetstrokecolor{currentstroke}%
\pgfsetstrokeopacity{0.200000}%
\pgfsetdash{}{0pt}%
\pgfpathmoveto{\pgfqpoint{0.640323in}{7.631307in}}%
\pgfpathlineto{\pgfqpoint{10.327822in}{7.631307in}}%
\pgfusepath{stroke}%
\end{pgfscope}%
\begin{pgfscope}%
\pgfsetbuttcap%
\pgfsetroundjoin%
\definecolor{currentfill}{rgb}{0.000000,0.000000,0.000000}%
\pgfsetfillcolor{currentfill}%
\pgfsetlinewidth{0.602250pt}%
\definecolor{currentstroke}{rgb}{0.000000,0.000000,0.000000}%
\pgfsetstrokecolor{currentstroke}%
\pgfsetdash{}{0pt}%
\pgfsys@defobject{currentmarker}{\pgfqpoint{-0.027778in}{0.000000in}}{\pgfqpoint{-0.000000in}{0.000000in}}{%
\pgfpathmoveto{\pgfqpoint{-0.000000in}{0.000000in}}%
\pgfpathlineto{\pgfqpoint{-0.027778in}{0.000000in}}%
\pgfusepath{stroke,fill}%
}%
\begin{pgfscope}%
\pgfsys@transformshift{0.640323in}{7.631307in}%
\pgfsys@useobject{currentmarker}{}%
\end{pgfscope}%
\end{pgfscope}%
\begin{pgfscope}%
\pgfpathrectangle{\pgfqpoint{0.640323in}{5.147436in}}{\pgfqpoint{9.687500in}{3.850000in}}%
\pgfusepath{clip}%
\pgfsetrectcap%
\pgfsetroundjoin%
\pgfsetlinewidth{0.803000pt}%
\definecolor{currentstroke}{rgb}{0.600000,0.600000,0.600000}%
\pgfsetstrokecolor{currentstroke}%
\pgfsetstrokeopacity{0.200000}%
\pgfsetdash{}{0pt}%
\pgfpathmoveto{\pgfqpoint{0.640323in}{7.879694in}}%
\pgfpathlineto{\pgfqpoint{10.327822in}{7.879694in}}%
\pgfusepath{stroke}%
\end{pgfscope}%
\begin{pgfscope}%
\pgfsetbuttcap%
\pgfsetroundjoin%
\definecolor{currentfill}{rgb}{0.000000,0.000000,0.000000}%
\pgfsetfillcolor{currentfill}%
\pgfsetlinewidth{0.602250pt}%
\definecolor{currentstroke}{rgb}{0.000000,0.000000,0.000000}%
\pgfsetstrokecolor{currentstroke}%
\pgfsetdash{}{0pt}%
\pgfsys@defobject{currentmarker}{\pgfqpoint{-0.027778in}{0.000000in}}{\pgfqpoint{-0.000000in}{0.000000in}}{%
\pgfpathmoveto{\pgfqpoint{-0.000000in}{0.000000in}}%
\pgfpathlineto{\pgfqpoint{-0.027778in}{0.000000in}}%
\pgfusepath{stroke,fill}%
}%
\begin{pgfscope}%
\pgfsys@transformshift{0.640323in}{7.879694in}%
\pgfsys@useobject{currentmarker}{}%
\end{pgfscope}%
\end{pgfscope}%
\begin{pgfscope}%
\pgfpathrectangle{\pgfqpoint{0.640323in}{5.147436in}}{\pgfqpoint{9.687500in}{3.850000in}}%
\pgfusepath{clip}%
\pgfsetrectcap%
\pgfsetroundjoin%
\pgfsetlinewidth{0.803000pt}%
\definecolor{currentstroke}{rgb}{0.600000,0.600000,0.600000}%
\pgfsetstrokecolor{currentstroke}%
\pgfsetstrokeopacity{0.200000}%
\pgfsetdash{}{0pt}%
\pgfpathmoveto{\pgfqpoint{0.640323in}{8.003887in}}%
\pgfpathlineto{\pgfqpoint{10.327822in}{8.003887in}}%
\pgfusepath{stroke}%
\end{pgfscope}%
\begin{pgfscope}%
\pgfsetbuttcap%
\pgfsetroundjoin%
\definecolor{currentfill}{rgb}{0.000000,0.000000,0.000000}%
\pgfsetfillcolor{currentfill}%
\pgfsetlinewidth{0.602250pt}%
\definecolor{currentstroke}{rgb}{0.000000,0.000000,0.000000}%
\pgfsetstrokecolor{currentstroke}%
\pgfsetdash{}{0pt}%
\pgfsys@defobject{currentmarker}{\pgfqpoint{-0.027778in}{0.000000in}}{\pgfqpoint{-0.000000in}{0.000000in}}{%
\pgfpathmoveto{\pgfqpoint{-0.000000in}{0.000000in}}%
\pgfpathlineto{\pgfqpoint{-0.027778in}{0.000000in}}%
\pgfusepath{stroke,fill}%
}%
\begin{pgfscope}%
\pgfsys@transformshift{0.640323in}{8.003887in}%
\pgfsys@useobject{currentmarker}{}%
\end{pgfscope}%
\end{pgfscope}%
\begin{pgfscope}%
\pgfpathrectangle{\pgfqpoint{0.640323in}{5.147436in}}{\pgfqpoint{9.687500in}{3.850000in}}%
\pgfusepath{clip}%
\pgfsetrectcap%
\pgfsetroundjoin%
\pgfsetlinewidth{0.803000pt}%
\definecolor{currentstroke}{rgb}{0.600000,0.600000,0.600000}%
\pgfsetstrokecolor{currentstroke}%
\pgfsetstrokeopacity{0.200000}%
\pgfsetdash{}{0pt}%
\pgfpathmoveto{\pgfqpoint{0.640323in}{8.128081in}}%
\pgfpathlineto{\pgfqpoint{10.327822in}{8.128081in}}%
\pgfusepath{stroke}%
\end{pgfscope}%
\begin{pgfscope}%
\pgfsetbuttcap%
\pgfsetroundjoin%
\definecolor{currentfill}{rgb}{0.000000,0.000000,0.000000}%
\pgfsetfillcolor{currentfill}%
\pgfsetlinewidth{0.602250pt}%
\definecolor{currentstroke}{rgb}{0.000000,0.000000,0.000000}%
\pgfsetstrokecolor{currentstroke}%
\pgfsetdash{}{0pt}%
\pgfsys@defobject{currentmarker}{\pgfqpoint{-0.027778in}{0.000000in}}{\pgfqpoint{-0.000000in}{0.000000in}}{%
\pgfpathmoveto{\pgfqpoint{-0.000000in}{0.000000in}}%
\pgfpathlineto{\pgfqpoint{-0.027778in}{0.000000in}}%
\pgfusepath{stroke,fill}%
}%
\begin{pgfscope}%
\pgfsys@transformshift{0.640323in}{8.128081in}%
\pgfsys@useobject{currentmarker}{}%
\end{pgfscope}%
\end{pgfscope}%
\begin{pgfscope}%
\pgfpathrectangle{\pgfqpoint{0.640323in}{5.147436in}}{\pgfqpoint{9.687500in}{3.850000in}}%
\pgfusepath{clip}%
\pgfsetrectcap%
\pgfsetroundjoin%
\pgfsetlinewidth{0.803000pt}%
\definecolor{currentstroke}{rgb}{0.600000,0.600000,0.600000}%
\pgfsetstrokecolor{currentstroke}%
\pgfsetstrokeopacity{0.200000}%
\pgfsetdash{}{0pt}%
\pgfpathmoveto{\pgfqpoint{0.640323in}{8.252275in}}%
\pgfpathlineto{\pgfqpoint{10.327822in}{8.252275in}}%
\pgfusepath{stroke}%
\end{pgfscope}%
\begin{pgfscope}%
\pgfsetbuttcap%
\pgfsetroundjoin%
\definecolor{currentfill}{rgb}{0.000000,0.000000,0.000000}%
\pgfsetfillcolor{currentfill}%
\pgfsetlinewidth{0.602250pt}%
\definecolor{currentstroke}{rgb}{0.000000,0.000000,0.000000}%
\pgfsetstrokecolor{currentstroke}%
\pgfsetdash{}{0pt}%
\pgfsys@defobject{currentmarker}{\pgfqpoint{-0.027778in}{0.000000in}}{\pgfqpoint{-0.000000in}{0.000000in}}{%
\pgfpathmoveto{\pgfqpoint{-0.000000in}{0.000000in}}%
\pgfpathlineto{\pgfqpoint{-0.027778in}{0.000000in}}%
\pgfusepath{stroke,fill}%
}%
\begin{pgfscope}%
\pgfsys@transformshift{0.640323in}{8.252275in}%
\pgfsys@useobject{currentmarker}{}%
\end{pgfscope}%
\end{pgfscope}%
\begin{pgfscope}%
\pgfpathrectangle{\pgfqpoint{0.640323in}{5.147436in}}{\pgfqpoint{9.687500in}{3.850000in}}%
\pgfusepath{clip}%
\pgfsetrectcap%
\pgfsetroundjoin%
\pgfsetlinewidth{0.803000pt}%
\definecolor{currentstroke}{rgb}{0.600000,0.600000,0.600000}%
\pgfsetstrokecolor{currentstroke}%
\pgfsetstrokeopacity{0.200000}%
\pgfsetdash{}{0pt}%
\pgfpathmoveto{\pgfqpoint{0.640323in}{8.500662in}}%
\pgfpathlineto{\pgfqpoint{10.327822in}{8.500662in}}%
\pgfusepath{stroke}%
\end{pgfscope}%
\begin{pgfscope}%
\pgfsetbuttcap%
\pgfsetroundjoin%
\definecolor{currentfill}{rgb}{0.000000,0.000000,0.000000}%
\pgfsetfillcolor{currentfill}%
\pgfsetlinewidth{0.602250pt}%
\definecolor{currentstroke}{rgb}{0.000000,0.000000,0.000000}%
\pgfsetstrokecolor{currentstroke}%
\pgfsetdash{}{0pt}%
\pgfsys@defobject{currentmarker}{\pgfqpoint{-0.027778in}{0.000000in}}{\pgfqpoint{-0.000000in}{0.000000in}}{%
\pgfpathmoveto{\pgfqpoint{-0.000000in}{0.000000in}}%
\pgfpathlineto{\pgfqpoint{-0.027778in}{0.000000in}}%
\pgfusepath{stroke,fill}%
}%
\begin{pgfscope}%
\pgfsys@transformshift{0.640323in}{8.500662in}%
\pgfsys@useobject{currentmarker}{}%
\end{pgfscope}%
\end{pgfscope}%
\begin{pgfscope}%
\pgfpathrectangle{\pgfqpoint{0.640323in}{5.147436in}}{\pgfqpoint{9.687500in}{3.850000in}}%
\pgfusepath{clip}%
\pgfsetrectcap%
\pgfsetroundjoin%
\pgfsetlinewidth{0.803000pt}%
\definecolor{currentstroke}{rgb}{0.600000,0.600000,0.600000}%
\pgfsetstrokecolor{currentstroke}%
\pgfsetstrokeopacity{0.200000}%
\pgfsetdash{}{0pt}%
\pgfpathmoveto{\pgfqpoint{0.640323in}{8.624855in}}%
\pgfpathlineto{\pgfqpoint{10.327822in}{8.624855in}}%
\pgfusepath{stroke}%
\end{pgfscope}%
\begin{pgfscope}%
\pgfsetbuttcap%
\pgfsetroundjoin%
\definecolor{currentfill}{rgb}{0.000000,0.000000,0.000000}%
\pgfsetfillcolor{currentfill}%
\pgfsetlinewidth{0.602250pt}%
\definecolor{currentstroke}{rgb}{0.000000,0.000000,0.000000}%
\pgfsetstrokecolor{currentstroke}%
\pgfsetdash{}{0pt}%
\pgfsys@defobject{currentmarker}{\pgfqpoint{-0.027778in}{0.000000in}}{\pgfqpoint{-0.000000in}{0.000000in}}{%
\pgfpathmoveto{\pgfqpoint{-0.000000in}{0.000000in}}%
\pgfpathlineto{\pgfqpoint{-0.027778in}{0.000000in}}%
\pgfusepath{stroke,fill}%
}%
\begin{pgfscope}%
\pgfsys@transformshift{0.640323in}{8.624855in}%
\pgfsys@useobject{currentmarker}{}%
\end{pgfscope}%
\end{pgfscope}%
\begin{pgfscope}%
\pgfpathrectangle{\pgfqpoint{0.640323in}{5.147436in}}{\pgfqpoint{9.687500in}{3.850000in}}%
\pgfusepath{clip}%
\pgfsetrectcap%
\pgfsetroundjoin%
\pgfsetlinewidth{0.803000pt}%
\definecolor{currentstroke}{rgb}{0.600000,0.600000,0.600000}%
\pgfsetstrokecolor{currentstroke}%
\pgfsetstrokeopacity{0.200000}%
\pgfsetdash{}{0pt}%
\pgfpathmoveto{\pgfqpoint{0.640323in}{8.749049in}}%
\pgfpathlineto{\pgfqpoint{10.327822in}{8.749049in}}%
\pgfusepath{stroke}%
\end{pgfscope}%
\begin{pgfscope}%
\pgfsetbuttcap%
\pgfsetroundjoin%
\definecolor{currentfill}{rgb}{0.000000,0.000000,0.000000}%
\pgfsetfillcolor{currentfill}%
\pgfsetlinewidth{0.602250pt}%
\definecolor{currentstroke}{rgb}{0.000000,0.000000,0.000000}%
\pgfsetstrokecolor{currentstroke}%
\pgfsetdash{}{0pt}%
\pgfsys@defobject{currentmarker}{\pgfqpoint{-0.027778in}{0.000000in}}{\pgfqpoint{-0.000000in}{0.000000in}}{%
\pgfpathmoveto{\pgfqpoint{-0.000000in}{0.000000in}}%
\pgfpathlineto{\pgfqpoint{-0.027778in}{0.000000in}}%
\pgfusepath{stroke,fill}%
}%
\begin{pgfscope}%
\pgfsys@transformshift{0.640323in}{8.749049in}%
\pgfsys@useobject{currentmarker}{}%
\end{pgfscope}%
\end{pgfscope}%
\begin{pgfscope}%
\pgfpathrectangle{\pgfqpoint{0.640323in}{5.147436in}}{\pgfqpoint{9.687500in}{3.850000in}}%
\pgfusepath{clip}%
\pgfsetrectcap%
\pgfsetroundjoin%
\pgfsetlinewidth{0.803000pt}%
\definecolor{currentstroke}{rgb}{0.600000,0.600000,0.600000}%
\pgfsetstrokecolor{currentstroke}%
\pgfsetstrokeopacity{0.200000}%
\pgfsetdash{}{0pt}%
\pgfpathmoveto{\pgfqpoint{0.640323in}{8.873242in}}%
\pgfpathlineto{\pgfqpoint{10.327822in}{8.873242in}}%
\pgfusepath{stroke}%
\end{pgfscope}%
\begin{pgfscope}%
\pgfsetbuttcap%
\pgfsetroundjoin%
\definecolor{currentfill}{rgb}{0.000000,0.000000,0.000000}%
\pgfsetfillcolor{currentfill}%
\pgfsetlinewidth{0.602250pt}%
\definecolor{currentstroke}{rgb}{0.000000,0.000000,0.000000}%
\pgfsetstrokecolor{currentstroke}%
\pgfsetdash{}{0pt}%
\pgfsys@defobject{currentmarker}{\pgfqpoint{-0.027778in}{0.000000in}}{\pgfqpoint{-0.000000in}{0.000000in}}{%
\pgfpathmoveto{\pgfqpoint{-0.000000in}{0.000000in}}%
\pgfpathlineto{\pgfqpoint{-0.027778in}{0.000000in}}%
\pgfusepath{stroke,fill}%
}%
\begin{pgfscope}%
\pgfsys@transformshift{0.640323in}{8.873242in}%
\pgfsys@useobject{currentmarker}{}%
\end{pgfscope}%
\end{pgfscope}%
\begin{pgfscope}%
\definecolor{textcolor}{rgb}{0.000000,0.000000,0.000000}%
\pgfsetstrokecolor{textcolor}%
\pgfsetfillcolor{textcolor}%
\pgftext[x=0.266665in,y=7.072436in,,bottom,rotate=90.000000]{\color{textcolor}\sffamily\fontsize{10.000000}{12.000000}\selectfont avg. infection rate \(\displaystyle \overline{\langle I\rangle}\)}%
\end{pgfscope}%
\begin{pgfscope}%
\pgfpathrectangle{\pgfqpoint{0.640323in}{5.147436in}}{\pgfqpoint{9.687500in}{3.850000in}}%
\pgfusepath{clip}%
\pgfsetbuttcap%
\pgfsetroundjoin%
\pgfsetlinewidth{1.003750pt}%
\definecolor{currentstroke}{rgb}{0.000000,0.000000,1.000000}%
\pgfsetstrokecolor{currentstroke}%
\pgfsetstrokeopacity{0.500000}%
\pgfsetdash{{3.700000pt}{1.600000pt}}{0.000000pt}%
\pgfpathmoveto{\pgfqpoint{1.080663in}{5.275389in}}%
\pgfpathlineto{\pgfqpoint{1.260394in}{5.275607in}}%
\pgfpathlineto{\pgfqpoint{1.440125in}{5.276505in}}%
\pgfpathlineto{\pgfqpoint{1.619856in}{5.276626in}}%
\pgfpathlineto{\pgfqpoint{1.799587in}{5.277742in}}%
\pgfpathlineto{\pgfqpoint{1.979318in}{5.277378in}}%
\pgfpathlineto{\pgfqpoint{2.159049in}{5.279489in}}%
\pgfpathlineto{\pgfqpoint{2.338780in}{5.278809in}}%
\pgfpathlineto{\pgfqpoint{2.518511in}{5.276917in}}%
\pgfpathlineto{\pgfqpoint{2.698242in}{5.278057in}}%
\pgfpathlineto{\pgfqpoint{2.877973in}{5.277475in}}%
\pgfpathlineto{\pgfqpoint{3.057704in}{5.281162in}}%
\pgfpathlineto{\pgfqpoint{3.237435in}{5.278518in}}%
\pgfpathlineto{\pgfqpoint{3.417166in}{5.280968in}}%
\pgfpathlineto{\pgfqpoint{3.596897in}{5.278785in}}%
\pgfpathlineto{\pgfqpoint{3.776628in}{5.283733in}}%
\pgfpathlineto{\pgfqpoint{3.956359in}{5.283345in}}%
\pgfpathlineto{\pgfqpoint{4.136090in}{5.281235in}}%
\pgfpathlineto{\pgfqpoint{4.315821in}{5.290477in}}%
\pgfpathlineto{\pgfqpoint{4.495552in}{5.279901in}}%
\pgfpathlineto{\pgfqpoint{4.675283in}{5.310513in}}%
\pgfpathlineto{\pgfqpoint{4.855014in}{5.287687in}}%
\pgfpathlineto{\pgfqpoint{5.034745in}{5.279537in}}%
\pgfpathlineto{\pgfqpoint{5.214476in}{5.323684in}}%
\pgfpathlineto{\pgfqpoint{5.394207in}{5.299646in}}%
\pgfpathlineto{\pgfqpoint{5.573938in}{5.309664in}}%
\pgfpathlineto{\pgfqpoint{5.753669in}{5.343817in}}%
\pgfpathlineto{\pgfqpoint{5.933400in}{5.524043in}}%
\pgfpathlineto{\pgfqpoint{6.113131in}{5.310319in}}%
\pgfpathlineto{\pgfqpoint{6.292862in}{5.909891in}}%
\pgfpathlineto{\pgfqpoint{6.472593in}{5.552738in}}%
\pgfpathlineto{\pgfqpoint{6.652324in}{5.463984in}}%
\pgfpathlineto{\pgfqpoint{6.832055in}{5.505948in}}%
\pgfpathlineto{\pgfqpoint{7.011786in}{5.527730in}}%
\pgfpathlineto{\pgfqpoint{7.191517in}{6.071175in}}%
\pgfpathlineto{\pgfqpoint{7.371248in}{5.485451in}}%
\pgfpathlineto{\pgfqpoint{7.550979in}{6.155900in}}%
\pgfpathlineto{\pgfqpoint{7.730710in}{6.217078in}}%
\pgfpathlineto{\pgfqpoint{7.910441in}{6.219506in}}%
\pgfpathlineto{\pgfqpoint{8.090172in}{6.263967in}}%
\pgfpathlineto{\pgfqpoint{8.269903in}{6.319388in}}%
\pgfpathlineto{\pgfqpoint{8.449634in}{6.361304in}}%
\pgfpathlineto{\pgfqpoint{8.629365in}{6.346872in}}%
\pgfpathlineto{\pgfqpoint{8.809096in}{6.368538in}}%
\pgfpathlineto{\pgfqpoint{8.988827in}{6.367929in}}%
\pgfpathlineto{\pgfqpoint{9.168558in}{6.419861in}}%
\pgfpathlineto{\pgfqpoint{9.348289in}{6.462018in}}%
\pgfpathlineto{\pgfqpoint{9.528020in}{6.480647in}}%
\pgfpathlineto{\pgfqpoint{9.707751in}{6.471674in}}%
\pgfpathlineto{\pgfqpoint{9.887482in}{6.488943in}}%
\pgfusepath{stroke}%
\end{pgfscope}%
\begin{pgfscope}%
\pgfpathrectangle{\pgfqpoint{0.640323in}{5.147436in}}{\pgfqpoint{9.687500in}{3.850000in}}%
\pgfusepath{clip}%
\pgfsetbuttcap%
\pgfsetroundjoin%
\pgfsetlinewidth{1.003750pt}%
\definecolor{currentstroke}{rgb}{0.980392,0.164706,0.333333}%
\pgfsetstrokecolor{currentstroke}%
\pgfsetstrokeopacity{0.500000}%
\pgfsetdash{{3.700000pt}{1.600000pt}}{0.000000pt}%
\pgfpathmoveto{\pgfqpoint{1.080663in}{5.276705in}}%
\pgfpathlineto{\pgfqpoint{1.260394in}{5.276469in}}%
\pgfpathlineto{\pgfqpoint{1.440125in}{5.276523in}}%
\pgfpathlineto{\pgfqpoint{1.619856in}{5.276893in}}%
\pgfpathlineto{\pgfqpoint{1.799587in}{5.277433in}}%
\pgfpathlineto{\pgfqpoint{1.979318in}{5.277633in}}%
\pgfpathlineto{\pgfqpoint{2.159049in}{5.277524in}}%
\pgfpathlineto{\pgfqpoint{2.338780in}{5.277809in}}%
\pgfpathlineto{\pgfqpoint{2.518511in}{5.277851in}}%
\pgfpathlineto{\pgfqpoint{2.698242in}{5.278730in}}%
\pgfpathlineto{\pgfqpoint{2.877973in}{5.278785in}}%
\pgfpathlineto{\pgfqpoint{3.057704in}{5.278542in}}%
\pgfpathlineto{\pgfqpoint{3.237435in}{5.280980in}}%
\pgfpathlineto{\pgfqpoint{3.417166in}{5.281532in}}%
\pgfpathlineto{\pgfqpoint{3.596897in}{5.279246in}}%
\pgfpathlineto{\pgfqpoint{3.776628in}{5.284643in}}%
\pgfpathlineto{\pgfqpoint{3.956359in}{5.280544in}}%
\pgfpathlineto{\pgfqpoint{4.136090in}{5.287057in}}%
\pgfpathlineto{\pgfqpoint{4.315821in}{5.283927in}}%
\pgfpathlineto{\pgfqpoint{4.495552in}{5.285977in}}%
\pgfpathlineto{\pgfqpoint{4.675283in}{5.298706in}}%
\pgfpathlineto{\pgfqpoint{4.855014in}{5.304655in}}%
\pgfpathlineto{\pgfqpoint{5.034745in}{5.314369in}}%
\pgfpathlineto{\pgfqpoint{5.214476in}{5.376624in}}%
\pgfpathlineto{\pgfqpoint{5.394207in}{5.585012in}}%
\pgfpathlineto{\pgfqpoint{5.573938in}{5.684949in}}%
\pgfpathlineto{\pgfqpoint{5.753669in}{5.807293in}}%
\pgfpathlineto{\pgfqpoint{5.933400in}{5.842271in}}%
\pgfpathlineto{\pgfqpoint{6.113131in}{5.905811in}}%
\pgfpathlineto{\pgfqpoint{6.292862in}{6.021479in}}%
\pgfpathlineto{\pgfqpoint{6.472593in}{6.038841in}}%
\pgfpathlineto{\pgfqpoint{6.652324in}{6.079912in}}%
\pgfpathlineto{\pgfqpoint{6.832055in}{6.124516in}}%
\pgfpathlineto{\pgfqpoint{7.011786in}{6.162743in}}%
\pgfpathlineto{\pgfqpoint{7.191517in}{6.221120in}}%
\pgfpathlineto{\pgfqpoint{7.371248in}{6.276051in}}%
\pgfpathlineto{\pgfqpoint{7.550979in}{6.268382in}}%
\pgfpathlineto{\pgfqpoint{7.730710in}{6.312328in}}%
\pgfpathlineto{\pgfqpoint{7.910441in}{6.350170in}}%
\pgfpathlineto{\pgfqpoint{8.090172in}{6.373388in}}%
\pgfpathlineto{\pgfqpoint{8.269903in}{6.400598in}}%
\pgfpathlineto{\pgfqpoint{8.449634in}{6.409627in}}%
\pgfpathlineto{\pgfqpoint{8.629365in}{6.434385in}}%
\pgfpathlineto{\pgfqpoint{8.809096in}{6.445445in}}%
\pgfpathlineto{\pgfqpoint{8.988827in}{6.475251in}}%
\pgfpathlineto{\pgfqpoint{9.168558in}{6.490061in}}%
\pgfpathlineto{\pgfqpoint{9.348289in}{6.506995in}}%
\pgfpathlineto{\pgfqpoint{9.528020in}{6.520184in}}%
\pgfpathlineto{\pgfqpoint{9.707751in}{6.544303in}}%
\pgfpathlineto{\pgfqpoint{9.887482in}{6.554959in}}%
\pgfusepath{stroke}%
\end{pgfscope}%
\begin{pgfscope}%
\pgfpathrectangle{\pgfqpoint{0.640323in}{5.147436in}}{\pgfqpoint{9.687500in}{3.850000in}}%
\pgfusepath{clip}%
\pgfsetbuttcap%
\pgfsetroundjoin%
\pgfsetlinewidth{1.003750pt}%
\definecolor{currentstroke}{rgb}{0.239216,0.478431,0.992157}%
\pgfsetstrokecolor{currentstroke}%
\pgfsetstrokeopacity{0.500000}%
\pgfsetdash{{3.700000pt}{1.600000pt}}{0.000000pt}%
\pgfpathmoveto{\pgfqpoint{1.080663in}{5.276250in}}%
\pgfpathlineto{\pgfqpoint{1.260394in}{5.276256in}}%
\pgfpathlineto{\pgfqpoint{1.440125in}{5.276743in}}%
\pgfpathlineto{\pgfqpoint{1.619856in}{5.276764in}}%
\pgfpathlineto{\pgfqpoint{1.799587in}{5.277131in}}%
\pgfpathlineto{\pgfqpoint{1.979318in}{5.277202in}}%
\pgfpathlineto{\pgfqpoint{2.159049in}{5.277747in}}%
\pgfpathlineto{\pgfqpoint{2.338780in}{5.277872in}}%
\pgfpathlineto{\pgfqpoint{2.518511in}{5.278391in}}%
\pgfpathlineto{\pgfqpoint{2.698242in}{5.278438in}}%
\pgfpathlineto{\pgfqpoint{2.877973in}{5.279081in}}%
\pgfpathlineto{\pgfqpoint{3.057704in}{5.279279in}}%
\pgfpathlineto{\pgfqpoint{3.237435in}{5.279899in}}%
\pgfpathlineto{\pgfqpoint{3.417166in}{5.281779in}}%
\pgfpathlineto{\pgfqpoint{3.596897in}{5.282287in}}%
\pgfpathlineto{\pgfqpoint{3.776628in}{5.282116in}}%
\pgfpathlineto{\pgfqpoint{3.956359in}{5.283697in}}%
\pgfpathlineto{\pgfqpoint{4.136090in}{5.284625in}}%
\pgfpathlineto{\pgfqpoint{4.315821in}{5.287903in}}%
\pgfpathlineto{\pgfqpoint{4.495552in}{5.292931in}}%
\pgfpathlineto{\pgfqpoint{4.675283in}{5.300599in}}%
\pgfpathlineto{\pgfqpoint{4.855014in}{5.322623in}}%
\pgfpathlineto{\pgfqpoint{5.034745in}{5.420359in}}%
\pgfpathlineto{\pgfqpoint{5.214476in}{5.518179in}}%
\pgfpathlineto{\pgfqpoint{5.394207in}{5.674770in}}%
\pgfpathlineto{\pgfqpoint{5.573938in}{5.769451in}}%
\pgfpathlineto{\pgfqpoint{5.753669in}{5.871621in}}%
\pgfpathlineto{\pgfqpoint{5.933400in}{5.959183in}}%
\pgfpathlineto{\pgfqpoint{6.113131in}{6.000292in}}%
\pgfpathlineto{\pgfqpoint{6.292862in}{6.066928in}}%
\pgfpathlineto{\pgfqpoint{6.472593in}{6.103111in}}%
\pgfpathlineto{\pgfqpoint{6.652324in}{6.141369in}}%
\pgfpathlineto{\pgfqpoint{6.832055in}{6.191748in}}%
\pgfpathlineto{\pgfqpoint{7.011786in}{6.245009in}}%
\pgfpathlineto{\pgfqpoint{7.191517in}{6.266047in}}%
\pgfpathlineto{\pgfqpoint{7.371248in}{6.301194in}}%
\pgfpathlineto{\pgfqpoint{7.550979in}{6.332093in}}%
\pgfpathlineto{\pgfqpoint{7.730710in}{6.362396in}}%
\pgfpathlineto{\pgfqpoint{7.910441in}{6.373996in}}%
\pgfpathlineto{\pgfqpoint{8.090172in}{6.410559in}}%
\pgfpathlineto{\pgfqpoint{8.269903in}{6.434006in}}%
\pgfpathlineto{\pgfqpoint{8.449634in}{6.460640in}}%
\pgfpathlineto{\pgfqpoint{8.629365in}{6.481896in}}%
\pgfpathlineto{\pgfqpoint{8.809096in}{6.502561in}}%
\pgfpathlineto{\pgfqpoint{8.988827in}{6.510205in}}%
\pgfpathlineto{\pgfqpoint{9.168558in}{6.531380in}}%
\pgfpathlineto{\pgfqpoint{9.348289in}{6.541744in}}%
\pgfpathlineto{\pgfqpoint{9.528020in}{6.555753in}}%
\pgfpathlineto{\pgfqpoint{9.707751in}{6.575891in}}%
\pgfpathlineto{\pgfqpoint{9.887482in}{6.591403in}}%
\pgfusepath{stroke}%
\end{pgfscope}%
\begin{pgfscope}%
\pgfpathrectangle{\pgfqpoint{0.640323in}{5.147436in}}{\pgfqpoint{9.687500in}{3.850000in}}%
\pgfusepath{clip}%
\pgfsetbuttcap%
\pgfsetroundjoin%
\pgfsetlinewidth{1.003750pt}%
\definecolor{currentstroke}{rgb}{0.000000,0.000000,0.000000}%
\pgfsetstrokecolor{currentstroke}%
\pgfsetstrokeopacity{0.500000}%
\pgfsetdash{{3.700000pt}{1.600000pt}}{0.000000pt}%
\pgfpathmoveto{\pgfqpoint{1.080663in}{5.276317in}}%
\pgfpathlineto{\pgfqpoint{1.260394in}{5.276409in}}%
\pgfpathlineto{\pgfqpoint{1.440125in}{5.276751in}}%
\pgfpathlineto{\pgfqpoint{1.619856in}{5.276889in}}%
\pgfpathlineto{\pgfqpoint{1.799587in}{5.277083in}}%
\pgfpathlineto{\pgfqpoint{1.979318in}{5.277261in}}%
\pgfpathlineto{\pgfqpoint{2.159049in}{5.277601in}}%
\pgfpathlineto{\pgfqpoint{2.338780in}{5.277905in}}%
\pgfpathlineto{\pgfqpoint{2.518511in}{5.278229in}}%
\pgfpathlineto{\pgfqpoint{2.698242in}{5.278620in}}%
\pgfpathlineto{\pgfqpoint{2.877973in}{5.278832in}}%
\pgfpathlineto{\pgfqpoint{3.057704in}{5.279708in}}%
\pgfpathlineto{\pgfqpoint{3.237435in}{5.280171in}}%
\pgfpathlineto{\pgfqpoint{3.417166in}{5.280771in}}%
\pgfpathlineto{\pgfqpoint{3.596897in}{5.282038in}}%
\pgfpathlineto{\pgfqpoint{3.776628in}{5.282917in}}%
\pgfpathlineto{\pgfqpoint{3.956359in}{5.283727in}}%
\pgfpathlineto{\pgfqpoint{4.136090in}{5.285470in}}%
\pgfpathlineto{\pgfqpoint{4.315821in}{5.288475in}}%
\pgfpathlineto{\pgfqpoint{4.495552in}{5.291315in}}%
\pgfpathlineto{\pgfqpoint{4.675283in}{5.296769in}}%
\pgfpathlineto{\pgfqpoint{4.855014in}{5.323733in}}%
\pgfpathlineto{\pgfqpoint{5.034745in}{5.428259in}}%
\pgfpathlineto{\pgfqpoint{5.214476in}{5.593213in}}%
\pgfpathlineto{\pgfqpoint{5.394207in}{5.708419in}}%
\pgfpathlineto{\pgfqpoint{5.573938in}{5.825315in}}%
\pgfpathlineto{\pgfqpoint{5.753669in}{5.898707in}}%
\pgfpathlineto{\pgfqpoint{5.933400in}{5.968647in}}%
\pgfpathlineto{\pgfqpoint{6.113131in}{6.037630in}}%
\pgfpathlineto{\pgfqpoint{6.292862in}{6.083787in}}%
\pgfpathlineto{\pgfqpoint{6.472593in}{6.138016in}}%
\pgfpathlineto{\pgfqpoint{6.652324in}{6.177137in}}%
\pgfpathlineto{\pgfqpoint{6.832055in}{6.222561in}}%
\pgfpathlineto{\pgfqpoint{7.011786in}{6.256372in}}%
\pgfpathlineto{\pgfqpoint{7.191517in}{6.291171in}}%
\pgfpathlineto{\pgfqpoint{7.371248in}{6.327728in}}%
\pgfpathlineto{\pgfqpoint{7.550979in}{6.351095in}}%
\pgfpathlineto{\pgfqpoint{7.730710in}{6.377424in}}%
\pgfpathlineto{\pgfqpoint{7.910441in}{6.405759in}}%
\pgfpathlineto{\pgfqpoint{8.090172in}{6.427561in}}%
\pgfpathlineto{\pgfqpoint{8.269903in}{6.450990in}}%
\pgfpathlineto{\pgfqpoint{8.449634in}{6.472084in}}%
\pgfpathlineto{\pgfqpoint{8.629365in}{6.491545in}}%
\pgfpathlineto{\pgfqpoint{8.809096in}{6.511776in}}%
\pgfpathlineto{\pgfqpoint{8.988827in}{6.530474in}}%
\pgfpathlineto{\pgfqpoint{9.168558in}{6.545656in}}%
\pgfpathlineto{\pgfqpoint{9.348289in}{6.564708in}}%
\pgfpathlineto{\pgfqpoint{9.528020in}{6.576997in}}%
\pgfpathlineto{\pgfqpoint{9.707751in}{6.591416in}}%
\pgfpathlineto{\pgfqpoint{9.887482in}{6.607548in}}%
\pgfusepath{stroke}%
\end{pgfscope}%
\begin{pgfscope}%
\pgfsetrectcap%
\pgfsetmiterjoin%
\pgfsetlinewidth{0.803000pt}%
\definecolor{currentstroke}{rgb}{0.000000,0.000000,0.000000}%
\pgfsetstrokecolor{currentstroke}%
\pgfsetdash{}{0pt}%
\pgfpathmoveto{\pgfqpoint{0.640323in}{5.147436in}}%
\pgfpathlineto{\pgfqpoint{0.640323in}{8.997436in}}%
\pgfusepath{stroke}%
\end{pgfscope}%
\begin{pgfscope}%
\pgfsetrectcap%
\pgfsetmiterjoin%
\pgfsetlinewidth{0.803000pt}%
\definecolor{currentstroke}{rgb}{0.000000,0.000000,0.000000}%
\pgfsetstrokecolor{currentstroke}%
\pgfsetdash{}{0pt}%
\pgfpathmoveto{\pgfqpoint{10.327822in}{5.147436in}}%
\pgfpathlineto{\pgfqpoint{10.327822in}{8.997436in}}%
\pgfusepath{stroke}%
\end{pgfscope}%
\begin{pgfscope}%
\pgfsetrectcap%
\pgfsetmiterjoin%
\pgfsetlinewidth{0.803000pt}%
\definecolor{currentstroke}{rgb}{0.000000,0.000000,0.000000}%
\pgfsetstrokecolor{currentstroke}%
\pgfsetdash{}{0pt}%
\pgfpathmoveto{\pgfqpoint{0.640322in}{5.147436in}}%
\pgfpathlineto{\pgfqpoint{10.327823in}{5.147436in}}%
\pgfusepath{stroke}%
\end{pgfscope}%
\begin{pgfscope}%
\pgfsetrectcap%
\pgfsetmiterjoin%
\pgfsetlinewidth{0.803000pt}%
\definecolor{currentstroke}{rgb}{0.000000,0.000000,0.000000}%
\pgfsetstrokecolor{currentstroke}%
\pgfsetdash{}{0pt}%
\pgfpathmoveto{\pgfqpoint{0.640322in}{8.997436in}}%
\pgfpathlineto{\pgfqpoint{10.327823in}{8.997436in}}%
\pgfusepath{stroke}%
\end{pgfscope}%
\begin{pgfscope}%
\definecolor{textcolor}{rgb}{0.000000,0.000000,0.000000}%
\pgfsetstrokecolor{textcolor}%
\pgfsetfillcolor{textcolor}%
\pgftext[x=5.484072in,y=9.080769in,,base]{\color{textcolor}\sffamily\fontsize{12.000000}{14.400000}\selectfont \(\displaystyle \overline{\langle I\rangle}\) over \(\displaystyle p_1\) for \(\displaystyle T=1000\) with \(\displaystyle p_2=0.6\), \(\displaystyle p_3=0.3\)}%
\end{pgfscope}%
\begin{pgfscope}%
\pgfsetbuttcap%
\pgfsetmiterjoin%
\definecolor{currentfill}{rgb}{1.000000,1.000000,1.000000}%
\pgfsetfillcolor{currentfill}%
\pgfsetfillopacity{0.800000}%
\pgfsetlinewidth{1.003750pt}%
\definecolor{currentstroke}{rgb}{0.800000,0.800000,0.800000}%
\pgfsetstrokecolor{currentstroke}%
\pgfsetstrokeopacity{0.800000}%
\pgfsetdash{}{0pt}%
\pgfpathmoveto{\pgfqpoint{0.737545in}{8.070896in}}%
\pgfpathlineto{\pgfqpoint{1.670029in}{8.070896in}}%
\pgfpathquadraticcurveto{\pgfqpoint{1.697806in}{8.070896in}}{\pgfqpoint{1.697806in}{8.098674in}}%
\pgfpathlineto{\pgfqpoint{1.697806in}{8.900214in}}%
\pgfpathquadraticcurveto{\pgfqpoint{1.697806in}{8.927991in}}{\pgfqpoint{1.670029in}{8.927991in}}%
\pgfpathlineto{\pgfqpoint{0.737545in}{8.927991in}}%
\pgfpathquadraticcurveto{\pgfqpoint{0.709767in}{8.927991in}}{\pgfqpoint{0.709767in}{8.900214in}}%
\pgfpathlineto{\pgfqpoint{0.709767in}{8.098674in}}%
\pgfpathquadraticcurveto{\pgfqpoint{0.709767in}{8.070896in}}{\pgfqpoint{0.737545in}{8.070896in}}%
\pgfpathlineto{\pgfqpoint{0.737545in}{8.070896in}}%
\pgfpathclose%
\pgfusepath{stroke,fill}%
\end{pgfscope}%
\begin{pgfscope}%
\pgfsetbuttcap%
\pgfsetroundjoin%
\definecolor{currentfill}{rgb}{0.000000,0.000000,1.000000}%
\pgfsetfillcolor{currentfill}%
\pgfsetfillopacity{0.500000}%
\pgfsetlinewidth{1.003750pt}%
\definecolor{currentstroke}{rgb}{0.000000,0.000000,1.000000}%
\pgfsetstrokecolor{currentstroke}%
\pgfsetstrokeopacity{0.500000}%
\pgfsetdash{}{0pt}%
\pgfsys@defobject{currentmarker}{\pgfqpoint{-0.021960in}{-0.021960in}}{\pgfqpoint{0.021960in}{0.021960in}}{%
\pgfpathmoveto{\pgfqpoint{0.000000in}{-0.021960in}}%
\pgfpathcurveto{\pgfqpoint{0.005824in}{-0.021960in}}{\pgfqpoint{0.011410in}{-0.019646in}}{\pgfqpoint{0.015528in}{-0.015528in}}%
\pgfpathcurveto{\pgfqpoint{0.019646in}{-0.011410in}}{\pgfqpoint{0.021960in}{-0.005824in}}{\pgfqpoint{0.021960in}{0.000000in}}%
\pgfpathcurveto{\pgfqpoint{0.021960in}{0.005824in}}{\pgfqpoint{0.019646in}{0.011410in}}{\pgfqpoint{0.015528in}{0.015528in}}%
\pgfpathcurveto{\pgfqpoint{0.011410in}{0.019646in}}{\pgfqpoint{0.005824in}{0.021960in}}{\pgfqpoint{0.000000in}{0.021960in}}%
\pgfpathcurveto{\pgfqpoint{-0.005824in}{0.021960in}}{\pgfqpoint{-0.011410in}{0.019646in}}{\pgfqpoint{-0.015528in}{0.015528in}}%
\pgfpathcurveto{\pgfqpoint{-0.019646in}{0.011410in}}{\pgfqpoint{-0.021960in}{0.005824in}}{\pgfqpoint{-0.021960in}{0.000000in}}%
\pgfpathcurveto{\pgfqpoint{-0.021960in}{-0.005824in}}{\pgfqpoint{-0.019646in}{-0.011410in}}{\pgfqpoint{-0.015528in}{-0.015528in}}%
\pgfpathcurveto{\pgfqpoint{-0.011410in}{-0.019646in}}{\pgfqpoint{-0.005824in}{-0.021960in}}{\pgfqpoint{0.000000in}{-0.021960in}}%
\pgfpathlineto{\pgfqpoint{0.000000in}{-0.021960in}}%
\pgfpathclose%
\pgfusepath{stroke,fill}%
}%
\begin{pgfscope}%
\pgfsys@transformshift{0.904211in}{8.803371in}%
\pgfsys@useobject{currentmarker}{}%
\end{pgfscope}%
\end{pgfscope}%
\begin{pgfscope}%
\definecolor{textcolor}{rgb}{0.000000,0.000000,0.000000}%
\pgfsetstrokecolor{textcolor}%
\pgfsetfillcolor{textcolor}%
\pgftext[x=1.154211in,y=8.766913in,left,base]{\color{textcolor}\sffamily\fontsize{10.000000}{12.000000}\selectfont \(\displaystyle L=16\)}%
\end{pgfscope}%
\begin{pgfscope}%
\pgfsetbuttcap%
\pgfsetroundjoin%
\definecolor{currentfill}{rgb}{0.980392,0.164706,0.333333}%
\pgfsetfillcolor{currentfill}%
\pgfsetfillopacity{0.500000}%
\pgfsetlinewidth{1.003750pt}%
\definecolor{currentstroke}{rgb}{0.980392,0.164706,0.333333}%
\pgfsetstrokecolor{currentstroke}%
\pgfsetstrokeopacity{0.500000}%
\pgfsetdash{}{0pt}%
\pgfsys@defobject{currentmarker}{\pgfqpoint{-0.021960in}{-0.021960in}}{\pgfqpoint{0.021960in}{0.021960in}}{%
\pgfpathmoveto{\pgfqpoint{0.000000in}{-0.021960in}}%
\pgfpathcurveto{\pgfqpoint{0.005824in}{-0.021960in}}{\pgfqpoint{0.011410in}{-0.019646in}}{\pgfqpoint{0.015528in}{-0.015528in}}%
\pgfpathcurveto{\pgfqpoint{0.019646in}{-0.011410in}}{\pgfqpoint{0.021960in}{-0.005824in}}{\pgfqpoint{0.021960in}{0.000000in}}%
\pgfpathcurveto{\pgfqpoint{0.021960in}{0.005824in}}{\pgfqpoint{0.019646in}{0.011410in}}{\pgfqpoint{0.015528in}{0.015528in}}%
\pgfpathcurveto{\pgfqpoint{0.011410in}{0.019646in}}{\pgfqpoint{0.005824in}{0.021960in}}{\pgfqpoint{0.000000in}{0.021960in}}%
\pgfpathcurveto{\pgfqpoint{-0.005824in}{0.021960in}}{\pgfqpoint{-0.011410in}{0.019646in}}{\pgfqpoint{-0.015528in}{0.015528in}}%
\pgfpathcurveto{\pgfqpoint{-0.019646in}{0.011410in}}{\pgfqpoint{-0.021960in}{0.005824in}}{\pgfqpoint{-0.021960in}{0.000000in}}%
\pgfpathcurveto{\pgfqpoint{-0.021960in}{-0.005824in}}{\pgfqpoint{-0.019646in}{-0.011410in}}{\pgfqpoint{-0.015528in}{-0.015528in}}%
\pgfpathcurveto{\pgfqpoint{-0.011410in}{-0.019646in}}{\pgfqpoint{-0.005824in}{-0.021960in}}{\pgfqpoint{0.000000in}{-0.021960in}}%
\pgfpathlineto{\pgfqpoint{0.000000in}{-0.021960in}}%
\pgfpathclose%
\pgfusepath{stroke,fill}%
}%
\begin{pgfscope}%
\pgfsys@transformshift{0.904211in}{8.599514in}%
\pgfsys@useobject{currentmarker}{}%
\end{pgfscope}%
\end{pgfscope}%
\begin{pgfscope}%
\definecolor{textcolor}{rgb}{0.000000,0.000000,0.000000}%
\pgfsetstrokecolor{textcolor}%
\pgfsetfillcolor{textcolor}%
\pgftext[x=1.154211in,y=8.563056in,left,base]{\color{textcolor}\sffamily\fontsize{10.000000}{12.000000}\selectfont \(\displaystyle L=32\)}%
\end{pgfscope}%
\begin{pgfscope}%
\pgfsetbuttcap%
\pgfsetroundjoin%
\definecolor{currentfill}{rgb}{0.239216,0.478431,0.992157}%
\pgfsetfillcolor{currentfill}%
\pgfsetfillopacity{0.500000}%
\pgfsetlinewidth{1.003750pt}%
\definecolor{currentstroke}{rgb}{0.239216,0.478431,0.992157}%
\pgfsetstrokecolor{currentstroke}%
\pgfsetstrokeopacity{0.500000}%
\pgfsetdash{}{0pt}%
\pgfsys@defobject{currentmarker}{\pgfqpoint{-0.021960in}{-0.021960in}}{\pgfqpoint{0.021960in}{0.021960in}}{%
\pgfpathmoveto{\pgfqpoint{0.000000in}{-0.021960in}}%
\pgfpathcurveto{\pgfqpoint{0.005824in}{-0.021960in}}{\pgfqpoint{0.011410in}{-0.019646in}}{\pgfqpoint{0.015528in}{-0.015528in}}%
\pgfpathcurveto{\pgfqpoint{0.019646in}{-0.011410in}}{\pgfqpoint{0.021960in}{-0.005824in}}{\pgfqpoint{0.021960in}{0.000000in}}%
\pgfpathcurveto{\pgfqpoint{0.021960in}{0.005824in}}{\pgfqpoint{0.019646in}{0.011410in}}{\pgfqpoint{0.015528in}{0.015528in}}%
\pgfpathcurveto{\pgfqpoint{0.011410in}{0.019646in}}{\pgfqpoint{0.005824in}{0.021960in}}{\pgfqpoint{0.000000in}{0.021960in}}%
\pgfpathcurveto{\pgfqpoint{-0.005824in}{0.021960in}}{\pgfqpoint{-0.011410in}{0.019646in}}{\pgfqpoint{-0.015528in}{0.015528in}}%
\pgfpathcurveto{\pgfqpoint{-0.019646in}{0.011410in}}{\pgfqpoint{-0.021960in}{0.005824in}}{\pgfqpoint{-0.021960in}{0.000000in}}%
\pgfpathcurveto{\pgfqpoint{-0.021960in}{-0.005824in}}{\pgfqpoint{-0.019646in}{-0.011410in}}{\pgfqpoint{-0.015528in}{-0.015528in}}%
\pgfpathcurveto{\pgfqpoint{-0.011410in}{-0.019646in}}{\pgfqpoint{-0.005824in}{-0.021960in}}{\pgfqpoint{0.000000in}{-0.021960in}}%
\pgfpathlineto{\pgfqpoint{0.000000in}{-0.021960in}}%
\pgfpathclose%
\pgfusepath{stroke,fill}%
}%
\begin{pgfscope}%
\pgfsys@transformshift{0.904211in}{8.395657in}%
\pgfsys@useobject{currentmarker}{}%
\end{pgfscope}%
\end{pgfscope}%
\begin{pgfscope}%
\definecolor{textcolor}{rgb}{0.000000,0.000000,0.000000}%
\pgfsetstrokecolor{textcolor}%
\pgfsetfillcolor{textcolor}%
\pgftext[x=1.154211in,y=8.359198in,left,base]{\color{textcolor}\sffamily\fontsize{10.000000}{12.000000}\selectfont \(\displaystyle L=64\)}%
\end{pgfscope}%
\begin{pgfscope}%
\pgfsetbuttcap%
\pgfsetroundjoin%
\definecolor{currentfill}{rgb}{0.000000,0.000000,0.000000}%
\pgfsetfillcolor{currentfill}%
\pgfsetfillopacity{0.500000}%
\pgfsetlinewidth{1.003750pt}%
\definecolor{currentstroke}{rgb}{0.000000,0.000000,0.000000}%
\pgfsetstrokecolor{currentstroke}%
\pgfsetstrokeopacity{0.500000}%
\pgfsetdash{}{0pt}%
\pgfsys@defobject{currentmarker}{\pgfqpoint{-0.021960in}{-0.021960in}}{\pgfqpoint{0.021960in}{0.021960in}}{%
\pgfpathmoveto{\pgfqpoint{0.000000in}{-0.021960in}}%
\pgfpathcurveto{\pgfqpoint{0.005824in}{-0.021960in}}{\pgfqpoint{0.011410in}{-0.019646in}}{\pgfqpoint{0.015528in}{-0.015528in}}%
\pgfpathcurveto{\pgfqpoint{0.019646in}{-0.011410in}}{\pgfqpoint{0.021960in}{-0.005824in}}{\pgfqpoint{0.021960in}{0.000000in}}%
\pgfpathcurveto{\pgfqpoint{0.021960in}{0.005824in}}{\pgfqpoint{0.019646in}{0.011410in}}{\pgfqpoint{0.015528in}{0.015528in}}%
\pgfpathcurveto{\pgfqpoint{0.011410in}{0.019646in}}{\pgfqpoint{0.005824in}{0.021960in}}{\pgfqpoint{0.000000in}{0.021960in}}%
\pgfpathcurveto{\pgfqpoint{-0.005824in}{0.021960in}}{\pgfqpoint{-0.011410in}{0.019646in}}{\pgfqpoint{-0.015528in}{0.015528in}}%
\pgfpathcurveto{\pgfqpoint{-0.019646in}{0.011410in}}{\pgfqpoint{-0.021960in}{0.005824in}}{\pgfqpoint{-0.021960in}{0.000000in}}%
\pgfpathcurveto{\pgfqpoint{-0.021960in}{-0.005824in}}{\pgfqpoint{-0.019646in}{-0.011410in}}{\pgfqpoint{-0.015528in}{-0.015528in}}%
\pgfpathcurveto{\pgfqpoint{-0.011410in}{-0.019646in}}{\pgfqpoint{-0.005824in}{-0.021960in}}{\pgfqpoint{0.000000in}{-0.021960in}}%
\pgfpathlineto{\pgfqpoint{0.000000in}{-0.021960in}}%
\pgfpathclose%
\pgfusepath{stroke,fill}%
}%
\begin{pgfscope}%
\pgfsys@transformshift{0.904211in}{8.191799in}%
\pgfsys@useobject{currentmarker}{}%
\end{pgfscope}%
\end{pgfscope}%
\begin{pgfscope}%
\definecolor{textcolor}{rgb}{0.000000,0.000000,0.000000}%
\pgfsetstrokecolor{textcolor}%
\pgfsetfillcolor{textcolor}%
\pgftext[x=1.154211in,y=8.155341in,left,base]{\color{textcolor}\sffamily\fontsize{10.000000}{12.000000}\selectfont \(\displaystyle L=128\)}%
\end{pgfscope}%
\begin{pgfscope}%
\pgfsetbuttcap%
\pgfsetmiterjoin%
\definecolor{currentfill}{rgb}{1.000000,1.000000,1.000000}%
\pgfsetfillcolor{currentfill}%
\pgfsetlinewidth{0.000000pt}%
\definecolor{currentstroke}{rgb}{0.000000,0.000000,0.000000}%
\pgfsetstrokecolor{currentstroke}%
\pgfsetstrokeopacity{0.000000}%
\pgfsetdash{}{0pt}%
\pgfpathmoveto{\pgfqpoint{0.640323in}{0.527436in}}%
\pgfpathlineto{\pgfqpoint{10.327822in}{0.527436in}}%
\pgfpathlineto{\pgfqpoint{10.327822in}{4.377436in}}%
\pgfpathlineto{\pgfqpoint{0.640323in}{4.377436in}}%
\pgfpathlineto{\pgfqpoint{0.640323in}{0.527436in}}%
\pgfpathclose%
\pgfusepath{fill}%
\end{pgfscope}%
\begin{pgfscope}%
\pgfpathrectangle{\pgfqpoint{0.640323in}{0.527436in}}{\pgfqpoint{9.687500in}{3.850000in}}%
\pgfusepath{clip}%
\pgfsetbuttcap%
\pgfsetroundjoin%
\definecolor{currentfill}{rgb}{0.000000,0.000000,1.000000}%
\pgfsetfillcolor{currentfill}%
\pgfsetfillopacity{0.500000}%
\pgfsetlinewidth{1.003750pt}%
\definecolor{currentstroke}{rgb}{0.000000,0.000000,1.000000}%
\pgfsetstrokecolor{currentstroke}%
\pgfsetstrokeopacity{0.500000}%
\pgfsetdash{{3.700000pt}{1.600000pt}}{0.000000pt}%
\pgfpathmoveto{\pgfqpoint{1.080663in}{0.637261in}}%
\pgfpathcurveto{\pgfqpoint{1.086487in}{0.637261in}}{\pgfqpoint{1.092074in}{0.639575in}}{\pgfqpoint{1.096192in}{0.643693in}}%
\pgfpathcurveto{\pgfqpoint{1.100310in}{0.647812in}}{\pgfqpoint{1.102624in}{0.653398in}}{\pgfqpoint{1.102624in}{0.659222in}}%
\pgfpathcurveto{\pgfqpoint{1.102624in}{0.665046in}}{\pgfqpoint{1.100310in}{0.670632in}}{\pgfqpoint{1.096192in}{0.674750in}}%
\pgfpathcurveto{\pgfqpoint{1.092074in}{0.678868in}}{\pgfqpoint{1.086487in}{0.681182in}}{\pgfqpoint{1.080663in}{0.681182in}}%
\pgfpathcurveto{\pgfqpoint{1.074839in}{0.681182in}}{\pgfqpoint{1.069253in}{0.678868in}}{\pgfqpoint{1.065135in}{0.674750in}}%
\pgfpathcurveto{\pgfqpoint{1.061017in}{0.670632in}}{\pgfqpoint{1.058703in}{0.665046in}}{\pgfqpoint{1.058703in}{0.659222in}}%
\pgfpathcurveto{\pgfqpoint{1.058703in}{0.653398in}}{\pgfqpoint{1.061017in}{0.647812in}}{\pgfqpoint{1.065135in}{0.643693in}}%
\pgfpathcurveto{\pgfqpoint{1.069253in}{0.639575in}}{\pgfqpoint{1.074839in}{0.637261in}}{\pgfqpoint{1.080663in}{0.637261in}}%
\pgfpathlineto{\pgfqpoint{1.080663in}{0.637261in}}%
\pgfpathclose%
\pgfusepath{stroke,fill}%
\end{pgfscope}%
\begin{pgfscope}%
\pgfpathrectangle{\pgfqpoint{0.640323in}{0.527436in}}{\pgfqpoint{9.687500in}{3.850000in}}%
\pgfusepath{clip}%
\pgfsetbuttcap%
\pgfsetroundjoin%
\definecolor{currentfill}{rgb}{0.000000,0.000000,1.000000}%
\pgfsetfillcolor{currentfill}%
\pgfsetfillopacity{0.500000}%
\pgfsetlinewidth{1.003750pt}%
\definecolor{currentstroke}{rgb}{0.000000,0.000000,1.000000}%
\pgfsetstrokecolor{currentstroke}%
\pgfsetstrokeopacity{0.500000}%
\pgfsetdash{{3.700000pt}{1.600000pt}}{0.000000pt}%
\pgfpathmoveto{\pgfqpoint{1.260394in}{0.639857in}}%
\pgfpathcurveto{\pgfqpoint{1.266218in}{0.639857in}}{\pgfqpoint{1.271805in}{0.642171in}}{\pgfqpoint{1.275923in}{0.646289in}}%
\pgfpathcurveto{\pgfqpoint{1.280041in}{0.650407in}}{\pgfqpoint{1.282355in}{0.655993in}}{\pgfqpoint{1.282355in}{0.661817in}}%
\pgfpathcurveto{\pgfqpoint{1.282355in}{0.667641in}}{\pgfqpoint{1.280041in}{0.673227in}}{\pgfqpoint{1.275923in}{0.677345in}}%
\pgfpathcurveto{\pgfqpoint{1.271805in}{0.681463in}}{\pgfqpoint{1.266218in}{0.683777in}}{\pgfqpoint{1.260394in}{0.683777in}}%
\pgfpathcurveto{\pgfqpoint{1.254570in}{0.683777in}}{\pgfqpoint{1.248984in}{0.681463in}}{\pgfqpoint{1.244866in}{0.677345in}}%
\pgfpathcurveto{\pgfqpoint{1.240748in}{0.673227in}}{\pgfqpoint{1.238434in}{0.667641in}}{\pgfqpoint{1.238434in}{0.661817in}}%
\pgfpathcurveto{\pgfqpoint{1.238434in}{0.655993in}}{\pgfqpoint{1.240748in}{0.650407in}}{\pgfqpoint{1.244866in}{0.646289in}}%
\pgfpathcurveto{\pgfqpoint{1.248984in}{0.642171in}}{\pgfqpoint{1.254570in}{0.639857in}}{\pgfqpoint{1.260394in}{0.639857in}}%
\pgfpathlineto{\pgfqpoint{1.260394in}{0.639857in}}%
\pgfpathclose%
\pgfusepath{stroke,fill}%
\end{pgfscope}%
\begin{pgfscope}%
\pgfpathrectangle{\pgfqpoint{0.640323in}{0.527436in}}{\pgfqpoint{9.687500in}{3.850000in}}%
\pgfusepath{clip}%
\pgfsetbuttcap%
\pgfsetroundjoin%
\definecolor{currentfill}{rgb}{0.000000,0.000000,1.000000}%
\pgfsetfillcolor{currentfill}%
\pgfsetfillopacity{0.500000}%
\pgfsetlinewidth{1.003750pt}%
\definecolor{currentstroke}{rgb}{0.000000,0.000000,1.000000}%
\pgfsetstrokecolor{currentstroke}%
\pgfsetstrokeopacity{0.500000}%
\pgfsetdash{{3.700000pt}{1.600000pt}}{0.000000pt}%
\pgfpathmoveto{\pgfqpoint{1.440125in}{0.641991in}}%
\pgfpathcurveto{\pgfqpoint{1.445949in}{0.641991in}}{\pgfqpoint{1.451535in}{0.644305in}}{\pgfqpoint{1.455654in}{0.648423in}}%
\pgfpathcurveto{\pgfqpoint{1.459772in}{0.652542in}}{\pgfqpoint{1.462086in}{0.658128in}}{\pgfqpoint{1.462086in}{0.663952in}}%
\pgfpathcurveto{\pgfqpoint{1.462086in}{0.669776in}}{\pgfqpoint{1.459772in}{0.675362in}}{\pgfqpoint{1.455654in}{0.679480in}}%
\pgfpathcurveto{\pgfqpoint{1.451535in}{0.683598in}}{\pgfqpoint{1.445949in}{0.685912in}}{\pgfqpoint{1.440125in}{0.685912in}}%
\pgfpathcurveto{\pgfqpoint{1.434301in}{0.685912in}}{\pgfqpoint{1.428715in}{0.683598in}}{\pgfqpoint{1.424597in}{0.679480in}}%
\pgfpathcurveto{\pgfqpoint{1.420479in}{0.675362in}}{\pgfqpoint{1.418165in}{0.669776in}}{\pgfqpoint{1.418165in}{0.663952in}}%
\pgfpathcurveto{\pgfqpoint{1.418165in}{0.658128in}}{\pgfqpoint{1.420479in}{0.652542in}}{\pgfqpoint{1.424597in}{0.648423in}}%
\pgfpathcurveto{\pgfqpoint{1.428715in}{0.644305in}}{\pgfqpoint{1.434301in}{0.641991in}}{\pgfqpoint{1.440125in}{0.641991in}}%
\pgfpathlineto{\pgfqpoint{1.440125in}{0.641991in}}%
\pgfpathclose%
\pgfusepath{stroke,fill}%
\end{pgfscope}%
\begin{pgfscope}%
\pgfpathrectangle{\pgfqpoint{0.640323in}{0.527436in}}{\pgfqpoint{9.687500in}{3.850000in}}%
\pgfusepath{clip}%
\pgfsetbuttcap%
\pgfsetroundjoin%
\definecolor{currentfill}{rgb}{0.000000,0.000000,1.000000}%
\pgfsetfillcolor{currentfill}%
\pgfsetfillopacity{0.500000}%
\pgfsetlinewidth{1.003750pt}%
\definecolor{currentstroke}{rgb}{0.000000,0.000000,1.000000}%
\pgfsetstrokecolor{currentstroke}%
\pgfsetstrokeopacity{0.500000}%
\pgfsetdash{{3.700000pt}{1.600000pt}}{0.000000pt}%
\pgfpathmoveto{\pgfqpoint{1.619856in}{0.641167in}}%
\pgfpathcurveto{\pgfqpoint{1.625680in}{0.641167in}}{\pgfqpoint{1.631266in}{0.643481in}}{\pgfqpoint{1.635385in}{0.647599in}}%
\pgfpathcurveto{\pgfqpoint{1.639503in}{0.651717in}}{\pgfqpoint{1.641817in}{0.657303in}}{\pgfqpoint{1.641817in}{0.663127in}}%
\pgfpathcurveto{\pgfqpoint{1.641817in}{0.668951in}}{\pgfqpoint{1.639503in}{0.674537in}}{\pgfqpoint{1.635385in}{0.678655in}}%
\pgfpathcurveto{\pgfqpoint{1.631266in}{0.682773in}}{\pgfqpoint{1.625680in}{0.685087in}}{\pgfqpoint{1.619856in}{0.685087in}}%
\pgfpathcurveto{\pgfqpoint{1.614032in}{0.685087in}}{\pgfqpoint{1.608446in}{0.682773in}}{\pgfqpoint{1.604328in}{0.678655in}}%
\pgfpathcurveto{\pgfqpoint{1.600210in}{0.674537in}}{\pgfqpoint{1.597896in}{0.668951in}}{\pgfqpoint{1.597896in}{0.663127in}}%
\pgfpathcurveto{\pgfqpoint{1.597896in}{0.657303in}}{\pgfqpoint{1.600210in}{0.651717in}}{\pgfqpoint{1.604328in}{0.647599in}}%
\pgfpathcurveto{\pgfqpoint{1.608446in}{0.643481in}}{\pgfqpoint{1.614032in}{0.641167in}}{\pgfqpoint{1.619856in}{0.641167in}}%
\pgfpathlineto{\pgfqpoint{1.619856in}{0.641167in}}%
\pgfpathclose%
\pgfusepath{stroke,fill}%
\end{pgfscope}%
\begin{pgfscope}%
\pgfpathrectangle{\pgfqpoint{0.640323in}{0.527436in}}{\pgfqpoint{9.687500in}{3.850000in}}%
\pgfusepath{clip}%
\pgfsetbuttcap%
\pgfsetroundjoin%
\definecolor{currentfill}{rgb}{0.000000,0.000000,1.000000}%
\pgfsetfillcolor{currentfill}%
\pgfsetfillopacity{0.500000}%
\pgfsetlinewidth{1.003750pt}%
\definecolor{currentstroke}{rgb}{0.000000,0.000000,1.000000}%
\pgfsetstrokecolor{currentstroke}%
\pgfsetstrokeopacity{0.500000}%
\pgfsetdash{{3.700000pt}{1.600000pt}}{0.000000pt}%
\pgfpathmoveto{\pgfqpoint{1.799587in}{0.640027in}}%
\pgfpathcurveto{\pgfqpoint{1.805411in}{0.640027in}}{\pgfqpoint{1.810997in}{0.642341in}}{\pgfqpoint{1.815116in}{0.646459in}}%
\pgfpathcurveto{\pgfqpoint{1.819234in}{0.650577in}}{\pgfqpoint{1.821548in}{0.656163in}}{\pgfqpoint{1.821548in}{0.661987in}}%
\pgfpathcurveto{\pgfqpoint{1.821548in}{0.667811in}}{\pgfqpoint{1.819234in}{0.673397in}}{\pgfqpoint{1.815116in}{0.677515in}}%
\pgfpathcurveto{\pgfqpoint{1.810997in}{0.681633in}}{\pgfqpoint{1.805411in}{0.683947in}}{\pgfqpoint{1.799587in}{0.683947in}}%
\pgfpathcurveto{\pgfqpoint{1.793763in}{0.683947in}}{\pgfqpoint{1.788177in}{0.681633in}}{\pgfqpoint{1.784059in}{0.677515in}}%
\pgfpathcurveto{\pgfqpoint{1.779941in}{0.673397in}}{\pgfqpoint{1.777627in}{0.667811in}}{\pgfqpoint{1.777627in}{0.661987in}}%
\pgfpathcurveto{\pgfqpoint{1.777627in}{0.656163in}}{\pgfqpoint{1.779941in}{0.650577in}}{\pgfqpoint{1.784059in}{0.646459in}}%
\pgfpathcurveto{\pgfqpoint{1.788177in}{0.642341in}}{\pgfqpoint{1.793763in}{0.640027in}}{\pgfqpoint{1.799587in}{0.640027in}}%
\pgfpathlineto{\pgfqpoint{1.799587in}{0.640027in}}%
\pgfpathclose%
\pgfusepath{stroke,fill}%
\end{pgfscope}%
\begin{pgfscope}%
\pgfpathrectangle{\pgfqpoint{0.640323in}{0.527436in}}{\pgfqpoint{9.687500in}{3.850000in}}%
\pgfusepath{clip}%
\pgfsetbuttcap%
\pgfsetroundjoin%
\definecolor{currentfill}{rgb}{0.000000,0.000000,1.000000}%
\pgfsetfillcolor{currentfill}%
\pgfsetfillopacity{0.500000}%
\pgfsetlinewidth{1.003750pt}%
\definecolor{currentstroke}{rgb}{0.000000,0.000000,1.000000}%
\pgfsetstrokecolor{currentstroke}%
\pgfsetstrokeopacity{0.500000}%
\pgfsetdash{{3.700000pt}{1.600000pt}}{0.000000pt}%
\pgfpathmoveto{\pgfqpoint{1.979318in}{0.648128in}}%
\pgfpathcurveto{\pgfqpoint{1.985142in}{0.648128in}}{\pgfqpoint{1.990728in}{0.650442in}}{\pgfqpoint{1.994847in}{0.654560in}}%
\pgfpathcurveto{\pgfqpoint{1.998965in}{0.658679in}}{\pgfqpoint{2.001279in}{0.664265in}}{\pgfqpoint{2.001279in}{0.670089in}}%
\pgfpathcurveto{\pgfqpoint{2.001279in}{0.675913in}}{\pgfqpoint{1.998965in}{0.681499in}}{\pgfqpoint{1.994847in}{0.685617in}}%
\pgfpathcurveto{\pgfqpoint{1.990728in}{0.689735in}}{\pgfqpoint{1.985142in}{0.692049in}}{\pgfqpoint{1.979318in}{0.692049in}}%
\pgfpathcurveto{\pgfqpoint{1.973494in}{0.692049in}}{\pgfqpoint{1.967908in}{0.689735in}}{\pgfqpoint{1.963790in}{0.685617in}}%
\pgfpathcurveto{\pgfqpoint{1.959672in}{0.681499in}}{\pgfqpoint{1.957358in}{0.675913in}}{\pgfqpoint{1.957358in}{0.670089in}}%
\pgfpathcurveto{\pgfqpoint{1.957358in}{0.664265in}}{\pgfqpoint{1.959672in}{0.658679in}}{\pgfqpoint{1.963790in}{0.654560in}}%
\pgfpathcurveto{\pgfqpoint{1.967908in}{0.650442in}}{\pgfqpoint{1.973494in}{0.648128in}}{\pgfqpoint{1.979318in}{0.648128in}}%
\pgfpathlineto{\pgfqpoint{1.979318in}{0.648128in}}%
\pgfpathclose%
\pgfusepath{stroke,fill}%
\end{pgfscope}%
\begin{pgfscope}%
\pgfpathrectangle{\pgfqpoint{0.640323in}{0.527436in}}{\pgfqpoint{9.687500in}{3.850000in}}%
\pgfusepath{clip}%
\pgfsetbuttcap%
\pgfsetroundjoin%
\definecolor{currentfill}{rgb}{0.000000,0.000000,1.000000}%
\pgfsetfillcolor{currentfill}%
\pgfsetfillopacity{0.500000}%
\pgfsetlinewidth{1.003750pt}%
\definecolor{currentstroke}{rgb}{0.000000,0.000000,1.000000}%
\pgfsetstrokecolor{currentstroke}%
\pgfsetstrokeopacity{0.500000}%
\pgfsetdash{{3.700000pt}{1.600000pt}}{0.000000pt}%
\pgfpathmoveto{\pgfqpoint{2.159049in}{0.661033in}}%
\pgfpathcurveto{\pgfqpoint{2.164873in}{0.661033in}}{\pgfqpoint{2.170459in}{0.663347in}}{\pgfqpoint{2.174578in}{0.667465in}}%
\pgfpathcurveto{\pgfqpoint{2.178696in}{0.671583in}}{\pgfqpoint{2.181010in}{0.677169in}}{\pgfqpoint{2.181010in}{0.682993in}}%
\pgfpathcurveto{\pgfqpoint{2.181010in}{0.688817in}}{\pgfqpoint{2.178696in}{0.694403in}}{\pgfqpoint{2.174578in}{0.698521in}}%
\pgfpathcurveto{\pgfqpoint{2.170459in}{0.702639in}}{\pgfqpoint{2.164873in}{0.704953in}}{\pgfqpoint{2.159049in}{0.704953in}}%
\pgfpathcurveto{\pgfqpoint{2.153225in}{0.704953in}}{\pgfqpoint{2.147639in}{0.702639in}}{\pgfqpoint{2.143521in}{0.698521in}}%
\pgfpathcurveto{\pgfqpoint{2.139403in}{0.694403in}}{\pgfqpoint{2.137089in}{0.688817in}}{\pgfqpoint{2.137089in}{0.682993in}}%
\pgfpathcurveto{\pgfqpoint{2.137089in}{0.677169in}}{\pgfqpoint{2.139403in}{0.671583in}}{\pgfqpoint{2.143521in}{0.667465in}}%
\pgfpathcurveto{\pgfqpoint{2.147639in}{0.663347in}}{\pgfqpoint{2.153225in}{0.661033in}}{\pgfqpoint{2.159049in}{0.661033in}}%
\pgfpathlineto{\pgfqpoint{2.159049in}{0.661033in}}%
\pgfpathclose%
\pgfusepath{stroke,fill}%
\end{pgfscope}%
\begin{pgfscope}%
\pgfpathrectangle{\pgfqpoint{0.640323in}{0.527436in}}{\pgfqpoint{9.687500in}{3.850000in}}%
\pgfusepath{clip}%
\pgfsetbuttcap%
\pgfsetroundjoin%
\definecolor{currentfill}{rgb}{0.000000,0.000000,1.000000}%
\pgfsetfillcolor{currentfill}%
\pgfsetfillopacity{0.500000}%
\pgfsetlinewidth{1.003750pt}%
\definecolor{currentstroke}{rgb}{0.000000,0.000000,1.000000}%
\pgfsetstrokecolor{currentstroke}%
\pgfsetstrokeopacity{0.500000}%
\pgfsetdash{{3.700000pt}{1.600000pt}}{0.000000pt}%
\pgfpathmoveto{\pgfqpoint{2.338780in}{0.663507in}}%
\pgfpathcurveto{\pgfqpoint{2.344604in}{0.663507in}}{\pgfqpoint{2.350190in}{0.665821in}}{\pgfqpoint{2.354309in}{0.669939in}}%
\pgfpathcurveto{\pgfqpoint{2.358427in}{0.674057in}}{\pgfqpoint{2.360741in}{0.679643in}}{\pgfqpoint{2.360741in}{0.685467in}}%
\pgfpathcurveto{\pgfqpoint{2.360741in}{0.691291in}}{\pgfqpoint{2.358427in}{0.696877in}}{\pgfqpoint{2.354309in}{0.700996in}}%
\pgfpathcurveto{\pgfqpoint{2.350190in}{0.705114in}}{\pgfqpoint{2.344604in}{0.707428in}}{\pgfqpoint{2.338780in}{0.707428in}}%
\pgfpathcurveto{\pgfqpoint{2.332956in}{0.707428in}}{\pgfqpoint{2.327370in}{0.705114in}}{\pgfqpoint{2.323252in}{0.700996in}}%
\pgfpathcurveto{\pgfqpoint{2.319134in}{0.696877in}}{\pgfqpoint{2.316820in}{0.691291in}}{\pgfqpoint{2.316820in}{0.685467in}}%
\pgfpathcurveto{\pgfqpoint{2.316820in}{0.679643in}}{\pgfqpoint{2.319134in}{0.674057in}}{\pgfqpoint{2.323252in}{0.669939in}}%
\pgfpathcurveto{\pgfqpoint{2.327370in}{0.665821in}}{\pgfqpoint{2.332956in}{0.663507in}}{\pgfqpoint{2.338780in}{0.663507in}}%
\pgfpathlineto{\pgfqpoint{2.338780in}{0.663507in}}%
\pgfpathclose%
\pgfusepath{stroke,fill}%
\end{pgfscope}%
\begin{pgfscope}%
\pgfpathrectangle{\pgfqpoint{0.640323in}{0.527436in}}{\pgfqpoint{9.687500in}{3.850000in}}%
\pgfusepath{clip}%
\pgfsetbuttcap%
\pgfsetroundjoin%
\definecolor{currentfill}{rgb}{0.000000,0.000000,1.000000}%
\pgfsetfillcolor{currentfill}%
\pgfsetfillopacity{0.500000}%
\pgfsetlinewidth{1.003750pt}%
\definecolor{currentstroke}{rgb}{0.000000,0.000000,1.000000}%
\pgfsetstrokecolor{currentstroke}%
\pgfsetstrokeopacity{0.500000}%
\pgfsetdash{{3.700000pt}{1.600000pt}}{0.000000pt}%
\pgfpathmoveto{\pgfqpoint{2.518511in}{0.666054in}}%
\pgfpathcurveto{\pgfqpoint{2.524335in}{0.666054in}}{\pgfqpoint{2.529921in}{0.668368in}}{\pgfqpoint{2.534040in}{0.672486in}}%
\pgfpathcurveto{\pgfqpoint{2.538158in}{0.676604in}}{\pgfqpoint{2.540472in}{0.682190in}}{\pgfqpoint{2.540472in}{0.688014in}}%
\pgfpathcurveto{\pgfqpoint{2.540472in}{0.693838in}}{\pgfqpoint{2.538158in}{0.699424in}}{\pgfqpoint{2.534040in}{0.703542in}}%
\pgfpathcurveto{\pgfqpoint{2.529921in}{0.707661in}}{\pgfqpoint{2.524335in}{0.709975in}}{\pgfqpoint{2.518511in}{0.709975in}}%
\pgfpathcurveto{\pgfqpoint{2.512687in}{0.709975in}}{\pgfqpoint{2.507101in}{0.707661in}}{\pgfqpoint{2.502983in}{0.703542in}}%
\pgfpathcurveto{\pgfqpoint{2.498865in}{0.699424in}}{\pgfqpoint{2.496551in}{0.693838in}}{\pgfqpoint{2.496551in}{0.688014in}}%
\pgfpathcurveto{\pgfqpoint{2.496551in}{0.682190in}}{\pgfqpoint{2.498865in}{0.676604in}}{\pgfqpoint{2.502983in}{0.672486in}}%
\pgfpathcurveto{\pgfqpoint{2.507101in}{0.668368in}}{\pgfqpoint{2.512687in}{0.666054in}}{\pgfqpoint{2.518511in}{0.666054in}}%
\pgfpathlineto{\pgfqpoint{2.518511in}{0.666054in}}%
\pgfpathclose%
\pgfusepath{stroke,fill}%
\end{pgfscope}%
\begin{pgfscope}%
\pgfpathrectangle{\pgfqpoint{0.640323in}{0.527436in}}{\pgfqpoint{9.687500in}{3.850000in}}%
\pgfusepath{clip}%
\pgfsetbuttcap%
\pgfsetroundjoin%
\definecolor{currentfill}{rgb}{0.000000,0.000000,1.000000}%
\pgfsetfillcolor{currentfill}%
\pgfsetfillopacity{0.500000}%
\pgfsetlinewidth{1.003750pt}%
\definecolor{currentstroke}{rgb}{0.000000,0.000000,1.000000}%
\pgfsetstrokecolor{currentstroke}%
\pgfsetstrokeopacity{0.500000}%
\pgfsetdash{{3.700000pt}{1.600000pt}}{0.000000pt}%
\pgfpathmoveto{\pgfqpoint{2.698242in}{0.674398in}}%
\pgfpathcurveto{\pgfqpoint{2.704066in}{0.674398in}}{\pgfqpoint{2.709652in}{0.676712in}}{\pgfqpoint{2.713771in}{0.680830in}}%
\pgfpathcurveto{\pgfqpoint{2.717889in}{0.684948in}}{\pgfqpoint{2.720203in}{0.690535in}}{\pgfqpoint{2.720203in}{0.696358in}}%
\pgfpathcurveto{\pgfqpoint{2.720203in}{0.702182in}}{\pgfqpoint{2.717889in}{0.707769in}}{\pgfqpoint{2.713771in}{0.711887in}}%
\pgfpathcurveto{\pgfqpoint{2.709652in}{0.716005in}}{\pgfqpoint{2.704066in}{0.718319in}}{\pgfqpoint{2.698242in}{0.718319in}}%
\pgfpathcurveto{\pgfqpoint{2.692418in}{0.718319in}}{\pgfqpoint{2.686832in}{0.716005in}}{\pgfqpoint{2.682714in}{0.711887in}}%
\pgfpathcurveto{\pgfqpoint{2.678596in}{0.707769in}}{\pgfqpoint{2.676282in}{0.702182in}}{\pgfqpoint{2.676282in}{0.696358in}}%
\pgfpathcurveto{\pgfqpoint{2.676282in}{0.690535in}}{\pgfqpoint{2.678596in}{0.684948in}}{\pgfqpoint{2.682714in}{0.680830in}}%
\pgfpathcurveto{\pgfqpoint{2.686832in}{0.676712in}}{\pgfqpoint{2.692418in}{0.674398in}}{\pgfqpoint{2.698242in}{0.674398in}}%
\pgfpathlineto{\pgfqpoint{2.698242in}{0.674398in}}%
\pgfpathclose%
\pgfusepath{stroke,fill}%
\end{pgfscope}%
\begin{pgfscope}%
\pgfpathrectangle{\pgfqpoint{0.640323in}{0.527436in}}{\pgfqpoint{9.687500in}{3.850000in}}%
\pgfusepath{clip}%
\pgfsetbuttcap%
\pgfsetroundjoin%
\definecolor{currentfill}{rgb}{0.000000,0.000000,1.000000}%
\pgfsetfillcolor{currentfill}%
\pgfsetfillopacity{0.500000}%
\pgfsetlinewidth{1.003750pt}%
\definecolor{currentstroke}{rgb}{0.000000,0.000000,1.000000}%
\pgfsetstrokecolor{currentstroke}%
\pgfsetstrokeopacity{0.500000}%
\pgfsetdash{{3.700000pt}{1.600000pt}}{0.000000pt}%
\pgfpathmoveto{\pgfqpoint{2.877973in}{1.612133in}}%
\pgfpathcurveto{\pgfqpoint{2.883797in}{1.612133in}}{\pgfqpoint{2.889383in}{1.614447in}}{\pgfqpoint{2.893501in}{1.618565in}}%
\pgfpathcurveto{\pgfqpoint{2.897620in}{1.622683in}}{\pgfqpoint{2.899934in}{1.628270in}}{\pgfqpoint{2.899934in}{1.634093in}}%
\pgfpathcurveto{\pgfqpoint{2.899934in}{1.639917in}}{\pgfqpoint{2.897620in}{1.645504in}}{\pgfqpoint{2.893501in}{1.649622in}}%
\pgfpathcurveto{\pgfqpoint{2.889383in}{1.653740in}}{\pgfqpoint{2.883797in}{1.656054in}}{\pgfqpoint{2.877973in}{1.656054in}}%
\pgfpathcurveto{\pgfqpoint{2.872149in}{1.656054in}}{\pgfqpoint{2.866563in}{1.653740in}}{\pgfqpoint{2.862445in}{1.649622in}}%
\pgfpathcurveto{\pgfqpoint{2.858327in}{1.645504in}}{\pgfqpoint{2.856013in}{1.639917in}}{\pgfqpoint{2.856013in}{1.634093in}}%
\pgfpathcurveto{\pgfqpoint{2.856013in}{1.628270in}}{\pgfqpoint{2.858327in}{1.622683in}}{\pgfqpoint{2.862445in}{1.618565in}}%
\pgfpathcurveto{\pgfqpoint{2.866563in}{1.614447in}}{\pgfqpoint{2.872149in}{1.612133in}}{\pgfqpoint{2.877973in}{1.612133in}}%
\pgfpathlineto{\pgfqpoint{2.877973in}{1.612133in}}%
\pgfpathclose%
\pgfusepath{stroke,fill}%
\end{pgfscope}%
\begin{pgfscope}%
\pgfpathrectangle{\pgfqpoint{0.640323in}{0.527436in}}{\pgfqpoint{9.687500in}{3.850000in}}%
\pgfusepath{clip}%
\pgfsetbuttcap%
\pgfsetroundjoin%
\definecolor{currentfill}{rgb}{0.000000,0.000000,1.000000}%
\pgfsetfillcolor{currentfill}%
\pgfsetfillopacity{0.500000}%
\pgfsetlinewidth{1.003750pt}%
\definecolor{currentstroke}{rgb}{0.000000,0.000000,1.000000}%
\pgfsetstrokecolor{currentstroke}%
\pgfsetstrokeopacity{0.500000}%
\pgfsetdash{{3.700000pt}{1.600000pt}}{0.000000pt}%
\pgfpathmoveto{\pgfqpoint{3.057704in}{1.862700in}}%
\pgfpathcurveto{\pgfqpoint{3.063528in}{1.862700in}}{\pgfqpoint{3.069114in}{1.865014in}}{\pgfqpoint{3.073232in}{1.869132in}}%
\pgfpathcurveto{\pgfqpoint{3.077351in}{1.873250in}}{\pgfqpoint{3.079664in}{1.878836in}}{\pgfqpoint{3.079664in}{1.884660in}}%
\pgfpathcurveto{\pgfqpoint{3.079664in}{1.890484in}}{\pgfqpoint{3.077351in}{1.896070in}}{\pgfqpoint{3.073232in}{1.900188in}}%
\pgfpathcurveto{\pgfqpoint{3.069114in}{1.904307in}}{\pgfqpoint{3.063528in}{1.906620in}}{\pgfqpoint{3.057704in}{1.906620in}}%
\pgfpathcurveto{\pgfqpoint{3.051880in}{1.906620in}}{\pgfqpoint{3.046294in}{1.904307in}}{\pgfqpoint{3.042176in}{1.900188in}}%
\pgfpathcurveto{\pgfqpoint{3.038058in}{1.896070in}}{\pgfqpoint{3.035744in}{1.890484in}}{\pgfqpoint{3.035744in}{1.884660in}}%
\pgfpathcurveto{\pgfqpoint{3.035744in}{1.878836in}}{\pgfqpoint{3.038058in}{1.873250in}}{\pgfqpoint{3.042176in}{1.869132in}}%
\pgfpathcurveto{\pgfqpoint{3.046294in}{1.865014in}}{\pgfqpoint{3.051880in}{1.862700in}}{\pgfqpoint{3.057704in}{1.862700in}}%
\pgfpathlineto{\pgfqpoint{3.057704in}{1.862700in}}%
\pgfpathclose%
\pgfusepath{stroke,fill}%
\end{pgfscope}%
\begin{pgfscope}%
\pgfpathrectangle{\pgfqpoint{0.640323in}{0.527436in}}{\pgfqpoint{9.687500in}{3.850000in}}%
\pgfusepath{clip}%
\pgfsetbuttcap%
\pgfsetroundjoin%
\definecolor{currentfill}{rgb}{0.000000,0.000000,1.000000}%
\pgfsetfillcolor{currentfill}%
\pgfsetfillopacity{0.500000}%
\pgfsetlinewidth{1.003750pt}%
\definecolor{currentstroke}{rgb}{0.000000,0.000000,1.000000}%
\pgfsetstrokecolor{currentstroke}%
\pgfsetstrokeopacity{0.500000}%
\pgfsetdash{{3.700000pt}{1.600000pt}}{0.000000pt}%
\pgfpathmoveto{\pgfqpoint{3.237435in}{2.097724in}}%
\pgfpathcurveto{\pgfqpoint{3.243259in}{2.097724in}}{\pgfqpoint{3.248845in}{2.100038in}}{\pgfqpoint{3.252963in}{2.104156in}}%
\pgfpathcurveto{\pgfqpoint{3.257082in}{2.108274in}}{\pgfqpoint{3.259395in}{2.113860in}}{\pgfqpoint{3.259395in}{2.119684in}}%
\pgfpathcurveto{\pgfqpoint{3.259395in}{2.125508in}}{\pgfqpoint{3.257082in}{2.131094in}}{\pgfqpoint{3.252963in}{2.135212in}}%
\pgfpathcurveto{\pgfqpoint{3.248845in}{2.139330in}}{\pgfqpoint{3.243259in}{2.141644in}}{\pgfqpoint{3.237435in}{2.141644in}}%
\pgfpathcurveto{\pgfqpoint{3.231611in}{2.141644in}}{\pgfqpoint{3.226025in}{2.139330in}}{\pgfqpoint{3.221907in}{2.135212in}}%
\pgfpathcurveto{\pgfqpoint{3.217789in}{2.131094in}}{\pgfqpoint{3.215475in}{2.125508in}}{\pgfqpoint{3.215475in}{2.119684in}}%
\pgfpathcurveto{\pgfqpoint{3.215475in}{2.113860in}}{\pgfqpoint{3.217789in}{2.108274in}}{\pgfqpoint{3.221907in}{2.104156in}}%
\pgfpathcurveto{\pgfqpoint{3.226025in}{2.100038in}}{\pgfqpoint{3.231611in}{2.097724in}}{\pgfqpoint{3.237435in}{2.097724in}}%
\pgfpathlineto{\pgfqpoint{3.237435in}{2.097724in}}%
\pgfpathclose%
\pgfusepath{stroke,fill}%
\end{pgfscope}%
\begin{pgfscope}%
\pgfpathrectangle{\pgfqpoint{0.640323in}{0.527436in}}{\pgfqpoint{9.687500in}{3.850000in}}%
\pgfusepath{clip}%
\pgfsetbuttcap%
\pgfsetroundjoin%
\definecolor{currentfill}{rgb}{0.000000,0.000000,1.000000}%
\pgfsetfillcolor{currentfill}%
\pgfsetfillopacity{0.500000}%
\pgfsetlinewidth{1.003750pt}%
\definecolor{currentstroke}{rgb}{0.000000,0.000000,1.000000}%
\pgfsetstrokecolor{currentstroke}%
\pgfsetstrokeopacity{0.500000}%
\pgfsetdash{{3.700000pt}{1.600000pt}}{0.000000pt}%
\pgfpathmoveto{\pgfqpoint{3.417166in}{2.309629in}}%
\pgfpathcurveto{\pgfqpoint{3.422990in}{2.309629in}}{\pgfqpoint{3.428576in}{2.311943in}}{\pgfqpoint{3.432694in}{2.316061in}}%
\pgfpathcurveto{\pgfqpoint{3.436813in}{2.320179in}}{\pgfqpoint{3.439126in}{2.325765in}}{\pgfqpoint{3.439126in}{2.331589in}}%
\pgfpathcurveto{\pgfqpoint{3.439126in}{2.337413in}}{\pgfqpoint{3.436813in}{2.342999in}}{\pgfqpoint{3.432694in}{2.347118in}}%
\pgfpathcurveto{\pgfqpoint{3.428576in}{2.351236in}}{\pgfqpoint{3.422990in}{2.353550in}}{\pgfqpoint{3.417166in}{2.353550in}}%
\pgfpathcurveto{\pgfqpoint{3.411342in}{2.353550in}}{\pgfqpoint{3.405756in}{2.351236in}}{\pgfqpoint{3.401638in}{2.347118in}}%
\pgfpathcurveto{\pgfqpoint{3.397520in}{2.342999in}}{\pgfqpoint{3.395206in}{2.337413in}}{\pgfqpoint{3.395206in}{2.331589in}}%
\pgfpathcurveto{\pgfqpoint{3.395206in}{2.325765in}}{\pgfqpoint{3.397520in}{2.320179in}}{\pgfqpoint{3.401638in}{2.316061in}}%
\pgfpathcurveto{\pgfqpoint{3.405756in}{2.311943in}}{\pgfqpoint{3.411342in}{2.309629in}}{\pgfqpoint{3.417166in}{2.309629in}}%
\pgfpathlineto{\pgfqpoint{3.417166in}{2.309629in}}%
\pgfpathclose%
\pgfusepath{stroke,fill}%
\end{pgfscope}%
\begin{pgfscope}%
\pgfpathrectangle{\pgfqpoint{0.640323in}{0.527436in}}{\pgfqpoint{9.687500in}{3.850000in}}%
\pgfusepath{clip}%
\pgfsetbuttcap%
\pgfsetroundjoin%
\definecolor{currentfill}{rgb}{0.000000,0.000000,1.000000}%
\pgfsetfillcolor{currentfill}%
\pgfsetfillopacity{0.500000}%
\pgfsetlinewidth{1.003750pt}%
\definecolor{currentstroke}{rgb}{0.000000,0.000000,1.000000}%
\pgfsetstrokecolor{currentstroke}%
\pgfsetstrokeopacity{0.500000}%
\pgfsetdash{{3.700000pt}{1.600000pt}}{0.000000pt}%
\pgfpathmoveto{\pgfqpoint{3.596897in}{2.484444in}}%
\pgfpathcurveto{\pgfqpoint{3.602721in}{2.484444in}}{\pgfqpoint{3.608307in}{2.486758in}}{\pgfqpoint{3.612425in}{2.490876in}}%
\pgfpathcurveto{\pgfqpoint{3.616544in}{2.494994in}}{\pgfqpoint{3.618857in}{2.500580in}}{\pgfqpoint{3.618857in}{2.506404in}}%
\pgfpathcurveto{\pgfqpoint{3.618857in}{2.512228in}}{\pgfqpoint{3.616544in}{2.517814in}}{\pgfqpoint{3.612425in}{2.521932in}}%
\pgfpathcurveto{\pgfqpoint{3.608307in}{2.526051in}}{\pgfqpoint{3.602721in}{2.528364in}}{\pgfqpoint{3.596897in}{2.528364in}}%
\pgfpathcurveto{\pgfqpoint{3.591073in}{2.528364in}}{\pgfqpoint{3.585487in}{2.526051in}}{\pgfqpoint{3.581369in}{2.521932in}}%
\pgfpathcurveto{\pgfqpoint{3.577251in}{2.517814in}}{\pgfqpoint{3.574937in}{2.512228in}}{\pgfqpoint{3.574937in}{2.506404in}}%
\pgfpathcurveto{\pgfqpoint{3.574937in}{2.500580in}}{\pgfqpoint{3.577251in}{2.494994in}}{\pgfqpoint{3.581369in}{2.490876in}}%
\pgfpathcurveto{\pgfqpoint{3.585487in}{2.486758in}}{\pgfqpoint{3.591073in}{2.484444in}}{\pgfqpoint{3.596897in}{2.484444in}}%
\pgfpathlineto{\pgfqpoint{3.596897in}{2.484444in}}%
\pgfpathclose%
\pgfusepath{stroke,fill}%
\end{pgfscope}%
\begin{pgfscope}%
\pgfpathrectangle{\pgfqpoint{0.640323in}{0.527436in}}{\pgfqpoint{9.687500in}{3.850000in}}%
\pgfusepath{clip}%
\pgfsetbuttcap%
\pgfsetroundjoin%
\definecolor{currentfill}{rgb}{0.000000,0.000000,1.000000}%
\pgfsetfillcolor{currentfill}%
\pgfsetfillopacity{0.500000}%
\pgfsetlinewidth{1.003750pt}%
\definecolor{currentstroke}{rgb}{0.000000,0.000000,1.000000}%
\pgfsetstrokecolor{currentstroke}%
\pgfsetstrokeopacity{0.500000}%
\pgfsetdash{{3.700000pt}{1.600000pt}}{0.000000pt}%
\pgfpathmoveto{\pgfqpoint{3.776628in}{2.584358in}}%
\pgfpathcurveto{\pgfqpoint{3.782452in}{2.584358in}}{\pgfqpoint{3.788038in}{2.586671in}}{\pgfqpoint{3.792156in}{2.590790in}}%
\pgfpathcurveto{\pgfqpoint{3.796275in}{2.594908in}}{\pgfqpoint{3.798588in}{2.600494in}}{\pgfqpoint{3.798588in}{2.606318in}}%
\pgfpathcurveto{\pgfqpoint{3.798588in}{2.612142in}}{\pgfqpoint{3.796275in}{2.617728in}}{\pgfqpoint{3.792156in}{2.621846in}}%
\pgfpathcurveto{\pgfqpoint{3.788038in}{2.625964in}}{\pgfqpoint{3.782452in}{2.628278in}}{\pgfqpoint{3.776628in}{2.628278in}}%
\pgfpathcurveto{\pgfqpoint{3.770804in}{2.628278in}}{\pgfqpoint{3.765218in}{2.625964in}}{\pgfqpoint{3.761100in}{2.621846in}}%
\pgfpathcurveto{\pgfqpoint{3.756982in}{2.617728in}}{\pgfqpoint{3.754668in}{2.612142in}}{\pgfqpoint{3.754668in}{2.606318in}}%
\pgfpathcurveto{\pgfqpoint{3.754668in}{2.600494in}}{\pgfqpoint{3.756982in}{2.594908in}}{\pgfqpoint{3.761100in}{2.590790in}}%
\pgfpathcurveto{\pgfqpoint{3.765218in}{2.586671in}}{\pgfqpoint{3.770804in}{2.584358in}}{\pgfqpoint{3.776628in}{2.584358in}}%
\pgfpathlineto{\pgfqpoint{3.776628in}{2.584358in}}%
\pgfpathclose%
\pgfusepath{stroke,fill}%
\end{pgfscope}%
\begin{pgfscope}%
\pgfpathrectangle{\pgfqpoint{0.640323in}{0.527436in}}{\pgfqpoint{9.687500in}{3.850000in}}%
\pgfusepath{clip}%
\pgfsetbuttcap%
\pgfsetroundjoin%
\definecolor{currentfill}{rgb}{0.000000,0.000000,1.000000}%
\pgfsetfillcolor{currentfill}%
\pgfsetfillopacity{0.500000}%
\pgfsetlinewidth{1.003750pt}%
\definecolor{currentstroke}{rgb}{0.000000,0.000000,1.000000}%
\pgfsetstrokecolor{currentstroke}%
\pgfsetstrokeopacity{0.500000}%
\pgfsetdash{{3.700000pt}{1.600000pt}}{0.000000pt}%
\pgfpathmoveto{\pgfqpoint{3.956359in}{2.767717in}}%
\pgfpathcurveto{\pgfqpoint{3.962183in}{2.767717in}}{\pgfqpoint{3.967769in}{2.770031in}}{\pgfqpoint{3.971887in}{2.774149in}}%
\pgfpathcurveto{\pgfqpoint{3.976006in}{2.778267in}}{\pgfqpoint{3.978319in}{2.783853in}}{\pgfqpoint{3.978319in}{2.789677in}}%
\pgfpathcurveto{\pgfqpoint{3.978319in}{2.795501in}}{\pgfqpoint{3.976006in}{2.801087in}}{\pgfqpoint{3.971887in}{2.805205in}}%
\pgfpathcurveto{\pgfqpoint{3.967769in}{2.809324in}}{\pgfqpoint{3.962183in}{2.811637in}}{\pgfqpoint{3.956359in}{2.811637in}}%
\pgfpathcurveto{\pgfqpoint{3.950535in}{2.811637in}}{\pgfqpoint{3.944949in}{2.809324in}}{\pgfqpoint{3.940831in}{2.805205in}}%
\pgfpathcurveto{\pgfqpoint{3.936713in}{2.801087in}}{\pgfqpoint{3.934399in}{2.795501in}}{\pgfqpoint{3.934399in}{2.789677in}}%
\pgfpathcurveto{\pgfqpoint{3.934399in}{2.783853in}}{\pgfqpoint{3.936713in}{2.778267in}}{\pgfqpoint{3.940831in}{2.774149in}}%
\pgfpathcurveto{\pgfqpoint{3.944949in}{2.770031in}}{\pgfqpoint{3.950535in}{2.767717in}}{\pgfqpoint{3.956359in}{2.767717in}}%
\pgfpathlineto{\pgfqpoint{3.956359in}{2.767717in}}%
\pgfpathclose%
\pgfusepath{stroke,fill}%
\end{pgfscope}%
\begin{pgfscope}%
\pgfpathrectangle{\pgfqpoint{0.640323in}{0.527436in}}{\pgfqpoint{9.687500in}{3.850000in}}%
\pgfusepath{clip}%
\pgfsetbuttcap%
\pgfsetroundjoin%
\definecolor{currentfill}{rgb}{0.000000,0.000000,1.000000}%
\pgfsetfillcolor{currentfill}%
\pgfsetfillopacity{0.500000}%
\pgfsetlinewidth{1.003750pt}%
\definecolor{currentstroke}{rgb}{0.000000,0.000000,1.000000}%
\pgfsetstrokecolor{currentstroke}%
\pgfsetstrokeopacity{0.500000}%
\pgfsetdash{{3.700000pt}{1.600000pt}}{0.000000pt}%
\pgfpathmoveto{\pgfqpoint{4.136090in}{2.807571in}}%
\pgfpathcurveto{\pgfqpoint{4.141914in}{2.807571in}}{\pgfqpoint{4.147500in}{2.809885in}}{\pgfqpoint{4.151618in}{2.814003in}}%
\pgfpathcurveto{\pgfqpoint{4.155737in}{2.818121in}}{\pgfqpoint{4.158050in}{2.823707in}}{\pgfqpoint{4.158050in}{2.829531in}}%
\pgfpathcurveto{\pgfqpoint{4.158050in}{2.835355in}}{\pgfqpoint{4.155737in}{2.840941in}}{\pgfqpoint{4.151618in}{2.845059in}}%
\pgfpathcurveto{\pgfqpoint{4.147500in}{2.849177in}}{\pgfqpoint{4.141914in}{2.851491in}}{\pgfqpoint{4.136090in}{2.851491in}}%
\pgfpathcurveto{\pgfqpoint{4.130266in}{2.851491in}}{\pgfqpoint{4.124680in}{2.849177in}}{\pgfqpoint{4.120562in}{2.845059in}}%
\pgfpathcurveto{\pgfqpoint{4.116444in}{2.840941in}}{\pgfqpoint{4.114130in}{2.835355in}}{\pgfqpoint{4.114130in}{2.829531in}}%
\pgfpathcurveto{\pgfqpoint{4.114130in}{2.823707in}}{\pgfqpoint{4.116444in}{2.818121in}}{\pgfqpoint{4.120562in}{2.814003in}}%
\pgfpathcurveto{\pgfqpoint{4.124680in}{2.809885in}}{\pgfqpoint{4.130266in}{2.807571in}}{\pgfqpoint{4.136090in}{2.807571in}}%
\pgfpathlineto{\pgfqpoint{4.136090in}{2.807571in}}%
\pgfpathclose%
\pgfusepath{stroke,fill}%
\end{pgfscope}%
\begin{pgfscope}%
\pgfpathrectangle{\pgfqpoint{0.640323in}{0.527436in}}{\pgfqpoint{9.687500in}{3.850000in}}%
\pgfusepath{clip}%
\pgfsetbuttcap%
\pgfsetroundjoin%
\definecolor{currentfill}{rgb}{0.000000,0.000000,1.000000}%
\pgfsetfillcolor{currentfill}%
\pgfsetfillopacity{0.500000}%
\pgfsetlinewidth{1.003750pt}%
\definecolor{currentstroke}{rgb}{0.000000,0.000000,1.000000}%
\pgfsetstrokecolor{currentstroke}%
\pgfsetstrokeopacity{0.500000}%
\pgfsetdash{{3.700000pt}{1.600000pt}}{0.000000pt}%
\pgfpathmoveto{\pgfqpoint{4.315821in}{2.951946in}}%
\pgfpathcurveto{\pgfqpoint{4.321645in}{2.951946in}}{\pgfqpoint{4.327231in}{2.954260in}}{\pgfqpoint{4.331349in}{2.958378in}}%
\pgfpathcurveto{\pgfqpoint{4.335467in}{2.962496in}}{\pgfqpoint{4.337781in}{2.968082in}}{\pgfqpoint{4.337781in}{2.973906in}}%
\pgfpathcurveto{\pgfqpoint{4.337781in}{2.979730in}}{\pgfqpoint{4.335467in}{2.985316in}}{\pgfqpoint{4.331349in}{2.989434in}}%
\pgfpathcurveto{\pgfqpoint{4.327231in}{2.993552in}}{\pgfqpoint{4.321645in}{2.995866in}}{\pgfqpoint{4.315821in}{2.995866in}}%
\pgfpathcurveto{\pgfqpoint{4.309997in}{2.995866in}}{\pgfqpoint{4.304411in}{2.993552in}}{\pgfqpoint{4.300293in}{2.989434in}}%
\pgfpathcurveto{\pgfqpoint{4.296175in}{2.985316in}}{\pgfqpoint{4.293861in}{2.979730in}}{\pgfqpoint{4.293861in}{2.973906in}}%
\pgfpathcurveto{\pgfqpoint{4.293861in}{2.968082in}}{\pgfqpoint{4.296175in}{2.962496in}}{\pgfqpoint{4.300293in}{2.958378in}}%
\pgfpathcurveto{\pgfqpoint{4.304411in}{2.954260in}}{\pgfqpoint{4.309997in}{2.951946in}}{\pgfqpoint{4.315821in}{2.951946in}}%
\pgfpathlineto{\pgfqpoint{4.315821in}{2.951946in}}%
\pgfpathclose%
\pgfusepath{stroke,fill}%
\end{pgfscope}%
\begin{pgfscope}%
\pgfpathrectangle{\pgfqpoint{0.640323in}{0.527436in}}{\pgfqpoint{9.687500in}{3.850000in}}%
\pgfusepath{clip}%
\pgfsetbuttcap%
\pgfsetroundjoin%
\definecolor{currentfill}{rgb}{0.000000,0.000000,1.000000}%
\pgfsetfillcolor{currentfill}%
\pgfsetfillopacity{0.500000}%
\pgfsetlinewidth{1.003750pt}%
\definecolor{currentstroke}{rgb}{0.000000,0.000000,1.000000}%
\pgfsetstrokecolor{currentstroke}%
\pgfsetstrokeopacity{0.500000}%
\pgfsetdash{{3.700000pt}{1.600000pt}}{0.000000pt}%
\pgfpathmoveto{\pgfqpoint{4.495552in}{3.010205in}}%
\pgfpathcurveto{\pgfqpoint{4.501376in}{3.010205in}}{\pgfqpoint{4.506962in}{3.012519in}}{\pgfqpoint{4.511080in}{3.016637in}}%
\pgfpathcurveto{\pgfqpoint{4.515198in}{3.020755in}}{\pgfqpoint{4.517512in}{3.026341in}}{\pgfqpoint{4.517512in}{3.032165in}}%
\pgfpathcurveto{\pgfqpoint{4.517512in}{3.037989in}}{\pgfqpoint{4.515198in}{3.043575in}}{\pgfqpoint{4.511080in}{3.047693in}}%
\pgfpathcurveto{\pgfqpoint{4.506962in}{3.051811in}}{\pgfqpoint{4.501376in}{3.054125in}}{\pgfqpoint{4.495552in}{3.054125in}}%
\pgfpathcurveto{\pgfqpoint{4.489728in}{3.054125in}}{\pgfqpoint{4.484142in}{3.051811in}}{\pgfqpoint{4.480024in}{3.047693in}}%
\pgfpathcurveto{\pgfqpoint{4.475906in}{3.043575in}}{\pgfqpoint{4.473592in}{3.037989in}}{\pgfqpoint{4.473592in}{3.032165in}}%
\pgfpathcurveto{\pgfqpoint{4.473592in}{3.026341in}}{\pgfqpoint{4.475906in}{3.020755in}}{\pgfqpoint{4.480024in}{3.016637in}}%
\pgfpathcurveto{\pgfqpoint{4.484142in}{3.012519in}}{\pgfqpoint{4.489728in}{3.010205in}}{\pgfqpoint{4.495552in}{3.010205in}}%
\pgfpathlineto{\pgfqpoint{4.495552in}{3.010205in}}%
\pgfpathclose%
\pgfusepath{stroke,fill}%
\end{pgfscope}%
\begin{pgfscope}%
\pgfpathrectangle{\pgfqpoint{0.640323in}{0.527436in}}{\pgfqpoint{9.687500in}{3.850000in}}%
\pgfusepath{clip}%
\pgfsetbuttcap%
\pgfsetroundjoin%
\definecolor{currentfill}{rgb}{0.000000,0.000000,1.000000}%
\pgfsetfillcolor{currentfill}%
\pgfsetfillopacity{0.500000}%
\pgfsetlinewidth{1.003750pt}%
\definecolor{currentstroke}{rgb}{0.000000,0.000000,1.000000}%
\pgfsetstrokecolor{currentstroke}%
\pgfsetstrokeopacity{0.500000}%
\pgfsetdash{{3.700000pt}{1.600000pt}}{0.000000pt}%
\pgfpathmoveto{\pgfqpoint{4.675283in}{3.101580in}}%
\pgfpathcurveto{\pgfqpoint{4.681107in}{3.101580in}}{\pgfqpoint{4.686693in}{3.103894in}}{\pgfqpoint{4.690811in}{3.108012in}}%
\pgfpathcurveto{\pgfqpoint{4.694929in}{3.112130in}}{\pgfqpoint{4.697243in}{3.117717in}}{\pgfqpoint{4.697243in}{3.123541in}}%
\pgfpathcurveto{\pgfqpoint{4.697243in}{3.129364in}}{\pgfqpoint{4.694929in}{3.134951in}}{\pgfqpoint{4.690811in}{3.139069in}}%
\pgfpathcurveto{\pgfqpoint{4.686693in}{3.143187in}}{\pgfqpoint{4.681107in}{3.145501in}}{\pgfqpoint{4.675283in}{3.145501in}}%
\pgfpathcurveto{\pgfqpoint{4.669459in}{3.145501in}}{\pgfqpoint{4.663873in}{3.143187in}}{\pgfqpoint{4.659755in}{3.139069in}}%
\pgfpathcurveto{\pgfqpoint{4.655637in}{3.134951in}}{\pgfqpoint{4.653323in}{3.129364in}}{\pgfqpoint{4.653323in}{3.123541in}}%
\pgfpathcurveto{\pgfqpoint{4.653323in}{3.117717in}}{\pgfqpoint{4.655637in}{3.112130in}}{\pgfqpoint{4.659755in}{3.108012in}}%
\pgfpathcurveto{\pgfqpoint{4.663873in}{3.103894in}}{\pgfqpoint{4.669459in}{3.101580in}}{\pgfqpoint{4.675283in}{3.101580in}}%
\pgfpathlineto{\pgfqpoint{4.675283in}{3.101580in}}%
\pgfpathclose%
\pgfusepath{stroke,fill}%
\end{pgfscope}%
\begin{pgfscope}%
\pgfpathrectangle{\pgfqpoint{0.640323in}{0.527436in}}{\pgfqpoint{9.687500in}{3.850000in}}%
\pgfusepath{clip}%
\pgfsetbuttcap%
\pgfsetroundjoin%
\definecolor{currentfill}{rgb}{0.000000,0.000000,1.000000}%
\pgfsetfillcolor{currentfill}%
\pgfsetfillopacity{0.500000}%
\pgfsetlinewidth{1.003750pt}%
\definecolor{currentstroke}{rgb}{0.000000,0.000000,1.000000}%
\pgfsetstrokecolor{currentstroke}%
\pgfsetstrokeopacity{0.500000}%
\pgfsetdash{{3.700000pt}{1.600000pt}}{0.000000pt}%
\pgfpathmoveto{\pgfqpoint{4.855014in}{3.177698in}}%
\pgfpathcurveto{\pgfqpoint{4.860838in}{3.177698in}}{\pgfqpoint{4.866424in}{3.180012in}}{\pgfqpoint{4.870542in}{3.184130in}}%
\pgfpathcurveto{\pgfqpoint{4.874660in}{3.188249in}}{\pgfqpoint{4.876974in}{3.193835in}}{\pgfqpoint{4.876974in}{3.199659in}}%
\pgfpathcurveto{\pgfqpoint{4.876974in}{3.205483in}}{\pgfqpoint{4.874660in}{3.211069in}}{\pgfqpoint{4.870542in}{3.215187in}}%
\pgfpathcurveto{\pgfqpoint{4.866424in}{3.219305in}}{\pgfqpoint{4.860838in}{3.221619in}}{\pgfqpoint{4.855014in}{3.221619in}}%
\pgfpathcurveto{\pgfqpoint{4.849190in}{3.221619in}}{\pgfqpoint{4.843604in}{3.219305in}}{\pgfqpoint{4.839486in}{3.215187in}}%
\pgfpathcurveto{\pgfqpoint{4.835368in}{3.211069in}}{\pgfqpoint{4.833054in}{3.205483in}}{\pgfqpoint{4.833054in}{3.199659in}}%
\pgfpathcurveto{\pgfqpoint{4.833054in}{3.193835in}}{\pgfqpoint{4.835368in}{3.188249in}}{\pgfqpoint{4.839486in}{3.184130in}}%
\pgfpathcurveto{\pgfqpoint{4.843604in}{3.180012in}}{\pgfqpoint{4.849190in}{3.177698in}}{\pgfqpoint{4.855014in}{3.177698in}}%
\pgfpathlineto{\pgfqpoint{4.855014in}{3.177698in}}%
\pgfpathclose%
\pgfusepath{stroke,fill}%
\end{pgfscope}%
\begin{pgfscope}%
\pgfpathrectangle{\pgfqpoint{0.640323in}{0.527436in}}{\pgfqpoint{9.687500in}{3.850000in}}%
\pgfusepath{clip}%
\pgfsetbuttcap%
\pgfsetroundjoin%
\definecolor{currentfill}{rgb}{0.000000,0.000000,1.000000}%
\pgfsetfillcolor{currentfill}%
\pgfsetfillopacity{0.500000}%
\pgfsetlinewidth{1.003750pt}%
\definecolor{currentstroke}{rgb}{0.000000,0.000000,1.000000}%
\pgfsetstrokecolor{currentstroke}%
\pgfsetstrokeopacity{0.500000}%
\pgfsetdash{{3.700000pt}{1.600000pt}}{0.000000pt}%
\pgfpathmoveto{\pgfqpoint{5.034745in}{3.210908in}}%
\pgfpathcurveto{\pgfqpoint{5.040569in}{3.210908in}}{\pgfqpoint{5.046155in}{3.213222in}}{\pgfqpoint{5.050273in}{3.217340in}}%
\pgfpathcurveto{\pgfqpoint{5.054391in}{3.221458in}}{\pgfqpoint{5.056705in}{3.227044in}}{\pgfqpoint{5.056705in}{3.232868in}}%
\pgfpathcurveto{\pgfqpoint{5.056705in}{3.238692in}}{\pgfqpoint{5.054391in}{3.244278in}}{\pgfqpoint{5.050273in}{3.248396in}}%
\pgfpathcurveto{\pgfqpoint{5.046155in}{3.252514in}}{\pgfqpoint{5.040569in}{3.254828in}}{\pgfqpoint{5.034745in}{3.254828in}}%
\pgfpathcurveto{\pgfqpoint{5.028921in}{3.254828in}}{\pgfqpoint{5.023335in}{3.252514in}}{\pgfqpoint{5.019217in}{3.248396in}}%
\pgfpathcurveto{\pgfqpoint{5.015099in}{3.244278in}}{\pgfqpoint{5.012785in}{3.238692in}}{\pgfqpoint{5.012785in}{3.232868in}}%
\pgfpathcurveto{\pgfqpoint{5.012785in}{3.227044in}}{\pgfqpoint{5.015099in}{3.221458in}}{\pgfqpoint{5.019217in}{3.217340in}}%
\pgfpathcurveto{\pgfqpoint{5.023335in}{3.213222in}}{\pgfqpoint{5.028921in}{3.210908in}}{\pgfqpoint{5.034745in}{3.210908in}}%
\pgfpathlineto{\pgfqpoint{5.034745in}{3.210908in}}%
\pgfpathclose%
\pgfusepath{stroke,fill}%
\end{pgfscope}%
\begin{pgfscope}%
\pgfpathrectangle{\pgfqpoint{0.640323in}{0.527436in}}{\pgfqpoint{9.687500in}{3.850000in}}%
\pgfusepath{clip}%
\pgfsetbuttcap%
\pgfsetroundjoin%
\definecolor{currentfill}{rgb}{0.000000,0.000000,1.000000}%
\pgfsetfillcolor{currentfill}%
\pgfsetfillopacity{0.500000}%
\pgfsetlinewidth{1.003750pt}%
\definecolor{currentstroke}{rgb}{0.000000,0.000000,1.000000}%
\pgfsetstrokecolor{currentstroke}%
\pgfsetstrokeopacity{0.500000}%
\pgfsetdash{{3.700000pt}{1.600000pt}}{0.000000pt}%
\pgfpathmoveto{\pgfqpoint{5.214476in}{3.277805in}}%
\pgfpathcurveto{\pgfqpoint{5.220300in}{3.277805in}}{\pgfqpoint{5.225886in}{3.280119in}}{\pgfqpoint{5.230004in}{3.284237in}}%
\pgfpathcurveto{\pgfqpoint{5.234122in}{3.288355in}}{\pgfqpoint{5.236436in}{3.293941in}}{\pgfqpoint{5.236436in}{3.299765in}}%
\pgfpathcurveto{\pgfqpoint{5.236436in}{3.305589in}}{\pgfqpoint{5.234122in}{3.311175in}}{\pgfqpoint{5.230004in}{3.315293in}}%
\pgfpathcurveto{\pgfqpoint{5.225886in}{3.319411in}}{\pgfqpoint{5.220300in}{3.321725in}}{\pgfqpoint{5.214476in}{3.321725in}}%
\pgfpathcurveto{\pgfqpoint{5.208652in}{3.321725in}}{\pgfqpoint{5.203066in}{3.319411in}}{\pgfqpoint{5.198948in}{3.315293in}}%
\pgfpathcurveto{\pgfqpoint{5.194830in}{3.311175in}}{\pgfqpoint{5.192516in}{3.305589in}}{\pgfqpoint{5.192516in}{3.299765in}}%
\pgfpathcurveto{\pgfqpoint{5.192516in}{3.293941in}}{\pgfqpoint{5.194830in}{3.288355in}}{\pgfqpoint{5.198948in}{3.284237in}}%
\pgfpathcurveto{\pgfqpoint{5.203066in}{3.280119in}}{\pgfqpoint{5.208652in}{3.277805in}}{\pgfqpoint{5.214476in}{3.277805in}}%
\pgfpathlineto{\pgfqpoint{5.214476in}{3.277805in}}%
\pgfpathclose%
\pgfusepath{stroke,fill}%
\end{pgfscope}%
\begin{pgfscope}%
\pgfpathrectangle{\pgfqpoint{0.640323in}{0.527436in}}{\pgfqpoint{9.687500in}{3.850000in}}%
\pgfusepath{clip}%
\pgfsetbuttcap%
\pgfsetroundjoin%
\definecolor{currentfill}{rgb}{0.000000,0.000000,1.000000}%
\pgfsetfillcolor{currentfill}%
\pgfsetfillopacity{0.500000}%
\pgfsetlinewidth{1.003750pt}%
\definecolor{currentstroke}{rgb}{0.000000,0.000000,1.000000}%
\pgfsetstrokecolor{currentstroke}%
\pgfsetstrokeopacity{0.500000}%
\pgfsetdash{{3.700000pt}{1.600000pt}}{0.000000pt}%
\pgfpathmoveto{\pgfqpoint{5.394207in}{3.337790in}}%
\pgfpathcurveto{\pgfqpoint{5.400031in}{3.337790in}}{\pgfqpoint{5.405617in}{3.340104in}}{\pgfqpoint{5.409735in}{3.344222in}}%
\pgfpathcurveto{\pgfqpoint{5.413853in}{3.348340in}}{\pgfqpoint{5.416167in}{3.353927in}}{\pgfqpoint{5.416167in}{3.359750in}}%
\pgfpathcurveto{\pgfqpoint{5.416167in}{3.365574in}}{\pgfqpoint{5.413853in}{3.371161in}}{\pgfqpoint{5.409735in}{3.375279in}}%
\pgfpathcurveto{\pgfqpoint{5.405617in}{3.379397in}}{\pgfqpoint{5.400031in}{3.381711in}}{\pgfqpoint{5.394207in}{3.381711in}}%
\pgfpathcurveto{\pgfqpoint{5.388383in}{3.381711in}}{\pgfqpoint{5.382797in}{3.379397in}}{\pgfqpoint{5.378679in}{3.375279in}}%
\pgfpathcurveto{\pgfqpoint{5.374561in}{3.371161in}}{\pgfqpoint{5.372247in}{3.365574in}}{\pgfqpoint{5.372247in}{3.359750in}}%
\pgfpathcurveto{\pgfqpoint{5.372247in}{3.353927in}}{\pgfqpoint{5.374561in}{3.348340in}}{\pgfqpoint{5.378679in}{3.344222in}}%
\pgfpathcurveto{\pgfqpoint{5.382797in}{3.340104in}}{\pgfqpoint{5.388383in}{3.337790in}}{\pgfqpoint{5.394207in}{3.337790in}}%
\pgfpathlineto{\pgfqpoint{5.394207in}{3.337790in}}%
\pgfpathclose%
\pgfusepath{stroke,fill}%
\end{pgfscope}%
\begin{pgfscope}%
\pgfpathrectangle{\pgfqpoint{0.640323in}{0.527436in}}{\pgfqpoint{9.687500in}{3.850000in}}%
\pgfusepath{clip}%
\pgfsetbuttcap%
\pgfsetroundjoin%
\definecolor{currentfill}{rgb}{0.000000,0.000000,1.000000}%
\pgfsetfillcolor{currentfill}%
\pgfsetfillopacity{0.500000}%
\pgfsetlinewidth{1.003750pt}%
\definecolor{currentstroke}{rgb}{0.000000,0.000000,1.000000}%
\pgfsetstrokecolor{currentstroke}%
\pgfsetstrokeopacity{0.500000}%
\pgfsetdash{{3.700000pt}{1.600000pt}}{0.000000pt}%
\pgfpathmoveto{\pgfqpoint{5.573938in}{3.384096in}}%
\pgfpathcurveto{\pgfqpoint{5.579762in}{3.384096in}}{\pgfqpoint{5.585348in}{3.386410in}}{\pgfqpoint{5.589466in}{3.390528in}}%
\pgfpathcurveto{\pgfqpoint{5.593584in}{3.394646in}}{\pgfqpoint{5.595898in}{3.400232in}}{\pgfqpoint{5.595898in}{3.406056in}}%
\pgfpathcurveto{\pgfqpoint{5.595898in}{3.411880in}}{\pgfqpoint{5.593584in}{3.417466in}}{\pgfqpoint{5.589466in}{3.421584in}}%
\pgfpathcurveto{\pgfqpoint{5.585348in}{3.425702in}}{\pgfqpoint{5.579762in}{3.428016in}}{\pgfqpoint{5.573938in}{3.428016in}}%
\pgfpathcurveto{\pgfqpoint{5.568114in}{3.428016in}}{\pgfqpoint{5.562528in}{3.425702in}}{\pgfqpoint{5.558410in}{3.421584in}}%
\pgfpathcurveto{\pgfqpoint{5.554292in}{3.417466in}}{\pgfqpoint{5.551978in}{3.411880in}}{\pgfqpoint{5.551978in}{3.406056in}}%
\pgfpathcurveto{\pgfqpoint{5.551978in}{3.400232in}}{\pgfqpoint{5.554292in}{3.394646in}}{\pgfqpoint{5.558410in}{3.390528in}}%
\pgfpathcurveto{\pgfqpoint{5.562528in}{3.386410in}}{\pgfqpoint{5.568114in}{3.384096in}}{\pgfqpoint{5.573938in}{3.384096in}}%
\pgfpathlineto{\pgfqpoint{5.573938in}{3.384096in}}%
\pgfpathclose%
\pgfusepath{stroke,fill}%
\end{pgfscope}%
\begin{pgfscope}%
\pgfpathrectangle{\pgfqpoint{0.640323in}{0.527436in}}{\pgfqpoint{9.687500in}{3.850000in}}%
\pgfusepath{clip}%
\pgfsetbuttcap%
\pgfsetroundjoin%
\definecolor{currentfill}{rgb}{0.000000,0.000000,1.000000}%
\pgfsetfillcolor{currentfill}%
\pgfsetfillopacity{0.500000}%
\pgfsetlinewidth{1.003750pt}%
\definecolor{currentstroke}{rgb}{0.000000,0.000000,1.000000}%
\pgfsetstrokecolor{currentstroke}%
\pgfsetstrokeopacity{0.500000}%
\pgfsetdash{{3.700000pt}{1.600000pt}}{0.000000pt}%
\pgfpathmoveto{\pgfqpoint{5.753669in}{3.404302in}}%
\pgfpathcurveto{\pgfqpoint{5.759493in}{3.404302in}}{\pgfqpoint{5.765079in}{3.406616in}}{\pgfqpoint{5.769197in}{3.410734in}}%
\pgfpathcurveto{\pgfqpoint{5.773315in}{3.414852in}}{\pgfqpoint{5.775629in}{3.420438in}}{\pgfqpoint{5.775629in}{3.426262in}}%
\pgfpathcurveto{\pgfqpoint{5.775629in}{3.432086in}}{\pgfqpoint{5.773315in}{3.437672in}}{\pgfqpoint{5.769197in}{3.441791in}}%
\pgfpathcurveto{\pgfqpoint{5.765079in}{3.445909in}}{\pgfqpoint{5.759493in}{3.448223in}}{\pgfqpoint{5.753669in}{3.448223in}}%
\pgfpathcurveto{\pgfqpoint{5.747845in}{3.448223in}}{\pgfqpoint{5.742259in}{3.445909in}}{\pgfqpoint{5.738141in}{3.441791in}}%
\pgfpathcurveto{\pgfqpoint{5.734023in}{3.437672in}}{\pgfqpoint{5.731709in}{3.432086in}}{\pgfqpoint{5.731709in}{3.426262in}}%
\pgfpathcurveto{\pgfqpoint{5.731709in}{3.420438in}}{\pgfqpoint{5.734023in}{3.414852in}}{\pgfqpoint{5.738141in}{3.410734in}}%
\pgfpathcurveto{\pgfqpoint{5.742259in}{3.406616in}}{\pgfqpoint{5.747845in}{3.404302in}}{\pgfqpoint{5.753669in}{3.404302in}}%
\pgfpathlineto{\pgfqpoint{5.753669in}{3.404302in}}%
\pgfpathclose%
\pgfusepath{stroke,fill}%
\end{pgfscope}%
\begin{pgfscope}%
\pgfpathrectangle{\pgfqpoint{0.640323in}{0.527436in}}{\pgfqpoint{9.687500in}{3.850000in}}%
\pgfusepath{clip}%
\pgfsetbuttcap%
\pgfsetroundjoin%
\definecolor{currentfill}{rgb}{0.000000,0.000000,1.000000}%
\pgfsetfillcolor{currentfill}%
\pgfsetfillopacity{0.500000}%
\pgfsetlinewidth{1.003750pt}%
\definecolor{currentstroke}{rgb}{0.000000,0.000000,1.000000}%
\pgfsetstrokecolor{currentstroke}%
\pgfsetstrokeopacity{0.500000}%
\pgfsetdash{{3.700000pt}{1.600000pt}}{0.000000pt}%
\pgfpathmoveto{\pgfqpoint{5.933400in}{3.482699in}}%
\pgfpathcurveto{\pgfqpoint{5.939224in}{3.482699in}}{\pgfqpoint{5.944810in}{3.485013in}}{\pgfqpoint{5.948928in}{3.489131in}}%
\pgfpathcurveto{\pgfqpoint{5.953046in}{3.493249in}}{\pgfqpoint{5.955360in}{3.498836in}}{\pgfqpoint{5.955360in}{3.504659in}}%
\pgfpathcurveto{\pgfqpoint{5.955360in}{3.510483in}}{\pgfqpoint{5.953046in}{3.516070in}}{\pgfqpoint{5.948928in}{3.520188in}}%
\pgfpathcurveto{\pgfqpoint{5.944810in}{3.524306in}}{\pgfqpoint{5.939224in}{3.526620in}}{\pgfqpoint{5.933400in}{3.526620in}}%
\pgfpathcurveto{\pgfqpoint{5.927576in}{3.526620in}}{\pgfqpoint{5.921990in}{3.524306in}}{\pgfqpoint{5.917872in}{3.520188in}}%
\pgfpathcurveto{\pgfqpoint{5.913754in}{3.516070in}}{\pgfqpoint{5.911440in}{3.510483in}}{\pgfqpoint{5.911440in}{3.504659in}}%
\pgfpathcurveto{\pgfqpoint{5.911440in}{3.498836in}}{\pgfqpoint{5.913754in}{3.493249in}}{\pgfqpoint{5.917872in}{3.489131in}}%
\pgfpathcurveto{\pgfqpoint{5.921990in}{3.485013in}}{\pgfqpoint{5.927576in}{3.482699in}}{\pgfqpoint{5.933400in}{3.482699in}}%
\pgfpathlineto{\pgfqpoint{5.933400in}{3.482699in}}%
\pgfpathclose%
\pgfusepath{stroke,fill}%
\end{pgfscope}%
\begin{pgfscope}%
\pgfpathrectangle{\pgfqpoint{0.640323in}{0.527436in}}{\pgfqpoint{9.687500in}{3.850000in}}%
\pgfusepath{clip}%
\pgfsetbuttcap%
\pgfsetroundjoin%
\definecolor{currentfill}{rgb}{0.000000,0.000000,1.000000}%
\pgfsetfillcolor{currentfill}%
\pgfsetfillopacity{0.500000}%
\pgfsetlinewidth{1.003750pt}%
\definecolor{currentstroke}{rgb}{0.000000,0.000000,1.000000}%
\pgfsetstrokecolor{currentstroke}%
\pgfsetstrokeopacity{0.500000}%
\pgfsetdash{{3.700000pt}{1.600000pt}}{0.000000pt}%
\pgfpathmoveto{\pgfqpoint{6.113131in}{3.532985in}}%
\pgfpathcurveto{\pgfqpoint{6.118955in}{3.532985in}}{\pgfqpoint{6.124541in}{3.535299in}}{\pgfqpoint{6.128659in}{3.539417in}}%
\pgfpathcurveto{\pgfqpoint{6.132777in}{3.543535in}}{\pgfqpoint{6.135091in}{3.549122in}}{\pgfqpoint{6.135091in}{3.554945in}}%
\pgfpathcurveto{\pgfqpoint{6.135091in}{3.560769in}}{\pgfqpoint{6.132777in}{3.566356in}}{\pgfqpoint{6.128659in}{3.570474in}}%
\pgfpathcurveto{\pgfqpoint{6.124541in}{3.574592in}}{\pgfqpoint{6.118955in}{3.576906in}}{\pgfqpoint{6.113131in}{3.576906in}}%
\pgfpathcurveto{\pgfqpoint{6.107307in}{3.576906in}}{\pgfqpoint{6.101721in}{3.574592in}}{\pgfqpoint{6.097603in}{3.570474in}}%
\pgfpathcurveto{\pgfqpoint{6.093485in}{3.566356in}}{\pgfqpoint{6.091171in}{3.560769in}}{\pgfqpoint{6.091171in}{3.554945in}}%
\pgfpathcurveto{\pgfqpoint{6.091171in}{3.549122in}}{\pgfqpoint{6.093485in}{3.543535in}}{\pgfqpoint{6.097603in}{3.539417in}}%
\pgfpathcurveto{\pgfqpoint{6.101721in}{3.535299in}}{\pgfqpoint{6.107307in}{3.532985in}}{\pgfqpoint{6.113131in}{3.532985in}}%
\pgfpathlineto{\pgfqpoint{6.113131in}{3.532985in}}%
\pgfpathclose%
\pgfusepath{stroke,fill}%
\end{pgfscope}%
\begin{pgfscope}%
\pgfpathrectangle{\pgfqpoint{0.640323in}{0.527436in}}{\pgfqpoint{9.687500in}{3.850000in}}%
\pgfusepath{clip}%
\pgfsetbuttcap%
\pgfsetroundjoin%
\definecolor{currentfill}{rgb}{0.000000,0.000000,1.000000}%
\pgfsetfillcolor{currentfill}%
\pgfsetfillopacity{0.500000}%
\pgfsetlinewidth{1.003750pt}%
\definecolor{currentstroke}{rgb}{0.000000,0.000000,1.000000}%
\pgfsetstrokecolor{currentstroke}%
\pgfsetstrokeopacity{0.500000}%
\pgfsetdash{{3.700000pt}{1.600000pt}}{0.000000pt}%
\pgfpathmoveto{\pgfqpoint{6.292862in}{3.552657in}}%
\pgfpathcurveto{\pgfqpoint{6.298686in}{3.552657in}}{\pgfqpoint{6.304272in}{3.554971in}}{\pgfqpoint{6.308390in}{3.559089in}}%
\pgfpathcurveto{\pgfqpoint{6.312508in}{3.563208in}}{\pgfqpoint{6.314822in}{3.568794in}}{\pgfqpoint{6.314822in}{3.574618in}}%
\pgfpathcurveto{\pgfqpoint{6.314822in}{3.580442in}}{\pgfqpoint{6.312508in}{3.586028in}}{\pgfqpoint{6.308390in}{3.590146in}}%
\pgfpathcurveto{\pgfqpoint{6.304272in}{3.594264in}}{\pgfqpoint{6.298686in}{3.596578in}}{\pgfqpoint{6.292862in}{3.596578in}}%
\pgfpathcurveto{\pgfqpoint{6.287038in}{3.596578in}}{\pgfqpoint{6.281452in}{3.594264in}}{\pgfqpoint{6.277334in}{3.590146in}}%
\pgfpathcurveto{\pgfqpoint{6.273216in}{3.586028in}}{\pgfqpoint{6.270902in}{3.580442in}}{\pgfqpoint{6.270902in}{3.574618in}}%
\pgfpathcurveto{\pgfqpoint{6.270902in}{3.568794in}}{\pgfqpoint{6.273216in}{3.563208in}}{\pgfqpoint{6.277334in}{3.559089in}}%
\pgfpathcurveto{\pgfqpoint{6.281452in}{3.554971in}}{\pgfqpoint{6.287038in}{3.552657in}}{\pgfqpoint{6.292862in}{3.552657in}}%
\pgfpathlineto{\pgfqpoint{6.292862in}{3.552657in}}%
\pgfpathclose%
\pgfusepath{stroke,fill}%
\end{pgfscope}%
\begin{pgfscope}%
\pgfpathrectangle{\pgfqpoint{0.640323in}{0.527436in}}{\pgfqpoint{9.687500in}{3.850000in}}%
\pgfusepath{clip}%
\pgfsetbuttcap%
\pgfsetroundjoin%
\definecolor{currentfill}{rgb}{0.000000,0.000000,1.000000}%
\pgfsetfillcolor{currentfill}%
\pgfsetfillopacity{0.500000}%
\pgfsetlinewidth{1.003750pt}%
\definecolor{currentstroke}{rgb}{0.000000,0.000000,1.000000}%
\pgfsetstrokecolor{currentstroke}%
\pgfsetstrokeopacity{0.500000}%
\pgfsetdash{{3.700000pt}{1.600000pt}}{0.000000pt}%
\pgfpathmoveto{\pgfqpoint{6.472593in}{3.580868in}}%
\pgfpathcurveto{\pgfqpoint{6.478417in}{3.580868in}}{\pgfqpoint{6.484003in}{3.583182in}}{\pgfqpoint{6.488121in}{3.587300in}}%
\pgfpathcurveto{\pgfqpoint{6.492239in}{3.591418in}}{\pgfqpoint{6.494553in}{3.597004in}}{\pgfqpoint{6.494553in}{3.602828in}}%
\pgfpathcurveto{\pgfqpoint{6.494553in}{3.608652in}}{\pgfqpoint{6.492239in}{3.614238in}}{\pgfqpoint{6.488121in}{3.618357in}}%
\pgfpathcurveto{\pgfqpoint{6.484003in}{3.622475in}}{\pgfqpoint{6.478417in}{3.624789in}}{\pgfqpoint{6.472593in}{3.624789in}}%
\pgfpathcurveto{\pgfqpoint{6.466769in}{3.624789in}}{\pgfqpoint{6.461183in}{3.622475in}}{\pgfqpoint{6.457065in}{3.618357in}}%
\pgfpathcurveto{\pgfqpoint{6.452947in}{3.614238in}}{\pgfqpoint{6.450633in}{3.608652in}}{\pgfqpoint{6.450633in}{3.602828in}}%
\pgfpathcurveto{\pgfqpoint{6.450633in}{3.597004in}}{\pgfqpoint{6.452947in}{3.591418in}}{\pgfqpoint{6.457065in}{3.587300in}}%
\pgfpathcurveto{\pgfqpoint{6.461183in}{3.583182in}}{\pgfqpoint{6.466769in}{3.580868in}}{\pgfqpoint{6.472593in}{3.580868in}}%
\pgfpathlineto{\pgfqpoint{6.472593in}{3.580868in}}%
\pgfpathclose%
\pgfusepath{stroke,fill}%
\end{pgfscope}%
\begin{pgfscope}%
\pgfpathrectangle{\pgfqpoint{0.640323in}{0.527436in}}{\pgfqpoint{9.687500in}{3.850000in}}%
\pgfusepath{clip}%
\pgfsetbuttcap%
\pgfsetroundjoin%
\definecolor{currentfill}{rgb}{0.000000,0.000000,1.000000}%
\pgfsetfillcolor{currentfill}%
\pgfsetfillopacity{0.500000}%
\pgfsetlinewidth{1.003750pt}%
\definecolor{currentstroke}{rgb}{0.000000,0.000000,1.000000}%
\pgfsetstrokecolor{currentstroke}%
\pgfsetstrokeopacity{0.500000}%
\pgfsetdash{{3.700000pt}{1.600000pt}}{0.000000pt}%
\pgfpathmoveto{\pgfqpoint{6.652324in}{3.631173in}}%
\pgfpathcurveto{\pgfqpoint{6.658148in}{3.631173in}}{\pgfqpoint{6.663734in}{3.633486in}}{\pgfqpoint{6.667852in}{3.637605in}}%
\pgfpathcurveto{\pgfqpoint{6.671970in}{3.641723in}}{\pgfqpoint{6.674284in}{3.647309in}}{\pgfqpoint{6.674284in}{3.653133in}}%
\pgfpathcurveto{\pgfqpoint{6.674284in}{3.658957in}}{\pgfqpoint{6.671970in}{3.664543in}}{\pgfqpoint{6.667852in}{3.668661in}}%
\pgfpathcurveto{\pgfqpoint{6.663734in}{3.672779in}}{\pgfqpoint{6.658148in}{3.675093in}}{\pgfqpoint{6.652324in}{3.675093in}}%
\pgfpathcurveto{\pgfqpoint{6.646500in}{3.675093in}}{\pgfqpoint{6.640914in}{3.672779in}}{\pgfqpoint{6.636796in}{3.668661in}}%
\pgfpathcurveto{\pgfqpoint{6.632678in}{3.664543in}}{\pgfqpoint{6.630364in}{3.658957in}}{\pgfqpoint{6.630364in}{3.653133in}}%
\pgfpathcurveto{\pgfqpoint{6.630364in}{3.647309in}}{\pgfqpoint{6.632678in}{3.641723in}}{\pgfqpoint{6.636796in}{3.637605in}}%
\pgfpathcurveto{\pgfqpoint{6.640914in}{3.633486in}}{\pgfqpoint{6.646500in}{3.631173in}}{\pgfqpoint{6.652324in}{3.631173in}}%
\pgfpathlineto{\pgfqpoint{6.652324in}{3.631173in}}%
\pgfpathclose%
\pgfusepath{stroke,fill}%
\end{pgfscope}%
\begin{pgfscope}%
\pgfpathrectangle{\pgfqpoint{0.640323in}{0.527436in}}{\pgfqpoint{9.687500in}{3.850000in}}%
\pgfusepath{clip}%
\pgfsetbuttcap%
\pgfsetroundjoin%
\definecolor{currentfill}{rgb}{0.000000,0.000000,1.000000}%
\pgfsetfillcolor{currentfill}%
\pgfsetfillopacity{0.500000}%
\pgfsetlinewidth{1.003750pt}%
\definecolor{currentstroke}{rgb}{0.000000,0.000000,1.000000}%
\pgfsetstrokecolor{currentstroke}%
\pgfsetstrokeopacity{0.500000}%
\pgfsetdash{{3.700000pt}{1.600000pt}}{0.000000pt}%
\pgfpathmoveto{\pgfqpoint{6.832055in}{3.665158in}}%
\pgfpathcurveto{\pgfqpoint{6.837879in}{3.665158in}}{\pgfqpoint{6.843465in}{3.667472in}}{\pgfqpoint{6.847583in}{3.671590in}}%
\pgfpathcurveto{\pgfqpoint{6.851701in}{3.675708in}}{\pgfqpoint{6.854015in}{3.681294in}}{\pgfqpoint{6.854015in}{3.687118in}}%
\pgfpathcurveto{\pgfqpoint{6.854015in}{3.692942in}}{\pgfqpoint{6.851701in}{3.698529in}}{\pgfqpoint{6.847583in}{3.702647in}}%
\pgfpathcurveto{\pgfqpoint{6.843465in}{3.706765in}}{\pgfqpoint{6.837879in}{3.709079in}}{\pgfqpoint{6.832055in}{3.709079in}}%
\pgfpathcurveto{\pgfqpoint{6.826231in}{3.709079in}}{\pgfqpoint{6.820645in}{3.706765in}}{\pgfqpoint{6.816527in}{3.702647in}}%
\pgfpathcurveto{\pgfqpoint{6.812408in}{3.698529in}}{\pgfqpoint{6.810095in}{3.692942in}}{\pgfqpoint{6.810095in}{3.687118in}}%
\pgfpathcurveto{\pgfqpoint{6.810095in}{3.681294in}}{\pgfqpoint{6.812408in}{3.675708in}}{\pgfqpoint{6.816527in}{3.671590in}}%
\pgfpathcurveto{\pgfqpoint{6.820645in}{3.667472in}}{\pgfqpoint{6.826231in}{3.665158in}}{\pgfqpoint{6.832055in}{3.665158in}}%
\pgfpathlineto{\pgfqpoint{6.832055in}{3.665158in}}%
\pgfpathclose%
\pgfusepath{stroke,fill}%
\end{pgfscope}%
\begin{pgfscope}%
\pgfpathrectangle{\pgfqpoint{0.640323in}{0.527436in}}{\pgfqpoint{9.687500in}{3.850000in}}%
\pgfusepath{clip}%
\pgfsetbuttcap%
\pgfsetroundjoin%
\definecolor{currentfill}{rgb}{0.000000,0.000000,1.000000}%
\pgfsetfillcolor{currentfill}%
\pgfsetfillopacity{0.500000}%
\pgfsetlinewidth{1.003750pt}%
\definecolor{currentstroke}{rgb}{0.000000,0.000000,1.000000}%
\pgfsetstrokecolor{currentstroke}%
\pgfsetstrokeopacity{0.500000}%
\pgfsetdash{{3.700000pt}{1.600000pt}}{0.000000pt}%
\pgfpathmoveto{\pgfqpoint{7.011786in}{3.672411in}}%
\pgfpathcurveto{\pgfqpoint{7.017610in}{3.672411in}}{\pgfqpoint{7.023196in}{3.674725in}}{\pgfqpoint{7.027314in}{3.678843in}}%
\pgfpathcurveto{\pgfqpoint{7.031432in}{3.682961in}}{\pgfqpoint{7.033746in}{3.688547in}}{\pgfqpoint{7.033746in}{3.694371in}}%
\pgfpathcurveto{\pgfqpoint{7.033746in}{3.700195in}}{\pgfqpoint{7.031432in}{3.705781in}}{\pgfqpoint{7.027314in}{3.709900in}}%
\pgfpathcurveto{\pgfqpoint{7.023196in}{3.714018in}}{\pgfqpoint{7.017610in}{3.716332in}}{\pgfqpoint{7.011786in}{3.716332in}}%
\pgfpathcurveto{\pgfqpoint{7.005962in}{3.716332in}}{\pgfqpoint{7.000376in}{3.714018in}}{\pgfqpoint{6.996258in}{3.709900in}}%
\pgfpathcurveto{\pgfqpoint{6.992139in}{3.705781in}}{\pgfqpoint{6.989826in}{3.700195in}}{\pgfqpoint{6.989826in}{3.694371in}}%
\pgfpathcurveto{\pgfqpoint{6.989826in}{3.688547in}}{\pgfqpoint{6.992139in}{3.682961in}}{\pgfqpoint{6.996258in}{3.678843in}}%
\pgfpathcurveto{\pgfqpoint{7.000376in}{3.674725in}}{\pgfqpoint{7.005962in}{3.672411in}}{\pgfqpoint{7.011786in}{3.672411in}}%
\pgfpathlineto{\pgfqpoint{7.011786in}{3.672411in}}%
\pgfpathclose%
\pgfusepath{stroke,fill}%
\end{pgfscope}%
\begin{pgfscope}%
\pgfpathrectangle{\pgfqpoint{0.640323in}{0.527436in}}{\pgfqpoint{9.687500in}{3.850000in}}%
\pgfusepath{clip}%
\pgfsetbuttcap%
\pgfsetroundjoin%
\definecolor{currentfill}{rgb}{0.000000,0.000000,1.000000}%
\pgfsetfillcolor{currentfill}%
\pgfsetfillopacity{0.500000}%
\pgfsetlinewidth{1.003750pt}%
\definecolor{currentstroke}{rgb}{0.000000,0.000000,1.000000}%
\pgfsetstrokecolor{currentstroke}%
\pgfsetstrokeopacity{0.500000}%
\pgfsetdash{{3.700000pt}{1.600000pt}}{0.000000pt}%
\pgfpathmoveto{\pgfqpoint{7.191517in}{3.714860in}}%
\pgfpathcurveto{\pgfqpoint{7.197341in}{3.714860in}}{\pgfqpoint{7.202927in}{3.717174in}}{\pgfqpoint{7.207045in}{3.721292in}}%
\pgfpathcurveto{\pgfqpoint{7.211163in}{3.725411in}}{\pgfqpoint{7.213477in}{3.730997in}}{\pgfqpoint{7.213477in}{3.736821in}}%
\pgfpathcurveto{\pgfqpoint{7.213477in}{3.742645in}}{\pgfqpoint{7.211163in}{3.748231in}}{\pgfqpoint{7.207045in}{3.752349in}}%
\pgfpathcurveto{\pgfqpoint{7.202927in}{3.756467in}}{\pgfqpoint{7.197341in}{3.758781in}}{\pgfqpoint{7.191517in}{3.758781in}}%
\pgfpathcurveto{\pgfqpoint{7.185693in}{3.758781in}}{\pgfqpoint{7.180107in}{3.756467in}}{\pgfqpoint{7.175989in}{3.752349in}}%
\pgfpathcurveto{\pgfqpoint{7.171870in}{3.748231in}}{\pgfqpoint{7.169557in}{3.742645in}}{\pgfqpoint{7.169557in}{3.736821in}}%
\pgfpathcurveto{\pgfqpoint{7.169557in}{3.730997in}}{\pgfqpoint{7.171870in}{3.725411in}}{\pgfqpoint{7.175989in}{3.721292in}}%
\pgfpathcurveto{\pgfqpoint{7.180107in}{3.717174in}}{\pgfqpoint{7.185693in}{3.714860in}}{\pgfqpoint{7.191517in}{3.714860in}}%
\pgfpathlineto{\pgfqpoint{7.191517in}{3.714860in}}%
\pgfpathclose%
\pgfusepath{stroke,fill}%
\end{pgfscope}%
\begin{pgfscope}%
\pgfpathrectangle{\pgfqpoint{0.640323in}{0.527436in}}{\pgfqpoint{9.687500in}{3.850000in}}%
\pgfusepath{clip}%
\pgfsetbuttcap%
\pgfsetroundjoin%
\definecolor{currentfill}{rgb}{0.000000,0.000000,1.000000}%
\pgfsetfillcolor{currentfill}%
\pgfsetfillopacity{0.500000}%
\pgfsetlinewidth{1.003750pt}%
\definecolor{currentstroke}{rgb}{0.000000,0.000000,1.000000}%
\pgfsetstrokecolor{currentstroke}%
\pgfsetstrokeopacity{0.500000}%
\pgfsetdash{{3.700000pt}{1.600000pt}}{0.000000pt}%
\pgfpathmoveto{\pgfqpoint{7.371248in}{3.734508in}}%
\pgfpathcurveto{\pgfqpoint{7.377072in}{3.734508in}}{\pgfqpoint{7.382658in}{3.736822in}}{\pgfqpoint{7.386776in}{3.740940in}}%
\pgfpathcurveto{\pgfqpoint{7.390894in}{3.745058in}}{\pgfqpoint{7.393208in}{3.750644in}}{\pgfqpoint{7.393208in}{3.756468in}}%
\pgfpathcurveto{\pgfqpoint{7.393208in}{3.762292in}}{\pgfqpoint{7.390894in}{3.767878in}}{\pgfqpoint{7.386776in}{3.771996in}}%
\pgfpathcurveto{\pgfqpoint{7.382658in}{3.776114in}}{\pgfqpoint{7.377072in}{3.778428in}}{\pgfqpoint{7.371248in}{3.778428in}}%
\pgfpathcurveto{\pgfqpoint{7.365424in}{3.778428in}}{\pgfqpoint{7.359838in}{3.776114in}}{\pgfqpoint{7.355720in}{3.771996in}}%
\pgfpathcurveto{\pgfqpoint{7.351601in}{3.767878in}}{\pgfqpoint{7.349288in}{3.762292in}}{\pgfqpoint{7.349288in}{3.756468in}}%
\pgfpathcurveto{\pgfqpoint{7.349288in}{3.750644in}}{\pgfqpoint{7.351601in}{3.745058in}}{\pgfqpoint{7.355720in}{3.740940in}}%
\pgfpathcurveto{\pgfqpoint{7.359838in}{3.736822in}}{\pgfqpoint{7.365424in}{3.734508in}}{\pgfqpoint{7.371248in}{3.734508in}}%
\pgfpathlineto{\pgfqpoint{7.371248in}{3.734508in}}%
\pgfpathclose%
\pgfusepath{stroke,fill}%
\end{pgfscope}%
\begin{pgfscope}%
\pgfpathrectangle{\pgfqpoint{0.640323in}{0.527436in}}{\pgfqpoint{9.687500in}{3.850000in}}%
\pgfusepath{clip}%
\pgfsetbuttcap%
\pgfsetroundjoin%
\definecolor{currentfill}{rgb}{0.000000,0.000000,1.000000}%
\pgfsetfillcolor{currentfill}%
\pgfsetfillopacity{0.500000}%
\pgfsetlinewidth{1.003750pt}%
\definecolor{currentstroke}{rgb}{0.000000,0.000000,1.000000}%
\pgfsetstrokecolor{currentstroke}%
\pgfsetstrokeopacity{0.500000}%
\pgfsetdash{{3.700000pt}{1.600000pt}}{0.000000pt}%
\pgfpathmoveto{\pgfqpoint{7.550979in}{3.766407in}}%
\pgfpathcurveto{\pgfqpoint{7.556803in}{3.766407in}}{\pgfqpoint{7.562389in}{3.768721in}}{\pgfqpoint{7.566507in}{3.772839in}}%
\pgfpathcurveto{\pgfqpoint{7.570625in}{3.776957in}}{\pgfqpoint{7.572939in}{3.782543in}}{\pgfqpoint{7.572939in}{3.788367in}}%
\pgfpathcurveto{\pgfqpoint{7.572939in}{3.794191in}}{\pgfqpoint{7.570625in}{3.799777in}}{\pgfqpoint{7.566507in}{3.803895in}}%
\pgfpathcurveto{\pgfqpoint{7.562389in}{3.808014in}}{\pgfqpoint{7.556803in}{3.810327in}}{\pgfqpoint{7.550979in}{3.810327in}}%
\pgfpathcurveto{\pgfqpoint{7.545155in}{3.810327in}}{\pgfqpoint{7.539569in}{3.808014in}}{\pgfqpoint{7.535451in}{3.803895in}}%
\pgfpathcurveto{\pgfqpoint{7.531332in}{3.799777in}}{\pgfqpoint{7.529019in}{3.794191in}}{\pgfqpoint{7.529019in}{3.788367in}}%
\pgfpathcurveto{\pgfqpoint{7.529019in}{3.782543in}}{\pgfqpoint{7.531332in}{3.776957in}}{\pgfqpoint{7.535451in}{3.772839in}}%
\pgfpathcurveto{\pgfqpoint{7.539569in}{3.768721in}}{\pgfqpoint{7.545155in}{3.766407in}}{\pgfqpoint{7.550979in}{3.766407in}}%
\pgfpathlineto{\pgfqpoint{7.550979in}{3.766407in}}%
\pgfpathclose%
\pgfusepath{stroke,fill}%
\end{pgfscope}%
\begin{pgfscope}%
\pgfpathrectangle{\pgfqpoint{0.640323in}{0.527436in}}{\pgfqpoint{9.687500in}{3.850000in}}%
\pgfusepath{clip}%
\pgfsetbuttcap%
\pgfsetroundjoin%
\definecolor{currentfill}{rgb}{0.000000,0.000000,1.000000}%
\pgfsetfillcolor{currentfill}%
\pgfsetfillopacity{0.500000}%
\pgfsetlinewidth{1.003750pt}%
\definecolor{currentstroke}{rgb}{0.000000,0.000000,1.000000}%
\pgfsetstrokecolor{currentstroke}%
\pgfsetstrokeopacity{0.500000}%
\pgfsetdash{{3.700000pt}{1.600000pt}}{0.000000pt}%
\pgfpathmoveto{\pgfqpoint{7.730710in}{3.786104in}}%
\pgfpathcurveto{\pgfqpoint{7.736534in}{3.786104in}}{\pgfqpoint{7.742120in}{3.788418in}}{\pgfqpoint{7.746238in}{3.792536in}}%
\pgfpathcurveto{\pgfqpoint{7.750356in}{3.796654in}}{\pgfqpoint{7.752670in}{3.802240in}}{\pgfqpoint{7.752670in}{3.808064in}}%
\pgfpathcurveto{\pgfqpoint{7.752670in}{3.813888in}}{\pgfqpoint{7.750356in}{3.819474in}}{\pgfqpoint{7.746238in}{3.823593in}}%
\pgfpathcurveto{\pgfqpoint{7.742120in}{3.827711in}}{\pgfqpoint{7.736534in}{3.830025in}}{\pgfqpoint{7.730710in}{3.830025in}}%
\pgfpathcurveto{\pgfqpoint{7.724886in}{3.830025in}}{\pgfqpoint{7.719300in}{3.827711in}}{\pgfqpoint{7.715182in}{3.823593in}}%
\pgfpathcurveto{\pgfqpoint{7.711063in}{3.819474in}}{\pgfqpoint{7.708750in}{3.813888in}}{\pgfqpoint{7.708750in}{3.808064in}}%
\pgfpathcurveto{\pgfqpoint{7.708750in}{3.802240in}}{\pgfqpoint{7.711063in}{3.796654in}}{\pgfqpoint{7.715182in}{3.792536in}}%
\pgfpathcurveto{\pgfqpoint{7.719300in}{3.788418in}}{\pgfqpoint{7.724886in}{3.786104in}}{\pgfqpoint{7.730710in}{3.786104in}}%
\pgfpathlineto{\pgfqpoint{7.730710in}{3.786104in}}%
\pgfpathclose%
\pgfusepath{stroke,fill}%
\end{pgfscope}%
\begin{pgfscope}%
\pgfpathrectangle{\pgfqpoint{0.640323in}{0.527436in}}{\pgfqpoint{9.687500in}{3.850000in}}%
\pgfusepath{clip}%
\pgfsetbuttcap%
\pgfsetroundjoin%
\definecolor{currentfill}{rgb}{0.000000,0.000000,1.000000}%
\pgfsetfillcolor{currentfill}%
\pgfsetfillopacity{0.500000}%
\pgfsetlinewidth{1.003750pt}%
\definecolor{currentstroke}{rgb}{0.000000,0.000000,1.000000}%
\pgfsetstrokecolor{currentstroke}%
\pgfsetstrokeopacity{0.500000}%
\pgfsetdash{{3.700000pt}{1.600000pt}}{0.000000pt}%
\pgfpathmoveto{\pgfqpoint{7.910441in}{3.816420in}}%
\pgfpathcurveto{\pgfqpoint{7.916265in}{3.816420in}}{\pgfqpoint{7.921851in}{3.818734in}}{\pgfqpoint{7.925969in}{3.822852in}}%
\pgfpathcurveto{\pgfqpoint{7.930087in}{3.826970in}}{\pgfqpoint{7.932401in}{3.832556in}}{\pgfqpoint{7.932401in}{3.838380in}}%
\pgfpathcurveto{\pgfqpoint{7.932401in}{3.844204in}}{\pgfqpoint{7.930087in}{3.849790in}}{\pgfqpoint{7.925969in}{3.853908in}}%
\pgfpathcurveto{\pgfqpoint{7.921851in}{3.858026in}}{\pgfqpoint{7.916265in}{3.860340in}}{\pgfqpoint{7.910441in}{3.860340in}}%
\pgfpathcurveto{\pgfqpoint{7.904617in}{3.860340in}}{\pgfqpoint{7.899031in}{3.858026in}}{\pgfqpoint{7.894913in}{3.853908in}}%
\pgfpathcurveto{\pgfqpoint{7.890794in}{3.849790in}}{\pgfqpoint{7.888481in}{3.844204in}}{\pgfqpoint{7.888481in}{3.838380in}}%
\pgfpathcurveto{\pgfqpoint{7.888481in}{3.832556in}}{\pgfqpoint{7.890794in}{3.826970in}}{\pgfqpoint{7.894913in}{3.822852in}}%
\pgfpathcurveto{\pgfqpoint{7.899031in}{3.818734in}}{\pgfqpoint{7.904617in}{3.816420in}}{\pgfqpoint{7.910441in}{3.816420in}}%
\pgfpathlineto{\pgfqpoint{7.910441in}{3.816420in}}%
\pgfpathclose%
\pgfusepath{stroke,fill}%
\end{pgfscope}%
\begin{pgfscope}%
\pgfpathrectangle{\pgfqpoint{0.640323in}{0.527436in}}{\pgfqpoint{9.687500in}{3.850000in}}%
\pgfusepath{clip}%
\pgfsetbuttcap%
\pgfsetroundjoin%
\definecolor{currentfill}{rgb}{0.000000,0.000000,1.000000}%
\pgfsetfillcolor{currentfill}%
\pgfsetfillopacity{0.500000}%
\pgfsetlinewidth{1.003750pt}%
\definecolor{currentstroke}{rgb}{0.000000,0.000000,1.000000}%
\pgfsetstrokecolor{currentstroke}%
\pgfsetstrokeopacity{0.500000}%
\pgfsetdash{{3.700000pt}{1.600000pt}}{0.000000pt}%
\pgfpathmoveto{\pgfqpoint{8.090172in}{3.833912in}}%
\pgfpathcurveto{\pgfqpoint{8.095996in}{3.833912in}}{\pgfqpoint{8.101582in}{3.836226in}}{\pgfqpoint{8.105700in}{3.840344in}}%
\pgfpathcurveto{\pgfqpoint{8.109818in}{3.844462in}}{\pgfqpoint{8.112132in}{3.850049in}}{\pgfqpoint{8.112132in}{3.855873in}}%
\pgfpathcurveto{\pgfqpoint{8.112132in}{3.861697in}}{\pgfqpoint{8.109818in}{3.867283in}}{\pgfqpoint{8.105700in}{3.871401in}}%
\pgfpathcurveto{\pgfqpoint{8.101582in}{3.875519in}}{\pgfqpoint{8.095996in}{3.877833in}}{\pgfqpoint{8.090172in}{3.877833in}}%
\pgfpathcurveto{\pgfqpoint{8.084348in}{3.877833in}}{\pgfqpoint{8.078762in}{3.875519in}}{\pgfqpoint{8.074644in}{3.871401in}}%
\pgfpathcurveto{\pgfqpoint{8.070525in}{3.867283in}}{\pgfqpoint{8.068211in}{3.861697in}}{\pgfqpoint{8.068211in}{3.855873in}}%
\pgfpathcurveto{\pgfqpoint{8.068211in}{3.850049in}}{\pgfqpoint{8.070525in}{3.844462in}}{\pgfqpoint{8.074644in}{3.840344in}}%
\pgfpathcurveto{\pgfqpoint{8.078762in}{3.836226in}}{\pgfqpoint{8.084348in}{3.833912in}}{\pgfqpoint{8.090172in}{3.833912in}}%
\pgfpathlineto{\pgfqpoint{8.090172in}{3.833912in}}%
\pgfpathclose%
\pgfusepath{stroke,fill}%
\end{pgfscope}%
\begin{pgfscope}%
\pgfpathrectangle{\pgfqpoint{0.640323in}{0.527436in}}{\pgfqpoint{9.687500in}{3.850000in}}%
\pgfusepath{clip}%
\pgfsetbuttcap%
\pgfsetroundjoin%
\definecolor{currentfill}{rgb}{0.000000,0.000000,1.000000}%
\pgfsetfillcolor{currentfill}%
\pgfsetfillopacity{0.500000}%
\pgfsetlinewidth{1.003750pt}%
\definecolor{currentstroke}{rgb}{0.000000,0.000000,1.000000}%
\pgfsetstrokecolor{currentstroke}%
\pgfsetstrokeopacity{0.500000}%
\pgfsetdash{{3.700000pt}{1.600000pt}}{0.000000pt}%
\pgfpathmoveto{\pgfqpoint{8.269903in}{3.837135in}}%
\pgfpathcurveto{\pgfqpoint{8.275727in}{3.837135in}}{\pgfqpoint{8.281313in}{3.839449in}}{\pgfqpoint{8.285431in}{3.843567in}}%
\pgfpathcurveto{\pgfqpoint{8.289549in}{3.847685in}}{\pgfqpoint{8.291863in}{3.853272in}}{\pgfqpoint{8.291863in}{3.859095in}}%
\pgfpathcurveto{\pgfqpoint{8.291863in}{3.864919in}}{\pgfqpoint{8.289549in}{3.870506in}}{\pgfqpoint{8.285431in}{3.874624in}}%
\pgfpathcurveto{\pgfqpoint{8.281313in}{3.878742in}}{\pgfqpoint{8.275727in}{3.881056in}}{\pgfqpoint{8.269903in}{3.881056in}}%
\pgfpathcurveto{\pgfqpoint{8.264079in}{3.881056in}}{\pgfqpoint{8.258493in}{3.878742in}}{\pgfqpoint{8.254374in}{3.874624in}}%
\pgfpathcurveto{\pgfqpoint{8.250256in}{3.870506in}}{\pgfqpoint{8.247942in}{3.864919in}}{\pgfqpoint{8.247942in}{3.859095in}}%
\pgfpathcurveto{\pgfqpoint{8.247942in}{3.853272in}}{\pgfqpoint{8.250256in}{3.847685in}}{\pgfqpoint{8.254374in}{3.843567in}}%
\pgfpathcurveto{\pgfqpoint{8.258493in}{3.839449in}}{\pgfqpoint{8.264079in}{3.837135in}}{\pgfqpoint{8.269903in}{3.837135in}}%
\pgfpathlineto{\pgfqpoint{8.269903in}{3.837135in}}%
\pgfpathclose%
\pgfusepath{stroke,fill}%
\end{pgfscope}%
\begin{pgfscope}%
\pgfpathrectangle{\pgfqpoint{0.640323in}{0.527436in}}{\pgfqpoint{9.687500in}{3.850000in}}%
\pgfusepath{clip}%
\pgfsetbuttcap%
\pgfsetroundjoin%
\definecolor{currentfill}{rgb}{0.000000,0.000000,1.000000}%
\pgfsetfillcolor{currentfill}%
\pgfsetfillopacity{0.500000}%
\pgfsetlinewidth{1.003750pt}%
\definecolor{currentstroke}{rgb}{0.000000,0.000000,1.000000}%
\pgfsetstrokecolor{currentstroke}%
\pgfsetstrokeopacity{0.500000}%
\pgfsetdash{{3.700000pt}{1.600000pt}}{0.000000pt}%
\pgfpathmoveto{\pgfqpoint{8.449634in}{3.871580in}}%
\pgfpathcurveto{\pgfqpoint{8.455458in}{3.871580in}}{\pgfqpoint{8.461044in}{3.873894in}}{\pgfqpoint{8.465162in}{3.878012in}}%
\pgfpathcurveto{\pgfqpoint{8.469280in}{3.882130in}}{\pgfqpoint{8.471594in}{3.887717in}}{\pgfqpoint{8.471594in}{3.893541in}}%
\pgfpathcurveto{\pgfqpoint{8.471594in}{3.899364in}}{\pgfqpoint{8.469280in}{3.904951in}}{\pgfqpoint{8.465162in}{3.909069in}}%
\pgfpathcurveto{\pgfqpoint{8.461044in}{3.913187in}}{\pgfqpoint{8.455458in}{3.915501in}}{\pgfqpoint{8.449634in}{3.915501in}}%
\pgfpathcurveto{\pgfqpoint{8.443810in}{3.915501in}}{\pgfqpoint{8.438224in}{3.913187in}}{\pgfqpoint{8.434105in}{3.909069in}}%
\pgfpathcurveto{\pgfqpoint{8.429987in}{3.904951in}}{\pgfqpoint{8.427673in}{3.899364in}}{\pgfqpoint{8.427673in}{3.893541in}}%
\pgfpathcurveto{\pgfqpoint{8.427673in}{3.887717in}}{\pgfqpoint{8.429987in}{3.882130in}}{\pgfqpoint{8.434105in}{3.878012in}}%
\pgfpathcurveto{\pgfqpoint{8.438224in}{3.873894in}}{\pgfqpoint{8.443810in}{3.871580in}}{\pgfqpoint{8.449634in}{3.871580in}}%
\pgfpathlineto{\pgfqpoint{8.449634in}{3.871580in}}%
\pgfpathclose%
\pgfusepath{stroke,fill}%
\end{pgfscope}%
\begin{pgfscope}%
\pgfpathrectangle{\pgfqpoint{0.640323in}{0.527436in}}{\pgfqpoint{9.687500in}{3.850000in}}%
\pgfusepath{clip}%
\pgfsetbuttcap%
\pgfsetroundjoin%
\definecolor{currentfill}{rgb}{0.000000,0.000000,1.000000}%
\pgfsetfillcolor{currentfill}%
\pgfsetfillopacity{0.500000}%
\pgfsetlinewidth{1.003750pt}%
\definecolor{currentstroke}{rgb}{0.000000,0.000000,1.000000}%
\pgfsetstrokecolor{currentstroke}%
\pgfsetstrokeopacity{0.500000}%
\pgfsetdash{{3.700000pt}{1.600000pt}}{0.000000pt}%
\pgfpathmoveto{\pgfqpoint{8.629365in}{3.888514in}}%
\pgfpathcurveto{\pgfqpoint{8.635189in}{3.888514in}}{\pgfqpoint{8.640775in}{3.890828in}}{\pgfqpoint{8.644893in}{3.894946in}}%
\pgfpathcurveto{\pgfqpoint{8.649011in}{3.899064in}}{\pgfqpoint{8.651325in}{3.904650in}}{\pgfqpoint{8.651325in}{3.910474in}}%
\pgfpathcurveto{\pgfqpoint{8.651325in}{3.916298in}}{\pgfqpoint{8.649011in}{3.921884in}}{\pgfqpoint{8.644893in}{3.926003in}}%
\pgfpathcurveto{\pgfqpoint{8.640775in}{3.930121in}}{\pgfqpoint{8.635189in}{3.932435in}}{\pgfqpoint{8.629365in}{3.932435in}}%
\pgfpathcurveto{\pgfqpoint{8.623541in}{3.932435in}}{\pgfqpoint{8.617955in}{3.930121in}}{\pgfqpoint{8.613836in}{3.926003in}}%
\pgfpathcurveto{\pgfqpoint{8.609718in}{3.921884in}}{\pgfqpoint{8.607404in}{3.916298in}}{\pgfqpoint{8.607404in}{3.910474in}}%
\pgfpathcurveto{\pgfqpoint{8.607404in}{3.904650in}}{\pgfqpoint{8.609718in}{3.899064in}}{\pgfqpoint{8.613836in}{3.894946in}}%
\pgfpathcurveto{\pgfqpoint{8.617955in}{3.890828in}}{\pgfqpoint{8.623541in}{3.888514in}}{\pgfqpoint{8.629365in}{3.888514in}}%
\pgfpathlineto{\pgfqpoint{8.629365in}{3.888514in}}%
\pgfpathclose%
\pgfusepath{stroke,fill}%
\end{pgfscope}%
\begin{pgfscope}%
\pgfpathrectangle{\pgfqpoint{0.640323in}{0.527436in}}{\pgfqpoint{9.687500in}{3.850000in}}%
\pgfusepath{clip}%
\pgfsetbuttcap%
\pgfsetroundjoin%
\definecolor{currentfill}{rgb}{0.000000,0.000000,1.000000}%
\pgfsetfillcolor{currentfill}%
\pgfsetfillopacity{0.500000}%
\pgfsetlinewidth{1.003750pt}%
\definecolor{currentstroke}{rgb}{0.000000,0.000000,1.000000}%
\pgfsetstrokecolor{currentstroke}%
\pgfsetstrokeopacity{0.500000}%
\pgfsetdash{{3.700000pt}{1.600000pt}}{0.000000pt}%
\pgfpathmoveto{\pgfqpoint{8.809096in}{3.899089in}}%
\pgfpathcurveto{\pgfqpoint{8.814920in}{3.899089in}}{\pgfqpoint{8.820506in}{3.901403in}}{\pgfqpoint{8.824624in}{3.905521in}}%
\pgfpathcurveto{\pgfqpoint{8.828742in}{3.909639in}}{\pgfqpoint{8.831056in}{3.915225in}}{\pgfqpoint{8.831056in}{3.921049in}}%
\pgfpathcurveto{\pgfqpoint{8.831056in}{3.926873in}}{\pgfqpoint{8.828742in}{3.932459in}}{\pgfqpoint{8.824624in}{3.936578in}}%
\pgfpathcurveto{\pgfqpoint{8.820506in}{3.940696in}}{\pgfqpoint{8.814920in}{3.943010in}}{\pgfqpoint{8.809096in}{3.943010in}}%
\pgfpathcurveto{\pgfqpoint{8.803272in}{3.943010in}}{\pgfqpoint{8.797686in}{3.940696in}}{\pgfqpoint{8.793567in}{3.936578in}}%
\pgfpathcurveto{\pgfqpoint{8.789449in}{3.932459in}}{\pgfqpoint{8.787135in}{3.926873in}}{\pgfqpoint{8.787135in}{3.921049in}}%
\pgfpathcurveto{\pgfqpoint{8.787135in}{3.915225in}}{\pgfqpoint{8.789449in}{3.909639in}}{\pgfqpoint{8.793567in}{3.905521in}}%
\pgfpathcurveto{\pgfqpoint{8.797686in}{3.901403in}}{\pgfqpoint{8.803272in}{3.899089in}}{\pgfqpoint{8.809096in}{3.899089in}}%
\pgfpathlineto{\pgfqpoint{8.809096in}{3.899089in}}%
\pgfpathclose%
\pgfusepath{stroke,fill}%
\end{pgfscope}%
\begin{pgfscope}%
\pgfpathrectangle{\pgfqpoint{0.640323in}{0.527436in}}{\pgfqpoint{9.687500in}{3.850000in}}%
\pgfusepath{clip}%
\pgfsetbuttcap%
\pgfsetroundjoin%
\definecolor{currentfill}{rgb}{0.000000,0.000000,1.000000}%
\pgfsetfillcolor{currentfill}%
\pgfsetfillopacity{0.500000}%
\pgfsetlinewidth{1.003750pt}%
\definecolor{currentstroke}{rgb}{0.000000,0.000000,1.000000}%
\pgfsetstrokecolor{currentstroke}%
\pgfsetstrokeopacity{0.500000}%
\pgfsetdash{{3.700000pt}{1.600000pt}}{0.000000pt}%
\pgfpathmoveto{\pgfqpoint{8.988827in}{3.928268in}}%
\pgfpathcurveto{\pgfqpoint{8.994651in}{3.928268in}}{\pgfqpoint{9.000237in}{3.930582in}}{\pgfqpoint{9.004355in}{3.934700in}}%
\pgfpathcurveto{\pgfqpoint{9.008473in}{3.938819in}}{\pgfqpoint{9.010787in}{3.944405in}}{\pgfqpoint{9.010787in}{3.950229in}}%
\pgfpathcurveto{\pgfqpoint{9.010787in}{3.956053in}}{\pgfqpoint{9.008473in}{3.961639in}}{\pgfqpoint{9.004355in}{3.965757in}}%
\pgfpathcurveto{\pgfqpoint{9.000237in}{3.969875in}}{\pgfqpoint{8.994651in}{3.972189in}}{\pgfqpoint{8.988827in}{3.972189in}}%
\pgfpathcurveto{\pgfqpoint{8.983003in}{3.972189in}}{\pgfqpoint{8.977417in}{3.969875in}}{\pgfqpoint{8.973298in}{3.965757in}}%
\pgfpathcurveto{\pgfqpoint{8.969180in}{3.961639in}}{\pgfqpoint{8.966866in}{3.956053in}}{\pgfqpoint{8.966866in}{3.950229in}}%
\pgfpathcurveto{\pgfqpoint{8.966866in}{3.944405in}}{\pgfqpoint{8.969180in}{3.938819in}}{\pgfqpoint{8.973298in}{3.934700in}}%
\pgfpathcurveto{\pgfqpoint{8.977417in}{3.930582in}}{\pgfqpoint{8.983003in}{3.928268in}}{\pgfqpoint{8.988827in}{3.928268in}}%
\pgfpathlineto{\pgfqpoint{8.988827in}{3.928268in}}%
\pgfpathclose%
\pgfusepath{stroke,fill}%
\end{pgfscope}%
\begin{pgfscope}%
\pgfpathrectangle{\pgfqpoint{0.640323in}{0.527436in}}{\pgfqpoint{9.687500in}{3.850000in}}%
\pgfusepath{clip}%
\pgfsetbuttcap%
\pgfsetroundjoin%
\definecolor{currentfill}{rgb}{0.000000,0.000000,1.000000}%
\pgfsetfillcolor{currentfill}%
\pgfsetfillopacity{0.500000}%
\pgfsetlinewidth{1.003750pt}%
\definecolor{currentstroke}{rgb}{0.000000,0.000000,1.000000}%
\pgfsetstrokecolor{currentstroke}%
\pgfsetstrokeopacity{0.500000}%
\pgfsetdash{{3.700000pt}{1.600000pt}}{0.000000pt}%
\pgfpathmoveto{\pgfqpoint{9.168558in}{3.937900in}}%
\pgfpathcurveto{\pgfqpoint{9.174382in}{3.937900in}}{\pgfqpoint{9.179968in}{3.940213in}}{\pgfqpoint{9.184086in}{3.944332in}}%
\pgfpathcurveto{\pgfqpoint{9.188204in}{3.948450in}}{\pgfqpoint{9.190518in}{3.954036in}}{\pgfqpoint{9.190518in}{3.959860in}}%
\pgfpathcurveto{\pgfqpoint{9.190518in}{3.965684in}}{\pgfqpoint{9.188204in}{3.971270in}}{\pgfqpoint{9.184086in}{3.975388in}}%
\pgfpathcurveto{\pgfqpoint{9.179968in}{3.979506in}}{\pgfqpoint{9.174382in}{3.981820in}}{\pgfqpoint{9.168558in}{3.981820in}}%
\pgfpathcurveto{\pgfqpoint{9.162734in}{3.981820in}}{\pgfqpoint{9.157148in}{3.979506in}}{\pgfqpoint{9.153029in}{3.975388in}}%
\pgfpathcurveto{\pgfqpoint{9.148911in}{3.971270in}}{\pgfqpoint{9.146597in}{3.965684in}}{\pgfqpoint{9.146597in}{3.959860in}}%
\pgfpathcurveto{\pgfqpoint{9.146597in}{3.954036in}}{\pgfqpoint{9.148911in}{3.948450in}}{\pgfqpoint{9.153029in}{3.944332in}}%
\pgfpathcurveto{\pgfqpoint{9.157148in}{3.940213in}}{\pgfqpoint{9.162734in}{3.937900in}}{\pgfqpoint{9.168558in}{3.937900in}}%
\pgfpathlineto{\pgfqpoint{9.168558in}{3.937900in}}%
\pgfpathclose%
\pgfusepath{stroke,fill}%
\end{pgfscope}%
\begin{pgfscope}%
\pgfpathrectangle{\pgfqpoint{0.640323in}{0.527436in}}{\pgfqpoint{9.687500in}{3.850000in}}%
\pgfusepath{clip}%
\pgfsetbuttcap%
\pgfsetroundjoin%
\definecolor{currentfill}{rgb}{0.000000,0.000000,1.000000}%
\pgfsetfillcolor{currentfill}%
\pgfsetfillopacity{0.500000}%
\pgfsetlinewidth{1.003750pt}%
\definecolor{currentstroke}{rgb}{0.000000,0.000000,1.000000}%
\pgfsetstrokecolor{currentstroke}%
\pgfsetstrokeopacity{0.500000}%
\pgfsetdash{{3.700000pt}{1.600000pt}}{0.000000pt}%
\pgfpathmoveto{\pgfqpoint{9.348289in}{3.950685in}}%
\pgfpathcurveto{\pgfqpoint{9.354113in}{3.950685in}}{\pgfqpoint{9.359699in}{3.952999in}}{\pgfqpoint{9.363817in}{3.957117in}}%
\pgfpathcurveto{\pgfqpoint{9.367935in}{3.961235in}}{\pgfqpoint{9.370249in}{3.966822in}}{\pgfqpoint{9.370249in}{3.972646in}}%
\pgfpathcurveto{\pgfqpoint{9.370249in}{3.978470in}}{\pgfqpoint{9.367935in}{3.984056in}}{\pgfqpoint{9.363817in}{3.988174in}}%
\pgfpathcurveto{\pgfqpoint{9.359699in}{3.992292in}}{\pgfqpoint{9.354113in}{3.994606in}}{\pgfqpoint{9.348289in}{3.994606in}}%
\pgfpathcurveto{\pgfqpoint{9.342465in}{3.994606in}}{\pgfqpoint{9.336879in}{3.992292in}}{\pgfqpoint{9.332760in}{3.988174in}}%
\pgfpathcurveto{\pgfqpoint{9.328642in}{3.984056in}}{\pgfqpoint{9.326328in}{3.978470in}}{\pgfqpoint{9.326328in}{3.972646in}}%
\pgfpathcurveto{\pgfqpoint{9.326328in}{3.966822in}}{\pgfqpoint{9.328642in}{3.961235in}}{\pgfqpoint{9.332760in}{3.957117in}}%
\pgfpathcurveto{\pgfqpoint{9.336879in}{3.952999in}}{\pgfqpoint{9.342465in}{3.950685in}}{\pgfqpoint{9.348289in}{3.950685in}}%
\pgfpathlineto{\pgfqpoint{9.348289in}{3.950685in}}%
\pgfpathclose%
\pgfusepath{stroke,fill}%
\end{pgfscope}%
\begin{pgfscope}%
\pgfpathrectangle{\pgfqpoint{0.640323in}{0.527436in}}{\pgfqpoint{9.687500in}{3.850000in}}%
\pgfusepath{clip}%
\pgfsetbuttcap%
\pgfsetroundjoin%
\definecolor{currentfill}{rgb}{0.000000,0.000000,1.000000}%
\pgfsetfillcolor{currentfill}%
\pgfsetfillopacity{0.500000}%
\pgfsetlinewidth{1.003750pt}%
\definecolor{currentstroke}{rgb}{0.000000,0.000000,1.000000}%
\pgfsetstrokecolor{currentstroke}%
\pgfsetstrokeopacity{0.500000}%
\pgfsetdash{{3.700000pt}{1.600000pt}}{0.000000pt}%
\pgfpathmoveto{\pgfqpoint{9.528020in}{3.971419in}}%
\pgfpathcurveto{\pgfqpoint{9.533844in}{3.971419in}}{\pgfqpoint{9.539430in}{3.973733in}}{\pgfqpoint{9.543548in}{3.977851in}}%
\pgfpathcurveto{\pgfqpoint{9.547666in}{3.981970in}}{\pgfqpoint{9.549980in}{3.987556in}}{\pgfqpoint{9.549980in}{3.993380in}}%
\pgfpathcurveto{\pgfqpoint{9.549980in}{3.999204in}}{\pgfqpoint{9.547666in}{4.004790in}}{\pgfqpoint{9.543548in}{4.008908in}}%
\pgfpathcurveto{\pgfqpoint{9.539430in}{4.013026in}}{\pgfqpoint{9.533844in}{4.015340in}}{\pgfqpoint{9.528020in}{4.015340in}}%
\pgfpathcurveto{\pgfqpoint{9.522196in}{4.015340in}}{\pgfqpoint{9.516610in}{4.013026in}}{\pgfqpoint{9.512491in}{4.008908in}}%
\pgfpathcurveto{\pgfqpoint{9.508373in}{4.004790in}}{\pgfqpoint{9.506059in}{3.999204in}}{\pgfqpoint{9.506059in}{3.993380in}}%
\pgfpathcurveto{\pgfqpoint{9.506059in}{3.987556in}}{\pgfqpoint{9.508373in}{3.981970in}}{\pgfqpoint{9.512491in}{3.977851in}}%
\pgfpathcurveto{\pgfqpoint{9.516610in}{3.973733in}}{\pgfqpoint{9.522196in}{3.971419in}}{\pgfqpoint{9.528020in}{3.971419in}}%
\pgfpathlineto{\pgfqpoint{9.528020in}{3.971419in}}%
\pgfpathclose%
\pgfusepath{stroke,fill}%
\end{pgfscope}%
\begin{pgfscope}%
\pgfpathrectangle{\pgfqpoint{0.640323in}{0.527436in}}{\pgfqpoint{9.687500in}{3.850000in}}%
\pgfusepath{clip}%
\pgfsetbuttcap%
\pgfsetroundjoin%
\definecolor{currentfill}{rgb}{0.000000,0.000000,1.000000}%
\pgfsetfillcolor{currentfill}%
\pgfsetfillopacity{0.500000}%
\pgfsetlinewidth{1.003750pt}%
\definecolor{currentstroke}{rgb}{0.000000,0.000000,1.000000}%
\pgfsetstrokecolor{currentstroke}%
\pgfsetstrokeopacity{0.500000}%
\pgfsetdash{{3.700000pt}{1.600000pt}}{0.000000pt}%
\pgfpathmoveto{\pgfqpoint{9.707751in}{4.004070in}}%
\pgfpathcurveto{\pgfqpoint{9.713575in}{4.004070in}}{\pgfqpoint{9.719161in}{4.006384in}}{\pgfqpoint{9.723279in}{4.010502in}}%
\pgfpathcurveto{\pgfqpoint{9.727397in}{4.014620in}}{\pgfqpoint{9.729711in}{4.020206in}}{\pgfqpoint{9.729711in}{4.026030in}}%
\pgfpathcurveto{\pgfqpoint{9.729711in}{4.031854in}}{\pgfqpoint{9.727397in}{4.037440in}}{\pgfqpoint{9.723279in}{4.041558in}}%
\pgfpathcurveto{\pgfqpoint{9.719161in}{4.045677in}}{\pgfqpoint{9.713575in}{4.047990in}}{\pgfqpoint{9.707751in}{4.047990in}}%
\pgfpathcurveto{\pgfqpoint{9.701927in}{4.047990in}}{\pgfqpoint{9.696340in}{4.045677in}}{\pgfqpoint{9.692222in}{4.041558in}}%
\pgfpathcurveto{\pgfqpoint{9.688104in}{4.037440in}}{\pgfqpoint{9.685790in}{4.031854in}}{\pgfqpoint{9.685790in}{4.026030in}}%
\pgfpathcurveto{\pgfqpoint{9.685790in}{4.020206in}}{\pgfqpoint{9.688104in}{4.014620in}}{\pgfqpoint{9.692222in}{4.010502in}}%
\pgfpathcurveto{\pgfqpoint{9.696340in}{4.006384in}}{\pgfqpoint{9.701927in}{4.004070in}}{\pgfqpoint{9.707751in}{4.004070in}}%
\pgfpathlineto{\pgfqpoint{9.707751in}{4.004070in}}%
\pgfpathclose%
\pgfusepath{stroke,fill}%
\end{pgfscope}%
\begin{pgfscope}%
\pgfpathrectangle{\pgfqpoint{0.640323in}{0.527436in}}{\pgfqpoint{9.687500in}{3.850000in}}%
\pgfusepath{clip}%
\pgfsetbuttcap%
\pgfsetroundjoin%
\definecolor{currentfill}{rgb}{0.000000,0.000000,1.000000}%
\pgfsetfillcolor{currentfill}%
\pgfsetfillopacity{0.500000}%
\pgfsetlinewidth{1.003750pt}%
\definecolor{currentstroke}{rgb}{0.000000,0.000000,1.000000}%
\pgfsetstrokecolor{currentstroke}%
\pgfsetstrokeopacity{0.500000}%
\pgfsetdash{{3.700000pt}{1.600000pt}}{0.000000pt}%
\pgfpathmoveto{\pgfqpoint{9.887482in}{4.000456in}}%
\pgfpathcurveto{\pgfqpoint{9.893306in}{4.000456in}}{\pgfqpoint{9.898892in}{4.002770in}}{\pgfqpoint{9.903010in}{4.006888in}}%
\pgfpathcurveto{\pgfqpoint{9.907128in}{4.011006in}}{\pgfqpoint{9.909442in}{4.016592in}}{\pgfqpoint{9.909442in}{4.022416in}}%
\pgfpathcurveto{\pgfqpoint{9.909442in}{4.028240in}}{\pgfqpoint{9.907128in}{4.033826in}}{\pgfqpoint{9.903010in}{4.037944in}}%
\pgfpathcurveto{\pgfqpoint{9.898892in}{4.042063in}}{\pgfqpoint{9.893306in}{4.044376in}}{\pgfqpoint{9.887482in}{4.044376in}}%
\pgfpathcurveto{\pgfqpoint{9.881658in}{4.044376in}}{\pgfqpoint{9.876071in}{4.042063in}}{\pgfqpoint{9.871953in}{4.037944in}}%
\pgfpathcurveto{\pgfqpoint{9.867835in}{4.033826in}}{\pgfqpoint{9.865521in}{4.028240in}}{\pgfqpoint{9.865521in}{4.022416in}}%
\pgfpathcurveto{\pgfqpoint{9.865521in}{4.016592in}}{\pgfqpoint{9.867835in}{4.011006in}}{\pgfqpoint{9.871953in}{4.006888in}}%
\pgfpathcurveto{\pgfqpoint{9.876071in}{4.002770in}}{\pgfqpoint{9.881658in}{4.000456in}}{\pgfqpoint{9.887482in}{4.000456in}}%
\pgfpathlineto{\pgfqpoint{9.887482in}{4.000456in}}%
\pgfpathclose%
\pgfusepath{stroke,fill}%
\end{pgfscope}%
\begin{pgfscope}%
\pgfpathrectangle{\pgfqpoint{0.640323in}{0.527436in}}{\pgfqpoint{9.687500in}{3.850000in}}%
\pgfusepath{clip}%
\pgfsetbuttcap%
\pgfsetroundjoin%
\definecolor{currentfill}{rgb}{0.980392,0.164706,0.333333}%
\pgfsetfillcolor{currentfill}%
\pgfsetfillopacity{0.500000}%
\pgfsetlinewidth{1.003750pt}%
\definecolor{currentstroke}{rgb}{0.980392,0.164706,0.333333}%
\pgfsetstrokecolor{currentstroke}%
\pgfsetstrokeopacity{0.500000}%
\pgfsetdash{{3.700000pt}{1.600000pt}}{0.000000pt}%
\pgfpathmoveto{\pgfqpoint{1.080663in}{0.637874in}}%
\pgfpathcurveto{\pgfqpoint{1.086487in}{0.637874in}}{\pgfqpoint{1.092074in}{0.640188in}}{\pgfqpoint{1.096192in}{0.644306in}}%
\pgfpathcurveto{\pgfqpoint{1.100310in}{0.648424in}}{\pgfqpoint{1.102624in}{0.654010in}}{\pgfqpoint{1.102624in}{0.659834in}}%
\pgfpathcurveto{\pgfqpoint{1.102624in}{0.665658in}}{\pgfqpoint{1.100310in}{0.671244in}}{\pgfqpoint{1.096192in}{0.675362in}}%
\pgfpathcurveto{\pgfqpoint{1.092074in}{0.679481in}}{\pgfqpoint{1.086487in}{0.681794in}}{\pgfqpoint{1.080663in}{0.681794in}}%
\pgfpathcurveto{\pgfqpoint{1.074839in}{0.681794in}}{\pgfqpoint{1.069253in}{0.679481in}}{\pgfqpoint{1.065135in}{0.675362in}}%
\pgfpathcurveto{\pgfqpoint{1.061017in}{0.671244in}}{\pgfqpoint{1.058703in}{0.665658in}}{\pgfqpoint{1.058703in}{0.659834in}}%
\pgfpathcurveto{\pgfqpoint{1.058703in}{0.654010in}}{\pgfqpoint{1.061017in}{0.648424in}}{\pgfqpoint{1.065135in}{0.644306in}}%
\pgfpathcurveto{\pgfqpoint{1.069253in}{0.640188in}}{\pgfqpoint{1.074839in}{0.637874in}}{\pgfqpoint{1.080663in}{0.637874in}}%
\pgfpathlineto{\pgfqpoint{1.080663in}{0.637874in}}%
\pgfpathclose%
\pgfusepath{stroke,fill}%
\end{pgfscope}%
\begin{pgfscope}%
\pgfpathrectangle{\pgfqpoint{0.640323in}{0.527436in}}{\pgfqpoint{9.687500in}{3.850000in}}%
\pgfusepath{clip}%
\pgfsetbuttcap%
\pgfsetroundjoin%
\definecolor{currentfill}{rgb}{0.980392,0.164706,0.333333}%
\pgfsetfillcolor{currentfill}%
\pgfsetfillopacity{0.500000}%
\pgfsetlinewidth{1.003750pt}%
\definecolor{currentstroke}{rgb}{0.980392,0.164706,0.333333}%
\pgfsetstrokecolor{currentstroke}%
\pgfsetstrokeopacity{0.500000}%
\pgfsetdash{{3.700000pt}{1.600000pt}}{0.000000pt}%
\pgfpathmoveto{\pgfqpoint{1.260394in}{0.638941in}}%
\pgfpathcurveto{\pgfqpoint{1.266218in}{0.638941in}}{\pgfqpoint{1.271805in}{0.641255in}}{\pgfqpoint{1.275923in}{0.645373in}}%
\pgfpathcurveto{\pgfqpoint{1.280041in}{0.649491in}}{\pgfqpoint{1.282355in}{0.655077in}}{\pgfqpoint{1.282355in}{0.660901in}}%
\pgfpathcurveto{\pgfqpoint{1.282355in}{0.666725in}}{\pgfqpoint{1.280041in}{0.672312in}}{\pgfqpoint{1.275923in}{0.676430in}}%
\pgfpathcurveto{\pgfqpoint{1.271805in}{0.680548in}}{\pgfqpoint{1.266218in}{0.682862in}}{\pgfqpoint{1.260394in}{0.682862in}}%
\pgfpathcurveto{\pgfqpoint{1.254570in}{0.682862in}}{\pgfqpoint{1.248984in}{0.680548in}}{\pgfqpoint{1.244866in}{0.676430in}}%
\pgfpathcurveto{\pgfqpoint{1.240748in}{0.672312in}}{\pgfqpoint{1.238434in}{0.666725in}}{\pgfqpoint{1.238434in}{0.660901in}}%
\pgfpathcurveto{\pgfqpoint{1.238434in}{0.655077in}}{\pgfqpoint{1.240748in}{0.649491in}}{\pgfqpoint{1.244866in}{0.645373in}}%
\pgfpathcurveto{\pgfqpoint{1.248984in}{0.641255in}}{\pgfqpoint{1.254570in}{0.638941in}}{\pgfqpoint{1.260394in}{0.638941in}}%
\pgfpathlineto{\pgfqpoint{1.260394in}{0.638941in}}%
\pgfpathclose%
\pgfusepath{stroke,fill}%
\end{pgfscope}%
\begin{pgfscope}%
\pgfpathrectangle{\pgfqpoint{0.640323in}{0.527436in}}{\pgfqpoint{9.687500in}{3.850000in}}%
\pgfusepath{clip}%
\pgfsetbuttcap%
\pgfsetroundjoin%
\definecolor{currentfill}{rgb}{0.980392,0.164706,0.333333}%
\pgfsetfillcolor{currentfill}%
\pgfsetfillopacity{0.500000}%
\pgfsetlinewidth{1.003750pt}%
\definecolor{currentstroke}{rgb}{0.980392,0.164706,0.333333}%
\pgfsetstrokecolor{currentstroke}%
\pgfsetstrokeopacity{0.500000}%
\pgfsetdash{{3.700000pt}{1.600000pt}}{0.000000pt}%
\pgfpathmoveto{\pgfqpoint{1.440125in}{0.639736in}}%
\pgfpathcurveto{\pgfqpoint{1.445949in}{0.639736in}}{\pgfqpoint{1.451535in}{0.642049in}}{\pgfqpoint{1.455654in}{0.646168in}}%
\pgfpathcurveto{\pgfqpoint{1.459772in}{0.650286in}}{\pgfqpoint{1.462086in}{0.655872in}}{\pgfqpoint{1.462086in}{0.661696in}}%
\pgfpathcurveto{\pgfqpoint{1.462086in}{0.667520in}}{\pgfqpoint{1.459772in}{0.673106in}}{\pgfqpoint{1.455654in}{0.677224in}}%
\pgfpathcurveto{\pgfqpoint{1.451535in}{0.681342in}}{\pgfqpoint{1.445949in}{0.683656in}}{\pgfqpoint{1.440125in}{0.683656in}}%
\pgfpathcurveto{\pgfqpoint{1.434301in}{0.683656in}}{\pgfqpoint{1.428715in}{0.681342in}}{\pgfqpoint{1.424597in}{0.677224in}}%
\pgfpathcurveto{\pgfqpoint{1.420479in}{0.673106in}}{\pgfqpoint{1.418165in}{0.667520in}}{\pgfqpoint{1.418165in}{0.661696in}}%
\pgfpathcurveto{\pgfqpoint{1.418165in}{0.655872in}}{\pgfqpoint{1.420479in}{0.650286in}}{\pgfqpoint{1.424597in}{0.646168in}}%
\pgfpathcurveto{\pgfqpoint{1.428715in}{0.642049in}}{\pgfqpoint{1.434301in}{0.639736in}}{\pgfqpoint{1.440125in}{0.639736in}}%
\pgfpathlineto{\pgfqpoint{1.440125in}{0.639736in}}%
\pgfpathclose%
\pgfusepath{stroke,fill}%
\end{pgfscope}%
\begin{pgfscope}%
\pgfpathrectangle{\pgfqpoint{0.640323in}{0.527436in}}{\pgfqpoint{9.687500in}{3.850000in}}%
\pgfusepath{clip}%
\pgfsetbuttcap%
\pgfsetroundjoin%
\definecolor{currentfill}{rgb}{0.980392,0.164706,0.333333}%
\pgfsetfillcolor{currentfill}%
\pgfsetfillopacity{0.500000}%
\pgfsetlinewidth{1.003750pt}%
\definecolor{currentstroke}{rgb}{0.980392,0.164706,0.333333}%
\pgfsetstrokecolor{currentstroke}%
\pgfsetstrokeopacity{0.500000}%
\pgfsetdash{{3.700000pt}{1.600000pt}}{0.000000pt}%
\pgfpathmoveto{\pgfqpoint{1.619856in}{0.642392in}}%
\pgfpathcurveto{\pgfqpoint{1.625680in}{0.642392in}}{\pgfqpoint{1.631266in}{0.644706in}}{\pgfqpoint{1.635385in}{0.648824in}}%
\pgfpathcurveto{\pgfqpoint{1.639503in}{0.652942in}}{\pgfqpoint{1.641817in}{0.658528in}}{\pgfqpoint{1.641817in}{0.664352in}}%
\pgfpathcurveto{\pgfqpoint{1.641817in}{0.670176in}}{\pgfqpoint{1.639503in}{0.675762in}}{\pgfqpoint{1.635385in}{0.679880in}}%
\pgfpathcurveto{\pgfqpoint{1.631266in}{0.683998in}}{\pgfqpoint{1.625680in}{0.686312in}}{\pgfqpoint{1.619856in}{0.686312in}}%
\pgfpathcurveto{\pgfqpoint{1.614032in}{0.686312in}}{\pgfqpoint{1.608446in}{0.683998in}}{\pgfqpoint{1.604328in}{0.679880in}}%
\pgfpathcurveto{\pgfqpoint{1.600210in}{0.675762in}}{\pgfqpoint{1.597896in}{0.670176in}}{\pgfqpoint{1.597896in}{0.664352in}}%
\pgfpathcurveto{\pgfqpoint{1.597896in}{0.658528in}}{\pgfqpoint{1.600210in}{0.652942in}}{\pgfqpoint{1.604328in}{0.648824in}}%
\pgfpathcurveto{\pgfqpoint{1.608446in}{0.644706in}}{\pgfqpoint{1.614032in}{0.642392in}}{\pgfqpoint{1.619856in}{0.642392in}}%
\pgfpathlineto{\pgfqpoint{1.619856in}{0.642392in}}%
\pgfpathclose%
\pgfusepath{stroke,fill}%
\end{pgfscope}%
\begin{pgfscope}%
\pgfpathrectangle{\pgfqpoint{0.640323in}{0.527436in}}{\pgfqpoint{9.687500in}{3.850000in}}%
\pgfusepath{clip}%
\pgfsetbuttcap%
\pgfsetroundjoin%
\definecolor{currentfill}{rgb}{0.980392,0.164706,0.333333}%
\pgfsetfillcolor{currentfill}%
\pgfsetfillopacity{0.500000}%
\pgfsetlinewidth{1.003750pt}%
\definecolor{currentstroke}{rgb}{0.980392,0.164706,0.333333}%
\pgfsetstrokecolor{currentstroke}%
\pgfsetstrokeopacity{0.500000}%
\pgfsetdash{{3.700000pt}{1.600000pt}}{0.000000pt}%
\pgfpathmoveto{\pgfqpoint{1.799587in}{0.643477in}}%
\pgfpathcurveto{\pgfqpoint{1.805411in}{0.643477in}}{\pgfqpoint{1.810997in}{0.645791in}}{\pgfqpoint{1.815116in}{0.649909in}}%
\pgfpathcurveto{\pgfqpoint{1.819234in}{0.654027in}}{\pgfqpoint{1.821548in}{0.659613in}}{\pgfqpoint{1.821548in}{0.665437in}}%
\pgfpathcurveto{\pgfqpoint{1.821548in}{0.671261in}}{\pgfqpoint{1.819234in}{0.676848in}}{\pgfqpoint{1.815116in}{0.680966in}}%
\pgfpathcurveto{\pgfqpoint{1.810997in}{0.685084in}}{\pgfqpoint{1.805411in}{0.687398in}}{\pgfqpoint{1.799587in}{0.687398in}}%
\pgfpathcurveto{\pgfqpoint{1.793763in}{0.687398in}}{\pgfqpoint{1.788177in}{0.685084in}}{\pgfqpoint{1.784059in}{0.680966in}}%
\pgfpathcurveto{\pgfqpoint{1.779941in}{0.676848in}}{\pgfqpoint{1.777627in}{0.671261in}}{\pgfqpoint{1.777627in}{0.665437in}}%
\pgfpathcurveto{\pgfqpoint{1.777627in}{0.659613in}}{\pgfqpoint{1.779941in}{0.654027in}}{\pgfqpoint{1.784059in}{0.649909in}}%
\pgfpathcurveto{\pgfqpoint{1.788177in}{0.645791in}}{\pgfqpoint{1.793763in}{0.643477in}}{\pgfqpoint{1.799587in}{0.643477in}}%
\pgfpathlineto{\pgfqpoint{1.799587in}{0.643477in}}%
\pgfpathclose%
\pgfusepath{stroke,fill}%
\end{pgfscope}%
\begin{pgfscope}%
\pgfpathrectangle{\pgfqpoint{0.640323in}{0.527436in}}{\pgfqpoint{9.687500in}{3.850000in}}%
\pgfusepath{clip}%
\pgfsetbuttcap%
\pgfsetroundjoin%
\definecolor{currentfill}{rgb}{0.980392,0.164706,0.333333}%
\pgfsetfillcolor{currentfill}%
\pgfsetfillopacity{0.500000}%
\pgfsetlinewidth{1.003750pt}%
\definecolor{currentstroke}{rgb}{0.980392,0.164706,0.333333}%
\pgfsetstrokecolor{currentstroke}%
\pgfsetstrokeopacity{0.500000}%
\pgfsetdash{{3.700000pt}{1.600000pt}}{0.000000pt}%
\pgfpathmoveto{\pgfqpoint{1.979318in}{0.652398in}}%
\pgfpathcurveto{\pgfqpoint{1.985142in}{0.652398in}}{\pgfqpoint{1.990728in}{0.654711in}}{\pgfqpoint{1.994847in}{0.658830in}}%
\pgfpathcurveto{\pgfqpoint{1.998965in}{0.662948in}}{\pgfqpoint{2.001279in}{0.668534in}}{\pgfqpoint{2.001279in}{0.674358in}}%
\pgfpathcurveto{\pgfqpoint{2.001279in}{0.680182in}}{\pgfqpoint{1.998965in}{0.685768in}}{\pgfqpoint{1.994847in}{0.689886in}}%
\pgfpathcurveto{\pgfqpoint{1.990728in}{0.694004in}}{\pgfqpoint{1.985142in}{0.696318in}}{\pgfqpoint{1.979318in}{0.696318in}}%
\pgfpathcurveto{\pgfqpoint{1.973494in}{0.696318in}}{\pgfqpoint{1.967908in}{0.694004in}}{\pgfqpoint{1.963790in}{0.689886in}}%
\pgfpathcurveto{\pgfqpoint{1.959672in}{0.685768in}}{\pgfqpoint{1.957358in}{0.680182in}}{\pgfqpoint{1.957358in}{0.674358in}}%
\pgfpathcurveto{\pgfqpoint{1.957358in}{0.668534in}}{\pgfqpoint{1.959672in}{0.662948in}}{\pgfqpoint{1.963790in}{0.658830in}}%
\pgfpathcurveto{\pgfqpoint{1.967908in}{0.654711in}}{\pgfqpoint{1.973494in}{0.652398in}}{\pgfqpoint{1.979318in}{0.652398in}}%
\pgfpathlineto{\pgfqpoint{1.979318in}{0.652398in}}%
\pgfpathclose%
\pgfusepath{stroke,fill}%
\end{pgfscope}%
\begin{pgfscope}%
\pgfpathrectangle{\pgfqpoint{0.640323in}{0.527436in}}{\pgfqpoint{9.687500in}{3.850000in}}%
\pgfusepath{clip}%
\pgfsetbuttcap%
\pgfsetroundjoin%
\definecolor{currentfill}{rgb}{0.980392,0.164706,0.333333}%
\pgfsetfillcolor{currentfill}%
\pgfsetfillopacity{0.500000}%
\pgfsetlinewidth{1.003750pt}%
\definecolor{currentstroke}{rgb}{0.980392,0.164706,0.333333}%
\pgfsetstrokecolor{currentstroke}%
\pgfsetstrokeopacity{0.500000}%
\pgfsetdash{{3.700000pt}{1.600000pt}}{0.000000pt}%
\pgfpathmoveto{\pgfqpoint{2.159049in}{0.655296in}}%
\pgfpathcurveto{\pgfqpoint{2.164873in}{0.655296in}}{\pgfqpoint{2.170459in}{0.657610in}}{\pgfqpoint{2.174578in}{0.661728in}}%
\pgfpathcurveto{\pgfqpoint{2.178696in}{0.665846in}}{\pgfqpoint{2.181010in}{0.671432in}}{\pgfqpoint{2.181010in}{0.677256in}}%
\pgfpathcurveto{\pgfqpoint{2.181010in}{0.683080in}}{\pgfqpoint{2.178696in}{0.688667in}}{\pgfqpoint{2.174578in}{0.692785in}}%
\pgfpathcurveto{\pgfqpoint{2.170459in}{0.696903in}}{\pgfqpoint{2.164873in}{0.699217in}}{\pgfqpoint{2.159049in}{0.699217in}}%
\pgfpathcurveto{\pgfqpoint{2.153225in}{0.699217in}}{\pgfqpoint{2.147639in}{0.696903in}}{\pgfqpoint{2.143521in}{0.692785in}}%
\pgfpathcurveto{\pgfqpoint{2.139403in}{0.688667in}}{\pgfqpoint{2.137089in}{0.683080in}}{\pgfqpoint{2.137089in}{0.677256in}}%
\pgfpathcurveto{\pgfqpoint{2.137089in}{0.671432in}}{\pgfqpoint{2.139403in}{0.665846in}}{\pgfqpoint{2.143521in}{0.661728in}}%
\pgfpathcurveto{\pgfqpoint{2.147639in}{0.657610in}}{\pgfqpoint{2.153225in}{0.655296in}}{\pgfqpoint{2.159049in}{0.655296in}}%
\pgfpathlineto{\pgfqpoint{2.159049in}{0.655296in}}%
\pgfpathclose%
\pgfusepath{stroke,fill}%
\end{pgfscope}%
\begin{pgfscope}%
\pgfpathrectangle{\pgfqpoint{0.640323in}{0.527436in}}{\pgfqpoint{9.687500in}{3.850000in}}%
\pgfusepath{clip}%
\pgfsetbuttcap%
\pgfsetroundjoin%
\definecolor{currentfill}{rgb}{0.980392,0.164706,0.333333}%
\pgfsetfillcolor{currentfill}%
\pgfsetfillopacity{0.500000}%
\pgfsetlinewidth{1.003750pt}%
\definecolor{currentstroke}{rgb}{0.980392,0.164706,0.333333}%
\pgfsetstrokecolor{currentstroke}%
\pgfsetstrokeopacity{0.500000}%
\pgfsetdash{{3.700000pt}{1.600000pt}}{0.000000pt}%
\pgfpathmoveto{\pgfqpoint{2.338780in}{0.693403in}}%
\pgfpathcurveto{\pgfqpoint{2.344604in}{0.693403in}}{\pgfqpoint{2.350190in}{0.695717in}}{\pgfqpoint{2.354309in}{0.699835in}}%
\pgfpathcurveto{\pgfqpoint{2.358427in}{0.703954in}}{\pgfqpoint{2.360741in}{0.709540in}}{\pgfqpoint{2.360741in}{0.715364in}}%
\pgfpathcurveto{\pgfqpoint{2.360741in}{0.721188in}}{\pgfqpoint{2.358427in}{0.726774in}}{\pgfqpoint{2.354309in}{0.730892in}}%
\pgfpathcurveto{\pgfqpoint{2.350190in}{0.735010in}}{\pgfqpoint{2.344604in}{0.737324in}}{\pgfqpoint{2.338780in}{0.737324in}}%
\pgfpathcurveto{\pgfqpoint{2.332956in}{0.737324in}}{\pgfqpoint{2.327370in}{0.735010in}}{\pgfqpoint{2.323252in}{0.730892in}}%
\pgfpathcurveto{\pgfqpoint{2.319134in}{0.726774in}}{\pgfqpoint{2.316820in}{0.721188in}}{\pgfqpoint{2.316820in}{0.715364in}}%
\pgfpathcurveto{\pgfqpoint{2.316820in}{0.709540in}}{\pgfqpoint{2.319134in}{0.703954in}}{\pgfqpoint{2.323252in}{0.699835in}}%
\pgfpathcurveto{\pgfqpoint{2.327370in}{0.695717in}}{\pgfqpoint{2.332956in}{0.693403in}}{\pgfqpoint{2.338780in}{0.693403in}}%
\pgfpathlineto{\pgfqpoint{2.338780in}{0.693403in}}%
\pgfpathclose%
\pgfusepath{stroke,fill}%
\end{pgfscope}%
\begin{pgfscope}%
\pgfpathrectangle{\pgfqpoint{0.640323in}{0.527436in}}{\pgfqpoint{9.687500in}{3.850000in}}%
\pgfusepath{clip}%
\pgfsetbuttcap%
\pgfsetroundjoin%
\definecolor{currentfill}{rgb}{0.980392,0.164706,0.333333}%
\pgfsetfillcolor{currentfill}%
\pgfsetfillopacity{0.500000}%
\pgfsetlinewidth{1.003750pt}%
\definecolor{currentstroke}{rgb}{0.980392,0.164706,0.333333}%
\pgfsetstrokecolor{currentstroke}%
\pgfsetstrokeopacity{0.500000}%
\pgfsetdash{{3.700000pt}{1.600000pt}}{0.000000pt}%
\pgfpathmoveto{\pgfqpoint{2.518511in}{0.731335in}}%
\pgfpathcurveto{\pgfqpoint{2.524335in}{0.731335in}}{\pgfqpoint{2.529921in}{0.733648in}}{\pgfqpoint{2.534040in}{0.737767in}}%
\pgfpathcurveto{\pgfqpoint{2.538158in}{0.741885in}}{\pgfqpoint{2.540472in}{0.747471in}}{\pgfqpoint{2.540472in}{0.753295in}}%
\pgfpathcurveto{\pgfqpoint{2.540472in}{0.759119in}}{\pgfqpoint{2.538158in}{0.764705in}}{\pgfqpoint{2.534040in}{0.768823in}}%
\pgfpathcurveto{\pgfqpoint{2.529921in}{0.772941in}}{\pgfqpoint{2.524335in}{0.775255in}}{\pgfqpoint{2.518511in}{0.775255in}}%
\pgfpathcurveto{\pgfqpoint{2.512687in}{0.775255in}}{\pgfqpoint{2.507101in}{0.772941in}}{\pgfqpoint{2.502983in}{0.768823in}}%
\pgfpathcurveto{\pgfqpoint{2.498865in}{0.764705in}}{\pgfqpoint{2.496551in}{0.759119in}}{\pgfqpoint{2.496551in}{0.753295in}}%
\pgfpathcurveto{\pgfqpoint{2.496551in}{0.747471in}}{\pgfqpoint{2.498865in}{0.741885in}}{\pgfqpoint{2.502983in}{0.737767in}}%
\pgfpathcurveto{\pgfqpoint{2.507101in}{0.733648in}}{\pgfqpoint{2.512687in}{0.731335in}}{\pgfqpoint{2.518511in}{0.731335in}}%
\pgfpathlineto{\pgfqpoint{2.518511in}{0.731335in}}%
\pgfpathclose%
\pgfusepath{stroke,fill}%
\end{pgfscope}%
\begin{pgfscope}%
\pgfpathrectangle{\pgfqpoint{0.640323in}{0.527436in}}{\pgfqpoint{9.687500in}{3.850000in}}%
\pgfusepath{clip}%
\pgfsetbuttcap%
\pgfsetroundjoin%
\definecolor{currentfill}{rgb}{0.980392,0.164706,0.333333}%
\pgfsetfillcolor{currentfill}%
\pgfsetfillopacity{0.500000}%
\pgfsetlinewidth{1.003750pt}%
\definecolor{currentstroke}{rgb}{0.980392,0.164706,0.333333}%
\pgfsetstrokecolor{currentstroke}%
\pgfsetstrokeopacity{0.500000}%
\pgfsetdash{{3.700000pt}{1.600000pt}}{0.000000pt}%
\pgfpathmoveto{\pgfqpoint{2.698242in}{1.175332in}}%
\pgfpathcurveto{\pgfqpoint{2.704066in}{1.175332in}}{\pgfqpoint{2.709652in}{1.177646in}}{\pgfqpoint{2.713771in}{1.181764in}}%
\pgfpathcurveto{\pgfqpoint{2.717889in}{1.185882in}}{\pgfqpoint{2.720203in}{1.191468in}}{\pgfqpoint{2.720203in}{1.197292in}}%
\pgfpathcurveto{\pgfqpoint{2.720203in}{1.203116in}}{\pgfqpoint{2.717889in}{1.208702in}}{\pgfqpoint{2.713771in}{1.212821in}}%
\pgfpathcurveto{\pgfqpoint{2.709652in}{1.216939in}}{\pgfqpoint{2.704066in}{1.219253in}}{\pgfqpoint{2.698242in}{1.219253in}}%
\pgfpathcurveto{\pgfqpoint{2.692418in}{1.219253in}}{\pgfqpoint{2.686832in}{1.216939in}}{\pgfqpoint{2.682714in}{1.212821in}}%
\pgfpathcurveto{\pgfqpoint{2.678596in}{1.208702in}}{\pgfqpoint{2.676282in}{1.203116in}}{\pgfqpoint{2.676282in}{1.197292in}}%
\pgfpathcurveto{\pgfqpoint{2.676282in}{1.191468in}}{\pgfqpoint{2.678596in}{1.185882in}}{\pgfqpoint{2.682714in}{1.181764in}}%
\pgfpathcurveto{\pgfqpoint{2.686832in}{1.177646in}}{\pgfqpoint{2.692418in}{1.175332in}}{\pgfqpoint{2.698242in}{1.175332in}}%
\pgfpathlineto{\pgfqpoint{2.698242in}{1.175332in}}%
\pgfpathclose%
\pgfusepath{stroke,fill}%
\end{pgfscope}%
\begin{pgfscope}%
\pgfpathrectangle{\pgfqpoint{0.640323in}{0.527436in}}{\pgfqpoint{9.687500in}{3.850000in}}%
\pgfusepath{clip}%
\pgfsetbuttcap%
\pgfsetroundjoin%
\definecolor{currentfill}{rgb}{0.980392,0.164706,0.333333}%
\pgfsetfillcolor{currentfill}%
\pgfsetfillopacity{0.500000}%
\pgfsetlinewidth{1.003750pt}%
\definecolor{currentstroke}{rgb}{0.980392,0.164706,0.333333}%
\pgfsetstrokecolor{currentstroke}%
\pgfsetstrokeopacity{0.500000}%
\pgfsetdash{{3.700000pt}{1.600000pt}}{0.000000pt}%
\pgfpathmoveto{\pgfqpoint{2.877973in}{1.680843in}}%
\pgfpathcurveto{\pgfqpoint{2.883797in}{1.680843in}}{\pgfqpoint{2.889383in}{1.683157in}}{\pgfqpoint{2.893501in}{1.687275in}}%
\pgfpathcurveto{\pgfqpoint{2.897620in}{1.691393in}}{\pgfqpoint{2.899934in}{1.696980in}}{\pgfqpoint{2.899934in}{1.702804in}}%
\pgfpathcurveto{\pgfqpoint{2.899934in}{1.708628in}}{\pgfqpoint{2.897620in}{1.714214in}}{\pgfqpoint{2.893501in}{1.718332in}}%
\pgfpathcurveto{\pgfqpoint{2.889383in}{1.722450in}}{\pgfqpoint{2.883797in}{1.724764in}}{\pgfqpoint{2.877973in}{1.724764in}}%
\pgfpathcurveto{\pgfqpoint{2.872149in}{1.724764in}}{\pgfqpoint{2.866563in}{1.722450in}}{\pgfqpoint{2.862445in}{1.718332in}}%
\pgfpathcurveto{\pgfqpoint{2.858327in}{1.714214in}}{\pgfqpoint{2.856013in}{1.708628in}}{\pgfqpoint{2.856013in}{1.702804in}}%
\pgfpathcurveto{\pgfqpoint{2.856013in}{1.696980in}}{\pgfqpoint{2.858327in}{1.691393in}}{\pgfqpoint{2.862445in}{1.687275in}}%
\pgfpathcurveto{\pgfqpoint{2.866563in}{1.683157in}}{\pgfqpoint{2.872149in}{1.680843in}}{\pgfqpoint{2.877973in}{1.680843in}}%
\pgfpathlineto{\pgfqpoint{2.877973in}{1.680843in}}%
\pgfpathclose%
\pgfusepath{stroke,fill}%
\end{pgfscope}%
\begin{pgfscope}%
\pgfpathrectangle{\pgfqpoint{0.640323in}{0.527436in}}{\pgfqpoint{9.687500in}{3.850000in}}%
\pgfusepath{clip}%
\pgfsetbuttcap%
\pgfsetroundjoin%
\definecolor{currentfill}{rgb}{0.980392,0.164706,0.333333}%
\pgfsetfillcolor{currentfill}%
\pgfsetfillopacity{0.500000}%
\pgfsetlinewidth{1.003750pt}%
\definecolor{currentstroke}{rgb}{0.980392,0.164706,0.333333}%
\pgfsetstrokecolor{currentstroke}%
\pgfsetstrokeopacity{0.500000}%
\pgfsetdash{{3.700000pt}{1.600000pt}}{0.000000pt}%
\pgfpathmoveto{\pgfqpoint{3.057704in}{2.052437in}}%
\pgfpathcurveto{\pgfqpoint{3.063528in}{2.052437in}}{\pgfqpoint{3.069114in}{2.054750in}}{\pgfqpoint{3.073232in}{2.058869in}}%
\pgfpathcurveto{\pgfqpoint{3.077351in}{2.062987in}}{\pgfqpoint{3.079664in}{2.068573in}}{\pgfqpoint{3.079664in}{2.074397in}}%
\pgfpathcurveto{\pgfqpoint{3.079664in}{2.080221in}}{\pgfqpoint{3.077351in}{2.085807in}}{\pgfqpoint{3.073232in}{2.089925in}}%
\pgfpathcurveto{\pgfqpoint{3.069114in}{2.094043in}}{\pgfqpoint{3.063528in}{2.096357in}}{\pgfqpoint{3.057704in}{2.096357in}}%
\pgfpathcurveto{\pgfqpoint{3.051880in}{2.096357in}}{\pgfqpoint{3.046294in}{2.094043in}}{\pgfqpoint{3.042176in}{2.089925in}}%
\pgfpathcurveto{\pgfqpoint{3.038058in}{2.085807in}}{\pgfqpoint{3.035744in}{2.080221in}}{\pgfqpoint{3.035744in}{2.074397in}}%
\pgfpathcurveto{\pgfqpoint{3.035744in}{2.068573in}}{\pgfqpoint{3.038058in}{2.062987in}}{\pgfqpoint{3.042176in}{2.058869in}}%
\pgfpathcurveto{\pgfqpoint{3.046294in}{2.054750in}}{\pgfqpoint{3.051880in}{2.052437in}}{\pgfqpoint{3.057704in}{2.052437in}}%
\pgfpathlineto{\pgfqpoint{3.057704in}{2.052437in}}%
\pgfpathclose%
\pgfusepath{stroke,fill}%
\end{pgfscope}%
\begin{pgfscope}%
\pgfpathrectangle{\pgfqpoint{0.640323in}{0.527436in}}{\pgfqpoint{9.687500in}{3.850000in}}%
\pgfusepath{clip}%
\pgfsetbuttcap%
\pgfsetroundjoin%
\definecolor{currentfill}{rgb}{0.980392,0.164706,0.333333}%
\pgfsetfillcolor{currentfill}%
\pgfsetfillopacity{0.500000}%
\pgfsetlinewidth{1.003750pt}%
\definecolor{currentstroke}{rgb}{0.980392,0.164706,0.333333}%
\pgfsetstrokecolor{currentstroke}%
\pgfsetstrokeopacity{0.500000}%
\pgfsetdash{{3.700000pt}{1.600000pt}}{0.000000pt}%
\pgfpathmoveto{\pgfqpoint{3.237435in}{2.232499in}}%
\pgfpathcurveto{\pgfqpoint{3.243259in}{2.232499in}}{\pgfqpoint{3.248845in}{2.234813in}}{\pgfqpoint{3.252963in}{2.238931in}}%
\pgfpathcurveto{\pgfqpoint{3.257082in}{2.243049in}}{\pgfqpoint{3.259395in}{2.248635in}}{\pgfqpoint{3.259395in}{2.254459in}}%
\pgfpathcurveto{\pgfqpoint{3.259395in}{2.260283in}}{\pgfqpoint{3.257082in}{2.265869in}}{\pgfqpoint{3.252963in}{2.269987in}}%
\pgfpathcurveto{\pgfqpoint{3.248845in}{2.274105in}}{\pgfqpoint{3.243259in}{2.276419in}}{\pgfqpoint{3.237435in}{2.276419in}}%
\pgfpathcurveto{\pgfqpoint{3.231611in}{2.276419in}}{\pgfqpoint{3.226025in}{2.274105in}}{\pgfqpoint{3.221907in}{2.269987in}}%
\pgfpathcurveto{\pgfqpoint{3.217789in}{2.265869in}}{\pgfqpoint{3.215475in}{2.260283in}}{\pgfqpoint{3.215475in}{2.254459in}}%
\pgfpathcurveto{\pgfqpoint{3.215475in}{2.248635in}}{\pgfqpoint{3.217789in}{2.243049in}}{\pgfqpoint{3.221907in}{2.238931in}}%
\pgfpathcurveto{\pgfqpoint{3.226025in}{2.234813in}}{\pgfqpoint{3.231611in}{2.232499in}}{\pgfqpoint{3.237435in}{2.232499in}}%
\pgfpathlineto{\pgfqpoint{3.237435in}{2.232499in}}%
\pgfpathclose%
\pgfusepath{stroke,fill}%
\end{pgfscope}%
\begin{pgfscope}%
\pgfpathrectangle{\pgfqpoint{0.640323in}{0.527436in}}{\pgfqpoint{9.687500in}{3.850000in}}%
\pgfusepath{clip}%
\pgfsetbuttcap%
\pgfsetroundjoin%
\definecolor{currentfill}{rgb}{0.980392,0.164706,0.333333}%
\pgfsetfillcolor{currentfill}%
\pgfsetfillopacity{0.500000}%
\pgfsetlinewidth{1.003750pt}%
\definecolor{currentstroke}{rgb}{0.980392,0.164706,0.333333}%
\pgfsetstrokecolor{currentstroke}%
\pgfsetstrokeopacity{0.500000}%
\pgfsetdash{{3.700000pt}{1.600000pt}}{0.000000pt}%
\pgfpathmoveto{\pgfqpoint{3.417166in}{2.409456in}}%
\pgfpathcurveto{\pgfqpoint{3.422990in}{2.409456in}}{\pgfqpoint{3.428576in}{2.411770in}}{\pgfqpoint{3.432694in}{2.415888in}}%
\pgfpathcurveto{\pgfqpoint{3.436813in}{2.420006in}}{\pgfqpoint{3.439126in}{2.425592in}}{\pgfqpoint{3.439126in}{2.431416in}}%
\pgfpathcurveto{\pgfqpoint{3.439126in}{2.437240in}}{\pgfqpoint{3.436813in}{2.442826in}}{\pgfqpoint{3.432694in}{2.446944in}}%
\pgfpathcurveto{\pgfqpoint{3.428576in}{2.451062in}}{\pgfqpoint{3.422990in}{2.453376in}}{\pgfqpoint{3.417166in}{2.453376in}}%
\pgfpathcurveto{\pgfqpoint{3.411342in}{2.453376in}}{\pgfqpoint{3.405756in}{2.451062in}}{\pgfqpoint{3.401638in}{2.446944in}}%
\pgfpathcurveto{\pgfqpoint{3.397520in}{2.442826in}}{\pgfqpoint{3.395206in}{2.437240in}}{\pgfqpoint{3.395206in}{2.431416in}}%
\pgfpathcurveto{\pgfqpoint{3.395206in}{2.425592in}}{\pgfqpoint{3.397520in}{2.420006in}}{\pgfqpoint{3.401638in}{2.415888in}}%
\pgfpathcurveto{\pgfqpoint{3.405756in}{2.411770in}}{\pgfqpoint{3.411342in}{2.409456in}}{\pgfqpoint{3.417166in}{2.409456in}}%
\pgfpathlineto{\pgfqpoint{3.417166in}{2.409456in}}%
\pgfpathclose%
\pgfusepath{stroke,fill}%
\end{pgfscope}%
\begin{pgfscope}%
\pgfpathrectangle{\pgfqpoint{0.640323in}{0.527436in}}{\pgfqpoint{9.687500in}{3.850000in}}%
\pgfusepath{clip}%
\pgfsetbuttcap%
\pgfsetroundjoin%
\definecolor{currentfill}{rgb}{0.980392,0.164706,0.333333}%
\pgfsetfillcolor{currentfill}%
\pgfsetfillopacity{0.500000}%
\pgfsetlinewidth{1.003750pt}%
\definecolor{currentstroke}{rgb}{0.980392,0.164706,0.333333}%
\pgfsetstrokecolor{currentstroke}%
\pgfsetstrokeopacity{0.500000}%
\pgfsetdash{{3.700000pt}{1.600000pt}}{0.000000pt}%
\pgfpathmoveto{\pgfqpoint{3.596897in}{2.547013in}}%
\pgfpathcurveto{\pgfqpoint{3.602721in}{2.547013in}}{\pgfqpoint{3.608307in}{2.549326in}}{\pgfqpoint{3.612425in}{2.553445in}}%
\pgfpathcurveto{\pgfqpoint{3.616544in}{2.557563in}}{\pgfqpoint{3.618857in}{2.563149in}}{\pgfqpoint{3.618857in}{2.568973in}}%
\pgfpathcurveto{\pgfqpoint{3.618857in}{2.574797in}}{\pgfqpoint{3.616544in}{2.580383in}}{\pgfqpoint{3.612425in}{2.584501in}}%
\pgfpathcurveto{\pgfqpoint{3.608307in}{2.588619in}}{\pgfqpoint{3.602721in}{2.590933in}}{\pgfqpoint{3.596897in}{2.590933in}}%
\pgfpathcurveto{\pgfqpoint{3.591073in}{2.590933in}}{\pgfqpoint{3.585487in}{2.588619in}}{\pgfqpoint{3.581369in}{2.584501in}}%
\pgfpathcurveto{\pgfqpoint{3.577251in}{2.580383in}}{\pgfqpoint{3.574937in}{2.574797in}}{\pgfqpoint{3.574937in}{2.568973in}}%
\pgfpathcurveto{\pgfqpoint{3.574937in}{2.563149in}}{\pgfqpoint{3.577251in}{2.557563in}}{\pgfqpoint{3.581369in}{2.553445in}}%
\pgfpathcurveto{\pgfqpoint{3.585487in}{2.549326in}}{\pgfqpoint{3.591073in}{2.547013in}}{\pgfqpoint{3.596897in}{2.547013in}}%
\pgfpathlineto{\pgfqpoint{3.596897in}{2.547013in}}%
\pgfpathclose%
\pgfusepath{stroke,fill}%
\end{pgfscope}%
\begin{pgfscope}%
\pgfpathrectangle{\pgfqpoint{0.640323in}{0.527436in}}{\pgfqpoint{9.687500in}{3.850000in}}%
\pgfusepath{clip}%
\pgfsetbuttcap%
\pgfsetroundjoin%
\definecolor{currentfill}{rgb}{0.980392,0.164706,0.333333}%
\pgfsetfillcolor{currentfill}%
\pgfsetfillopacity{0.500000}%
\pgfsetlinewidth{1.003750pt}%
\definecolor{currentstroke}{rgb}{0.980392,0.164706,0.333333}%
\pgfsetstrokecolor{currentstroke}%
\pgfsetstrokeopacity{0.500000}%
\pgfsetdash{{3.700000pt}{1.600000pt}}{0.000000pt}%
\pgfpathmoveto{\pgfqpoint{3.776628in}{2.697784in}}%
\pgfpathcurveto{\pgfqpoint{3.782452in}{2.697784in}}{\pgfqpoint{3.788038in}{2.700097in}}{\pgfqpoint{3.792156in}{2.704216in}}%
\pgfpathcurveto{\pgfqpoint{3.796275in}{2.708334in}}{\pgfqpoint{3.798588in}{2.713920in}}{\pgfqpoint{3.798588in}{2.719744in}}%
\pgfpathcurveto{\pgfqpoint{3.798588in}{2.725568in}}{\pgfqpoint{3.796275in}{2.731154in}}{\pgfqpoint{3.792156in}{2.735272in}}%
\pgfpathcurveto{\pgfqpoint{3.788038in}{2.739390in}}{\pgfqpoint{3.782452in}{2.741704in}}{\pgfqpoint{3.776628in}{2.741704in}}%
\pgfpathcurveto{\pgfqpoint{3.770804in}{2.741704in}}{\pgfqpoint{3.765218in}{2.739390in}}{\pgfqpoint{3.761100in}{2.735272in}}%
\pgfpathcurveto{\pgfqpoint{3.756982in}{2.731154in}}{\pgfqpoint{3.754668in}{2.725568in}}{\pgfqpoint{3.754668in}{2.719744in}}%
\pgfpathcurveto{\pgfqpoint{3.754668in}{2.713920in}}{\pgfqpoint{3.756982in}{2.708334in}}{\pgfqpoint{3.761100in}{2.704216in}}%
\pgfpathcurveto{\pgfqpoint{3.765218in}{2.700097in}}{\pgfqpoint{3.770804in}{2.697784in}}{\pgfqpoint{3.776628in}{2.697784in}}%
\pgfpathlineto{\pgfqpoint{3.776628in}{2.697784in}}%
\pgfpathclose%
\pgfusepath{stroke,fill}%
\end{pgfscope}%
\begin{pgfscope}%
\pgfpathrectangle{\pgfqpoint{0.640323in}{0.527436in}}{\pgfqpoint{9.687500in}{3.850000in}}%
\pgfusepath{clip}%
\pgfsetbuttcap%
\pgfsetroundjoin%
\definecolor{currentfill}{rgb}{0.980392,0.164706,0.333333}%
\pgfsetfillcolor{currentfill}%
\pgfsetfillopacity{0.500000}%
\pgfsetlinewidth{1.003750pt}%
\definecolor{currentstroke}{rgb}{0.980392,0.164706,0.333333}%
\pgfsetstrokecolor{currentstroke}%
\pgfsetstrokeopacity{0.500000}%
\pgfsetdash{{3.700000pt}{1.600000pt}}{0.000000pt}%
\pgfpathmoveto{\pgfqpoint{3.956359in}{2.820605in}}%
\pgfpathcurveto{\pgfqpoint{3.962183in}{2.820605in}}{\pgfqpoint{3.967769in}{2.822919in}}{\pgfqpoint{3.971887in}{2.827037in}}%
\pgfpathcurveto{\pgfqpoint{3.976006in}{2.831155in}}{\pgfqpoint{3.978319in}{2.836741in}}{\pgfqpoint{3.978319in}{2.842565in}}%
\pgfpathcurveto{\pgfqpoint{3.978319in}{2.848389in}}{\pgfqpoint{3.976006in}{2.853975in}}{\pgfqpoint{3.971887in}{2.858093in}}%
\pgfpathcurveto{\pgfqpoint{3.967769in}{2.862211in}}{\pgfqpoint{3.962183in}{2.864525in}}{\pgfqpoint{3.956359in}{2.864525in}}%
\pgfpathcurveto{\pgfqpoint{3.950535in}{2.864525in}}{\pgfqpoint{3.944949in}{2.862211in}}{\pgfqpoint{3.940831in}{2.858093in}}%
\pgfpathcurveto{\pgfqpoint{3.936713in}{2.853975in}}{\pgfqpoint{3.934399in}{2.848389in}}{\pgfqpoint{3.934399in}{2.842565in}}%
\pgfpathcurveto{\pgfqpoint{3.934399in}{2.836741in}}{\pgfqpoint{3.936713in}{2.831155in}}{\pgfqpoint{3.940831in}{2.827037in}}%
\pgfpathcurveto{\pgfqpoint{3.944949in}{2.822919in}}{\pgfqpoint{3.950535in}{2.820605in}}{\pgfqpoint{3.956359in}{2.820605in}}%
\pgfpathlineto{\pgfqpoint{3.956359in}{2.820605in}}%
\pgfpathclose%
\pgfusepath{stroke,fill}%
\end{pgfscope}%
\begin{pgfscope}%
\pgfpathrectangle{\pgfqpoint{0.640323in}{0.527436in}}{\pgfqpoint{9.687500in}{3.850000in}}%
\pgfusepath{clip}%
\pgfsetbuttcap%
\pgfsetroundjoin%
\definecolor{currentfill}{rgb}{0.980392,0.164706,0.333333}%
\pgfsetfillcolor{currentfill}%
\pgfsetfillopacity{0.500000}%
\pgfsetlinewidth{1.003750pt}%
\definecolor{currentstroke}{rgb}{0.980392,0.164706,0.333333}%
\pgfsetstrokecolor{currentstroke}%
\pgfsetstrokeopacity{0.500000}%
\pgfsetdash{{3.700000pt}{1.600000pt}}{0.000000pt}%
\pgfpathmoveto{\pgfqpoint{4.136090in}{2.903094in}}%
\pgfpathcurveto{\pgfqpoint{4.141914in}{2.903094in}}{\pgfqpoint{4.147500in}{2.905408in}}{\pgfqpoint{4.151618in}{2.909526in}}%
\pgfpathcurveto{\pgfqpoint{4.155737in}{2.913644in}}{\pgfqpoint{4.158050in}{2.919230in}}{\pgfqpoint{4.158050in}{2.925054in}}%
\pgfpathcurveto{\pgfqpoint{4.158050in}{2.930878in}}{\pgfqpoint{4.155737in}{2.936464in}}{\pgfqpoint{4.151618in}{2.940583in}}%
\pgfpathcurveto{\pgfqpoint{4.147500in}{2.944701in}}{\pgfqpoint{4.141914in}{2.947015in}}{\pgfqpoint{4.136090in}{2.947015in}}%
\pgfpathcurveto{\pgfqpoint{4.130266in}{2.947015in}}{\pgfqpoint{4.124680in}{2.944701in}}{\pgfqpoint{4.120562in}{2.940583in}}%
\pgfpathcurveto{\pgfqpoint{4.116444in}{2.936464in}}{\pgfqpoint{4.114130in}{2.930878in}}{\pgfqpoint{4.114130in}{2.925054in}}%
\pgfpathcurveto{\pgfqpoint{4.114130in}{2.919230in}}{\pgfqpoint{4.116444in}{2.913644in}}{\pgfqpoint{4.120562in}{2.909526in}}%
\pgfpathcurveto{\pgfqpoint{4.124680in}{2.905408in}}{\pgfqpoint{4.130266in}{2.903094in}}{\pgfqpoint{4.136090in}{2.903094in}}%
\pgfpathlineto{\pgfqpoint{4.136090in}{2.903094in}}%
\pgfpathclose%
\pgfusepath{stroke,fill}%
\end{pgfscope}%
\begin{pgfscope}%
\pgfpathrectangle{\pgfqpoint{0.640323in}{0.527436in}}{\pgfqpoint{9.687500in}{3.850000in}}%
\pgfusepath{clip}%
\pgfsetbuttcap%
\pgfsetroundjoin%
\definecolor{currentfill}{rgb}{0.980392,0.164706,0.333333}%
\pgfsetfillcolor{currentfill}%
\pgfsetfillopacity{0.500000}%
\pgfsetlinewidth{1.003750pt}%
\definecolor{currentstroke}{rgb}{0.980392,0.164706,0.333333}%
\pgfsetstrokecolor{currentstroke}%
\pgfsetstrokeopacity{0.500000}%
\pgfsetdash{{3.700000pt}{1.600000pt}}{0.000000pt}%
\pgfpathmoveto{\pgfqpoint{4.315821in}{3.004840in}}%
\pgfpathcurveto{\pgfqpoint{4.321645in}{3.004840in}}{\pgfqpoint{4.327231in}{3.007154in}}{\pgfqpoint{4.331349in}{3.011272in}}%
\pgfpathcurveto{\pgfqpoint{4.335467in}{3.015390in}}{\pgfqpoint{4.337781in}{3.020976in}}{\pgfqpoint{4.337781in}{3.026800in}}%
\pgfpathcurveto{\pgfqpoint{4.337781in}{3.032624in}}{\pgfqpoint{4.335467in}{3.038210in}}{\pgfqpoint{4.331349in}{3.042328in}}%
\pgfpathcurveto{\pgfqpoint{4.327231in}{3.046446in}}{\pgfqpoint{4.321645in}{3.048760in}}{\pgfqpoint{4.315821in}{3.048760in}}%
\pgfpathcurveto{\pgfqpoint{4.309997in}{3.048760in}}{\pgfqpoint{4.304411in}{3.046446in}}{\pgfqpoint{4.300293in}{3.042328in}}%
\pgfpathcurveto{\pgfqpoint{4.296175in}{3.038210in}}{\pgfqpoint{4.293861in}{3.032624in}}{\pgfqpoint{4.293861in}{3.026800in}}%
\pgfpathcurveto{\pgfqpoint{4.293861in}{3.020976in}}{\pgfqpoint{4.296175in}{3.015390in}}{\pgfqpoint{4.300293in}{3.011272in}}%
\pgfpathcurveto{\pgfqpoint{4.304411in}{3.007154in}}{\pgfqpoint{4.309997in}{3.004840in}}{\pgfqpoint{4.315821in}{3.004840in}}%
\pgfpathlineto{\pgfqpoint{4.315821in}{3.004840in}}%
\pgfpathclose%
\pgfusepath{stroke,fill}%
\end{pgfscope}%
\begin{pgfscope}%
\pgfpathrectangle{\pgfqpoint{0.640323in}{0.527436in}}{\pgfqpoint{9.687500in}{3.850000in}}%
\pgfusepath{clip}%
\pgfsetbuttcap%
\pgfsetroundjoin%
\definecolor{currentfill}{rgb}{0.980392,0.164706,0.333333}%
\pgfsetfillcolor{currentfill}%
\pgfsetfillopacity{0.500000}%
\pgfsetlinewidth{1.003750pt}%
\definecolor{currentstroke}{rgb}{0.980392,0.164706,0.333333}%
\pgfsetstrokecolor{currentstroke}%
\pgfsetstrokeopacity{0.500000}%
\pgfsetdash{{3.700000pt}{1.600000pt}}{0.000000pt}%
\pgfpathmoveto{\pgfqpoint{4.495552in}{3.082386in}}%
\pgfpathcurveto{\pgfqpoint{4.501376in}{3.082386in}}{\pgfqpoint{4.506962in}{3.084700in}}{\pgfqpoint{4.511080in}{3.088818in}}%
\pgfpathcurveto{\pgfqpoint{4.515198in}{3.092936in}}{\pgfqpoint{4.517512in}{3.098522in}}{\pgfqpoint{4.517512in}{3.104346in}}%
\pgfpathcurveto{\pgfqpoint{4.517512in}{3.110170in}}{\pgfqpoint{4.515198in}{3.115757in}}{\pgfqpoint{4.511080in}{3.119875in}}%
\pgfpathcurveto{\pgfqpoint{4.506962in}{3.123993in}}{\pgfqpoint{4.501376in}{3.126307in}}{\pgfqpoint{4.495552in}{3.126307in}}%
\pgfpathcurveto{\pgfqpoint{4.489728in}{3.126307in}}{\pgfqpoint{4.484142in}{3.123993in}}{\pgfqpoint{4.480024in}{3.119875in}}%
\pgfpathcurveto{\pgfqpoint{4.475906in}{3.115757in}}{\pgfqpoint{4.473592in}{3.110170in}}{\pgfqpoint{4.473592in}{3.104346in}}%
\pgfpathcurveto{\pgfqpoint{4.473592in}{3.098522in}}{\pgfqpoint{4.475906in}{3.092936in}}{\pgfqpoint{4.480024in}{3.088818in}}%
\pgfpathcurveto{\pgfqpoint{4.484142in}{3.084700in}}{\pgfqpoint{4.489728in}{3.082386in}}{\pgfqpoint{4.495552in}{3.082386in}}%
\pgfpathlineto{\pgfqpoint{4.495552in}{3.082386in}}%
\pgfpathclose%
\pgfusepath{stroke,fill}%
\end{pgfscope}%
\begin{pgfscope}%
\pgfpathrectangle{\pgfqpoint{0.640323in}{0.527436in}}{\pgfqpoint{9.687500in}{3.850000in}}%
\pgfusepath{clip}%
\pgfsetbuttcap%
\pgfsetroundjoin%
\definecolor{currentfill}{rgb}{0.980392,0.164706,0.333333}%
\pgfsetfillcolor{currentfill}%
\pgfsetfillopacity{0.500000}%
\pgfsetlinewidth{1.003750pt}%
\definecolor{currentstroke}{rgb}{0.980392,0.164706,0.333333}%
\pgfsetstrokecolor{currentstroke}%
\pgfsetstrokeopacity{0.500000}%
\pgfsetdash{{3.700000pt}{1.600000pt}}{0.000000pt}%
\pgfpathmoveto{\pgfqpoint{4.675283in}{3.168042in}}%
\pgfpathcurveto{\pgfqpoint{4.681107in}{3.168042in}}{\pgfqpoint{4.686693in}{3.170356in}}{\pgfqpoint{4.690811in}{3.174474in}}%
\pgfpathcurveto{\pgfqpoint{4.694929in}{3.178593in}}{\pgfqpoint{4.697243in}{3.184179in}}{\pgfqpoint{4.697243in}{3.190003in}}%
\pgfpathcurveto{\pgfqpoint{4.697243in}{3.195827in}}{\pgfqpoint{4.694929in}{3.201413in}}{\pgfqpoint{4.690811in}{3.205531in}}%
\pgfpathcurveto{\pgfqpoint{4.686693in}{3.209649in}}{\pgfqpoint{4.681107in}{3.211963in}}{\pgfqpoint{4.675283in}{3.211963in}}%
\pgfpathcurveto{\pgfqpoint{4.669459in}{3.211963in}}{\pgfqpoint{4.663873in}{3.209649in}}{\pgfqpoint{4.659755in}{3.205531in}}%
\pgfpathcurveto{\pgfqpoint{4.655637in}{3.201413in}}{\pgfqpoint{4.653323in}{3.195827in}}{\pgfqpoint{4.653323in}{3.190003in}}%
\pgfpathcurveto{\pgfqpoint{4.653323in}{3.184179in}}{\pgfqpoint{4.655637in}{3.178593in}}{\pgfqpoint{4.659755in}{3.174474in}}%
\pgfpathcurveto{\pgfqpoint{4.663873in}{3.170356in}}{\pgfqpoint{4.669459in}{3.168042in}}{\pgfqpoint{4.675283in}{3.168042in}}%
\pgfpathlineto{\pgfqpoint{4.675283in}{3.168042in}}%
\pgfpathclose%
\pgfusepath{stroke,fill}%
\end{pgfscope}%
\begin{pgfscope}%
\pgfpathrectangle{\pgfqpoint{0.640323in}{0.527436in}}{\pgfqpoint{9.687500in}{3.850000in}}%
\pgfusepath{clip}%
\pgfsetbuttcap%
\pgfsetroundjoin%
\definecolor{currentfill}{rgb}{0.980392,0.164706,0.333333}%
\pgfsetfillcolor{currentfill}%
\pgfsetfillopacity{0.500000}%
\pgfsetlinewidth{1.003750pt}%
\definecolor{currentstroke}{rgb}{0.980392,0.164706,0.333333}%
\pgfsetstrokecolor{currentstroke}%
\pgfsetstrokeopacity{0.500000}%
\pgfsetdash{{3.700000pt}{1.600000pt}}{0.000000pt}%
\pgfpathmoveto{\pgfqpoint{4.855014in}{3.205450in}}%
\pgfpathcurveto{\pgfqpoint{4.860838in}{3.205450in}}{\pgfqpoint{4.866424in}{3.207763in}}{\pgfqpoint{4.870542in}{3.211882in}}%
\pgfpathcurveto{\pgfqpoint{4.874660in}{3.216000in}}{\pgfqpoint{4.876974in}{3.221586in}}{\pgfqpoint{4.876974in}{3.227410in}}%
\pgfpathcurveto{\pgfqpoint{4.876974in}{3.233234in}}{\pgfqpoint{4.874660in}{3.238820in}}{\pgfqpoint{4.870542in}{3.242938in}}%
\pgfpathcurveto{\pgfqpoint{4.866424in}{3.247056in}}{\pgfqpoint{4.860838in}{3.249370in}}{\pgfqpoint{4.855014in}{3.249370in}}%
\pgfpathcurveto{\pgfqpoint{4.849190in}{3.249370in}}{\pgfqpoint{4.843604in}{3.247056in}}{\pgfqpoint{4.839486in}{3.242938in}}%
\pgfpathcurveto{\pgfqpoint{4.835368in}{3.238820in}}{\pgfqpoint{4.833054in}{3.233234in}}{\pgfqpoint{4.833054in}{3.227410in}}%
\pgfpathcurveto{\pgfqpoint{4.833054in}{3.221586in}}{\pgfqpoint{4.835368in}{3.216000in}}{\pgfqpoint{4.839486in}{3.211882in}}%
\pgfpathcurveto{\pgfqpoint{4.843604in}{3.207763in}}{\pgfqpoint{4.849190in}{3.205450in}}{\pgfqpoint{4.855014in}{3.205450in}}%
\pgfpathlineto{\pgfqpoint{4.855014in}{3.205450in}}%
\pgfpathclose%
\pgfusepath{stroke,fill}%
\end{pgfscope}%
\begin{pgfscope}%
\pgfpathrectangle{\pgfqpoint{0.640323in}{0.527436in}}{\pgfqpoint{9.687500in}{3.850000in}}%
\pgfusepath{clip}%
\pgfsetbuttcap%
\pgfsetroundjoin%
\definecolor{currentfill}{rgb}{0.980392,0.164706,0.333333}%
\pgfsetfillcolor{currentfill}%
\pgfsetfillopacity{0.500000}%
\pgfsetlinewidth{1.003750pt}%
\definecolor{currentstroke}{rgb}{0.980392,0.164706,0.333333}%
\pgfsetstrokecolor{currentstroke}%
\pgfsetstrokeopacity{0.500000}%
\pgfsetdash{{3.700000pt}{1.600000pt}}{0.000000pt}%
\pgfpathmoveto{\pgfqpoint{5.034745in}{3.260107in}}%
\pgfpathcurveto{\pgfqpoint{5.040569in}{3.260107in}}{\pgfqpoint{5.046155in}{3.262421in}}{\pgfqpoint{5.050273in}{3.266539in}}%
\pgfpathcurveto{\pgfqpoint{5.054391in}{3.270657in}}{\pgfqpoint{5.056705in}{3.276243in}}{\pgfqpoint{5.056705in}{3.282067in}}%
\pgfpathcurveto{\pgfqpoint{5.056705in}{3.287891in}}{\pgfqpoint{5.054391in}{3.293477in}}{\pgfqpoint{5.050273in}{3.297596in}}%
\pgfpathcurveto{\pgfqpoint{5.046155in}{3.301714in}}{\pgfqpoint{5.040569in}{3.304028in}}{\pgfqpoint{5.034745in}{3.304028in}}%
\pgfpathcurveto{\pgfqpoint{5.028921in}{3.304028in}}{\pgfqpoint{5.023335in}{3.301714in}}{\pgfqpoint{5.019217in}{3.297596in}}%
\pgfpathcurveto{\pgfqpoint{5.015099in}{3.293477in}}{\pgfqpoint{5.012785in}{3.287891in}}{\pgfqpoint{5.012785in}{3.282067in}}%
\pgfpathcurveto{\pgfqpoint{5.012785in}{3.276243in}}{\pgfqpoint{5.015099in}{3.270657in}}{\pgfqpoint{5.019217in}{3.266539in}}%
\pgfpathcurveto{\pgfqpoint{5.023335in}{3.262421in}}{\pgfqpoint{5.028921in}{3.260107in}}{\pgfqpoint{5.034745in}{3.260107in}}%
\pgfpathlineto{\pgfqpoint{5.034745in}{3.260107in}}%
\pgfpathclose%
\pgfusepath{stroke,fill}%
\end{pgfscope}%
\begin{pgfscope}%
\pgfpathrectangle{\pgfqpoint{0.640323in}{0.527436in}}{\pgfqpoint{9.687500in}{3.850000in}}%
\pgfusepath{clip}%
\pgfsetbuttcap%
\pgfsetroundjoin%
\definecolor{currentfill}{rgb}{0.980392,0.164706,0.333333}%
\pgfsetfillcolor{currentfill}%
\pgfsetfillopacity{0.500000}%
\pgfsetlinewidth{1.003750pt}%
\definecolor{currentstroke}{rgb}{0.980392,0.164706,0.333333}%
\pgfsetstrokecolor{currentstroke}%
\pgfsetstrokeopacity{0.500000}%
\pgfsetdash{{3.700000pt}{1.600000pt}}{0.000000pt}%
\pgfpathmoveto{\pgfqpoint{5.214476in}{3.330842in}}%
\pgfpathcurveto{\pgfqpoint{5.220300in}{3.330842in}}{\pgfqpoint{5.225886in}{3.333155in}}{\pgfqpoint{5.230004in}{3.337274in}}%
\pgfpathcurveto{\pgfqpoint{5.234122in}{3.341392in}}{\pgfqpoint{5.236436in}{3.346978in}}{\pgfqpoint{5.236436in}{3.352802in}}%
\pgfpathcurveto{\pgfqpoint{5.236436in}{3.358626in}}{\pgfqpoint{5.234122in}{3.364212in}}{\pgfqpoint{5.230004in}{3.368330in}}%
\pgfpathcurveto{\pgfqpoint{5.225886in}{3.372448in}}{\pgfqpoint{5.220300in}{3.374762in}}{\pgfqpoint{5.214476in}{3.374762in}}%
\pgfpathcurveto{\pgfqpoint{5.208652in}{3.374762in}}{\pgfqpoint{5.203066in}{3.372448in}}{\pgfqpoint{5.198948in}{3.368330in}}%
\pgfpathcurveto{\pgfqpoint{5.194830in}{3.364212in}}{\pgfqpoint{5.192516in}{3.358626in}}{\pgfqpoint{5.192516in}{3.352802in}}%
\pgfpathcurveto{\pgfqpoint{5.192516in}{3.346978in}}{\pgfqpoint{5.194830in}{3.341392in}}{\pgfqpoint{5.198948in}{3.337274in}}%
\pgfpathcurveto{\pgfqpoint{5.203066in}{3.333155in}}{\pgfqpoint{5.208652in}{3.330842in}}{\pgfqpoint{5.214476in}{3.330842in}}%
\pgfpathlineto{\pgfqpoint{5.214476in}{3.330842in}}%
\pgfpathclose%
\pgfusepath{stroke,fill}%
\end{pgfscope}%
\begin{pgfscope}%
\pgfpathrectangle{\pgfqpoint{0.640323in}{0.527436in}}{\pgfqpoint{9.687500in}{3.850000in}}%
\pgfusepath{clip}%
\pgfsetbuttcap%
\pgfsetroundjoin%
\definecolor{currentfill}{rgb}{0.980392,0.164706,0.333333}%
\pgfsetfillcolor{currentfill}%
\pgfsetfillopacity{0.500000}%
\pgfsetlinewidth{1.003750pt}%
\definecolor{currentstroke}{rgb}{0.980392,0.164706,0.333333}%
\pgfsetstrokecolor{currentstroke}%
\pgfsetstrokeopacity{0.500000}%
\pgfsetdash{{3.700000pt}{1.600000pt}}{0.000000pt}%
\pgfpathmoveto{\pgfqpoint{5.394207in}{3.374899in}}%
\pgfpathcurveto{\pgfqpoint{5.400031in}{3.374899in}}{\pgfqpoint{5.405617in}{3.377213in}}{\pgfqpoint{5.409735in}{3.381331in}}%
\pgfpathcurveto{\pgfqpoint{5.413853in}{3.385449in}}{\pgfqpoint{5.416167in}{3.391036in}}{\pgfqpoint{5.416167in}{3.396859in}}%
\pgfpathcurveto{\pgfqpoint{5.416167in}{3.402683in}}{\pgfqpoint{5.413853in}{3.408270in}}{\pgfqpoint{5.409735in}{3.412388in}}%
\pgfpathcurveto{\pgfqpoint{5.405617in}{3.416506in}}{\pgfqpoint{5.400031in}{3.418820in}}{\pgfqpoint{5.394207in}{3.418820in}}%
\pgfpathcurveto{\pgfqpoint{5.388383in}{3.418820in}}{\pgfqpoint{5.382797in}{3.416506in}}{\pgfqpoint{5.378679in}{3.412388in}}%
\pgfpathcurveto{\pgfqpoint{5.374561in}{3.408270in}}{\pgfqpoint{5.372247in}{3.402683in}}{\pgfqpoint{5.372247in}{3.396859in}}%
\pgfpathcurveto{\pgfqpoint{5.372247in}{3.391036in}}{\pgfqpoint{5.374561in}{3.385449in}}{\pgfqpoint{5.378679in}{3.381331in}}%
\pgfpathcurveto{\pgfqpoint{5.382797in}{3.377213in}}{\pgfqpoint{5.388383in}{3.374899in}}{\pgfqpoint{5.394207in}{3.374899in}}%
\pgfpathlineto{\pgfqpoint{5.394207in}{3.374899in}}%
\pgfpathclose%
\pgfusepath{stroke,fill}%
\end{pgfscope}%
\begin{pgfscope}%
\pgfpathrectangle{\pgfqpoint{0.640323in}{0.527436in}}{\pgfqpoint{9.687500in}{3.850000in}}%
\pgfusepath{clip}%
\pgfsetbuttcap%
\pgfsetroundjoin%
\definecolor{currentfill}{rgb}{0.980392,0.164706,0.333333}%
\pgfsetfillcolor{currentfill}%
\pgfsetfillopacity{0.500000}%
\pgfsetlinewidth{1.003750pt}%
\definecolor{currentstroke}{rgb}{0.980392,0.164706,0.333333}%
\pgfsetstrokecolor{currentstroke}%
\pgfsetstrokeopacity{0.500000}%
\pgfsetdash{{3.700000pt}{1.600000pt}}{0.000000pt}%
\pgfpathmoveto{\pgfqpoint{5.573938in}{3.416479in}}%
\pgfpathcurveto{\pgfqpoint{5.579762in}{3.416479in}}{\pgfqpoint{5.585348in}{3.418793in}}{\pgfqpoint{5.589466in}{3.422911in}}%
\pgfpathcurveto{\pgfqpoint{5.593584in}{3.427029in}}{\pgfqpoint{5.595898in}{3.432616in}}{\pgfqpoint{5.595898in}{3.438439in}}%
\pgfpathcurveto{\pgfqpoint{5.595898in}{3.444263in}}{\pgfqpoint{5.593584in}{3.449850in}}{\pgfqpoint{5.589466in}{3.453968in}}%
\pgfpathcurveto{\pgfqpoint{5.585348in}{3.458086in}}{\pgfqpoint{5.579762in}{3.460400in}}{\pgfqpoint{5.573938in}{3.460400in}}%
\pgfpathcurveto{\pgfqpoint{5.568114in}{3.460400in}}{\pgfqpoint{5.562528in}{3.458086in}}{\pgfqpoint{5.558410in}{3.453968in}}%
\pgfpathcurveto{\pgfqpoint{5.554292in}{3.449850in}}{\pgfqpoint{5.551978in}{3.444263in}}{\pgfqpoint{5.551978in}{3.438439in}}%
\pgfpathcurveto{\pgfqpoint{5.551978in}{3.432616in}}{\pgfqpoint{5.554292in}{3.427029in}}{\pgfqpoint{5.558410in}{3.422911in}}%
\pgfpathcurveto{\pgfqpoint{5.562528in}{3.418793in}}{\pgfqpoint{5.568114in}{3.416479in}}{\pgfqpoint{5.573938in}{3.416479in}}%
\pgfpathlineto{\pgfqpoint{5.573938in}{3.416479in}}%
\pgfpathclose%
\pgfusepath{stroke,fill}%
\end{pgfscope}%
\begin{pgfscope}%
\pgfpathrectangle{\pgfqpoint{0.640323in}{0.527436in}}{\pgfqpoint{9.687500in}{3.850000in}}%
\pgfusepath{clip}%
\pgfsetbuttcap%
\pgfsetroundjoin%
\definecolor{currentfill}{rgb}{0.980392,0.164706,0.333333}%
\pgfsetfillcolor{currentfill}%
\pgfsetfillopacity{0.500000}%
\pgfsetlinewidth{1.003750pt}%
\definecolor{currentstroke}{rgb}{0.980392,0.164706,0.333333}%
\pgfsetstrokecolor{currentstroke}%
\pgfsetstrokeopacity{0.500000}%
\pgfsetdash{{3.700000pt}{1.600000pt}}{0.000000pt}%
\pgfpathmoveto{\pgfqpoint{5.753669in}{3.461934in}}%
\pgfpathcurveto{\pgfqpoint{5.759493in}{3.461934in}}{\pgfqpoint{5.765079in}{3.464248in}}{\pgfqpoint{5.769197in}{3.468366in}}%
\pgfpathcurveto{\pgfqpoint{5.773315in}{3.472484in}}{\pgfqpoint{5.775629in}{3.478070in}}{\pgfqpoint{5.775629in}{3.483894in}}%
\pgfpathcurveto{\pgfqpoint{5.775629in}{3.489718in}}{\pgfqpoint{5.773315in}{3.495304in}}{\pgfqpoint{5.769197in}{3.499423in}}%
\pgfpathcurveto{\pgfqpoint{5.765079in}{3.503541in}}{\pgfqpoint{5.759493in}{3.505855in}}{\pgfqpoint{5.753669in}{3.505855in}}%
\pgfpathcurveto{\pgfqpoint{5.747845in}{3.505855in}}{\pgfqpoint{5.742259in}{3.503541in}}{\pgfqpoint{5.738141in}{3.499423in}}%
\pgfpathcurveto{\pgfqpoint{5.734023in}{3.495304in}}{\pgfqpoint{5.731709in}{3.489718in}}{\pgfqpoint{5.731709in}{3.483894in}}%
\pgfpathcurveto{\pgfqpoint{5.731709in}{3.478070in}}{\pgfqpoint{5.734023in}{3.472484in}}{\pgfqpoint{5.738141in}{3.468366in}}%
\pgfpathcurveto{\pgfqpoint{5.742259in}{3.464248in}}{\pgfqpoint{5.747845in}{3.461934in}}{\pgfqpoint{5.753669in}{3.461934in}}%
\pgfpathlineto{\pgfqpoint{5.753669in}{3.461934in}}%
\pgfpathclose%
\pgfusepath{stroke,fill}%
\end{pgfscope}%
\begin{pgfscope}%
\pgfpathrectangle{\pgfqpoint{0.640323in}{0.527436in}}{\pgfqpoint{9.687500in}{3.850000in}}%
\pgfusepath{clip}%
\pgfsetbuttcap%
\pgfsetroundjoin%
\definecolor{currentfill}{rgb}{0.980392,0.164706,0.333333}%
\pgfsetfillcolor{currentfill}%
\pgfsetfillopacity{0.500000}%
\pgfsetlinewidth{1.003750pt}%
\definecolor{currentstroke}{rgb}{0.980392,0.164706,0.333333}%
\pgfsetstrokecolor{currentstroke}%
\pgfsetstrokeopacity{0.500000}%
\pgfsetdash{{3.700000pt}{1.600000pt}}{0.000000pt}%
\pgfpathmoveto{\pgfqpoint{5.933400in}{3.502781in}}%
\pgfpathcurveto{\pgfqpoint{5.939224in}{3.502781in}}{\pgfqpoint{5.944810in}{3.505095in}}{\pgfqpoint{5.948928in}{3.509213in}}%
\pgfpathcurveto{\pgfqpoint{5.953046in}{3.513331in}}{\pgfqpoint{5.955360in}{3.518918in}}{\pgfqpoint{5.955360in}{3.524742in}}%
\pgfpathcurveto{\pgfqpoint{5.955360in}{3.530565in}}{\pgfqpoint{5.953046in}{3.536152in}}{\pgfqpoint{5.948928in}{3.540270in}}%
\pgfpathcurveto{\pgfqpoint{5.944810in}{3.544388in}}{\pgfqpoint{5.939224in}{3.546702in}}{\pgfqpoint{5.933400in}{3.546702in}}%
\pgfpathcurveto{\pgfqpoint{5.927576in}{3.546702in}}{\pgfqpoint{5.921990in}{3.544388in}}{\pgfqpoint{5.917872in}{3.540270in}}%
\pgfpathcurveto{\pgfqpoint{5.913754in}{3.536152in}}{\pgfqpoint{5.911440in}{3.530565in}}{\pgfqpoint{5.911440in}{3.524742in}}%
\pgfpathcurveto{\pgfqpoint{5.911440in}{3.518918in}}{\pgfqpoint{5.913754in}{3.513331in}}{\pgfqpoint{5.917872in}{3.509213in}}%
\pgfpathcurveto{\pgfqpoint{5.921990in}{3.505095in}}{\pgfqpoint{5.927576in}{3.502781in}}{\pgfqpoint{5.933400in}{3.502781in}}%
\pgfpathlineto{\pgfqpoint{5.933400in}{3.502781in}}%
\pgfpathclose%
\pgfusepath{stroke,fill}%
\end{pgfscope}%
\begin{pgfscope}%
\pgfpathrectangle{\pgfqpoint{0.640323in}{0.527436in}}{\pgfqpoint{9.687500in}{3.850000in}}%
\pgfusepath{clip}%
\pgfsetbuttcap%
\pgfsetroundjoin%
\definecolor{currentfill}{rgb}{0.980392,0.164706,0.333333}%
\pgfsetfillcolor{currentfill}%
\pgfsetfillopacity{0.500000}%
\pgfsetlinewidth{1.003750pt}%
\definecolor{currentstroke}{rgb}{0.980392,0.164706,0.333333}%
\pgfsetstrokecolor{currentstroke}%
\pgfsetstrokeopacity{0.500000}%
\pgfsetdash{{3.700000pt}{1.600000pt}}{0.000000pt}%
\pgfpathmoveto{\pgfqpoint{6.113131in}{3.538251in}}%
\pgfpathcurveto{\pgfqpoint{6.118955in}{3.538251in}}{\pgfqpoint{6.124541in}{3.540565in}}{\pgfqpoint{6.128659in}{3.544683in}}%
\pgfpathcurveto{\pgfqpoint{6.132777in}{3.548801in}}{\pgfqpoint{6.135091in}{3.554387in}}{\pgfqpoint{6.135091in}{3.560211in}}%
\pgfpathcurveto{\pgfqpoint{6.135091in}{3.566035in}}{\pgfqpoint{6.132777in}{3.571621in}}{\pgfqpoint{6.128659in}{3.575739in}}%
\pgfpathcurveto{\pgfqpoint{6.124541in}{3.579858in}}{\pgfqpoint{6.118955in}{3.582171in}}{\pgfqpoint{6.113131in}{3.582171in}}%
\pgfpathcurveto{\pgfqpoint{6.107307in}{3.582171in}}{\pgfqpoint{6.101721in}{3.579858in}}{\pgfqpoint{6.097603in}{3.575739in}}%
\pgfpathcurveto{\pgfqpoint{6.093485in}{3.571621in}}{\pgfqpoint{6.091171in}{3.566035in}}{\pgfqpoint{6.091171in}{3.560211in}}%
\pgfpathcurveto{\pgfqpoint{6.091171in}{3.554387in}}{\pgfqpoint{6.093485in}{3.548801in}}{\pgfqpoint{6.097603in}{3.544683in}}%
\pgfpathcurveto{\pgfqpoint{6.101721in}{3.540565in}}{\pgfqpoint{6.107307in}{3.538251in}}{\pgfqpoint{6.113131in}{3.538251in}}%
\pgfpathlineto{\pgfqpoint{6.113131in}{3.538251in}}%
\pgfpathclose%
\pgfusepath{stroke,fill}%
\end{pgfscope}%
\begin{pgfscope}%
\pgfpathrectangle{\pgfqpoint{0.640323in}{0.527436in}}{\pgfqpoint{9.687500in}{3.850000in}}%
\pgfusepath{clip}%
\pgfsetbuttcap%
\pgfsetroundjoin%
\definecolor{currentfill}{rgb}{0.980392,0.164706,0.333333}%
\pgfsetfillcolor{currentfill}%
\pgfsetfillopacity{0.500000}%
\pgfsetlinewidth{1.003750pt}%
\definecolor{currentstroke}{rgb}{0.980392,0.164706,0.333333}%
\pgfsetstrokecolor{currentstroke}%
\pgfsetstrokeopacity{0.500000}%
\pgfsetdash{{3.700000pt}{1.600000pt}}{0.000000pt}%
\pgfpathmoveto{\pgfqpoint{6.292862in}{3.581210in}}%
\pgfpathcurveto{\pgfqpoint{6.298686in}{3.581210in}}{\pgfqpoint{6.304272in}{3.583523in}}{\pgfqpoint{6.308390in}{3.587642in}}%
\pgfpathcurveto{\pgfqpoint{6.312508in}{3.591760in}}{\pgfqpoint{6.314822in}{3.597346in}}{\pgfqpoint{6.314822in}{3.603170in}}%
\pgfpathcurveto{\pgfqpoint{6.314822in}{3.608994in}}{\pgfqpoint{6.312508in}{3.614580in}}{\pgfqpoint{6.308390in}{3.618698in}}%
\pgfpathcurveto{\pgfqpoint{6.304272in}{3.622816in}}{\pgfqpoint{6.298686in}{3.625130in}}{\pgfqpoint{6.292862in}{3.625130in}}%
\pgfpathcurveto{\pgfqpoint{6.287038in}{3.625130in}}{\pgfqpoint{6.281452in}{3.622816in}}{\pgfqpoint{6.277334in}{3.618698in}}%
\pgfpathcurveto{\pgfqpoint{6.273216in}{3.614580in}}{\pgfqpoint{6.270902in}{3.608994in}}{\pgfqpoint{6.270902in}{3.603170in}}%
\pgfpathcurveto{\pgfqpoint{6.270902in}{3.597346in}}{\pgfqpoint{6.273216in}{3.591760in}}{\pgfqpoint{6.277334in}{3.587642in}}%
\pgfpathcurveto{\pgfqpoint{6.281452in}{3.583523in}}{\pgfqpoint{6.287038in}{3.581210in}}{\pgfqpoint{6.292862in}{3.581210in}}%
\pgfpathlineto{\pgfqpoint{6.292862in}{3.581210in}}%
\pgfpathclose%
\pgfusepath{stroke,fill}%
\end{pgfscope}%
\begin{pgfscope}%
\pgfpathrectangle{\pgfqpoint{0.640323in}{0.527436in}}{\pgfqpoint{9.687500in}{3.850000in}}%
\pgfusepath{clip}%
\pgfsetbuttcap%
\pgfsetroundjoin%
\definecolor{currentfill}{rgb}{0.980392,0.164706,0.333333}%
\pgfsetfillcolor{currentfill}%
\pgfsetfillopacity{0.500000}%
\pgfsetlinewidth{1.003750pt}%
\definecolor{currentstroke}{rgb}{0.980392,0.164706,0.333333}%
\pgfsetstrokecolor{currentstroke}%
\pgfsetstrokeopacity{0.500000}%
\pgfsetdash{{3.700000pt}{1.600000pt}}{0.000000pt}%
\pgfpathmoveto{\pgfqpoint{6.472593in}{3.617486in}}%
\pgfpathcurveto{\pgfqpoint{6.478417in}{3.617486in}}{\pgfqpoint{6.484003in}{3.619800in}}{\pgfqpoint{6.488121in}{3.623918in}}%
\pgfpathcurveto{\pgfqpoint{6.492239in}{3.628037in}}{\pgfqpoint{6.494553in}{3.633623in}}{\pgfqpoint{6.494553in}{3.639447in}}%
\pgfpathcurveto{\pgfqpoint{6.494553in}{3.645271in}}{\pgfqpoint{6.492239in}{3.650857in}}{\pgfqpoint{6.488121in}{3.654975in}}%
\pgfpathcurveto{\pgfqpoint{6.484003in}{3.659093in}}{\pgfqpoint{6.478417in}{3.661407in}}{\pgfqpoint{6.472593in}{3.661407in}}%
\pgfpathcurveto{\pgfqpoint{6.466769in}{3.661407in}}{\pgfqpoint{6.461183in}{3.659093in}}{\pgfqpoint{6.457065in}{3.654975in}}%
\pgfpathcurveto{\pgfqpoint{6.452947in}{3.650857in}}{\pgfqpoint{6.450633in}{3.645271in}}{\pgfqpoint{6.450633in}{3.639447in}}%
\pgfpathcurveto{\pgfqpoint{6.450633in}{3.633623in}}{\pgfqpoint{6.452947in}{3.628037in}}{\pgfqpoint{6.457065in}{3.623918in}}%
\pgfpathcurveto{\pgfqpoint{6.461183in}{3.619800in}}{\pgfqpoint{6.466769in}{3.617486in}}{\pgfqpoint{6.472593in}{3.617486in}}%
\pgfpathlineto{\pgfqpoint{6.472593in}{3.617486in}}%
\pgfpathclose%
\pgfusepath{stroke,fill}%
\end{pgfscope}%
\begin{pgfscope}%
\pgfpathrectangle{\pgfqpoint{0.640323in}{0.527436in}}{\pgfqpoint{9.687500in}{3.850000in}}%
\pgfusepath{clip}%
\pgfsetbuttcap%
\pgfsetroundjoin%
\definecolor{currentfill}{rgb}{0.980392,0.164706,0.333333}%
\pgfsetfillcolor{currentfill}%
\pgfsetfillopacity{0.500000}%
\pgfsetlinewidth{1.003750pt}%
\definecolor{currentstroke}{rgb}{0.980392,0.164706,0.333333}%
\pgfsetstrokecolor{currentstroke}%
\pgfsetstrokeopacity{0.500000}%
\pgfsetdash{{3.700000pt}{1.600000pt}}{0.000000pt}%
\pgfpathmoveto{\pgfqpoint{6.652324in}{3.634544in}}%
\pgfpathcurveto{\pgfqpoint{6.658148in}{3.634544in}}{\pgfqpoint{6.663734in}{3.636858in}}{\pgfqpoint{6.667852in}{3.640976in}}%
\pgfpathcurveto{\pgfqpoint{6.671970in}{3.645095in}}{\pgfqpoint{6.674284in}{3.650681in}}{\pgfqpoint{6.674284in}{3.656505in}}%
\pgfpathcurveto{\pgfqpoint{6.674284in}{3.662329in}}{\pgfqpoint{6.671970in}{3.667915in}}{\pgfqpoint{6.667852in}{3.672033in}}%
\pgfpathcurveto{\pgfqpoint{6.663734in}{3.676151in}}{\pgfqpoint{6.658148in}{3.678465in}}{\pgfqpoint{6.652324in}{3.678465in}}%
\pgfpathcurveto{\pgfqpoint{6.646500in}{3.678465in}}{\pgfqpoint{6.640914in}{3.676151in}}{\pgfqpoint{6.636796in}{3.672033in}}%
\pgfpathcurveto{\pgfqpoint{6.632678in}{3.667915in}}{\pgfqpoint{6.630364in}{3.662329in}}{\pgfqpoint{6.630364in}{3.656505in}}%
\pgfpathcurveto{\pgfqpoint{6.630364in}{3.650681in}}{\pgfqpoint{6.632678in}{3.645095in}}{\pgfqpoint{6.636796in}{3.640976in}}%
\pgfpathcurveto{\pgfqpoint{6.640914in}{3.636858in}}{\pgfqpoint{6.646500in}{3.634544in}}{\pgfqpoint{6.652324in}{3.634544in}}%
\pgfpathlineto{\pgfqpoint{6.652324in}{3.634544in}}%
\pgfpathclose%
\pgfusepath{stroke,fill}%
\end{pgfscope}%
\begin{pgfscope}%
\pgfpathrectangle{\pgfqpoint{0.640323in}{0.527436in}}{\pgfqpoint{9.687500in}{3.850000in}}%
\pgfusepath{clip}%
\pgfsetbuttcap%
\pgfsetroundjoin%
\definecolor{currentfill}{rgb}{0.980392,0.164706,0.333333}%
\pgfsetfillcolor{currentfill}%
\pgfsetfillopacity{0.500000}%
\pgfsetlinewidth{1.003750pt}%
\definecolor{currentstroke}{rgb}{0.980392,0.164706,0.333333}%
\pgfsetstrokecolor{currentstroke}%
\pgfsetstrokeopacity{0.500000}%
\pgfsetdash{{3.700000pt}{1.600000pt}}{0.000000pt}%
\pgfpathmoveto{\pgfqpoint{6.832055in}{3.665661in}}%
\pgfpathcurveto{\pgfqpoint{6.837879in}{3.665661in}}{\pgfqpoint{6.843465in}{3.667975in}}{\pgfqpoint{6.847583in}{3.672093in}}%
\pgfpathcurveto{\pgfqpoint{6.851701in}{3.676211in}}{\pgfqpoint{6.854015in}{3.681797in}}{\pgfqpoint{6.854015in}{3.687621in}}%
\pgfpathcurveto{\pgfqpoint{6.854015in}{3.693445in}}{\pgfqpoint{6.851701in}{3.699032in}}{\pgfqpoint{6.847583in}{3.703150in}}%
\pgfpathcurveto{\pgfqpoint{6.843465in}{3.707268in}}{\pgfqpoint{6.837879in}{3.709582in}}{\pgfqpoint{6.832055in}{3.709582in}}%
\pgfpathcurveto{\pgfqpoint{6.826231in}{3.709582in}}{\pgfqpoint{6.820645in}{3.707268in}}{\pgfqpoint{6.816527in}{3.703150in}}%
\pgfpathcurveto{\pgfqpoint{6.812408in}{3.699032in}}{\pgfqpoint{6.810095in}{3.693445in}}{\pgfqpoint{6.810095in}{3.687621in}}%
\pgfpathcurveto{\pgfqpoint{6.810095in}{3.681797in}}{\pgfqpoint{6.812408in}{3.676211in}}{\pgfqpoint{6.816527in}{3.672093in}}%
\pgfpathcurveto{\pgfqpoint{6.820645in}{3.667975in}}{\pgfqpoint{6.826231in}{3.665661in}}{\pgfqpoint{6.832055in}{3.665661in}}%
\pgfpathlineto{\pgfqpoint{6.832055in}{3.665661in}}%
\pgfpathclose%
\pgfusepath{stroke,fill}%
\end{pgfscope}%
\begin{pgfscope}%
\pgfpathrectangle{\pgfqpoint{0.640323in}{0.527436in}}{\pgfqpoint{9.687500in}{3.850000in}}%
\pgfusepath{clip}%
\pgfsetbuttcap%
\pgfsetroundjoin%
\definecolor{currentfill}{rgb}{0.980392,0.164706,0.333333}%
\pgfsetfillcolor{currentfill}%
\pgfsetfillopacity{0.500000}%
\pgfsetlinewidth{1.003750pt}%
\definecolor{currentstroke}{rgb}{0.980392,0.164706,0.333333}%
\pgfsetstrokecolor{currentstroke}%
\pgfsetstrokeopacity{0.500000}%
\pgfsetdash{{3.700000pt}{1.600000pt}}{0.000000pt}%
\pgfpathmoveto{\pgfqpoint{7.011786in}{3.694393in}}%
\pgfpathcurveto{\pgfqpoint{7.017610in}{3.694393in}}{\pgfqpoint{7.023196in}{3.696707in}}{\pgfqpoint{7.027314in}{3.700825in}}%
\pgfpathcurveto{\pgfqpoint{7.031432in}{3.704943in}}{\pgfqpoint{7.033746in}{3.710530in}}{\pgfqpoint{7.033746in}{3.716354in}}%
\pgfpathcurveto{\pgfqpoint{7.033746in}{3.722178in}}{\pgfqpoint{7.031432in}{3.727764in}}{\pgfqpoint{7.027314in}{3.731882in}}%
\pgfpathcurveto{\pgfqpoint{7.023196in}{3.736000in}}{\pgfqpoint{7.017610in}{3.738314in}}{\pgfqpoint{7.011786in}{3.738314in}}%
\pgfpathcurveto{\pgfqpoint{7.005962in}{3.738314in}}{\pgfqpoint{7.000376in}{3.736000in}}{\pgfqpoint{6.996258in}{3.731882in}}%
\pgfpathcurveto{\pgfqpoint{6.992139in}{3.727764in}}{\pgfqpoint{6.989826in}{3.722178in}}{\pgfqpoint{6.989826in}{3.716354in}}%
\pgfpathcurveto{\pgfqpoint{6.989826in}{3.710530in}}{\pgfqpoint{6.992139in}{3.704943in}}{\pgfqpoint{6.996258in}{3.700825in}}%
\pgfpathcurveto{\pgfqpoint{7.000376in}{3.696707in}}{\pgfqpoint{7.005962in}{3.694393in}}{\pgfqpoint{7.011786in}{3.694393in}}%
\pgfpathlineto{\pgfqpoint{7.011786in}{3.694393in}}%
\pgfpathclose%
\pgfusepath{stroke,fill}%
\end{pgfscope}%
\begin{pgfscope}%
\pgfpathrectangle{\pgfqpoint{0.640323in}{0.527436in}}{\pgfqpoint{9.687500in}{3.850000in}}%
\pgfusepath{clip}%
\pgfsetbuttcap%
\pgfsetroundjoin%
\definecolor{currentfill}{rgb}{0.980392,0.164706,0.333333}%
\pgfsetfillcolor{currentfill}%
\pgfsetfillopacity{0.500000}%
\pgfsetlinewidth{1.003750pt}%
\definecolor{currentstroke}{rgb}{0.980392,0.164706,0.333333}%
\pgfsetstrokecolor{currentstroke}%
\pgfsetstrokeopacity{0.500000}%
\pgfsetdash{{3.700000pt}{1.600000pt}}{0.000000pt}%
\pgfpathmoveto{\pgfqpoint{7.191517in}{3.737855in}}%
\pgfpathcurveto{\pgfqpoint{7.197341in}{3.737855in}}{\pgfqpoint{7.202927in}{3.740169in}}{\pgfqpoint{7.207045in}{3.744287in}}%
\pgfpathcurveto{\pgfqpoint{7.211163in}{3.748405in}}{\pgfqpoint{7.213477in}{3.753991in}}{\pgfqpoint{7.213477in}{3.759815in}}%
\pgfpathcurveto{\pgfqpoint{7.213477in}{3.765639in}}{\pgfqpoint{7.211163in}{3.771225in}}{\pgfqpoint{7.207045in}{3.775343in}}%
\pgfpathcurveto{\pgfqpoint{7.202927in}{3.779461in}}{\pgfqpoint{7.197341in}{3.781775in}}{\pgfqpoint{7.191517in}{3.781775in}}%
\pgfpathcurveto{\pgfqpoint{7.185693in}{3.781775in}}{\pgfqpoint{7.180107in}{3.779461in}}{\pgfqpoint{7.175989in}{3.775343in}}%
\pgfpathcurveto{\pgfqpoint{7.171870in}{3.771225in}}{\pgfqpoint{7.169557in}{3.765639in}}{\pgfqpoint{7.169557in}{3.759815in}}%
\pgfpathcurveto{\pgfqpoint{7.169557in}{3.753991in}}{\pgfqpoint{7.171870in}{3.748405in}}{\pgfqpoint{7.175989in}{3.744287in}}%
\pgfpathcurveto{\pgfqpoint{7.180107in}{3.740169in}}{\pgfqpoint{7.185693in}{3.737855in}}{\pgfqpoint{7.191517in}{3.737855in}}%
\pgfpathlineto{\pgfqpoint{7.191517in}{3.737855in}}%
\pgfpathclose%
\pgfusepath{stroke,fill}%
\end{pgfscope}%
\begin{pgfscope}%
\pgfpathrectangle{\pgfqpoint{0.640323in}{0.527436in}}{\pgfqpoint{9.687500in}{3.850000in}}%
\pgfusepath{clip}%
\pgfsetbuttcap%
\pgfsetroundjoin%
\definecolor{currentfill}{rgb}{0.980392,0.164706,0.333333}%
\pgfsetfillcolor{currentfill}%
\pgfsetfillopacity{0.500000}%
\pgfsetlinewidth{1.003750pt}%
\definecolor{currentstroke}{rgb}{0.980392,0.164706,0.333333}%
\pgfsetstrokecolor{currentstroke}%
\pgfsetstrokeopacity{0.500000}%
\pgfsetdash{{3.700000pt}{1.600000pt}}{0.000000pt}%
\pgfpathmoveto{\pgfqpoint{7.371248in}{3.762495in}}%
\pgfpathcurveto{\pgfqpoint{7.377072in}{3.762495in}}{\pgfqpoint{7.382658in}{3.764809in}}{\pgfqpoint{7.386776in}{3.768927in}}%
\pgfpathcurveto{\pgfqpoint{7.390894in}{3.773045in}}{\pgfqpoint{7.393208in}{3.778631in}}{\pgfqpoint{7.393208in}{3.784455in}}%
\pgfpathcurveto{\pgfqpoint{7.393208in}{3.790279in}}{\pgfqpoint{7.390894in}{3.795865in}}{\pgfqpoint{7.386776in}{3.799983in}}%
\pgfpathcurveto{\pgfqpoint{7.382658in}{3.804101in}}{\pgfqpoint{7.377072in}{3.806415in}}{\pgfqpoint{7.371248in}{3.806415in}}%
\pgfpathcurveto{\pgfqpoint{7.365424in}{3.806415in}}{\pgfqpoint{7.359838in}{3.804101in}}{\pgfqpoint{7.355720in}{3.799983in}}%
\pgfpathcurveto{\pgfqpoint{7.351601in}{3.795865in}}{\pgfqpoint{7.349288in}{3.790279in}}{\pgfqpoint{7.349288in}{3.784455in}}%
\pgfpathcurveto{\pgfqpoint{7.349288in}{3.778631in}}{\pgfqpoint{7.351601in}{3.773045in}}{\pgfqpoint{7.355720in}{3.768927in}}%
\pgfpathcurveto{\pgfqpoint{7.359838in}{3.764809in}}{\pgfqpoint{7.365424in}{3.762495in}}{\pgfqpoint{7.371248in}{3.762495in}}%
\pgfpathlineto{\pgfqpoint{7.371248in}{3.762495in}}%
\pgfpathclose%
\pgfusepath{stroke,fill}%
\end{pgfscope}%
\begin{pgfscope}%
\pgfpathrectangle{\pgfqpoint{0.640323in}{0.527436in}}{\pgfqpoint{9.687500in}{3.850000in}}%
\pgfusepath{clip}%
\pgfsetbuttcap%
\pgfsetroundjoin%
\definecolor{currentfill}{rgb}{0.980392,0.164706,0.333333}%
\pgfsetfillcolor{currentfill}%
\pgfsetfillopacity{0.500000}%
\pgfsetlinewidth{1.003750pt}%
\definecolor{currentstroke}{rgb}{0.980392,0.164706,0.333333}%
\pgfsetstrokecolor{currentstroke}%
\pgfsetstrokeopacity{0.500000}%
\pgfsetdash{{3.700000pt}{1.600000pt}}{0.000000pt}%
\pgfpathmoveto{\pgfqpoint{7.550979in}{3.776578in}}%
\pgfpathcurveto{\pgfqpoint{7.556803in}{3.776578in}}{\pgfqpoint{7.562389in}{3.778892in}}{\pgfqpoint{7.566507in}{3.783010in}}%
\pgfpathcurveto{\pgfqpoint{7.570625in}{3.787129in}}{\pgfqpoint{7.572939in}{3.792715in}}{\pgfqpoint{7.572939in}{3.798539in}}%
\pgfpathcurveto{\pgfqpoint{7.572939in}{3.804363in}}{\pgfqpoint{7.570625in}{3.809949in}}{\pgfqpoint{7.566507in}{3.814067in}}%
\pgfpathcurveto{\pgfqpoint{7.562389in}{3.818185in}}{\pgfqpoint{7.556803in}{3.820499in}}{\pgfqpoint{7.550979in}{3.820499in}}%
\pgfpathcurveto{\pgfqpoint{7.545155in}{3.820499in}}{\pgfqpoint{7.539569in}{3.818185in}}{\pgfqpoint{7.535451in}{3.814067in}}%
\pgfpathcurveto{\pgfqpoint{7.531332in}{3.809949in}}{\pgfqpoint{7.529019in}{3.804363in}}{\pgfqpoint{7.529019in}{3.798539in}}%
\pgfpathcurveto{\pgfqpoint{7.529019in}{3.792715in}}{\pgfqpoint{7.531332in}{3.787129in}}{\pgfqpoint{7.535451in}{3.783010in}}%
\pgfpathcurveto{\pgfqpoint{7.539569in}{3.778892in}}{\pgfqpoint{7.545155in}{3.776578in}}{\pgfqpoint{7.550979in}{3.776578in}}%
\pgfpathlineto{\pgfqpoint{7.550979in}{3.776578in}}%
\pgfpathclose%
\pgfusepath{stroke,fill}%
\end{pgfscope}%
\begin{pgfscope}%
\pgfpathrectangle{\pgfqpoint{0.640323in}{0.527436in}}{\pgfqpoint{9.687500in}{3.850000in}}%
\pgfusepath{clip}%
\pgfsetbuttcap%
\pgfsetroundjoin%
\definecolor{currentfill}{rgb}{0.980392,0.164706,0.333333}%
\pgfsetfillcolor{currentfill}%
\pgfsetfillopacity{0.500000}%
\pgfsetlinewidth{1.003750pt}%
\definecolor{currentstroke}{rgb}{0.980392,0.164706,0.333333}%
\pgfsetstrokecolor{currentstroke}%
\pgfsetstrokeopacity{0.500000}%
\pgfsetdash{{3.700000pt}{1.600000pt}}{0.000000pt}%
\pgfpathmoveto{\pgfqpoint{7.730710in}{3.800380in}}%
\pgfpathcurveto{\pgfqpoint{7.736534in}{3.800380in}}{\pgfqpoint{7.742120in}{3.802694in}}{\pgfqpoint{7.746238in}{3.806812in}}%
\pgfpathcurveto{\pgfqpoint{7.750356in}{3.810930in}}{\pgfqpoint{7.752670in}{3.816516in}}{\pgfqpoint{7.752670in}{3.822340in}}%
\pgfpathcurveto{\pgfqpoint{7.752670in}{3.828164in}}{\pgfqpoint{7.750356in}{3.833750in}}{\pgfqpoint{7.746238in}{3.837869in}}%
\pgfpathcurveto{\pgfqpoint{7.742120in}{3.841987in}}{\pgfqpoint{7.736534in}{3.844301in}}{\pgfqpoint{7.730710in}{3.844301in}}%
\pgfpathcurveto{\pgfqpoint{7.724886in}{3.844301in}}{\pgfqpoint{7.719300in}{3.841987in}}{\pgfqpoint{7.715182in}{3.837869in}}%
\pgfpathcurveto{\pgfqpoint{7.711063in}{3.833750in}}{\pgfqpoint{7.708750in}{3.828164in}}{\pgfqpoint{7.708750in}{3.822340in}}%
\pgfpathcurveto{\pgfqpoint{7.708750in}{3.816516in}}{\pgfqpoint{7.711063in}{3.810930in}}{\pgfqpoint{7.715182in}{3.806812in}}%
\pgfpathcurveto{\pgfqpoint{7.719300in}{3.802694in}}{\pgfqpoint{7.724886in}{3.800380in}}{\pgfqpoint{7.730710in}{3.800380in}}%
\pgfpathlineto{\pgfqpoint{7.730710in}{3.800380in}}%
\pgfpathclose%
\pgfusepath{stroke,fill}%
\end{pgfscope}%
\begin{pgfscope}%
\pgfpathrectangle{\pgfqpoint{0.640323in}{0.527436in}}{\pgfqpoint{9.687500in}{3.850000in}}%
\pgfusepath{clip}%
\pgfsetbuttcap%
\pgfsetroundjoin%
\definecolor{currentfill}{rgb}{0.980392,0.164706,0.333333}%
\pgfsetfillcolor{currentfill}%
\pgfsetfillopacity{0.500000}%
\pgfsetlinewidth{1.003750pt}%
\definecolor{currentstroke}{rgb}{0.980392,0.164706,0.333333}%
\pgfsetstrokecolor{currentstroke}%
\pgfsetstrokeopacity{0.500000}%
\pgfsetdash{{3.700000pt}{1.600000pt}}{0.000000pt}%
\pgfpathmoveto{\pgfqpoint{7.910441in}{3.827131in}}%
\pgfpathcurveto{\pgfqpoint{7.916265in}{3.827131in}}{\pgfqpoint{7.921851in}{3.829445in}}{\pgfqpoint{7.925969in}{3.833563in}}%
\pgfpathcurveto{\pgfqpoint{7.930087in}{3.837682in}}{\pgfqpoint{7.932401in}{3.843268in}}{\pgfqpoint{7.932401in}{3.849092in}}%
\pgfpathcurveto{\pgfqpoint{7.932401in}{3.854916in}}{\pgfqpoint{7.930087in}{3.860502in}}{\pgfqpoint{7.925969in}{3.864620in}}%
\pgfpathcurveto{\pgfqpoint{7.921851in}{3.868738in}}{\pgfqpoint{7.916265in}{3.871052in}}{\pgfqpoint{7.910441in}{3.871052in}}%
\pgfpathcurveto{\pgfqpoint{7.904617in}{3.871052in}}{\pgfqpoint{7.899031in}{3.868738in}}{\pgfqpoint{7.894913in}{3.864620in}}%
\pgfpathcurveto{\pgfqpoint{7.890794in}{3.860502in}}{\pgfqpoint{7.888481in}{3.854916in}}{\pgfqpoint{7.888481in}{3.849092in}}%
\pgfpathcurveto{\pgfqpoint{7.888481in}{3.843268in}}{\pgfqpoint{7.890794in}{3.837682in}}{\pgfqpoint{7.894913in}{3.833563in}}%
\pgfpathcurveto{\pgfqpoint{7.899031in}{3.829445in}}{\pgfqpoint{7.904617in}{3.827131in}}{\pgfqpoint{7.910441in}{3.827131in}}%
\pgfpathlineto{\pgfqpoint{7.910441in}{3.827131in}}%
\pgfpathclose%
\pgfusepath{stroke,fill}%
\end{pgfscope}%
\begin{pgfscope}%
\pgfpathrectangle{\pgfqpoint{0.640323in}{0.527436in}}{\pgfqpoint{9.687500in}{3.850000in}}%
\pgfusepath{clip}%
\pgfsetbuttcap%
\pgfsetroundjoin%
\definecolor{currentfill}{rgb}{0.980392,0.164706,0.333333}%
\pgfsetfillcolor{currentfill}%
\pgfsetfillopacity{0.500000}%
\pgfsetlinewidth{1.003750pt}%
\definecolor{currentstroke}{rgb}{0.980392,0.164706,0.333333}%
\pgfsetstrokecolor{currentstroke}%
\pgfsetstrokeopacity{0.500000}%
\pgfsetdash{{3.700000pt}{1.600000pt}}{0.000000pt}%
\pgfpathmoveto{\pgfqpoint{8.090172in}{3.845506in}}%
\pgfpathcurveto{\pgfqpoint{8.095996in}{3.845506in}}{\pgfqpoint{8.101582in}{3.847820in}}{\pgfqpoint{8.105700in}{3.851938in}}%
\pgfpathcurveto{\pgfqpoint{8.109818in}{3.856056in}}{\pgfqpoint{8.112132in}{3.861642in}}{\pgfqpoint{8.112132in}{3.867466in}}%
\pgfpathcurveto{\pgfqpoint{8.112132in}{3.873290in}}{\pgfqpoint{8.109818in}{3.878876in}}{\pgfqpoint{8.105700in}{3.882994in}}%
\pgfpathcurveto{\pgfqpoint{8.101582in}{3.887112in}}{\pgfqpoint{8.095996in}{3.889426in}}{\pgfqpoint{8.090172in}{3.889426in}}%
\pgfpathcurveto{\pgfqpoint{8.084348in}{3.889426in}}{\pgfqpoint{8.078762in}{3.887112in}}{\pgfqpoint{8.074644in}{3.882994in}}%
\pgfpathcurveto{\pgfqpoint{8.070525in}{3.878876in}}{\pgfqpoint{8.068211in}{3.873290in}}{\pgfqpoint{8.068211in}{3.867466in}}%
\pgfpathcurveto{\pgfqpoint{8.068211in}{3.861642in}}{\pgfqpoint{8.070525in}{3.856056in}}{\pgfqpoint{8.074644in}{3.851938in}}%
\pgfpathcurveto{\pgfqpoint{8.078762in}{3.847820in}}{\pgfqpoint{8.084348in}{3.845506in}}{\pgfqpoint{8.090172in}{3.845506in}}%
\pgfpathlineto{\pgfqpoint{8.090172in}{3.845506in}}%
\pgfpathclose%
\pgfusepath{stroke,fill}%
\end{pgfscope}%
\begin{pgfscope}%
\pgfpathrectangle{\pgfqpoint{0.640323in}{0.527436in}}{\pgfqpoint{9.687500in}{3.850000in}}%
\pgfusepath{clip}%
\pgfsetbuttcap%
\pgfsetroundjoin%
\definecolor{currentfill}{rgb}{0.980392,0.164706,0.333333}%
\pgfsetfillcolor{currentfill}%
\pgfsetfillopacity{0.500000}%
\pgfsetlinewidth{1.003750pt}%
\definecolor{currentstroke}{rgb}{0.980392,0.164706,0.333333}%
\pgfsetstrokecolor{currentstroke}%
\pgfsetstrokeopacity{0.500000}%
\pgfsetdash{{3.700000pt}{1.600000pt}}{0.000000pt}%
\pgfpathmoveto{\pgfqpoint{8.269903in}{3.866377in}}%
\pgfpathcurveto{\pgfqpoint{8.275727in}{3.866377in}}{\pgfqpoint{8.281313in}{3.868690in}}{\pgfqpoint{8.285431in}{3.872809in}}%
\pgfpathcurveto{\pgfqpoint{8.289549in}{3.876927in}}{\pgfqpoint{8.291863in}{3.882513in}}{\pgfqpoint{8.291863in}{3.888337in}}%
\pgfpathcurveto{\pgfqpoint{8.291863in}{3.894161in}}{\pgfqpoint{8.289549in}{3.899747in}}{\pgfqpoint{8.285431in}{3.903865in}}%
\pgfpathcurveto{\pgfqpoint{8.281313in}{3.907983in}}{\pgfqpoint{8.275727in}{3.910297in}}{\pgfqpoint{8.269903in}{3.910297in}}%
\pgfpathcurveto{\pgfqpoint{8.264079in}{3.910297in}}{\pgfqpoint{8.258493in}{3.907983in}}{\pgfqpoint{8.254374in}{3.903865in}}%
\pgfpathcurveto{\pgfqpoint{8.250256in}{3.899747in}}{\pgfqpoint{8.247942in}{3.894161in}}{\pgfqpoint{8.247942in}{3.888337in}}%
\pgfpathcurveto{\pgfqpoint{8.247942in}{3.882513in}}{\pgfqpoint{8.250256in}{3.876927in}}{\pgfqpoint{8.254374in}{3.872809in}}%
\pgfpathcurveto{\pgfqpoint{8.258493in}{3.868690in}}{\pgfqpoint{8.264079in}{3.866377in}}{\pgfqpoint{8.269903in}{3.866377in}}%
\pgfpathlineto{\pgfqpoint{8.269903in}{3.866377in}}%
\pgfpathclose%
\pgfusepath{stroke,fill}%
\end{pgfscope}%
\begin{pgfscope}%
\pgfpathrectangle{\pgfqpoint{0.640323in}{0.527436in}}{\pgfqpoint{9.687500in}{3.850000in}}%
\pgfusepath{clip}%
\pgfsetbuttcap%
\pgfsetroundjoin%
\definecolor{currentfill}{rgb}{0.980392,0.164706,0.333333}%
\pgfsetfillcolor{currentfill}%
\pgfsetfillopacity{0.500000}%
\pgfsetlinewidth{1.003750pt}%
\definecolor{currentstroke}{rgb}{0.980392,0.164706,0.333333}%
\pgfsetstrokecolor{currentstroke}%
\pgfsetstrokeopacity{0.500000}%
\pgfsetdash{{3.700000pt}{1.600000pt}}{0.000000pt}%
\pgfpathmoveto{\pgfqpoint{8.449634in}{3.884776in}}%
\pgfpathcurveto{\pgfqpoint{8.455458in}{3.884776in}}{\pgfqpoint{8.461044in}{3.887090in}}{\pgfqpoint{8.465162in}{3.891208in}}%
\pgfpathcurveto{\pgfqpoint{8.469280in}{3.895326in}}{\pgfqpoint{8.471594in}{3.900912in}}{\pgfqpoint{8.471594in}{3.906736in}}%
\pgfpathcurveto{\pgfqpoint{8.471594in}{3.912560in}}{\pgfqpoint{8.469280in}{3.918146in}}{\pgfqpoint{8.465162in}{3.922264in}}%
\pgfpathcurveto{\pgfqpoint{8.461044in}{3.926382in}}{\pgfqpoint{8.455458in}{3.928696in}}{\pgfqpoint{8.449634in}{3.928696in}}%
\pgfpathcurveto{\pgfqpoint{8.443810in}{3.928696in}}{\pgfqpoint{8.438224in}{3.926382in}}{\pgfqpoint{8.434105in}{3.922264in}}%
\pgfpathcurveto{\pgfqpoint{8.429987in}{3.918146in}}{\pgfqpoint{8.427673in}{3.912560in}}{\pgfqpoint{8.427673in}{3.906736in}}%
\pgfpathcurveto{\pgfqpoint{8.427673in}{3.900912in}}{\pgfqpoint{8.429987in}{3.895326in}}{\pgfqpoint{8.434105in}{3.891208in}}%
\pgfpathcurveto{\pgfqpoint{8.438224in}{3.887090in}}{\pgfqpoint{8.443810in}{3.884776in}}{\pgfqpoint{8.449634in}{3.884776in}}%
\pgfpathlineto{\pgfqpoint{8.449634in}{3.884776in}}%
\pgfpathclose%
\pgfusepath{stroke,fill}%
\end{pgfscope}%
\begin{pgfscope}%
\pgfpathrectangle{\pgfqpoint{0.640323in}{0.527436in}}{\pgfqpoint{9.687500in}{3.850000in}}%
\pgfusepath{clip}%
\pgfsetbuttcap%
\pgfsetroundjoin%
\definecolor{currentfill}{rgb}{0.980392,0.164706,0.333333}%
\pgfsetfillcolor{currentfill}%
\pgfsetfillopacity{0.500000}%
\pgfsetlinewidth{1.003750pt}%
\definecolor{currentstroke}{rgb}{0.980392,0.164706,0.333333}%
\pgfsetstrokecolor{currentstroke}%
\pgfsetstrokeopacity{0.500000}%
\pgfsetdash{{3.700000pt}{1.600000pt}}{0.000000pt}%
\pgfpathmoveto{\pgfqpoint{8.629365in}{3.903784in}}%
\pgfpathcurveto{\pgfqpoint{8.635189in}{3.903784in}}{\pgfqpoint{8.640775in}{3.906098in}}{\pgfqpoint{8.644893in}{3.910216in}}%
\pgfpathcurveto{\pgfqpoint{8.649011in}{3.914334in}}{\pgfqpoint{8.651325in}{3.919920in}}{\pgfqpoint{8.651325in}{3.925744in}}%
\pgfpathcurveto{\pgfqpoint{8.651325in}{3.931568in}}{\pgfqpoint{8.649011in}{3.937154in}}{\pgfqpoint{8.644893in}{3.941272in}}%
\pgfpathcurveto{\pgfqpoint{8.640775in}{3.945390in}}{\pgfqpoint{8.635189in}{3.947704in}}{\pgfqpoint{8.629365in}{3.947704in}}%
\pgfpathcurveto{\pgfqpoint{8.623541in}{3.947704in}}{\pgfqpoint{8.617955in}{3.945390in}}{\pgfqpoint{8.613836in}{3.941272in}}%
\pgfpathcurveto{\pgfqpoint{8.609718in}{3.937154in}}{\pgfqpoint{8.607404in}{3.931568in}}{\pgfqpoint{8.607404in}{3.925744in}}%
\pgfpathcurveto{\pgfqpoint{8.607404in}{3.919920in}}{\pgfqpoint{8.609718in}{3.914334in}}{\pgfqpoint{8.613836in}{3.910216in}}%
\pgfpathcurveto{\pgfqpoint{8.617955in}{3.906098in}}{\pgfqpoint{8.623541in}{3.903784in}}{\pgfqpoint{8.629365in}{3.903784in}}%
\pgfpathlineto{\pgfqpoint{8.629365in}{3.903784in}}%
\pgfpathclose%
\pgfusepath{stroke,fill}%
\end{pgfscope}%
\begin{pgfscope}%
\pgfpathrectangle{\pgfqpoint{0.640323in}{0.527436in}}{\pgfqpoint{9.687500in}{3.850000in}}%
\pgfusepath{clip}%
\pgfsetbuttcap%
\pgfsetroundjoin%
\definecolor{currentfill}{rgb}{0.980392,0.164706,0.333333}%
\pgfsetfillcolor{currentfill}%
\pgfsetfillopacity{0.500000}%
\pgfsetlinewidth{1.003750pt}%
\definecolor{currentstroke}{rgb}{0.980392,0.164706,0.333333}%
\pgfsetstrokecolor{currentstroke}%
\pgfsetstrokeopacity{0.500000}%
\pgfsetdash{{3.700000pt}{1.600000pt}}{0.000000pt}%
\pgfpathmoveto{\pgfqpoint{8.809096in}{3.919271in}}%
\pgfpathcurveto{\pgfqpoint{8.814920in}{3.919271in}}{\pgfqpoint{8.820506in}{3.921584in}}{\pgfqpoint{8.824624in}{3.925703in}}%
\pgfpathcurveto{\pgfqpoint{8.828742in}{3.929821in}}{\pgfqpoint{8.831056in}{3.935407in}}{\pgfqpoint{8.831056in}{3.941231in}}%
\pgfpathcurveto{\pgfqpoint{8.831056in}{3.947055in}}{\pgfqpoint{8.828742in}{3.952641in}}{\pgfqpoint{8.824624in}{3.956759in}}%
\pgfpathcurveto{\pgfqpoint{8.820506in}{3.960877in}}{\pgfqpoint{8.814920in}{3.963191in}}{\pgfqpoint{8.809096in}{3.963191in}}%
\pgfpathcurveto{\pgfqpoint{8.803272in}{3.963191in}}{\pgfqpoint{8.797686in}{3.960877in}}{\pgfqpoint{8.793567in}{3.956759in}}%
\pgfpathcurveto{\pgfqpoint{8.789449in}{3.952641in}}{\pgfqpoint{8.787135in}{3.947055in}}{\pgfqpoint{8.787135in}{3.941231in}}%
\pgfpathcurveto{\pgfqpoint{8.787135in}{3.935407in}}{\pgfqpoint{8.789449in}{3.929821in}}{\pgfqpoint{8.793567in}{3.925703in}}%
\pgfpathcurveto{\pgfqpoint{8.797686in}{3.921584in}}{\pgfqpoint{8.803272in}{3.919271in}}{\pgfqpoint{8.809096in}{3.919271in}}%
\pgfpathlineto{\pgfqpoint{8.809096in}{3.919271in}}%
\pgfpathclose%
\pgfusepath{stroke,fill}%
\end{pgfscope}%
\begin{pgfscope}%
\pgfpathrectangle{\pgfqpoint{0.640323in}{0.527436in}}{\pgfqpoint{9.687500in}{3.850000in}}%
\pgfusepath{clip}%
\pgfsetbuttcap%
\pgfsetroundjoin%
\definecolor{currentfill}{rgb}{0.980392,0.164706,0.333333}%
\pgfsetfillcolor{currentfill}%
\pgfsetfillopacity{0.500000}%
\pgfsetlinewidth{1.003750pt}%
\definecolor{currentstroke}{rgb}{0.980392,0.164706,0.333333}%
\pgfsetstrokecolor{currentstroke}%
\pgfsetstrokeopacity{0.500000}%
\pgfsetdash{{3.700000pt}{1.600000pt}}{0.000000pt}%
\pgfpathmoveto{\pgfqpoint{8.988827in}{3.934807in}}%
\pgfpathcurveto{\pgfqpoint{8.994651in}{3.934807in}}{\pgfqpoint{9.000237in}{3.937121in}}{\pgfqpoint{9.004355in}{3.941239in}}%
\pgfpathcurveto{\pgfqpoint{9.008473in}{3.945357in}}{\pgfqpoint{9.010787in}{3.950944in}}{\pgfqpoint{9.010787in}{3.956767in}}%
\pgfpathcurveto{\pgfqpoint{9.010787in}{3.962591in}}{\pgfqpoint{9.008473in}{3.968178in}}{\pgfqpoint{9.004355in}{3.972296in}}%
\pgfpathcurveto{\pgfqpoint{9.000237in}{3.976414in}}{\pgfqpoint{8.994651in}{3.978728in}}{\pgfqpoint{8.988827in}{3.978728in}}%
\pgfpathcurveto{\pgfqpoint{8.983003in}{3.978728in}}{\pgfqpoint{8.977417in}{3.976414in}}{\pgfqpoint{8.973298in}{3.972296in}}%
\pgfpathcurveto{\pgfqpoint{8.969180in}{3.968178in}}{\pgfqpoint{8.966866in}{3.962591in}}{\pgfqpoint{8.966866in}{3.956767in}}%
\pgfpathcurveto{\pgfqpoint{8.966866in}{3.950944in}}{\pgfqpoint{8.969180in}{3.945357in}}{\pgfqpoint{8.973298in}{3.941239in}}%
\pgfpathcurveto{\pgfqpoint{8.977417in}{3.937121in}}{\pgfqpoint{8.983003in}{3.934807in}}{\pgfqpoint{8.988827in}{3.934807in}}%
\pgfpathlineto{\pgfqpoint{8.988827in}{3.934807in}}%
\pgfpathclose%
\pgfusepath{stroke,fill}%
\end{pgfscope}%
\begin{pgfscope}%
\pgfpathrectangle{\pgfqpoint{0.640323in}{0.527436in}}{\pgfqpoint{9.687500in}{3.850000in}}%
\pgfusepath{clip}%
\pgfsetbuttcap%
\pgfsetroundjoin%
\definecolor{currentfill}{rgb}{0.980392,0.164706,0.333333}%
\pgfsetfillcolor{currentfill}%
\pgfsetfillopacity{0.500000}%
\pgfsetlinewidth{1.003750pt}%
\definecolor{currentstroke}{rgb}{0.980392,0.164706,0.333333}%
\pgfsetstrokecolor{currentstroke}%
\pgfsetstrokeopacity{0.500000}%
\pgfsetdash{{3.700000pt}{1.600000pt}}{0.000000pt}%
\pgfpathmoveto{\pgfqpoint{9.168558in}{3.957882in}}%
\pgfpathcurveto{\pgfqpoint{9.174382in}{3.957882in}}{\pgfqpoint{9.179968in}{3.960196in}}{\pgfqpoint{9.184086in}{3.964314in}}%
\pgfpathcurveto{\pgfqpoint{9.188204in}{3.968432in}}{\pgfqpoint{9.190518in}{3.974019in}}{\pgfqpoint{9.190518in}{3.979843in}}%
\pgfpathcurveto{\pgfqpoint{9.190518in}{3.985667in}}{\pgfqpoint{9.188204in}{3.991253in}}{\pgfqpoint{9.184086in}{3.995371in}}%
\pgfpathcurveto{\pgfqpoint{9.179968in}{3.999489in}}{\pgfqpoint{9.174382in}{4.001803in}}{\pgfqpoint{9.168558in}{4.001803in}}%
\pgfpathcurveto{\pgfqpoint{9.162734in}{4.001803in}}{\pgfqpoint{9.157148in}{3.999489in}}{\pgfqpoint{9.153029in}{3.995371in}}%
\pgfpathcurveto{\pgfqpoint{9.148911in}{3.991253in}}{\pgfqpoint{9.146597in}{3.985667in}}{\pgfqpoint{9.146597in}{3.979843in}}%
\pgfpathcurveto{\pgfqpoint{9.146597in}{3.974019in}}{\pgfqpoint{9.148911in}{3.968432in}}{\pgfqpoint{9.153029in}{3.964314in}}%
\pgfpathcurveto{\pgfqpoint{9.157148in}{3.960196in}}{\pgfqpoint{9.162734in}{3.957882in}}{\pgfqpoint{9.168558in}{3.957882in}}%
\pgfpathlineto{\pgfqpoint{9.168558in}{3.957882in}}%
\pgfpathclose%
\pgfusepath{stroke,fill}%
\end{pgfscope}%
\begin{pgfscope}%
\pgfpathrectangle{\pgfqpoint{0.640323in}{0.527436in}}{\pgfqpoint{9.687500in}{3.850000in}}%
\pgfusepath{clip}%
\pgfsetbuttcap%
\pgfsetroundjoin%
\definecolor{currentfill}{rgb}{0.980392,0.164706,0.333333}%
\pgfsetfillcolor{currentfill}%
\pgfsetfillopacity{0.500000}%
\pgfsetlinewidth{1.003750pt}%
\definecolor{currentstroke}{rgb}{0.980392,0.164706,0.333333}%
\pgfsetstrokecolor{currentstroke}%
\pgfsetstrokeopacity{0.500000}%
\pgfsetdash{{3.700000pt}{1.600000pt}}{0.000000pt}%
\pgfpathmoveto{\pgfqpoint{9.348289in}{3.961931in}}%
\pgfpathcurveto{\pgfqpoint{9.354113in}{3.961931in}}{\pgfqpoint{9.359699in}{3.964245in}}{\pgfqpoint{9.363817in}{3.968363in}}%
\pgfpathcurveto{\pgfqpoint{9.367935in}{3.972481in}}{\pgfqpoint{9.370249in}{3.978067in}}{\pgfqpoint{9.370249in}{3.983891in}}%
\pgfpathcurveto{\pgfqpoint{9.370249in}{3.989715in}}{\pgfqpoint{9.367935in}{3.995301in}}{\pgfqpoint{9.363817in}{3.999420in}}%
\pgfpathcurveto{\pgfqpoint{9.359699in}{4.003538in}}{\pgfqpoint{9.354113in}{4.005852in}}{\pgfqpoint{9.348289in}{4.005852in}}%
\pgfpathcurveto{\pgfqpoint{9.342465in}{4.005852in}}{\pgfqpoint{9.336879in}{4.003538in}}{\pgfqpoint{9.332760in}{3.999420in}}%
\pgfpathcurveto{\pgfqpoint{9.328642in}{3.995301in}}{\pgfqpoint{9.326328in}{3.989715in}}{\pgfqpoint{9.326328in}{3.983891in}}%
\pgfpathcurveto{\pgfqpoint{9.326328in}{3.978067in}}{\pgfqpoint{9.328642in}{3.972481in}}{\pgfqpoint{9.332760in}{3.968363in}}%
\pgfpathcurveto{\pgfqpoint{9.336879in}{3.964245in}}{\pgfqpoint{9.342465in}{3.961931in}}{\pgfqpoint{9.348289in}{3.961931in}}%
\pgfpathlineto{\pgfqpoint{9.348289in}{3.961931in}}%
\pgfpathclose%
\pgfusepath{stroke,fill}%
\end{pgfscope}%
\begin{pgfscope}%
\pgfpathrectangle{\pgfqpoint{0.640323in}{0.527436in}}{\pgfqpoint{9.687500in}{3.850000in}}%
\pgfusepath{clip}%
\pgfsetbuttcap%
\pgfsetroundjoin%
\definecolor{currentfill}{rgb}{0.980392,0.164706,0.333333}%
\pgfsetfillcolor{currentfill}%
\pgfsetfillopacity{0.500000}%
\pgfsetlinewidth{1.003750pt}%
\definecolor{currentstroke}{rgb}{0.980392,0.164706,0.333333}%
\pgfsetstrokecolor{currentstroke}%
\pgfsetstrokeopacity{0.500000}%
\pgfsetdash{{3.700000pt}{1.600000pt}}{0.000000pt}%
\pgfpathmoveto{\pgfqpoint{9.528020in}{3.987409in}}%
\pgfpathcurveto{\pgfqpoint{9.533844in}{3.987409in}}{\pgfqpoint{9.539430in}{3.989723in}}{\pgfqpoint{9.543548in}{3.993841in}}%
\pgfpathcurveto{\pgfqpoint{9.547666in}{3.997960in}}{\pgfqpoint{9.549980in}{4.003546in}}{\pgfqpoint{9.549980in}{4.009370in}}%
\pgfpathcurveto{\pgfqpoint{9.549980in}{4.015194in}}{\pgfqpoint{9.547666in}{4.020780in}}{\pgfqpoint{9.543548in}{4.024898in}}%
\pgfpathcurveto{\pgfqpoint{9.539430in}{4.029016in}}{\pgfqpoint{9.533844in}{4.031330in}}{\pgfqpoint{9.528020in}{4.031330in}}%
\pgfpathcurveto{\pgfqpoint{9.522196in}{4.031330in}}{\pgfqpoint{9.516610in}{4.029016in}}{\pgfqpoint{9.512491in}{4.024898in}}%
\pgfpathcurveto{\pgfqpoint{9.508373in}{4.020780in}}{\pgfqpoint{9.506059in}{4.015194in}}{\pgfqpoint{9.506059in}{4.009370in}}%
\pgfpathcurveto{\pgfqpoint{9.506059in}{4.003546in}}{\pgfqpoint{9.508373in}{3.997960in}}{\pgfqpoint{9.512491in}{3.993841in}}%
\pgfpathcurveto{\pgfqpoint{9.516610in}{3.989723in}}{\pgfqpoint{9.522196in}{3.987409in}}{\pgfqpoint{9.528020in}{3.987409in}}%
\pgfpathlineto{\pgfqpoint{9.528020in}{3.987409in}}%
\pgfpathclose%
\pgfusepath{stroke,fill}%
\end{pgfscope}%
\begin{pgfscope}%
\pgfpathrectangle{\pgfqpoint{0.640323in}{0.527436in}}{\pgfqpoint{9.687500in}{3.850000in}}%
\pgfusepath{clip}%
\pgfsetbuttcap%
\pgfsetroundjoin%
\definecolor{currentfill}{rgb}{0.980392,0.164706,0.333333}%
\pgfsetfillcolor{currentfill}%
\pgfsetfillopacity{0.500000}%
\pgfsetlinewidth{1.003750pt}%
\definecolor{currentstroke}{rgb}{0.980392,0.164706,0.333333}%
\pgfsetstrokecolor{currentstroke}%
\pgfsetstrokeopacity{0.500000}%
\pgfsetdash{{3.700000pt}{1.600000pt}}{0.000000pt}%
\pgfpathmoveto{\pgfqpoint{9.707751in}{3.994886in}}%
\pgfpathcurveto{\pgfqpoint{9.713575in}{3.994886in}}{\pgfqpoint{9.719161in}{3.997200in}}{\pgfqpoint{9.723279in}{4.001318in}}%
\pgfpathcurveto{\pgfqpoint{9.727397in}{4.005436in}}{\pgfqpoint{9.729711in}{4.011022in}}{\pgfqpoint{9.729711in}{4.016846in}}%
\pgfpathcurveto{\pgfqpoint{9.729711in}{4.022670in}}{\pgfqpoint{9.727397in}{4.028256in}}{\pgfqpoint{9.723279in}{4.032374in}}%
\pgfpathcurveto{\pgfqpoint{9.719161in}{4.036492in}}{\pgfqpoint{9.713575in}{4.038806in}}{\pgfqpoint{9.707751in}{4.038806in}}%
\pgfpathcurveto{\pgfqpoint{9.701927in}{4.038806in}}{\pgfqpoint{9.696340in}{4.036492in}}{\pgfqpoint{9.692222in}{4.032374in}}%
\pgfpathcurveto{\pgfqpoint{9.688104in}{4.028256in}}{\pgfqpoint{9.685790in}{4.022670in}}{\pgfqpoint{9.685790in}{4.016846in}}%
\pgfpathcurveto{\pgfqpoint{9.685790in}{4.011022in}}{\pgfqpoint{9.688104in}{4.005436in}}{\pgfqpoint{9.692222in}{4.001318in}}%
\pgfpathcurveto{\pgfqpoint{9.696340in}{3.997200in}}{\pgfqpoint{9.701927in}{3.994886in}}{\pgfqpoint{9.707751in}{3.994886in}}%
\pgfpathlineto{\pgfqpoint{9.707751in}{3.994886in}}%
\pgfpathclose%
\pgfusepath{stroke,fill}%
\end{pgfscope}%
\begin{pgfscope}%
\pgfpathrectangle{\pgfqpoint{0.640323in}{0.527436in}}{\pgfqpoint{9.687500in}{3.850000in}}%
\pgfusepath{clip}%
\pgfsetbuttcap%
\pgfsetroundjoin%
\definecolor{currentfill}{rgb}{0.980392,0.164706,0.333333}%
\pgfsetfillcolor{currentfill}%
\pgfsetfillopacity{0.500000}%
\pgfsetlinewidth{1.003750pt}%
\definecolor{currentstroke}{rgb}{0.980392,0.164706,0.333333}%
\pgfsetstrokecolor{currentstroke}%
\pgfsetstrokeopacity{0.500000}%
\pgfsetdash{{3.700000pt}{1.600000pt}}{0.000000pt}%
\pgfpathmoveto{\pgfqpoint{9.887482in}{4.015614in}}%
\pgfpathcurveto{\pgfqpoint{9.893306in}{4.015614in}}{\pgfqpoint{9.898892in}{4.017928in}}{\pgfqpoint{9.903010in}{4.022046in}}%
\pgfpathcurveto{\pgfqpoint{9.907128in}{4.026164in}}{\pgfqpoint{9.909442in}{4.031750in}}{\pgfqpoint{9.909442in}{4.037574in}}%
\pgfpathcurveto{\pgfqpoint{9.909442in}{4.043398in}}{\pgfqpoint{9.907128in}{4.048984in}}{\pgfqpoint{9.903010in}{4.053102in}}%
\pgfpathcurveto{\pgfqpoint{9.898892in}{4.057220in}}{\pgfqpoint{9.893306in}{4.059534in}}{\pgfqpoint{9.887482in}{4.059534in}}%
\pgfpathcurveto{\pgfqpoint{9.881658in}{4.059534in}}{\pgfqpoint{9.876071in}{4.057220in}}{\pgfqpoint{9.871953in}{4.053102in}}%
\pgfpathcurveto{\pgfqpoint{9.867835in}{4.048984in}}{\pgfqpoint{9.865521in}{4.043398in}}{\pgfqpoint{9.865521in}{4.037574in}}%
\pgfpathcurveto{\pgfqpoint{9.865521in}{4.031750in}}{\pgfqpoint{9.867835in}{4.026164in}}{\pgfqpoint{9.871953in}{4.022046in}}%
\pgfpathcurveto{\pgfqpoint{9.876071in}{4.017928in}}{\pgfqpoint{9.881658in}{4.015614in}}{\pgfqpoint{9.887482in}{4.015614in}}%
\pgfpathlineto{\pgfqpoint{9.887482in}{4.015614in}}%
\pgfpathclose%
\pgfusepath{stroke,fill}%
\end{pgfscope}%
\begin{pgfscope}%
\pgfpathrectangle{\pgfqpoint{0.640323in}{0.527436in}}{\pgfqpoint{9.687500in}{3.850000in}}%
\pgfusepath{clip}%
\pgfsetbuttcap%
\pgfsetroundjoin%
\definecolor{currentfill}{rgb}{0.239216,0.478431,0.992157}%
\pgfsetfillcolor{currentfill}%
\pgfsetfillopacity{0.500000}%
\pgfsetlinewidth{1.003750pt}%
\definecolor{currentstroke}{rgb}{0.239216,0.478431,0.992157}%
\pgfsetstrokecolor{currentstroke}%
\pgfsetstrokeopacity{0.500000}%
\pgfsetdash{{3.700000pt}{1.600000pt}}{0.000000pt}%
\pgfpathmoveto{\pgfqpoint{1.080663in}{0.637815in}}%
\pgfpathcurveto{\pgfqpoint{1.086487in}{0.637815in}}{\pgfqpoint{1.092074in}{0.640129in}}{\pgfqpoint{1.096192in}{0.644247in}}%
\pgfpathcurveto{\pgfqpoint{1.100310in}{0.648365in}}{\pgfqpoint{1.102624in}{0.653951in}}{\pgfqpoint{1.102624in}{0.659775in}}%
\pgfpathcurveto{\pgfqpoint{1.102624in}{0.665599in}}{\pgfqpoint{1.100310in}{0.671185in}}{\pgfqpoint{1.096192in}{0.675303in}}%
\pgfpathcurveto{\pgfqpoint{1.092074in}{0.679421in}}{\pgfqpoint{1.086487in}{0.681735in}}{\pgfqpoint{1.080663in}{0.681735in}}%
\pgfpathcurveto{\pgfqpoint{1.074839in}{0.681735in}}{\pgfqpoint{1.069253in}{0.679421in}}{\pgfqpoint{1.065135in}{0.675303in}}%
\pgfpathcurveto{\pgfqpoint{1.061017in}{0.671185in}}{\pgfqpoint{1.058703in}{0.665599in}}{\pgfqpoint{1.058703in}{0.659775in}}%
\pgfpathcurveto{\pgfqpoint{1.058703in}{0.653951in}}{\pgfqpoint{1.061017in}{0.648365in}}{\pgfqpoint{1.065135in}{0.644247in}}%
\pgfpathcurveto{\pgfqpoint{1.069253in}{0.640129in}}{\pgfqpoint{1.074839in}{0.637815in}}{\pgfqpoint{1.080663in}{0.637815in}}%
\pgfpathlineto{\pgfqpoint{1.080663in}{0.637815in}}%
\pgfpathclose%
\pgfusepath{stroke,fill}%
\end{pgfscope}%
\begin{pgfscope}%
\pgfpathrectangle{\pgfqpoint{0.640323in}{0.527436in}}{\pgfqpoint{9.687500in}{3.850000in}}%
\pgfusepath{clip}%
\pgfsetbuttcap%
\pgfsetroundjoin%
\definecolor{currentfill}{rgb}{0.239216,0.478431,0.992157}%
\pgfsetfillcolor{currentfill}%
\pgfsetfillopacity{0.500000}%
\pgfsetlinewidth{1.003750pt}%
\definecolor{currentstroke}{rgb}{0.239216,0.478431,0.992157}%
\pgfsetstrokecolor{currentstroke}%
\pgfsetstrokeopacity{0.500000}%
\pgfsetdash{{3.700000pt}{1.600000pt}}{0.000000pt}%
\pgfpathmoveto{\pgfqpoint{1.260394in}{0.638123in}}%
\pgfpathcurveto{\pgfqpoint{1.266218in}{0.638123in}}{\pgfqpoint{1.271805in}{0.640436in}}{\pgfqpoint{1.275923in}{0.644555in}}%
\pgfpathcurveto{\pgfqpoint{1.280041in}{0.648673in}}{\pgfqpoint{1.282355in}{0.654259in}}{\pgfqpoint{1.282355in}{0.660083in}}%
\pgfpathcurveto{\pgfqpoint{1.282355in}{0.665907in}}{\pgfqpoint{1.280041in}{0.671493in}}{\pgfqpoint{1.275923in}{0.675611in}}%
\pgfpathcurveto{\pgfqpoint{1.271805in}{0.679729in}}{\pgfqpoint{1.266218in}{0.682043in}}{\pgfqpoint{1.260394in}{0.682043in}}%
\pgfpathcurveto{\pgfqpoint{1.254570in}{0.682043in}}{\pgfqpoint{1.248984in}{0.679729in}}{\pgfqpoint{1.244866in}{0.675611in}}%
\pgfpathcurveto{\pgfqpoint{1.240748in}{0.671493in}}{\pgfqpoint{1.238434in}{0.665907in}}{\pgfqpoint{1.238434in}{0.660083in}}%
\pgfpathcurveto{\pgfqpoint{1.238434in}{0.654259in}}{\pgfqpoint{1.240748in}{0.648673in}}{\pgfqpoint{1.244866in}{0.644555in}}%
\pgfpathcurveto{\pgfqpoint{1.248984in}{0.640436in}}{\pgfqpoint{1.254570in}{0.638123in}}{\pgfqpoint{1.260394in}{0.638123in}}%
\pgfpathlineto{\pgfqpoint{1.260394in}{0.638123in}}%
\pgfpathclose%
\pgfusepath{stroke,fill}%
\end{pgfscope}%
\begin{pgfscope}%
\pgfpathrectangle{\pgfqpoint{0.640323in}{0.527436in}}{\pgfqpoint{9.687500in}{3.850000in}}%
\pgfusepath{clip}%
\pgfsetbuttcap%
\pgfsetroundjoin%
\definecolor{currentfill}{rgb}{0.239216,0.478431,0.992157}%
\pgfsetfillcolor{currentfill}%
\pgfsetfillopacity{0.500000}%
\pgfsetlinewidth{1.003750pt}%
\definecolor{currentstroke}{rgb}{0.239216,0.478431,0.992157}%
\pgfsetstrokecolor{currentstroke}%
\pgfsetstrokeopacity{0.500000}%
\pgfsetdash{{3.700000pt}{1.600000pt}}{0.000000pt}%
\pgfpathmoveto{\pgfqpoint{1.440125in}{0.640518in}}%
\pgfpathcurveto{\pgfqpoint{1.445949in}{0.640518in}}{\pgfqpoint{1.451535in}{0.642832in}}{\pgfqpoint{1.455654in}{0.646950in}}%
\pgfpathcurveto{\pgfqpoint{1.459772in}{0.651068in}}{\pgfqpoint{1.462086in}{0.656654in}}{\pgfqpoint{1.462086in}{0.662478in}}%
\pgfpathcurveto{\pgfqpoint{1.462086in}{0.668302in}}{\pgfqpoint{1.459772in}{0.673888in}}{\pgfqpoint{1.455654in}{0.678006in}}%
\pgfpathcurveto{\pgfqpoint{1.451535in}{0.682125in}}{\pgfqpoint{1.445949in}{0.684438in}}{\pgfqpoint{1.440125in}{0.684438in}}%
\pgfpathcurveto{\pgfqpoint{1.434301in}{0.684438in}}{\pgfqpoint{1.428715in}{0.682125in}}{\pgfqpoint{1.424597in}{0.678006in}}%
\pgfpathcurveto{\pgfqpoint{1.420479in}{0.673888in}}{\pgfqpoint{1.418165in}{0.668302in}}{\pgfqpoint{1.418165in}{0.662478in}}%
\pgfpathcurveto{\pgfqpoint{1.418165in}{0.656654in}}{\pgfqpoint{1.420479in}{0.651068in}}{\pgfqpoint{1.424597in}{0.646950in}}%
\pgfpathcurveto{\pgfqpoint{1.428715in}{0.642832in}}{\pgfqpoint{1.434301in}{0.640518in}}{\pgfqpoint{1.440125in}{0.640518in}}%
\pgfpathlineto{\pgfqpoint{1.440125in}{0.640518in}}%
\pgfpathclose%
\pgfusepath{stroke,fill}%
\end{pgfscope}%
\begin{pgfscope}%
\pgfpathrectangle{\pgfqpoint{0.640323in}{0.527436in}}{\pgfqpoint{9.687500in}{3.850000in}}%
\pgfusepath{clip}%
\pgfsetbuttcap%
\pgfsetroundjoin%
\definecolor{currentfill}{rgb}{0.239216,0.478431,0.992157}%
\pgfsetfillcolor{currentfill}%
\pgfsetfillopacity{0.500000}%
\pgfsetlinewidth{1.003750pt}%
\definecolor{currentstroke}{rgb}{0.239216,0.478431,0.992157}%
\pgfsetstrokecolor{currentstroke}%
\pgfsetstrokeopacity{0.500000}%
\pgfsetdash{{3.700000pt}{1.600000pt}}{0.000000pt}%
\pgfpathmoveto{\pgfqpoint{1.619856in}{0.642237in}}%
\pgfpathcurveto{\pgfqpoint{1.625680in}{0.642237in}}{\pgfqpoint{1.631266in}{0.644551in}}{\pgfqpoint{1.635385in}{0.648669in}}%
\pgfpathcurveto{\pgfqpoint{1.639503in}{0.652787in}}{\pgfqpoint{1.641817in}{0.658373in}}{\pgfqpoint{1.641817in}{0.664197in}}%
\pgfpathcurveto{\pgfqpoint{1.641817in}{0.670021in}}{\pgfqpoint{1.639503in}{0.675607in}}{\pgfqpoint{1.635385in}{0.679726in}}%
\pgfpathcurveto{\pgfqpoint{1.631266in}{0.683844in}}{\pgfqpoint{1.625680in}{0.686158in}}{\pgfqpoint{1.619856in}{0.686158in}}%
\pgfpathcurveto{\pgfqpoint{1.614032in}{0.686158in}}{\pgfqpoint{1.608446in}{0.683844in}}{\pgfqpoint{1.604328in}{0.679726in}}%
\pgfpathcurveto{\pgfqpoint{1.600210in}{0.675607in}}{\pgfqpoint{1.597896in}{0.670021in}}{\pgfqpoint{1.597896in}{0.664197in}}%
\pgfpathcurveto{\pgfqpoint{1.597896in}{0.658373in}}{\pgfqpoint{1.600210in}{0.652787in}}{\pgfqpoint{1.604328in}{0.648669in}}%
\pgfpathcurveto{\pgfqpoint{1.608446in}{0.644551in}}{\pgfqpoint{1.614032in}{0.642237in}}{\pgfqpoint{1.619856in}{0.642237in}}%
\pgfpathlineto{\pgfqpoint{1.619856in}{0.642237in}}%
\pgfpathclose%
\pgfusepath{stroke,fill}%
\end{pgfscope}%
\begin{pgfscope}%
\pgfpathrectangle{\pgfqpoint{0.640323in}{0.527436in}}{\pgfqpoint{9.687500in}{3.850000in}}%
\pgfusepath{clip}%
\pgfsetbuttcap%
\pgfsetroundjoin%
\definecolor{currentfill}{rgb}{0.239216,0.478431,0.992157}%
\pgfsetfillcolor{currentfill}%
\pgfsetfillopacity{0.500000}%
\pgfsetlinewidth{1.003750pt}%
\definecolor{currentstroke}{rgb}{0.239216,0.478431,0.992157}%
\pgfsetstrokecolor{currentstroke}%
\pgfsetstrokeopacity{0.500000}%
\pgfsetdash{{3.700000pt}{1.600000pt}}{0.000000pt}%
\pgfpathmoveto{\pgfqpoint{1.799587in}{0.644875in}}%
\pgfpathcurveto{\pgfqpoint{1.805411in}{0.644875in}}{\pgfqpoint{1.810997in}{0.647189in}}{\pgfqpoint{1.815116in}{0.651307in}}%
\pgfpathcurveto{\pgfqpoint{1.819234in}{0.655425in}}{\pgfqpoint{1.821548in}{0.661011in}}{\pgfqpoint{1.821548in}{0.666835in}}%
\pgfpathcurveto{\pgfqpoint{1.821548in}{0.672659in}}{\pgfqpoint{1.819234in}{0.678245in}}{\pgfqpoint{1.815116in}{0.682363in}}%
\pgfpathcurveto{\pgfqpoint{1.810997in}{0.686482in}}{\pgfqpoint{1.805411in}{0.688795in}}{\pgfqpoint{1.799587in}{0.688795in}}%
\pgfpathcurveto{\pgfqpoint{1.793763in}{0.688795in}}{\pgfqpoint{1.788177in}{0.686482in}}{\pgfqpoint{1.784059in}{0.682363in}}%
\pgfpathcurveto{\pgfqpoint{1.779941in}{0.678245in}}{\pgfqpoint{1.777627in}{0.672659in}}{\pgfqpoint{1.777627in}{0.666835in}}%
\pgfpathcurveto{\pgfqpoint{1.777627in}{0.661011in}}{\pgfqpoint{1.779941in}{0.655425in}}{\pgfqpoint{1.784059in}{0.651307in}}%
\pgfpathcurveto{\pgfqpoint{1.788177in}{0.647189in}}{\pgfqpoint{1.793763in}{0.644875in}}{\pgfqpoint{1.799587in}{0.644875in}}%
\pgfpathlineto{\pgfqpoint{1.799587in}{0.644875in}}%
\pgfpathclose%
\pgfusepath{stroke,fill}%
\end{pgfscope}%
\begin{pgfscope}%
\pgfpathrectangle{\pgfqpoint{0.640323in}{0.527436in}}{\pgfqpoint{9.687500in}{3.850000in}}%
\pgfusepath{clip}%
\pgfsetbuttcap%
\pgfsetroundjoin%
\definecolor{currentfill}{rgb}{0.239216,0.478431,0.992157}%
\pgfsetfillcolor{currentfill}%
\pgfsetfillopacity{0.500000}%
\pgfsetlinewidth{1.003750pt}%
\definecolor{currentstroke}{rgb}{0.239216,0.478431,0.992157}%
\pgfsetstrokecolor{currentstroke}%
\pgfsetstrokeopacity{0.500000}%
\pgfsetdash{{3.700000pt}{1.600000pt}}{0.000000pt}%
\pgfpathmoveto{\pgfqpoint{1.979318in}{0.651476in}}%
\pgfpathcurveto{\pgfqpoint{1.985142in}{0.651476in}}{\pgfqpoint{1.990728in}{0.653790in}}{\pgfqpoint{1.994847in}{0.657908in}}%
\pgfpathcurveto{\pgfqpoint{1.998965in}{0.662026in}}{\pgfqpoint{2.001279in}{0.667612in}}{\pgfqpoint{2.001279in}{0.673436in}}%
\pgfpathcurveto{\pgfqpoint{2.001279in}{0.679260in}}{\pgfqpoint{1.998965in}{0.684846in}}{\pgfqpoint{1.994847in}{0.688964in}}%
\pgfpathcurveto{\pgfqpoint{1.990728in}{0.693082in}}{\pgfqpoint{1.985142in}{0.695396in}}{\pgfqpoint{1.979318in}{0.695396in}}%
\pgfpathcurveto{\pgfqpoint{1.973494in}{0.695396in}}{\pgfqpoint{1.967908in}{0.693082in}}{\pgfqpoint{1.963790in}{0.688964in}}%
\pgfpathcurveto{\pgfqpoint{1.959672in}{0.684846in}}{\pgfqpoint{1.957358in}{0.679260in}}{\pgfqpoint{1.957358in}{0.673436in}}%
\pgfpathcurveto{\pgfqpoint{1.957358in}{0.667612in}}{\pgfqpoint{1.959672in}{0.662026in}}{\pgfqpoint{1.963790in}{0.657908in}}%
\pgfpathcurveto{\pgfqpoint{1.967908in}{0.653790in}}{\pgfqpoint{1.973494in}{0.651476in}}{\pgfqpoint{1.979318in}{0.651476in}}%
\pgfpathlineto{\pgfqpoint{1.979318in}{0.651476in}}%
\pgfpathclose%
\pgfusepath{stroke,fill}%
\end{pgfscope}%
\begin{pgfscope}%
\pgfpathrectangle{\pgfqpoint{0.640323in}{0.527436in}}{\pgfqpoint{9.687500in}{3.850000in}}%
\pgfusepath{clip}%
\pgfsetbuttcap%
\pgfsetroundjoin%
\definecolor{currentfill}{rgb}{0.239216,0.478431,0.992157}%
\pgfsetfillcolor{currentfill}%
\pgfsetfillopacity{0.500000}%
\pgfsetlinewidth{1.003750pt}%
\definecolor{currentstroke}{rgb}{0.239216,0.478431,0.992157}%
\pgfsetstrokecolor{currentstroke}%
\pgfsetstrokeopacity{0.500000}%
\pgfsetdash{{3.700000pt}{1.600000pt}}{0.000000pt}%
\pgfpathmoveto{\pgfqpoint{2.159049in}{0.662170in}}%
\pgfpathcurveto{\pgfqpoint{2.164873in}{0.662170in}}{\pgfqpoint{2.170459in}{0.664484in}}{\pgfqpoint{2.174578in}{0.668602in}}%
\pgfpathcurveto{\pgfqpoint{2.178696in}{0.672720in}}{\pgfqpoint{2.181010in}{0.678306in}}{\pgfqpoint{2.181010in}{0.684130in}}%
\pgfpathcurveto{\pgfqpoint{2.181010in}{0.689954in}}{\pgfqpoint{2.178696in}{0.695540in}}{\pgfqpoint{2.174578in}{0.699658in}}%
\pgfpathcurveto{\pgfqpoint{2.170459in}{0.703777in}}{\pgfqpoint{2.164873in}{0.706090in}}{\pgfqpoint{2.159049in}{0.706090in}}%
\pgfpathcurveto{\pgfqpoint{2.153225in}{0.706090in}}{\pgfqpoint{2.147639in}{0.703777in}}{\pgfqpoint{2.143521in}{0.699658in}}%
\pgfpathcurveto{\pgfqpoint{2.139403in}{0.695540in}}{\pgfqpoint{2.137089in}{0.689954in}}{\pgfqpoint{2.137089in}{0.684130in}}%
\pgfpathcurveto{\pgfqpoint{2.137089in}{0.678306in}}{\pgfqpoint{2.139403in}{0.672720in}}{\pgfqpoint{2.143521in}{0.668602in}}%
\pgfpathcurveto{\pgfqpoint{2.147639in}{0.664484in}}{\pgfqpoint{2.153225in}{0.662170in}}{\pgfqpoint{2.159049in}{0.662170in}}%
\pgfpathlineto{\pgfqpoint{2.159049in}{0.662170in}}%
\pgfpathclose%
\pgfusepath{stroke,fill}%
\end{pgfscope}%
\begin{pgfscope}%
\pgfpathrectangle{\pgfqpoint{0.640323in}{0.527436in}}{\pgfqpoint{9.687500in}{3.850000in}}%
\pgfusepath{clip}%
\pgfsetbuttcap%
\pgfsetroundjoin%
\definecolor{currentfill}{rgb}{0.239216,0.478431,0.992157}%
\pgfsetfillcolor{currentfill}%
\pgfsetfillopacity{0.500000}%
\pgfsetlinewidth{1.003750pt}%
\definecolor{currentstroke}{rgb}{0.239216,0.478431,0.992157}%
\pgfsetstrokecolor{currentstroke}%
\pgfsetstrokeopacity{0.500000}%
\pgfsetdash{{3.700000pt}{1.600000pt}}{0.000000pt}%
\pgfpathmoveto{\pgfqpoint{2.338780in}{0.677615in}}%
\pgfpathcurveto{\pgfqpoint{2.344604in}{0.677615in}}{\pgfqpoint{2.350190in}{0.679929in}}{\pgfqpoint{2.354309in}{0.684047in}}%
\pgfpathcurveto{\pgfqpoint{2.358427in}{0.688165in}}{\pgfqpoint{2.360741in}{0.693752in}}{\pgfqpoint{2.360741in}{0.699575in}}%
\pgfpathcurveto{\pgfqpoint{2.360741in}{0.705399in}}{\pgfqpoint{2.358427in}{0.710986in}}{\pgfqpoint{2.354309in}{0.715104in}}%
\pgfpathcurveto{\pgfqpoint{2.350190in}{0.719222in}}{\pgfqpoint{2.344604in}{0.721536in}}{\pgfqpoint{2.338780in}{0.721536in}}%
\pgfpathcurveto{\pgfqpoint{2.332956in}{0.721536in}}{\pgfqpoint{2.327370in}{0.719222in}}{\pgfqpoint{2.323252in}{0.715104in}}%
\pgfpathcurveto{\pgfqpoint{2.319134in}{0.710986in}}{\pgfqpoint{2.316820in}{0.705399in}}{\pgfqpoint{2.316820in}{0.699575in}}%
\pgfpathcurveto{\pgfqpoint{2.316820in}{0.693752in}}{\pgfqpoint{2.319134in}{0.688165in}}{\pgfqpoint{2.323252in}{0.684047in}}%
\pgfpathcurveto{\pgfqpoint{2.327370in}{0.679929in}}{\pgfqpoint{2.332956in}{0.677615in}}{\pgfqpoint{2.338780in}{0.677615in}}%
\pgfpathlineto{\pgfqpoint{2.338780in}{0.677615in}}%
\pgfpathclose%
\pgfusepath{stroke,fill}%
\end{pgfscope}%
\begin{pgfscope}%
\pgfpathrectangle{\pgfqpoint{0.640323in}{0.527436in}}{\pgfqpoint{9.687500in}{3.850000in}}%
\pgfusepath{clip}%
\pgfsetbuttcap%
\pgfsetroundjoin%
\definecolor{currentfill}{rgb}{0.239216,0.478431,0.992157}%
\pgfsetfillcolor{currentfill}%
\pgfsetfillopacity{0.500000}%
\pgfsetlinewidth{1.003750pt}%
\definecolor{currentstroke}{rgb}{0.239216,0.478431,0.992157}%
\pgfsetstrokecolor{currentstroke}%
\pgfsetstrokeopacity{0.500000}%
\pgfsetdash{{3.700000pt}{1.600000pt}}{0.000000pt}%
\pgfpathmoveto{\pgfqpoint{2.518511in}{0.795312in}}%
\pgfpathcurveto{\pgfqpoint{2.524335in}{0.795312in}}{\pgfqpoint{2.529921in}{0.797626in}}{\pgfqpoint{2.534040in}{0.801744in}}%
\pgfpathcurveto{\pgfqpoint{2.538158in}{0.805862in}}{\pgfqpoint{2.540472in}{0.811449in}}{\pgfqpoint{2.540472in}{0.817273in}}%
\pgfpathcurveto{\pgfqpoint{2.540472in}{0.823096in}}{\pgfqpoint{2.538158in}{0.828683in}}{\pgfqpoint{2.534040in}{0.832801in}}%
\pgfpathcurveto{\pgfqpoint{2.529921in}{0.836919in}}{\pgfqpoint{2.524335in}{0.839233in}}{\pgfqpoint{2.518511in}{0.839233in}}%
\pgfpathcurveto{\pgfqpoint{2.512687in}{0.839233in}}{\pgfqpoint{2.507101in}{0.836919in}}{\pgfqpoint{2.502983in}{0.832801in}}%
\pgfpathcurveto{\pgfqpoint{2.498865in}{0.828683in}}{\pgfqpoint{2.496551in}{0.823096in}}{\pgfqpoint{2.496551in}{0.817273in}}%
\pgfpathcurveto{\pgfqpoint{2.496551in}{0.811449in}}{\pgfqpoint{2.498865in}{0.805862in}}{\pgfqpoint{2.502983in}{0.801744in}}%
\pgfpathcurveto{\pgfqpoint{2.507101in}{0.797626in}}{\pgfqpoint{2.512687in}{0.795312in}}{\pgfqpoint{2.518511in}{0.795312in}}%
\pgfpathlineto{\pgfqpoint{2.518511in}{0.795312in}}%
\pgfpathclose%
\pgfusepath{stroke,fill}%
\end{pgfscope}%
\begin{pgfscope}%
\pgfpathrectangle{\pgfqpoint{0.640323in}{0.527436in}}{\pgfqpoint{9.687500in}{3.850000in}}%
\pgfusepath{clip}%
\pgfsetbuttcap%
\pgfsetroundjoin%
\definecolor{currentfill}{rgb}{0.239216,0.478431,0.992157}%
\pgfsetfillcolor{currentfill}%
\pgfsetfillopacity{0.500000}%
\pgfsetlinewidth{1.003750pt}%
\definecolor{currentstroke}{rgb}{0.239216,0.478431,0.992157}%
\pgfsetstrokecolor{currentstroke}%
\pgfsetstrokeopacity{0.500000}%
\pgfsetdash{{3.700000pt}{1.600000pt}}{0.000000pt}%
\pgfpathmoveto{\pgfqpoint{2.698242in}{1.383015in}}%
\pgfpathcurveto{\pgfqpoint{2.704066in}{1.383015in}}{\pgfqpoint{2.709652in}{1.385329in}}{\pgfqpoint{2.713771in}{1.389447in}}%
\pgfpathcurveto{\pgfqpoint{2.717889in}{1.393565in}}{\pgfqpoint{2.720203in}{1.399151in}}{\pgfqpoint{2.720203in}{1.404975in}}%
\pgfpathcurveto{\pgfqpoint{2.720203in}{1.410799in}}{\pgfqpoint{2.717889in}{1.416385in}}{\pgfqpoint{2.713771in}{1.420503in}}%
\pgfpathcurveto{\pgfqpoint{2.709652in}{1.424621in}}{\pgfqpoint{2.704066in}{1.426935in}}{\pgfqpoint{2.698242in}{1.426935in}}%
\pgfpathcurveto{\pgfqpoint{2.692418in}{1.426935in}}{\pgfqpoint{2.686832in}{1.424621in}}{\pgfqpoint{2.682714in}{1.420503in}}%
\pgfpathcurveto{\pgfqpoint{2.678596in}{1.416385in}}{\pgfqpoint{2.676282in}{1.410799in}}{\pgfqpoint{2.676282in}{1.404975in}}%
\pgfpathcurveto{\pgfqpoint{2.676282in}{1.399151in}}{\pgfqpoint{2.678596in}{1.393565in}}{\pgfqpoint{2.682714in}{1.389447in}}%
\pgfpathcurveto{\pgfqpoint{2.686832in}{1.385329in}}{\pgfqpoint{2.692418in}{1.383015in}}{\pgfqpoint{2.698242in}{1.383015in}}%
\pgfpathlineto{\pgfqpoint{2.698242in}{1.383015in}}%
\pgfpathclose%
\pgfusepath{stroke,fill}%
\end{pgfscope}%
\begin{pgfscope}%
\pgfpathrectangle{\pgfqpoint{0.640323in}{0.527436in}}{\pgfqpoint{9.687500in}{3.850000in}}%
\pgfusepath{clip}%
\pgfsetbuttcap%
\pgfsetroundjoin%
\definecolor{currentfill}{rgb}{0.239216,0.478431,0.992157}%
\pgfsetfillcolor{currentfill}%
\pgfsetfillopacity{0.500000}%
\pgfsetlinewidth{1.003750pt}%
\definecolor{currentstroke}{rgb}{0.239216,0.478431,0.992157}%
\pgfsetstrokecolor{currentstroke}%
\pgfsetstrokeopacity{0.500000}%
\pgfsetdash{{3.700000pt}{1.600000pt}}{0.000000pt}%
\pgfpathmoveto{\pgfqpoint{2.877973in}{1.817829in}}%
\pgfpathcurveto{\pgfqpoint{2.883797in}{1.817829in}}{\pgfqpoint{2.889383in}{1.820143in}}{\pgfqpoint{2.893501in}{1.824261in}}%
\pgfpathcurveto{\pgfqpoint{2.897620in}{1.828379in}}{\pgfqpoint{2.899934in}{1.833965in}}{\pgfqpoint{2.899934in}{1.839789in}}%
\pgfpathcurveto{\pgfqpoint{2.899934in}{1.845613in}}{\pgfqpoint{2.897620in}{1.851199in}}{\pgfqpoint{2.893501in}{1.855317in}}%
\pgfpathcurveto{\pgfqpoint{2.889383in}{1.859435in}}{\pgfqpoint{2.883797in}{1.861749in}}{\pgfqpoint{2.877973in}{1.861749in}}%
\pgfpathcurveto{\pgfqpoint{2.872149in}{1.861749in}}{\pgfqpoint{2.866563in}{1.859435in}}{\pgfqpoint{2.862445in}{1.855317in}}%
\pgfpathcurveto{\pgfqpoint{2.858327in}{1.851199in}}{\pgfqpoint{2.856013in}{1.845613in}}{\pgfqpoint{2.856013in}{1.839789in}}%
\pgfpathcurveto{\pgfqpoint{2.856013in}{1.833965in}}{\pgfqpoint{2.858327in}{1.828379in}}{\pgfqpoint{2.862445in}{1.824261in}}%
\pgfpathcurveto{\pgfqpoint{2.866563in}{1.820143in}}{\pgfqpoint{2.872149in}{1.817829in}}{\pgfqpoint{2.877973in}{1.817829in}}%
\pgfpathlineto{\pgfqpoint{2.877973in}{1.817829in}}%
\pgfpathclose%
\pgfusepath{stroke,fill}%
\end{pgfscope}%
\begin{pgfscope}%
\pgfpathrectangle{\pgfqpoint{0.640323in}{0.527436in}}{\pgfqpoint{9.687500in}{3.850000in}}%
\pgfusepath{clip}%
\pgfsetbuttcap%
\pgfsetroundjoin%
\definecolor{currentfill}{rgb}{0.239216,0.478431,0.992157}%
\pgfsetfillcolor{currentfill}%
\pgfsetfillopacity{0.500000}%
\pgfsetlinewidth{1.003750pt}%
\definecolor{currentstroke}{rgb}{0.239216,0.478431,0.992157}%
\pgfsetstrokecolor{currentstroke}%
\pgfsetstrokeopacity{0.500000}%
\pgfsetdash{{3.700000pt}{1.600000pt}}{0.000000pt}%
\pgfpathmoveto{\pgfqpoint{3.057704in}{2.062204in}}%
\pgfpathcurveto{\pgfqpoint{3.063528in}{2.062204in}}{\pgfqpoint{3.069114in}{2.064518in}}{\pgfqpoint{3.073232in}{2.068636in}}%
\pgfpathcurveto{\pgfqpoint{3.077351in}{2.072755in}}{\pgfqpoint{3.079664in}{2.078341in}}{\pgfqpoint{3.079664in}{2.084165in}}%
\pgfpathcurveto{\pgfqpoint{3.079664in}{2.089989in}}{\pgfqpoint{3.077351in}{2.095575in}}{\pgfqpoint{3.073232in}{2.099693in}}%
\pgfpathcurveto{\pgfqpoint{3.069114in}{2.103811in}}{\pgfqpoint{3.063528in}{2.106125in}}{\pgfqpoint{3.057704in}{2.106125in}}%
\pgfpathcurveto{\pgfqpoint{3.051880in}{2.106125in}}{\pgfqpoint{3.046294in}{2.103811in}}{\pgfqpoint{3.042176in}{2.099693in}}%
\pgfpathcurveto{\pgfqpoint{3.038058in}{2.095575in}}{\pgfqpoint{3.035744in}{2.089989in}}{\pgfqpoint{3.035744in}{2.084165in}}%
\pgfpathcurveto{\pgfqpoint{3.035744in}{2.078341in}}{\pgfqpoint{3.038058in}{2.072755in}}{\pgfqpoint{3.042176in}{2.068636in}}%
\pgfpathcurveto{\pgfqpoint{3.046294in}{2.064518in}}{\pgfqpoint{3.051880in}{2.062204in}}{\pgfqpoint{3.057704in}{2.062204in}}%
\pgfpathlineto{\pgfqpoint{3.057704in}{2.062204in}}%
\pgfpathclose%
\pgfusepath{stroke,fill}%
\end{pgfscope}%
\begin{pgfscope}%
\pgfpathrectangle{\pgfqpoint{0.640323in}{0.527436in}}{\pgfqpoint{9.687500in}{3.850000in}}%
\pgfusepath{clip}%
\pgfsetbuttcap%
\pgfsetroundjoin%
\definecolor{currentfill}{rgb}{0.239216,0.478431,0.992157}%
\pgfsetfillcolor{currentfill}%
\pgfsetfillopacity{0.500000}%
\pgfsetlinewidth{1.003750pt}%
\definecolor{currentstroke}{rgb}{0.239216,0.478431,0.992157}%
\pgfsetstrokecolor{currentstroke}%
\pgfsetstrokeopacity{0.500000}%
\pgfsetdash{{3.700000pt}{1.600000pt}}{0.000000pt}%
\pgfpathmoveto{\pgfqpoint{3.237435in}{2.282573in}}%
\pgfpathcurveto{\pgfqpoint{3.243259in}{2.282573in}}{\pgfqpoint{3.248845in}{2.284887in}}{\pgfqpoint{3.252963in}{2.289005in}}%
\pgfpathcurveto{\pgfqpoint{3.257082in}{2.293124in}}{\pgfqpoint{3.259395in}{2.298710in}}{\pgfqpoint{3.259395in}{2.304534in}}%
\pgfpathcurveto{\pgfqpoint{3.259395in}{2.310358in}}{\pgfqpoint{3.257082in}{2.315944in}}{\pgfqpoint{3.252963in}{2.320062in}}%
\pgfpathcurveto{\pgfqpoint{3.248845in}{2.324180in}}{\pgfqpoint{3.243259in}{2.326494in}}{\pgfqpoint{3.237435in}{2.326494in}}%
\pgfpathcurveto{\pgfqpoint{3.231611in}{2.326494in}}{\pgfqpoint{3.226025in}{2.324180in}}{\pgfqpoint{3.221907in}{2.320062in}}%
\pgfpathcurveto{\pgfqpoint{3.217789in}{2.315944in}}{\pgfqpoint{3.215475in}{2.310358in}}{\pgfqpoint{3.215475in}{2.304534in}}%
\pgfpathcurveto{\pgfqpoint{3.215475in}{2.298710in}}{\pgfqpoint{3.217789in}{2.293124in}}{\pgfqpoint{3.221907in}{2.289005in}}%
\pgfpathcurveto{\pgfqpoint{3.226025in}{2.284887in}}{\pgfqpoint{3.231611in}{2.282573in}}{\pgfqpoint{3.237435in}{2.282573in}}%
\pgfpathlineto{\pgfqpoint{3.237435in}{2.282573in}}%
\pgfpathclose%
\pgfusepath{stroke,fill}%
\end{pgfscope}%
\begin{pgfscope}%
\pgfpathrectangle{\pgfqpoint{0.640323in}{0.527436in}}{\pgfqpoint{9.687500in}{3.850000in}}%
\pgfusepath{clip}%
\pgfsetbuttcap%
\pgfsetroundjoin%
\definecolor{currentfill}{rgb}{0.239216,0.478431,0.992157}%
\pgfsetfillcolor{currentfill}%
\pgfsetfillopacity{0.500000}%
\pgfsetlinewidth{1.003750pt}%
\definecolor{currentstroke}{rgb}{0.239216,0.478431,0.992157}%
\pgfsetstrokecolor{currentstroke}%
\pgfsetstrokeopacity{0.500000}%
\pgfsetdash{{3.700000pt}{1.600000pt}}{0.000000pt}%
\pgfpathmoveto{\pgfqpoint{3.417166in}{2.451719in}}%
\pgfpathcurveto{\pgfqpoint{3.422990in}{2.451719in}}{\pgfqpoint{3.428576in}{2.454033in}}{\pgfqpoint{3.432694in}{2.458151in}}%
\pgfpathcurveto{\pgfqpoint{3.436813in}{2.462269in}}{\pgfqpoint{3.439126in}{2.467855in}}{\pgfqpoint{3.439126in}{2.473679in}}%
\pgfpathcurveto{\pgfqpoint{3.439126in}{2.479503in}}{\pgfqpoint{3.436813in}{2.485089in}}{\pgfqpoint{3.432694in}{2.489207in}}%
\pgfpathcurveto{\pgfqpoint{3.428576in}{2.493326in}}{\pgfqpoint{3.422990in}{2.495639in}}{\pgfqpoint{3.417166in}{2.495639in}}%
\pgfpathcurveto{\pgfqpoint{3.411342in}{2.495639in}}{\pgfqpoint{3.405756in}{2.493326in}}{\pgfqpoint{3.401638in}{2.489207in}}%
\pgfpathcurveto{\pgfqpoint{3.397520in}{2.485089in}}{\pgfqpoint{3.395206in}{2.479503in}}{\pgfqpoint{3.395206in}{2.473679in}}%
\pgfpathcurveto{\pgfqpoint{3.395206in}{2.467855in}}{\pgfqpoint{3.397520in}{2.462269in}}{\pgfqpoint{3.401638in}{2.458151in}}%
\pgfpathcurveto{\pgfqpoint{3.405756in}{2.454033in}}{\pgfqpoint{3.411342in}{2.451719in}}{\pgfqpoint{3.417166in}{2.451719in}}%
\pgfpathlineto{\pgfqpoint{3.417166in}{2.451719in}}%
\pgfpathclose%
\pgfusepath{stroke,fill}%
\end{pgfscope}%
\begin{pgfscope}%
\pgfpathrectangle{\pgfqpoint{0.640323in}{0.527436in}}{\pgfqpoint{9.687500in}{3.850000in}}%
\pgfusepath{clip}%
\pgfsetbuttcap%
\pgfsetroundjoin%
\definecolor{currentfill}{rgb}{0.239216,0.478431,0.992157}%
\pgfsetfillcolor{currentfill}%
\pgfsetfillopacity{0.500000}%
\pgfsetlinewidth{1.003750pt}%
\definecolor{currentstroke}{rgb}{0.239216,0.478431,0.992157}%
\pgfsetstrokecolor{currentstroke}%
\pgfsetstrokeopacity{0.500000}%
\pgfsetdash{{3.700000pt}{1.600000pt}}{0.000000pt}%
\pgfpathmoveto{\pgfqpoint{3.596897in}{2.594715in}}%
\pgfpathcurveto{\pgfqpoint{3.602721in}{2.594715in}}{\pgfqpoint{3.608307in}{2.597029in}}{\pgfqpoint{3.612425in}{2.601147in}}%
\pgfpathcurveto{\pgfqpoint{3.616544in}{2.605265in}}{\pgfqpoint{3.618857in}{2.610852in}}{\pgfqpoint{3.618857in}{2.616676in}}%
\pgfpathcurveto{\pgfqpoint{3.618857in}{2.622500in}}{\pgfqpoint{3.616544in}{2.628086in}}{\pgfqpoint{3.612425in}{2.632204in}}%
\pgfpathcurveto{\pgfqpoint{3.608307in}{2.636322in}}{\pgfqpoint{3.602721in}{2.638636in}}{\pgfqpoint{3.596897in}{2.638636in}}%
\pgfpathcurveto{\pgfqpoint{3.591073in}{2.638636in}}{\pgfqpoint{3.585487in}{2.636322in}}{\pgfqpoint{3.581369in}{2.632204in}}%
\pgfpathcurveto{\pgfqpoint{3.577251in}{2.628086in}}{\pgfqpoint{3.574937in}{2.622500in}}{\pgfqpoint{3.574937in}{2.616676in}}%
\pgfpathcurveto{\pgfqpoint{3.574937in}{2.610852in}}{\pgfqpoint{3.577251in}{2.605265in}}{\pgfqpoint{3.581369in}{2.601147in}}%
\pgfpathcurveto{\pgfqpoint{3.585487in}{2.597029in}}{\pgfqpoint{3.591073in}{2.594715in}}{\pgfqpoint{3.596897in}{2.594715in}}%
\pgfpathlineto{\pgfqpoint{3.596897in}{2.594715in}}%
\pgfpathclose%
\pgfusepath{stroke,fill}%
\end{pgfscope}%
\begin{pgfscope}%
\pgfpathrectangle{\pgfqpoint{0.640323in}{0.527436in}}{\pgfqpoint{9.687500in}{3.850000in}}%
\pgfusepath{clip}%
\pgfsetbuttcap%
\pgfsetroundjoin%
\definecolor{currentfill}{rgb}{0.239216,0.478431,0.992157}%
\pgfsetfillcolor{currentfill}%
\pgfsetfillopacity{0.500000}%
\pgfsetlinewidth{1.003750pt}%
\definecolor{currentstroke}{rgb}{0.239216,0.478431,0.992157}%
\pgfsetstrokecolor{currentstroke}%
\pgfsetstrokeopacity{0.500000}%
\pgfsetdash{{3.700000pt}{1.600000pt}}{0.000000pt}%
\pgfpathmoveto{\pgfqpoint{3.776628in}{2.731452in}}%
\pgfpathcurveto{\pgfqpoint{3.782452in}{2.731452in}}{\pgfqpoint{3.788038in}{2.733766in}}{\pgfqpoint{3.792156in}{2.737884in}}%
\pgfpathcurveto{\pgfqpoint{3.796275in}{2.742003in}}{\pgfqpoint{3.798588in}{2.747589in}}{\pgfqpoint{3.798588in}{2.753413in}}%
\pgfpathcurveto{\pgfqpoint{3.798588in}{2.759237in}}{\pgfqpoint{3.796275in}{2.764823in}}{\pgfqpoint{3.792156in}{2.768941in}}%
\pgfpathcurveto{\pgfqpoint{3.788038in}{2.773059in}}{\pgfqpoint{3.782452in}{2.775373in}}{\pgfqpoint{3.776628in}{2.775373in}}%
\pgfpathcurveto{\pgfqpoint{3.770804in}{2.775373in}}{\pgfqpoint{3.765218in}{2.773059in}}{\pgfqpoint{3.761100in}{2.768941in}}%
\pgfpathcurveto{\pgfqpoint{3.756982in}{2.764823in}}{\pgfqpoint{3.754668in}{2.759237in}}{\pgfqpoint{3.754668in}{2.753413in}}%
\pgfpathcurveto{\pgfqpoint{3.754668in}{2.747589in}}{\pgfqpoint{3.756982in}{2.742003in}}{\pgfqpoint{3.761100in}{2.737884in}}%
\pgfpathcurveto{\pgfqpoint{3.765218in}{2.733766in}}{\pgfqpoint{3.770804in}{2.731452in}}{\pgfqpoint{3.776628in}{2.731452in}}%
\pgfpathlineto{\pgfqpoint{3.776628in}{2.731452in}}%
\pgfpathclose%
\pgfusepath{stroke,fill}%
\end{pgfscope}%
\begin{pgfscope}%
\pgfpathrectangle{\pgfqpoint{0.640323in}{0.527436in}}{\pgfqpoint{9.687500in}{3.850000in}}%
\pgfusepath{clip}%
\pgfsetbuttcap%
\pgfsetroundjoin%
\definecolor{currentfill}{rgb}{0.239216,0.478431,0.992157}%
\pgfsetfillcolor{currentfill}%
\pgfsetfillopacity{0.500000}%
\pgfsetlinewidth{1.003750pt}%
\definecolor{currentstroke}{rgb}{0.239216,0.478431,0.992157}%
\pgfsetstrokecolor{currentstroke}%
\pgfsetstrokeopacity{0.500000}%
\pgfsetdash{{3.700000pt}{1.600000pt}}{0.000000pt}%
\pgfpathmoveto{\pgfqpoint{3.956359in}{2.844940in}}%
\pgfpathcurveto{\pgfqpoint{3.962183in}{2.844940in}}{\pgfqpoint{3.967769in}{2.847254in}}{\pgfqpoint{3.971887in}{2.851373in}}%
\pgfpathcurveto{\pgfqpoint{3.976006in}{2.855491in}}{\pgfqpoint{3.978319in}{2.861077in}}{\pgfqpoint{3.978319in}{2.866901in}}%
\pgfpathcurveto{\pgfqpoint{3.978319in}{2.872725in}}{\pgfqpoint{3.976006in}{2.878311in}}{\pgfqpoint{3.971887in}{2.882429in}}%
\pgfpathcurveto{\pgfqpoint{3.967769in}{2.886547in}}{\pgfqpoint{3.962183in}{2.888861in}}{\pgfqpoint{3.956359in}{2.888861in}}%
\pgfpathcurveto{\pgfqpoint{3.950535in}{2.888861in}}{\pgfqpoint{3.944949in}{2.886547in}}{\pgfqpoint{3.940831in}{2.882429in}}%
\pgfpathcurveto{\pgfqpoint{3.936713in}{2.878311in}}{\pgfqpoint{3.934399in}{2.872725in}}{\pgfqpoint{3.934399in}{2.866901in}}%
\pgfpathcurveto{\pgfqpoint{3.934399in}{2.861077in}}{\pgfqpoint{3.936713in}{2.855491in}}{\pgfqpoint{3.940831in}{2.851373in}}%
\pgfpathcurveto{\pgfqpoint{3.944949in}{2.847254in}}{\pgfqpoint{3.950535in}{2.844940in}}{\pgfqpoint{3.956359in}{2.844940in}}%
\pgfpathlineto{\pgfqpoint{3.956359in}{2.844940in}}%
\pgfpathclose%
\pgfusepath{stroke,fill}%
\end{pgfscope}%
\begin{pgfscope}%
\pgfpathrectangle{\pgfqpoint{0.640323in}{0.527436in}}{\pgfqpoint{9.687500in}{3.850000in}}%
\pgfusepath{clip}%
\pgfsetbuttcap%
\pgfsetroundjoin%
\definecolor{currentfill}{rgb}{0.239216,0.478431,0.992157}%
\pgfsetfillcolor{currentfill}%
\pgfsetfillopacity{0.500000}%
\pgfsetlinewidth{1.003750pt}%
\definecolor{currentstroke}{rgb}{0.239216,0.478431,0.992157}%
\pgfsetstrokecolor{currentstroke}%
\pgfsetstrokeopacity{0.500000}%
\pgfsetdash{{3.700000pt}{1.600000pt}}{0.000000pt}%
\pgfpathmoveto{\pgfqpoint{4.136090in}{2.936043in}}%
\pgfpathcurveto{\pgfqpoint{4.141914in}{2.936043in}}{\pgfqpoint{4.147500in}{2.938357in}}{\pgfqpoint{4.151618in}{2.942475in}}%
\pgfpathcurveto{\pgfqpoint{4.155737in}{2.946593in}}{\pgfqpoint{4.158050in}{2.952179in}}{\pgfqpoint{4.158050in}{2.958003in}}%
\pgfpathcurveto{\pgfqpoint{4.158050in}{2.963827in}}{\pgfqpoint{4.155737in}{2.969413in}}{\pgfqpoint{4.151618in}{2.973531in}}%
\pgfpathcurveto{\pgfqpoint{4.147500in}{2.977649in}}{\pgfqpoint{4.141914in}{2.979963in}}{\pgfqpoint{4.136090in}{2.979963in}}%
\pgfpathcurveto{\pgfqpoint{4.130266in}{2.979963in}}{\pgfqpoint{4.124680in}{2.977649in}}{\pgfqpoint{4.120562in}{2.973531in}}%
\pgfpathcurveto{\pgfqpoint{4.116444in}{2.969413in}}{\pgfqpoint{4.114130in}{2.963827in}}{\pgfqpoint{4.114130in}{2.958003in}}%
\pgfpathcurveto{\pgfqpoint{4.114130in}{2.952179in}}{\pgfqpoint{4.116444in}{2.946593in}}{\pgfqpoint{4.120562in}{2.942475in}}%
\pgfpathcurveto{\pgfqpoint{4.124680in}{2.938357in}}{\pgfqpoint{4.130266in}{2.936043in}}{\pgfqpoint{4.136090in}{2.936043in}}%
\pgfpathlineto{\pgfqpoint{4.136090in}{2.936043in}}%
\pgfpathclose%
\pgfusepath{stroke,fill}%
\end{pgfscope}%
\begin{pgfscope}%
\pgfpathrectangle{\pgfqpoint{0.640323in}{0.527436in}}{\pgfqpoint{9.687500in}{3.850000in}}%
\pgfusepath{clip}%
\pgfsetbuttcap%
\pgfsetroundjoin%
\definecolor{currentfill}{rgb}{0.239216,0.478431,0.992157}%
\pgfsetfillcolor{currentfill}%
\pgfsetfillopacity{0.500000}%
\pgfsetlinewidth{1.003750pt}%
\definecolor{currentstroke}{rgb}{0.239216,0.478431,0.992157}%
\pgfsetstrokecolor{currentstroke}%
\pgfsetstrokeopacity{0.500000}%
\pgfsetdash{{3.700000pt}{1.600000pt}}{0.000000pt}%
\pgfpathmoveto{\pgfqpoint{4.315821in}{3.029796in}}%
\pgfpathcurveto{\pgfqpoint{4.321645in}{3.029796in}}{\pgfqpoint{4.327231in}{3.032110in}}{\pgfqpoint{4.331349in}{3.036228in}}%
\pgfpathcurveto{\pgfqpoint{4.335467in}{3.040347in}}{\pgfqpoint{4.337781in}{3.045933in}}{\pgfqpoint{4.337781in}{3.051757in}}%
\pgfpathcurveto{\pgfqpoint{4.337781in}{3.057581in}}{\pgfqpoint{4.335467in}{3.063167in}}{\pgfqpoint{4.331349in}{3.067285in}}%
\pgfpathcurveto{\pgfqpoint{4.327231in}{3.071403in}}{\pgfqpoint{4.321645in}{3.073717in}}{\pgfqpoint{4.315821in}{3.073717in}}%
\pgfpathcurveto{\pgfqpoint{4.309997in}{3.073717in}}{\pgfqpoint{4.304411in}{3.071403in}}{\pgfqpoint{4.300293in}{3.067285in}}%
\pgfpathcurveto{\pgfqpoint{4.296175in}{3.063167in}}{\pgfqpoint{4.293861in}{3.057581in}}{\pgfqpoint{4.293861in}{3.051757in}}%
\pgfpathcurveto{\pgfqpoint{4.293861in}{3.045933in}}{\pgfqpoint{4.296175in}{3.040347in}}{\pgfqpoint{4.300293in}{3.036228in}}%
\pgfpathcurveto{\pgfqpoint{4.304411in}{3.032110in}}{\pgfqpoint{4.309997in}{3.029796in}}{\pgfqpoint{4.315821in}{3.029796in}}%
\pgfpathlineto{\pgfqpoint{4.315821in}{3.029796in}}%
\pgfpathclose%
\pgfusepath{stroke,fill}%
\end{pgfscope}%
\begin{pgfscope}%
\pgfpathrectangle{\pgfqpoint{0.640323in}{0.527436in}}{\pgfqpoint{9.687500in}{3.850000in}}%
\pgfusepath{clip}%
\pgfsetbuttcap%
\pgfsetroundjoin%
\definecolor{currentfill}{rgb}{0.239216,0.478431,0.992157}%
\pgfsetfillcolor{currentfill}%
\pgfsetfillopacity{0.500000}%
\pgfsetlinewidth{1.003750pt}%
\definecolor{currentstroke}{rgb}{0.239216,0.478431,0.992157}%
\pgfsetstrokecolor{currentstroke}%
\pgfsetstrokeopacity{0.500000}%
\pgfsetdash{{3.700000pt}{1.600000pt}}{0.000000pt}%
\pgfpathmoveto{\pgfqpoint{4.495552in}{3.100301in}}%
\pgfpathcurveto{\pgfqpoint{4.501376in}{3.100301in}}{\pgfqpoint{4.506962in}{3.102615in}}{\pgfqpoint{4.511080in}{3.106733in}}%
\pgfpathcurveto{\pgfqpoint{4.515198in}{3.110851in}}{\pgfqpoint{4.517512in}{3.116437in}}{\pgfqpoint{4.517512in}{3.122261in}}%
\pgfpathcurveto{\pgfqpoint{4.517512in}{3.128085in}}{\pgfqpoint{4.515198in}{3.133671in}}{\pgfqpoint{4.511080in}{3.137790in}}%
\pgfpathcurveto{\pgfqpoint{4.506962in}{3.141908in}}{\pgfqpoint{4.501376in}{3.144222in}}{\pgfqpoint{4.495552in}{3.144222in}}%
\pgfpathcurveto{\pgfqpoint{4.489728in}{3.144222in}}{\pgfqpoint{4.484142in}{3.141908in}}{\pgfqpoint{4.480024in}{3.137790in}}%
\pgfpathcurveto{\pgfqpoint{4.475906in}{3.133671in}}{\pgfqpoint{4.473592in}{3.128085in}}{\pgfqpoint{4.473592in}{3.122261in}}%
\pgfpathcurveto{\pgfqpoint{4.473592in}{3.116437in}}{\pgfqpoint{4.475906in}{3.110851in}}{\pgfqpoint{4.480024in}{3.106733in}}%
\pgfpathcurveto{\pgfqpoint{4.484142in}{3.102615in}}{\pgfqpoint{4.489728in}{3.100301in}}{\pgfqpoint{4.495552in}{3.100301in}}%
\pgfpathlineto{\pgfqpoint{4.495552in}{3.100301in}}%
\pgfpathclose%
\pgfusepath{stroke,fill}%
\end{pgfscope}%
\begin{pgfscope}%
\pgfpathrectangle{\pgfqpoint{0.640323in}{0.527436in}}{\pgfqpoint{9.687500in}{3.850000in}}%
\pgfusepath{clip}%
\pgfsetbuttcap%
\pgfsetroundjoin%
\definecolor{currentfill}{rgb}{0.239216,0.478431,0.992157}%
\pgfsetfillcolor{currentfill}%
\pgfsetfillopacity{0.500000}%
\pgfsetlinewidth{1.003750pt}%
\definecolor{currentstroke}{rgb}{0.239216,0.478431,0.992157}%
\pgfsetstrokecolor{currentstroke}%
\pgfsetstrokeopacity{0.500000}%
\pgfsetdash{{3.700000pt}{1.600000pt}}{0.000000pt}%
\pgfpathmoveto{\pgfqpoint{4.675283in}{3.165366in}}%
\pgfpathcurveto{\pgfqpoint{4.681107in}{3.165366in}}{\pgfqpoint{4.686693in}{3.167680in}}{\pgfqpoint{4.690811in}{3.171798in}}%
\pgfpathcurveto{\pgfqpoint{4.694929in}{3.175916in}}{\pgfqpoint{4.697243in}{3.181502in}}{\pgfqpoint{4.697243in}{3.187326in}}%
\pgfpathcurveto{\pgfqpoint{4.697243in}{3.193150in}}{\pgfqpoint{4.694929in}{3.198736in}}{\pgfqpoint{4.690811in}{3.202855in}}%
\pgfpathcurveto{\pgfqpoint{4.686693in}{3.206973in}}{\pgfqpoint{4.681107in}{3.209287in}}{\pgfqpoint{4.675283in}{3.209287in}}%
\pgfpathcurveto{\pgfqpoint{4.669459in}{3.209287in}}{\pgfqpoint{4.663873in}{3.206973in}}{\pgfqpoint{4.659755in}{3.202855in}}%
\pgfpathcurveto{\pgfqpoint{4.655637in}{3.198736in}}{\pgfqpoint{4.653323in}{3.193150in}}{\pgfqpoint{4.653323in}{3.187326in}}%
\pgfpathcurveto{\pgfqpoint{4.653323in}{3.181502in}}{\pgfqpoint{4.655637in}{3.175916in}}{\pgfqpoint{4.659755in}{3.171798in}}%
\pgfpathcurveto{\pgfqpoint{4.663873in}{3.167680in}}{\pgfqpoint{4.669459in}{3.165366in}}{\pgfqpoint{4.675283in}{3.165366in}}%
\pgfpathlineto{\pgfqpoint{4.675283in}{3.165366in}}%
\pgfpathclose%
\pgfusepath{stroke,fill}%
\end{pgfscope}%
\begin{pgfscope}%
\pgfpathrectangle{\pgfqpoint{0.640323in}{0.527436in}}{\pgfqpoint{9.687500in}{3.850000in}}%
\pgfusepath{clip}%
\pgfsetbuttcap%
\pgfsetroundjoin%
\definecolor{currentfill}{rgb}{0.239216,0.478431,0.992157}%
\pgfsetfillcolor{currentfill}%
\pgfsetfillopacity{0.500000}%
\pgfsetlinewidth{1.003750pt}%
\definecolor{currentstroke}{rgb}{0.239216,0.478431,0.992157}%
\pgfsetstrokecolor{currentstroke}%
\pgfsetstrokeopacity{0.500000}%
\pgfsetdash{{3.700000pt}{1.600000pt}}{0.000000pt}%
\pgfpathmoveto{\pgfqpoint{4.855014in}{3.224333in}}%
\pgfpathcurveto{\pgfqpoint{4.860838in}{3.224333in}}{\pgfqpoint{4.866424in}{3.226647in}}{\pgfqpoint{4.870542in}{3.230765in}}%
\pgfpathcurveto{\pgfqpoint{4.874660in}{3.234883in}}{\pgfqpoint{4.876974in}{3.240469in}}{\pgfqpoint{4.876974in}{3.246293in}}%
\pgfpathcurveto{\pgfqpoint{4.876974in}{3.252117in}}{\pgfqpoint{4.874660in}{3.257704in}}{\pgfqpoint{4.870542in}{3.261822in}}%
\pgfpathcurveto{\pgfqpoint{4.866424in}{3.265940in}}{\pgfqpoint{4.860838in}{3.268254in}}{\pgfqpoint{4.855014in}{3.268254in}}%
\pgfpathcurveto{\pgfqpoint{4.849190in}{3.268254in}}{\pgfqpoint{4.843604in}{3.265940in}}{\pgfqpoint{4.839486in}{3.261822in}}%
\pgfpathcurveto{\pgfqpoint{4.835368in}{3.257704in}}{\pgfqpoint{4.833054in}{3.252117in}}{\pgfqpoint{4.833054in}{3.246293in}}%
\pgfpathcurveto{\pgfqpoint{4.833054in}{3.240469in}}{\pgfqpoint{4.835368in}{3.234883in}}{\pgfqpoint{4.839486in}{3.230765in}}%
\pgfpathcurveto{\pgfqpoint{4.843604in}{3.226647in}}{\pgfqpoint{4.849190in}{3.224333in}}{\pgfqpoint{4.855014in}{3.224333in}}%
\pgfpathlineto{\pgfqpoint{4.855014in}{3.224333in}}%
\pgfpathclose%
\pgfusepath{stroke,fill}%
\end{pgfscope}%
\begin{pgfscope}%
\pgfpathrectangle{\pgfqpoint{0.640323in}{0.527436in}}{\pgfqpoint{9.687500in}{3.850000in}}%
\pgfusepath{clip}%
\pgfsetbuttcap%
\pgfsetroundjoin%
\definecolor{currentfill}{rgb}{0.239216,0.478431,0.992157}%
\pgfsetfillcolor{currentfill}%
\pgfsetfillopacity{0.500000}%
\pgfsetlinewidth{1.003750pt}%
\definecolor{currentstroke}{rgb}{0.239216,0.478431,0.992157}%
\pgfsetstrokecolor{currentstroke}%
\pgfsetstrokeopacity{0.500000}%
\pgfsetdash{{3.700000pt}{1.600000pt}}{0.000000pt}%
\pgfpathmoveto{\pgfqpoint{5.034745in}{3.294000in}}%
\pgfpathcurveto{\pgfqpoint{5.040569in}{3.294000in}}{\pgfqpoint{5.046155in}{3.296313in}}{\pgfqpoint{5.050273in}{3.300432in}}%
\pgfpathcurveto{\pgfqpoint{5.054391in}{3.304550in}}{\pgfqpoint{5.056705in}{3.310136in}}{\pgfqpoint{5.056705in}{3.315960in}}%
\pgfpathcurveto{\pgfqpoint{5.056705in}{3.321784in}}{\pgfqpoint{5.054391in}{3.327370in}}{\pgfqpoint{5.050273in}{3.331488in}}%
\pgfpathcurveto{\pgfqpoint{5.046155in}{3.335606in}}{\pgfqpoint{5.040569in}{3.337920in}}{\pgfqpoint{5.034745in}{3.337920in}}%
\pgfpathcurveto{\pgfqpoint{5.028921in}{3.337920in}}{\pgfqpoint{5.023335in}{3.335606in}}{\pgfqpoint{5.019217in}{3.331488in}}%
\pgfpathcurveto{\pgfqpoint{5.015099in}{3.327370in}}{\pgfqpoint{5.012785in}{3.321784in}}{\pgfqpoint{5.012785in}{3.315960in}}%
\pgfpathcurveto{\pgfqpoint{5.012785in}{3.310136in}}{\pgfqpoint{5.015099in}{3.304550in}}{\pgfqpoint{5.019217in}{3.300432in}}%
\pgfpathcurveto{\pgfqpoint{5.023335in}{3.296313in}}{\pgfqpoint{5.028921in}{3.294000in}}{\pgfqpoint{5.034745in}{3.294000in}}%
\pgfpathlineto{\pgfqpoint{5.034745in}{3.294000in}}%
\pgfpathclose%
\pgfusepath{stroke,fill}%
\end{pgfscope}%
\begin{pgfscope}%
\pgfpathrectangle{\pgfqpoint{0.640323in}{0.527436in}}{\pgfqpoint{9.687500in}{3.850000in}}%
\pgfusepath{clip}%
\pgfsetbuttcap%
\pgfsetroundjoin%
\definecolor{currentfill}{rgb}{0.239216,0.478431,0.992157}%
\pgfsetfillcolor{currentfill}%
\pgfsetfillopacity{0.500000}%
\pgfsetlinewidth{1.003750pt}%
\definecolor{currentstroke}{rgb}{0.239216,0.478431,0.992157}%
\pgfsetstrokecolor{currentstroke}%
\pgfsetstrokeopacity{0.500000}%
\pgfsetdash{{3.700000pt}{1.600000pt}}{0.000000pt}%
\pgfpathmoveto{\pgfqpoint{5.214476in}{3.349377in}}%
\pgfpathcurveto{\pgfqpoint{5.220300in}{3.349377in}}{\pgfqpoint{5.225886in}{3.351691in}}{\pgfqpoint{5.230004in}{3.355809in}}%
\pgfpathcurveto{\pgfqpoint{5.234122in}{3.359928in}}{\pgfqpoint{5.236436in}{3.365514in}}{\pgfqpoint{5.236436in}{3.371338in}}%
\pgfpathcurveto{\pgfqpoint{5.236436in}{3.377162in}}{\pgfqpoint{5.234122in}{3.382748in}}{\pgfqpoint{5.230004in}{3.386866in}}%
\pgfpathcurveto{\pgfqpoint{5.225886in}{3.390984in}}{\pgfqpoint{5.220300in}{3.393298in}}{\pgfqpoint{5.214476in}{3.393298in}}%
\pgfpathcurveto{\pgfqpoint{5.208652in}{3.393298in}}{\pgfqpoint{5.203066in}{3.390984in}}{\pgfqpoint{5.198948in}{3.386866in}}%
\pgfpathcurveto{\pgfqpoint{5.194830in}{3.382748in}}{\pgfqpoint{5.192516in}{3.377162in}}{\pgfqpoint{5.192516in}{3.371338in}}%
\pgfpathcurveto{\pgfqpoint{5.192516in}{3.365514in}}{\pgfqpoint{5.194830in}{3.359928in}}{\pgfqpoint{5.198948in}{3.355809in}}%
\pgfpathcurveto{\pgfqpoint{5.203066in}{3.351691in}}{\pgfqpoint{5.208652in}{3.349377in}}{\pgfqpoint{5.214476in}{3.349377in}}%
\pgfpathlineto{\pgfqpoint{5.214476in}{3.349377in}}%
\pgfpathclose%
\pgfusepath{stroke,fill}%
\end{pgfscope}%
\begin{pgfscope}%
\pgfpathrectangle{\pgfqpoint{0.640323in}{0.527436in}}{\pgfqpoint{9.687500in}{3.850000in}}%
\pgfusepath{clip}%
\pgfsetbuttcap%
\pgfsetroundjoin%
\definecolor{currentfill}{rgb}{0.239216,0.478431,0.992157}%
\pgfsetfillcolor{currentfill}%
\pgfsetfillopacity{0.500000}%
\pgfsetlinewidth{1.003750pt}%
\definecolor{currentstroke}{rgb}{0.239216,0.478431,0.992157}%
\pgfsetstrokecolor{currentstroke}%
\pgfsetstrokeopacity{0.500000}%
\pgfsetdash{{3.700000pt}{1.600000pt}}{0.000000pt}%
\pgfpathmoveto{\pgfqpoint{5.394207in}{3.389585in}}%
\pgfpathcurveto{\pgfqpoint{5.400031in}{3.389585in}}{\pgfqpoint{5.405617in}{3.391899in}}{\pgfqpoint{5.409735in}{3.396017in}}%
\pgfpathcurveto{\pgfqpoint{5.413853in}{3.400135in}}{\pgfqpoint{5.416167in}{3.405721in}}{\pgfqpoint{5.416167in}{3.411545in}}%
\pgfpathcurveto{\pgfqpoint{5.416167in}{3.417369in}}{\pgfqpoint{5.413853in}{3.422955in}}{\pgfqpoint{5.409735in}{3.427074in}}%
\pgfpathcurveto{\pgfqpoint{5.405617in}{3.431192in}}{\pgfqpoint{5.400031in}{3.433506in}}{\pgfqpoint{5.394207in}{3.433506in}}%
\pgfpathcurveto{\pgfqpoint{5.388383in}{3.433506in}}{\pgfqpoint{5.382797in}{3.431192in}}{\pgfqpoint{5.378679in}{3.427074in}}%
\pgfpathcurveto{\pgfqpoint{5.374561in}{3.422955in}}{\pgfqpoint{5.372247in}{3.417369in}}{\pgfqpoint{5.372247in}{3.411545in}}%
\pgfpathcurveto{\pgfqpoint{5.372247in}{3.405721in}}{\pgfqpoint{5.374561in}{3.400135in}}{\pgfqpoint{5.378679in}{3.396017in}}%
\pgfpathcurveto{\pgfqpoint{5.382797in}{3.391899in}}{\pgfqpoint{5.388383in}{3.389585in}}{\pgfqpoint{5.394207in}{3.389585in}}%
\pgfpathlineto{\pgfqpoint{5.394207in}{3.389585in}}%
\pgfpathclose%
\pgfusepath{stroke,fill}%
\end{pgfscope}%
\begin{pgfscope}%
\pgfpathrectangle{\pgfqpoint{0.640323in}{0.527436in}}{\pgfqpoint{9.687500in}{3.850000in}}%
\pgfusepath{clip}%
\pgfsetbuttcap%
\pgfsetroundjoin%
\definecolor{currentfill}{rgb}{0.239216,0.478431,0.992157}%
\pgfsetfillcolor{currentfill}%
\pgfsetfillopacity{0.500000}%
\pgfsetlinewidth{1.003750pt}%
\definecolor{currentstroke}{rgb}{0.239216,0.478431,0.992157}%
\pgfsetstrokecolor{currentstroke}%
\pgfsetstrokeopacity{0.500000}%
\pgfsetdash{{3.700000pt}{1.600000pt}}{0.000000pt}%
\pgfpathmoveto{\pgfqpoint{5.573938in}{3.442610in}}%
\pgfpathcurveto{\pgfqpoint{5.579762in}{3.442610in}}{\pgfqpoint{5.585348in}{3.444923in}}{\pgfqpoint{5.589466in}{3.449042in}}%
\pgfpathcurveto{\pgfqpoint{5.593584in}{3.453160in}}{\pgfqpoint{5.595898in}{3.458746in}}{\pgfqpoint{5.595898in}{3.464570in}}%
\pgfpathcurveto{\pgfqpoint{5.595898in}{3.470394in}}{\pgfqpoint{5.593584in}{3.475980in}}{\pgfqpoint{5.589466in}{3.480098in}}%
\pgfpathcurveto{\pgfqpoint{5.585348in}{3.484216in}}{\pgfqpoint{5.579762in}{3.486530in}}{\pgfqpoint{5.573938in}{3.486530in}}%
\pgfpathcurveto{\pgfqpoint{5.568114in}{3.486530in}}{\pgfqpoint{5.562528in}{3.484216in}}{\pgfqpoint{5.558410in}{3.480098in}}%
\pgfpathcurveto{\pgfqpoint{5.554292in}{3.475980in}}{\pgfqpoint{5.551978in}{3.470394in}}{\pgfqpoint{5.551978in}{3.464570in}}%
\pgfpathcurveto{\pgfqpoint{5.551978in}{3.458746in}}{\pgfqpoint{5.554292in}{3.453160in}}{\pgfqpoint{5.558410in}{3.449042in}}%
\pgfpathcurveto{\pgfqpoint{5.562528in}{3.444923in}}{\pgfqpoint{5.568114in}{3.442610in}}{\pgfqpoint{5.573938in}{3.442610in}}%
\pgfpathlineto{\pgfqpoint{5.573938in}{3.442610in}}%
\pgfpathclose%
\pgfusepath{stroke,fill}%
\end{pgfscope}%
\begin{pgfscope}%
\pgfpathrectangle{\pgfqpoint{0.640323in}{0.527436in}}{\pgfqpoint{9.687500in}{3.850000in}}%
\pgfusepath{clip}%
\pgfsetbuttcap%
\pgfsetroundjoin%
\definecolor{currentfill}{rgb}{0.239216,0.478431,0.992157}%
\pgfsetfillcolor{currentfill}%
\pgfsetfillopacity{0.500000}%
\pgfsetlinewidth{1.003750pt}%
\definecolor{currentstroke}{rgb}{0.239216,0.478431,0.992157}%
\pgfsetstrokecolor{currentstroke}%
\pgfsetstrokeopacity{0.500000}%
\pgfsetdash{{3.700000pt}{1.600000pt}}{0.000000pt}%
\pgfpathmoveto{\pgfqpoint{5.753669in}{3.477849in}}%
\pgfpathcurveto{\pgfqpoint{5.759493in}{3.477849in}}{\pgfqpoint{5.765079in}{3.480163in}}{\pgfqpoint{5.769197in}{3.484281in}}%
\pgfpathcurveto{\pgfqpoint{5.773315in}{3.488400in}}{\pgfqpoint{5.775629in}{3.493986in}}{\pgfqpoint{5.775629in}{3.499810in}}%
\pgfpathcurveto{\pgfqpoint{5.775629in}{3.505634in}}{\pgfqpoint{5.773315in}{3.511220in}}{\pgfqpoint{5.769197in}{3.515338in}}%
\pgfpathcurveto{\pgfqpoint{5.765079in}{3.519456in}}{\pgfqpoint{5.759493in}{3.521770in}}{\pgfqpoint{5.753669in}{3.521770in}}%
\pgfpathcurveto{\pgfqpoint{5.747845in}{3.521770in}}{\pgfqpoint{5.742259in}{3.519456in}}{\pgfqpoint{5.738141in}{3.515338in}}%
\pgfpathcurveto{\pgfqpoint{5.734023in}{3.511220in}}{\pgfqpoint{5.731709in}{3.505634in}}{\pgfqpoint{5.731709in}{3.499810in}}%
\pgfpathcurveto{\pgfqpoint{5.731709in}{3.493986in}}{\pgfqpoint{5.734023in}{3.488400in}}{\pgfqpoint{5.738141in}{3.484281in}}%
\pgfpathcurveto{\pgfqpoint{5.742259in}{3.480163in}}{\pgfqpoint{5.747845in}{3.477849in}}{\pgfqpoint{5.753669in}{3.477849in}}%
\pgfpathlineto{\pgfqpoint{5.753669in}{3.477849in}}%
\pgfpathclose%
\pgfusepath{stroke,fill}%
\end{pgfscope}%
\begin{pgfscope}%
\pgfpathrectangle{\pgfqpoint{0.640323in}{0.527436in}}{\pgfqpoint{9.687500in}{3.850000in}}%
\pgfusepath{clip}%
\pgfsetbuttcap%
\pgfsetroundjoin%
\definecolor{currentfill}{rgb}{0.239216,0.478431,0.992157}%
\pgfsetfillcolor{currentfill}%
\pgfsetfillopacity{0.500000}%
\pgfsetlinewidth{1.003750pt}%
\definecolor{currentstroke}{rgb}{0.239216,0.478431,0.992157}%
\pgfsetstrokecolor{currentstroke}%
\pgfsetstrokeopacity{0.500000}%
\pgfsetdash{{3.700000pt}{1.600000pt}}{0.000000pt}%
\pgfpathmoveto{\pgfqpoint{5.933400in}{3.521758in}}%
\pgfpathcurveto{\pgfqpoint{5.939224in}{3.521758in}}{\pgfqpoint{5.944810in}{3.524072in}}{\pgfqpoint{5.948928in}{3.528190in}}%
\pgfpathcurveto{\pgfqpoint{5.953046in}{3.532308in}}{\pgfqpoint{5.955360in}{3.537894in}}{\pgfqpoint{5.955360in}{3.543718in}}%
\pgfpathcurveto{\pgfqpoint{5.955360in}{3.549542in}}{\pgfqpoint{5.953046in}{3.555128in}}{\pgfqpoint{5.948928in}{3.559247in}}%
\pgfpathcurveto{\pgfqpoint{5.944810in}{3.563365in}}{\pgfqpoint{5.939224in}{3.565679in}}{\pgfqpoint{5.933400in}{3.565679in}}%
\pgfpathcurveto{\pgfqpoint{5.927576in}{3.565679in}}{\pgfqpoint{5.921990in}{3.563365in}}{\pgfqpoint{5.917872in}{3.559247in}}%
\pgfpathcurveto{\pgfqpoint{5.913754in}{3.555128in}}{\pgfqpoint{5.911440in}{3.549542in}}{\pgfqpoint{5.911440in}{3.543718in}}%
\pgfpathcurveto{\pgfqpoint{5.911440in}{3.537894in}}{\pgfqpoint{5.913754in}{3.532308in}}{\pgfqpoint{5.917872in}{3.528190in}}%
\pgfpathcurveto{\pgfqpoint{5.921990in}{3.524072in}}{\pgfqpoint{5.927576in}{3.521758in}}{\pgfqpoint{5.933400in}{3.521758in}}%
\pgfpathlineto{\pgfqpoint{5.933400in}{3.521758in}}%
\pgfpathclose%
\pgfusepath{stroke,fill}%
\end{pgfscope}%
\begin{pgfscope}%
\pgfpathrectangle{\pgfqpoint{0.640323in}{0.527436in}}{\pgfqpoint{9.687500in}{3.850000in}}%
\pgfusepath{clip}%
\pgfsetbuttcap%
\pgfsetroundjoin%
\definecolor{currentfill}{rgb}{0.239216,0.478431,0.992157}%
\pgfsetfillcolor{currentfill}%
\pgfsetfillopacity{0.500000}%
\pgfsetlinewidth{1.003750pt}%
\definecolor{currentstroke}{rgb}{0.239216,0.478431,0.992157}%
\pgfsetstrokecolor{currentstroke}%
\pgfsetstrokeopacity{0.500000}%
\pgfsetdash{{3.700000pt}{1.600000pt}}{0.000000pt}%
\pgfpathmoveto{\pgfqpoint{6.113131in}{3.551341in}}%
\pgfpathcurveto{\pgfqpoint{6.118955in}{3.551341in}}{\pgfqpoint{6.124541in}{3.553655in}}{\pgfqpoint{6.128659in}{3.557773in}}%
\pgfpathcurveto{\pgfqpoint{6.132777in}{3.561891in}}{\pgfqpoint{6.135091in}{3.567477in}}{\pgfqpoint{6.135091in}{3.573301in}}%
\pgfpathcurveto{\pgfqpoint{6.135091in}{3.579125in}}{\pgfqpoint{6.132777in}{3.584711in}}{\pgfqpoint{6.128659in}{3.588829in}}%
\pgfpathcurveto{\pgfqpoint{6.124541in}{3.592948in}}{\pgfqpoint{6.118955in}{3.595261in}}{\pgfqpoint{6.113131in}{3.595261in}}%
\pgfpathcurveto{\pgfqpoint{6.107307in}{3.595261in}}{\pgfqpoint{6.101721in}{3.592948in}}{\pgfqpoint{6.097603in}{3.588829in}}%
\pgfpathcurveto{\pgfqpoint{6.093485in}{3.584711in}}{\pgfqpoint{6.091171in}{3.579125in}}{\pgfqpoint{6.091171in}{3.573301in}}%
\pgfpathcurveto{\pgfqpoint{6.091171in}{3.567477in}}{\pgfqpoint{6.093485in}{3.561891in}}{\pgfqpoint{6.097603in}{3.557773in}}%
\pgfpathcurveto{\pgfqpoint{6.101721in}{3.553655in}}{\pgfqpoint{6.107307in}{3.551341in}}{\pgfqpoint{6.113131in}{3.551341in}}%
\pgfpathlineto{\pgfqpoint{6.113131in}{3.551341in}}%
\pgfpathclose%
\pgfusepath{stroke,fill}%
\end{pgfscope}%
\begin{pgfscope}%
\pgfpathrectangle{\pgfqpoint{0.640323in}{0.527436in}}{\pgfqpoint{9.687500in}{3.850000in}}%
\pgfusepath{clip}%
\pgfsetbuttcap%
\pgfsetroundjoin%
\definecolor{currentfill}{rgb}{0.239216,0.478431,0.992157}%
\pgfsetfillcolor{currentfill}%
\pgfsetfillopacity{0.500000}%
\pgfsetlinewidth{1.003750pt}%
\definecolor{currentstroke}{rgb}{0.239216,0.478431,0.992157}%
\pgfsetstrokecolor{currentstroke}%
\pgfsetstrokeopacity{0.500000}%
\pgfsetdash{{3.700000pt}{1.600000pt}}{0.000000pt}%
\pgfpathmoveto{\pgfqpoint{6.292862in}{3.598944in}}%
\pgfpathcurveto{\pgfqpoint{6.298686in}{3.598944in}}{\pgfqpoint{6.304272in}{3.601258in}}{\pgfqpoint{6.308390in}{3.605376in}}%
\pgfpathcurveto{\pgfqpoint{6.312508in}{3.609495in}}{\pgfqpoint{6.314822in}{3.615081in}}{\pgfqpoint{6.314822in}{3.620905in}}%
\pgfpathcurveto{\pgfqpoint{6.314822in}{3.626729in}}{\pgfqpoint{6.312508in}{3.632315in}}{\pgfqpoint{6.308390in}{3.636433in}}%
\pgfpathcurveto{\pgfqpoint{6.304272in}{3.640551in}}{\pgfqpoint{6.298686in}{3.642865in}}{\pgfqpoint{6.292862in}{3.642865in}}%
\pgfpathcurveto{\pgfqpoint{6.287038in}{3.642865in}}{\pgfqpoint{6.281452in}{3.640551in}}{\pgfqpoint{6.277334in}{3.636433in}}%
\pgfpathcurveto{\pgfqpoint{6.273216in}{3.632315in}}{\pgfqpoint{6.270902in}{3.626729in}}{\pgfqpoint{6.270902in}{3.620905in}}%
\pgfpathcurveto{\pgfqpoint{6.270902in}{3.615081in}}{\pgfqpoint{6.273216in}{3.609495in}}{\pgfqpoint{6.277334in}{3.605376in}}%
\pgfpathcurveto{\pgfqpoint{6.281452in}{3.601258in}}{\pgfqpoint{6.287038in}{3.598944in}}{\pgfqpoint{6.292862in}{3.598944in}}%
\pgfpathlineto{\pgfqpoint{6.292862in}{3.598944in}}%
\pgfpathclose%
\pgfusepath{stroke,fill}%
\end{pgfscope}%
\begin{pgfscope}%
\pgfpathrectangle{\pgfqpoint{0.640323in}{0.527436in}}{\pgfqpoint{9.687500in}{3.850000in}}%
\pgfusepath{clip}%
\pgfsetbuttcap%
\pgfsetroundjoin%
\definecolor{currentfill}{rgb}{0.239216,0.478431,0.992157}%
\pgfsetfillcolor{currentfill}%
\pgfsetfillopacity{0.500000}%
\pgfsetlinewidth{1.003750pt}%
\definecolor{currentstroke}{rgb}{0.239216,0.478431,0.992157}%
\pgfsetstrokecolor{currentstroke}%
\pgfsetstrokeopacity{0.500000}%
\pgfsetdash{{3.700000pt}{1.600000pt}}{0.000000pt}%
\pgfpathmoveto{\pgfqpoint{6.472593in}{3.627881in}}%
\pgfpathcurveto{\pgfqpoint{6.478417in}{3.627881in}}{\pgfqpoint{6.484003in}{3.630195in}}{\pgfqpoint{6.488121in}{3.634313in}}%
\pgfpathcurveto{\pgfqpoint{6.492239in}{3.638432in}}{\pgfqpoint{6.494553in}{3.644018in}}{\pgfqpoint{6.494553in}{3.649842in}}%
\pgfpathcurveto{\pgfqpoint{6.494553in}{3.655666in}}{\pgfqpoint{6.492239in}{3.661252in}}{\pgfqpoint{6.488121in}{3.665370in}}%
\pgfpathcurveto{\pgfqpoint{6.484003in}{3.669488in}}{\pgfqpoint{6.478417in}{3.671802in}}{\pgfqpoint{6.472593in}{3.671802in}}%
\pgfpathcurveto{\pgfqpoint{6.466769in}{3.671802in}}{\pgfqpoint{6.461183in}{3.669488in}}{\pgfqpoint{6.457065in}{3.665370in}}%
\pgfpathcurveto{\pgfqpoint{6.452947in}{3.661252in}}{\pgfqpoint{6.450633in}{3.655666in}}{\pgfqpoint{6.450633in}{3.649842in}}%
\pgfpathcurveto{\pgfqpoint{6.450633in}{3.644018in}}{\pgfqpoint{6.452947in}{3.638432in}}{\pgfqpoint{6.457065in}{3.634313in}}%
\pgfpathcurveto{\pgfqpoint{6.461183in}{3.630195in}}{\pgfqpoint{6.466769in}{3.627881in}}{\pgfqpoint{6.472593in}{3.627881in}}%
\pgfpathlineto{\pgfqpoint{6.472593in}{3.627881in}}%
\pgfpathclose%
\pgfusepath{stroke,fill}%
\end{pgfscope}%
\begin{pgfscope}%
\pgfpathrectangle{\pgfqpoint{0.640323in}{0.527436in}}{\pgfqpoint{9.687500in}{3.850000in}}%
\pgfusepath{clip}%
\pgfsetbuttcap%
\pgfsetroundjoin%
\definecolor{currentfill}{rgb}{0.239216,0.478431,0.992157}%
\pgfsetfillcolor{currentfill}%
\pgfsetfillopacity{0.500000}%
\pgfsetlinewidth{1.003750pt}%
\definecolor{currentstroke}{rgb}{0.239216,0.478431,0.992157}%
\pgfsetstrokecolor{currentstroke}%
\pgfsetstrokeopacity{0.500000}%
\pgfsetdash{{3.700000pt}{1.600000pt}}{0.000000pt}%
\pgfpathmoveto{\pgfqpoint{6.652324in}{3.649758in}}%
\pgfpathcurveto{\pgfqpoint{6.658148in}{3.649758in}}{\pgfqpoint{6.663734in}{3.652072in}}{\pgfqpoint{6.667852in}{3.656190in}}%
\pgfpathcurveto{\pgfqpoint{6.671970in}{3.660308in}}{\pgfqpoint{6.674284in}{3.665894in}}{\pgfqpoint{6.674284in}{3.671718in}}%
\pgfpathcurveto{\pgfqpoint{6.674284in}{3.677542in}}{\pgfqpoint{6.671970in}{3.683129in}}{\pgfqpoint{6.667852in}{3.687247in}}%
\pgfpathcurveto{\pgfqpoint{6.663734in}{3.691365in}}{\pgfqpoint{6.658148in}{3.693679in}}{\pgfqpoint{6.652324in}{3.693679in}}%
\pgfpathcurveto{\pgfqpoint{6.646500in}{3.693679in}}{\pgfqpoint{6.640914in}{3.691365in}}{\pgfqpoint{6.636796in}{3.687247in}}%
\pgfpathcurveto{\pgfqpoint{6.632678in}{3.683129in}}{\pgfqpoint{6.630364in}{3.677542in}}{\pgfqpoint{6.630364in}{3.671718in}}%
\pgfpathcurveto{\pgfqpoint{6.630364in}{3.665894in}}{\pgfqpoint{6.632678in}{3.660308in}}{\pgfqpoint{6.636796in}{3.656190in}}%
\pgfpathcurveto{\pgfqpoint{6.640914in}{3.652072in}}{\pgfqpoint{6.646500in}{3.649758in}}{\pgfqpoint{6.652324in}{3.649758in}}%
\pgfpathlineto{\pgfqpoint{6.652324in}{3.649758in}}%
\pgfpathclose%
\pgfusepath{stroke,fill}%
\end{pgfscope}%
\begin{pgfscope}%
\pgfpathrectangle{\pgfqpoint{0.640323in}{0.527436in}}{\pgfqpoint{9.687500in}{3.850000in}}%
\pgfusepath{clip}%
\pgfsetbuttcap%
\pgfsetroundjoin%
\definecolor{currentfill}{rgb}{0.239216,0.478431,0.992157}%
\pgfsetfillcolor{currentfill}%
\pgfsetfillopacity{0.500000}%
\pgfsetlinewidth{1.003750pt}%
\definecolor{currentstroke}{rgb}{0.239216,0.478431,0.992157}%
\pgfsetstrokecolor{currentstroke}%
\pgfsetstrokeopacity{0.500000}%
\pgfsetdash{{3.700000pt}{1.600000pt}}{0.000000pt}%
\pgfpathmoveto{\pgfqpoint{6.832055in}{3.689544in}}%
\pgfpathcurveto{\pgfqpoint{6.837879in}{3.689544in}}{\pgfqpoint{6.843465in}{3.691857in}}{\pgfqpoint{6.847583in}{3.695976in}}%
\pgfpathcurveto{\pgfqpoint{6.851701in}{3.700094in}}{\pgfqpoint{6.854015in}{3.705680in}}{\pgfqpoint{6.854015in}{3.711504in}}%
\pgfpathcurveto{\pgfqpoint{6.854015in}{3.717328in}}{\pgfqpoint{6.851701in}{3.722914in}}{\pgfqpoint{6.847583in}{3.727032in}}%
\pgfpathcurveto{\pgfqpoint{6.843465in}{3.731150in}}{\pgfqpoint{6.837879in}{3.733464in}}{\pgfqpoint{6.832055in}{3.733464in}}%
\pgfpathcurveto{\pgfqpoint{6.826231in}{3.733464in}}{\pgfqpoint{6.820645in}{3.731150in}}{\pgfqpoint{6.816527in}{3.727032in}}%
\pgfpathcurveto{\pgfqpoint{6.812408in}{3.722914in}}{\pgfqpoint{6.810095in}{3.717328in}}{\pgfqpoint{6.810095in}{3.711504in}}%
\pgfpathcurveto{\pgfqpoint{6.810095in}{3.705680in}}{\pgfqpoint{6.812408in}{3.700094in}}{\pgfqpoint{6.816527in}{3.695976in}}%
\pgfpathcurveto{\pgfqpoint{6.820645in}{3.691857in}}{\pgfqpoint{6.826231in}{3.689544in}}{\pgfqpoint{6.832055in}{3.689544in}}%
\pgfpathlineto{\pgfqpoint{6.832055in}{3.689544in}}%
\pgfpathclose%
\pgfusepath{stroke,fill}%
\end{pgfscope}%
\begin{pgfscope}%
\pgfpathrectangle{\pgfqpoint{0.640323in}{0.527436in}}{\pgfqpoint{9.687500in}{3.850000in}}%
\pgfusepath{clip}%
\pgfsetbuttcap%
\pgfsetroundjoin%
\definecolor{currentfill}{rgb}{0.239216,0.478431,0.992157}%
\pgfsetfillcolor{currentfill}%
\pgfsetfillopacity{0.500000}%
\pgfsetlinewidth{1.003750pt}%
\definecolor{currentstroke}{rgb}{0.239216,0.478431,0.992157}%
\pgfsetstrokecolor{currentstroke}%
\pgfsetstrokeopacity{0.500000}%
\pgfsetdash{{3.700000pt}{1.600000pt}}{0.000000pt}%
\pgfpathmoveto{\pgfqpoint{7.011786in}{3.711458in}}%
\pgfpathcurveto{\pgfqpoint{7.017610in}{3.711458in}}{\pgfqpoint{7.023196in}{3.713771in}}{\pgfqpoint{7.027314in}{3.717890in}}%
\pgfpathcurveto{\pgfqpoint{7.031432in}{3.722008in}}{\pgfqpoint{7.033746in}{3.727594in}}{\pgfqpoint{7.033746in}{3.733418in}}%
\pgfpathcurveto{\pgfqpoint{7.033746in}{3.739242in}}{\pgfqpoint{7.031432in}{3.744828in}}{\pgfqpoint{7.027314in}{3.748946in}}%
\pgfpathcurveto{\pgfqpoint{7.023196in}{3.753064in}}{\pgfqpoint{7.017610in}{3.755378in}}{\pgfqpoint{7.011786in}{3.755378in}}%
\pgfpathcurveto{\pgfqpoint{7.005962in}{3.755378in}}{\pgfqpoint{7.000376in}{3.753064in}}{\pgfqpoint{6.996258in}{3.748946in}}%
\pgfpathcurveto{\pgfqpoint{6.992139in}{3.744828in}}{\pgfqpoint{6.989826in}{3.739242in}}{\pgfqpoint{6.989826in}{3.733418in}}%
\pgfpathcurveto{\pgfqpoint{6.989826in}{3.727594in}}{\pgfqpoint{6.992139in}{3.722008in}}{\pgfqpoint{6.996258in}{3.717890in}}%
\pgfpathcurveto{\pgfqpoint{7.000376in}{3.713771in}}{\pgfqpoint{7.005962in}{3.711458in}}{\pgfqpoint{7.011786in}{3.711458in}}%
\pgfpathlineto{\pgfqpoint{7.011786in}{3.711458in}}%
\pgfpathclose%
\pgfusepath{stroke,fill}%
\end{pgfscope}%
\begin{pgfscope}%
\pgfpathrectangle{\pgfqpoint{0.640323in}{0.527436in}}{\pgfqpoint{9.687500in}{3.850000in}}%
\pgfusepath{clip}%
\pgfsetbuttcap%
\pgfsetroundjoin%
\definecolor{currentfill}{rgb}{0.239216,0.478431,0.992157}%
\pgfsetfillcolor{currentfill}%
\pgfsetfillopacity{0.500000}%
\pgfsetlinewidth{1.003750pt}%
\definecolor{currentstroke}{rgb}{0.239216,0.478431,0.992157}%
\pgfsetstrokecolor{currentstroke}%
\pgfsetstrokeopacity{0.500000}%
\pgfsetdash{{3.700000pt}{1.600000pt}}{0.000000pt}%
\pgfpathmoveto{\pgfqpoint{7.191517in}{3.742431in}}%
\pgfpathcurveto{\pgfqpoint{7.197341in}{3.742431in}}{\pgfqpoint{7.202927in}{3.744745in}}{\pgfqpoint{7.207045in}{3.748863in}}%
\pgfpathcurveto{\pgfqpoint{7.211163in}{3.752982in}}{\pgfqpoint{7.213477in}{3.758568in}}{\pgfqpoint{7.213477in}{3.764392in}}%
\pgfpathcurveto{\pgfqpoint{7.213477in}{3.770216in}}{\pgfqpoint{7.211163in}{3.775802in}}{\pgfqpoint{7.207045in}{3.779920in}}%
\pgfpathcurveto{\pgfqpoint{7.202927in}{3.784038in}}{\pgfqpoint{7.197341in}{3.786352in}}{\pgfqpoint{7.191517in}{3.786352in}}%
\pgfpathcurveto{\pgfqpoint{7.185693in}{3.786352in}}{\pgfqpoint{7.180107in}{3.784038in}}{\pgfqpoint{7.175989in}{3.779920in}}%
\pgfpathcurveto{\pgfqpoint{7.171870in}{3.775802in}}{\pgfqpoint{7.169557in}{3.770216in}}{\pgfqpoint{7.169557in}{3.764392in}}%
\pgfpathcurveto{\pgfqpoint{7.169557in}{3.758568in}}{\pgfqpoint{7.171870in}{3.752982in}}{\pgfqpoint{7.175989in}{3.748863in}}%
\pgfpathcurveto{\pgfqpoint{7.180107in}{3.744745in}}{\pgfqpoint{7.185693in}{3.742431in}}{\pgfqpoint{7.191517in}{3.742431in}}%
\pgfpathlineto{\pgfqpoint{7.191517in}{3.742431in}}%
\pgfpathclose%
\pgfusepath{stroke,fill}%
\end{pgfscope}%
\begin{pgfscope}%
\pgfpathrectangle{\pgfqpoint{0.640323in}{0.527436in}}{\pgfqpoint{9.687500in}{3.850000in}}%
\pgfusepath{clip}%
\pgfsetbuttcap%
\pgfsetroundjoin%
\definecolor{currentfill}{rgb}{0.239216,0.478431,0.992157}%
\pgfsetfillcolor{currentfill}%
\pgfsetfillopacity{0.500000}%
\pgfsetlinewidth{1.003750pt}%
\definecolor{currentstroke}{rgb}{0.239216,0.478431,0.992157}%
\pgfsetstrokecolor{currentstroke}%
\pgfsetstrokeopacity{0.500000}%
\pgfsetdash{{3.700000pt}{1.600000pt}}{0.000000pt}%
\pgfpathmoveto{\pgfqpoint{7.371248in}{3.767891in}}%
\pgfpathcurveto{\pgfqpoint{7.377072in}{3.767891in}}{\pgfqpoint{7.382658in}{3.770205in}}{\pgfqpoint{7.386776in}{3.774323in}}%
\pgfpathcurveto{\pgfqpoint{7.390894in}{3.778441in}}{\pgfqpoint{7.393208in}{3.784027in}}{\pgfqpoint{7.393208in}{3.789851in}}%
\pgfpathcurveto{\pgfqpoint{7.393208in}{3.795675in}}{\pgfqpoint{7.390894in}{3.801261in}}{\pgfqpoint{7.386776in}{3.805380in}}%
\pgfpathcurveto{\pgfqpoint{7.382658in}{3.809498in}}{\pgfqpoint{7.377072in}{3.811812in}}{\pgfqpoint{7.371248in}{3.811812in}}%
\pgfpathcurveto{\pgfqpoint{7.365424in}{3.811812in}}{\pgfqpoint{7.359838in}{3.809498in}}{\pgfqpoint{7.355720in}{3.805380in}}%
\pgfpathcurveto{\pgfqpoint{7.351601in}{3.801261in}}{\pgfqpoint{7.349288in}{3.795675in}}{\pgfqpoint{7.349288in}{3.789851in}}%
\pgfpathcurveto{\pgfqpoint{7.349288in}{3.784027in}}{\pgfqpoint{7.351601in}{3.778441in}}{\pgfqpoint{7.355720in}{3.774323in}}%
\pgfpathcurveto{\pgfqpoint{7.359838in}{3.770205in}}{\pgfqpoint{7.365424in}{3.767891in}}{\pgfqpoint{7.371248in}{3.767891in}}%
\pgfpathlineto{\pgfqpoint{7.371248in}{3.767891in}}%
\pgfpathclose%
\pgfusepath{stroke,fill}%
\end{pgfscope}%
\begin{pgfscope}%
\pgfpathrectangle{\pgfqpoint{0.640323in}{0.527436in}}{\pgfqpoint{9.687500in}{3.850000in}}%
\pgfusepath{clip}%
\pgfsetbuttcap%
\pgfsetroundjoin%
\definecolor{currentfill}{rgb}{0.239216,0.478431,0.992157}%
\pgfsetfillcolor{currentfill}%
\pgfsetfillopacity{0.500000}%
\pgfsetlinewidth{1.003750pt}%
\definecolor{currentstroke}{rgb}{0.239216,0.478431,0.992157}%
\pgfsetstrokecolor{currentstroke}%
\pgfsetstrokeopacity{0.500000}%
\pgfsetdash{{3.700000pt}{1.600000pt}}{0.000000pt}%
\pgfpathmoveto{\pgfqpoint{7.550979in}{3.788010in}}%
\pgfpathcurveto{\pgfqpoint{7.556803in}{3.788010in}}{\pgfqpoint{7.562389in}{3.790324in}}{\pgfqpoint{7.566507in}{3.794442in}}%
\pgfpathcurveto{\pgfqpoint{7.570625in}{3.798561in}}{\pgfqpoint{7.572939in}{3.804147in}}{\pgfqpoint{7.572939in}{3.809971in}}%
\pgfpathcurveto{\pgfqpoint{7.572939in}{3.815795in}}{\pgfqpoint{7.570625in}{3.821381in}}{\pgfqpoint{7.566507in}{3.825499in}}%
\pgfpathcurveto{\pgfqpoint{7.562389in}{3.829617in}}{\pgfqpoint{7.556803in}{3.831931in}}{\pgfqpoint{7.550979in}{3.831931in}}%
\pgfpathcurveto{\pgfqpoint{7.545155in}{3.831931in}}{\pgfqpoint{7.539569in}{3.829617in}}{\pgfqpoint{7.535451in}{3.825499in}}%
\pgfpathcurveto{\pgfqpoint{7.531332in}{3.821381in}}{\pgfqpoint{7.529019in}{3.815795in}}{\pgfqpoint{7.529019in}{3.809971in}}%
\pgfpathcurveto{\pgfqpoint{7.529019in}{3.804147in}}{\pgfqpoint{7.531332in}{3.798561in}}{\pgfqpoint{7.535451in}{3.794442in}}%
\pgfpathcurveto{\pgfqpoint{7.539569in}{3.790324in}}{\pgfqpoint{7.545155in}{3.788010in}}{\pgfqpoint{7.550979in}{3.788010in}}%
\pgfpathlineto{\pgfqpoint{7.550979in}{3.788010in}}%
\pgfpathclose%
\pgfusepath{stroke,fill}%
\end{pgfscope}%
\begin{pgfscope}%
\pgfpathrectangle{\pgfqpoint{0.640323in}{0.527436in}}{\pgfqpoint{9.687500in}{3.850000in}}%
\pgfusepath{clip}%
\pgfsetbuttcap%
\pgfsetroundjoin%
\definecolor{currentfill}{rgb}{0.239216,0.478431,0.992157}%
\pgfsetfillcolor{currentfill}%
\pgfsetfillopacity{0.500000}%
\pgfsetlinewidth{1.003750pt}%
\definecolor{currentstroke}{rgb}{0.239216,0.478431,0.992157}%
\pgfsetstrokecolor{currentstroke}%
\pgfsetstrokeopacity{0.500000}%
\pgfsetdash{{3.700000pt}{1.600000pt}}{0.000000pt}%
\pgfpathmoveto{\pgfqpoint{7.730710in}{3.813458in}}%
\pgfpathcurveto{\pgfqpoint{7.736534in}{3.813458in}}{\pgfqpoint{7.742120in}{3.815772in}}{\pgfqpoint{7.746238in}{3.819890in}}%
\pgfpathcurveto{\pgfqpoint{7.750356in}{3.824008in}}{\pgfqpoint{7.752670in}{3.829594in}}{\pgfqpoint{7.752670in}{3.835418in}}%
\pgfpathcurveto{\pgfqpoint{7.752670in}{3.841242in}}{\pgfqpoint{7.750356in}{3.846828in}}{\pgfqpoint{7.746238in}{3.850946in}}%
\pgfpathcurveto{\pgfqpoint{7.742120in}{3.855064in}}{\pgfqpoint{7.736534in}{3.857378in}}{\pgfqpoint{7.730710in}{3.857378in}}%
\pgfpathcurveto{\pgfqpoint{7.724886in}{3.857378in}}{\pgfqpoint{7.719300in}{3.855064in}}{\pgfqpoint{7.715182in}{3.850946in}}%
\pgfpathcurveto{\pgfqpoint{7.711063in}{3.846828in}}{\pgfqpoint{7.708750in}{3.841242in}}{\pgfqpoint{7.708750in}{3.835418in}}%
\pgfpathcurveto{\pgfqpoint{7.708750in}{3.829594in}}{\pgfqpoint{7.711063in}{3.824008in}}{\pgfqpoint{7.715182in}{3.819890in}}%
\pgfpathcurveto{\pgfqpoint{7.719300in}{3.815772in}}{\pgfqpoint{7.724886in}{3.813458in}}{\pgfqpoint{7.730710in}{3.813458in}}%
\pgfpathlineto{\pgfqpoint{7.730710in}{3.813458in}}%
\pgfpathclose%
\pgfusepath{stroke,fill}%
\end{pgfscope}%
\begin{pgfscope}%
\pgfpathrectangle{\pgfqpoint{0.640323in}{0.527436in}}{\pgfqpoint{9.687500in}{3.850000in}}%
\pgfusepath{clip}%
\pgfsetbuttcap%
\pgfsetroundjoin%
\definecolor{currentfill}{rgb}{0.239216,0.478431,0.992157}%
\pgfsetfillcolor{currentfill}%
\pgfsetfillopacity{0.500000}%
\pgfsetlinewidth{1.003750pt}%
\definecolor{currentstroke}{rgb}{0.239216,0.478431,0.992157}%
\pgfsetstrokecolor{currentstroke}%
\pgfsetstrokeopacity{0.500000}%
\pgfsetdash{{3.700000pt}{1.600000pt}}{0.000000pt}%
\pgfpathmoveto{\pgfqpoint{7.910441in}{3.828479in}}%
\pgfpathcurveto{\pgfqpoint{7.916265in}{3.828479in}}{\pgfqpoint{7.921851in}{3.830793in}}{\pgfqpoint{7.925969in}{3.834911in}}%
\pgfpathcurveto{\pgfqpoint{7.930087in}{3.839029in}}{\pgfqpoint{7.932401in}{3.844615in}}{\pgfqpoint{7.932401in}{3.850439in}}%
\pgfpathcurveto{\pgfqpoint{7.932401in}{3.856263in}}{\pgfqpoint{7.930087in}{3.861849in}}{\pgfqpoint{7.925969in}{3.865967in}}%
\pgfpathcurveto{\pgfqpoint{7.921851in}{3.870086in}}{\pgfqpoint{7.916265in}{3.872399in}}{\pgfqpoint{7.910441in}{3.872399in}}%
\pgfpathcurveto{\pgfqpoint{7.904617in}{3.872399in}}{\pgfqpoint{7.899031in}{3.870086in}}{\pgfqpoint{7.894913in}{3.865967in}}%
\pgfpathcurveto{\pgfqpoint{7.890794in}{3.861849in}}{\pgfqpoint{7.888481in}{3.856263in}}{\pgfqpoint{7.888481in}{3.850439in}}%
\pgfpathcurveto{\pgfqpoint{7.888481in}{3.844615in}}{\pgfqpoint{7.890794in}{3.839029in}}{\pgfqpoint{7.894913in}{3.834911in}}%
\pgfpathcurveto{\pgfqpoint{7.899031in}{3.830793in}}{\pgfqpoint{7.904617in}{3.828479in}}{\pgfqpoint{7.910441in}{3.828479in}}%
\pgfpathlineto{\pgfqpoint{7.910441in}{3.828479in}}%
\pgfpathclose%
\pgfusepath{stroke,fill}%
\end{pgfscope}%
\begin{pgfscope}%
\pgfpathrectangle{\pgfqpoint{0.640323in}{0.527436in}}{\pgfqpoint{9.687500in}{3.850000in}}%
\pgfusepath{clip}%
\pgfsetbuttcap%
\pgfsetroundjoin%
\definecolor{currentfill}{rgb}{0.239216,0.478431,0.992157}%
\pgfsetfillcolor{currentfill}%
\pgfsetfillopacity{0.500000}%
\pgfsetlinewidth{1.003750pt}%
\definecolor{currentstroke}{rgb}{0.239216,0.478431,0.992157}%
\pgfsetstrokecolor{currentstroke}%
\pgfsetstrokeopacity{0.500000}%
\pgfsetdash{{3.700000pt}{1.600000pt}}{0.000000pt}%
\pgfpathmoveto{\pgfqpoint{8.090172in}{3.858683in}}%
\pgfpathcurveto{\pgfqpoint{8.095996in}{3.858683in}}{\pgfqpoint{8.101582in}{3.860997in}}{\pgfqpoint{8.105700in}{3.865115in}}%
\pgfpathcurveto{\pgfqpoint{8.109818in}{3.869233in}}{\pgfqpoint{8.112132in}{3.874819in}}{\pgfqpoint{8.112132in}{3.880643in}}%
\pgfpathcurveto{\pgfqpoint{8.112132in}{3.886467in}}{\pgfqpoint{8.109818in}{3.892053in}}{\pgfqpoint{8.105700in}{3.896171in}}%
\pgfpathcurveto{\pgfqpoint{8.101582in}{3.900289in}}{\pgfqpoint{8.095996in}{3.902603in}}{\pgfqpoint{8.090172in}{3.902603in}}%
\pgfpathcurveto{\pgfqpoint{8.084348in}{3.902603in}}{\pgfqpoint{8.078762in}{3.900289in}}{\pgfqpoint{8.074644in}{3.896171in}}%
\pgfpathcurveto{\pgfqpoint{8.070525in}{3.892053in}}{\pgfqpoint{8.068211in}{3.886467in}}{\pgfqpoint{8.068211in}{3.880643in}}%
\pgfpathcurveto{\pgfqpoint{8.068211in}{3.874819in}}{\pgfqpoint{8.070525in}{3.869233in}}{\pgfqpoint{8.074644in}{3.865115in}}%
\pgfpathcurveto{\pgfqpoint{8.078762in}{3.860997in}}{\pgfqpoint{8.084348in}{3.858683in}}{\pgfqpoint{8.090172in}{3.858683in}}%
\pgfpathlineto{\pgfqpoint{8.090172in}{3.858683in}}%
\pgfpathclose%
\pgfusepath{stroke,fill}%
\end{pgfscope}%
\begin{pgfscope}%
\pgfpathrectangle{\pgfqpoint{0.640323in}{0.527436in}}{\pgfqpoint{9.687500in}{3.850000in}}%
\pgfusepath{clip}%
\pgfsetbuttcap%
\pgfsetroundjoin%
\definecolor{currentfill}{rgb}{0.239216,0.478431,0.992157}%
\pgfsetfillcolor{currentfill}%
\pgfsetfillopacity{0.500000}%
\pgfsetlinewidth{1.003750pt}%
\definecolor{currentstroke}{rgb}{0.239216,0.478431,0.992157}%
\pgfsetstrokecolor{currentstroke}%
\pgfsetstrokeopacity{0.500000}%
\pgfsetdash{{3.700000pt}{1.600000pt}}{0.000000pt}%
\pgfpathmoveto{\pgfqpoint{8.269903in}{3.873903in}}%
\pgfpathcurveto{\pgfqpoint{8.275727in}{3.873903in}}{\pgfqpoint{8.281313in}{3.876217in}}{\pgfqpoint{8.285431in}{3.880335in}}%
\pgfpathcurveto{\pgfqpoint{8.289549in}{3.884453in}}{\pgfqpoint{8.291863in}{3.890039in}}{\pgfqpoint{8.291863in}{3.895863in}}%
\pgfpathcurveto{\pgfqpoint{8.291863in}{3.901687in}}{\pgfqpoint{8.289549in}{3.907273in}}{\pgfqpoint{8.285431in}{3.911391in}}%
\pgfpathcurveto{\pgfqpoint{8.281313in}{3.915509in}}{\pgfqpoint{8.275727in}{3.917823in}}{\pgfqpoint{8.269903in}{3.917823in}}%
\pgfpathcurveto{\pgfqpoint{8.264079in}{3.917823in}}{\pgfqpoint{8.258493in}{3.915509in}}{\pgfqpoint{8.254374in}{3.911391in}}%
\pgfpathcurveto{\pgfqpoint{8.250256in}{3.907273in}}{\pgfqpoint{8.247942in}{3.901687in}}{\pgfqpoint{8.247942in}{3.895863in}}%
\pgfpathcurveto{\pgfqpoint{8.247942in}{3.890039in}}{\pgfqpoint{8.250256in}{3.884453in}}{\pgfqpoint{8.254374in}{3.880335in}}%
\pgfpathcurveto{\pgfqpoint{8.258493in}{3.876217in}}{\pgfqpoint{8.264079in}{3.873903in}}{\pgfqpoint{8.269903in}{3.873903in}}%
\pgfpathlineto{\pgfqpoint{8.269903in}{3.873903in}}%
\pgfpathclose%
\pgfusepath{stroke,fill}%
\end{pgfscope}%
\begin{pgfscope}%
\pgfpathrectangle{\pgfqpoint{0.640323in}{0.527436in}}{\pgfqpoint{9.687500in}{3.850000in}}%
\pgfusepath{clip}%
\pgfsetbuttcap%
\pgfsetroundjoin%
\definecolor{currentfill}{rgb}{0.239216,0.478431,0.992157}%
\pgfsetfillcolor{currentfill}%
\pgfsetfillopacity{0.500000}%
\pgfsetlinewidth{1.003750pt}%
\definecolor{currentstroke}{rgb}{0.239216,0.478431,0.992157}%
\pgfsetstrokecolor{currentstroke}%
\pgfsetstrokeopacity{0.500000}%
\pgfsetdash{{3.700000pt}{1.600000pt}}{0.000000pt}%
\pgfpathmoveto{\pgfqpoint{8.449634in}{3.892724in}}%
\pgfpathcurveto{\pgfqpoint{8.455458in}{3.892724in}}{\pgfqpoint{8.461044in}{3.895038in}}{\pgfqpoint{8.465162in}{3.899156in}}%
\pgfpathcurveto{\pgfqpoint{8.469280in}{3.903274in}}{\pgfqpoint{8.471594in}{3.908861in}}{\pgfqpoint{8.471594in}{3.914684in}}%
\pgfpathcurveto{\pgfqpoint{8.471594in}{3.920508in}}{\pgfqpoint{8.469280in}{3.926095in}}{\pgfqpoint{8.465162in}{3.930213in}}%
\pgfpathcurveto{\pgfqpoint{8.461044in}{3.934331in}}{\pgfqpoint{8.455458in}{3.936645in}}{\pgfqpoint{8.449634in}{3.936645in}}%
\pgfpathcurveto{\pgfqpoint{8.443810in}{3.936645in}}{\pgfqpoint{8.438224in}{3.934331in}}{\pgfqpoint{8.434105in}{3.930213in}}%
\pgfpathcurveto{\pgfqpoint{8.429987in}{3.926095in}}{\pgfqpoint{8.427673in}{3.920508in}}{\pgfqpoint{8.427673in}{3.914684in}}%
\pgfpathcurveto{\pgfqpoint{8.427673in}{3.908861in}}{\pgfqpoint{8.429987in}{3.903274in}}{\pgfqpoint{8.434105in}{3.899156in}}%
\pgfpathcurveto{\pgfqpoint{8.438224in}{3.895038in}}{\pgfqpoint{8.443810in}{3.892724in}}{\pgfqpoint{8.449634in}{3.892724in}}%
\pgfpathlineto{\pgfqpoint{8.449634in}{3.892724in}}%
\pgfpathclose%
\pgfusepath{stroke,fill}%
\end{pgfscope}%
\begin{pgfscope}%
\pgfpathrectangle{\pgfqpoint{0.640323in}{0.527436in}}{\pgfqpoint{9.687500in}{3.850000in}}%
\pgfusepath{clip}%
\pgfsetbuttcap%
\pgfsetroundjoin%
\definecolor{currentfill}{rgb}{0.239216,0.478431,0.992157}%
\pgfsetfillcolor{currentfill}%
\pgfsetfillopacity{0.500000}%
\pgfsetlinewidth{1.003750pt}%
\definecolor{currentstroke}{rgb}{0.239216,0.478431,0.992157}%
\pgfsetstrokecolor{currentstroke}%
\pgfsetstrokeopacity{0.500000}%
\pgfsetdash{{3.700000pt}{1.600000pt}}{0.000000pt}%
\pgfpathmoveto{\pgfqpoint{8.629365in}{3.907534in}}%
\pgfpathcurveto{\pgfqpoint{8.635189in}{3.907534in}}{\pgfqpoint{8.640775in}{3.909848in}}{\pgfqpoint{8.644893in}{3.913966in}}%
\pgfpathcurveto{\pgfqpoint{8.649011in}{3.918084in}}{\pgfqpoint{8.651325in}{3.923671in}}{\pgfqpoint{8.651325in}{3.929495in}}%
\pgfpathcurveto{\pgfqpoint{8.651325in}{3.935318in}}{\pgfqpoint{8.649011in}{3.940905in}}{\pgfqpoint{8.644893in}{3.945023in}}%
\pgfpathcurveto{\pgfqpoint{8.640775in}{3.949141in}}{\pgfqpoint{8.635189in}{3.951455in}}{\pgfqpoint{8.629365in}{3.951455in}}%
\pgfpathcurveto{\pgfqpoint{8.623541in}{3.951455in}}{\pgfqpoint{8.617955in}{3.949141in}}{\pgfqpoint{8.613836in}{3.945023in}}%
\pgfpathcurveto{\pgfqpoint{8.609718in}{3.940905in}}{\pgfqpoint{8.607404in}{3.935318in}}{\pgfqpoint{8.607404in}{3.929495in}}%
\pgfpathcurveto{\pgfqpoint{8.607404in}{3.923671in}}{\pgfqpoint{8.609718in}{3.918084in}}{\pgfqpoint{8.613836in}{3.913966in}}%
\pgfpathcurveto{\pgfqpoint{8.617955in}{3.909848in}}{\pgfqpoint{8.623541in}{3.907534in}}{\pgfqpoint{8.629365in}{3.907534in}}%
\pgfpathlineto{\pgfqpoint{8.629365in}{3.907534in}}%
\pgfpathclose%
\pgfusepath{stroke,fill}%
\end{pgfscope}%
\begin{pgfscope}%
\pgfpathrectangle{\pgfqpoint{0.640323in}{0.527436in}}{\pgfqpoint{9.687500in}{3.850000in}}%
\pgfusepath{clip}%
\pgfsetbuttcap%
\pgfsetroundjoin%
\definecolor{currentfill}{rgb}{0.239216,0.478431,0.992157}%
\pgfsetfillcolor{currentfill}%
\pgfsetfillopacity{0.500000}%
\pgfsetlinewidth{1.003750pt}%
\definecolor{currentstroke}{rgb}{0.239216,0.478431,0.992157}%
\pgfsetstrokecolor{currentstroke}%
\pgfsetstrokeopacity{0.500000}%
\pgfsetdash{{3.700000pt}{1.600000pt}}{0.000000pt}%
\pgfpathmoveto{\pgfqpoint{8.809096in}{3.926269in}}%
\pgfpathcurveto{\pgfqpoint{8.814920in}{3.926269in}}{\pgfqpoint{8.820506in}{3.928583in}}{\pgfqpoint{8.824624in}{3.932701in}}%
\pgfpathcurveto{\pgfqpoint{8.828742in}{3.936819in}}{\pgfqpoint{8.831056in}{3.942405in}}{\pgfqpoint{8.831056in}{3.948229in}}%
\pgfpathcurveto{\pgfqpoint{8.831056in}{3.954053in}}{\pgfqpoint{8.828742in}{3.959639in}}{\pgfqpoint{8.824624in}{3.963757in}}%
\pgfpathcurveto{\pgfqpoint{8.820506in}{3.967876in}}{\pgfqpoint{8.814920in}{3.970189in}}{\pgfqpoint{8.809096in}{3.970189in}}%
\pgfpathcurveto{\pgfqpoint{8.803272in}{3.970189in}}{\pgfqpoint{8.797686in}{3.967876in}}{\pgfqpoint{8.793567in}{3.963757in}}%
\pgfpathcurveto{\pgfqpoint{8.789449in}{3.959639in}}{\pgfqpoint{8.787135in}{3.954053in}}{\pgfqpoint{8.787135in}{3.948229in}}%
\pgfpathcurveto{\pgfqpoint{8.787135in}{3.942405in}}{\pgfqpoint{8.789449in}{3.936819in}}{\pgfqpoint{8.793567in}{3.932701in}}%
\pgfpathcurveto{\pgfqpoint{8.797686in}{3.928583in}}{\pgfqpoint{8.803272in}{3.926269in}}{\pgfqpoint{8.809096in}{3.926269in}}%
\pgfpathlineto{\pgfqpoint{8.809096in}{3.926269in}}%
\pgfpathclose%
\pgfusepath{stroke,fill}%
\end{pgfscope}%
\begin{pgfscope}%
\pgfpathrectangle{\pgfqpoint{0.640323in}{0.527436in}}{\pgfqpoint{9.687500in}{3.850000in}}%
\pgfusepath{clip}%
\pgfsetbuttcap%
\pgfsetroundjoin%
\definecolor{currentfill}{rgb}{0.239216,0.478431,0.992157}%
\pgfsetfillcolor{currentfill}%
\pgfsetfillopacity{0.500000}%
\pgfsetlinewidth{1.003750pt}%
\definecolor{currentstroke}{rgb}{0.239216,0.478431,0.992157}%
\pgfsetstrokecolor{currentstroke}%
\pgfsetstrokeopacity{0.500000}%
\pgfsetdash{{3.700000pt}{1.600000pt}}{0.000000pt}%
\pgfpathmoveto{\pgfqpoint{8.988827in}{3.945997in}}%
\pgfpathcurveto{\pgfqpoint{8.994651in}{3.945997in}}{\pgfqpoint{9.000237in}{3.948311in}}{\pgfqpoint{9.004355in}{3.952429in}}%
\pgfpathcurveto{\pgfqpoint{9.008473in}{3.956547in}}{\pgfqpoint{9.010787in}{3.962133in}}{\pgfqpoint{9.010787in}{3.967957in}}%
\pgfpathcurveto{\pgfqpoint{9.010787in}{3.973781in}}{\pgfqpoint{9.008473in}{3.979367in}}{\pgfqpoint{9.004355in}{3.983486in}}%
\pgfpathcurveto{\pgfqpoint{9.000237in}{3.987604in}}{\pgfqpoint{8.994651in}{3.989918in}}{\pgfqpoint{8.988827in}{3.989918in}}%
\pgfpathcurveto{\pgfqpoint{8.983003in}{3.989918in}}{\pgfqpoint{8.977417in}{3.987604in}}{\pgfqpoint{8.973298in}{3.983486in}}%
\pgfpathcurveto{\pgfqpoint{8.969180in}{3.979367in}}{\pgfqpoint{8.966866in}{3.973781in}}{\pgfqpoint{8.966866in}{3.967957in}}%
\pgfpathcurveto{\pgfqpoint{8.966866in}{3.962133in}}{\pgfqpoint{8.969180in}{3.956547in}}{\pgfqpoint{8.973298in}{3.952429in}}%
\pgfpathcurveto{\pgfqpoint{8.977417in}{3.948311in}}{\pgfqpoint{8.983003in}{3.945997in}}{\pgfqpoint{8.988827in}{3.945997in}}%
\pgfpathlineto{\pgfqpoint{8.988827in}{3.945997in}}%
\pgfpathclose%
\pgfusepath{stroke,fill}%
\end{pgfscope}%
\begin{pgfscope}%
\pgfpathrectangle{\pgfqpoint{0.640323in}{0.527436in}}{\pgfqpoint{9.687500in}{3.850000in}}%
\pgfusepath{clip}%
\pgfsetbuttcap%
\pgfsetroundjoin%
\definecolor{currentfill}{rgb}{0.239216,0.478431,0.992157}%
\pgfsetfillcolor{currentfill}%
\pgfsetfillopacity{0.500000}%
\pgfsetlinewidth{1.003750pt}%
\definecolor{currentstroke}{rgb}{0.239216,0.478431,0.992157}%
\pgfsetstrokecolor{currentstroke}%
\pgfsetstrokeopacity{0.500000}%
\pgfsetdash{{3.700000pt}{1.600000pt}}{0.000000pt}%
\pgfpathmoveto{\pgfqpoint{9.168558in}{3.962353in}}%
\pgfpathcurveto{\pgfqpoint{9.174382in}{3.962353in}}{\pgfqpoint{9.179968in}{3.964667in}}{\pgfqpoint{9.184086in}{3.968785in}}%
\pgfpathcurveto{\pgfqpoint{9.188204in}{3.972903in}}{\pgfqpoint{9.190518in}{3.978490in}}{\pgfqpoint{9.190518in}{3.984314in}}%
\pgfpathcurveto{\pgfqpoint{9.190518in}{3.990138in}}{\pgfqpoint{9.188204in}{3.995724in}}{\pgfqpoint{9.184086in}{3.999842in}}%
\pgfpathcurveto{\pgfqpoint{9.179968in}{4.003960in}}{\pgfqpoint{9.174382in}{4.006274in}}{\pgfqpoint{9.168558in}{4.006274in}}%
\pgfpathcurveto{\pgfqpoint{9.162734in}{4.006274in}}{\pgfqpoint{9.157148in}{4.003960in}}{\pgfqpoint{9.153029in}{3.999842in}}%
\pgfpathcurveto{\pgfqpoint{9.148911in}{3.995724in}}{\pgfqpoint{9.146597in}{3.990138in}}{\pgfqpoint{9.146597in}{3.984314in}}%
\pgfpathcurveto{\pgfqpoint{9.146597in}{3.978490in}}{\pgfqpoint{9.148911in}{3.972903in}}{\pgfqpoint{9.153029in}{3.968785in}}%
\pgfpathcurveto{\pgfqpoint{9.157148in}{3.964667in}}{\pgfqpoint{9.162734in}{3.962353in}}{\pgfqpoint{9.168558in}{3.962353in}}%
\pgfpathlineto{\pgfqpoint{9.168558in}{3.962353in}}%
\pgfpathclose%
\pgfusepath{stroke,fill}%
\end{pgfscope}%
\begin{pgfscope}%
\pgfpathrectangle{\pgfqpoint{0.640323in}{0.527436in}}{\pgfqpoint{9.687500in}{3.850000in}}%
\pgfusepath{clip}%
\pgfsetbuttcap%
\pgfsetroundjoin%
\definecolor{currentfill}{rgb}{0.239216,0.478431,0.992157}%
\pgfsetfillcolor{currentfill}%
\pgfsetfillopacity{0.500000}%
\pgfsetlinewidth{1.003750pt}%
\definecolor{currentstroke}{rgb}{0.239216,0.478431,0.992157}%
\pgfsetstrokecolor{currentstroke}%
\pgfsetstrokeopacity{0.500000}%
\pgfsetdash{{3.700000pt}{1.600000pt}}{0.000000pt}%
\pgfpathmoveto{\pgfqpoint{9.348289in}{3.980392in}}%
\pgfpathcurveto{\pgfqpoint{9.354113in}{3.980392in}}{\pgfqpoint{9.359699in}{3.982706in}}{\pgfqpoint{9.363817in}{3.986824in}}%
\pgfpathcurveto{\pgfqpoint{9.367935in}{3.990943in}}{\pgfqpoint{9.370249in}{3.996529in}}{\pgfqpoint{9.370249in}{4.002353in}}%
\pgfpathcurveto{\pgfqpoint{9.370249in}{4.008177in}}{\pgfqpoint{9.367935in}{4.013763in}}{\pgfqpoint{9.363817in}{4.017881in}}%
\pgfpathcurveto{\pgfqpoint{9.359699in}{4.021999in}}{\pgfqpoint{9.354113in}{4.024313in}}{\pgfqpoint{9.348289in}{4.024313in}}%
\pgfpathcurveto{\pgfqpoint{9.342465in}{4.024313in}}{\pgfqpoint{9.336879in}{4.021999in}}{\pgfqpoint{9.332760in}{4.017881in}}%
\pgfpathcurveto{\pgfqpoint{9.328642in}{4.013763in}}{\pgfqpoint{9.326328in}{4.008177in}}{\pgfqpoint{9.326328in}{4.002353in}}%
\pgfpathcurveto{\pgfqpoint{9.326328in}{3.996529in}}{\pgfqpoint{9.328642in}{3.990943in}}{\pgfqpoint{9.332760in}{3.986824in}}%
\pgfpathcurveto{\pgfqpoint{9.336879in}{3.982706in}}{\pgfqpoint{9.342465in}{3.980392in}}{\pgfqpoint{9.348289in}{3.980392in}}%
\pgfpathlineto{\pgfqpoint{9.348289in}{3.980392in}}%
\pgfpathclose%
\pgfusepath{stroke,fill}%
\end{pgfscope}%
\begin{pgfscope}%
\pgfpathrectangle{\pgfqpoint{0.640323in}{0.527436in}}{\pgfqpoint{9.687500in}{3.850000in}}%
\pgfusepath{clip}%
\pgfsetbuttcap%
\pgfsetroundjoin%
\definecolor{currentfill}{rgb}{0.239216,0.478431,0.992157}%
\pgfsetfillcolor{currentfill}%
\pgfsetfillopacity{0.500000}%
\pgfsetlinewidth{1.003750pt}%
\definecolor{currentstroke}{rgb}{0.239216,0.478431,0.992157}%
\pgfsetstrokecolor{currentstroke}%
\pgfsetstrokeopacity{0.500000}%
\pgfsetdash{{3.700000pt}{1.600000pt}}{0.000000pt}%
\pgfpathmoveto{\pgfqpoint{9.528020in}{3.990011in}}%
\pgfpathcurveto{\pgfqpoint{9.533844in}{3.990011in}}{\pgfqpoint{9.539430in}{3.992325in}}{\pgfqpoint{9.543548in}{3.996443in}}%
\pgfpathcurveto{\pgfqpoint{9.547666in}{4.000561in}}{\pgfqpoint{9.549980in}{4.006148in}}{\pgfqpoint{9.549980in}{4.011971in}}%
\pgfpathcurveto{\pgfqpoint{9.549980in}{4.017795in}}{\pgfqpoint{9.547666in}{4.023382in}}{\pgfqpoint{9.543548in}{4.027500in}}%
\pgfpathcurveto{\pgfqpoint{9.539430in}{4.031618in}}{\pgfqpoint{9.533844in}{4.033932in}}{\pgfqpoint{9.528020in}{4.033932in}}%
\pgfpathcurveto{\pgfqpoint{9.522196in}{4.033932in}}{\pgfqpoint{9.516610in}{4.031618in}}{\pgfqpoint{9.512491in}{4.027500in}}%
\pgfpathcurveto{\pgfqpoint{9.508373in}{4.023382in}}{\pgfqpoint{9.506059in}{4.017795in}}{\pgfqpoint{9.506059in}{4.011971in}}%
\pgfpathcurveto{\pgfqpoint{9.506059in}{4.006148in}}{\pgfqpoint{9.508373in}{4.000561in}}{\pgfqpoint{9.512491in}{3.996443in}}%
\pgfpathcurveto{\pgfqpoint{9.516610in}{3.992325in}}{\pgfqpoint{9.522196in}{3.990011in}}{\pgfqpoint{9.528020in}{3.990011in}}%
\pgfpathlineto{\pgfqpoint{9.528020in}{3.990011in}}%
\pgfpathclose%
\pgfusepath{stroke,fill}%
\end{pgfscope}%
\begin{pgfscope}%
\pgfpathrectangle{\pgfqpoint{0.640323in}{0.527436in}}{\pgfqpoint{9.687500in}{3.850000in}}%
\pgfusepath{clip}%
\pgfsetbuttcap%
\pgfsetroundjoin%
\definecolor{currentfill}{rgb}{0.239216,0.478431,0.992157}%
\pgfsetfillcolor{currentfill}%
\pgfsetfillopacity{0.500000}%
\pgfsetlinewidth{1.003750pt}%
\definecolor{currentstroke}{rgb}{0.239216,0.478431,0.992157}%
\pgfsetstrokecolor{currentstroke}%
\pgfsetstrokeopacity{0.500000}%
\pgfsetdash{{3.700000pt}{1.600000pt}}{0.000000pt}%
\pgfpathmoveto{\pgfqpoint{9.707751in}{3.999512in}}%
\pgfpathcurveto{\pgfqpoint{9.713575in}{3.999512in}}{\pgfqpoint{9.719161in}{4.001826in}}{\pgfqpoint{9.723279in}{4.005944in}}%
\pgfpathcurveto{\pgfqpoint{9.727397in}{4.010062in}}{\pgfqpoint{9.729711in}{4.015648in}}{\pgfqpoint{9.729711in}{4.021472in}}%
\pgfpathcurveto{\pgfqpoint{9.729711in}{4.027296in}}{\pgfqpoint{9.727397in}{4.032882in}}{\pgfqpoint{9.723279in}{4.037001in}}%
\pgfpathcurveto{\pgfqpoint{9.719161in}{4.041119in}}{\pgfqpoint{9.713575in}{4.043433in}}{\pgfqpoint{9.707751in}{4.043433in}}%
\pgfpathcurveto{\pgfqpoint{9.701927in}{4.043433in}}{\pgfqpoint{9.696340in}{4.041119in}}{\pgfqpoint{9.692222in}{4.037001in}}%
\pgfpathcurveto{\pgfqpoint{9.688104in}{4.032882in}}{\pgfqpoint{9.685790in}{4.027296in}}{\pgfqpoint{9.685790in}{4.021472in}}%
\pgfpathcurveto{\pgfqpoint{9.685790in}{4.015648in}}{\pgfqpoint{9.688104in}{4.010062in}}{\pgfqpoint{9.692222in}{4.005944in}}%
\pgfpathcurveto{\pgfqpoint{9.696340in}{4.001826in}}{\pgfqpoint{9.701927in}{3.999512in}}{\pgfqpoint{9.707751in}{3.999512in}}%
\pgfpathlineto{\pgfqpoint{9.707751in}{3.999512in}}%
\pgfpathclose%
\pgfusepath{stroke,fill}%
\end{pgfscope}%
\begin{pgfscope}%
\pgfpathrectangle{\pgfqpoint{0.640323in}{0.527436in}}{\pgfqpoint{9.687500in}{3.850000in}}%
\pgfusepath{clip}%
\pgfsetbuttcap%
\pgfsetroundjoin%
\definecolor{currentfill}{rgb}{0.239216,0.478431,0.992157}%
\pgfsetfillcolor{currentfill}%
\pgfsetfillopacity{0.500000}%
\pgfsetlinewidth{1.003750pt}%
\definecolor{currentstroke}{rgb}{0.239216,0.478431,0.992157}%
\pgfsetstrokecolor{currentstroke}%
\pgfsetstrokeopacity{0.500000}%
\pgfsetdash{{3.700000pt}{1.600000pt}}{0.000000pt}%
\pgfpathmoveto{\pgfqpoint{9.887482in}{4.021302in}}%
\pgfpathcurveto{\pgfqpoint{9.893306in}{4.021302in}}{\pgfqpoint{9.898892in}{4.023616in}}{\pgfqpoint{9.903010in}{4.027734in}}%
\pgfpathcurveto{\pgfqpoint{9.907128in}{4.031852in}}{\pgfqpoint{9.909442in}{4.037438in}}{\pgfqpoint{9.909442in}{4.043262in}}%
\pgfpathcurveto{\pgfqpoint{9.909442in}{4.049086in}}{\pgfqpoint{9.907128in}{4.054672in}}{\pgfqpoint{9.903010in}{4.058790in}}%
\pgfpathcurveto{\pgfqpoint{9.898892in}{4.062908in}}{\pgfqpoint{9.893306in}{4.065222in}}{\pgfqpoint{9.887482in}{4.065222in}}%
\pgfpathcurveto{\pgfqpoint{9.881658in}{4.065222in}}{\pgfqpoint{9.876071in}{4.062908in}}{\pgfqpoint{9.871953in}{4.058790in}}%
\pgfpathcurveto{\pgfqpoint{9.867835in}{4.054672in}}{\pgfqpoint{9.865521in}{4.049086in}}{\pgfqpoint{9.865521in}{4.043262in}}%
\pgfpathcurveto{\pgfqpoint{9.865521in}{4.037438in}}{\pgfqpoint{9.867835in}{4.031852in}}{\pgfqpoint{9.871953in}{4.027734in}}%
\pgfpathcurveto{\pgfqpoint{9.876071in}{4.023616in}}{\pgfqpoint{9.881658in}{4.021302in}}{\pgfqpoint{9.887482in}{4.021302in}}%
\pgfpathlineto{\pgfqpoint{9.887482in}{4.021302in}}%
\pgfpathclose%
\pgfusepath{stroke,fill}%
\end{pgfscope}%
\begin{pgfscope}%
\pgfpathrectangle{\pgfqpoint{0.640323in}{0.527436in}}{\pgfqpoint{9.687500in}{3.850000in}}%
\pgfusepath{clip}%
\pgfsetbuttcap%
\pgfsetroundjoin%
\definecolor{currentfill}{rgb}{0.000000,0.000000,0.000000}%
\pgfsetfillcolor{currentfill}%
\pgfsetfillopacity{0.500000}%
\pgfsetlinewidth{1.003750pt}%
\definecolor{currentstroke}{rgb}{0.000000,0.000000,0.000000}%
\pgfsetstrokecolor{currentstroke}%
\pgfsetstrokeopacity{0.500000}%
\pgfsetdash{{3.700000pt}{1.600000pt}}{0.000000pt}%
\pgfpathmoveto{\pgfqpoint{1.080663in}{0.637576in}}%
\pgfpathcurveto{\pgfqpoint{1.086487in}{0.637576in}}{\pgfqpoint{1.092074in}{0.639890in}}{\pgfqpoint{1.096192in}{0.644008in}}%
\pgfpathcurveto{\pgfqpoint{1.100310in}{0.648126in}}{\pgfqpoint{1.102624in}{0.653712in}}{\pgfqpoint{1.102624in}{0.659536in}}%
\pgfpathcurveto{\pgfqpoint{1.102624in}{0.665360in}}{\pgfqpoint{1.100310in}{0.670946in}}{\pgfqpoint{1.096192in}{0.675065in}}%
\pgfpathcurveto{\pgfqpoint{1.092074in}{0.679183in}}{\pgfqpoint{1.086487in}{0.681497in}}{\pgfqpoint{1.080663in}{0.681497in}}%
\pgfpathcurveto{\pgfqpoint{1.074839in}{0.681497in}}{\pgfqpoint{1.069253in}{0.679183in}}{\pgfqpoint{1.065135in}{0.675065in}}%
\pgfpathcurveto{\pgfqpoint{1.061017in}{0.670946in}}{\pgfqpoint{1.058703in}{0.665360in}}{\pgfqpoint{1.058703in}{0.659536in}}%
\pgfpathcurveto{\pgfqpoint{1.058703in}{0.653712in}}{\pgfqpoint{1.061017in}{0.648126in}}{\pgfqpoint{1.065135in}{0.644008in}}%
\pgfpathcurveto{\pgfqpoint{1.069253in}{0.639890in}}{\pgfqpoint{1.074839in}{0.637576in}}{\pgfqpoint{1.080663in}{0.637576in}}%
\pgfpathlineto{\pgfqpoint{1.080663in}{0.637576in}}%
\pgfpathclose%
\pgfusepath{stroke,fill}%
\end{pgfscope}%
\begin{pgfscope}%
\pgfpathrectangle{\pgfqpoint{0.640323in}{0.527436in}}{\pgfqpoint{9.687500in}{3.850000in}}%
\pgfusepath{clip}%
\pgfsetbuttcap%
\pgfsetroundjoin%
\definecolor{currentfill}{rgb}{0.000000,0.000000,0.000000}%
\pgfsetfillcolor{currentfill}%
\pgfsetfillopacity{0.500000}%
\pgfsetlinewidth{1.003750pt}%
\definecolor{currentstroke}{rgb}{0.000000,0.000000,0.000000}%
\pgfsetstrokecolor{currentstroke}%
\pgfsetstrokeopacity{0.500000}%
\pgfsetdash{{3.700000pt}{1.600000pt}}{0.000000pt}%
\pgfpathmoveto{\pgfqpoint{1.260394in}{0.638566in}}%
\pgfpathcurveto{\pgfqpoint{1.266218in}{0.638566in}}{\pgfqpoint{1.271805in}{0.640880in}}{\pgfqpoint{1.275923in}{0.644998in}}%
\pgfpathcurveto{\pgfqpoint{1.280041in}{0.649116in}}{\pgfqpoint{1.282355in}{0.654702in}}{\pgfqpoint{1.282355in}{0.660526in}}%
\pgfpathcurveto{\pgfqpoint{1.282355in}{0.666350in}}{\pgfqpoint{1.280041in}{0.671936in}}{\pgfqpoint{1.275923in}{0.676054in}}%
\pgfpathcurveto{\pgfqpoint{1.271805in}{0.680173in}}{\pgfqpoint{1.266218in}{0.682486in}}{\pgfqpoint{1.260394in}{0.682486in}}%
\pgfpathcurveto{\pgfqpoint{1.254570in}{0.682486in}}{\pgfqpoint{1.248984in}{0.680173in}}{\pgfqpoint{1.244866in}{0.676054in}}%
\pgfpathcurveto{\pgfqpoint{1.240748in}{0.671936in}}{\pgfqpoint{1.238434in}{0.666350in}}{\pgfqpoint{1.238434in}{0.660526in}}%
\pgfpathcurveto{\pgfqpoint{1.238434in}{0.654702in}}{\pgfqpoint{1.240748in}{0.649116in}}{\pgfqpoint{1.244866in}{0.644998in}}%
\pgfpathcurveto{\pgfqpoint{1.248984in}{0.640880in}}{\pgfqpoint{1.254570in}{0.638566in}}{\pgfqpoint{1.260394in}{0.638566in}}%
\pgfpathlineto{\pgfqpoint{1.260394in}{0.638566in}}%
\pgfpathclose%
\pgfusepath{stroke,fill}%
\end{pgfscope}%
\begin{pgfscope}%
\pgfpathrectangle{\pgfqpoint{0.640323in}{0.527436in}}{\pgfqpoint{9.687500in}{3.850000in}}%
\pgfusepath{clip}%
\pgfsetbuttcap%
\pgfsetroundjoin%
\definecolor{currentfill}{rgb}{0.000000,0.000000,0.000000}%
\pgfsetfillcolor{currentfill}%
\pgfsetfillopacity{0.500000}%
\pgfsetlinewidth{1.003750pt}%
\definecolor{currentstroke}{rgb}{0.000000,0.000000,0.000000}%
\pgfsetstrokecolor{currentstroke}%
\pgfsetstrokeopacity{0.500000}%
\pgfsetdash{{3.700000pt}{1.600000pt}}{0.000000pt}%
\pgfpathmoveto{\pgfqpoint{1.440125in}{0.640284in}}%
\pgfpathcurveto{\pgfqpoint{1.445949in}{0.640284in}}{\pgfqpoint{1.451535in}{0.642598in}}{\pgfqpoint{1.455654in}{0.646716in}}%
\pgfpathcurveto{\pgfqpoint{1.459772in}{0.650834in}}{\pgfqpoint{1.462086in}{0.656420in}}{\pgfqpoint{1.462086in}{0.662244in}}%
\pgfpathcurveto{\pgfqpoint{1.462086in}{0.668068in}}{\pgfqpoint{1.459772in}{0.673654in}}{\pgfqpoint{1.455654in}{0.677773in}}%
\pgfpathcurveto{\pgfqpoint{1.451535in}{0.681891in}}{\pgfqpoint{1.445949in}{0.684205in}}{\pgfqpoint{1.440125in}{0.684205in}}%
\pgfpathcurveto{\pgfqpoint{1.434301in}{0.684205in}}{\pgfqpoint{1.428715in}{0.681891in}}{\pgfqpoint{1.424597in}{0.677773in}}%
\pgfpathcurveto{\pgfqpoint{1.420479in}{0.673654in}}{\pgfqpoint{1.418165in}{0.668068in}}{\pgfqpoint{1.418165in}{0.662244in}}%
\pgfpathcurveto{\pgfqpoint{1.418165in}{0.656420in}}{\pgfqpoint{1.420479in}{0.650834in}}{\pgfqpoint{1.424597in}{0.646716in}}%
\pgfpathcurveto{\pgfqpoint{1.428715in}{0.642598in}}{\pgfqpoint{1.434301in}{0.640284in}}{\pgfqpoint{1.440125in}{0.640284in}}%
\pgfpathlineto{\pgfqpoint{1.440125in}{0.640284in}}%
\pgfpathclose%
\pgfusepath{stroke,fill}%
\end{pgfscope}%
\begin{pgfscope}%
\pgfpathrectangle{\pgfqpoint{0.640323in}{0.527436in}}{\pgfqpoint{9.687500in}{3.850000in}}%
\pgfusepath{clip}%
\pgfsetbuttcap%
\pgfsetroundjoin%
\definecolor{currentfill}{rgb}{0.000000,0.000000,0.000000}%
\pgfsetfillcolor{currentfill}%
\pgfsetfillopacity{0.500000}%
\pgfsetlinewidth{1.003750pt}%
\definecolor{currentstroke}{rgb}{0.000000,0.000000,0.000000}%
\pgfsetstrokecolor{currentstroke}%
\pgfsetstrokeopacity{0.500000}%
\pgfsetdash{{3.700000pt}{1.600000pt}}{0.000000pt}%
\pgfpathmoveto{\pgfqpoint{1.619856in}{0.642325in}}%
\pgfpathcurveto{\pgfqpoint{1.625680in}{0.642325in}}{\pgfqpoint{1.631266in}{0.644638in}}{\pgfqpoint{1.635385in}{0.648757in}}%
\pgfpathcurveto{\pgfqpoint{1.639503in}{0.652875in}}{\pgfqpoint{1.641817in}{0.658461in}}{\pgfqpoint{1.641817in}{0.664285in}}%
\pgfpathcurveto{\pgfqpoint{1.641817in}{0.670109in}}{\pgfqpoint{1.639503in}{0.675695in}}{\pgfqpoint{1.635385in}{0.679813in}}%
\pgfpathcurveto{\pgfqpoint{1.631266in}{0.683931in}}{\pgfqpoint{1.625680in}{0.686245in}}{\pgfqpoint{1.619856in}{0.686245in}}%
\pgfpathcurveto{\pgfqpoint{1.614032in}{0.686245in}}{\pgfqpoint{1.608446in}{0.683931in}}{\pgfqpoint{1.604328in}{0.679813in}}%
\pgfpathcurveto{\pgfqpoint{1.600210in}{0.675695in}}{\pgfqpoint{1.597896in}{0.670109in}}{\pgfqpoint{1.597896in}{0.664285in}}%
\pgfpathcurveto{\pgfqpoint{1.597896in}{0.658461in}}{\pgfqpoint{1.600210in}{0.652875in}}{\pgfqpoint{1.604328in}{0.648757in}}%
\pgfpathcurveto{\pgfqpoint{1.608446in}{0.644638in}}{\pgfqpoint{1.614032in}{0.642325in}}{\pgfqpoint{1.619856in}{0.642325in}}%
\pgfpathlineto{\pgfqpoint{1.619856in}{0.642325in}}%
\pgfpathclose%
\pgfusepath{stroke,fill}%
\end{pgfscope}%
\begin{pgfscope}%
\pgfpathrectangle{\pgfqpoint{0.640323in}{0.527436in}}{\pgfqpoint{9.687500in}{3.850000in}}%
\pgfusepath{clip}%
\pgfsetbuttcap%
\pgfsetroundjoin%
\definecolor{currentfill}{rgb}{0.000000,0.000000,0.000000}%
\pgfsetfillcolor{currentfill}%
\pgfsetfillopacity{0.500000}%
\pgfsetlinewidth{1.003750pt}%
\definecolor{currentstroke}{rgb}{0.000000,0.000000,0.000000}%
\pgfsetstrokecolor{currentstroke}%
\pgfsetstrokeopacity{0.500000}%
\pgfsetdash{{3.700000pt}{1.600000pt}}{0.000000pt}%
\pgfpathmoveto{\pgfqpoint{1.799587in}{0.644563in}}%
\pgfpathcurveto{\pgfqpoint{1.805411in}{0.644563in}}{\pgfqpoint{1.810997in}{0.646877in}}{\pgfqpoint{1.815116in}{0.650995in}}%
\pgfpathcurveto{\pgfqpoint{1.819234in}{0.655113in}}{\pgfqpoint{1.821548in}{0.660699in}}{\pgfqpoint{1.821548in}{0.666523in}}%
\pgfpathcurveto{\pgfqpoint{1.821548in}{0.672347in}}{\pgfqpoint{1.819234in}{0.677933in}}{\pgfqpoint{1.815116in}{0.682051in}}%
\pgfpathcurveto{\pgfqpoint{1.810997in}{0.686169in}}{\pgfqpoint{1.805411in}{0.688483in}}{\pgfqpoint{1.799587in}{0.688483in}}%
\pgfpathcurveto{\pgfqpoint{1.793763in}{0.688483in}}{\pgfqpoint{1.788177in}{0.686169in}}{\pgfqpoint{1.784059in}{0.682051in}}%
\pgfpathcurveto{\pgfqpoint{1.779941in}{0.677933in}}{\pgfqpoint{1.777627in}{0.672347in}}{\pgfqpoint{1.777627in}{0.666523in}}%
\pgfpathcurveto{\pgfqpoint{1.777627in}{0.660699in}}{\pgfqpoint{1.779941in}{0.655113in}}{\pgfqpoint{1.784059in}{0.650995in}}%
\pgfpathcurveto{\pgfqpoint{1.788177in}{0.646877in}}{\pgfqpoint{1.793763in}{0.644563in}}{\pgfqpoint{1.799587in}{0.644563in}}%
\pgfpathlineto{\pgfqpoint{1.799587in}{0.644563in}}%
\pgfpathclose%
\pgfusepath{stroke,fill}%
\end{pgfscope}%
\begin{pgfscope}%
\pgfpathrectangle{\pgfqpoint{0.640323in}{0.527436in}}{\pgfqpoint{9.687500in}{3.850000in}}%
\pgfusepath{clip}%
\pgfsetbuttcap%
\pgfsetroundjoin%
\definecolor{currentfill}{rgb}{0.000000,0.000000,0.000000}%
\pgfsetfillcolor{currentfill}%
\pgfsetfillopacity{0.500000}%
\pgfsetlinewidth{1.003750pt}%
\definecolor{currentstroke}{rgb}{0.000000,0.000000,0.000000}%
\pgfsetstrokecolor{currentstroke}%
\pgfsetstrokeopacity{0.500000}%
\pgfsetdash{{3.700000pt}{1.600000pt}}{0.000000pt}%
\pgfpathmoveto{\pgfqpoint{1.979318in}{0.649628in}}%
\pgfpathcurveto{\pgfqpoint{1.985142in}{0.649628in}}{\pgfqpoint{1.990728in}{0.651942in}}{\pgfqpoint{1.994847in}{0.656060in}}%
\pgfpathcurveto{\pgfqpoint{1.998965in}{0.660179in}}{\pgfqpoint{2.001279in}{0.665765in}}{\pgfqpoint{2.001279in}{0.671589in}}%
\pgfpathcurveto{\pgfqpoint{2.001279in}{0.677413in}}{\pgfqpoint{1.998965in}{0.682999in}}{\pgfqpoint{1.994847in}{0.687117in}}%
\pgfpathcurveto{\pgfqpoint{1.990728in}{0.691235in}}{\pgfqpoint{1.985142in}{0.693549in}}{\pgfqpoint{1.979318in}{0.693549in}}%
\pgfpathcurveto{\pgfqpoint{1.973494in}{0.693549in}}{\pgfqpoint{1.967908in}{0.691235in}}{\pgfqpoint{1.963790in}{0.687117in}}%
\pgfpathcurveto{\pgfqpoint{1.959672in}{0.682999in}}{\pgfqpoint{1.957358in}{0.677413in}}{\pgfqpoint{1.957358in}{0.671589in}}%
\pgfpathcurveto{\pgfqpoint{1.957358in}{0.665765in}}{\pgfqpoint{1.959672in}{0.660179in}}{\pgfqpoint{1.963790in}{0.656060in}}%
\pgfpathcurveto{\pgfqpoint{1.967908in}{0.651942in}}{\pgfqpoint{1.973494in}{0.649628in}}{\pgfqpoint{1.979318in}{0.649628in}}%
\pgfpathlineto{\pgfqpoint{1.979318in}{0.649628in}}%
\pgfpathclose%
\pgfusepath{stroke,fill}%
\end{pgfscope}%
\begin{pgfscope}%
\pgfpathrectangle{\pgfqpoint{0.640323in}{0.527436in}}{\pgfqpoint{9.687500in}{3.850000in}}%
\pgfusepath{clip}%
\pgfsetbuttcap%
\pgfsetroundjoin%
\definecolor{currentfill}{rgb}{0.000000,0.000000,0.000000}%
\pgfsetfillcolor{currentfill}%
\pgfsetfillopacity{0.500000}%
\pgfsetlinewidth{1.003750pt}%
\definecolor{currentstroke}{rgb}{0.000000,0.000000,0.000000}%
\pgfsetstrokecolor{currentstroke}%
\pgfsetstrokeopacity{0.500000}%
\pgfsetdash{{3.700000pt}{1.600000pt}}{0.000000pt}%
\pgfpathmoveto{\pgfqpoint{2.159049in}{0.658756in}}%
\pgfpathcurveto{\pgfqpoint{2.164873in}{0.658756in}}{\pgfqpoint{2.170459in}{0.661070in}}{\pgfqpoint{2.174578in}{0.665188in}}%
\pgfpathcurveto{\pgfqpoint{2.178696in}{0.669306in}}{\pgfqpoint{2.181010in}{0.674892in}}{\pgfqpoint{2.181010in}{0.680716in}}%
\pgfpathcurveto{\pgfqpoint{2.181010in}{0.686540in}}{\pgfqpoint{2.178696in}{0.692126in}}{\pgfqpoint{2.174578in}{0.696244in}}%
\pgfpathcurveto{\pgfqpoint{2.170459in}{0.700362in}}{\pgfqpoint{2.164873in}{0.702676in}}{\pgfqpoint{2.159049in}{0.702676in}}%
\pgfpathcurveto{\pgfqpoint{2.153225in}{0.702676in}}{\pgfqpoint{2.147639in}{0.700362in}}{\pgfqpoint{2.143521in}{0.696244in}}%
\pgfpathcurveto{\pgfqpoint{2.139403in}{0.692126in}}{\pgfqpoint{2.137089in}{0.686540in}}{\pgfqpoint{2.137089in}{0.680716in}}%
\pgfpathcurveto{\pgfqpoint{2.137089in}{0.674892in}}{\pgfqpoint{2.139403in}{0.669306in}}{\pgfqpoint{2.143521in}{0.665188in}}%
\pgfpathcurveto{\pgfqpoint{2.147639in}{0.661070in}}{\pgfqpoint{2.153225in}{0.658756in}}{\pgfqpoint{2.159049in}{0.658756in}}%
\pgfpathlineto{\pgfqpoint{2.159049in}{0.658756in}}%
\pgfpathclose%
\pgfusepath{stroke,fill}%
\end{pgfscope}%
\begin{pgfscope}%
\pgfpathrectangle{\pgfqpoint{0.640323in}{0.527436in}}{\pgfqpoint{9.687500in}{3.850000in}}%
\pgfusepath{clip}%
\pgfsetbuttcap%
\pgfsetroundjoin%
\definecolor{currentfill}{rgb}{0.000000,0.000000,0.000000}%
\pgfsetfillcolor{currentfill}%
\pgfsetfillopacity{0.500000}%
\pgfsetlinewidth{1.003750pt}%
\definecolor{currentstroke}{rgb}{0.000000,0.000000,0.000000}%
\pgfsetstrokecolor{currentstroke}%
\pgfsetstrokeopacity{0.500000}%
\pgfsetdash{{3.700000pt}{1.600000pt}}{0.000000pt}%
\pgfpathmoveto{\pgfqpoint{2.338780in}{0.679264in}}%
\pgfpathcurveto{\pgfqpoint{2.344604in}{0.679264in}}{\pgfqpoint{2.350190in}{0.681577in}}{\pgfqpoint{2.354309in}{0.685696in}}%
\pgfpathcurveto{\pgfqpoint{2.358427in}{0.689814in}}{\pgfqpoint{2.360741in}{0.695400in}}{\pgfqpoint{2.360741in}{0.701224in}}%
\pgfpathcurveto{\pgfqpoint{2.360741in}{0.707048in}}{\pgfqpoint{2.358427in}{0.712634in}}{\pgfqpoint{2.354309in}{0.716752in}}%
\pgfpathcurveto{\pgfqpoint{2.350190in}{0.720870in}}{\pgfqpoint{2.344604in}{0.723184in}}{\pgfqpoint{2.338780in}{0.723184in}}%
\pgfpathcurveto{\pgfqpoint{2.332956in}{0.723184in}}{\pgfqpoint{2.327370in}{0.720870in}}{\pgfqpoint{2.323252in}{0.716752in}}%
\pgfpathcurveto{\pgfqpoint{2.319134in}{0.712634in}}{\pgfqpoint{2.316820in}{0.707048in}}{\pgfqpoint{2.316820in}{0.701224in}}%
\pgfpathcurveto{\pgfqpoint{2.316820in}{0.695400in}}{\pgfqpoint{2.319134in}{0.689814in}}{\pgfqpoint{2.323252in}{0.685696in}}%
\pgfpathcurveto{\pgfqpoint{2.327370in}{0.681577in}}{\pgfqpoint{2.332956in}{0.679264in}}{\pgfqpoint{2.338780in}{0.679264in}}%
\pgfpathlineto{\pgfqpoint{2.338780in}{0.679264in}}%
\pgfpathclose%
\pgfusepath{stroke,fill}%
\end{pgfscope}%
\begin{pgfscope}%
\pgfpathrectangle{\pgfqpoint{0.640323in}{0.527436in}}{\pgfqpoint{9.687500in}{3.850000in}}%
\pgfusepath{clip}%
\pgfsetbuttcap%
\pgfsetroundjoin%
\definecolor{currentfill}{rgb}{0.000000,0.000000,0.000000}%
\pgfsetfillcolor{currentfill}%
\pgfsetfillopacity{0.500000}%
\pgfsetlinewidth{1.003750pt}%
\definecolor{currentstroke}{rgb}{0.000000,0.000000,0.000000}%
\pgfsetstrokecolor{currentstroke}%
\pgfsetstrokeopacity{0.500000}%
\pgfsetdash{{3.700000pt}{1.600000pt}}{0.000000pt}%
\pgfpathmoveto{\pgfqpoint{2.518511in}{0.806554in}}%
\pgfpathcurveto{\pgfqpoint{2.524335in}{0.806554in}}{\pgfqpoint{2.529921in}{0.808868in}}{\pgfqpoint{2.534040in}{0.812986in}}%
\pgfpathcurveto{\pgfqpoint{2.538158in}{0.817104in}}{\pgfqpoint{2.540472in}{0.822691in}}{\pgfqpoint{2.540472in}{0.828515in}}%
\pgfpathcurveto{\pgfqpoint{2.540472in}{0.834338in}}{\pgfqpoint{2.538158in}{0.839925in}}{\pgfqpoint{2.534040in}{0.844043in}}%
\pgfpathcurveto{\pgfqpoint{2.529921in}{0.848161in}}{\pgfqpoint{2.524335in}{0.850475in}}{\pgfqpoint{2.518511in}{0.850475in}}%
\pgfpathcurveto{\pgfqpoint{2.512687in}{0.850475in}}{\pgfqpoint{2.507101in}{0.848161in}}{\pgfqpoint{2.502983in}{0.844043in}}%
\pgfpathcurveto{\pgfqpoint{2.498865in}{0.839925in}}{\pgfqpoint{2.496551in}{0.834338in}}{\pgfqpoint{2.496551in}{0.828515in}}%
\pgfpathcurveto{\pgfqpoint{2.496551in}{0.822691in}}{\pgfqpoint{2.498865in}{0.817104in}}{\pgfqpoint{2.502983in}{0.812986in}}%
\pgfpathcurveto{\pgfqpoint{2.507101in}{0.808868in}}{\pgfqpoint{2.512687in}{0.806554in}}{\pgfqpoint{2.518511in}{0.806554in}}%
\pgfpathlineto{\pgfqpoint{2.518511in}{0.806554in}}%
\pgfpathclose%
\pgfusepath{stroke,fill}%
\end{pgfscope}%
\begin{pgfscope}%
\pgfpathrectangle{\pgfqpoint{0.640323in}{0.527436in}}{\pgfqpoint{9.687500in}{3.850000in}}%
\pgfusepath{clip}%
\pgfsetbuttcap%
\pgfsetroundjoin%
\definecolor{currentfill}{rgb}{0.000000,0.000000,0.000000}%
\pgfsetfillcolor{currentfill}%
\pgfsetfillopacity{0.500000}%
\pgfsetlinewidth{1.003750pt}%
\definecolor{currentstroke}{rgb}{0.000000,0.000000,0.000000}%
\pgfsetstrokecolor{currentstroke}%
\pgfsetstrokeopacity{0.500000}%
\pgfsetdash{{3.700000pt}{1.600000pt}}{0.000000pt}%
\pgfpathmoveto{\pgfqpoint{2.698242in}{1.424061in}}%
\pgfpathcurveto{\pgfqpoint{2.704066in}{1.424061in}}{\pgfqpoint{2.709652in}{1.426375in}}{\pgfqpoint{2.713771in}{1.430493in}}%
\pgfpathcurveto{\pgfqpoint{2.717889in}{1.434611in}}{\pgfqpoint{2.720203in}{1.440197in}}{\pgfqpoint{2.720203in}{1.446021in}}%
\pgfpathcurveto{\pgfqpoint{2.720203in}{1.451845in}}{\pgfqpoint{2.717889in}{1.457431in}}{\pgfqpoint{2.713771in}{1.461549in}}%
\pgfpathcurveto{\pgfqpoint{2.709652in}{1.465667in}}{\pgfqpoint{2.704066in}{1.467981in}}{\pgfqpoint{2.698242in}{1.467981in}}%
\pgfpathcurveto{\pgfqpoint{2.692418in}{1.467981in}}{\pgfqpoint{2.686832in}{1.465667in}}{\pgfqpoint{2.682714in}{1.461549in}}%
\pgfpathcurveto{\pgfqpoint{2.678596in}{1.457431in}}{\pgfqpoint{2.676282in}{1.451845in}}{\pgfqpoint{2.676282in}{1.446021in}}%
\pgfpathcurveto{\pgfqpoint{2.676282in}{1.440197in}}{\pgfqpoint{2.678596in}{1.434611in}}{\pgfqpoint{2.682714in}{1.430493in}}%
\pgfpathcurveto{\pgfqpoint{2.686832in}{1.426375in}}{\pgfqpoint{2.692418in}{1.424061in}}{\pgfqpoint{2.698242in}{1.424061in}}%
\pgfpathlineto{\pgfqpoint{2.698242in}{1.424061in}}%
\pgfpathclose%
\pgfusepath{stroke,fill}%
\end{pgfscope}%
\begin{pgfscope}%
\pgfpathrectangle{\pgfqpoint{0.640323in}{0.527436in}}{\pgfqpoint{9.687500in}{3.850000in}}%
\pgfusepath{clip}%
\pgfsetbuttcap%
\pgfsetroundjoin%
\definecolor{currentfill}{rgb}{0.000000,0.000000,0.000000}%
\pgfsetfillcolor{currentfill}%
\pgfsetfillopacity{0.500000}%
\pgfsetlinewidth{1.003750pt}%
\definecolor{currentstroke}{rgb}{0.000000,0.000000,0.000000}%
\pgfsetstrokecolor{currentstroke}%
\pgfsetstrokeopacity{0.500000}%
\pgfsetdash{{3.700000pt}{1.600000pt}}{0.000000pt}%
\pgfpathmoveto{\pgfqpoint{2.877973in}{1.818922in}}%
\pgfpathcurveto{\pgfqpoint{2.883797in}{1.818922in}}{\pgfqpoint{2.889383in}{1.821236in}}{\pgfqpoint{2.893501in}{1.825354in}}%
\pgfpathcurveto{\pgfqpoint{2.897620in}{1.829472in}}{\pgfqpoint{2.899934in}{1.835058in}}{\pgfqpoint{2.899934in}{1.840882in}}%
\pgfpathcurveto{\pgfqpoint{2.899934in}{1.846706in}}{\pgfqpoint{2.897620in}{1.852292in}}{\pgfqpoint{2.893501in}{1.856410in}}%
\pgfpathcurveto{\pgfqpoint{2.889383in}{1.860528in}}{\pgfqpoint{2.883797in}{1.862842in}}{\pgfqpoint{2.877973in}{1.862842in}}%
\pgfpathcurveto{\pgfqpoint{2.872149in}{1.862842in}}{\pgfqpoint{2.866563in}{1.860528in}}{\pgfqpoint{2.862445in}{1.856410in}}%
\pgfpathcurveto{\pgfqpoint{2.858327in}{1.852292in}}{\pgfqpoint{2.856013in}{1.846706in}}{\pgfqpoint{2.856013in}{1.840882in}}%
\pgfpathcurveto{\pgfqpoint{2.856013in}{1.835058in}}{\pgfqpoint{2.858327in}{1.829472in}}{\pgfqpoint{2.862445in}{1.825354in}}%
\pgfpathcurveto{\pgfqpoint{2.866563in}{1.821236in}}{\pgfqpoint{2.872149in}{1.818922in}}{\pgfqpoint{2.877973in}{1.818922in}}%
\pgfpathlineto{\pgfqpoint{2.877973in}{1.818922in}}%
\pgfpathclose%
\pgfusepath{stroke,fill}%
\end{pgfscope}%
\begin{pgfscope}%
\pgfpathrectangle{\pgfqpoint{0.640323in}{0.527436in}}{\pgfqpoint{9.687500in}{3.850000in}}%
\pgfusepath{clip}%
\pgfsetbuttcap%
\pgfsetroundjoin%
\definecolor{currentfill}{rgb}{0.000000,0.000000,0.000000}%
\pgfsetfillcolor{currentfill}%
\pgfsetfillopacity{0.500000}%
\pgfsetlinewidth{1.003750pt}%
\definecolor{currentstroke}{rgb}{0.000000,0.000000,0.000000}%
\pgfsetstrokecolor{currentstroke}%
\pgfsetstrokeopacity{0.500000}%
\pgfsetdash{{3.700000pt}{1.600000pt}}{0.000000pt}%
\pgfpathmoveto{\pgfqpoint{3.057704in}{2.105666in}}%
\pgfpathcurveto{\pgfqpoint{3.063528in}{2.105666in}}{\pgfqpoint{3.069114in}{2.107980in}}{\pgfqpoint{3.073232in}{2.112098in}}%
\pgfpathcurveto{\pgfqpoint{3.077351in}{2.116216in}}{\pgfqpoint{3.079664in}{2.121802in}}{\pgfqpoint{3.079664in}{2.127626in}}%
\pgfpathcurveto{\pgfqpoint{3.079664in}{2.133450in}}{\pgfqpoint{3.077351in}{2.139036in}}{\pgfqpoint{3.073232in}{2.143154in}}%
\pgfpathcurveto{\pgfqpoint{3.069114in}{2.147273in}}{\pgfqpoint{3.063528in}{2.149586in}}{\pgfqpoint{3.057704in}{2.149586in}}%
\pgfpathcurveto{\pgfqpoint{3.051880in}{2.149586in}}{\pgfqpoint{3.046294in}{2.147273in}}{\pgfqpoint{3.042176in}{2.143154in}}%
\pgfpathcurveto{\pgfqpoint{3.038058in}{2.139036in}}{\pgfqpoint{3.035744in}{2.133450in}}{\pgfqpoint{3.035744in}{2.127626in}}%
\pgfpathcurveto{\pgfqpoint{3.035744in}{2.121802in}}{\pgfqpoint{3.038058in}{2.116216in}}{\pgfqpoint{3.042176in}{2.112098in}}%
\pgfpathcurveto{\pgfqpoint{3.046294in}{2.107980in}}{\pgfqpoint{3.051880in}{2.105666in}}{\pgfqpoint{3.057704in}{2.105666in}}%
\pgfpathlineto{\pgfqpoint{3.057704in}{2.105666in}}%
\pgfpathclose%
\pgfusepath{stroke,fill}%
\end{pgfscope}%
\begin{pgfscope}%
\pgfpathrectangle{\pgfqpoint{0.640323in}{0.527436in}}{\pgfqpoint{9.687500in}{3.850000in}}%
\pgfusepath{clip}%
\pgfsetbuttcap%
\pgfsetroundjoin%
\definecolor{currentfill}{rgb}{0.000000,0.000000,0.000000}%
\pgfsetfillcolor{currentfill}%
\pgfsetfillopacity{0.500000}%
\pgfsetlinewidth{1.003750pt}%
\definecolor{currentstroke}{rgb}{0.000000,0.000000,0.000000}%
\pgfsetstrokecolor{currentstroke}%
\pgfsetstrokeopacity{0.500000}%
\pgfsetdash{{3.700000pt}{1.600000pt}}{0.000000pt}%
\pgfpathmoveto{\pgfqpoint{3.237435in}{2.312846in}}%
\pgfpathcurveto{\pgfqpoint{3.243259in}{2.312846in}}{\pgfqpoint{3.248845in}{2.315160in}}{\pgfqpoint{3.252963in}{2.319278in}}%
\pgfpathcurveto{\pgfqpoint{3.257082in}{2.323396in}}{\pgfqpoint{3.259395in}{2.328982in}}{\pgfqpoint{3.259395in}{2.334806in}}%
\pgfpathcurveto{\pgfqpoint{3.259395in}{2.340630in}}{\pgfqpoint{3.257082in}{2.346216in}}{\pgfqpoint{3.252963in}{2.350334in}}%
\pgfpathcurveto{\pgfqpoint{3.248845in}{2.354452in}}{\pgfqpoint{3.243259in}{2.356766in}}{\pgfqpoint{3.237435in}{2.356766in}}%
\pgfpathcurveto{\pgfqpoint{3.231611in}{2.356766in}}{\pgfqpoint{3.226025in}{2.354452in}}{\pgfqpoint{3.221907in}{2.350334in}}%
\pgfpathcurveto{\pgfqpoint{3.217789in}{2.346216in}}{\pgfqpoint{3.215475in}{2.340630in}}{\pgfqpoint{3.215475in}{2.334806in}}%
\pgfpathcurveto{\pgfqpoint{3.215475in}{2.328982in}}{\pgfqpoint{3.217789in}{2.323396in}}{\pgfqpoint{3.221907in}{2.319278in}}%
\pgfpathcurveto{\pgfqpoint{3.226025in}{2.315160in}}{\pgfqpoint{3.231611in}{2.312846in}}{\pgfqpoint{3.237435in}{2.312846in}}%
\pgfpathlineto{\pgfqpoint{3.237435in}{2.312846in}}%
\pgfpathclose%
\pgfusepath{stroke,fill}%
\end{pgfscope}%
\begin{pgfscope}%
\pgfpathrectangle{\pgfqpoint{0.640323in}{0.527436in}}{\pgfqpoint{9.687500in}{3.850000in}}%
\pgfusepath{clip}%
\pgfsetbuttcap%
\pgfsetroundjoin%
\definecolor{currentfill}{rgb}{0.000000,0.000000,0.000000}%
\pgfsetfillcolor{currentfill}%
\pgfsetfillopacity{0.500000}%
\pgfsetlinewidth{1.003750pt}%
\definecolor{currentstroke}{rgb}{0.000000,0.000000,0.000000}%
\pgfsetstrokecolor{currentstroke}%
\pgfsetstrokeopacity{0.500000}%
\pgfsetdash{{3.700000pt}{1.600000pt}}{0.000000pt}%
\pgfpathmoveto{\pgfqpoint{3.417166in}{2.481823in}}%
\pgfpathcurveto{\pgfqpoint{3.422990in}{2.481823in}}{\pgfqpoint{3.428576in}{2.484137in}}{\pgfqpoint{3.432694in}{2.488255in}}%
\pgfpathcurveto{\pgfqpoint{3.436813in}{2.492374in}}{\pgfqpoint{3.439126in}{2.497960in}}{\pgfqpoint{3.439126in}{2.503784in}}%
\pgfpathcurveto{\pgfqpoint{3.439126in}{2.509608in}}{\pgfqpoint{3.436813in}{2.515194in}}{\pgfqpoint{3.432694in}{2.519312in}}%
\pgfpathcurveto{\pgfqpoint{3.428576in}{2.523430in}}{\pgfqpoint{3.422990in}{2.525744in}}{\pgfqpoint{3.417166in}{2.525744in}}%
\pgfpathcurveto{\pgfqpoint{3.411342in}{2.525744in}}{\pgfqpoint{3.405756in}{2.523430in}}{\pgfqpoint{3.401638in}{2.519312in}}%
\pgfpathcurveto{\pgfqpoint{3.397520in}{2.515194in}}{\pgfqpoint{3.395206in}{2.509608in}}{\pgfqpoint{3.395206in}{2.503784in}}%
\pgfpathcurveto{\pgfqpoint{3.395206in}{2.497960in}}{\pgfqpoint{3.397520in}{2.492374in}}{\pgfqpoint{3.401638in}{2.488255in}}%
\pgfpathcurveto{\pgfqpoint{3.405756in}{2.484137in}}{\pgfqpoint{3.411342in}{2.481823in}}{\pgfqpoint{3.417166in}{2.481823in}}%
\pgfpathlineto{\pgfqpoint{3.417166in}{2.481823in}}%
\pgfpathclose%
\pgfusepath{stroke,fill}%
\end{pgfscope}%
\begin{pgfscope}%
\pgfpathrectangle{\pgfqpoint{0.640323in}{0.527436in}}{\pgfqpoint{9.687500in}{3.850000in}}%
\pgfusepath{clip}%
\pgfsetbuttcap%
\pgfsetroundjoin%
\definecolor{currentfill}{rgb}{0.000000,0.000000,0.000000}%
\pgfsetfillcolor{currentfill}%
\pgfsetfillopacity{0.500000}%
\pgfsetlinewidth{1.003750pt}%
\definecolor{currentstroke}{rgb}{0.000000,0.000000,0.000000}%
\pgfsetstrokecolor{currentstroke}%
\pgfsetstrokeopacity{0.500000}%
\pgfsetdash{{3.700000pt}{1.600000pt}}{0.000000pt}%
\pgfpathmoveto{\pgfqpoint{3.596897in}{2.629663in}}%
\pgfpathcurveto{\pgfqpoint{3.602721in}{2.629663in}}{\pgfqpoint{3.608307in}{2.631977in}}{\pgfqpoint{3.612425in}{2.636095in}}%
\pgfpathcurveto{\pgfqpoint{3.616544in}{2.640214in}}{\pgfqpoint{3.618857in}{2.645800in}}{\pgfqpoint{3.618857in}{2.651624in}}%
\pgfpathcurveto{\pgfqpoint{3.618857in}{2.657448in}}{\pgfqpoint{3.616544in}{2.663034in}}{\pgfqpoint{3.612425in}{2.667152in}}%
\pgfpathcurveto{\pgfqpoint{3.608307in}{2.671270in}}{\pgfqpoint{3.602721in}{2.673584in}}{\pgfqpoint{3.596897in}{2.673584in}}%
\pgfpathcurveto{\pgfqpoint{3.591073in}{2.673584in}}{\pgfqpoint{3.585487in}{2.671270in}}{\pgfqpoint{3.581369in}{2.667152in}}%
\pgfpathcurveto{\pgfqpoint{3.577251in}{2.663034in}}{\pgfqpoint{3.574937in}{2.657448in}}{\pgfqpoint{3.574937in}{2.651624in}}%
\pgfpathcurveto{\pgfqpoint{3.574937in}{2.645800in}}{\pgfqpoint{3.577251in}{2.640214in}}{\pgfqpoint{3.581369in}{2.636095in}}%
\pgfpathcurveto{\pgfqpoint{3.585487in}{2.631977in}}{\pgfqpoint{3.591073in}{2.629663in}}{\pgfqpoint{3.596897in}{2.629663in}}%
\pgfpathlineto{\pgfqpoint{3.596897in}{2.629663in}}%
\pgfpathclose%
\pgfusepath{stroke,fill}%
\end{pgfscope}%
\begin{pgfscope}%
\pgfpathrectangle{\pgfqpoint{0.640323in}{0.527436in}}{\pgfqpoint{9.687500in}{3.850000in}}%
\pgfusepath{clip}%
\pgfsetbuttcap%
\pgfsetroundjoin%
\definecolor{currentfill}{rgb}{0.000000,0.000000,0.000000}%
\pgfsetfillcolor{currentfill}%
\pgfsetfillopacity{0.500000}%
\pgfsetlinewidth{1.003750pt}%
\definecolor{currentstroke}{rgb}{0.000000,0.000000,0.000000}%
\pgfsetstrokecolor{currentstroke}%
\pgfsetstrokeopacity{0.500000}%
\pgfsetdash{{3.700000pt}{1.600000pt}}{0.000000pt}%
\pgfpathmoveto{\pgfqpoint{3.776628in}{2.750510in}}%
\pgfpathcurveto{\pgfqpoint{3.782452in}{2.750510in}}{\pgfqpoint{3.788038in}{2.752824in}}{\pgfqpoint{3.792156in}{2.756942in}}%
\pgfpathcurveto{\pgfqpoint{3.796275in}{2.761060in}}{\pgfqpoint{3.798588in}{2.766646in}}{\pgfqpoint{3.798588in}{2.772470in}}%
\pgfpathcurveto{\pgfqpoint{3.798588in}{2.778294in}}{\pgfqpoint{3.796275in}{2.783880in}}{\pgfqpoint{3.792156in}{2.787998in}}%
\pgfpathcurveto{\pgfqpoint{3.788038in}{2.792117in}}{\pgfqpoint{3.782452in}{2.794430in}}{\pgfqpoint{3.776628in}{2.794430in}}%
\pgfpathcurveto{\pgfqpoint{3.770804in}{2.794430in}}{\pgfqpoint{3.765218in}{2.792117in}}{\pgfqpoint{3.761100in}{2.787998in}}%
\pgfpathcurveto{\pgfqpoint{3.756982in}{2.783880in}}{\pgfqpoint{3.754668in}{2.778294in}}{\pgfqpoint{3.754668in}{2.772470in}}%
\pgfpathcurveto{\pgfqpoint{3.754668in}{2.766646in}}{\pgfqpoint{3.756982in}{2.761060in}}{\pgfqpoint{3.761100in}{2.756942in}}%
\pgfpathcurveto{\pgfqpoint{3.765218in}{2.752824in}}{\pgfqpoint{3.770804in}{2.750510in}}{\pgfqpoint{3.776628in}{2.750510in}}%
\pgfpathlineto{\pgfqpoint{3.776628in}{2.750510in}}%
\pgfpathclose%
\pgfusepath{stroke,fill}%
\end{pgfscope}%
\begin{pgfscope}%
\pgfpathrectangle{\pgfqpoint{0.640323in}{0.527436in}}{\pgfqpoint{9.687500in}{3.850000in}}%
\pgfusepath{clip}%
\pgfsetbuttcap%
\pgfsetroundjoin%
\definecolor{currentfill}{rgb}{0.000000,0.000000,0.000000}%
\pgfsetfillcolor{currentfill}%
\pgfsetfillopacity{0.500000}%
\pgfsetlinewidth{1.003750pt}%
\definecolor{currentstroke}{rgb}{0.000000,0.000000,0.000000}%
\pgfsetstrokecolor{currentstroke}%
\pgfsetstrokeopacity{0.500000}%
\pgfsetdash{{3.700000pt}{1.600000pt}}{0.000000pt}%
\pgfpathmoveto{\pgfqpoint{3.956359in}{2.860483in}}%
\pgfpathcurveto{\pgfqpoint{3.962183in}{2.860483in}}{\pgfqpoint{3.967769in}{2.862797in}}{\pgfqpoint{3.971887in}{2.866915in}}%
\pgfpathcurveto{\pgfqpoint{3.976006in}{2.871033in}}{\pgfqpoint{3.978319in}{2.876620in}}{\pgfqpoint{3.978319in}{2.882444in}}%
\pgfpathcurveto{\pgfqpoint{3.978319in}{2.888268in}}{\pgfqpoint{3.976006in}{2.893854in}}{\pgfqpoint{3.971887in}{2.897972in}}%
\pgfpathcurveto{\pgfqpoint{3.967769in}{2.902090in}}{\pgfqpoint{3.962183in}{2.904404in}}{\pgfqpoint{3.956359in}{2.904404in}}%
\pgfpathcurveto{\pgfqpoint{3.950535in}{2.904404in}}{\pgfqpoint{3.944949in}{2.902090in}}{\pgfqpoint{3.940831in}{2.897972in}}%
\pgfpathcurveto{\pgfqpoint{3.936713in}{2.893854in}}{\pgfqpoint{3.934399in}{2.888268in}}{\pgfqpoint{3.934399in}{2.882444in}}%
\pgfpathcurveto{\pgfqpoint{3.934399in}{2.876620in}}{\pgfqpoint{3.936713in}{2.871033in}}{\pgfqpoint{3.940831in}{2.866915in}}%
\pgfpathcurveto{\pgfqpoint{3.944949in}{2.862797in}}{\pgfqpoint{3.950535in}{2.860483in}}{\pgfqpoint{3.956359in}{2.860483in}}%
\pgfpathlineto{\pgfqpoint{3.956359in}{2.860483in}}%
\pgfpathclose%
\pgfusepath{stroke,fill}%
\end{pgfscope}%
\begin{pgfscope}%
\pgfpathrectangle{\pgfqpoint{0.640323in}{0.527436in}}{\pgfqpoint{9.687500in}{3.850000in}}%
\pgfusepath{clip}%
\pgfsetbuttcap%
\pgfsetroundjoin%
\definecolor{currentfill}{rgb}{0.000000,0.000000,0.000000}%
\pgfsetfillcolor{currentfill}%
\pgfsetfillopacity{0.500000}%
\pgfsetlinewidth{1.003750pt}%
\definecolor{currentstroke}{rgb}{0.000000,0.000000,0.000000}%
\pgfsetstrokecolor{currentstroke}%
\pgfsetstrokeopacity{0.500000}%
\pgfsetdash{{3.700000pt}{1.600000pt}}{0.000000pt}%
\pgfpathmoveto{\pgfqpoint{4.136090in}{2.951499in}}%
\pgfpathcurveto{\pgfqpoint{4.141914in}{2.951499in}}{\pgfqpoint{4.147500in}{2.953812in}}{\pgfqpoint{4.151618in}{2.957931in}}%
\pgfpathcurveto{\pgfqpoint{4.155737in}{2.962049in}}{\pgfqpoint{4.158050in}{2.967635in}}{\pgfqpoint{4.158050in}{2.973459in}}%
\pgfpathcurveto{\pgfqpoint{4.158050in}{2.979283in}}{\pgfqpoint{4.155737in}{2.984869in}}{\pgfqpoint{4.151618in}{2.988987in}}%
\pgfpathcurveto{\pgfqpoint{4.147500in}{2.993105in}}{\pgfqpoint{4.141914in}{2.995419in}}{\pgfqpoint{4.136090in}{2.995419in}}%
\pgfpathcurveto{\pgfqpoint{4.130266in}{2.995419in}}{\pgfqpoint{4.124680in}{2.993105in}}{\pgfqpoint{4.120562in}{2.988987in}}%
\pgfpathcurveto{\pgfqpoint{4.116444in}{2.984869in}}{\pgfqpoint{4.114130in}{2.979283in}}{\pgfqpoint{4.114130in}{2.973459in}}%
\pgfpathcurveto{\pgfqpoint{4.114130in}{2.967635in}}{\pgfqpoint{4.116444in}{2.962049in}}{\pgfqpoint{4.120562in}{2.957931in}}%
\pgfpathcurveto{\pgfqpoint{4.124680in}{2.953812in}}{\pgfqpoint{4.130266in}{2.951499in}}{\pgfqpoint{4.136090in}{2.951499in}}%
\pgfpathlineto{\pgfqpoint{4.136090in}{2.951499in}}%
\pgfpathclose%
\pgfusepath{stroke,fill}%
\end{pgfscope}%
\begin{pgfscope}%
\pgfpathrectangle{\pgfqpoint{0.640323in}{0.527436in}}{\pgfqpoint{9.687500in}{3.850000in}}%
\pgfusepath{clip}%
\pgfsetbuttcap%
\pgfsetroundjoin%
\definecolor{currentfill}{rgb}{0.000000,0.000000,0.000000}%
\pgfsetfillcolor{currentfill}%
\pgfsetfillopacity{0.500000}%
\pgfsetlinewidth{1.003750pt}%
\definecolor{currentstroke}{rgb}{0.000000,0.000000,0.000000}%
\pgfsetstrokecolor{currentstroke}%
\pgfsetstrokeopacity{0.500000}%
\pgfsetdash{{3.700000pt}{1.600000pt}}{0.000000pt}%
\pgfpathmoveto{\pgfqpoint{4.315821in}{3.036372in}}%
\pgfpathcurveto{\pgfqpoint{4.321645in}{3.036372in}}{\pgfqpoint{4.327231in}{3.038686in}}{\pgfqpoint{4.331349in}{3.042804in}}%
\pgfpathcurveto{\pgfqpoint{4.335467in}{3.046923in}}{\pgfqpoint{4.337781in}{3.052509in}}{\pgfqpoint{4.337781in}{3.058333in}}%
\pgfpathcurveto{\pgfqpoint{4.337781in}{3.064157in}}{\pgfqpoint{4.335467in}{3.069743in}}{\pgfqpoint{4.331349in}{3.073861in}}%
\pgfpathcurveto{\pgfqpoint{4.327231in}{3.077979in}}{\pgfqpoint{4.321645in}{3.080293in}}{\pgfqpoint{4.315821in}{3.080293in}}%
\pgfpathcurveto{\pgfqpoint{4.309997in}{3.080293in}}{\pgfqpoint{4.304411in}{3.077979in}}{\pgfqpoint{4.300293in}{3.073861in}}%
\pgfpathcurveto{\pgfqpoint{4.296175in}{3.069743in}}{\pgfqpoint{4.293861in}{3.064157in}}{\pgfqpoint{4.293861in}{3.058333in}}%
\pgfpathcurveto{\pgfqpoint{4.293861in}{3.052509in}}{\pgfqpoint{4.296175in}{3.046923in}}{\pgfqpoint{4.300293in}{3.042804in}}%
\pgfpathcurveto{\pgfqpoint{4.304411in}{3.038686in}}{\pgfqpoint{4.309997in}{3.036372in}}{\pgfqpoint{4.315821in}{3.036372in}}%
\pgfpathlineto{\pgfqpoint{4.315821in}{3.036372in}}%
\pgfpathclose%
\pgfusepath{stroke,fill}%
\end{pgfscope}%
\begin{pgfscope}%
\pgfpathrectangle{\pgfqpoint{0.640323in}{0.527436in}}{\pgfqpoint{9.687500in}{3.850000in}}%
\pgfusepath{clip}%
\pgfsetbuttcap%
\pgfsetroundjoin%
\definecolor{currentfill}{rgb}{0.000000,0.000000,0.000000}%
\pgfsetfillcolor{currentfill}%
\pgfsetfillopacity{0.500000}%
\pgfsetlinewidth{1.003750pt}%
\definecolor{currentstroke}{rgb}{0.000000,0.000000,0.000000}%
\pgfsetstrokecolor{currentstroke}%
\pgfsetstrokeopacity{0.500000}%
\pgfsetdash{{3.700000pt}{1.600000pt}}{0.000000pt}%
\pgfpathmoveto{\pgfqpoint{4.495552in}{3.110274in}}%
\pgfpathcurveto{\pgfqpoint{4.501376in}{3.110274in}}{\pgfqpoint{4.506962in}{3.112588in}}{\pgfqpoint{4.511080in}{3.116706in}}%
\pgfpathcurveto{\pgfqpoint{4.515198in}{3.120824in}}{\pgfqpoint{4.517512in}{3.126410in}}{\pgfqpoint{4.517512in}{3.132234in}}%
\pgfpathcurveto{\pgfqpoint{4.517512in}{3.138058in}}{\pgfqpoint{4.515198in}{3.143644in}}{\pgfqpoint{4.511080in}{3.147762in}}%
\pgfpathcurveto{\pgfqpoint{4.506962in}{3.151880in}}{\pgfqpoint{4.501376in}{3.154194in}}{\pgfqpoint{4.495552in}{3.154194in}}%
\pgfpathcurveto{\pgfqpoint{4.489728in}{3.154194in}}{\pgfqpoint{4.484142in}{3.151880in}}{\pgfqpoint{4.480024in}{3.147762in}}%
\pgfpathcurveto{\pgfqpoint{4.475906in}{3.143644in}}{\pgfqpoint{4.473592in}{3.138058in}}{\pgfqpoint{4.473592in}{3.132234in}}%
\pgfpathcurveto{\pgfqpoint{4.473592in}{3.126410in}}{\pgfqpoint{4.475906in}{3.120824in}}{\pgfqpoint{4.480024in}{3.116706in}}%
\pgfpathcurveto{\pgfqpoint{4.484142in}{3.112588in}}{\pgfqpoint{4.489728in}{3.110274in}}{\pgfqpoint{4.495552in}{3.110274in}}%
\pgfpathlineto{\pgfqpoint{4.495552in}{3.110274in}}%
\pgfpathclose%
\pgfusepath{stroke,fill}%
\end{pgfscope}%
\begin{pgfscope}%
\pgfpathrectangle{\pgfqpoint{0.640323in}{0.527436in}}{\pgfqpoint{9.687500in}{3.850000in}}%
\pgfusepath{clip}%
\pgfsetbuttcap%
\pgfsetroundjoin%
\definecolor{currentfill}{rgb}{0.000000,0.000000,0.000000}%
\pgfsetfillcolor{currentfill}%
\pgfsetfillopacity{0.500000}%
\pgfsetlinewidth{1.003750pt}%
\definecolor{currentstroke}{rgb}{0.000000,0.000000,0.000000}%
\pgfsetstrokecolor{currentstroke}%
\pgfsetstrokeopacity{0.500000}%
\pgfsetdash{{3.700000pt}{1.600000pt}}{0.000000pt}%
\pgfpathmoveto{\pgfqpoint{4.675283in}{3.181493in}}%
\pgfpathcurveto{\pgfqpoint{4.681107in}{3.181493in}}{\pgfqpoint{4.686693in}{3.183806in}}{\pgfqpoint{4.690811in}{3.187925in}}%
\pgfpathcurveto{\pgfqpoint{4.694929in}{3.192043in}}{\pgfqpoint{4.697243in}{3.197629in}}{\pgfqpoint{4.697243in}{3.203453in}}%
\pgfpathcurveto{\pgfqpoint{4.697243in}{3.209277in}}{\pgfqpoint{4.694929in}{3.214863in}}{\pgfqpoint{4.690811in}{3.218981in}}%
\pgfpathcurveto{\pgfqpoint{4.686693in}{3.223099in}}{\pgfqpoint{4.681107in}{3.225413in}}{\pgfqpoint{4.675283in}{3.225413in}}%
\pgfpathcurveto{\pgfqpoint{4.669459in}{3.225413in}}{\pgfqpoint{4.663873in}{3.223099in}}{\pgfqpoint{4.659755in}{3.218981in}}%
\pgfpathcurveto{\pgfqpoint{4.655637in}{3.214863in}}{\pgfqpoint{4.653323in}{3.209277in}}{\pgfqpoint{4.653323in}{3.203453in}}%
\pgfpathcurveto{\pgfqpoint{4.653323in}{3.197629in}}{\pgfqpoint{4.655637in}{3.192043in}}{\pgfqpoint{4.659755in}{3.187925in}}%
\pgfpathcurveto{\pgfqpoint{4.663873in}{3.183806in}}{\pgfqpoint{4.669459in}{3.181493in}}{\pgfqpoint{4.675283in}{3.181493in}}%
\pgfpathlineto{\pgfqpoint{4.675283in}{3.181493in}}%
\pgfpathclose%
\pgfusepath{stroke,fill}%
\end{pgfscope}%
\begin{pgfscope}%
\pgfpathrectangle{\pgfqpoint{0.640323in}{0.527436in}}{\pgfqpoint{9.687500in}{3.850000in}}%
\pgfusepath{clip}%
\pgfsetbuttcap%
\pgfsetroundjoin%
\definecolor{currentfill}{rgb}{0.000000,0.000000,0.000000}%
\pgfsetfillcolor{currentfill}%
\pgfsetfillopacity{0.500000}%
\pgfsetlinewidth{1.003750pt}%
\definecolor{currentstroke}{rgb}{0.000000,0.000000,0.000000}%
\pgfsetstrokecolor{currentstroke}%
\pgfsetstrokeopacity{0.500000}%
\pgfsetdash{{3.700000pt}{1.600000pt}}{0.000000pt}%
\pgfpathmoveto{\pgfqpoint{4.855014in}{3.241199in}}%
\pgfpathcurveto{\pgfqpoint{4.860838in}{3.241199in}}{\pgfqpoint{4.866424in}{3.243513in}}{\pgfqpoint{4.870542in}{3.247631in}}%
\pgfpathcurveto{\pgfqpoint{4.874660in}{3.251749in}}{\pgfqpoint{4.876974in}{3.257335in}}{\pgfqpoint{4.876974in}{3.263159in}}%
\pgfpathcurveto{\pgfqpoint{4.876974in}{3.268983in}}{\pgfqpoint{4.874660in}{3.274569in}}{\pgfqpoint{4.870542in}{3.278687in}}%
\pgfpathcurveto{\pgfqpoint{4.866424in}{3.282805in}}{\pgfqpoint{4.860838in}{3.285119in}}{\pgfqpoint{4.855014in}{3.285119in}}%
\pgfpathcurveto{\pgfqpoint{4.849190in}{3.285119in}}{\pgfqpoint{4.843604in}{3.282805in}}{\pgfqpoint{4.839486in}{3.278687in}}%
\pgfpathcurveto{\pgfqpoint{4.835368in}{3.274569in}}{\pgfqpoint{4.833054in}{3.268983in}}{\pgfqpoint{4.833054in}{3.263159in}}%
\pgfpathcurveto{\pgfqpoint{4.833054in}{3.257335in}}{\pgfqpoint{4.835368in}{3.251749in}}{\pgfqpoint{4.839486in}{3.247631in}}%
\pgfpathcurveto{\pgfqpoint{4.843604in}{3.243513in}}{\pgfqpoint{4.849190in}{3.241199in}}{\pgfqpoint{4.855014in}{3.241199in}}%
\pgfpathlineto{\pgfqpoint{4.855014in}{3.241199in}}%
\pgfpathclose%
\pgfusepath{stroke,fill}%
\end{pgfscope}%
\begin{pgfscope}%
\pgfpathrectangle{\pgfqpoint{0.640323in}{0.527436in}}{\pgfqpoint{9.687500in}{3.850000in}}%
\pgfusepath{clip}%
\pgfsetbuttcap%
\pgfsetroundjoin%
\definecolor{currentfill}{rgb}{0.000000,0.000000,0.000000}%
\pgfsetfillcolor{currentfill}%
\pgfsetfillopacity{0.500000}%
\pgfsetlinewidth{1.003750pt}%
\definecolor{currentstroke}{rgb}{0.000000,0.000000,0.000000}%
\pgfsetstrokecolor{currentstroke}%
\pgfsetstrokeopacity{0.500000}%
\pgfsetdash{{3.700000pt}{1.600000pt}}{0.000000pt}%
\pgfpathmoveto{\pgfqpoint{5.034745in}{3.293813in}}%
\pgfpathcurveto{\pgfqpoint{5.040569in}{3.293813in}}{\pgfqpoint{5.046155in}{3.296127in}}{\pgfqpoint{5.050273in}{3.300245in}}%
\pgfpathcurveto{\pgfqpoint{5.054391in}{3.304363in}}{\pgfqpoint{5.056705in}{3.309950in}}{\pgfqpoint{5.056705in}{3.315773in}}%
\pgfpathcurveto{\pgfqpoint{5.056705in}{3.321597in}}{\pgfqpoint{5.054391in}{3.327184in}}{\pgfqpoint{5.050273in}{3.331302in}}%
\pgfpathcurveto{\pgfqpoint{5.046155in}{3.335420in}}{\pgfqpoint{5.040569in}{3.337734in}}{\pgfqpoint{5.034745in}{3.337734in}}%
\pgfpathcurveto{\pgfqpoint{5.028921in}{3.337734in}}{\pgfqpoint{5.023335in}{3.335420in}}{\pgfqpoint{5.019217in}{3.331302in}}%
\pgfpathcurveto{\pgfqpoint{5.015099in}{3.327184in}}{\pgfqpoint{5.012785in}{3.321597in}}{\pgfqpoint{5.012785in}{3.315773in}}%
\pgfpathcurveto{\pgfqpoint{5.012785in}{3.309950in}}{\pgfqpoint{5.015099in}{3.304363in}}{\pgfqpoint{5.019217in}{3.300245in}}%
\pgfpathcurveto{\pgfqpoint{5.023335in}{3.296127in}}{\pgfqpoint{5.028921in}{3.293813in}}{\pgfqpoint{5.034745in}{3.293813in}}%
\pgfpathlineto{\pgfqpoint{5.034745in}{3.293813in}}%
\pgfpathclose%
\pgfusepath{stroke,fill}%
\end{pgfscope}%
\begin{pgfscope}%
\pgfpathrectangle{\pgfqpoint{0.640323in}{0.527436in}}{\pgfqpoint{9.687500in}{3.850000in}}%
\pgfusepath{clip}%
\pgfsetbuttcap%
\pgfsetroundjoin%
\definecolor{currentfill}{rgb}{0.000000,0.000000,0.000000}%
\pgfsetfillcolor{currentfill}%
\pgfsetfillopacity{0.500000}%
\pgfsetlinewidth{1.003750pt}%
\definecolor{currentstroke}{rgb}{0.000000,0.000000,0.000000}%
\pgfsetstrokecolor{currentstroke}%
\pgfsetstrokeopacity{0.500000}%
\pgfsetdash{{3.700000pt}{1.600000pt}}{0.000000pt}%
\pgfpathmoveto{\pgfqpoint{5.214476in}{3.351414in}}%
\pgfpathcurveto{\pgfqpoint{5.220300in}{3.351414in}}{\pgfqpoint{5.225886in}{3.353728in}}{\pgfqpoint{5.230004in}{3.357846in}}%
\pgfpathcurveto{\pgfqpoint{5.234122in}{3.361964in}}{\pgfqpoint{5.236436in}{3.367551in}}{\pgfqpoint{5.236436in}{3.373374in}}%
\pgfpathcurveto{\pgfqpoint{5.236436in}{3.379198in}}{\pgfqpoint{5.234122in}{3.384785in}}{\pgfqpoint{5.230004in}{3.388903in}}%
\pgfpathcurveto{\pgfqpoint{5.225886in}{3.393021in}}{\pgfqpoint{5.220300in}{3.395335in}}{\pgfqpoint{5.214476in}{3.395335in}}%
\pgfpathcurveto{\pgfqpoint{5.208652in}{3.395335in}}{\pgfqpoint{5.203066in}{3.393021in}}{\pgfqpoint{5.198948in}{3.388903in}}%
\pgfpathcurveto{\pgfqpoint{5.194830in}{3.384785in}}{\pgfqpoint{5.192516in}{3.379198in}}{\pgfqpoint{5.192516in}{3.373374in}}%
\pgfpathcurveto{\pgfqpoint{5.192516in}{3.367551in}}{\pgfqpoint{5.194830in}{3.361964in}}{\pgfqpoint{5.198948in}{3.357846in}}%
\pgfpathcurveto{\pgfqpoint{5.203066in}{3.353728in}}{\pgfqpoint{5.208652in}{3.351414in}}{\pgfqpoint{5.214476in}{3.351414in}}%
\pgfpathlineto{\pgfqpoint{5.214476in}{3.351414in}}%
\pgfpathclose%
\pgfusepath{stroke,fill}%
\end{pgfscope}%
\begin{pgfscope}%
\pgfpathrectangle{\pgfqpoint{0.640323in}{0.527436in}}{\pgfqpoint{9.687500in}{3.850000in}}%
\pgfusepath{clip}%
\pgfsetbuttcap%
\pgfsetroundjoin%
\definecolor{currentfill}{rgb}{0.000000,0.000000,0.000000}%
\pgfsetfillcolor{currentfill}%
\pgfsetfillopacity{0.500000}%
\pgfsetlinewidth{1.003750pt}%
\definecolor{currentstroke}{rgb}{0.000000,0.000000,0.000000}%
\pgfsetstrokecolor{currentstroke}%
\pgfsetstrokeopacity{0.500000}%
\pgfsetdash{{3.700000pt}{1.600000pt}}{0.000000pt}%
\pgfpathmoveto{\pgfqpoint{5.394207in}{3.399415in}}%
\pgfpathcurveto{\pgfqpoint{5.400031in}{3.399415in}}{\pgfqpoint{5.405617in}{3.401729in}}{\pgfqpoint{5.409735in}{3.405847in}}%
\pgfpathcurveto{\pgfqpoint{5.413853in}{3.409965in}}{\pgfqpoint{5.416167in}{3.415551in}}{\pgfqpoint{5.416167in}{3.421375in}}%
\pgfpathcurveto{\pgfqpoint{5.416167in}{3.427199in}}{\pgfqpoint{5.413853in}{3.432785in}}{\pgfqpoint{5.409735in}{3.436904in}}%
\pgfpathcurveto{\pgfqpoint{5.405617in}{3.441022in}}{\pgfqpoint{5.400031in}{3.443336in}}{\pgfqpoint{5.394207in}{3.443336in}}%
\pgfpathcurveto{\pgfqpoint{5.388383in}{3.443336in}}{\pgfqpoint{5.382797in}{3.441022in}}{\pgfqpoint{5.378679in}{3.436904in}}%
\pgfpathcurveto{\pgfqpoint{5.374561in}{3.432785in}}{\pgfqpoint{5.372247in}{3.427199in}}{\pgfqpoint{5.372247in}{3.421375in}}%
\pgfpathcurveto{\pgfqpoint{5.372247in}{3.415551in}}{\pgfqpoint{5.374561in}{3.409965in}}{\pgfqpoint{5.378679in}{3.405847in}}%
\pgfpathcurveto{\pgfqpoint{5.382797in}{3.401729in}}{\pgfqpoint{5.388383in}{3.399415in}}{\pgfqpoint{5.394207in}{3.399415in}}%
\pgfpathlineto{\pgfqpoint{5.394207in}{3.399415in}}%
\pgfpathclose%
\pgfusepath{stroke,fill}%
\end{pgfscope}%
\begin{pgfscope}%
\pgfpathrectangle{\pgfqpoint{0.640323in}{0.527436in}}{\pgfqpoint{9.687500in}{3.850000in}}%
\pgfusepath{clip}%
\pgfsetbuttcap%
\pgfsetroundjoin%
\definecolor{currentfill}{rgb}{0.000000,0.000000,0.000000}%
\pgfsetfillcolor{currentfill}%
\pgfsetfillopacity{0.500000}%
\pgfsetlinewidth{1.003750pt}%
\definecolor{currentstroke}{rgb}{0.000000,0.000000,0.000000}%
\pgfsetstrokecolor{currentstroke}%
\pgfsetstrokeopacity{0.500000}%
\pgfsetdash{{3.700000pt}{1.600000pt}}{0.000000pt}%
\pgfpathmoveto{\pgfqpoint{5.573938in}{3.443740in}}%
\pgfpathcurveto{\pgfqpoint{5.579762in}{3.443740in}}{\pgfqpoint{5.585348in}{3.446054in}}{\pgfqpoint{5.589466in}{3.450172in}}%
\pgfpathcurveto{\pgfqpoint{5.593584in}{3.454290in}}{\pgfqpoint{5.595898in}{3.459876in}}{\pgfqpoint{5.595898in}{3.465700in}}%
\pgfpathcurveto{\pgfqpoint{5.595898in}{3.471524in}}{\pgfqpoint{5.593584in}{3.477110in}}{\pgfqpoint{5.589466in}{3.481228in}}%
\pgfpathcurveto{\pgfqpoint{5.585348in}{3.485346in}}{\pgfqpoint{5.579762in}{3.487660in}}{\pgfqpoint{5.573938in}{3.487660in}}%
\pgfpathcurveto{\pgfqpoint{5.568114in}{3.487660in}}{\pgfqpoint{5.562528in}{3.485346in}}{\pgfqpoint{5.558410in}{3.481228in}}%
\pgfpathcurveto{\pgfqpoint{5.554292in}{3.477110in}}{\pgfqpoint{5.551978in}{3.471524in}}{\pgfqpoint{5.551978in}{3.465700in}}%
\pgfpathcurveto{\pgfqpoint{5.551978in}{3.459876in}}{\pgfqpoint{5.554292in}{3.454290in}}{\pgfqpoint{5.558410in}{3.450172in}}%
\pgfpathcurveto{\pgfqpoint{5.562528in}{3.446054in}}{\pgfqpoint{5.568114in}{3.443740in}}{\pgfqpoint{5.573938in}{3.443740in}}%
\pgfpathlineto{\pgfqpoint{5.573938in}{3.443740in}}%
\pgfpathclose%
\pgfusepath{stroke,fill}%
\end{pgfscope}%
\begin{pgfscope}%
\pgfpathrectangle{\pgfqpoint{0.640323in}{0.527436in}}{\pgfqpoint{9.687500in}{3.850000in}}%
\pgfusepath{clip}%
\pgfsetbuttcap%
\pgfsetroundjoin%
\definecolor{currentfill}{rgb}{0.000000,0.000000,0.000000}%
\pgfsetfillcolor{currentfill}%
\pgfsetfillopacity{0.500000}%
\pgfsetlinewidth{1.003750pt}%
\definecolor{currentstroke}{rgb}{0.000000,0.000000,0.000000}%
\pgfsetstrokecolor{currentstroke}%
\pgfsetstrokeopacity{0.500000}%
\pgfsetdash{{3.700000pt}{1.600000pt}}{0.000000pt}%
\pgfpathmoveto{\pgfqpoint{5.753669in}{3.487071in}}%
\pgfpathcurveto{\pgfqpoint{5.759493in}{3.487071in}}{\pgfqpoint{5.765079in}{3.489385in}}{\pgfqpoint{5.769197in}{3.493503in}}%
\pgfpathcurveto{\pgfqpoint{5.773315in}{3.497621in}}{\pgfqpoint{5.775629in}{3.503207in}}{\pgfqpoint{5.775629in}{3.509031in}}%
\pgfpathcurveto{\pgfqpoint{5.775629in}{3.514855in}}{\pgfqpoint{5.773315in}{3.520441in}}{\pgfqpoint{5.769197in}{3.524559in}}%
\pgfpathcurveto{\pgfqpoint{5.765079in}{3.528677in}}{\pgfqpoint{5.759493in}{3.530991in}}{\pgfqpoint{5.753669in}{3.530991in}}%
\pgfpathcurveto{\pgfqpoint{5.747845in}{3.530991in}}{\pgfqpoint{5.742259in}{3.528677in}}{\pgfqpoint{5.738141in}{3.524559in}}%
\pgfpathcurveto{\pgfqpoint{5.734023in}{3.520441in}}{\pgfqpoint{5.731709in}{3.514855in}}{\pgfqpoint{5.731709in}{3.509031in}}%
\pgfpathcurveto{\pgfqpoint{5.731709in}{3.503207in}}{\pgfqpoint{5.734023in}{3.497621in}}{\pgfqpoint{5.738141in}{3.493503in}}%
\pgfpathcurveto{\pgfqpoint{5.742259in}{3.489385in}}{\pgfqpoint{5.747845in}{3.487071in}}{\pgfqpoint{5.753669in}{3.487071in}}%
\pgfpathlineto{\pgfqpoint{5.753669in}{3.487071in}}%
\pgfpathclose%
\pgfusepath{stroke,fill}%
\end{pgfscope}%
\begin{pgfscope}%
\pgfpathrectangle{\pgfqpoint{0.640323in}{0.527436in}}{\pgfqpoint{9.687500in}{3.850000in}}%
\pgfusepath{clip}%
\pgfsetbuttcap%
\pgfsetroundjoin%
\definecolor{currentfill}{rgb}{0.000000,0.000000,0.000000}%
\pgfsetfillcolor{currentfill}%
\pgfsetfillopacity{0.500000}%
\pgfsetlinewidth{1.003750pt}%
\definecolor{currentstroke}{rgb}{0.000000,0.000000,0.000000}%
\pgfsetstrokecolor{currentstroke}%
\pgfsetstrokeopacity{0.500000}%
\pgfsetdash{{3.700000pt}{1.600000pt}}{0.000000pt}%
\pgfpathmoveto{\pgfqpoint{5.933400in}{3.527396in}}%
\pgfpathcurveto{\pgfqpoint{5.939224in}{3.527396in}}{\pgfqpoint{5.944810in}{3.529710in}}{\pgfqpoint{5.948928in}{3.533828in}}%
\pgfpathcurveto{\pgfqpoint{5.953046in}{3.537947in}}{\pgfqpoint{5.955360in}{3.543533in}}{\pgfqpoint{5.955360in}{3.549357in}}%
\pgfpathcurveto{\pgfqpoint{5.955360in}{3.555181in}}{\pgfqpoint{5.953046in}{3.560767in}}{\pgfqpoint{5.948928in}{3.564885in}}%
\pgfpathcurveto{\pgfqpoint{5.944810in}{3.569003in}}{\pgfqpoint{5.939224in}{3.571317in}}{\pgfqpoint{5.933400in}{3.571317in}}%
\pgfpathcurveto{\pgfqpoint{5.927576in}{3.571317in}}{\pgfqpoint{5.921990in}{3.569003in}}{\pgfqpoint{5.917872in}{3.564885in}}%
\pgfpathcurveto{\pgfqpoint{5.913754in}{3.560767in}}{\pgfqpoint{5.911440in}{3.555181in}}{\pgfqpoint{5.911440in}{3.549357in}}%
\pgfpathcurveto{\pgfqpoint{5.911440in}{3.543533in}}{\pgfqpoint{5.913754in}{3.537947in}}{\pgfqpoint{5.917872in}{3.533828in}}%
\pgfpathcurveto{\pgfqpoint{5.921990in}{3.529710in}}{\pgfqpoint{5.927576in}{3.527396in}}{\pgfqpoint{5.933400in}{3.527396in}}%
\pgfpathlineto{\pgfqpoint{5.933400in}{3.527396in}}%
\pgfpathclose%
\pgfusepath{stroke,fill}%
\end{pgfscope}%
\begin{pgfscope}%
\pgfpathrectangle{\pgfqpoint{0.640323in}{0.527436in}}{\pgfqpoint{9.687500in}{3.850000in}}%
\pgfusepath{clip}%
\pgfsetbuttcap%
\pgfsetroundjoin%
\definecolor{currentfill}{rgb}{0.000000,0.000000,0.000000}%
\pgfsetfillcolor{currentfill}%
\pgfsetfillopacity{0.500000}%
\pgfsetlinewidth{1.003750pt}%
\definecolor{currentstroke}{rgb}{0.000000,0.000000,0.000000}%
\pgfsetstrokecolor{currentstroke}%
\pgfsetstrokeopacity{0.500000}%
\pgfsetdash{{3.700000pt}{1.600000pt}}{0.000000pt}%
\pgfpathmoveto{\pgfqpoint{6.113131in}{3.564270in}}%
\pgfpathcurveto{\pgfqpoint{6.118955in}{3.564270in}}{\pgfqpoint{6.124541in}{3.566583in}}{\pgfqpoint{6.128659in}{3.570702in}}%
\pgfpathcurveto{\pgfqpoint{6.132777in}{3.574820in}}{\pgfqpoint{6.135091in}{3.580406in}}{\pgfqpoint{6.135091in}{3.586230in}}%
\pgfpathcurveto{\pgfqpoint{6.135091in}{3.592054in}}{\pgfqpoint{6.132777in}{3.597640in}}{\pgfqpoint{6.128659in}{3.601758in}}%
\pgfpathcurveto{\pgfqpoint{6.124541in}{3.605876in}}{\pgfqpoint{6.118955in}{3.608190in}}{\pgfqpoint{6.113131in}{3.608190in}}%
\pgfpathcurveto{\pgfqpoint{6.107307in}{3.608190in}}{\pgfqpoint{6.101721in}{3.605876in}}{\pgfqpoint{6.097603in}{3.601758in}}%
\pgfpathcurveto{\pgfqpoint{6.093485in}{3.597640in}}{\pgfqpoint{6.091171in}{3.592054in}}{\pgfqpoint{6.091171in}{3.586230in}}%
\pgfpathcurveto{\pgfqpoint{6.091171in}{3.580406in}}{\pgfqpoint{6.093485in}{3.574820in}}{\pgfqpoint{6.097603in}{3.570702in}}%
\pgfpathcurveto{\pgfqpoint{6.101721in}{3.566583in}}{\pgfqpoint{6.107307in}{3.564270in}}{\pgfqpoint{6.113131in}{3.564270in}}%
\pgfpathlineto{\pgfqpoint{6.113131in}{3.564270in}}%
\pgfpathclose%
\pgfusepath{stroke,fill}%
\end{pgfscope}%
\begin{pgfscope}%
\pgfpathrectangle{\pgfqpoint{0.640323in}{0.527436in}}{\pgfqpoint{9.687500in}{3.850000in}}%
\pgfusepath{clip}%
\pgfsetbuttcap%
\pgfsetroundjoin%
\definecolor{currentfill}{rgb}{0.000000,0.000000,0.000000}%
\pgfsetfillcolor{currentfill}%
\pgfsetfillopacity{0.500000}%
\pgfsetlinewidth{1.003750pt}%
\definecolor{currentstroke}{rgb}{0.000000,0.000000,0.000000}%
\pgfsetstrokecolor{currentstroke}%
\pgfsetstrokeopacity{0.500000}%
\pgfsetdash{{3.700000pt}{1.600000pt}}{0.000000pt}%
\pgfpathmoveto{\pgfqpoint{6.292862in}{3.599509in}}%
\pgfpathcurveto{\pgfqpoint{6.298686in}{3.599509in}}{\pgfqpoint{6.304272in}{3.601823in}}{\pgfqpoint{6.308390in}{3.605941in}}%
\pgfpathcurveto{\pgfqpoint{6.312508in}{3.610060in}}{\pgfqpoint{6.314822in}{3.615646in}}{\pgfqpoint{6.314822in}{3.621470in}}%
\pgfpathcurveto{\pgfqpoint{6.314822in}{3.627294in}}{\pgfqpoint{6.312508in}{3.632880in}}{\pgfqpoint{6.308390in}{3.636998in}}%
\pgfpathcurveto{\pgfqpoint{6.304272in}{3.641116in}}{\pgfqpoint{6.298686in}{3.643430in}}{\pgfqpoint{6.292862in}{3.643430in}}%
\pgfpathcurveto{\pgfqpoint{6.287038in}{3.643430in}}{\pgfqpoint{6.281452in}{3.641116in}}{\pgfqpoint{6.277334in}{3.636998in}}%
\pgfpathcurveto{\pgfqpoint{6.273216in}{3.632880in}}{\pgfqpoint{6.270902in}{3.627294in}}{\pgfqpoint{6.270902in}{3.621470in}}%
\pgfpathcurveto{\pgfqpoint{6.270902in}{3.615646in}}{\pgfqpoint{6.273216in}{3.610060in}}{\pgfqpoint{6.277334in}{3.605941in}}%
\pgfpathcurveto{\pgfqpoint{6.281452in}{3.601823in}}{\pgfqpoint{6.287038in}{3.599509in}}{\pgfqpoint{6.292862in}{3.599509in}}%
\pgfpathlineto{\pgfqpoint{6.292862in}{3.599509in}}%
\pgfpathclose%
\pgfusepath{stroke,fill}%
\end{pgfscope}%
\begin{pgfscope}%
\pgfpathrectangle{\pgfqpoint{0.640323in}{0.527436in}}{\pgfqpoint{9.687500in}{3.850000in}}%
\pgfusepath{clip}%
\pgfsetbuttcap%
\pgfsetroundjoin%
\definecolor{currentfill}{rgb}{0.000000,0.000000,0.000000}%
\pgfsetfillcolor{currentfill}%
\pgfsetfillopacity{0.500000}%
\pgfsetlinewidth{1.003750pt}%
\definecolor{currentstroke}{rgb}{0.000000,0.000000,0.000000}%
\pgfsetstrokecolor{currentstroke}%
\pgfsetstrokeopacity{0.500000}%
\pgfsetdash{{3.700000pt}{1.600000pt}}{0.000000pt}%
\pgfpathmoveto{\pgfqpoint{6.472593in}{3.632222in}}%
\pgfpathcurveto{\pgfqpoint{6.478417in}{3.632222in}}{\pgfqpoint{6.484003in}{3.634536in}}{\pgfqpoint{6.488121in}{3.638654in}}%
\pgfpathcurveto{\pgfqpoint{6.492239in}{3.642772in}}{\pgfqpoint{6.494553in}{3.648358in}}{\pgfqpoint{6.494553in}{3.654182in}}%
\pgfpathcurveto{\pgfqpoint{6.494553in}{3.660006in}}{\pgfqpoint{6.492239in}{3.665592in}}{\pgfqpoint{6.488121in}{3.669711in}}%
\pgfpathcurveto{\pgfqpoint{6.484003in}{3.673829in}}{\pgfqpoint{6.478417in}{3.676143in}}{\pgfqpoint{6.472593in}{3.676143in}}%
\pgfpathcurveto{\pgfqpoint{6.466769in}{3.676143in}}{\pgfqpoint{6.461183in}{3.673829in}}{\pgfqpoint{6.457065in}{3.669711in}}%
\pgfpathcurveto{\pgfqpoint{6.452947in}{3.665592in}}{\pgfqpoint{6.450633in}{3.660006in}}{\pgfqpoint{6.450633in}{3.654182in}}%
\pgfpathcurveto{\pgfqpoint{6.450633in}{3.648358in}}{\pgfqpoint{6.452947in}{3.642772in}}{\pgfqpoint{6.457065in}{3.638654in}}%
\pgfpathcurveto{\pgfqpoint{6.461183in}{3.634536in}}{\pgfqpoint{6.466769in}{3.632222in}}{\pgfqpoint{6.472593in}{3.632222in}}%
\pgfpathlineto{\pgfqpoint{6.472593in}{3.632222in}}%
\pgfpathclose%
\pgfusepath{stroke,fill}%
\end{pgfscope}%
\begin{pgfscope}%
\pgfpathrectangle{\pgfqpoint{0.640323in}{0.527436in}}{\pgfqpoint{9.687500in}{3.850000in}}%
\pgfusepath{clip}%
\pgfsetbuttcap%
\pgfsetroundjoin%
\definecolor{currentfill}{rgb}{0.000000,0.000000,0.000000}%
\pgfsetfillcolor{currentfill}%
\pgfsetfillopacity{0.500000}%
\pgfsetlinewidth{1.003750pt}%
\definecolor{currentstroke}{rgb}{0.000000,0.000000,0.000000}%
\pgfsetstrokecolor{currentstroke}%
\pgfsetstrokeopacity{0.500000}%
\pgfsetdash{{3.700000pt}{1.600000pt}}{0.000000pt}%
\pgfpathmoveto{\pgfqpoint{6.652324in}{3.659402in}}%
\pgfpathcurveto{\pgfqpoint{6.658148in}{3.659402in}}{\pgfqpoint{6.663734in}{3.661716in}}{\pgfqpoint{6.667852in}{3.665834in}}%
\pgfpathcurveto{\pgfqpoint{6.671970in}{3.669952in}}{\pgfqpoint{6.674284in}{3.675538in}}{\pgfqpoint{6.674284in}{3.681362in}}%
\pgfpathcurveto{\pgfqpoint{6.674284in}{3.687186in}}{\pgfqpoint{6.671970in}{3.692772in}}{\pgfqpoint{6.667852in}{3.696890in}}%
\pgfpathcurveto{\pgfqpoint{6.663734in}{3.701008in}}{\pgfqpoint{6.658148in}{3.703322in}}{\pgfqpoint{6.652324in}{3.703322in}}%
\pgfpathcurveto{\pgfqpoint{6.646500in}{3.703322in}}{\pgfqpoint{6.640914in}{3.701008in}}{\pgfqpoint{6.636796in}{3.696890in}}%
\pgfpathcurveto{\pgfqpoint{6.632678in}{3.692772in}}{\pgfqpoint{6.630364in}{3.687186in}}{\pgfqpoint{6.630364in}{3.681362in}}%
\pgfpathcurveto{\pgfqpoint{6.630364in}{3.675538in}}{\pgfqpoint{6.632678in}{3.669952in}}{\pgfqpoint{6.636796in}{3.665834in}}%
\pgfpathcurveto{\pgfqpoint{6.640914in}{3.661716in}}{\pgfqpoint{6.646500in}{3.659402in}}{\pgfqpoint{6.652324in}{3.659402in}}%
\pgfpathlineto{\pgfqpoint{6.652324in}{3.659402in}}%
\pgfpathclose%
\pgfusepath{stroke,fill}%
\end{pgfscope}%
\begin{pgfscope}%
\pgfpathrectangle{\pgfqpoint{0.640323in}{0.527436in}}{\pgfqpoint{9.687500in}{3.850000in}}%
\pgfusepath{clip}%
\pgfsetbuttcap%
\pgfsetroundjoin%
\definecolor{currentfill}{rgb}{0.000000,0.000000,0.000000}%
\pgfsetfillcolor{currentfill}%
\pgfsetfillopacity{0.500000}%
\pgfsetlinewidth{1.003750pt}%
\definecolor{currentstroke}{rgb}{0.000000,0.000000,0.000000}%
\pgfsetstrokecolor{currentstroke}%
\pgfsetstrokeopacity{0.500000}%
\pgfsetdash{{3.700000pt}{1.600000pt}}{0.000000pt}%
\pgfpathmoveto{\pgfqpoint{6.832055in}{3.690599in}}%
\pgfpathcurveto{\pgfqpoint{6.837879in}{3.690599in}}{\pgfqpoint{6.843465in}{3.692913in}}{\pgfqpoint{6.847583in}{3.697031in}}%
\pgfpathcurveto{\pgfqpoint{6.851701in}{3.701149in}}{\pgfqpoint{6.854015in}{3.706736in}}{\pgfqpoint{6.854015in}{3.712559in}}%
\pgfpathcurveto{\pgfqpoint{6.854015in}{3.718383in}}{\pgfqpoint{6.851701in}{3.723970in}}{\pgfqpoint{6.847583in}{3.728088in}}%
\pgfpathcurveto{\pgfqpoint{6.843465in}{3.732206in}}{\pgfqpoint{6.837879in}{3.734520in}}{\pgfqpoint{6.832055in}{3.734520in}}%
\pgfpathcurveto{\pgfqpoint{6.826231in}{3.734520in}}{\pgfqpoint{6.820645in}{3.732206in}}{\pgfqpoint{6.816527in}{3.728088in}}%
\pgfpathcurveto{\pgfqpoint{6.812408in}{3.723970in}}{\pgfqpoint{6.810095in}{3.718383in}}{\pgfqpoint{6.810095in}{3.712559in}}%
\pgfpathcurveto{\pgfqpoint{6.810095in}{3.706736in}}{\pgfqpoint{6.812408in}{3.701149in}}{\pgfqpoint{6.816527in}{3.697031in}}%
\pgfpathcurveto{\pgfqpoint{6.820645in}{3.692913in}}{\pgfqpoint{6.826231in}{3.690599in}}{\pgfqpoint{6.832055in}{3.690599in}}%
\pgfpathlineto{\pgfqpoint{6.832055in}{3.690599in}}%
\pgfpathclose%
\pgfusepath{stroke,fill}%
\end{pgfscope}%
\begin{pgfscope}%
\pgfpathrectangle{\pgfqpoint{0.640323in}{0.527436in}}{\pgfqpoint{9.687500in}{3.850000in}}%
\pgfusepath{clip}%
\pgfsetbuttcap%
\pgfsetroundjoin%
\definecolor{currentfill}{rgb}{0.000000,0.000000,0.000000}%
\pgfsetfillcolor{currentfill}%
\pgfsetfillopacity{0.500000}%
\pgfsetlinewidth{1.003750pt}%
\definecolor{currentstroke}{rgb}{0.000000,0.000000,0.000000}%
\pgfsetstrokecolor{currentstroke}%
\pgfsetstrokeopacity{0.500000}%
\pgfsetdash{{3.700000pt}{1.600000pt}}{0.000000pt}%
\pgfpathmoveto{\pgfqpoint{7.011786in}{3.717108in}}%
\pgfpathcurveto{\pgfqpoint{7.017610in}{3.717108in}}{\pgfqpoint{7.023196in}{3.719422in}}{\pgfqpoint{7.027314in}{3.723540in}}%
\pgfpathcurveto{\pgfqpoint{7.031432in}{3.727658in}}{\pgfqpoint{7.033746in}{3.733245in}}{\pgfqpoint{7.033746in}{3.739069in}}%
\pgfpathcurveto{\pgfqpoint{7.033746in}{3.744893in}}{\pgfqpoint{7.031432in}{3.750479in}}{\pgfqpoint{7.027314in}{3.754597in}}%
\pgfpathcurveto{\pgfqpoint{7.023196in}{3.758715in}}{\pgfqpoint{7.017610in}{3.761029in}}{\pgfqpoint{7.011786in}{3.761029in}}%
\pgfpathcurveto{\pgfqpoint{7.005962in}{3.761029in}}{\pgfqpoint{7.000376in}{3.758715in}}{\pgfqpoint{6.996258in}{3.754597in}}%
\pgfpathcurveto{\pgfqpoint{6.992139in}{3.750479in}}{\pgfqpoint{6.989826in}{3.744893in}}{\pgfqpoint{6.989826in}{3.739069in}}%
\pgfpathcurveto{\pgfqpoint{6.989826in}{3.733245in}}{\pgfqpoint{6.992139in}{3.727658in}}{\pgfqpoint{6.996258in}{3.723540in}}%
\pgfpathcurveto{\pgfqpoint{7.000376in}{3.719422in}}{\pgfqpoint{7.005962in}{3.717108in}}{\pgfqpoint{7.011786in}{3.717108in}}%
\pgfpathlineto{\pgfqpoint{7.011786in}{3.717108in}}%
\pgfpathclose%
\pgfusepath{stroke,fill}%
\end{pgfscope}%
\begin{pgfscope}%
\pgfpathrectangle{\pgfqpoint{0.640323in}{0.527436in}}{\pgfqpoint{9.687500in}{3.850000in}}%
\pgfusepath{clip}%
\pgfsetbuttcap%
\pgfsetroundjoin%
\definecolor{currentfill}{rgb}{0.000000,0.000000,0.000000}%
\pgfsetfillcolor{currentfill}%
\pgfsetfillopacity{0.500000}%
\pgfsetlinewidth{1.003750pt}%
\definecolor{currentstroke}{rgb}{0.000000,0.000000,0.000000}%
\pgfsetstrokecolor{currentstroke}%
\pgfsetstrokeopacity{0.500000}%
\pgfsetdash{{3.700000pt}{1.600000pt}}{0.000000pt}%
\pgfpathmoveto{\pgfqpoint{7.191517in}{3.744878in}}%
\pgfpathcurveto{\pgfqpoint{7.197341in}{3.744878in}}{\pgfqpoint{7.202927in}{3.747192in}}{\pgfqpoint{7.207045in}{3.751310in}}%
\pgfpathcurveto{\pgfqpoint{7.211163in}{3.755428in}}{\pgfqpoint{7.213477in}{3.761014in}}{\pgfqpoint{7.213477in}{3.766838in}}%
\pgfpathcurveto{\pgfqpoint{7.213477in}{3.772662in}}{\pgfqpoint{7.211163in}{3.778248in}}{\pgfqpoint{7.207045in}{3.782367in}}%
\pgfpathcurveto{\pgfqpoint{7.202927in}{3.786485in}}{\pgfqpoint{7.197341in}{3.788799in}}{\pgfqpoint{7.191517in}{3.788799in}}%
\pgfpathcurveto{\pgfqpoint{7.185693in}{3.788799in}}{\pgfqpoint{7.180107in}{3.786485in}}{\pgfqpoint{7.175989in}{3.782367in}}%
\pgfpathcurveto{\pgfqpoint{7.171870in}{3.778248in}}{\pgfqpoint{7.169557in}{3.772662in}}{\pgfqpoint{7.169557in}{3.766838in}}%
\pgfpathcurveto{\pgfqpoint{7.169557in}{3.761014in}}{\pgfqpoint{7.171870in}{3.755428in}}{\pgfqpoint{7.175989in}{3.751310in}}%
\pgfpathcurveto{\pgfqpoint{7.180107in}{3.747192in}}{\pgfqpoint{7.185693in}{3.744878in}}{\pgfqpoint{7.191517in}{3.744878in}}%
\pgfpathlineto{\pgfqpoint{7.191517in}{3.744878in}}%
\pgfpathclose%
\pgfusepath{stroke,fill}%
\end{pgfscope}%
\begin{pgfscope}%
\pgfpathrectangle{\pgfqpoint{0.640323in}{0.527436in}}{\pgfqpoint{9.687500in}{3.850000in}}%
\pgfusepath{clip}%
\pgfsetbuttcap%
\pgfsetroundjoin%
\definecolor{currentfill}{rgb}{0.000000,0.000000,0.000000}%
\pgfsetfillcolor{currentfill}%
\pgfsetfillopacity{0.500000}%
\pgfsetlinewidth{1.003750pt}%
\definecolor{currentstroke}{rgb}{0.000000,0.000000,0.000000}%
\pgfsetstrokecolor{currentstroke}%
\pgfsetstrokeopacity{0.500000}%
\pgfsetdash{{3.700000pt}{1.600000pt}}{0.000000pt}%
\pgfpathmoveto{\pgfqpoint{7.371248in}{3.771859in}}%
\pgfpathcurveto{\pgfqpoint{7.377072in}{3.771859in}}{\pgfqpoint{7.382658in}{3.774173in}}{\pgfqpoint{7.386776in}{3.778291in}}%
\pgfpathcurveto{\pgfqpoint{7.390894in}{3.782409in}}{\pgfqpoint{7.393208in}{3.787995in}}{\pgfqpoint{7.393208in}{3.793819in}}%
\pgfpathcurveto{\pgfqpoint{7.393208in}{3.799643in}}{\pgfqpoint{7.390894in}{3.805229in}}{\pgfqpoint{7.386776in}{3.809348in}}%
\pgfpathcurveto{\pgfqpoint{7.382658in}{3.813466in}}{\pgfqpoint{7.377072in}{3.815780in}}{\pgfqpoint{7.371248in}{3.815780in}}%
\pgfpathcurveto{\pgfqpoint{7.365424in}{3.815780in}}{\pgfqpoint{7.359838in}{3.813466in}}{\pgfqpoint{7.355720in}{3.809348in}}%
\pgfpathcurveto{\pgfqpoint{7.351601in}{3.805229in}}{\pgfqpoint{7.349288in}{3.799643in}}{\pgfqpoint{7.349288in}{3.793819in}}%
\pgfpathcurveto{\pgfqpoint{7.349288in}{3.787995in}}{\pgfqpoint{7.351601in}{3.782409in}}{\pgfqpoint{7.355720in}{3.778291in}}%
\pgfpathcurveto{\pgfqpoint{7.359838in}{3.774173in}}{\pgfqpoint{7.365424in}{3.771859in}}{\pgfqpoint{7.371248in}{3.771859in}}%
\pgfpathlineto{\pgfqpoint{7.371248in}{3.771859in}}%
\pgfpathclose%
\pgfusepath{stroke,fill}%
\end{pgfscope}%
\begin{pgfscope}%
\pgfpathrectangle{\pgfqpoint{0.640323in}{0.527436in}}{\pgfqpoint{9.687500in}{3.850000in}}%
\pgfusepath{clip}%
\pgfsetbuttcap%
\pgfsetroundjoin%
\definecolor{currentfill}{rgb}{0.000000,0.000000,0.000000}%
\pgfsetfillcolor{currentfill}%
\pgfsetfillopacity{0.500000}%
\pgfsetlinewidth{1.003750pt}%
\definecolor{currentstroke}{rgb}{0.000000,0.000000,0.000000}%
\pgfsetstrokecolor{currentstroke}%
\pgfsetstrokeopacity{0.500000}%
\pgfsetdash{{3.700000pt}{1.600000pt}}{0.000000pt}%
\pgfpathmoveto{\pgfqpoint{7.550979in}{3.794021in}}%
\pgfpathcurveto{\pgfqpoint{7.556803in}{3.794021in}}{\pgfqpoint{7.562389in}{3.796335in}}{\pgfqpoint{7.566507in}{3.800453in}}%
\pgfpathcurveto{\pgfqpoint{7.570625in}{3.804572in}}{\pgfqpoint{7.572939in}{3.810158in}}{\pgfqpoint{7.572939in}{3.815982in}}%
\pgfpathcurveto{\pgfqpoint{7.572939in}{3.821806in}}{\pgfqpoint{7.570625in}{3.827392in}}{\pgfqpoint{7.566507in}{3.831510in}}%
\pgfpathcurveto{\pgfqpoint{7.562389in}{3.835628in}}{\pgfqpoint{7.556803in}{3.837942in}}{\pgfqpoint{7.550979in}{3.837942in}}%
\pgfpathcurveto{\pgfqpoint{7.545155in}{3.837942in}}{\pgfqpoint{7.539569in}{3.835628in}}{\pgfqpoint{7.535451in}{3.831510in}}%
\pgfpathcurveto{\pgfqpoint{7.531332in}{3.827392in}}{\pgfqpoint{7.529019in}{3.821806in}}{\pgfqpoint{7.529019in}{3.815982in}}%
\pgfpathcurveto{\pgfqpoint{7.529019in}{3.810158in}}{\pgfqpoint{7.531332in}{3.804572in}}{\pgfqpoint{7.535451in}{3.800453in}}%
\pgfpathcurveto{\pgfqpoint{7.539569in}{3.796335in}}{\pgfqpoint{7.545155in}{3.794021in}}{\pgfqpoint{7.550979in}{3.794021in}}%
\pgfpathlineto{\pgfqpoint{7.550979in}{3.794021in}}%
\pgfpathclose%
\pgfusepath{stroke,fill}%
\end{pgfscope}%
\begin{pgfscope}%
\pgfpathrectangle{\pgfqpoint{0.640323in}{0.527436in}}{\pgfqpoint{9.687500in}{3.850000in}}%
\pgfusepath{clip}%
\pgfsetbuttcap%
\pgfsetroundjoin%
\definecolor{currentfill}{rgb}{0.000000,0.000000,0.000000}%
\pgfsetfillcolor{currentfill}%
\pgfsetfillopacity{0.500000}%
\pgfsetlinewidth{1.003750pt}%
\definecolor{currentstroke}{rgb}{0.000000,0.000000,0.000000}%
\pgfsetstrokecolor{currentstroke}%
\pgfsetstrokeopacity{0.500000}%
\pgfsetdash{{3.700000pt}{1.600000pt}}{0.000000pt}%
\pgfpathmoveto{\pgfqpoint{7.730710in}{3.814700in}}%
\pgfpathcurveto{\pgfqpoint{7.736534in}{3.814700in}}{\pgfqpoint{7.742120in}{3.817013in}}{\pgfqpoint{7.746238in}{3.821132in}}%
\pgfpathcurveto{\pgfqpoint{7.750356in}{3.825250in}}{\pgfqpoint{7.752670in}{3.830836in}}{\pgfqpoint{7.752670in}{3.836660in}}%
\pgfpathcurveto{\pgfqpoint{7.752670in}{3.842484in}}{\pgfqpoint{7.750356in}{3.848070in}}{\pgfqpoint{7.746238in}{3.852188in}}%
\pgfpathcurveto{\pgfqpoint{7.742120in}{3.856306in}}{\pgfqpoint{7.736534in}{3.858620in}}{\pgfqpoint{7.730710in}{3.858620in}}%
\pgfpathcurveto{\pgfqpoint{7.724886in}{3.858620in}}{\pgfqpoint{7.719300in}{3.856306in}}{\pgfqpoint{7.715182in}{3.852188in}}%
\pgfpathcurveto{\pgfqpoint{7.711063in}{3.848070in}}{\pgfqpoint{7.708750in}{3.842484in}}{\pgfqpoint{7.708750in}{3.836660in}}%
\pgfpathcurveto{\pgfqpoint{7.708750in}{3.830836in}}{\pgfqpoint{7.711063in}{3.825250in}}{\pgfqpoint{7.715182in}{3.821132in}}%
\pgfpathcurveto{\pgfqpoint{7.719300in}{3.817013in}}{\pgfqpoint{7.724886in}{3.814700in}}{\pgfqpoint{7.730710in}{3.814700in}}%
\pgfpathlineto{\pgfqpoint{7.730710in}{3.814700in}}%
\pgfpathclose%
\pgfusepath{stroke,fill}%
\end{pgfscope}%
\begin{pgfscope}%
\pgfpathrectangle{\pgfqpoint{0.640323in}{0.527436in}}{\pgfqpoint{9.687500in}{3.850000in}}%
\pgfusepath{clip}%
\pgfsetbuttcap%
\pgfsetroundjoin%
\definecolor{currentfill}{rgb}{0.000000,0.000000,0.000000}%
\pgfsetfillcolor{currentfill}%
\pgfsetfillopacity{0.500000}%
\pgfsetlinewidth{1.003750pt}%
\definecolor{currentstroke}{rgb}{0.000000,0.000000,0.000000}%
\pgfsetstrokecolor{currentstroke}%
\pgfsetstrokeopacity{0.500000}%
\pgfsetdash{{3.700000pt}{1.600000pt}}{0.000000pt}%
\pgfpathmoveto{\pgfqpoint{7.910441in}{3.838452in}}%
\pgfpathcurveto{\pgfqpoint{7.916265in}{3.838452in}}{\pgfqpoint{7.921851in}{3.840765in}}{\pgfqpoint{7.925969in}{3.844884in}}%
\pgfpathcurveto{\pgfqpoint{7.930087in}{3.849002in}}{\pgfqpoint{7.932401in}{3.854588in}}{\pgfqpoint{7.932401in}{3.860412in}}%
\pgfpathcurveto{\pgfqpoint{7.932401in}{3.866236in}}{\pgfqpoint{7.930087in}{3.871822in}}{\pgfqpoint{7.925969in}{3.875940in}}%
\pgfpathcurveto{\pgfqpoint{7.921851in}{3.880058in}}{\pgfqpoint{7.916265in}{3.882372in}}{\pgfqpoint{7.910441in}{3.882372in}}%
\pgfpathcurveto{\pgfqpoint{7.904617in}{3.882372in}}{\pgfqpoint{7.899031in}{3.880058in}}{\pgfqpoint{7.894913in}{3.875940in}}%
\pgfpathcurveto{\pgfqpoint{7.890794in}{3.871822in}}{\pgfqpoint{7.888481in}{3.866236in}}{\pgfqpoint{7.888481in}{3.860412in}}%
\pgfpathcurveto{\pgfqpoint{7.888481in}{3.854588in}}{\pgfqpoint{7.890794in}{3.849002in}}{\pgfqpoint{7.894913in}{3.844884in}}%
\pgfpathcurveto{\pgfqpoint{7.899031in}{3.840765in}}{\pgfqpoint{7.904617in}{3.838452in}}{\pgfqpoint{7.910441in}{3.838452in}}%
\pgfpathlineto{\pgfqpoint{7.910441in}{3.838452in}}%
\pgfpathclose%
\pgfusepath{stroke,fill}%
\end{pgfscope}%
\begin{pgfscope}%
\pgfpathrectangle{\pgfqpoint{0.640323in}{0.527436in}}{\pgfqpoint{9.687500in}{3.850000in}}%
\pgfusepath{clip}%
\pgfsetbuttcap%
\pgfsetroundjoin%
\definecolor{currentfill}{rgb}{0.000000,0.000000,0.000000}%
\pgfsetfillcolor{currentfill}%
\pgfsetfillopacity{0.500000}%
\pgfsetlinewidth{1.003750pt}%
\definecolor{currentstroke}{rgb}{0.000000,0.000000,0.000000}%
\pgfsetstrokecolor{currentstroke}%
\pgfsetstrokeopacity{0.500000}%
\pgfsetdash{{3.700000pt}{1.600000pt}}{0.000000pt}%
\pgfpathmoveto{\pgfqpoint{8.090172in}{3.856385in}}%
\pgfpathcurveto{\pgfqpoint{8.095996in}{3.856385in}}{\pgfqpoint{8.101582in}{3.858699in}}{\pgfqpoint{8.105700in}{3.862817in}}%
\pgfpathcurveto{\pgfqpoint{8.109818in}{3.866935in}}{\pgfqpoint{8.112132in}{3.872522in}}{\pgfqpoint{8.112132in}{3.878345in}}%
\pgfpathcurveto{\pgfqpoint{8.112132in}{3.884169in}}{\pgfqpoint{8.109818in}{3.889756in}}{\pgfqpoint{8.105700in}{3.893874in}}%
\pgfpathcurveto{\pgfqpoint{8.101582in}{3.897992in}}{\pgfqpoint{8.095996in}{3.900306in}}{\pgfqpoint{8.090172in}{3.900306in}}%
\pgfpathcurveto{\pgfqpoint{8.084348in}{3.900306in}}{\pgfqpoint{8.078762in}{3.897992in}}{\pgfqpoint{8.074644in}{3.893874in}}%
\pgfpathcurveto{\pgfqpoint{8.070525in}{3.889756in}}{\pgfqpoint{8.068211in}{3.884169in}}{\pgfqpoint{8.068211in}{3.878345in}}%
\pgfpathcurveto{\pgfqpoint{8.068211in}{3.872522in}}{\pgfqpoint{8.070525in}{3.866935in}}{\pgfqpoint{8.074644in}{3.862817in}}%
\pgfpathcurveto{\pgfqpoint{8.078762in}{3.858699in}}{\pgfqpoint{8.084348in}{3.856385in}}{\pgfqpoint{8.090172in}{3.856385in}}%
\pgfpathlineto{\pgfqpoint{8.090172in}{3.856385in}}%
\pgfpathclose%
\pgfusepath{stroke,fill}%
\end{pgfscope}%
\begin{pgfscope}%
\pgfpathrectangle{\pgfqpoint{0.640323in}{0.527436in}}{\pgfqpoint{9.687500in}{3.850000in}}%
\pgfusepath{clip}%
\pgfsetbuttcap%
\pgfsetroundjoin%
\definecolor{currentfill}{rgb}{0.000000,0.000000,0.000000}%
\pgfsetfillcolor{currentfill}%
\pgfsetfillopacity{0.500000}%
\pgfsetlinewidth{1.003750pt}%
\definecolor{currentstroke}{rgb}{0.000000,0.000000,0.000000}%
\pgfsetstrokecolor{currentstroke}%
\pgfsetstrokeopacity{0.500000}%
\pgfsetdash{{3.700000pt}{1.600000pt}}{0.000000pt}%
\pgfpathmoveto{\pgfqpoint{8.269903in}{3.876467in}}%
\pgfpathcurveto{\pgfqpoint{8.275727in}{3.876467in}}{\pgfqpoint{8.281313in}{3.878781in}}{\pgfqpoint{8.285431in}{3.882899in}}%
\pgfpathcurveto{\pgfqpoint{8.289549in}{3.887017in}}{\pgfqpoint{8.291863in}{3.892604in}}{\pgfqpoint{8.291863in}{3.898428in}}%
\pgfpathcurveto{\pgfqpoint{8.291863in}{3.904251in}}{\pgfqpoint{8.289549in}{3.909838in}}{\pgfqpoint{8.285431in}{3.913956in}}%
\pgfpathcurveto{\pgfqpoint{8.281313in}{3.918074in}}{\pgfqpoint{8.275727in}{3.920388in}}{\pgfqpoint{8.269903in}{3.920388in}}%
\pgfpathcurveto{\pgfqpoint{8.264079in}{3.920388in}}{\pgfqpoint{8.258493in}{3.918074in}}{\pgfqpoint{8.254374in}{3.913956in}}%
\pgfpathcurveto{\pgfqpoint{8.250256in}{3.909838in}}{\pgfqpoint{8.247942in}{3.904251in}}{\pgfqpoint{8.247942in}{3.898428in}}%
\pgfpathcurveto{\pgfqpoint{8.247942in}{3.892604in}}{\pgfqpoint{8.250256in}{3.887017in}}{\pgfqpoint{8.254374in}{3.882899in}}%
\pgfpathcurveto{\pgfqpoint{8.258493in}{3.878781in}}{\pgfqpoint{8.264079in}{3.876467in}}{\pgfqpoint{8.269903in}{3.876467in}}%
\pgfpathlineto{\pgfqpoint{8.269903in}{3.876467in}}%
\pgfpathclose%
\pgfusepath{stroke,fill}%
\end{pgfscope}%
\begin{pgfscope}%
\pgfpathrectangle{\pgfqpoint{0.640323in}{0.527436in}}{\pgfqpoint{9.687500in}{3.850000in}}%
\pgfusepath{clip}%
\pgfsetbuttcap%
\pgfsetroundjoin%
\definecolor{currentfill}{rgb}{0.000000,0.000000,0.000000}%
\pgfsetfillcolor{currentfill}%
\pgfsetfillopacity{0.500000}%
\pgfsetlinewidth{1.003750pt}%
\definecolor{currentstroke}{rgb}{0.000000,0.000000,0.000000}%
\pgfsetstrokecolor{currentstroke}%
\pgfsetstrokeopacity{0.500000}%
\pgfsetdash{{3.700000pt}{1.600000pt}}{0.000000pt}%
\pgfpathmoveto{\pgfqpoint{8.449634in}{3.895624in}}%
\pgfpathcurveto{\pgfqpoint{8.455458in}{3.895624in}}{\pgfqpoint{8.461044in}{3.897938in}}{\pgfqpoint{8.465162in}{3.902056in}}%
\pgfpathcurveto{\pgfqpoint{8.469280in}{3.906174in}}{\pgfqpoint{8.471594in}{3.911760in}}{\pgfqpoint{8.471594in}{3.917584in}}%
\pgfpathcurveto{\pgfqpoint{8.471594in}{3.923408in}}{\pgfqpoint{8.469280in}{3.928994in}}{\pgfqpoint{8.465162in}{3.933113in}}%
\pgfpathcurveto{\pgfqpoint{8.461044in}{3.937231in}}{\pgfqpoint{8.455458in}{3.939545in}}{\pgfqpoint{8.449634in}{3.939545in}}%
\pgfpathcurveto{\pgfqpoint{8.443810in}{3.939545in}}{\pgfqpoint{8.438224in}{3.937231in}}{\pgfqpoint{8.434105in}{3.933113in}}%
\pgfpathcurveto{\pgfqpoint{8.429987in}{3.928994in}}{\pgfqpoint{8.427673in}{3.923408in}}{\pgfqpoint{8.427673in}{3.917584in}}%
\pgfpathcurveto{\pgfqpoint{8.427673in}{3.911760in}}{\pgfqpoint{8.429987in}{3.906174in}}{\pgfqpoint{8.434105in}{3.902056in}}%
\pgfpathcurveto{\pgfqpoint{8.438224in}{3.897938in}}{\pgfqpoint{8.443810in}{3.895624in}}{\pgfqpoint{8.449634in}{3.895624in}}%
\pgfpathlineto{\pgfqpoint{8.449634in}{3.895624in}}%
\pgfpathclose%
\pgfusepath{stroke,fill}%
\end{pgfscope}%
\begin{pgfscope}%
\pgfpathrectangle{\pgfqpoint{0.640323in}{0.527436in}}{\pgfqpoint{9.687500in}{3.850000in}}%
\pgfusepath{clip}%
\pgfsetbuttcap%
\pgfsetroundjoin%
\definecolor{currentfill}{rgb}{0.000000,0.000000,0.000000}%
\pgfsetfillcolor{currentfill}%
\pgfsetfillopacity{0.500000}%
\pgfsetlinewidth{1.003750pt}%
\definecolor{currentstroke}{rgb}{0.000000,0.000000,0.000000}%
\pgfsetstrokecolor{currentstroke}%
\pgfsetstrokeopacity{0.500000}%
\pgfsetdash{{3.700000pt}{1.600000pt}}{0.000000pt}%
\pgfpathmoveto{\pgfqpoint{8.629365in}{3.912875in}}%
\pgfpathcurveto{\pgfqpoint{8.635189in}{3.912875in}}{\pgfqpoint{8.640775in}{3.915188in}}{\pgfqpoint{8.644893in}{3.919307in}}%
\pgfpathcurveto{\pgfqpoint{8.649011in}{3.923425in}}{\pgfqpoint{8.651325in}{3.929011in}}{\pgfqpoint{8.651325in}{3.934835in}}%
\pgfpathcurveto{\pgfqpoint{8.651325in}{3.940659in}}{\pgfqpoint{8.649011in}{3.946245in}}{\pgfqpoint{8.644893in}{3.950363in}}%
\pgfpathcurveto{\pgfqpoint{8.640775in}{3.954481in}}{\pgfqpoint{8.635189in}{3.956795in}}{\pgfqpoint{8.629365in}{3.956795in}}%
\pgfpathcurveto{\pgfqpoint{8.623541in}{3.956795in}}{\pgfqpoint{8.617955in}{3.954481in}}{\pgfqpoint{8.613836in}{3.950363in}}%
\pgfpathcurveto{\pgfqpoint{8.609718in}{3.946245in}}{\pgfqpoint{8.607404in}{3.940659in}}{\pgfqpoint{8.607404in}{3.934835in}}%
\pgfpathcurveto{\pgfqpoint{8.607404in}{3.929011in}}{\pgfqpoint{8.609718in}{3.923425in}}{\pgfqpoint{8.613836in}{3.919307in}}%
\pgfpathcurveto{\pgfqpoint{8.617955in}{3.915188in}}{\pgfqpoint{8.623541in}{3.912875in}}{\pgfqpoint{8.629365in}{3.912875in}}%
\pgfpathlineto{\pgfqpoint{8.629365in}{3.912875in}}%
\pgfpathclose%
\pgfusepath{stroke,fill}%
\end{pgfscope}%
\begin{pgfscope}%
\pgfpathrectangle{\pgfqpoint{0.640323in}{0.527436in}}{\pgfqpoint{9.687500in}{3.850000in}}%
\pgfusepath{clip}%
\pgfsetbuttcap%
\pgfsetroundjoin%
\definecolor{currentfill}{rgb}{0.000000,0.000000,0.000000}%
\pgfsetfillcolor{currentfill}%
\pgfsetfillopacity{0.500000}%
\pgfsetlinewidth{1.003750pt}%
\definecolor{currentstroke}{rgb}{0.000000,0.000000,0.000000}%
\pgfsetstrokecolor{currentstroke}%
\pgfsetstrokeopacity{0.500000}%
\pgfsetdash{{3.700000pt}{1.600000pt}}{0.000000pt}%
\pgfpathmoveto{\pgfqpoint{8.809096in}{3.930566in}}%
\pgfpathcurveto{\pgfqpoint{8.814920in}{3.930566in}}{\pgfqpoint{8.820506in}{3.932880in}}{\pgfqpoint{8.824624in}{3.936998in}}%
\pgfpathcurveto{\pgfqpoint{8.828742in}{3.941116in}}{\pgfqpoint{8.831056in}{3.946702in}}{\pgfqpoint{8.831056in}{3.952526in}}%
\pgfpathcurveto{\pgfqpoint{8.831056in}{3.958350in}}{\pgfqpoint{8.828742in}{3.963936in}}{\pgfqpoint{8.824624in}{3.968054in}}%
\pgfpathcurveto{\pgfqpoint{8.820506in}{3.972173in}}{\pgfqpoint{8.814920in}{3.974486in}}{\pgfqpoint{8.809096in}{3.974486in}}%
\pgfpathcurveto{\pgfqpoint{8.803272in}{3.974486in}}{\pgfqpoint{8.797686in}{3.972173in}}{\pgfqpoint{8.793567in}{3.968054in}}%
\pgfpathcurveto{\pgfqpoint{8.789449in}{3.963936in}}{\pgfqpoint{8.787135in}{3.958350in}}{\pgfqpoint{8.787135in}{3.952526in}}%
\pgfpathcurveto{\pgfqpoint{8.787135in}{3.946702in}}{\pgfqpoint{8.789449in}{3.941116in}}{\pgfqpoint{8.793567in}{3.936998in}}%
\pgfpathcurveto{\pgfqpoint{8.797686in}{3.932880in}}{\pgfqpoint{8.803272in}{3.930566in}}{\pgfqpoint{8.809096in}{3.930566in}}%
\pgfpathlineto{\pgfqpoint{8.809096in}{3.930566in}}%
\pgfpathclose%
\pgfusepath{stroke,fill}%
\end{pgfscope}%
\begin{pgfscope}%
\pgfpathrectangle{\pgfqpoint{0.640323in}{0.527436in}}{\pgfqpoint{9.687500in}{3.850000in}}%
\pgfusepath{clip}%
\pgfsetbuttcap%
\pgfsetroundjoin%
\definecolor{currentfill}{rgb}{0.000000,0.000000,0.000000}%
\pgfsetfillcolor{currentfill}%
\pgfsetfillopacity{0.500000}%
\pgfsetlinewidth{1.003750pt}%
\definecolor{currentstroke}{rgb}{0.000000,0.000000,0.000000}%
\pgfsetstrokecolor{currentstroke}%
\pgfsetstrokeopacity{0.500000}%
\pgfsetdash{{3.700000pt}{1.600000pt}}{0.000000pt}%
\pgfpathmoveto{\pgfqpoint{8.988827in}{3.945469in}}%
\pgfpathcurveto{\pgfqpoint{8.994651in}{3.945469in}}{\pgfqpoint{9.000237in}{3.947783in}}{\pgfqpoint{9.004355in}{3.951901in}}%
\pgfpathcurveto{\pgfqpoint{9.008473in}{3.956019in}}{\pgfqpoint{9.010787in}{3.961606in}}{\pgfqpoint{9.010787in}{3.967429in}}%
\pgfpathcurveto{\pgfqpoint{9.010787in}{3.973253in}}{\pgfqpoint{9.008473in}{3.978840in}}{\pgfqpoint{9.004355in}{3.982958in}}%
\pgfpathcurveto{\pgfqpoint{9.000237in}{3.987076in}}{\pgfqpoint{8.994651in}{3.989390in}}{\pgfqpoint{8.988827in}{3.989390in}}%
\pgfpathcurveto{\pgfqpoint{8.983003in}{3.989390in}}{\pgfqpoint{8.977417in}{3.987076in}}{\pgfqpoint{8.973298in}{3.982958in}}%
\pgfpathcurveto{\pgfqpoint{8.969180in}{3.978840in}}{\pgfqpoint{8.966866in}{3.973253in}}{\pgfqpoint{8.966866in}{3.967429in}}%
\pgfpathcurveto{\pgfqpoint{8.966866in}{3.961606in}}{\pgfqpoint{8.969180in}{3.956019in}}{\pgfqpoint{8.973298in}{3.951901in}}%
\pgfpathcurveto{\pgfqpoint{8.977417in}{3.947783in}}{\pgfqpoint{8.983003in}{3.945469in}}{\pgfqpoint{8.988827in}{3.945469in}}%
\pgfpathlineto{\pgfqpoint{8.988827in}{3.945469in}}%
\pgfpathclose%
\pgfusepath{stroke,fill}%
\end{pgfscope}%
\begin{pgfscope}%
\pgfpathrectangle{\pgfqpoint{0.640323in}{0.527436in}}{\pgfqpoint{9.687500in}{3.850000in}}%
\pgfusepath{clip}%
\pgfsetbuttcap%
\pgfsetroundjoin%
\definecolor{currentfill}{rgb}{0.000000,0.000000,0.000000}%
\pgfsetfillcolor{currentfill}%
\pgfsetfillopacity{0.500000}%
\pgfsetlinewidth{1.003750pt}%
\definecolor{currentstroke}{rgb}{0.000000,0.000000,0.000000}%
\pgfsetstrokecolor{currentstroke}%
\pgfsetstrokeopacity{0.500000}%
\pgfsetdash{{3.700000pt}{1.600000pt}}{0.000000pt}%
\pgfpathmoveto{\pgfqpoint{9.168558in}{3.960950in}}%
\pgfpathcurveto{\pgfqpoint{9.174382in}{3.960950in}}{\pgfqpoint{9.179968in}{3.963264in}}{\pgfqpoint{9.184086in}{3.967382in}}%
\pgfpathcurveto{\pgfqpoint{9.188204in}{3.971500in}}{\pgfqpoint{9.190518in}{3.977086in}}{\pgfqpoint{9.190518in}{3.982910in}}%
\pgfpathcurveto{\pgfqpoint{9.190518in}{3.988734in}}{\pgfqpoint{9.188204in}{3.994320in}}{\pgfqpoint{9.184086in}{3.998438in}}%
\pgfpathcurveto{\pgfqpoint{9.179968in}{4.002557in}}{\pgfqpoint{9.174382in}{4.004870in}}{\pgfqpoint{9.168558in}{4.004870in}}%
\pgfpathcurveto{\pgfqpoint{9.162734in}{4.004870in}}{\pgfqpoint{9.157148in}{4.002557in}}{\pgfqpoint{9.153029in}{3.998438in}}%
\pgfpathcurveto{\pgfqpoint{9.148911in}{3.994320in}}{\pgfqpoint{9.146597in}{3.988734in}}{\pgfqpoint{9.146597in}{3.982910in}}%
\pgfpathcurveto{\pgfqpoint{9.146597in}{3.977086in}}{\pgfqpoint{9.148911in}{3.971500in}}{\pgfqpoint{9.153029in}{3.967382in}}%
\pgfpathcurveto{\pgfqpoint{9.157148in}{3.963264in}}{\pgfqpoint{9.162734in}{3.960950in}}{\pgfqpoint{9.168558in}{3.960950in}}%
\pgfpathlineto{\pgfqpoint{9.168558in}{3.960950in}}%
\pgfpathclose%
\pgfusepath{stroke,fill}%
\end{pgfscope}%
\begin{pgfscope}%
\pgfpathrectangle{\pgfqpoint{0.640323in}{0.527436in}}{\pgfqpoint{9.687500in}{3.850000in}}%
\pgfusepath{clip}%
\pgfsetbuttcap%
\pgfsetroundjoin%
\definecolor{currentfill}{rgb}{0.000000,0.000000,0.000000}%
\pgfsetfillcolor{currentfill}%
\pgfsetfillopacity{0.500000}%
\pgfsetlinewidth{1.003750pt}%
\definecolor{currentstroke}{rgb}{0.000000,0.000000,0.000000}%
\pgfsetstrokecolor{currentstroke}%
\pgfsetstrokeopacity{0.500000}%
\pgfsetdash{{3.700000pt}{1.600000pt}}{0.000000pt}%
\pgfpathmoveto{\pgfqpoint{9.348289in}{3.979418in}}%
\pgfpathcurveto{\pgfqpoint{9.354113in}{3.979418in}}{\pgfqpoint{9.359699in}{3.981731in}}{\pgfqpoint{9.363817in}{3.985850in}}%
\pgfpathcurveto{\pgfqpoint{9.367935in}{3.989968in}}{\pgfqpoint{9.370249in}{3.995554in}}{\pgfqpoint{9.370249in}{4.001378in}}%
\pgfpathcurveto{\pgfqpoint{9.370249in}{4.007202in}}{\pgfqpoint{9.367935in}{4.012788in}}{\pgfqpoint{9.363817in}{4.016906in}}%
\pgfpathcurveto{\pgfqpoint{9.359699in}{4.021024in}}{\pgfqpoint{9.354113in}{4.023338in}}{\pgfqpoint{9.348289in}{4.023338in}}%
\pgfpathcurveto{\pgfqpoint{9.342465in}{4.023338in}}{\pgfqpoint{9.336879in}{4.021024in}}{\pgfqpoint{9.332760in}{4.016906in}}%
\pgfpathcurveto{\pgfqpoint{9.328642in}{4.012788in}}{\pgfqpoint{9.326328in}{4.007202in}}{\pgfqpoint{9.326328in}{4.001378in}}%
\pgfpathcurveto{\pgfqpoint{9.326328in}{3.995554in}}{\pgfqpoint{9.328642in}{3.989968in}}{\pgfqpoint{9.332760in}{3.985850in}}%
\pgfpathcurveto{\pgfqpoint{9.336879in}{3.981731in}}{\pgfqpoint{9.342465in}{3.979418in}}{\pgfqpoint{9.348289in}{3.979418in}}%
\pgfpathlineto{\pgfqpoint{9.348289in}{3.979418in}}%
\pgfpathclose%
\pgfusepath{stroke,fill}%
\end{pgfscope}%
\begin{pgfscope}%
\pgfpathrectangle{\pgfqpoint{0.640323in}{0.527436in}}{\pgfqpoint{9.687500in}{3.850000in}}%
\pgfusepath{clip}%
\pgfsetbuttcap%
\pgfsetroundjoin%
\definecolor{currentfill}{rgb}{0.000000,0.000000,0.000000}%
\pgfsetfillcolor{currentfill}%
\pgfsetfillopacity{0.500000}%
\pgfsetlinewidth{1.003750pt}%
\definecolor{currentstroke}{rgb}{0.000000,0.000000,0.000000}%
\pgfsetstrokecolor{currentstroke}%
\pgfsetstrokeopacity{0.500000}%
\pgfsetdash{{3.700000pt}{1.600000pt}}{0.000000pt}%
\pgfpathmoveto{\pgfqpoint{9.528020in}{3.992340in}}%
\pgfpathcurveto{\pgfqpoint{9.533844in}{3.992340in}}{\pgfqpoint{9.539430in}{3.994654in}}{\pgfqpoint{9.543548in}{3.998772in}}%
\pgfpathcurveto{\pgfqpoint{9.547666in}{4.002890in}}{\pgfqpoint{9.549980in}{4.008476in}}{\pgfqpoint{9.549980in}{4.014300in}}%
\pgfpathcurveto{\pgfqpoint{9.549980in}{4.020124in}}{\pgfqpoint{9.547666in}{4.025710in}}{\pgfqpoint{9.543548in}{4.029828in}}%
\pgfpathcurveto{\pgfqpoint{9.539430in}{4.033946in}}{\pgfqpoint{9.533844in}{4.036260in}}{\pgfqpoint{9.528020in}{4.036260in}}%
\pgfpathcurveto{\pgfqpoint{9.522196in}{4.036260in}}{\pgfqpoint{9.516610in}{4.033946in}}{\pgfqpoint{9.512491in}{4.029828in}}%
\pgfpathcurveto{\pgfqpoint{9.508373in}{4.025710in}}{\pgfqpoint{9.506059in}{4.020124in}}{\pgfqpoint{9.506059in}{4.014300in}}%
\pgfpathcurveto{\pgfqpoint{9.506059in}{4.008476in}}{\pgfqpoint{9.508373in}{4.002890in}}{\pgfqpoint{9.512491in}{3.998772in}}%
\pgfpathcurveto{\pgfqpoint{9.516610in}{3.994654in}}{\pgfqpoint{9.522196in}{3.992340in}}{\pgfqpoint{9.528020in}{3.992340in}}%
\pgfpathlineto{\pgfqpoint{9.528020in}{3.992340in}}%
\pgfpathclose%
\pgfusepath{stroke,fill}%
\end{pgfscope}%
\begin{pgfscope}%
\pgfpathrectangle{\pgfqpoint{0.640323in}{0.527436in}}{\pgfqpoint{9.687500in}{3.850000in}}%
\pgfusepath{clip}%
\pgfsetbuttcap%
\pgfsetroundjoin%
\definecolor{currentfill}{rgb}{0.000000,0.000000,0.000000}%
\pgfsetfillcolor{currentfill}%
\pgfsetfillopacity{0.500000}%
\pgfsetlinewidth{1.003750pt}%
\definecolor{currentstroke}{rgb}{0.000000,0.000000,0.000000}%
\pgfsetstrokecolor{currentstroke}%
\pgfsetstrokeopacity{0.500000}%
\pgfsetdash{{3.700000pt}{1.600000pt}}{0.000000pt}%
\pgfpathmoveto{\pgfqpoint{9.707751in}{4.006908in}}%
\pgfpathcurveto{\pgfqpoint{9.713575in}{4.006908in}}{\pgfqpoint{9.719161in}{4.009222in}}{\pgfqpoint{9.723279in}{4.013340in}}%
\pgfpathcurveto{\pgfqpoint{9.727397in}{4.017458in}}{\pgfqpoint{9.729711in}{4.023044in}}{\pgfqpoint{9.729711in}{4.028868in}}%
\pgfpathcurveto{\pgfqpoint{9.729711in}{4.034692in}}{\pgfqpoint{9.727397in}{4.040278in}}{\pgfqpoint{9.723279in}{4.044396in}}%
\pgfpathcurveto{\pgfqpoint{9.719161in}{4.048514in}}{\pgfqpoint{9.713575in}{4.050828in}}{\pgfqpoint{9.707751in}{4.050828in}}%
\pgfpathcurveto{\pgfqpoint{9.701927in}{4.050828in}}{\pgfqpoint{9.696340in}{4.048514in}}{\pgfqpoint{9.692222in}{4.044396in}}%
\pgfpathcurveto{\pgfqpoint{9.688104in}{4.040278in}}{\pgfqpoint{9.685790in}{4.034692in}}{\pgfqpoint{9.685790in}{4.028868in}}%
\pgfpathcurveto{\pgfqpoint{9.685790in}{4.023044in}}{\pgfqpoint{9.688104in}{4.017458in}}{\pgfqpoint{9.692222in}{4.013340in}}%
\pgfpathcurveto{\pgfqpoint{9.696340in}{4.009222in}}{\pgfqpoint{9.701927in}{4.006908in}}{\pgfqpoint{9.707751in}{4.006908in}}%
\pgfpathlineto{\pgfqpoint{9.707751in}{4.006908in}}%
\pgfpathclose%
\pgfusepath{stroke,fill}%
\end{pgfscope}%
\begin{pgfscope}%
\pgfpathrectangle{\pgfqpoint{0.640323in}{0.527436in}}{\pgfqpoint{9.687500in}{3.850000in}}%
\pgfusepath{clip}%
\pgfsetbuttcap%
\pgfsetroundjoin%
\definecolor{currentfill}{rgb}{0.000000,0.000000,0.000000}%
\pgfsetfillcolor{currentfill}%
\pgfsetfillopacity{0.500000}%
\pgfsetlinewidth{1.003750pt}%
\definecolor{currentstroke}{rgb}{0.000000,0.000000,0.000000}%
\pgfsetstrokecolor{currentstroke}%
\pgfsetstrokeopacity{0.500000}%
\pgfsetdash{{3.700000pt}{1.600000pt}}{0.000000pt}%
\pgfpathmoveto{\pgfqpoint{9.887482in}{4.020463in}}%
\pgfpathcurveto{\pgfqpoint{9.893306in}{4.020463in}}{\pgfqpoint{9.898892in}{4.022777in}}{\pgfqpoint{9.903010in}{4.026895in}}%
\pgfpathcurveto{\pgfqpoint{9.907128in}{4.031014in}}{\pgfqpoint{9.909442in}{4.036600in}}{\pgfqpoint{9.909442in}{4.042424in}}%
\pgfpathcurveto{\pgfqpoint{9.909442in}{4.048248in}}{\pgfqpoint{9.907128in}{4.053834in}}{\pgfqpoint{9.903010in}{4.057952in}}%
\pgfpathcurveto{\pgfqpoint{9.898892in}{4.062070in}}{\pgfqpoint{9.893306in}{4.064384in}}{\pgfqpoint{9.887482in}{4.064384in}}%
\pgfpathcurveto{\pgfqpoint{9.881658in}{4.064384in}}{\pgfqpoint{9.876071in}{4.062070in}}{\pgfqpoint{9.871953in}{4.057952in}}%
\pgfpathcurveto{\pgfqpoint{9.867835in}{4.053834in}}{\pgfqpoint{9.865521in}{4.048248in}}{\pgfqpoint{9.865521in}{4.042424in}}%
\pgfpathcurveto{\pgfqpoint{9.865521in}{4.036600in}}{\pgfqpoint{9.867835in}{4.031014in}}{\pgfqpoint{9.871953in}{4.026895in}}%
\pgfpathcurveto{\pgfqpoint{9.876071in}{4.022777in}}{\pgfqpoint{9.881658in}{4.020463in}}{\pgfqpoint{9.887482in}{4.020463in}}%
\pgfpathlineto{\pgfqpoint{9.887482in}{4.020463in}}%
\pgfpathclose%
\pgfusepath{stroke,fill}%
\end{pgfscope}%
\begin{pgfscope}%
\pgfpathrectangle{\pgfqpoint{0.640323in}{0.527436in}}{\pgfqpoint{9.687500in}{3.850000in}}%
\pgfusepath{clip}%
\pgfsetrectcap%
\pgfsetroundjoin%
\pgfsetlinewidth{0.803000pt}%
\definecolor{currentstroke}{rgb}{0.690196,0.690196,0.690196}%
\pgfsetstrokecolor{currentstroke}%
\pgfsetdash{}{0pt}%
\pgfpathmoveto{\pgfqpoint{1.080663in}{0.527436in}}%
\pgfpathlineto{\pgfqpoint{1.080663in}{4.377436in}}%
\pgfusepath{stroke}%
\end{pgfscope}%
\begin{pgfscope}%
\pgfsetbuttcap%
\pgfsetroundjoin%
\definecolor{currentfill}{rgb}{0.000000,0.000000,0.000000}%
\pgfsetfillcolor{currentfill}%
\pgfsetlinewidth{0.803000pt}%
\definecolor{currentstroke}{rgb}{0.000000,0.000000,0.000000}%
\pgfsetstrokecolor{currentstroke}%
\pgfsetdash{}{0pt}%
\pgfsys@defobject{currentmarker}{\pgfqpoint{0.000000in}{-0.048611in}}{\pgfqpoint{0.000000in}{0.000000in}}{%
\pgfpathmoveto{\pgfqpoint{0.000000in}{0.000000in}}%
\pgfpathlineto{\pgfqpoint{0.000000in}{-0.048611in}}%
\pgfusepath{stroke,fill}%
}%
\begin{pgfscope}%
\pgfsys@transformshift{1.080663in}{0.527436in}%
\pgfsys@useobject{currentmarker}{}%
\end{pgfscope}%
\end{pgfscope}%
\begin{pgfscope}%
\definecolor{textcolor}{rgb}{0.000000,0.000000,0.000000}%
\pgfsetstrokecolor{textcolor}%
\pgfsetfillcolor{textcolor}%
\pgftext[x=1.080663in,y=0.430214in,,top]{\color{textcolor}\sffamily\fontsize{10.000000}{12.000000}\selectfont 0.0}%
\end{pgfscope}%
\begin{pgfscope}%
\pgfpathrectangle{\pgfqpoint{0.640323in}{0.527436in}}{\pgfqpoint{9.687500in}{3.850000in}}%
\pgfusepath{clip}%
\pgfsetrectcap%
\pgfsetroundjoin%
\pgfsetlinewidth{0.803000pt}%
\definecolor{currentstroke}{rgb}{0.690196,0.690196,0.690196}%
\pgfsetstrokecolor{currentstroke}%
\pgfsetdash{}{0pt}%
\pgfpathmoveto{\pgfqpoint{2.877973in}{0.527436in}}%
\pgfpathlineto{\pgfqpoint{2.877973in}{4.377436in}}%
\pgfusepath{stroke}%
\end{pgfscope}%
\begin{pgfscope}%
\pgfsetbuttcap%
\pgfsetroundjoin%
\definecolor{currentfill}{rgb}{0.000000,0.000000,0.000000}%
\pgfsetfillcolor{currentfill}%
\pgfsetlinewidth{0.803000pt}%
\definecolor{currentstroke}{rgb}{0.000000,0.000000,0.000000}%
\pgfsetstrokecolor{currentstroke}%
\pgfsetdash{}{0pt}%
\pgfsys@defobject{currentmarker}{\pgfqpoint{0.000000in}{-0.048611in}}{\pgfqpoint{0.000000in}{0.000000in}}{%
\pgfpathmoveto{\pgfqpoint{0.000000in}{0.000000in}}%
\pgfpathlineto{\pgfqpoint{0.000000in}{-0.048611in}}%
\pgfusepath{stroke,fill}%
}%
\begin{pgfscope}%
\pgfsys@transformshift{2.877973in}{0.527436in}%
\pgfsys@useobject{currentmarker}{}%
\end{pgfscope}%
\end{pgfscope}%
\begin{pgfscope}%
\definecolor{textcolor}{rgb}{0.000000,0.000000,0.000000}%
\pgfsetstrokecolor{textcolor}%
\pgfsetfillcolor{textcolor}%
\pgftext[x=2.877973in,y=0.430214in,,top]{\color{textcolor}\sffamily\fontsize{10.000000}{12.000000}\selectfont 0.2}%
\end{pgfscope}%
\begin{pgfscope}%
\pgfpathrectangle{\pgfqpoint{0.640323in}{0.527436in}}{\pgfqpoint{9.687500in}{3.850000in}}%
\pgfusepath{clip}%
\pgfsetrectcap%
\pgfsetroundjoin%
\pgfsetlinewidth{0.803000pt}%
\definecolor{currentstroke}{rgb}{0.690196,0.690196,0.690196}%
\pgfsetstrokecolor{currentstroke}%
\pgfsetdash{}{0pt}%
\pgfpathmoveto{\pgfqpoint{4.675283in}{0.527436in}}%
\pgfpathlineto{\pgfqpoint{4.675283in}{4.377436in}}%
\pgfusepath{stroke}%
\end{pgfscope}%
\begin{pgfscope}%
\pgfsetbuttcap%
\pgfsetroundjoin%
\definecolor{currentfill}{rgb}{0.000000,0.000000,0.000000}%
\pgfsetfillcolor{currentfill}%
\pgfsetlinewidth{0.803000pt}%
\definecolor{currentstroke}{rgb}{0.000000,0.000000,0.000000}%
\pgfsetstrokecolor{currentstroke}%
\pgfsetdash{}{0pt}%
\pgfsys@defobject{currentmarker}{\pgfqpoint{0.000000in}{-0.048611in}}{\pgfqpoint{0.000000in}{0.000000in}}{%
\pgfpathmoveto{\pgfqpoint{0.000000in}{0.000000in}}%
\pgfpathlineto{\pgfqpoint{0.000000in}{-0.048611in}}%
\pgfusepath{stroke,fill}%
}%
\begin{pgfscope}%
\pgfsys@transformshift{4.675283in}{0.527436in}%
\pgfsys@useobject{currentmarker}{}%
\end{pgfscope}%
\end{pgfscope}%
\begin{pgfscope}%
\definecolor{textcolor}{rgb}{0.000000,0.000000,0.000000}%
\pgfsetstrokecolor{textcolor}%
\pgfsetfillcolor{textcolor}%
\pgftext[x=4.675283in,y=0.430214in,,top]{\color{textcolor}\sffamily\fontsize{10.000000}{12.000000}\selectfont 0.4}%
\end{pgfscope}%
\begin{pgfscope}%
\pgfpathrectangle{\pgfqpoint{0.640323in}{0.527436in}}{\pgfqpoint{9.687500in}{3.850000in}}%
\pgfusepath{clip}%
\pgfsetrectcap%
\pgfsetroundjoin%
\pgfsetlinewidth{0.803000pt}%
\definecolor{currentstroke}{rgb}{0.690196,0.690196,0.690196}%
\pgfsetstrokecolor{currentstroke}%
\pgfsetdash{}{0pt}%
\pgfpathmoveto{\pgfqpoint{6.472593in}{0.527436in}}%
\pgfpathlineto{\pgfqpoint{6.472593in}{4.377436in}}%
\pgfusepath{stroke}%
\end{pgfscope}%
\begin{pgfscope}%
\pgfsetbuttcap%
\pgfsetroundjoin%
\definecolor{currentfill}{rgb}{0.000000,0.000000,0.000000}%
\pgfsetfillcolor{currentfill}%
\pgfsetlinewidth{0.803000pt}%
\definecolor{currentstroke}{rgb}{0.000000,0.000000,0.000000}%
\pgfsetstrokecolor{currentstroke}%
\pgfsetdash{}{0pt}%
\pgfsys@defobject{currentmarker}{\pgfqpoint{0.000000in}{-0.048611in}}{\pgfqpoint{0.000000in}{0.000000in}}{%
\pgfpathmoveto{\pgfqpoint{0.000000in}{0.000000in}}%
\pgfpathlineto{\pgfqpoint{0.000000in}{-0.048611in}}%
\pgfusepath{stroke,fill}%
}%
\begin{pgfscope}%
\pgfsys@transformshift{6.472593in}{0.527436in}%
\pgfsys@useobject{currentmarker}{}%
\end{pgfscope}%
\end{pgfscope}%
\begin{pgfscope}%
\definecolor{textcolor}{rgb}{0.000000,0.000000,0.000000}%
\pgfsetstrokecolor{textcolor}%
\pgfsetfillcolor{textcolor}%
\pgftext[x=6.472593in,y=0.430214in,,top]{\color{textcolor}\sffamily\fontsize{10.000000}{12.000000}\selectfont 0.6}%
\end{pgfscope}%
\begin{pgfscope}%
\pgfpathrectangle{\pgfqpoint{0.640323in}{0.527436in}}{\pgfqpoint{9.687500in}{3.850000in}}%
\pgfusepath{clip}%
\pgfsetrectcap%
\pgfsetroundjoin%
\pgfsetlinewidth{0.803000pt}%
\definecolor{currentstroke}{rgb}{0.690196,0.690196,0.690196}%
\pgfsetstrokecolor{currentstroke}%
\pgfsetdash{}{0pt}%
\pgfpathmoveto{\pgfqpoint{8.269903in}{0.527436in}}%
\pgfpathlineto{\pgfqpoint{8.269903in}{4.377436in}}%
\pgfusepath{stroke}%
\end{pgfscope}%
\begin{pgfscope}%
\pgfsetbuttcap%
\pgfsetroundjoin%
\definecolor{currentfill}{rgb}{0.000000,0.000000,0.000000}%
\pgfsetfillcolor{currentfill}%
\pgfsetlinewidth{0.803000pt}%
\definecolor{currentstroke}{rgb}{0.000000,0.000000,0.000000}%
\pgfsetstrokecolor{currentstroke}%
\pgfsetdash{}{0pt}%
\pgfsys@defobject{currentmarker}{\pgfqpoint{0.000000in}{-0.048611in}}{\pgfqpoint{0.000000in}{0.000000in}}{%
\pgfpathmoveto{\pgfqpoint{0.000000in}{0.000000in}}%
\pgfpathlineto{\pgfqpoint{0.000000in}{-0.048611in}}%
\pgfusepath{stroke,fill}%
}%
\begin{pgfscope}%
\pgfsys@transformshift{8.269903in}{0.527436in}%
\pgfsys@useobject{currentmarker}{}%
\end{pgfscope}%
\end{pgfscope}%
\begin{pgfscope}%
\definecolor{textcolor}{rgb}{0.000000,0.000000,0.000000}%
\pgfsetstrokecolor{textcolor}%
\pgfsetfillcolor{textcolor}%
\pgftext[x=8.269903in,y=0.430214in,,top]{\color{textcolor}\sffamily\fontsize{10.000000}{12.000000}\selectfont 0.8}%
\end{pgfscope}%
\begin{pgfscope}%
\pgfpathrectangle{\pgfqpoint{0.640323in}{0.527436in}}{\pgfqpoint{9.687500in}{3.850000in}}%
\pgfusepath{clip}%
\pgfsetrectcap%
\pgfsetroundjoin%
\pgfsetlinewidth{0.803000pt}%
\definecolor{currentstroke}{rgb}{0.690196,0.690196,0.690196}%
\pgfsetstrokecolor{currentstroke}%
\pgfsetdash{}{0pt}%
\pgfpathmoveto{\pgfqpoint{10.067213in}{0.527436in}}%
\pgfpathlineto{\pgfqpoint{10.067213in}{4.377436in}}%
\pgfusepath{stroke}%
\end{pgfscope}%
\begin{pgfscope}%
\pgfsetbuttcap%
\pgfsetroundjoin%
\definecolor{currentfill}{rgb}{0.000000,0.000000,0.000000}%
\pgfsetfillcolor{currentfill}%
\pgfsetlinewidth{0.803000pt}%
\definecolor{currentstroke}{rgb}{0.000000,0.000000,0.000000}%
\pgfsetstrokecolor{currentstroke}%
\pgfsetdash{}{0pt}%
\pgfsys@defobject{currentmarker}{\pgfqpoint{0.000000in}{-0.048611in}}{\pgfqpoint{0.000000in}{0.000000in}}{%
\pgfpathmoveto{\pgfqpoint{0.000000in}{0.000000in}}%
\pgfpathlineto{\pgfqpoint{0.000000in}{-0.048611in}}%
\pgfusepath{stroke,fill}%
}%
\begin{pgfscope}%
\pgfsys@transformshift{10.067213in}{0.527436in}%
\pgfsys@useobject{currentmarker}{}%
\end{pgfscope}%
\end{pgfscope}%
\begin{pgfscope}%
\definecolor{textcolor}{rgb}{0.000000,0.000000,0.000000}%
\pgfsetstrokecolor{textcolor}%
\pgfsetfillcolor{textcolor}%
\pgftext[x=10.067213in,y=0.430214in,,top]{\color{textcolor}\sffamily\fontsize{10.000000}{12.000000}\selectfont 1.0}%
\end{pgfscope}%
\begin{pgfscope}%
\pgfpathrectangle{\pgfqpoint{0.640323in}{0.527436in}}{\pgfqpoint{9.687500in}{3.850000in}}%
\pgfusepath{clip}%
\pgfsetrectcap%
\pgfsetroundjoin%
\pgfsetlinewidth{0.803000pt}%
\definecolor{currentstroke}{rgb}{0.600000,0.600000,0.600000}%
\pgfsetstrokecolor{currentstroke}%
\pgfsetstrokeopacity{0.200000}%
\pgfsetdash{}{0pt}%
\pgfpathmoveto{\pgfqpoint{1.529991in}{0.527436in}}%
\pgfpathlineto{\pgfqpoint{1.529991in}{4.377436in}}%
\pgfusepath{stroke}%
\end{pgfscope}%
\begin{pgfscope}%
\pgfsetbuttcap%
\pgfsetroundjoin%
\definecolor{currentfill}{rgb}{0.000000,0.000000,0.000000}%
\pgfsetfillcolor{currentfill}%
\pgfsetlinewidth{0.602250pt}%
\definecolor{currentstroke}{rgb}{0.000000,0.000000,0.000000}%
\pgfsetstrokecolor{currentstroke}%
\pgfsetdash{}{0pt}%
\pgfsys@defobject{currentmarker}{\pgfqpoint{0.000000in}{-0.027778in}}{\pgfqpoint{0.000000in}{0.000000in}}{%
\pgfpathmoveto{\pgfqpoint{0.000000in}{0.000000in}}%
\pgfpathlineto{\pgfqpoint{0.000000in}{-0.027778in}}%
\pgfusepath{stroke,fill}%
}%
\begin{pgfscope}%
\pgfsys@transformshift{1.529991in}{0.527436in}%
\pgfsys@useobject{currentmarker}{}%
\end{pgfscope}%
\end{pgfscope}%
\begin{pgfscope}%
\pgfpathrectangle{\pgfqpoint{0.640323in}{0.527436in}}{\pgfqpoint{9.687500in}{3.850000in}}%
\pgfusepath{clip}%
\pgfsetrectcap%
\pgfsetroundjoin%
\pgfsetlinewidth{0.803000pt}%
\definecolor{currentstroke}{rgb}{0.600000,0.600000,0.600000}%
\pgfsetstrokecolor{currentstroke}%
\pgfsetstrokeopacity{0.200000}%
\pgfsetdash{}{0pt}%
\pgfpathmoveto{\pgfqpoint{1.979318in}{0.527436in}}%
\pgfpathlineto{\pgfqpoint{1.979318in}{4.377436in}}%
\pgfusepath{stroke}%
\end{pgfscope}%
\begin{pgfscope}%
\pgfsetbuttcap%
\pgfsetroundjoin%
\definecolor{currentfill}{rgb}{0.000000,0.000000,0.000000}%
\pgfsetfillcolor{currentfill}%
\pgfsetlinewidth{0.602250pt}%
\definecolor{currentstroke}{rgb}{0.000000,0.000000,0.000000}%
\pgfsetstrokecolor{currentstroke}%
\pgfsetdash{}{0pt}%
\pgfsys@defobject{currentmarker}{\pgfqpoint{0.000000in}{-0.027778in}}{\pgfqpoint{0.000000in}{0.000000in}}{%
\pgfpathmoveto{\pgfqpoint{0.000000in}{0.000000in}}%
\pgfpathlineto{\pgfqpoint{0.000000in}{-0.027778in}}%
\pgfusepath{stroke,fill}%
}%
\begin{pgfscope}%
\pgfsys@transformshift{1.979318in}{0.527436in}%
\pgfsys@useobject{currentmarker}{}%
\end{pgfscope}%
\end{pgfscope}%
\begin{pgfscope}%
\pgfpathrectangle{\pgfqpoint{0.640323in}{0.527436in}}{\pgfqpoint{9.687500in}{3.850000in}}%
\pgfusepath{clip}%
\pgfsetrectcap%
\pgfsetroundjoin%
\pgfsetlinewidth{0.803000pt}%
\definecolor{currentstroke}{rgb}{0.600000,0.600000,0.600000}%
\pgfsetstrokecolor{currentstroke}%
\pgfsetstrokeopacity{0.200000}%
\pgfsetdash{}{0pt}%
\pgfpathmoveto{\pgfqpoint{2.428646in}{0.527436in}}%
\pgfpathlineto{\pgfqpoint{2.428646in}{4.377436in}}%
\pgfusepath{stroke}%
\end{pgfscope}%
\begin{pgfscope}%
\pgfsetbuttcap%
\pgfsetroundjoin%
\definecolor{currentfill}{rgb}{0.000000,0.000000,0.000000}%
\pgfsetfillcolor{currentfill}%
\pgfsetlinewidth{0.602250pt}%
\definecolor{currentstroke}{rgb}{0.000000,0.000000,0.000000}%
\pgfsetstrokecolor{currentstroke}%
\pgfsetdash{}{0pt}%
\pgfsys@defobject{currentmarker}{\pgfqpoint{0.000000in}{-0.027778in}}{\pgfqpoint{0.000000in}{0.000000in}}{%
\pgfpathmoveto{\pgfqpoint{0.000000in}{0.000000in}}%
\pgfpathlineto{\pgfqpoint{0.000000in}{-0.027778in}}%
\pgfusepath{stroke,fill}%
}%
\begin{pgfscope}%
\pgfsys@transformshift{2.428646in}{0.527436in}%
\pgfsys@useobject{currentmarker}{}%
\end{pgfscope}%
\end{pgfscope}%
\begin{pgfscope}%
\pgfpathrectangle{\pgfqpoint{0.640323in}{0.527436in}}{\pgfqpoint{9.687500in}{3.850000in}}%
\pgfusepath{clip}%
\pgfsetrectcap%
\pgfsetroundjoin%
\pgfsetlinewidth{0.803000pt}%
\definecolor{currentstroke}{rgb}{0.600000,0.600000,0.600000}%
\pgfsetstrokecolor{currentstroke}%
\pgfsetstrokeopacity{0.200000}%
\pgfsetdash{}{0pt}%
\pgfpathmoveto{\pgfqpoint{3.327301in}{0.527436in}}%
\pgfpathlineto{\pgfqpoint{3.327301in}{4.377436in}}%
\pgfusepath{stroke}%
\end{pgfscope}%
\begin{pgfscope}%
\pgfsetbuttcap%
\pgfsetroundjoin%
\definecolor{currentfill}{rgb}{0.000000,0.000000,0.000000}%
\pgfsetfillcolor{currentfill}%
\pgfsetlinewidth{0.602250pt}%
\definecolor{currentstroke}{rgb}{0.000000,0.000000,0.000000}%
\pgfsetstrokecolor{currentstroke}%
\pgfsetdash{}{0pt}%
\pgfsys@defobject{currentmarker}{\pgfqpoint{0.000000in}{-0.027778in}}{\pgfqpoint{0.000000in}{0.000000in}}{%
\pgfpathmoveto{\pgfqpoint{0.000000in}{0.000000in}}%
\pgfpathlineto{\pgfqpoint{0.000000in}{-0.027778in}}%
\pgfusepath{stroke,fill}%
}%
\begin{pgfscope}%
\pgfsys@transformshift{3.327301in}{0.527436in}%
\pgfsys@useobject{currentmarker}{}%
\end{pgfscope}%
\end{pgfscope}%
\begin{pgfscope}%
\pgfpathrectangle{\pgfqpoint{0.640323in}{0.527436in}}{\pgfqpoint{9.687500in}{3.850000in}}%
\pgfusepath{clip}%
\pgfsetrectcap%
\pgfsetroundjoin%
\pgfsetlinewidth{0.803000pt}%
\definecolor{currentstroke}{rgb}{0.600000,0.600000,0.600000}%
\pgfsetstrokecolor{currentstroke}%
\pgfsetstrokeopacity{0.200000}%
\pgfsetdash{}{0pt}%
\pgfpathmoveto{\pgfqpoint{3.776628in}{0.527436in}}%
\pgfpathlineto{\pgfqpoint{3.776628in}{4.377436in}}%
\pgfusepath{stroke}%
\end{pgfscope}%
\begin{pgfscope}%
\pgfsetbuttcap%
\pgfsetroundjoin%
\definecolor{currentfill}{rgb}{0.000000,0.000000,0.000000}%
\pgfsetfillcolor{currentfill}%
\pgfsetlinewidth{0.602250pt}%
\definecolor{currentstroke}{rgb}{0.000000,0.000000,0.000000}%
\pgfsetstrokecolor{currentstroke}%
\pgfsetdash{}{0pt}%
\pgfsys@defobject{currentmarker}{\pgfqpoint{0.000000in}{-0.027778in}}{\pgfqpoint{0.000000in}{0.000000in}}{%
\pgfpathmoveto{\pgfqpoint{0.000000in}{0.000000in}}%
\pgfpathlineto{\pgfqpoint{0.000000in}{-0.027778in}}%
\pgfusepath{stroke,fill}%
}%
\begin{pgfscope}%
\pgfsys@transformshift{3.776628in}{0.527436in}%
\pgfsys@useobject{currentmarker}{}%
\end{pgfscope}%
\end{pgfscope}%
\begin{pgfscope}%
\pgfpathrectangle{\pgfqpoint{0.640323in}{0.527436in}}{\pgfqpoint{9.687500in}{3.850000in}}%
\pgfusepath{clip}%
\pgfsetrectcap%
\pgfsetroundjoin%
\pgfsetlinewidth{0.803000pt}%
\definecolor{currentstroke}{rgb}{0.600000,0.600000,0.600000}%
\pgfsetstrokecolor{currentstroke}%
\pgfsetstrokeopacity{0.200000}%
\pgfsetdash{}{0pt}%
\pgfpathmoveto{\pgfqpoint{4.225956in}{0.527436in}}%
\pgfpathlineto{\pgfqpoint{4.225956in}{4.377436in}}%
\pgfusepath{stroke}%
\end{pgfscope}%
\begin{pgfscope}%
\pgfsetbuttcap%
\pgfsetroundjoin%
\definecolor{currentfill}{rgb}{0.000000,0.000000,0.000000}%
\pgfsetfillcolor{currentfill}%
\pgfsetlinewidth{0.602250pt}%
\definecolor{currentstroke}{rgb}{0.000000,0.000000,0.000000}%
\pgfsetstrokecolor{currentstroke}%
\pgfsetdash{}{0pt}%
\pgfsys@defobject{currentmarker}{\pgfqpoint{0.000000in}{-0.027778in}}{\pgfqpoint{0.000000in}{0.000000in}}{%
\pgfpathmoveto{\pgfqpoint{0.000000in}{0.000000in}}%
\pgfpathlineto{\pgfqpoint{0.000000in}{-0.027778in}}%
\pgfusepath{stroke,fill}%
}%
\begin{pgfscope}%
\pgfsys@transformshift{4.225956in}{0.527436in}%
\pgfsys@useobject{currentmarker}{}%
\end{pgfscope}%
\end{pgfscope}%
\begin{pgfscope}%
\pgfpathrectangle{\pgfqpoint{0.640323in}{0.527436in}}{\pgfqpoint{9.687500in}{3.850000in}}%
\pgfusepath{clip}%
\pgfsetrectcap%
\pgfsetroundjoin%
\pgfsetlinewidth{0.803000pt}%
\definecolor{currentstroke}{rgb}{0.600000,0.600000,0.600000}%
\pgfsetstrokecolor{currentstroke}%
\pgfsetstrokeopacity{0.200000}%
\pgfsetdash{}{0pt}%
\pgfpathmoveto{\pgfqpoint{5.124611in}{0.527436in}}%
\pgfpathlineto{\pgfqpoint{5.124611in}{4.377436in}}%
\pgfusepath{stroke}%
\end{pgfscope}%
\begin{pgfscope}%
\pgfsetbuttcap%
\pgfsetroundjoin%
\definecolor{currentfill}{rgb}{0.000000,0.000000,0.000000}%
\pgfsetfillcolor{currentfill}%
\pgfsetlinewidth{0.602250pt}%
\definecolor{currentstroke}{rgb}{0.000000,0.000000,0.000000}%
\pgfsetstrokecolor{currentstroke}%
\pgfsetdash{}{0pt}%
\pgfsys@defobject{currentmarker}{\pgfqpoint{0.000000in}{-0.027778in}}{\pgfqpoint{0.000000in}{0.000000in}}{%
\pgfpathmoveto{\pgfqpoint{0.000000in}{0.000000in}}%
\pgfpathlineto{\pgfqpoint{0.000000in}{-0.027778in}}%
\pgfusepath{stroke,fill}%
}%
\begin{pgfscope}%
\pgfsys@transformshift{5.124611in}{0.527436in}%
\pgfsys@useobject{currentmarker}{}%
\end{pgfscope}%
\end{pgfscope}%
\begin{pgfscope}%
\pgfpathrectangle{\pgfqpoint{0.640323in}{0.527436in}}{\pgfqpoint{9.687500in}{3.850000in}}%
\pgfusepath{clip}%
\pgfsetrectcap%
\pgfsetroundjoin%
\pgfsetlinewidth{0.803000pt}%
\definecolor{currentstroke}{rgb}{0.600000,0.600000,0.600000}%
\pgfsetstrokecolor{currentstroke}%
\pgfsetstrokeopacity{0.200000}%
\pgfsetdash{}{0pt}%
\pgfpathmoveto{\pgfqpoint{5.573938in}{0.527436in}}%
\pgfpathlineto{\pgfqpoint{5.573938in}{4.377436in}}%
\pgfusepath{stroke}%
\end{pgfscope}%
\begin{pgfscope}%
\pgfsetbuttcap%
\pgfsetroundjoin%
\definecolor{currentfill}{rgb}{0.000000,0.000000,0.000000}%
\pgfsetfillcolor{currentfill}%
\pgfsetlinewidth{0.602250pt}%
\definecolor{currentstroke}{rgb}{0.000000,0.000000,0.000000}%
\pgfsetstrokecolor{currentstroke}%
\pgfsetdash{}{0pt}%
\pgfsys@defobject{currentmarker}{\pgfqpoint{0.000000in}{-0.027778in}}{\pgfqpoint{0.000000in}{0.000000in}}{%
\pgfpathmoveto{\pgfqpoint{0.000000in}{0.000000in}}%
\pgfpathlineto{\pgfqpoint{0.000000in}{-0.027778in}}%
\pgfusepath{stroke,fill}%
}%
\begin{pgfscope}%
\pgfsys@transformshift{5.573938in}{0.527436in}%
\pgfsys@useobject{currentmarker}{}%
\end{pgfscope}%
\end{pgfscope}%
\begin{pgfscope}%
\pgfpathrectangle{\pgfqpoint{0.640323in}{0.527436in}}{\pgfqpoint{9.687500in}{3.850000in}}%
\pgfusepath{clip}%
\pgfsetrectcap%
\pgfsetroundjoin%
\pgfsetlinewidth{0.803000pt}%
\definecolor{currentstroke}{rgb}{0.600000,0.600000,0.600000}%
\pgfsetstrokecolor{currentstroke}%
\pgfsetstrokeopacity{0.200000}%
\pgfsetdash{}{0pt}%
\pgfpathmoveto{\pgfqpoint{6.023265in}{0.527436in}}%
\pgfpathlineto{\pgfqpoint{6.023265in}{4.377436in}}%
\pgfusepath{stroke}%
\end{pgfscope}%
\begin{pgfscope}%
\pgfsetbuttcap%
\pgfsetroundjoin%
\definecolor{currentfill}{rgb}{0.000000,0.000000,0.000000}%
\pgfsetfillcolor{currentfill}%
\pgfsetlinewidth{0.602250pt}%
\definecolor{currentstroke}{rgb}{0.000000,0.000000,0.000000}%
\pgfsetstrokecolor{currentstroke}%
\pgfsetdash{}{0pt}%
\pgfsys@defobject{currentmarker}{\pgfqpoint{0.000000in}{-0.027778in}}{\pgfqpoint{0.000000in}{0.000000in}}{%
\pgfpathmoveto{\pgfqpoint{0.000000in}{0.000000in}}%
\pgfpathlineto{\pgfqpoint{0.000000in}{-0.027778in}}%
\pgfusepath{stroke,fill}%
}%
\begin{pgfscope}%
\pgfsys@transformshift{6.023265in}{0.527436in}%
\pgfsys@useobject{currentmarker}{}%
\end{pgfscope}%
\end{pgfscope}%
\begin{pgfscope}%
\pgfpathrectangle{\pgfqpoint{0.640323in}{0.527436in}}{\pgfqpoint{9.687500in}{3.850000in}}%
\pgfusepath{clip}%
\pgfsetrectcap%
\pgfsetroundjoin%
\pgfsetlinewidth{0.803000pt}%
\definecolor{currentstroke}{rgb}{0.600000,0.600000,0.600000}%
\pgfsetstrokecolor{currentstroke}%
\pgfsetstrokeopacity{0.200000}%
\pgfsetdash{}{0pt}%
\pgfpathmoveto{\pgfqpoint{6.921920in}{0.527436in}}%
\pgfpathlineto{\pgfqpoint{6.921920in}{4.377436in}}%
\pgfusepath{stroke}%
\end{pgfscope}%
\begin{pgfscope}%
\pgfsetbuttcap%
\pgfsetroundjoin%
\definecolor{currentfill}{rgb}{0.000000,0.000000,0.000000}%
\pgfsetfillcolor{currentfill}%
\pgfsetlinewidth{0.602250pt}%
\definecolor{currentstroke}{rgb}{0.000000,0.000000,0.000000}%
\pgfsetstrokecolor{currentstroke}%
\pgfsetdash{}{0pt}%
\pgfsys@defobject{currentmarker}{\pgfqpoint{0.000000in}{-0.027778in}}{\pgfqpoint{0.000000in}{0.000000in}}{%
\pgfpathmoveto{\pgfqpoint{0.000000in}{0.000000in}}%
\pgfpathlineto{\pgfqpoint{0.000000in}{-0.027778in}}%
\pgfusepath{stroke,fill}%
}%
\begin{pgfscope}%
\pgfsys@transformshift{6.921920in}{0.527436in}%
\pgfsys@useobject{currentmarker}{}%
\end{pgfscope}%
\end{pgfscope}%
\begin{pgfscope}%
\pgfpathrectangle{\pgfqpoint{0.640323in}{0.527436in}}{\pgfqpoint{9.687500in}{3.850000in}}%
\pgfusepath{clip}%
\pgfsetrectcap%
\pgfsetroundjoin%
\pgfsetlinewidth{0.803000pt}%
\definecolor{currentstroke}{rgb}{0.600000,0.600000,0.600000}%
\pgfsetstrokecolor{currentstroke}%
\pgfsetstrokeopacity{0.200000}%
\pgfsetdash{}{0pt}%
\pgfpathmoveto{\pgfqpoint{7.371248in}{0.527436in}}%
\pgfpathlineto{\pgfqpoint{7.371248in}{4.377436in}}%
\pgfusepath{stroke}%
\end{pgfscope}%
\begin{pgfscope}%
\pgfsetbuttcap%
\pgfsetroundjoin%
\definecolor{currentfill}{rgb}{0.000000,0.000000,0.000000}%
\pgfsetfillcolor{currentfill}%
\pgfsetlinewidth{0.602250pt}%
\definecolor{currentstroke}{rgb}{0.000000,0.000000,0.000000}%
\pgfsetstrokecolor{currentstroke}%
\pgfsetdash{}{0pt}%
\pgfsys@defobject{currentmarker}{\pgfqpoint{0.000000in}{-0.027778in}}{\pgfqpoint{0.000000in}{0.000000in}}{%
\pgfpathmoveto{\pgfqpoint{0.000000in}{0.000000in}}%
\pgfpathlineto{\pgfqpoint{0.000000in}{-0.027778in}}%
\pgfusepath{stroke,fill}%
}%
\begin{pgfscope}%
\pgfsys@transformshift{7.371248in}{0.527436in}%
\pgfsys@useobject{currentmarker}{}%
\end{pgfscope}%
\end{pgfscope}%
\begin{pgfscope}%
\pgfpathrectangle{\pgfqpoint{0.640323in}{0.527436in}}{\pgfqpoint{9.687500in}{3.850000in}}%
\pgfusepath{clip}%
\pgfsetrectcap%
\pgfsetroundjoin%
\pgfsetlinewidth{0.803000pt}%
\definecolor{currentstroke}{rgb}{0.600000,0.600000,0.600000}%
\pgfsetstrokecolor{currentstroke}%
\pgfsetstrokeopacity{0.200000}%
\pgfsetdash{}{0pt}%
\pgfpathmoveto{\pgfqpoint{7.820575in}{0.527436in}}%
\pgfpathlineto{\pgfqpoint{7.820575in}{4.377436in}}%
\pgfusepath{stroke}%
\end{pgfscope}%
\begin{pgfscope}%
\pgfsetbuttcap%
\pgfsetroundjoin%
\definecolor{currentfill}{rgb}{0.000000,0.000000,0.000000}%
\pgfsetfillcolor{currentfill}%
\pgfsetlinewidth{0.602250pt}%
\definecolor{currentstroke}{rgb}{0.000000,0.000000,0.000000}%
\pgfsetstrokecolor{currentstroke}%
\pgfsetdash{}{0pt}%
\pgfsys@defobject{currentmarker}{\pgfqpoint{0.000000in}{-0.027778in}}{\pgfqpoint{0.000000in}{0.000000in}}{%
\pgfpathmoveto{\pgfqpoint{0.000000in}{0.000000in}}%
\pgfpathlineto{\pgfqpoint{0.000000in}{-0.027778in}}%
\pgfusepath{stroke,fill}%
}%
\begin{pgfscope}%
\pgfsys@transformshift{7.820575in}{0.527436in}%
\pgfsys@useobject{currentmarker}{}%
\end{pgfscope}%
\end{pgfscope}%
\begin{pgfscope}%
\pgfpathrectangle{\pgfqpoint{0.640323in}{0.527436in}}{\pgfqpoint{9.687500in}{3.850000in}}%
\pgfusepath{clip}%
\pgfsetrectcap%
\pgfsetroundjoin%
\pgfsetlinewidth{0.803000pt}%
\definecolor{currentstroke}{rgb}{0.600000,0.600000,0.600000}%
\pgfsetstrokecolor{currentstroke}%
\pgfsetstrokeopacity{0.200000}%
\pgfsetdash{}{0pt}%
\pgfpathmoveto{\pgfqpoint{8.719230in}{0.527436in}}%
\pgfpathlineto{\pgfqpoint{8.719230in}{4.377436in}}%
\pgfusepath{stroke}%
\end{pgfscope}%
\begin{pgfscope}%
\pgfsetbuttcap%
\pgfsetroundjoin%
\definecolor{currentfill}{rgb}{0.000000,0.000000,0.000000}%
\pgfsetfillcolor{currentfill}%
\pgfsetlinewidth{0.602250pt}%
\definecolor{currentstroke}{rgb}{0.000000,0.000000,0.000000}%
\pgfsetstrokecolor{currentstroke}%
\pgfsetdash{}{0pt}%
\pgfsys@defobject{currentmarker}{\pgfqpoint{0.000000in}{-0.027778in}}{\pgfqpoint{0.000000in}{0.000000in}}{%
\pgfpathmoveto{\pgfqpoint{0.000000in}{0.000000in}}%
\pgfpathlineto{\pgfqpoint{0.000000in}{-0.027778in}}%
\pgfusepath{stroke,fill}%
}%
\begin{pgfscope}%
\pgfsys@transformshift{8.719230in}{0.527436in}%
\pgfsys@useobject{currentmarker}{}%
\end{pgfscope}%
\end{pgfscope}%
\begin{pgfscope}%
\pgfpathrectangle{\pgfqpoint{0.640323in}{0.527436in}}{\pgfqpoint{9.687500in}{3.850000in}}%
\pgfusepath{clip}%
\pgfsetrectcap%
\pgfsetroundjoin%
\pgfsetlinewidth{0.803000pt}%
\definecolor{currentstroke}{rgb}{0.600000,0.600000,0.600000}%
\pgfsetstrokecolor{currentstroke}%
\pgfsetstrokeopacity{0.200000}%
\pgfsetdash{}{0pt}%
\pgfpathmoveto{\pgfqpoint{9.168558in}{0.527436in}}%
\pgfpathlineto{\pgfqpoint{9.168558in}{4.377436in}}%
\pgfusepath{stroke}%
\end{pgfscope}%
\begin{pgfscope}%
\pgfsetbuttcap%
\pgfsetroundjoin%
\definecolor{currentfill}{rgb}{0.000000,0.000000,0.000000}%
\pgfsetfillcolor{currentfill}%
\pgfsetlinewidth{0.602250pt}%
\definecolor{currentstroke}{rgb}{0.000000,0.000000,0.000000}%
\pgfsetstrokecolor{currentstroke}%
\pgfsetdash{}{0pt}%
\pgfsys@defobject{currentmarker}{\pgfqpoint{0.000000in}{-0.027778in}}{\pgfqpoint{0.000000in}{0.000000in}}{%
\pgfpathmoveto{\pgfqpoint{0.000000in}{0.000000in}}%
\pgfpathlineto{\pgfqpoint{0.000000in}{-0.027778in}}%
\pgfusepath{stroke,fill}%
}%
\begin{pgfscope}%
\pgfsys@transformshift{9.168558in}{0.527436in}%
\pgfsys@useobject{currentmarker}{}%
\end{pgfscope}%
\end{pgfscope}%
\begin{pgfscope}%
\pgfpathrectangle{\pgfqpoint{0.640323in}{0.527436in}}{\pgfqpoint{9.687500in}{3.850000in}}%
\pgfusepath{clip}%
\pgfsetrectcap%
\pgfsetroundjoin%
\pgfsetlinewidth{0.803000pt}%
\definecolor{currentstroke}{rgb}{0.600000,0.600000,0.600000}%
\pgfsetstrokecolor{currentstroke}%
\pgfsetstrokeopacity{0.200000}%
\pgfsetdash{}{0pt}%
\pgfpathmoveto{\pgfqpoint{9.617885in}{0.527436in}}%
\pgfpathlineto{\pgfqpoint{9.617885in}{4.377436in}}%
\pgfusepath{stroke}%
\end{pgfscope}%
\begin{pgfscope}%
\pgfsetbuttcap%
\pgfsetroundjoin%
\definecolor{currentfill}{rgb}{0.000000,0.000000,0.000000}%
\pgfsetfillcolor{currentfill}%
\pgfsetlinewidth{0.602250pt}%
\definecolor{currentstroke}{rgb}{0.000000,0.000000,0.000000}%
\pgfsetstrokecolor{currentstroke}%
\pgfsetdash{}{0pt}%
\pgfsys@defobject{currentmarker}{\pgfqpoint{0.000000in}{-0.027778in}}{\pgfqpoint{0.000000in}{0.000000in}}{%
\pgfpathmoveto{\pgfqpoint{0.000000in}{0.000000in}}%
\pgfpathlineto{\pgfqpoint{0.000000in}{-0.027778in}}%
\pgfusepath{stroke,fill}%
}%
\begin{pgfscope}%
\pgfsys@transformshift{9.617885in}{0.527436in}%
\pgfsys@useobject{currentmarker}{}%
\end{pgfscope}%
\end{pgfscope}%
\begin{pgfscope}%
\definecolor{textcolor}{rgb}{0.000000,0.000000,0.000000}%
\pgfsetstrokecolor{textcolor}%
\pgfsetfillcolor{textcolor}%
\pgftext[x=5.484072in,y=0.240245in,,top]{\color{textcolor}\sffamily\fontsize{10.000000}{12.000000}\selectfont turnover probability \(\displaystyle p_1\,(S\rightarrow I\,)\)}%
\end{pgfscope}%
\begin{pgfscope}%
\pgfpathrectangle{\pgfqpoint{0.640323in}{0.527436in}}{\pgfqpoint{9.687500in}{3.850000in}}%
\pgfusepath{clip}%
\pgfsetrectcap%
\pgfsetroundjoin%
\pgfsetlinewidth{0.803000pt}%
\definecolor{currentstroke}{rgb}{0.690196,0.690196,0.690196}%
\pgfsetstrokecolor{currentstroke}%
\pgfsetdash{}{0pt}%
\pgfpathmoveto{\pgfqpoint{0.640323in}{0.651629in}}%
\pgfpathlineto{\pgfqpoint{10.327822in}{0.651629in}}%
\pgfusepath{stroke}%
\end{pgfscope}%
\begin{pgfscope}%
\pgfsetbuttcap%
\pgfsetroundjoin%
\definecolor{currentfill}{rgb}{0.000000,0.000000,0.000000}%
\pgfsetfillcolor{currentfill}%
\pgfsetlinewidth{0.803000pt}%
\definecolor{currentstroke}{rgb}{0.000000,0.000000,0.000000}%
\pgfsetstrokecolor{currentstroke}%
\pgfsetdash{}{0pt}%
\pgfsys@defobject{currentmarker}{\pgfqpoint{-0.048611in}{0.000000in}}{\pgfqpoint{-0.000000in}{0.000000in}}{%
\pgfpathmoveto{\pgfqpoint{-0.000000in}{0.000000in}}%
\pgfpathlineto{\pgfqpoint{-0.048611in}{0.000000in}}%
\pgfusepath{stroke,fill}%
}%
\begin{pgfscope}%
\pgfsys@transformshift{0.640323in}{0.651629in}%
\pgfsys@useobject{currentmarker}{}%
\end{pgfscope}%
\end{pgfscope}%
\begin{pgfscope}%
\definecolor{textcolor}{rgb}{0.000000,0.000000,0.000000}%
\pgfsetstrokecolor{textcolor}%
\pgfsetfillcolor{textcolor}%
\pgftext[x=0.322221in, y=0.598868in, left, base]{\color{textcolor}\sffamily\fontsize{10.000000}{12.000000}\selectfont 0.0}%
\end{pgfscope}%
\begin{pgfscope}%
\pgfpathrectangle{\pgfqpoint{0.640323in}{0.527436in}}{\pgfqpoint{9.687500in}{3.850000in}}%
\pgfusepath{clip}%
\pgfsetrectcap%
\pgfsetroundjoin%
\pgfsetlinewidth{0.803000pt}%
\definecolor{currentstroke}{rgb}{0.690196,0.690196,0.690196}%
\pgfsetstrokecolor{currentstroke}%
\pgfsetdash{}{0pt}%
\pgfpathmoveto{\pgfqpoint{0.640323in}{1.272597in}}%
\pgfpathlineto{\pgfqpoint{10.327822in}{1.272597in}}%
\pgfusepath{stroke}%
\end{pgfscope}%
\begin{pgfscope}%
\pgfsetbuttcap%
\pgfsetroundjoin%
\definecolor{currentfill}{rgb}{0.000000,0.000000,0.000000}%
\pgfsetfillcolor{currentfill}%
\pgfsetlinewidth{0.803000pt}%
\definecolor{currentstroke}{rgb}{0.000000,0.000000,0.000000}%
\pgfsetstrokecolor{currentstroke}%
\pgfsetdash{}{0pt}%
\pgfsys@defobject{currentmarker}{\pgfqpoint{-0.048611in}{0.000000in}}{\pgfqpoint{-0.000000in}{0.000000in}}{%
\pgfpathmoveto{\pgfqpoint{-0.000000in}{0.000000in}}%
\pgfpathlineto{\pgfqpoint{-0.048611in}{0.000000in}}%
\pgfusepath{stroke,fill}%
}%
\begin{pgfscope}%
\pgfsys@transformshift{0.640323in}{1.272597in}%
\pgfsys@useobject{currentmarker}{}%
\end{pgfscope}%
\end{pgfscope}%
\begin{pgfscope}%
\definecolor{textcolor}{rgb}{0.000000,0.000000,0.000000}%
\pgfsetstrokecolor{textcolor}%
\pgfsetfillcolor{textcolor}%
\pgftext[x=0.322221in, y=1.219836in, left, base]{\color{textcolor}\sffamily\fontsize{10.000000}{12.000000}\selectfont 0.1}%
\end{pgfscope}%
\begin{pgfscope}%
\pgfpathrectangle{\pgfqpoint{0.640323in}{0.527436in}}{\pgfqpoint{9.687500in}{3.850000in}}%
\pgfusepath{clip}%
\pgfsetrectcap%
\pgfsetroundjoin%
\pgfsetlinewidth{0.803000pt}%
\definecolor{currentstroke}{rgb}{0.690196,0.690196,0.690196}%
\pgfsetstrokecolor{currentstroke}%
\pgfsetdash{}{0pt}%
\pgfpathmoveto{\pgfqpoint{0.640323in}{1.893565in}}%
\pgfpathlineto{\pgfqpoint{10.327822in}{1.893565in}}%
\pgfusepath{stroke}%
\end{pgfscope}%
\begin{pgfscope}%
\pgfsetbuttcap%
\pgfsetroundjoin%
\definecolor{currentfill}{rgb}{0.000000,0.000000,0.000000}%
\pgfsetfillcolor{currentfill}%
\pgfsetlinewidth{0.803000pt}%
\definecolor{currentstroke}{rgb}{0.000000,0.000000,0.000000}%
\pgfsetstrokecolor{currentstroke}%
\pgfsetdash{}{0pt}%
\pgfsys@defobject{currentmarker}{\pgfqpoint{-0.048611in}{0.000000in}}{\pgfqpoint{-0.000000in}{0.000000in}}{%
\pgfpathmoveto{\pgfqpoint{-0.000000in}{0.000000in}}%
\pgfpathlineto{\pgfqpoint{-0.048611in}{0.000000in}}%
\pgfusepath{stroke,fill}%
}%
\begin{pgfscope}%
\pgfsys@transformshift{0.640323in}{1.893565in}%
\pgfsys@useobject{currentmarker}{}%
\end{pgfscope}%
\end{pgfscope}%
\begin{pgfscope}%
\definecolor{textcolor}{rgb}{0.000000,0.000000,0.000000}%
\pgfsetstrokecolor{textcolor}%
\pgfsetfillcolor{textcolor}%
\pgftext[x=0.322221in, y=1.840803in, left, base]{\color{textcolor}\sffamily\fontsize{10.000000}{12.000000}\selectfont 0.2}%
\end{pgfscope}%
\begin{pgfscope}%
\pgfpathrectangle{\pgfqpoint{0.640323in}{0.527436in}}{\pgfqpoint{9.687500in}{3.850000in}}%
\pgfusepath{clip}%
\pgfsetrectcap%
\pgfsetroundjoin%
\pgfsetlinewidth{0.803000pt}%
\definecolor{currentstroke}{rgb}{0.690196,0.690196,0.690196}%
\pgfsetstrokecolor{currentstroke}%
\pgfsetdash{}{0pt}%
\pgfpathmoveto{\pgfqpoint{0.640323in}{2.514533in}}%
\pgfpathlineto{\pgfqpoint{10.327822in}{2.514533in}}%
\pgfusepath{stroke}%
\end{pgfscope}%
\begin{pgfscope}%
\pgfsetbuttcap%
\pgfsetroundjoin%
\definecolor{currentfill}{rgb}{0.000000,0.000000,0.000000}%
\pgfsetfillcolor{currentfill}%
\pgfsetlinewidth{0.803000pt}%
\definecolor{currentstroke}{rgb}{0.000000,0.000000,0.000000}%
\pgfsetstrokecolor{currentstroke}%
\pgfsetdash{}{0pt}%
\pgfsys@defobject{currentmarker}{\pgfqpoint{-0.048611in}{0.000000in}}{\pgfqpoint{-0.000000in}{0.000000in}}{%
\pgfpathmoveto{\pgfqpoint{-0.000000in}{0.000000in}}%
\pgfpathlineto{\pgfqpoint{-0.048611in}{0.000000in}}%
\pgfusepath{stroke,fill}%
}%
\begin{pgfscope}%
\pgfsys@transformshift{0.640323in}{2.514533in}%
\pgfsys@useobject{currentmarker}{}%
\end{pgfscope}%
\end{pgfscope}%
\begin{pgfscope}%
\definecolor{textcolor}{rgb}{0.000000,0.000000,0.000000}%
\pgfsetstrokecolor{textcolor}%
\pgfsetfillcolor{textcolor}%
\pgftext[x=0.322221in, y=2.461771in, left, base]{\color{textcolor}\sffamily\fontsize{10.000000}{12.000000}\selectfont 0.3}%
\end{pgfscope}%
\begin{pgfscope}%
\pgfpathrectangle{\pgfqpoint{0.640323in}{0.527436in}}{\pgfqpoint{9.687500in}{3.850000in}}%
\pgfusepath{clip}%
\pgfsetrectcap%
\pgfsetroundjoin%
\pgfsetlinewidth{0.803000pt}%
\definecolor{currentstroke}{rgb}{0.690196,0.690196,0.690196}%
\pgfsetstrokecolor{currentstroke}%
\pgfsetdash{}{0pt}%
\pgfpathmoveto{\pgfqpoint{0.640323in}{3.135500in}}%
\pgfpathlineto{\pgfqpoint{10.327822in}{3.135500in}}%
\pgfusepath{stroke}%
\end{pgfscope}%
\begin{pgfscope}%
\pgfsetbuttcap%
\pgfsetroundjoin%
\definecolor{currentfill}{rgb}{0.000000,0.000000,0.000000}%
\pgfsetfillcolor{currentfill}%
\pgfsetlinewidth{0.803000pt}%
\definecolor{currentstroke}{rgb}{0.000000,0.000000,0.000000}%
\pgfsetstrokecolor{currentstroke}%
\pgfsetdash{}{0pt}%
\pgfsys@defobject{currentmarker}{\pgfqpoint{-0.048611in}{0.000000in}}{\pgfqpoint{-0.000000in}{0.000000in}}{%
\pgfpathmoveto{\pgfqpoint{-0.000000in}{0.000000in}}%
\pgfpathlineto{\pgfqpoint{-0.048611in}{0.000000in}}%
\pgfusepath{stroke,fill}%
}%
\begin{pgfscope}%
\pgfsys@transformshift{0.640323in}{3.135500in}%
\pgfsys@useobject{currentmarker}{}%
\end{pgfscope}%
\end{pgfscope}%
\begin{pgfscope}%
\definecolor{textcolor}{rgb}{0.000000,0.000000,0.000000}%
\pgfsetstrokecolor{textcolor}%
\pgfsetfillcolor{textcolor}%
\pgftext[x=0.322221in, y=3.082739in, left, base]{\color{textcolor}\sffamily\fontsize{10.000000}{12.000000}\selectfont 0.4}%
\end{pgfscope}%
\begin{pgfscope}%
\pgfpathrectangle{\pgfqpoint{0.640323in}{0.527436in}}{\pgfqpoint{9.687500in}{3.850000in}}%
\pgfusepath{clip}%
\pgfsetrectcap%
\pgfsetroundjoin%
\pgfsetlinewidth{0.803000pt}%
\definecolor{currentstroke}{rgb}{0.690196,0.690196,0.690196}%
\pgfsetstrokecolor{currentstroke}%
\pgfsetdash{}{0pt}%
\pgfpathmoveto{\pgfqpoint{0.640323in}{3.756468in}}%
\pgfpathlineto{\pgfqpoint{10.327822in}{3.756468in}}%
\pgfusepath{stroke}%
\end{pgfscope}%
\begin{pgfscope}%
\pgfsetbuttcap%
\pgfsetroundjoin%
\definecolor{currentfill}{rgb}{0.000000,0.000000,0.000000}%
\pgfsetfillcolor{currentfill}%
\pgfsetlinewidth{0.803000pt}%
\definecolor{currentstroke}{rgb}{0.000000,0.000000,0.000000}%
\pgfsetstrokecolor{currentstroke}%
\pgfsetdash{}{0pt}%
\pgfsys@defobject{currentmarker}{\pgfqpoint{-0.048611in}{0.000000in}}{\pgfqpoint{-0.000000in}{0.000000in}}{%
\pgfpathmoveto{\pgfqpoint{-0.000000in}{0.000000in}}%
\pgfpathlineto{\pgfqpoint{-0.048611in}{0.000000in}}%
\pgfusepath{stroke,fill}%
}%
\begin{pgfscope}%
\pgfsys@transformshift{0.640323in}{3.756468in}%
\pgfsys@useobject{currentmarker}{}%
\end{pgfscope}%
\end{pgfscope}%
\begin{pgfscope}%
\definecolor{textcolor}{rgb}{0.000000,0.000000,0.000000}%
\pgfsetstrokecolor{textcolor}%
\pgfsetfillcolor{textcolor}%
\pgftext[x=0.322221in, y=3.703707in, left, base]{\color{textcolor}\sffamily\fontsize{10.000000}{12.000000}\selectfont 0.5}%
\end{pgfscope}%
\begin{pgfscope}%
\pgfpathrectangle{\pgfqpoint{0.640323in}{0.527436in}}{\pgfqpoint{9.687500in}{3.850000in}}%
\pgfusepath{clip}%
\pgfsetrectcap%
\pgfsetroundjoin%
\pgfsetlinewidth{0.803000pt}%
\definecolor{currentstroke}{rgb}{0.690196,0.690196,0.690196}%
\pgfsetstrokecolor{currentstroke}%
\pgfsetdash{}{0pt}%
\pgfpathmoveto{\pgfqpoint{0.640323in}{4.377436in}}%
\pgfpathlineto{\pgfqpoint{10.327822in}{4.377436in}}%
\pgfusepath{stroke}%
\end{pgfscope}%
\begin{pgfscope}%
\pgfsetbuttcap%
\pgfsetroundjoin%
\definecolor{currentfill}{rgb}{0.000000,0.000000,0.000000}%
\pgfsetfillcolor{currentfill}%
\pgfsetlinewidth{0.803000pt}%
\definecolor{currentstroke}{rgb}{0.000000,0.000000,0.000000}%
\pgfsetstrokecolor{currentstroke}%
\pgfsetdash{}{0pt}%
\pgfsys@defobject{currentmarker}{\pgfqpoint{-0.048611in}{0.000000in}}{\pgfqpoint{-0.000000in}{0.000000in}}{%
\pgfpathmoveto{\pgfqpoint{-0.000000in}{0.000000in}}%
\pgfpathlineto{\pgfqpoint{-0.048611in}{0.000000in}}%
\pgfusepath{stroke,fill}%
}%
\begin{pgfscope}%
\pgfsys@transformshift{0.640323in}{4.377436in}%
\pgfsys@useobject{currentmarker}{}%
\end{pgfscope}%
\end{pgfscope}%
\begin{pgfscope}%
\definecolor{textcolor}{rgb}{0.000000,0.000000,0.000000}%
\pgfsetstrokecolor{textcolor}%
\pgfsetfillcolor{textcolor}%
\pgftext[x=0.322221in, y=4.324674in, left, base]{\color{textcolor}\sffamily\fontsize{10.000000}{12.000000}\selectfont 0.6}%
\end{pgfscope}%
\begin{pgfscope}%
\pgfpathrectangle{\pgfqpoint{0.640323in}{0.527436in}}{\pgfqpoint{9.687500in}{3.850000in}}%
\pgfusepath{clip}%
\pgfsetrectcap%
\pgfsetroundjoin%
\pgfsetlinewidth{0.803000pt}%
\definecolor{currentstroke}{rgb}{0.600000,0.600000,0.600000}%
\pgfsetstrokecolor{currentstroke}%
\pgfsetstrokeopacity{0.200000}%
\pgfsetdash{}{0pt}%
\pgfpathmoveto{\pgfqpoint{0.640323in}{0.775823in}}%
\pgfpathlineto{\pgfqpoint{10.327822in}{0.775823in}}%
\pgfusepath{stroke}%
\end{pgfscope}%
\begin{pgfscope}%
\pgfsetbuttcap%
\pgfsetroundjoin%
\definecolor{currentfill}{rgb}{0.000000,0.000000,0.000000}%
\pgfsetfillcolor{currentfill}%
\pgfsetlinewidth{0.602250pt}%
\definecolor{currentstroke}{rgb}{0.000000,0.000000,0.000000}%
\pgfsetstrokecolor{currentstroke}%
\pgfsetdash{}{0pt}%
\pgfsys@defobject{currentmarker}{\pgfqpoint{-0.027778in}{0.000000in}}{\pgfqpoint{-0.000000in}{0.000000in}}{%
\pgfpathmoveto{\pgfqpoint{-0.000000in}{0.000000in}}%
\pgfpathlineto{\pgfqpoint{-0.027778in}{0.000000in}}%
\pgfusepath{stroke,fill}%
}%
\begin{pgfscope}%
\pgfsys@transformshift{0.640323in}{0.775823in}%
\pgfsys@useobject{currentmarker}{}%
\end{pgfscope}%
\end{pgfscope}%
\begin{pgfscope}%
\pgfpathrectangle{\pgfqpoint{0.640323in}{0.527436in}}{\pgfqpoint{9.687500in}{3.850000in}}%
\pgfusepath{clip}%
\pgfsetrectcap%
\pgfsetroundjoin%
\pgfsetlinewidth{0.803000pt}%
\definecolor{currentstroke}{rgb}{0.600000,0.600000,0.600000}%
\pgfsetstrokecolor{currentstroke}%
\pgfsetstrokeopacity{0.200000}%
\pgfsetdash{}{0pt}%
\pgfpathmoveto{\pgfqpoint{0.640323in}{0.900016in}}%
\pgfpathlineto{\pgfqpoint{10.327822in}{0.900016in}}%
\pgfusepath{stroke}%
\end{pgfscope}%
\begin{pgfscope}%
\pgfsetbuttcap%
\pgfsetroundjoin%
\definecolor{currentfill}{rgb}{0.000000,0.000000,0.000000}%
\pgfsetfillcolor{currentfill}%
\pgfsetlinewidth{0.602250pt}%
\definecolor{currentstroke}{rgb}{0.000000,0.000000,0.000000}%
\pgfsetstrokecolor{currentstroke}%
\pgfsetdash{}{0pt}%
\pgfsys@defobject{currentmarker}{\pgfqpoint{-0.027778in}{0.000000in}}{\pgfqpoint{-0.000000in}{0.000000in}}{%
\pgfpathmoveto{\pgfqpoint{-0.000000in}{0.000000in}}%
\pgfpathlineto{\pgfqpoint{-0.027778in}{0.000000in}}%
\pgfusepath{stroke,fill}%
}%
\begin{pgfscope}%
\pgfsys@transformshift{0.640323in}{0.900016in}%
\pgfsys@useobject{currentmarker}{}%
\end{pgfscope}%
\end{pgfscope}%
\begin{pgfscope}%
\pgfpathrectangle{\pgfqpoint{0.640323in}{0.527436in}}{\pgfqpoint{9.687500in}{3.850000in}}%
\pgfusepath{clip}%
\pgfsetrectcap%
\pgfsetroundjoin%
\pgfsetlinewidth{0.803000pt}%
\definecolor{currentstroke}{rgb}{0.600000,0.600000,0.600000}%
\pgfsetstrokecolor{currentstroke}%
\pgfsetstrokeopacity{0.200000}%
\pgfsetdash{}{0pt}%
\pgfpathmoveto{\pgfqpoint{0.640323in}{1.024210in}}%
\pgfpathlineto{\pgfqpoint{10.327822in}{1.024210in}}%
\pgfusepath{stroke}%
\end{pgfscope}%
\begin{pgfscope}%
\pgfsetbuttcap%
\pgfsetroundjoin%
\definecolor{currentfill}{rgb}{0.000000,0.000000,0.000000}%
\pgfsetfillcolor{currentfill}%
\pgfsetlinewidth{0.602250pt}%
\definecolor{currentstroke}{rgb}{0.000000,0.000000,0.000000}%
\pgfsetstrokecolor{currentstroke}%
\pgfsetdash{}{0pt}%
\pgfsys@defobject{currentmarker}{\pgfqpoint{-0.027778in}{0.000000in}}{\pgfqpoint{-0.000000in}{0.000000in}}{%
\pgfpathmoveto{\pgfqpoint{-0.000000in}{0.000000in}}%
\pgfpathlineto{\pgfqpoint{-0.027778in}{0.000000in}}%
\pgfusepath{stroke,fill}%
}%
\begin{pgfscope}%
\pgfsys@transformshift{0.640323in}{1.024210in}%
\pgfsys@useobject{currentmarker}{}%
\end{pgfscope}%
\end{pgfscope}%
\begin{pgfscope}%
\pgfpathrectangle{\pgfqpoint{0.640323in}{0.527436in}}{\pgfqpoint{9.687500in}{3.850000in}}%
\pgfusepath{clip}%
\pgfsetrectcap%
\pgfsetroundjoin%
\pgfsetlinewidth{0.803000pt}%
\definecolor{currentstroke}{rgb}{0.600000,0.600000,0.600000}%
\pgfsetstrokecolor{currentstroke}%
\pgfsetstrokeopacity{0.200000}%
\pgfsetdash{}{0pt}%
\pgfpathmoveto{\pgfqpoint{0.640323in}{1.148404in}}%
\pgfpathlineto{\pgfqpoint{10.327822in}{1.148404in}}%
\pgfusepath{stroke}%
\end{pgfscope}%
\begin{pgfscope}%
\pgfsetbuttcap%
\pgfsetroundjoin%
\definecolor{currentfill}{rgb}{0.000000,0.000000,0.000000}%
\pgfsetfillcolor{currentfill}%
\pgfsetlinewidth{0.602250pt}%
\definecolor{currentstroke}{rgb}{0.000000,0.000000,0.000000}%
\pgfsetstrokecolor{currentstroke}%
\pgfsetdash{}{0pt}%
\pgfsys@defobject{currentmarker}{\pgfqpoint{-0.027778in}{0.000000in}}{\pgfqpoint{-0.000000in}{0.000000in}}{%
\pgfpathmoveto{\pgfqpoint{-0.000000in}{0.000000in}}%
\pgfpathlineto{\pgfqpoint{-0.027778in}{0.000000in}}%
\pgfusepath{stroke,fill}%
}%
\begin{pgfscope}%
\pgfsys@transformshift{0.640323in}{1.148404in}%
\pgfsys@useobject{currentmarker}{}%
\end{pgfscope}%
\end{pgfscope}%
\begin{pgfscope}%
\pgfpathrectangle{\pgfqpoint{0.640323in}{0.527436in}}{\pgfqpoint{9.687500in}{3.850000in}}%
\pgfusepath{clip}%
\pgfsetrectcap%
\pgfsetroundjoin%
\pgfsetlinewidth{0.803000pt}%
\definecolor{currentstroke}{rgb}{0.600000,0.600000,0.600000}%
\pgfsetstrokecolor{currentstroke}%
\pgfsetstrokeopacity{0.200000}%
\pgfsetdash{}{0pt}%
\pgfpathmoveto{\pgfqpoint{0.640323in}{1.396791in}}%
\pgfpathlineto{\pgfqpoint{10.327822in}{1.396791in}}%
\pgfusepath{stroke}%
\end{pgfscope}%
\begin{pgfscope}%
\pgfsetbuttcap%
\pgfsetroundjoin%
\definecolor{currentfill}{rgb}{0.000000,0.000000,0.000000}%
\pgfsetfillcolor{currentfill}%
\pgfsetlinewidth{0.602250pt}%
\definecolor{currentstroke}{rgb}{0.000000,0.000000,0.000000}%
\pgfsetstrokecolor{currentstroke}%
\pgfsetdash{}{0pt}%
\pgfsys@defobject{currentmarker}{\pgfqpoint{-0.027778in}{0.000000in}}{\pgfqpoint{-0.000000in}{0.000000in}}{%
\pgfpathmoveto{\pgfqpoint{-0.000000in}{0.000000in}}%
\pgfpathlineto{\pgfqpoint{-0.027778in}{0.000000in}}%
\pgfusepath{stroke,fill}%
}%
\begin{pgfscope}%
\pgfsys@transformshift{0.640323in}{1.396791in}%
\pgfsys@useobject{currentmarker}{}%
\end{pgfscope}%
\end{pgfscope}%
\begin{pgfscope}%
\pgfpathrectangle{\pgfqpoint{0.640323in}{0.527436in}}{\pgfqpoint{9.687500in}{3.850000in}}%
\pgfusepath{clip}%
\pgfsetrectcap%
\pgfsetroundjoin%
\pgfsetlinewidth{0.803000pt}%
\definecolor{currentstroke}{rgb}{0.600000,0.600000,0.600000}%
\pgfsetstrokecolor{currentstroke}%
\pgfsetstrokeopacity{0.200000}%
\pgfsetdash{}{0pt}%
\pgfpathmoveto{\pgfqpoint{0.640323in}{1.520984in}}%
\pgfpathlineto{\pgfqpoint{10.327822in}{1.520984in}}%
\pgfusepath{stroke}%
\end{pgfscope}%
\begin{pgfscope}%
\pgfsetbuttcap%
\pgfsetroundjoin%
\definecolor{currentfill}{rgb}{0.000000,0.000000,0.000000}%
\pgfsetfillcolor{currentfill}%
\pgfsetlinewidth{0.602250pt}%
\definecolor{currentstroke}{rgb}{0.000000,0.000000,0.000000}%
\pgfsetstrokecolor{currentstroke}%
\pgfsetdash{}{0pt}%
\pgfsys@defobject{currentmarker}{\pgfqpoint{-0.027778in}{0.000000in}}{\pgfqpoint{-0.000000in}{0.000000in}}{%
\pgfpathmoveto{\pgfqpoint{-0.000000in}{0.000000in}}%
\pgfpathlineto{\pgfqpoint{-0.027778in}{0.000000in}}%
\pgfusepath{stroke,fill}%
}%
\begin{pgfscope}%
\pgfsys@transformshift{0.640323in}{1.520984in}%
\pgfsys@useobject{currentmarker}{}%
\end{pgfscope}%
\end{pgfscope}%
\begin{pgfscope}%
\pgfpathrectangle{\pgfqpoint{0.640323in}{0.527436in}}{\pgfqpoint{9.687500in}{3.850000in}}%
\pgfusepath{clip}%
\pgfsetrectcap%
\pgfsetroundjoin%
\pgfsetlinewidth{0.803000pt}%
\definecolor{currentstroke}{rgb}{0.600000,0.600000,0.600000}%
\pgfsetstrokecolor{currentstroke}%
\pgfsetstrokeopacity{0.200000}%
\pgfsetdash{}{0pt}%
\pgfpathmoveto{\pgfqpoint{0.640323in}{1.645178in}}%
\pgfpathlineto{\pgfqpoint{10.327822in}{1.645178in}}%
\pgfusepath{stroke}%
\end{pgfscope}%
\begin{pgfscope}%
\pgfsetbuttcap%
\pgfsetroundjoin%
\definecolor{currentfill}{rgb}{0.000000,0.000000,0.000000}%
\pgfsetfillcolor{currentfill}%
\pgfsetlinewidth{0.602250pt}%
\definecolor{currentstroke}{rgb}{0.000000,0.000000,0.000000}%
\pgfsetstrokecolor{currentstroke}%
\pgfsetdash{}{0pt}%
\pgfsys@defobject{currentmarker}{\pgfqpoint{-0.027778in}{0.000000in}}{\pgfqpoint{-0.000000in}{0.000000in}}{%
\pgfpathmoveto{\pgfqpoint{-0.000000in}{0.000000in}}%
\pgfpathlineto{\pgfqpoint{-0.027778in}{0.000000in}}%
\pgfusepath{stroke,fill}%
}%
\begin{pgfscope}%
\pgfsys@transformshift{0.640323in}{1.645178in}%
\pgfsys@useobject{currentmarker}{}%
\end{pgfscope}%
\end{pgfscope}%
\begin{pgfscope}%
\pgfpathrectangle{\pgfqpoint{0.640323in}{0.527436in}}{\pgfqpoint{9.687500in}{3.850000in}}%
\pgfusepath{clip}%
\pgfsetrectcap%
\pgfsetroundjoin%
\pgfsetlinewidth{0.803000pt}%
\definecolor{currentstroke}{rgb}{0.600000,0.600000,0.600000}%
\pgfsetstrokecolor{currentstroke}%
\pgfsetstrokeopacity{0.200000}%
\pgfsetdash{}{0pt}%
\pgfpathmoveto{\pgfqpoint{0.640323in}{1.769371in}}%
\pgfpathlineto{\pgfqpoint{10.327822in}{1.769371in}}%
\pgfusepath{stroke}%
\end{pgfscope}%
\begin{pgfscope}%
\pgfsetbuttcap%
\pgfsetroundjoin%
\definecolor{currentfill}{rgb}{0.000000,0.000000,0.000000}%
\pgfsetfillcolor{currentfill}%
\pgfsetlinewidth{0.602250pt}%
\definecolor{currentstroke}{rgb}{0.000000,0.000000,0.000000}%
\pgfsetstrokecolor{currentstroke}%
\pgfsetdash{}{0pt}%
\pgfsys@defobject{currentmarker}{\pgfqpoint{-0.027778in}{0.000000in}}{\pgfqpoint{-0.000000in}{0.000000in}}{%
\pgfpathmoveto{\pgfqpoint{-0.000000in}{0.000000in}}%
\pgfpathlineto{\pgfqpoint{-0.027778in}{0.000000in}}%
\pgfusepath{stroke,fill}%
}%
\begin{pgfscope}%
\pgfsys@transformshift{0.640323in}{1.769371in}%
\pgfsys@useobject{currentmarker}{}%
\end{pgfscope}%
\end{pgfscope}%
\begin{pgfscope}%
\pgfpathrectangle{\pgfqpoint{0.640323in}{0.527436in}}{\pgfqpoint{9.687500in}{3.850000in}}%
\pgfusepath{clip}%
\pgfsetrectcap%
\pgfsetroundjoin%
\pgfsetlinewidth{0.803000pt}%
\definecolor{currentstroke}{rgb}{0.600000,0.600000,0.600000}%
\pgfsetstrokecolor{currentstroke}%
\pgfsetstrokeopacity{0.200000}%
\pgfsetdash{}{0pt}%
\pgfpathmoveto{\pgfqpoint{0.640323in}{2.017758in}}%
\pgfpathlineto{\pgfqpoint{10.327822in}{2.017758in}}%
\pgfusepath{stroke}%
\end{pgfscope}%
\begin{pgfscope}%
\pgfsetbuttcap%
\pgfsetroundjoin%
\definecolor{currentfill}{rgb}{0.000000,0.000000,0.000000}%
\pgfsetfillcolor{currentfill}%
\pgfsetlinewidth{0.602250pt}%
\definecolor{currentstroke}{rgb}{0.000000,0.000000,0.000000}%
\pgfsetstrokecolor{currentstroke}%
\pgfsetdash{}{0pt}%
\pgfsys@defobject{currentmarker}{\pgfqpoint{-0.027778in}{0.000000in}}{\pgfqpoint{-0.000000in}{0.000000in}}{%
\pgfpathmoveto{\pgfqpoint{-0.000000in}{0.000000in}}%
\pgfpathlineto{\pgfqpoint{-0.027778in}{0.000000in}}%
\pgfusepath{stroke,fill}%
}%
\begin{pgfscope}%
\pgfsys@transformshift{0.640323in}{2.017758in}%
\pgfsys@useobject{currentmarker}{}%
\end{pgfscope}%
\end{pgfscope}%
\begin{pgfscope}%
\pgfpathrectangle{\pgfqpoint{0.640323in}{0.527436in}}{\pgfqpoint{9.687500in}{3.850000in}}%
\pgfusepath{clip}%
\pgfsetrectcap%
\pgfsetroundjoin%
\pgfsetlinewidth{0.803000pt}%
\definecolor{currentstroke}{rgb}{0.600000,0.600000,0.600000}%
\pgfsetstrokecolor{currentstroke}%
\pgfsetstrokeopacity{0.200000}%
\pgfsetdash{}{0pt}%
\pgfpathmoveto{\pgfqpoint{0.640323in}{2.141952in}}%
\pgfpathlineto{\pgfqpoint{10.327822in}{2.141952in}}%
\pgfusepath{stroke}%
\end{pgfscope}%
\begin{pgfscope}%
\pgfsetbuttcap%
\pgfsetroundjoin%
\definecolor{currentfill}{rgb}{0.000000,0.000000,0.000000}%
\pgfsetfillcolor{currentfill}%
\pgfsetlinewidth{0.602250pt}%
\definecolor{currentstroke}{rgb}{0.000000,0.000000,0.000000}%
\pgfsetstrokecolor{currentstroke}%
\pgfsetdash{}{0pt}%
\pgfsys@defobject{currentmarker}{\pgfqpoint{-0.027778in}{0.000000in}}{\pgfqpoint{-0.000000in}{0.000000in}}{%
\pgfpathmoveto{\pgfqpoint{-0.000000in}{0.000000in}}%
\pgfpathlineto{\pgfqpoint{-0.027778in}{0.000000in}}%
\pgfusepath{stroke,fill}%
}%
\begin{pgfscope}%
\pgfsys@transformshift{0.640323in}{2.141952in}%
\pgfsys@useobject{currentmarker}{}%
\end{pgfscope}%
\end{pgfscope}%
\begin{pgfscope}%
\pgfpathrectangle{\pgfqpoint{0.640323in}{0.527436in}}{\pgfqpoint{9.687500in}{3.850000in}}%
\pgfusepath{clip}%
\pgfsetrectcap%
\pgfsetroundjoin%
\pgfsetlinewidth{0.803000pt}%
\definecolor{currentstroke}{rgb}{0.600000,0.600000,0.600000}%
\pgfsetstrokecolor{currentstroke}%
\pgfsetstrokeopacity{0.200000}%
\pgfsetdash{}{0pt}%
\pgfpathmoveto{\pgfqpoint{0.640323in}{2.266146in}}%
\pgfpathlineto{\pgfqpoint{10.327822in}{2.266146in}}%
\pgfusepath{stroke}%
\end{pgfscope}%
\begin{pgfscope}%
\pgfsetbuttcap%
\pgfsetroundjoin%
\definecolor{currentfill}{rgb}{0.000000,0.000000,0.000000}%
\pgfsetfillcolor{currentfill}%
\pgfsetlinewidth{0.602250pt}%
\definecolor{currentstroke}{rgb}{0.000000,0.000000,0.000000}%
\pgfsetstrokecolor{currentstroke}%
\pgfsetdash{}{0pt}%
\pgfsys@defobject{currentmarker}{\pgfqpoint{-0.027778in}{0.000000in}}{\pgfqpoint{-0.000000in}{0.000000in}}{%
\pgfpathmoveto{\pgfqpoint{-0.000000in}{0.000000in}}%
\pgfpathlineto{\pgfqpoint{-0.027778in}{0.000000in}}%
\pgfusepath{stroke,fill}%
}%
\begin{pgfscope}%
\pgfsys@transformshift{0.640323in}{2.266146in}%
\pgfsys@useobject{currentmarker}{}%
\end{pgfscope}%
\end{pgfscope}%
\begin{pgfscope}%
\pgfpathrectangle{\pgfqpoint{0.640323in}{0.527436in}}{\pgfqpoint{9.687500in}{3.850000in}}%
\pgfusepath{clip}%
\pgfsetrectcap%
\pgfsetroundjoin%
\pgfsetlinewidth{0.803000pt}%
\definecolor{currentstroke}{rgb}{0.600000,0.600000,0.600000}%
\pgfsetstrokecolor{currentstroke}%
\pgfsetstrokeopacity{0.200000}%
\pgfsetdash{}{0pt}%
\pgfpathmoveto{\pgfqpoint{0.640323in}{2.390339in}}%
\pgfpathlineto{\pgfqpoint{10.327822in}{2.390339in}}%
\pgfusepath{stroke}%
\end{pgfscope}%
\begin{pgfscope}%
\pgfsetbuttcap%
\pgfsetroundjoin%
\definecolor{currentfill}{rgb}{0.000000,0.000000,0.000000}%
\pgfsetfillcolor{currentfill}%
\pgfsetlinewidth{0.602250pt}%
\definecolor{currentstroke}{rgb}{0.000000,0.000000,0.000000}%
\pgfsetstrokecolor{currentstroke}%
\pgfsetdash{}{0pt}%
\pgfsys@defobject{currentmarker}{\pgfqpoint{-0.027778in}{0.000000in}}{\pgfqpoint{-0.000000in}{0.000000in}}{%
\pgfpathmoveto{\pgfqpoint{-0.000000in}{0.000000in}}%
\pgfpathlineto{\pgfqpoint{-0.027778in}{0.000000in}}%
\pgfusepath{stroke,fill}%
}%
\begin{pgfscope}%
\pgfsys@transformshift{0.640323in}{2.390339in}%
\pgfsys@useobject{currentmarker}{}%
\end{pgfscope}%
\end{pgfscope}%
\begin{pgfscope}%
\pgfpathrectangle{\pgfqpoint{0.640323in}{0.527436in}}{\pgfqpoint{9.687500in}{3.850000in}}%
\pgfusepath{clip}%
\pgfsetrectcap%
\pgfsetroundjoin%
\pgfsetlinewidth{0.803000pt}%
\definecolor{currentstroke}{rgb}{0.600000,0.600000,0.600000}%
\pgfsetstrokecolor{currentstroke}%
\pgfsetstrokeopacity{0.200000}%
\pgfsetdash{}{0pt}%
\pgfpathmoveto{\pgfqpoint{0.640323in}{2.638726in}}%
\pgfpathlineto{\pgfqpoint{10.327822in}{2.638726in}}%
\pgfusepath{stroke}%
\end{pgfscope}%
\begin{pgfscope}%
\pgfsetbuttcap%
\pgfsetroundjoin%
\definecolor{currentfill}{rgb}{0.000000,0.000000,0.000000}%
\pgfsetfillcolor{currentfill}%
\pgfsetlinewidth{0.602250pt}%
\definecolor{currentstroke}{rgb}{0.000000,0.000000,0.000000}%
\pgfsetstrokecolor{currentstroke}%
\pgfsetdash{}{0pt}%
\pgfsys@defobject{currentmarker}{\pgfqpoint{-0.027778in}{0.000000in}}{\pgfqpoint{-0.000000in}{0.000000in}}{%
\pgfpathmoveto{\pgfqpoint{-0.000000in}{0.000000in}}%
\pgfpathlineto{\pgfqpoint{-0.027778in}{0.000000in}}%
\pgfusepath{stroke,fill}%
}%
\begin{pgfscope}%
\pgfsys@transformshift{0.640323in}{2.638726in}%
\pgfsys@useobject{currentmarker}{}%
\end{pgfscope}%
\end{pgfscope}%
\begin{pgfscope}%
\pgfpathrectangle{\pgfqpoint{0.640323in}{0.527436in}}{\pgfqpoint{9.687500in}{3.850000in}}%
\pgfusepath{clip}%
\pgfsetrectcap%
\pgfsetroundjoin%
\pgfsetlinewidth{0.803000pt}%
\definecolor{currentstroke}{rgb}{0.600000,0.600000,0.600000}%
\pgfsetstrokecolor{currentstroke}%
\pgfsetstrokeopacity{0.200000}%
\pgfsetdash{}{0pt}%
\pgfpathmoveto{\pgfqpoint{0.640323in}{2.762920in}}%
\pgfpathlineto{\pgfqpoint{10.327822in}{2.762920in}}%
\pgfusepath{stroke}%
\end{pgfscope}%
\begin{pgfscope}%
\pgfsetbuttcap%
\pgfsetroundjoin%
\definecolor{currentfill}{rgb}{0.000000,0.000000,0.000000}%
\pgfsetfillcolor{currentfill}%
\pgfsetlinewidth{0.602250pt}%
\definecolor{currentstroke}{rgb}{0.000000,0.000000,0.000000}%
\pgfsetstrokecolor{currentstroke}%
\pgfsetdash{}{0pt}%
\pgfsys@defobject{currentmarker}{\pgfqpoint{-0.027778in}{0.000000in}}{\pgfqpoint{-0.000000in}{0.000000in}}{%
\pgfpathmoveto{\pgfqpoint{-0.000000in}{0.000000in}}%
\pgfpathlineto{\pgfqpoint{-0.027778in}{0.000000in}}%
\pgfusepath{stroke,fill}%
}%
\begin{pgfscope}%
\pgfsys@transformshift{0.640323in}{2.762920in}%
\pgfsys@useobject{currentmarker}{}%
\end{pgfscope}%
\end{pgfscope}%
\begin{pgfscope}%
\pgfpathrectangle{\pgfqpoint{0.640323in}{0.527436in}}{\pgfqpoint{9.687500in}{3.850000in}}%
\pgfusepath{clip}%
\pgfsetrectcap%
\pgfsetroundjoin%
\pgfsetlinewidth{0.803000pt}%
\definecolor{currentstroke}{rgb}{0.600000,0.600000,0.600000}%
\pgfsetstrokecolor{currentstroke}%
\pgfsetstrokeopacity{0.200000}%
\pgfsetdash{}{0pt}%
\pgfpathmoveto{\pgfqpoint{0.640323in}{2.887113in}}%
\pgfpathlineto{\pgfqpoint{10.327822in}{2.887113in}}%
\pgfusepath{stroke}%
\end{pgfscope}%
\begin{pgfscope}%
\pgfsetbuttcap%
\pgfsetroundjoin%
\definecolor{currentfill}{rgb}{0.000000,0.000000,0.000000}%
\pgfsetfillcolor{currentfill}%
\pgfsetlinewidth{0.602250pt}%
\definecolor{currentstroke}{rgb}{0.000000,0.000000,0.000000}%
\pgfsetstrokecolor{currentstroke}%
\pgfsetdash{}{0pt}%
\pgfsys@defobject{currentmarker}{\pgfqpoint{-0.027778in}{0.000000in}}{\pgfqpoint{-0.000000in}{0.000000in}}{%
\pgfpathmoveto{\pgfqpoint{-0.000000in}{0.000000in}}%
\pgfpathlineto{\pgfqpoint{-0.027778in}{0.000000in}}%
\pgfusepath{stroke,fill}%
}%
\begin{pgfscope}%
\pgfsys@transformshift{0.640323in}{2.887113in}%
\pgfsys@useobject{currentmarker}{}%
\end{pgfscope}%
\end{pgfscope}%
\begin{pgfscope}%
\pgfpathrectangle{\pgfqpoint{0.640323in}{0.527436in}}{\pgfqpoint{9.687500in}{3.850000in}}%
\pgfusepath{clip}%
\pgfsetrectcap%
\pgfsetroundjoin%
\pgfsetlinewidth{0.803000pt}%
\definecolor{currentstroke}{rgb}{0.600000,0.600000,0.600000}%
\pgfsetstrokecolor{currentstroke}%
\pgfsetstrokeopacity{0.200000}%
\pgfsetdash{}{0pt}%
\pgfpathmoveto{\pgfqpoint{0.640323in}{3.011307in}}%
\pgfpathlineto{\pgfqpoint{10.327822in}{3.011307in}}%
\pgfusepath{stroke}%
\end{pgfscope}%
\begin{pgfscope}%
\pgfsetbuttcap%
\pgfsetroundjoin%
\definecolor{currentfill}{rgb}{0.000000,0.000000,0.000000}%
\pgfsetfillcolor{currentfill}%
\pgfsetlinewidth{0.602250pt}%
\definecolor{currentstroke}{rgb}{0.000000,0.000000,0.000000}%
\pgfsetstrokecolor{currentstroke}%
\pgfsetdash{}{0pt}%
\pgfsys@defobject{currentmarker}{\pgfqpoint{-0.027778in}{0.000000in}}{\pgfqpoint{-0.000000in}{0.000000in}}{%
\pgfpathmoveto{\pgfqpoint{-0.000000in}{0.000000in}}%
\pgfpathlineto{\pgfqpoint{-0.027778in}{0.000000in}}%
\pgfusepath{stroke,fill}%
}%
\begin{pgfscope}%
\pgfsys@transformshift{0.640323in}{3.011307in}%
\pgfsys@useobject{currentmarker}{}%
\end{pgfscope}%
\end{pgfscope}%
\begin{pgfscope}%
\pgfpathrectangle{\pgfqpoint{0.640323in}{0.527436in}}{\pgfqpoint{9.687500in}{3.850000in}}%
\pgfusepath{clip}%
\pgfsetrectcap%
\pgfsetroundjoin%
\pgfsetlinewidth{0.803000pt}%
\definecolor{currentstroke}{rgb}{0.600000,0.600000,0.600000}%
\pgfsetstrokecolor{currentstroke}%
\pgfsetstrokeopacity{0.200000}%
\pgfsetdash{}{0pt}%
\pgfpathmoveto{\pgfqpoint{0.640323in}{3.259694in}}%
\pgfpathlineto{\pgfqpoint{10.327822in}{3.259694in}}%
\pgfusepath{stroke}%
\end{pgfscope}%
\begin{pgfscope}%
\pgfsetbuttcap%
\pgfsetroundjoin%
\definecolor{currentfill}{rgb}{0.000000,0.000000,0.000000}%
\pgfsetfillcolor{currentfill}%
\pgfsetlinewidth{0.602250pt}%
\definecolor{currentstroke}{rgb}{0.000000,0.000000,0.000000}%
\pgfsetstrokecolor{currentstroke}%
\pgfsetdash{}{0pt}%
\pgfsys@defobject{currentmarker}{\pgfqpoint{-0.027778in}{0.000000in}}{\pgfqpoint{-0.000000in}{0.000000in}}{%
\pgfpathmoveto{\pgfqpoint{-0.000000in}{0.000000in}}%
\pgfpathlineto{\pgfqpoint{-0.027778in}{0.000000in}}%
\pgfusepath{stroke,fill}%
}%
\begin{pgfscope}%
\pgfsys@transformshift{0.640323in}{3.259694in}%
\pgfsys@useobject{currentmarker}{}%
\end{pgfscope}%
\end{pgfscope}%
\begin{pgfscope}%
\pgfpathrectangle{\pgfqpoint{0.640323in}{0.527436in}}{\pgfqpoint{9.687500in}{3.850000in}}%
\pgfusepath{clip}%
\pgfsetrectcap%
\pgfsetroundjoin%
\pgfsetlinewidth{0.803000pt}%
\definecolor{currentstroke}{rgb}{0.600000,0.600000,0.600000}%
\pgfsetstrokecolor{currentstroke}%
\pgfsetstrokeopacity{0.200000}%
\pgfsetdash{}{0pt}%
\pgfpathmoveto{\pgfqpoint{0.640323in}{3.383887in}}%
\pgfpathlineto{\pgfqpoint{10.327822in}{3.383887in}}%
\pgfusepath{stroke}%
\end{pgfscope}%
\begin{pgfscope}%
\pgfsetbuttcap%
\pgfsetroundjoin%
\definecolor{currentfill}{rgb}{0.000000,0.000000,0.000000}%
\pgfsetfillcolor{currentfill}%
\pgfsetlinewidth{0.602250pt}%
\definecolor{currentstroke}{rgb}{0.000000,0.000000,0.000000}%
\pgfsetstrokecolor{currentstroke}%
\pgfsetdash{}{0pt}%
\pgfsys@defobject{currentmarker}{\pgfqpoint{-0.027778in}{0.000000in}}{\pgfqpoint{-0.000000in}{0.000000in}}{%
\pgfpathmoveto{\pgfqpoint{-0.000000in}{0.000000in}}%
\pgfpathlineto{\pgfqpoint{-0.027778in}{0.000000in}}%
\pgfusepath{stroke,fill}%
}%
\begin{pgfscope}%
\pgfsys@transformshift{0.640323in}{3.383887in}%
\pgfsys@useobject{currentmarker}{}%
\end{pgfscope}%
\end{pgfscope}%
\begin{pgfscope}%
\pgfpathrectangle{\pgfqpoint{0.640323in}{0.527436in}}{\pgfqpoint{9.687500in}{3.850000in}}%
\pgfusepath{clip}%
\pgfsetrectcap%
\pgfsetroundjoin%
\pgfsetlinewidth{0.803000pt}%
\definecolor{currentstroke}{rgb}{0.600000,0.600000,0.600000}%
\pgfsetstrokecolor{currentstroke}%
\pgfsetstrokeopacity{0.200000}%
\pgfsetdash{}{0pt}%
\pgfpathmoveto{\pgfqpoint{0.640323in}{3.508081in}}%
\pgfpathlineto{\pgfqpoint{10.327822in}{3.508081in}}%
\pgfusepath{stroke}%
\end{pgfscope}%
\begin{pgfscope}%
\pgfsetbuttcap%
\pgfsetroundjoin%
\definecolor{currentfill}{rgb}{0.000000,0.000000,0.000000}%
\pgfsetfillcolor{currentfill}%
\pgfsetlinewidth{0.602250pt}%
\definecolor{currentstroke}{rgb}{0.000000,0.000000,0.000000}%
\pgfsetstrokecolor{currentstroke}%
\pgfsetdash{}{0pt}%
\pgfsys@defobject{currentmarker}{\pgfqpoint{-0.027778in}{0.000000in}}{\pgfqpoint{-0.000000in}{0.000000in}}{%
\pgfpathmoveto{\pgfqpoint{-0.000000in}{0.000000in}}%
\pgfpathlineto{\pgfqpoint{-0.027778in}{0.000000in}}%
\pgfusepath{stroke,fill}%
}%
\begin{pgfscope}%
\pgfsys@transformshift{0.640323in}{3.508081in}%
\pgfsys@useobject{currentmarker}{}%
\end{pgfscope}%
\end{pgfscope}%
\begin{pgfscope}%
\pgfpathrectangle{\pgfqpoint{0.640323in}{0.527436in}}{\pgfqpoint{9.687500in}{3.850000in}}%
\pgfusepath{clip}%
\pgfsetrectcap%
\pgfsetroundjoin%
\pgfsetlinewidth{0.803000pt}%
\definecolor{currentstroke}{rgb}{0.600000,0.600000,0.600000}%
\pgfsetstrokecolor{currentstroke}%
\pgfsetstrokeopacity{0.200000}%
\pgfsetdash{}{0pt}%
\pgfpathmoveto{\pgfqpoint{0.640323in}{3.632275in}}%
\pgfpathlineto{\pgfqpoint{10.327822in}{3.632275in}}%
\pgfusepath{stroke}%
\end{pgfscope}%
\begin{pgfscope}%
\pgfsetbuttcap%
\pgfsetroundjoin%
\definecolor{currentfill}{rgb}{0.000000,0.000000,0.000000}%
\pgfsetfillcolor{currentfill}%
\pgfsetlinewidth{0.602250pt}%
\definecolor{currentstroke}{rgb}{0.000000,0.000000,0.000000}%
\pgfsetstrokecolor{currentstroke}%
\pgfsetdash{}{0pt}%
\pgfsys@defobject{currentmarker}{\pgfqpoint{-0.027778in}{0.000000in}}{\pgfqpoint{-0.000000in}{0.000000in}}{%
\pgfpathmoveto{\pgfqpoint{-0.000000in}{0.000000in}}%
\pgfpathlineto{\pgfqpoint{-0.027778in}{0.000000in}}%
\pgfusepath{stroke,fill}%
}%
\begin{pgfscope}%
\pgfsys@transformshift{0.640323in}{3.632275in}%
\pgfsys@useobject{currentmarker}{}%
\end{pgfscope}%
\end{pgfscope}%
\begin{pgfscope}%
\pgfpathrectangle{\pgfqpoint{0.640323in}{0.527436in}}{\pgfqpoint{9.687500in}{3.850000in}}%
\pgfusepath{clip}%
\pgfsetrectcap%
\pgfsetroundjoin%
\pgfsetlinewidth{0.803000pt}%
\definecolor{currentstroke}{rgb}{0.600000,0.600000,0.600000}%
\pgfsetstrokecolor{currentstroke}%
\pgfsetstrokeopacity{0.200000}%
\pgfsetdash{}{0pt}%
\pgfpathmoveto{\pgfqpoint{0.640323in}{3.880662in}}%
\pgfpathlineto{\pgfqpoint{10.327822in}{3.880662in}}%
\pgfusepath{stroke}%
\end{pgfscope}%
\begin{pgfscope}%
\pgfsetbuttcap%
\pgfsetroundjoin%
\definecolor{currentfill}{rgb}{0.000000,0.000000,0.000000}%
\pgfsetfillcolor{currentfill}%
\pgfsetlinewidth{0.602250pt}%
\definecolor{currentstroke}{rgb}{0.000000,0.000000,0.000000}%
\pgfsetstrokecolor{currentstroke}%
\pgfsetdash{}{0pt}%
\pgfsys@defobject{currentmarker}{\pgfqpoint{-0.027778in}{0.000000in}}{\pgfqpoint{-0.000000in}{0.000000in}}{%
\pgfpathmoveto{\pgfqpoint{-0.000000in}{0.000000in}}%
\pgfpathlineto{\pgfqpoint{-0.027778in}{0.000000in}}%
\pgfusepath{stroke,fill}%
}%
\begin{pgfscope}%
\pgfsys@transformshift{0.640323in}{3.880662in}%
\pgfsys@useobject{currentmarker}{}%
\end{pgfscope}%
\end{pgfscope}%
\begin{pgfscope}%
\pgfpathrectangle{\pgfqpoint{0.640323in}{0.527436in}}{\pgfqpoint{9.687500in}{3.850000in}}%
\pgfusepath{clip}%
\pgfsetrectcap%
\pgfsetroundjoin%
\pgfsetlinewidth{0.803000pt}%
\definecolor{currentstroke}{rgb}{0.600000,0.600000,0.600000}%
\pgfsetstrokecolor{currentstroke}%
\pgfsetstrokeopacity{0.200000}%
\pgfsetdash{}{0pt}%
\pgfpathmoveto{\pgfqpoint{0.640323in}{4.004855in}}%
\pgfpathlineto{\pgfqpoint{10.327822in}{4.004855in}}%
\pgfusepath{stroke}%
\end{pgfscope}%
\begin{pgfscope}%
\pgfsetbuttcap%
\pgfsetroundjoin%
\definecolor{currentfill}{rgb}{0.000000,0.000000,0.000000}%
\pgfsetfillcolor{currentfill}%
\pgfsetlinewidth{0.602250pt}%
\definecolor{currentstroke}{rgb}{0.000000,0.000000,0.000000}%
\pgfsetstrokecolor{currentstroke}%
\pgfsetdash{}{0pt}%
\pgfsys@defobject{currentmarker}{\pgfqpoint{-0.027778in}{0.000000in}}{\pgfqpoint{-0.000000in}{0.000000in}}{%
\pgfpathmoveto{\pgfqpoint{-0.000000in}{0.000000in}}%
\pgfpathlineto{\pgfqpoint{-0.027778in}{0.000000in}}%
\pgfusepath{stroke,fill}%
}%
\begin{pgfscope}%
\pgfsys@transformshift{0.640323in}{4.004855in}%
\pgfsys@useobject{currentmarker}{}%
\end{pgfscope}%
\end{pgfscope}%
\begin{pgfscope}%
\pgfpathrectangle{\pgfqpoint{0.640323in}{0.527436in}}{\pgfqpoint{9.687500in}{3.850000in}}%
\pgfusepath{clip}%
\pgfsetrectcap%
\pgfsetroundjoin%
\pgfsetlinewidth{0.803000pt}%
\definecolor{currentstroke}{rgb}{0.600000,0.600000,0.600000}%
\pgfsetstrokecolor{currentstroke}%
\pgfsetstrokeopacity{0.200000}%
\pgfsetdash{}{0pt}%
\pgfpathmoveto{\pgfqpoint{0.640323in}{4.129049in}}%
\pgfpathlineto{\pgfqpoint{10.327822in}{4.129049in}}%
\pgfusepath{stroke}%
\end{pgfscope}%
\begin{pgfscope}%
\pgfsetbuttcap%
\pgfsetroundjoin%
\definecolor{currentfill}{rgb}{0.000000,0.000000,0.000000}%
\pgfsetfillcolor{currentfill}%
\pgfsetlinewidth{0.602250pt}%
\definecolor{currentstroke}{rgb}{0.000000,0.000000,0.000000}%
\pgfsetstrokecolor{currentstroke}%
\pgfsetdash{}{0pt}%
\pgfsys@defobject{currentmarker}{\pgfqpoint{-0.027778in}{0.000000in}}{\pgfqpoint{-0.000000in}{0.000000in}}{%
\pgfpathmoveto{\pgfqpoint{-0.000000in}{0.000000in}}%
\pgfpathlineto{\pgfqpoint{-0.027778in}{0.000000in}}%
\pgfusepath{stroke,fill}%
}%
\begin{pgfscope}%
\pgfsys@transformshift{0.640323in}{4.129049in}%
\pgfsys@useobject{currentmarker}{}%
\end{pgfscope}%
\end{pgfscope}%
\begin{pgfscope}%
\pgfpathrectangle{\pgfqpoint{0.640323in}{0.527436in}}{\pgfqpoint{9.687500in}{3.850000in}}%
\pgfusepath{clip}%
\pgfsetrectcap%
\pgfsetroundjoin%
\pgfsetlinewidth{0.803000pt}%
\definecolor{currentstroke}{rgb}{0.600000,0.600000,0.600000}%
\pgfsetstrokecolor{currentstroke}%
\pgfsetstrokeopacity{0.200000}%
\pgfsetdash{}{0pt}%
\pgfpathmoveto{\pgfqpoint{0.640323in}{4.253242in}}%
\pgfpathlineto{\pgfqpoint{10.327822in}{4.253242in}}%
\pgfusepath{stroke}%
\end{pgfscope}%
\begin{pgfscope}%
\pgfsetbuttcap%
\pgfsetroundjoin%
\definecolor{currentfill}{rgb}{0.000000,0.000000,0.000000}%
\pgfsetfillcolor{currentfill}%
\pgfsetlinewidth{0.602250pt}%
\definecolor{currentstroke}{rgb}{0.000000,0.000000,0.000000}%
\pgfsetstrokecolor{currentstroke}%
\pgfsetdash{}{0pt}%
\pgfsys@defobject{currentmarker}{\pgfqpoint{-0.027778in}{0.000000in}}{\pgfqpoint{-0.000000in}{0.000000in}}{%
\pgfpathmoveto{\pgfqpoint{-0.000000in}{0.000000in}}%
\pgfpathlineto{\pgfqpoint{-0.027778in}{0.000000in}}%
\pgfusepath{stroke,fill}%
}%
\begin{pgfscope}%
\pgfsys@transformshift{0.640323in}{4.253242in}%
\pgfsys@useobject{currentmarker}{}%
\end{pgfscope}%
\end{pgfscope}%
\begin{pgfscope}%
\definecolor{textcolor}{rgb}{0.000000,0.000000,0.000000}%
\pgfsetstrokecolor{textcolor}%
\pgfsetfillcolor{textcolor}%
\pgftext[x=0.266665in,y=2.452436in,,bottom,rotate=90.000000]{\color{textcolor}\sffamily\fontsize{10.000000}{12.000000}\selectfont avg. infection rate \(\displaystyle \overline{\langle I\rangle}\)}%
\end{pgfscope}%
\begin{pgfscope}%
\pgfpathrectangle{\pgfqpoint{0.640323in}{0.527436in}}{\pgfqpoint{9.687500in}{3.850000in}}%
\pgfusepath{clip}%
\pgfsetbuttcap%
\pgfsetroundjoin%
\pgfsetlinewidth{1.003750pt}%
\definecolor{currentstroke}{rgb}{0.000000,0.000000,1.000000}%
\pgfsetstrokecolor{currentstroke}%
\pgfsetstrokeopacity{0.500000}%
\pgfsetdash{{3.700000pt}{1.600000pt}}{0.000000pt}%
\pgfpathmoveto{\pgfqpoint{1.080663in}{0.659222in}}%
\pgfpathlineto{\pgfqpoint{1.260394in}{0.661817in}}%
\pgfpathlineto{\pgfqpoint{1.440125in}{0.663952in}}%
\pgfpathlineto{\pgfqpoint{1.619856in}{0.663127in}}%
\pgfpathlineto{\pgfqpoint{1.799587in}{0.661987in}}%
\pgfpathlineto{\pgfqpoint{1.979318in}{0.670089in}}%
\pgfpathlineto{\pgfqpoint{2.159049in}{0.682993in}}%
\pgfpathlineto{\pgfqpoint{2.338780in}{0.685467in}}%
\pgfpathlineto{\pgfqpoint{2.518511in}{0.688014in}}%
\pgfpathlineto{\pgfqpoint{2.698242in}{0.696358in}}%
\pgfpathlineto{\pgfqpoint{2.877973in}{1.634093in}}%
\pgfpathlineto{\pgfqpoint{3.057704in}{1.884660in}}%
\pgfpathlineto{\pgfqpoint{3.237435in}{2.119684in}}%
\pgfpathlineto{\pgfqpoint{3.417166in}{2.331589in}}%
\pgfpathlineto{\pgfqpoint{3.596897in}{2.506404in}}%
\pgfpathlineto{\pgfqpoint{3.776628in}{2.606318in}}%
\pgfpathlineto{\pgfqpoint{3.956359in}{2.789677in}}%
\pgfpathlineto{\pgfqpoint{4.136090in}{2.829531in}}%
\pgfpathlineto{\pgfqpoint{4.315821in}{2.973906in}}%
\pgfpathlineto{\pgfqpoint{4.495552in}{3.032165in}}%
\pgfpathlineto{\pgfqpoint{4.675283in}{3.123541in}}%
\pgfpathlineto{\pgfqpoint{4.855014in}{3.199659in}}%
\pgfpathlineto{\pgfqpoint{5.034745in}{3.232868in}}%
\pgfpathlineto{\pgfqpoint{5.214476in}{3.299765in}}%
\pgfpathlineto{\pgfqpoint{5.394207in}{3.359750in}}%
\pgfpathlineto{\pgfqpoint{5.573938in}{3.406056in}}%
\pgfpathlineto{\pgfqpoint{5.753669in}{3.426262in}}%
\pgfpathlineto{\pgfqpoint{5.933400in}{3.504659in}}%
\pgfpathlineto{\pgfqpoint{6.113131in}{3.554945in}}%
\pgfpathlineto{\pgfqpoint{6.292862in}{3.574618in}}%
\pgfpathlineto{\pgfqpoint{6.472593in}{3.602828in}}%
\pgfpathlineto{\pgfqpoint{6.652324in}{3.653133in}}%
\pgfpathlineto{\pgfqpoint{6.832055in}{3.687118in}}%
\pgfpathlineto{\pgfqpoint{7.011786in}{3.694371in}}%
\pgfpathlineto{\pgfqpoint{7.191517in}{3.736821in}}%
\pgfpathlineto{\pgfqpoint{7.371248in}{3.756468in}}%
\pgfpathlineto{\pgfqpoint{7.550979in}{3.788367in}}%
\pgfpathlineto{\pgfqpoint{7.730710in}{3.808064in}}%
\pgfpathlineto{\pgfqpoint{7.910441in}{3.838380in}}%
\pgfpathlineto{\pgfqpoint{8.090172in}{3.855873in}}%
\pgfpathlineto{\pgfqpoint{8.269903in}{3.859095in}}%
\pgfpathlineto{\pgfqpoint{8.449634in}{3.893541in}}%
\pgfpathlineto{\pgfqpoint{8.629365in}{3.910474in}}%
\pgfpathlineto{\pgfqpoint{8.809096in}{3.921049in}}%
\pgfpathlineto{\pgfqpoint{8.988827in}{3.950229in}}%
\pgfpathlineto{\pgfqpoint{9.168558in}{3.959860in}}%
\pgfpathlineto{\pgfqpoint{9.348289in}{3.972646in}}%
\pgfpathlineto{\pgfqpoint{9.528020in}{3.993380in}}%
\pgfpathlineto{\pgfqpoint{9.707751in}{4.026030in}}%
\pgfpathlineto{\pgfqpoint{9.887482in}{4.022416in}}%
\pgfusepath{stroke}%
\end{pgfscope}%
\begin{pgfscope}%
\pgfpathrectangle{\pgfqpoint{0.640323in}{0.527436in}}{\pgfqpoint{9.687500in}{3.850000in}}%
\pgfusepath{clip}%
\pgfsetbuttcap%
\pgfsetroundjoin%
\pgfsetlinewidth{1.003750pt}%
\definecolor{currentstroke}{rgb}{0.980392,0.164706,0.333333}%
\pgfsetstrokecolor{currentstroke}%
\pgfsetstrokeopacity{0.500000}%
\pgfsetdash{{3.700000pt}{1.600000pt}}{0.000000pt}%
\pgfpathmoveto{\pgfqpoint{1.080663in}{0.659834in}}%
\pgfpathlineto{\pgfqpoint{1.260394in}{0.660901in}}%
\pgfpathlineto{\pgfqpoint{1.440125in}{0.661696in}}%
\pgfpathlineto{\pgfqpoint{1.619856in}{0.664352in}}%
\pgfpathlineto{\pgfqpoint{1.799587in}{0.665437in}}%
\pgfpathlineto{\pgfqpoint{1.979318in}{0.674358in}}%
\pgfpathlineto{\pgfqpoint{2.159049in}{0.677256in}}%
\pgfpathlineto{\pgfqpoint{2.338780in}{0.715364in}}%
\pgfpathlineto{\pgfqpoint{2.518511in}{0.753295in}}%
\pgfpathlineto{\pgfqpoint{2.698242in}{1.197292in}}%
\pgfpathlineto{\pgfqpoint{2.877973in}{1.702804in}}%
\pgfpathlineto{\pgfqpoint{3.057704in}{2.074397in}}%
\pgfpathlineto{\pgfqpoint{3.237435in}{2.254459in}}%
\pgfpathlineto{\pgfqpoint{3.417166in}{2.431416in}}%
\pgfpathlineto{\pgfqpoint{3.596897in}{2.568973in}}%
\pgfpathlineto{\pgfqpoint{3.776628in}{2.719744in}}%
\pgfpathlineto{\pgfqpoint{3.956359in}{2.842565in}}%
\pgfpathlineto{\pgfqpoint{4.136090in}{2.925054in}}%
\pgfpathlineto{\pgfqpoint{4.315821in}{3.026800in}}%
\pgfpathlineto{\pgfqpoint{4.495552in}{3.104346in}}%
\pgfpathlineto{\pgfqpoint{4.675283in}{3.190003in}}%
\pgfpathlineto{\pgfqpoint{4.855014in}{3.227410in}}%
\pgfpathlineto{\pgfqpoint{5.034745in}{3.282067in}}%
\pgfpathlineto{\pgfqpoint{5.214476in}{3.352802in}}%
\pgfpathlineto{\pgfqpoint{5.394207in}{3.396859in}}%
\pgfpathlineto{\pgfqpoint{5.573938in}{3.438439in}}%
\pgfpathlineto{\pgfqpoint{5.753669in}{3.483894in}}%
\pgfpathlineto{\pgfqpoint{5.933400in}{3.524742in}}%
\pgfpathlineto{\pgfqpoint{6.113131in}{3.560211in}}%
\pgfpathlineto{\pgfqpoint{6.292862in}{3.603170in}}%
\pgfpathlineto{\pgfqpoint{6.472593in}{3.639447in}}%
\pgfpathlineto{\pgfqpoint{6.652324in}{3.656505in}}%
\pgfpathlineto{\pgfqpoint{6.832055in}{3.687621in}}%
\pgfpathlineto{\pgfqpoint{7.011786in}{3.716354in}}%
\pgfpathlineto{\pgfqpoint{7.191517in}{3.759815in}}%
\pgfpathlineto{\pgfqpoint{7.371248in}{3.784455in}}%
\pgfpathlineto{\pgfqpoint{7.550979in}{3.798539in}}%
\pgfpathlineto{\pgfqpoint{7.730710in}{3.822340in}}%
\pgfpathlineto{\pgfqpoint{7.910441in}{3.849092in}}%
\pgfpathlineto{\pgfqpoint{8.090172in}{3.867466in}}%
\pgfpathlineto{\pgfqpoint{8.269903in}{3.888337in}}%
\pgfpathlineto{\pgfqpoint{8.449634in}{3.906736in}}%
\pgfpathlineto{\pgfqpoint{8.629365in}{3.925744in}}%
\pgfpathlineto{\pgfqpoint{8.809096in}{3.941231in}}%
\pgfpathlineto{\pgfqpoint{8.988827in}{3.956767in}}%
\pgfpathlineto{\pgfqpoint{9.168558in}{3.979843in}}%
\pgfpathlineto{\pgfqpoint{9.348289in}{3.983891in}}%
\pgfpathlineto{\pgfqpoint{9.528020in}{4.009370in}}%
\pgfpathlineto{\pgfqpoint{9.707751in}{4.016846in}}%
\pgfpathlineto{\pgfqpoint{9.887482in}{4.037574in}}%
\pgfusepath{stroke}%
\end{pgfscope}%
\begin{pgfscope}%
\pgfpathrectangle{\pgfqpoint{0.640323in}{0.527436in}}{\pgfqpoint{9.687500in}{3.850000in}}%
\pgfusepath{clip}%
\pgfsetbuttcap%
\pgfsetroundjoin%
\pgfsetlinewidth{1.003750pt}%
\definecolor{currentstroke}{rgb}{0.239216,0.478431,0.992157}%
\pgfsetstrokecolor{currentstroke}%
\pgfsetstrokeopacity{0.500000}%
\pgfsetdash{{3.700000pt}{1.600000pt}}{0.000000pt}%
\pgfpathmoveto{\pgfqpoint{1.080663in}{0.659775in}}%
\pgfpathlineto{\pgfqpoint{1.260394in}{0.660083in}}%
\pgfpathlineto{\pgfqpoint{1.440125in}{0.662478in}}%
\pgfpathlineto{\pgfqpoint{1.619856in}{0.664197in}}%
\pgfpathlineto{\pgfqpoint{1.799587in}{0.666835in}}%
\pgfpathlineto{\pgfqpoint{1.979318in}{0.673436in}}%
\pgfpathlineto{\pgfqpoint{2.159049in}{0.684130in}}%
\pgfpathlineto{\pgfqpoint{2.338780in}{0.699575in}}%
\pgfpathlineto{\pgfqpoint{2.518511in}{0.817273in}}%
\pgfpathlineto{\pgfqpoint{2.698242in}{1.404975in}}%
\pgfpathlineto{\pgfqpoint{2.877973in}{1.839789in}}%
\pgfpathlineto{\pgfqpoint{3.057704in}{2.084165in}}%
\pgfpathlineto{\pgfqpoint{3.237435in}{2.304534in}}%
\pgfpathlineto{\pgfqpoint{3.417166in}{2.473679in}}%
\pgfpathlineto{\pgfqpoint{3.596897in}{2.616676in}}%
\pgfpathlineto{\pgfqpoint{3.776628in}{2.753413in}}%
\pgfpathlineto{\pgfqpoint{3.956359in}{2.866901in}}%
\pgfpathlineto{\pgfqpoint{4.136090in}{2.958003in}}%
\pgfpathlineto{\pgfqpoint{4.315821in}{3.051757in}}%
\pgfpathlineto{\pgfqpoint{4.495552in}{3.122261in}}%
\pgfpathlineto{\pgfqpoint{4.675283in}{3.187326in}}%
\pgfpathlineto{\pgfqpoint{4.855014in}{3.246293in}}%
\pgfpathlineto{\pgfqpoint{5.034745in}{3.315960in}}%
\pgfpathlineto{\pgfqpoint{5.214476in}{3.371338in}}%
\pgfpathlineto{\pgfqpoint{5.394207in}{3.411545in}}%
\pgfpathlineto{\pgfqpoint{5.573938in}{3.464570in}}%
\pgfpathlineto{\pgfqpoint{5.753669in}{3.499810in}}%
\pgfpathlineto{\pgfqpoint{5.933400in}{3.543718in}}%
\pgfpathlineto{\pgfqpoint{6.113131in}{3.573301in}}%
\pgfpathlineto{\pgfqpoint{6.292862in}{3.620905in}}%
\pgfpathlineto{\pgfqpoint{6.472593in}{3.649842in}}%
\pgfpathlineto{\pgfqpoint{6.652324in}{3.671718in}}%
\pgfpathlineto{\pgfqpoint{6.832055in}{3.711504in}}%
\pgfpathlineto{\pgfqpoint{7.011786in}{3.733418in}}%
\pgfpathlineto{\pgfqpoint{7.191517in}{3.764392in}}%
\pgfpathlineto{\pgfqpoint{7.371248in}{3.789851in}}%
\pgfpathlineto{\pgfqpoint{7.550979in}{3.809971in}}%
\pgfpathlineto{\pgfqpoint{7.730710in}{3.835418in}}%
\pgfpathlineto{\pgfqpoint{7.910441in}{3.850439in}}%
\pgfpathlineto{\pgfqpoint{8.090172in}{3.880643in}}%
\pgfpathlineto{\pgfqpoint{8.269903in}{3.895863in}}%
\pgfpathlineto{\pgfqpoint{8.449634in}{3.914684in}}%
\pgfpathlineto{\pgfqpoint{8.629365in}{3.929495in}}%
\pgfpathlineto{\pgfqpoint{8.809096in}{3.948229in}}%
\pgfpathlineto{\pgfqpoint{8.988827in}{3.967957in}}%
\pgfpathlineto{\pgfqpoint{9.168558in}{3.984314in}}%
\pgfpathlineto{\pgfqpoint{9.348289in}{4.002353in}}%
\pgfpathlineto{\pgfqpoint{9.528020in}{4.011971in}}%
\pgfpathlineto{\pgfqpoint{9.707751in}{4.021472in}}%
\pgfpathlineto{\pgfqpoint{9.887482in}{4.043262in}}%
\pgfusepath{stroke}%
\end{pgfscope}%
\begin{pgfscope}%
\pgfpathrectangle{\pgfqpoint{0.640323in}{0.527436in}}{\pgfqpoint{9.687500in}{3.850000in}}%
\pgfusepath{clip}%
\pgfsetbuttcap%
\pgfsetroundjoin%
\pgfsetlinewidth{1.003750pt}%
\definecolor{currentstroke}{rgb}{0.000000,0.000000,0.000000}%
\pgfsetstrokecolor{currentstroke}%
\pgfsetstrokeopacity{0.500000}%
\pgfsetdash{{3.700000pt}{1.600000pt}}{0.000000pt}%
\pgfpathmoveto{\pgfqpoint{1.080663in}{0.659536in}}%
\pgfpathlineto{\pgfqpoint{1.260394in}{0.660526in}}%
\pgfpathlineto{\pgfqpoint{1.440125in}{0.662244in}}%
\pgfpathlineto{\pgfqpoint{1.619856in}{0.664285in}}%
\pgfpathlineto{\pgfqpoint{1.799587in}{0.666523in}}%
\pgfpathlineto{\pgfqpoint{1.979318in}{0.671589in}}%
\pgfpathlineto{\pgfqpoint{2.159049in}{0.680716in}}%
\pgfpathlineto{\pgfqpoint{2.338780in}{0.701224in}}%
\pgfpathlineto{\pgfqpoint{2.518511in}{0.828515in}}%
\pgfpathlineto{\pgfqpoint{2.698242in}{1.446021in}}%
\pgfpathlineto{\pgfqpoint{2.877973in}{1.840882in}}%
\pgfpathlineto{\pgfqpoint{3.057704in}{2.127626in}}%
\pgfpathlineto{\pgfqpoint{3.237435in}{2.334806in}}%
\pgfpathlineto{\pgfqpoint{3.417166in}{2.503784in}}%
\pgfpathlineto{\pgfqpoint{3.596897in}{2.651624in}}%
\pgfpathlineto{\pgfqpoint{3.776628in}{2.772470in}}%
\pgfpathlineto{\pgfqpoint{3.956359in}{2.882444in}}%
\pgfpathlineto{\pgfqpoint{4.136090in}{2.973459in}}%
\pgfpathlineto{\pgfqpoint{4.315821in}{3.058333in}}%
\pgfpathlineto{\pgfqpoint{4.495552in}{3.132234in}}%
\pgfpathlineto{\pgfqpoint{4.675283in}{3.203453in}}%
\pgfpathlineto{\pgfqpoint{4.855014in}{3.263159in}}%
\pgfpathlineto{\pgfqpoint{5.034745in}{3.315773in}}%
\pgfpathlineto{\pgfqpoint{5.214476in}{3.373374in}}%
\pgfpathlineto{\pgfqpoint{5.394207in}{3.421375in}}%
\pgfpathlineto{\pgfqpoint{5.573938in}{3.465700in}}%
\pgfpathlineto{\pgfqpoint{5.753669in}{3.509031in}}%
\pgfpathlineto{\pgfqpoint{5.933400in}{3.549357in}}%
\pgfpathlineto{\pgfqpoint{6.113131in}{3.586230in}}%
\pgfpathlineto{\pgfqpoint{6.292862in}{3.621470in}}%
\pgfpathlineto{\pgfqpoint{6.472593in}{3.654182in}}%
\pgfpathlineto{\pgfqpoint{6.652324in}{3.681362in}}%
\pgfpathlineto{\pgfqpoint{6.832055in}{3.712559in}}%
\pgfpathlineto{\pgfqpoint{7.011786in}{3.739069in}}%
\pgfpathlineto{\pgfqpoint{7.191517in}{3.766838in}}%
\pgfpathlineto{\pgfqpoint{7.371248in}{3.793819in}}%
\pgfpathlineto{\pgfqpoint{7.550979in}{3.815982in}}%
\pgfpathlineto{\pgfqpoint{7.730710in}{3.836660in}}%
\pgfpathlineto{\pgfqpoint{7.910441in}{3.860412in}}%
\pgfpathlineto{\pgfqpoint{8.090172in}{3.878345in}}%
\pgfpathlineto{\pgfqpoint{8.269903in}{3.898428in}}%
\pgfpathlineto{\pgfqpoint{8.449634in}{3.917584in}}%
\pgfpathlineto{\pgfqpoint{8.629365in}{3.934835in}}%
\pgfpathlineto{\pgfqpoint{8.809096in}{3.952526in}}%
\pgfpathlineto{\pgfqpoint{8.988827in}{3.967429in}}%
\pgfpathlineto{\pgfqpoint{9.168558in}{3.982910in}}%
\pgfpathlineto{\pgfqpoint{9.348289in}{4.001378in}}%
\pgfpathlineto{\pgfqpoint{9.528020in}{4.014300in}}%
\pgfpathlineto{\pgfqpoint{9.707751in}{4.028868in}}%
\pgfpathlineto{\pgfqpoint{9.887482in}{4.042424in}}%
\pgfusepath{stroke}%
\end{pgfscope}%
\begin{pgfscope}%
\pgfsetrectcap%
\pgfsetmiterjoin%
\pgfsetlinewidth{0.803000pt}%
\definecolor{currentstroke}{rgb}{0.000000,0.000000,0.000000}%
\pgfsetstrokecolor{currentstroke}%
\pgfsetdash{}{0pt}%
\pgfpathmoveto{\pgfqpoint{0.640323in}{0.527436in}}%
\pgfpathlineto{\pgfqpoint{0.640323in}{4.377436in}}%
\pgfusepath{stroke}%
\end{pgfscope}%
\begin{pgfscope}%
\pgfsetrectcap%
\pgfsetmiterjoin%
\pgfsetlinewidth{0.803000pt}%
\definecolor{currentstroke}{rgb}{0.000000,0.000000,0.000000}%
\pgfsetstrokecolor{currentstroke}%
\pgfsetdash{}{0pt}%
\pgfpathmoveto{\pgfqpoint{10.327822in}{0.527436in}}%
\pgfpathlineto{\pgfqpoint{10.327822in}{4.377436in}}%
\pgfusepath{stroke}%
\end{pgfscope}%
\begin{pgfscope}%
\pgfsetrectcap%
\pgfsetmiterjoin%
\pgfsetlinewidth{0.803000pt}%
\definecolor{currentstroke}{rgb}{0.000000,0.000000,0.000000}%
\pgfsetstrokecolor{currentstroke}%
\pgfsetdash{}{0pt}%
\pgfpathmoveto{\pgfqpoint{0.640322in}{0.527436in}}%
\pgfpathlineto{\pgfqpoint{10.327823in}{0.527436in}}%
\pgfusepath{stroke}%
\end{pgfscope}%
\begin{pgfscope}%
\pgfsetrectcap%
\pgfsetmiterjoin%
\pgfsetlinewidth{0.803000pt}%
\definecolor{currentstroke}{rgb}{0.000000,0.000000,0.000000}%
\pgfsetstrokecolor{currentstroke}%
\pgfsetdash{}{0pt}%
\pgfpathmoveto{\pgfqpoint{0.640322in}{4.377436in}}%
\pgfpathlineto{\pgfqpoint{10.327823in}{4.377436in}}%
\pgfusepath{stroke}%
\end{pgfscope}%
\begin{pgfscope}%
\definecolor{textcolor}{rgb}{0.000000,0.000000,0.000000}%
\pgfsetstrokecolor{textcolor}%
\pgfsetfillcolor{textcolor}%
\pgftext[x=5.484072in,y=4.460769in,,base]{\color{textcolor}\sffamily\fontsize{12.000000}{14.400000}\selectfont \(\displaystyle \overline{\langle I\rangle}\) over \(\displaystyle p_1\) for \(\displaystyle T=1000\) with \(\displaystyle p_2=0.3\), \(\displaystyle p_3=0.6\)}%
\end{pgfscope}%
\begin{pgfscope}%
\pgfsetbuttcap%
\pgfsetmiterjoin%
\definecolor{currentfill}{rgb}{1.000000,1.000000,1.000000}%
\pgfsetfillcolor{currentfill}%
\pgfsetfillopacity{0.800000}%
\pgfsetlinewidth{1.003750pt}%
\definecolor{currentstroke}{rgb}{0.800000,0.800000,0.800000}%
\pgfsetstrokecolor{currentstroke}%
\pgfsetstrokeopacity{0.800000}%
\pgfsetdash{}{0pt}%
\pgfpathmoveto{\pgfqpoint{0.737545in}{3.450896in}}%
\pgfpathlineto{\pgfqpoint{1.670029in}{3.450896in}}%
\pgfpathquadraticcurveto{\pgfqpoint{1.697806in}{3.450896in}}{\pgfqpoint{1.697806in}{3.478674in}}%
\pgfpathlineto{\pgfqpoint{1.697806in}{4.280214in}}%
\pgfpathquadraticcurveto{\pgfqpoint{1.697806in}{4.307991in}}{\pgfqpoint{1.670029in}{4.307991in}}%
\pgfpathlineto{\pgfqpoint{0.737545in}{4.307991in}}%
\pgfpathquadraticcurveto{\pgfqpoint{0.709767in}{4.307991in}}{\pgfqpoint{0.709767in}{4.280214in}}%
\pgfpathlineto{\pgfqpoint{0.709767in}{3.478674in}}%
\pgfpathquadraticcurveto{\pgfqpoint{0.709767in}{3.450896in}}{\pgfqpoint{0.737545in}{3.450896in}}%
\pgfpathlineto{\pgfqpoint{0.737545in}{3.450896in}}%
\pgfpathclose%
\pgfusepath{stroke,fill}%
\end{pgfscope}%
\begin{pgfscope}%
\pgfsetbuttcap%
\pgfsetroundjoin%
\definecolor{currentfill}{rgb}{0.000000,0.000000,1.000000}%
\pgfsetfillcolor{currentfill}%
\pgfsetfillopacity{0.500000}%
\pgfsetlinewidth{1.003750pt}%
\definecolor{currentstroke}{rgb}{0.000000,0.000000,1.000000}%
\pgfsetstrokecolor{currentstroke}%
\pgfsetstrokeopacity{0.500000}%
\pgfsetdash{{3.700000pt}{1.600000pt}}{0.000000pt}%
\pgfpathmoveto{\pgfqpoint{0.904211in}{4.161411in}}%
\pgfpathcurveto{\pgfqpoint{0.910035in}{4.161411in}}{\pgfqpoint{0.915621in}{4.163725in}}{\pgfqpoint{0.919740in}{4.167843in}}%
\pgfpathcurveto{\pgfqpoint{0.923858in}{4.171961in}}{\pgfqpoint{0.926172in}{4.177547in}}{\pgfqpoint{0.926172in}{4.183371in}}%
\pgfpathcurveto{\pgfqpoint{0.926172in}{4.189195in}}{\pgfqpoint{0.923858in}{4.194781in}}{\pgfqpoint{0.919740in}{4.198899in}}%
\pgfpathcurveto{\pgfqpoint{0.915621in}{4.203018in}}{\pgfqpoint{0.910035in}{4.205331in}}{\pgfqpoint{0.904211in}{4.205331in}}%
\pgfpathcurveto{\pgfqpoint{0.898387in}{4.205331in}}{\pgfqpoint{0.892801in}{4.203018in}}{\pgfqpoint{0.888683in}{4.198899in}}%
\pgfpathcurveto{\pgfqpoint{0.884565in}{4.194781in}}{\pgfqpoint{0.882251in}{4.189195in}}{\pgfqpoint{0.882251in}{4.183371in}}%
\pgfpathcurveto{\pgfqpoint{0.882251in}{4.177547in}}{\pgfqpoint{0.884565in}{4.171961in}}{\pgfqpoint{0.888683in}{4.167843in}}%
\pgfpathcurveto{\pgfqpoint{0.892801in}{4.163725in}}{\pgfqpoint{0.898387in}{4.161411in}}{\pgfqpoint{0.904211in}{4.161411in}}%
\pgfpathlineto{\pgfqpoint{0.904211in}{4.161411in}}%
\pgfpathclose%
\pgfusepath{stroke,fill}%
\end{pgfscope}%
\begin{pgfscope}%
\definecolor{textcolor}{rgb}{0.000000,0.000000,0.000000}%
\pgfsetstrokecolor{textcolor}%
\pgfsetfillcolor{textcolor}%
\pgftext[x=1.154211in,y=4.146913in,left,base]{\color{textcolor}\sffamily\fontsize{10.000000}{12.000000}\selectfont \(\displaystyle L=16\)}%
\end{pgfscope}%
\begin{pgfscope}%
\pgfsetbuttcap%
\pgfsetroundjoin%
\definecolor{currentfill}{rgb}{0.980392,0.164706,0.333333}%
\pgfsetfillcolor{currentfill}%
\pgfsetfillopacity{0.500000}%
\pgfsetlinewidth{1.003750pt}%
\definecolor{currentstroke}{rgb}{0.980392,0.164706,0.333333}%
\pgfsetstrokecolor{currentstroke}%
\pgfsetstrokeopacity{0.500000}%
\pgfsetdash{{3.700000pt}{1.600000pt}}{0.000000pt}%
\pgfpathmoveto{\pgfqpoint{0.904211in}{3.957554in}}%
\pgfpathcurveto{\pgfqpoint{0.910035in}{3.957554in}}{\pgfqpoint{0.915621in}{3.959867in}}{\pgfqpoint{0.919740in}{3.963986in}}%
\pgfpathcurveto{\pgfqpoint{0.923858in}{3.968104in}}{\pgfqpoint{0.926172in}{3.973690in}}{\pgfqpoint{0.926172in}{3.979514in}}%
\pgfpathcurveto{\pgfqpoint{0.926172in}{3.985338in}}{\pgfqpoint{0.923858in}{3.990924in}}{\pgfqpoint{0.919740in}{3.995042in}}%
\pgfpathcurveto{\pgfqpoint{0.915621in}{3.999160in}}{\pgfqpoint{0.910035in}{4.001474in}}{\pgfqpoint{0.904211in}{4.001474in}}%
\pgfpathcurveto{\pgfqpoint{0.898387in}{4.001474in}}{\pgfqpoint{0.892801in}{3.999160in}}{\pgfqpoint{0.888683in}{3.995042in}}%
\pgfpathcurveto{\pgfqpoint{0.884565in}{3.990924in}}{\pgfqpoint{0.882251in}{3.985338in}}{\pgfqpoint{0.882251in}{3.979514in}}%
\pgfpathcurveto{\pgfqpoint{0.882251in}{3.973690in}}{\pgfqpoint{0.884565in}{3.968104in}}{\pgfqpoint{0.888683in}{3.963986in}}%
\pgfpathcurveto{\pgfqpoint{0.892801in}{3.959867in}}{\pgfqpoint{0.898387in}{3.957554in}}{\pgfqpoint{0.904211in}{3.957554in}}%
\pgfpathlineto{\pgfqpoint{0.904211in}{3.957554in}}%
\pgfpathclose%
\pgfusepath{stroke,fill}%
\end{pgfscope}%
\begin{pgfscope}%
\definecolor{textcolor}{rgb}{0.000000,0.000000,0.000000}%
\pgfsetstrokecolor{textcolor}%
\pgfsetfillcolor{textcolor}%
\pgftext[x=1.154211in,y=3.943056in,left,base]{\color{textcolor}\sffamily\fontsize{10.000000}{12.000000}\selectfont \(\displaystyle L=32\)}%
\end{pgfscope}%
\begin{pgfscope}%
\pgfsetbuttcap%
\pgfsetroundjoin%
\definecolor{currentfill}{rgb}{0.239216,0.478431,0.992157}%
\pgfsetfillcolor{currentfill}%
\pgfsetfillopacity{0.500000}%
\pgfsetlinewidth{1.003750pt}%
\definecolor{currentstroke}{rgb}{0.239216,0.478431,0.992157}%
\pgfsetstrokecolor{currentstroke}%
\pgfsetstrokeopacity{0.500000}%
\pgfsetdash{{3.700000pt}{1.600000pt}}{0.000000pt}%
\pgfpathmoveto{\pgfqpoint{0.904211in}{3.753696in}}%
\pgfpathcurveto{\pgfqpoint{0.910035in}{3.753696in}}{\pgfqpoint{0.915621in}{3.756010in}}{\pgfqpoint{0.919740in}{3.760128in}}%
\pgfpathcurveto{\pgfqpoint{0.923858in}{3.764247in}}{\pgfqpoint{0.926172in}{3.769833in}}{\pgfqpoint{0.926172in}{3.775657in}}%
\pgfpathcurveto{\pgfqpoint{0.926172in}{3.781481in}}{\pgfqpoint{0.923858in}{3.787067in}}{\pgfqpoint{0.919740in}{3.791185in}}%
\pgfpathcurveto{\pgfqpoint{0.915621in}{3.795303in}}{\pgfqpoint{0.910035in}{3.797617in}}{\pgfqpoint{0.904211in}{3.797617in}}%
\pgfpathcurveto{\pgfqpoint{0.898387in}{3.797617in}}{\pgfqpoint{0.892801in}{3.795303in}}{\pgfqpoint{0.888683in}{3.791185in}}%
\pgfpathcurveto{\pgfqpoint{0.884565in}{3.787067in}}{\pgfqpoint{0.882251in}{3.781481in}}{\pgfqpoint{0.882251in}{3.775657in}}%
\pgfpathcurveto{\pgfqpoint{0.882251in}{3.769833in}}{\pgfqpoint{0.884565in}{3.764247in}}{\pgfqpoint{0.888683in}{3.760128in}}%
\pgfpathcurveto{\pgfqpoint{0.892801in}{3.756010in}}{\pgfqpoint{0.898387in}{3.753696in}}{\pgfqpoint{0.904211in}{3.753696in}}%
\pgfpathlineto{\pgfqpoint{0.904211in}{3.753696in}}%
\pgfpathclose%
\pgfusepath{stroke,fill}%
\end{pgfscope}%
\begin{pgfscope}%
\definecolor{textcolor}{rgb}{0.000000,0.000000,0.000000}%
\pgfsetstrokecolor{textcolor}%
\pgfsetfillcolor{textcolor}%
\pgftext[x=1.154211in,y=3.739198in,left,base]{\color{textcolor}\sffamily\fontsize{10.000000}{12.000000}\selectfont \(\displaystyle L=64\)}%
\end{pgfscope}%
\begin{pgfscope}%
\pgfsetbuttcap%
\pgfsetroundjoin%
\definecolor{currentfill}{rgb}{0.000000,0.000000,0.000000}%
\pgfsetfillcolor{currentfill}%
\pgfsetfillopacity{0.500000}%
\pgfsetlinewidth{1.003750pt}%
\definecolor{currentstroke}{rgb}{0.000000,0.000000,0.000000}%
\pgfsetstrokecolor{currentstroke}%
\pgfsetstrokeopacity{0.500000}%
\pgfsetdash{{3.700000pt}{1.600000pt}}{0.000000pt}%
\pgfpathmoveto{\pgfqpoint{0.904211in}{3.549839in}}%
\pgfpathcurveto{\pgfqpoint{0.910035in}{3.549839in}}{\pgfqpoint{0.915621in}{3.552153in}}{\pgfqpoint{0.919740in}{3.556271in}}%
\pgfpathcurveto{\pgfqpoint{0.923858in}{3.560389in}}{\pgfqpoint{0.926172in}{3.565976in}}{\pgfqpoint{0.926172in}{3.571799in}}%
\pgfpathcurveto{\pgfqpoint{0.926172in}{3.577623in}}{\pgfqpoint{0.923858in}{3.583210in}}{\pgfqpoint{0.919740in}{3.587328in}}%
\pgfpathcurveto{\pgfqpoint{0.915621in}{3.591446in}}{\pgfqpoint{0.910035in}{3.593760in}}{\pgfqpoint{0.904211in}{3.593760in}}%
\pgfpathcurveto{\pgfqpoint{0.898387in}{3.593760in}}{\pgfqpoint{0.892801in}{3.591446in}}{\pgfqpoint{0.888683in}{3.587328in}}%
\pgfpathcurveto{\pgfqpoint{0.884565in}{3.583210in}}{\pgfqpoint{0.882251in}{3.577623in}}{\pgfqpoint{0.882251in}{3.571799in}}%
\pgfpathcurveto{\pgfqpoint{0.882251in}{3.565976in}}{\pgfqpoint{0.884565in}{3.560389in}}{\pgfqpoint{0.888683in}{3.556271in}}%
\pgfpathcurveto{\pgfqpoint{0.892801in}{3.552153in}}{\pgfqpoint{0.898387in}{3.549839in}}{\pgfqpoint{0.904211in}{3.549839in}}%
\pgfpathlineto{\pgfqpoint{0.904211in}{3.549839in}}%
\pgfpathclose%
\pgfusepath{stroke,fill}%
\end{pgfscope}%
\begin{pgfscope}%
\definecolor{textcolor}{rgb}{0.000000,0.000000,0.000000}%
\pgfsetstrokecolor{textcolor}%
\pgfsetfillcolor{textcolor}%
\pgftext[x=1.154211in,y=3.535341in,left,base]{\color{textcolor}\sffamily\fontsize{10.000000}{12.000000}\selectfont \(\displaystyle L=128\)}%
\end{pgfscope}%
\end{pgfpicture}%
\makeatother%
\endgroup%
}
    \caption{Plots showing the time-averaged infection rates $\overline{\langle I\rangle}$ in dependence of the turnover probability $p_1\left(\susceptible\rightarrow\infected\,\right)$
    for the different grid sizes $L=16$, $L=32$, $L=64$ and $L=128$ with $T=1000$ simulation steps each. The upper plot shows has now been generated using the turnover rates 
    $p_2\left(\susceptible\rightarrow\infected\,\right)=0.3$ and $p_3\left(\susceptible\rightarrow\infected\,\right)=0.3$, 
    the middle one using $p_2=0.6$ and $p_3=0.3$, and the lower one $p_2=0.3$ and $p_3=0.6$.}\label{fig:Res_Dis_Avg_Inf_over_p1}
\end{figure}

To further focus on the variations of $p_2$ and $p_3$ as well as the exact shape of the plotted curves, the time-averaged infection rates $\overline{\langle I\rangle}$ for the different probability combinations are 
shown together in \prettyref{fig:Res_Dis_Avg_Inf_over_p1_L128}. The plot only comprises the grid size $L=128$ for better visibility, while the biggest grid size was chosen for accuracy reason. 

\begin{figure}[ht]
    \centering
    \resizebox{\textwidth}{!}{%% Creator: Matplotlib, PGF backend
%%
%% To include the figure in your LaTeX document, write
%%   \input{<filename>.pgf}
%%
%% Make sure the required packages are loaded in your preamble
%%   \usepackage{pgf}
%%
%% Also ensure that all the required font packages are loaded; for instance,
%% the lmodern package is sometimes necessary when using math font.
%%   \usepackage{lmodern}
%%
%% Figures using additional raster images can only be included by \input if
%% they are in the same directory as the main LaTeX file. For loading figures
%% from other directories you can use the `import` package
%%   \usepackage{import}
%%
%% and then include the figures with
%%   \import{<path to file>}{<filename>.pgf}
%%
%% Matplotlib used the following preamble
%%   
%%   \usepackage{fontspec}
%%   \setmainfont{DejaVuSerif.ttf}[Path=\detokenize{/home/carlo/.local/lib/python3.10/site-packages/matplotlib/mpl-data/fonts/ttf/}]
%%   \setsansfont{DejaVuSans.ttf}[Path=\detokenize{/home/carlo/.local/lib/python3.10/site-packages/matplotlib/mpl-data/fonts/ttf/}]
%%   \setmonofont{DejaVuSansMono.ttf}[Path=\detokenize{/home/carlo/.local/lib/python3.10/site-packages/matplotlib/mpl-data/fonts/ttf/}]
%%   \makeatletter\@ifpackageloaded{underscore}{}{\usepackage[strings]{underscore}}\makeatother
%%
\begingroup%
\makeatletter%
\begin{pgfpicture}%
\pgfpathrectangle{\pgfpointorigin}{\pgfqpoint{10.538262in}{5.098546in}}%
\pgfusepath{use as bounding box, clip}%
\begin{pgfscope}%
\pgfsetbuttcap%
\pgfsetmiterjoin%
\definecolor{currentfill}{rgb}{1.000000,1.000000,1.000000}%
\pgfsetfillcolor{currentfill}%
\pgfsetlinewidth{0.000000pt}%
\definecolor{currentstroke}{rgb}{1.000000,1.000000,1.000000}%
\pgfsetstrokecolor{currentstroke}%
\pgfsetdash{}{0pt}%
\pgfpathmoveto{\pgfqpoint{0.000000in}{0.000000in}}%
\pgfpathlineto{\pgfqpoint{10.538262in}{0.000000in}}%
\pgfpathlineto{\pgfqpoint{10.538262in}{5.098546in}}%
\pgfpathlineto{\pgfqpoint{0.000000in}{5.098546in}}%
\pgfpathlineto{\pgfqpoint{0.000000in}{0.000000in}}%
\pgfpathclose%
\pgfusepath{fill}%
\end{pgfscope}%
\begin{pgfscope}%
\pgfsetbuttcap%
\pgfsetmiterjoin%
\definecolor{currentfill}{rgb}{1.000000,1.000000,1.000000}%
\pgfsetfillcolor{currentfill}%
\pgfsetlinewidth{0.000000pt}%
\definecolor{currentstroke}{rgb}{0.000000,0.000000,0.000000}%
\pgfsetstrokecolor{currentstroke}%
\pgfsetstrokeopacity{0.000000}%
\pgfsetdash{}{0pt}%
\pgfpathmoveto{\pgfqpoint{0.640323in}{0.527436in}}%
\pgfpathlineto{\pgfqpoint{5.043732in}{0.527436in}}%
\pgfpathlineto{\pgfqpoint{5.043732in}{4.762436in}}%
\pgfpathlineto{\pgfqpoint{0.640323in}{4.762436in}}%
\pgfpathlineto{\pgfqpoint{0.640323in}{0.527436in}}%
\pgfpathclose%
\pgfusepath{fill}%
\end{pgfscope}%
\begin{pgfscope}%
\pgfpathrectangle{\pgfqpoint{0.640323in}{0.527436in}}{\pgfqpoint{4.403409in}{4.235000in}}%
\pgfusepath{clip}%
\pgfsetbuttcap%
\pgfsetroundjoin%
\definecolor{currentfill}{rgb}{0.980392,0.164706,0.333333}%
\pgfsetfillcolor{currentfill}%
\pgfsetfillopacity{0.500000}%
\pgfsetlinewidth{1.003750pt}%
\definecolor{currentstroke}{rgb}{0.980392,0.164706,0.333333}%
\pgfsetstrokecolor{currentstroke}%
\pgfsetstrokeopacity{0.500000}%
\pgfsetdash{}{0pt}%
\pgfsys@defobject{currentmarker}{\pgfqpoint{-0.021960in}{-0.021960in}}{\pgfqpoint{0.021960in}{0.021960in}}{%
\pgfpathmoveto{\pgfqpoint{0.000000in}{-0.021960in}}%
\pgfpathcurveto{\pgfqpoint{0.005824in}{-0.021960in}}{\pgfqpoint{0.011410in}{-0.019646in}}{\pgfqpoint{0.015528in}{-0.015528in}}%
\pgfpathcurveto{\pgfqpoint{0.019646in}{-0.011410in}}{\pgfqpoint{0.021960in}{-0.005824in}}{\pgfqpoint{0.021960in}{0.000000in}}%
\pgfpathcurveto{\pgfqpoint{0.021960in}{0.005824in}}{\pgfqpoint{0.019646in}{0.011410in}}{\pgfqpoint{0.015528in}{0.015528in}}%
\pgfpathcurveto{\pgfqpoint{0.011410in}{0.019646in}}{\pgfqpoint{0.005824in}{0.021960in}}{\pgfqpoint{0.000000in}{0.021960in}}%
\pgfpathcurveto{\pgfqpoint{-0.005824in}{0.021960in}}{\pgfqpoint{-0.011410in}{0.019646in}}{\pgfqpoint{-0.015528in}{0.015528in}}%
\pgfpathcurveto{\pgfqpoint{-0.019646in}{0.011410in}}{\pgfqpoint{-0.021960in}{0.005824in}}{\pgfqpoint{-0.021960in}{0.000000in}}%
\pgfpathcurveto{\pgfqpoint{-0.021960in}{-0.005824in}}{\pgfqpoint{-0.019646in}{-0.011410in}}{\pgfqpoint{-0.015528in}{-0.015528in}}%
\pgfpathcurveto{\pgfqpoint{-0.011410in}{-0.019646in}}{\pgfqpoint{-0.005824in}{-0.021960in}}{\pgfqpoint{0.000000in}{-0.021960in}}%
\pgfpathlineto{\pgfqpoint{0.000000in}{-0.021960in}}%
\pgfpathclose%
\pgfusepath{stroke,fill}%
}%
\begin{pgfscope}%
\pgfsys@transformshift{0.840477in}{0.673013in}%
\pgfsys@useobject{currentmarker}{}%
\end{pgfscope}%
\begin{pgfscope}%
\pgfsys@transformshift{0.920539in}{0.673804in}%
\pgfsys@useobject{currentmarker}{}%
\end{pgfscope}%
\begin{pgfscope}%
\pgfsys@transformshift{1.000601in}{0.675232in}%
\pgfsys@useobject{currentmarker}{}%
\end{pgfscope}%
\begin{pgfscope}%
\pgfsys@transformshift{1.080663in}{0.676986in}%
\pgfsys@useobject{currentmarker}{}%
\end{pgfscope}%
\begin{pgfscope}%
\pgfsys@transformshift{1.160725in}{0.679653in}%
\pgfsys@useobject{currentmarker}{}%
\end{pgfscope}%
\begin{pgfscope}%
\pgfsys@transformshift{1.240787in}{0.682544in}%
\pgfsys@useobject{currentmarker}{}%
\end{pgfscope}%
\begin{pgfscope}%
\pgfsys@transformshift{1.320849in}{0.688079in}%
\pgfsys@useobject{currentmarker}{}%
\end{pgfscope}%
\begin{pgfscope}%
\pgfsys@transformshift{1.400911in}{0.696104in}%
\pgfsys@useobject{currentmarker}{}%
\end{pgfscope}%
\begin{pgfscope}%
\pgfsys@transformshift{1.480973in}{0.720753in}%
\pgfsys@useobject{currentmarker}{}%
\end{pgfscope}%
\begin{pgfscope}%
\pgfsys@transformshift{1.561035in}{0.852172in}%
\pgfsys@useobject{currentmarker}{}%
\end{pgfscope}%
\begin{pgfscope}%
\pgfsys@transformshift{1.641097in}{1.357469in}%
\pgfsys@useobject{currentmarker}{}%
\end{pgfscope}%
\begin{pgfscope}%
\pgfsys@transformshift{1.721159in}{1.681323in}%
\pgfsys@useobject{currentmarker}{}%
\end{pgfscope}%
\begin{pgfscope}%
\pgfsys@transformshift{1.801221in}{1.927308in}%
\pgfsys@useobject{currentmarker}{}%
\end{pgfscope}%
\begin{pgfscope}%
\pgfsys@transformshift{1.881283in}{2.115752in}%
\pgfsys@useobject{currentmarker}{}%
\end{pgfscope}%
\begin{pgfscope}%
\pgfsys@transformshift{1.961345in}{2.259141in}%
\pgfsys@useobject{currentmarker}{}%
\end{pgfscope}%
\begin{pgfscope}%
\pgfsys@transformshift{2.041407in}{2.386485in}%
\pgfsys@useobject{currentmarker}{}%
\end{pgfscope}%
\begin{pgfscope}%
\pgfsys@transformshift{2.121469in}{2.492223in}%
\pgfsys@useobject{currentmarker}{}%
\end{pgfscope}%
\begin{pgfscope}%
\pgfsys@transformshift{2.201531in}{2.581226in}%
\pgfsys@useobject{currentmarker}{}%
\end{pgfscope}%
\begin{pgfscope}%
\pgfsys@transformshift{2.281593in}{2.662689in}%
\pgfsys@useobject{currentmarker}{}%
\end{pgfscope}%
\begin{pgfscope}%
\pgfsys@transformshift{2.361655in}{2.725387in}%
\pgfsys@useobject{currentmarker}{}%
\end{pgfscope}%
\begin{pgfscope}%
\pgfsys@transformshift{2.441717in}{2.793174in}%
\pgfsys@useobject{currentmarker}{}%
\end{pgfscope}%
\begin{pgfscope}%
\pgfsys@transformshift{2.521779in}{2.852389in}%
\pgfsys@useobject{currentmarker}{}%
\end{pgfscope}%
\begin{pgfscope}%
\pgfsys@transformshift{2.601841in}{2.901495in}%
\pgfsys@useobject{currentmarker}{}%
\end{pgfscope}%
\begin{pgfscope}%
\pgfsys@transformshift{2.681903in}{2.948032in}%
\pgfsys@useobject{currentmarker}{}%
\end{pgfscope}%
\begin{pgfscope}%
\pgfsys@transformshift{2.761965in}{2.987957in}%
\pgfsys@useobject{currentmarker}{}%
\end{pgfscope}%
\begin{pgfscope}%
\pgfsys@transformshift{2.842027in}{3.028347in}%
\pgfsys@useobject{currentmarker}{}%
\end{pgfscope}%
\begin{pgfscope}%
\pgfsys@transformshift{2.922089in}{3.065847in}%
\pgfsys@useobject{currentmarker}{}%
\end{pgfscope}%
\begin{pgfscope}%
\pgfsys@transformshift{3.002151in}{3.096476in}%
\pgfsys@useobject{currentmarker}{}%
\end{pgfscope}%
\begin{pgfscope}%
\pgfsys@transformshift{3.082213in}{3.129522in}%
\pgfsys@useobject{currentmarker}{}%
\end{pgfscope}%
\begin{pgfscope}%
\pgfsys@transformshift{3.162275in}{3.162432in}%
\pgfsys@useobject{currentmarker}{}%
\end{pgfscope}%
\begin{pgfscope}%
\pgfsys@transformshift{3.242337in}{3.190322in}%
\pgfsys@useobject{currentmarker}{}%
\end{pgfscope}%
\begin{pgfscope}%
\pgfsys@transformshift{3.322399in}{3.215814in}%
\pgfsys@useobject{currentmarker}{}%
\end{pgfscope}%
\begin{pgfscope}%
\pgfsys@transformshift{3.402461in}{3.239482in}%
\pgfsys@useobject{currentmarker}{}%
\end{pgfscope}%
\begin{pgfscope}%
\pgfsys@transformshift{3.482523in}{3.262201in}%
\pgfsys@useobject{currentmarker}{}%
\end{pgfscope}%
\begin{pgfscope}%
\pgfsys@transformshift{3.562585in}{3.284243in}%
\pgfsys@useobject{currentmarker}{}%
\end{pgfscope}%
\begin{pgfscope}%
\pgfsys@transformshift{3.642647in}{3.303048in}%
\pgfsys@useobject{currentmarker}{}%
\end{pgfscope}%
\begin{pgfscope}%
\pgfsys@transformshift{3.722709in}{3.324571in}%
\pgfsys@useobject{currentmarker}{}%
\end{pgfscope}%
\begin{pgfscope}%
\pgfsys@transformshift{3.802771in}{3.338342in}%
\pgfsys@useobject{currentmarker}{}%
\end{pgfscope}%
\begin{pgfscope}%
\pgfsys@transformshift{3.882833in}{3.360084in}%
\pgfsys@useobject{currentmarker}{}%
\end{pgfscope}%
\begin{pgfscope}%
\pgfsys@transformshift{3.962895in}{3.374462in}%
\pgfsys@useobject{currentmarker}{}%
\end{pgfscope}%
\begin{pgfscope}%
\pgfsys@transformshift{4.042957in}{3.390665in}%
\pgfsys@useobject{currentmarker}{}%
\end{pgfscope}%
\begin{pgfscope}%
\pgfsys@transformshift{4.123019in}{3.407748in}%
\pgfsys@useobject{currentmarker}{}%
\end{pgfscope}%
\begin{pgfscope}%
\pgfsys@transformshift{4.203081in}{3.418602in}%
\pgfsys@useobject{currentmarker}{}%
\end{pgfscope}%
\begin{pgfscope}%
\pgfsys@transformshift{4.283143in}{3.436041in}%
\pgfsys@useobject{currentmarker}{}%
\end{pgfscope}%
\begin{pgfscope}%
\pgfsys@transformshift{4.363205in}{3.446553in}%
\pgfsys@useobject{currentmarker}{}%
\end{pgfscope}%
\begin{pgfscope}%
\pgfsys@transformshift{4.443267in}{3.460371in}%
\pgfsys@useobject{currentmarker}{}%
\end{pgfscope}%
\begin{pgfscope}%
\pgfsys@transformshift{4.523329in}{3.473951in}%
\pgfsys@useobject{currentmarker}{}%
\end{pgfscope}%
\begin{pgfscope}%
\pgfsys@transformshift{4.603391in}{3.482462in}%
\pgfsys@useobject{currentmarker}{}%
\end{pgfscope}%
\begin{pgfscope}%
\pgfsys@transformshift{4.683453in}{3.492913in}%
\pgfsys@useobject{currentmarker}{}%
\end{pgfscope}%
\begin{pgfscope}%
\pgfsys@transformshift{4.763515in}{3.504894in}%
\pgfsys@useobject{currentmarker}{}%
\end{pgfscope}%
\begin{pgfscope}%
\pgfsys@transformshift{4.843577in}{3.516328in}%
\pgfsys@useobject{currentmarker}{}%
\end{pgfscope}%
\end{pgfscope}%
\begin{pgfscope}%
\pgfpathrectangle{\pgfqpoint{0.640323in}{0.527436in}}{\pgfqpoint{4.403409in}{4.235000in}}%
\pgfusepath{clip}%
\pgfsetbuttcap%
\pgfsetroundjoin%
\definecolor{currentfill}{rgb}{0.000000,0.000000,1.000000}%
\pgfsetfillcolor{currentfill}%
\pgfsetfillopacity{0.500000}%
\pgfsetlinewidth{1.003750pt}%
\definecolor{currentstroke}{rgb}{0.000000,0.000000,1.000000}%
\pgfsetstrokecolor{currentstroke}%
\pgfsetstrokeopacity{0.500000}%
\pgfsetdash{}{0pt}%
\pgfsys@defobject{currentmarker}{\pgfqpoint{-0.021960in}{-0.021960in}}{\pgfqpoint{0.021960in}{0.021960in}}{%
\pgfpathmoveto{\pgfqpoint{0.000000in}{-0.021960in}}%
\pgfpathcurveto{\pgfqpoint{0.005824in}{-0.021960in}}{\pgfqpoint{0.011410in}{-0.019646in}}{\pgfqpoint{0.015528in}{-0.015528in}}%
\pgfpathcurveto{\pgfqpoint{0.019646in}{-0.011410in}}{\pgfqpoint{0.021960in}{-0.005824in}}{\pgfqpoint{0.021960in}{0.000000in}}%
\pgfpathcurveto{\pgfqpoint{0.021960in}{0.005824in}}{\pgfqpoint{0.019646in}{0.011410in}}{\pgfqpoint{0.015528in}{0.015528in}}%
\pgfpathcurveto{\pgfqpoint{0.011410in}{0.019646in}}{\pgfqpoint{0.005824in}{0.021960in}}{\pgfqpoint{0.000000in}{0.021960in}}%
\pgfpathcurveto{\pgfqpoint{-0.005824in}{0.021960in}}{\pgfqpoint{-0.011410in}{0.019646in}}{\pgfqpoint{-0.015528in}{0.015528in}}%
\pgfpathcurveto{\pgfqpoint{-0.019646in}{0.011410in}}{\pgfqpoint{-0.021960in}{0.005824in}}{\pgfqpoint{-0.021960in}{0.000000in}}%
\pgfpathcurveto{\pgfqpoint{-0.021960in}{-0.005824in}}{\pgfqpoint{-0.019646in}{-0.011410in}}{\pgfqpoint{-0.015528in}{-0.015528in}}%
\pgfpathcurveto{\pgfqpoint{-0.011410in}{-0.019646in}}{\pgfqpoint{-0.005824in}{-0.021960in}}{\pgfqpoint{0.000000in}{-0.021960in}}%
\pgfpathlineto{\pgfqpoint{0.000000in}{-0.021960in}}%
\pgfpathclose%
\pgfusepath{stroke,fill}%
}%
\begin{pgfscope}%
\pgfsys@transformshift{0.840477in}{0.669115in}%
\pgfsys@useobject{currentmarker}{}%
\end{pgfscope}%
\begin{pgfscope}%
\pgfsys@transformshift{0.920539in}{0.669045in}%
\pgfsys@useobject{currentmarker}{}%
\end{pgfscope}%
\begin{pgfscope}%
\pgfsys@transformshift{1.000601in}{0.669505in}%
\pgfsys@useobject{currentmarker}{}%
\end{pgfscope}%
\begin{pgfscope}%
\pgfsys@transformshift{1.080663in}{0.669854in}%
\pgfsys@useobject{currentmarker}{}%
\end{pgfscope}%
\begin{pgfscope}%
\pgfsys@transformshift{1.160725in}{0.670177in}%
\pgfsys@useobject{currentmarker}{}%
\end{pgfscope}%
\begin{pgfscope}%
\pgfsys@transformshift{1.240787in}{0.670457in}%
\pgfsys@useobject{currentmarker}{}%
\end{pgfscope}%
\begin{pgfscope}%
\pgfsys@transformshift{1.320849in}{0.670753in}%
\pgfsys@useobject{currentmarker}{}%
\end{pgfscope}%
\begin{pgfscope}%
\pgfsys@transformshift{1.400911in}{0.670921in}%
\pgfsys@useobject{currentmarker}{}%
\end{pgfscope}%
\begin{pgfscope}%
\pgfsys@transformshift{1.480973in}{0.671593in}%
\pgfsys@useobject{currentmarker}{}%
\end{pgfscope}%
\begin{pgfscope}%
\pgfsys@transformshift{1.561035in}{0.672077in}%
\pgfsys@useobject{currentmarker}{}%
\end{pgfscope}%
\begin{pgfscope}%
\pgfsys@transformshift{1.641097in}{0.672227in}%
\pgfsys@useobject{currentmarker}{}%
\end{pgfscope}%
\begin{pgfscope}%
\pgfsys@transformshift{1.721159in}{0.673004in}%
\pgfsys@useobject{currentmarker}{}%
\end{pgfscope}%
\begin{pgfscope}%
\pgfsys@transformshift{1.801221in}{0.673616in}%
\pgfsys@useobject{currentmarker}{}%
\end{pgfscope}%
\begin{pgfscope}%
\pgfsys@transformshift{1.881283in}{0.674351in}%
\pgfsys@useobject{currentmarker}{}%
\end{pgfscope}%
\begin{pgfscope}%
\pgfsys@transformshift{1.961345in}{0.675458in}%
\pgfsys@useobject{currentmarker}{}%
\end{pgfscope}%
\begin{pgfscope}%
\pgfsys@transformshift{2.041407in}{0.676207in}%
\pgfsys@useobject{currentmarker}{}%
\end{pgfscope}%
\begin{pgfscope}%
\pgfsys@transformshift{2.121469in}{0.677151in}%
\pgfsys@useobject{currentmarker}{}%
\end{pgfscope}%
\begin{pgfscope}%
\pgfsys@transformshift{2.201531in}{0.680067in}%
\pgfsys@useobject{currentmarker}{}%
\end{pgfscope}%
\begin{pgfscope}%
\pgfsys@transformshift{2.281593in}{0.682179in}%
\pgfsys@useobject{currentmarker}{}%
\end{pgfscope}%
\begin{pgfscope}%
\pgfsys@transformshift{2.361655in}{0.688698in}%
\pgfsys@useobject{currentmarker}{}%
\end{pgfscope}%
\begin{pgfscope}%
\pgfsys@transformshift{2.441717in}{0.697883in}%
\pgfsys@useobject{currentmarker}{}%
\end{pgfscope}%
\begin{pgfscope}%
\pgfsys@transformshift{2.521779in}{0.717923in}%
\pgfsys@useobject{currentmarker}{}%
\end{pgfscope}%
\begin{pgfscope}%
\pgfsys@transformshift{2.601841in}{0.849337in}%
\pgfsys@useobject{currentmarker}{}%
\end{pgfscope}%
\begin{pgfscope}%
\pgfsys@transformshift{2.681903in}{0.996383in}%
\pgfsys@useobject{currentmarker}{}%
\end{pgfscope}%
\begin{pgfscope}%
\pgfsys@transformshift{2.761965in}{1.141953in}%
\pgfsys@useobject{currentmarker}{}%
\end{pgfscope}%
\begin{pgfscope}%
\pgfsys@transformshift{2.842027in}{1.265157in}%
\pgfsys@useobject{currentmarker}{}%
\end{pgfscope}%
\begin{pgfscope}%
\pgfsys@transformshift{2.922089in}{1.364012in}%
\pgfsys@useobject{currentmarker}{}%
\end{pgfscope}%
\begin{pgfscope}%
\pgfsys@transformshift{3.002151in}{1.426082in}%
\pgfsys@useobject{currentmarker}{}%
\end{pgfscope}%
\begin{pgfscope}%
\pgfsys@transformshift{3.082213in}{1.497729in}%
\pgfsys@useobject{currentmarker}{}%
\end{pgfscope}%
\begin{pgfscope}%
\pgfsys@transformshift{3.162275in}{1.559157in}%
\pgfsys@useobject{currentmarker}{}%
\end{pgfscope}%
\begin{pgfscope}%
\pgfsys@transformshift{3.242337in}{1.615742in}%
\pgfsys@useobject{currentmarker}{}%
\end{pgfscope}%
\begin{pgfscope}%
\pgfsys@transformshift{3.322399in}{1.665134in}%
\pgfsys@useobject{currentmarker}{}%
\end{pgfscope}%
\begin{pgfscope}%
\pgfsys@transformshift{3.402461in}{1.709916in}%
\pgfsys@useobject{currentmarker}{}%
\end{pgfscope}%
\begin{pgfscope}%
\pgfsys@transformshift{3.482523in}{1.745968in}%
\pgfsys@useobject{currentmarker}{}%
\end{pgfscope}%
\begin{pgfscope}%
\pgfsys@transformshift{3.562585in}{1.780846in}%
\pgfsys@useobject{currentmarker}{}%
\end{pgfscope}%
\begin{pgfscope}%
\pgfsys@transformshift{3.642647in}{1.823161in}%
\pgfsys@useobject{currentmarker}{}%
\end{pgfscope}%
\begin{pgfscope}%
\pgfsys@transformshift{3.722709in}{1.852479in}%
\pgfsys@useobject{currentmarker}{}%
\end{pgfscope}%
\begin{pgfscope}%
\pgfsys@transformshift{3.802771in}{1.885040in}%
\pgfsys@useobject{currentmarker}{}%
\end{pgfscope}%
\begin{pgfscope}%
\pgfsys@transformshift{3.882833in}{1.907383in}%
\pgfsys@useobject{currentmarker}{}%
\end{pgfscope}%
\begin{pgfscope}%
\pgfsys@transformshift{3.962895in}{1.936885in}%
\pgfsys@useobject{currentmarker}{}%
\end{pgfscope}%
\begin{pgfscope}%
\pgfsys@transformshift{4.042957in}{1.961004in}%
\pgfsys@useobject{currentmarker}{}%
\end{pgfscope}%
\begin{pgfscope}%
\pgfsys@transformshift{4.123019in}{1.987746in}%
\pgfsys@useobject{currentmarker}{}%
\end{pgfscope}%
\begin{pgfscope}%
\pgfsys@transformshift{4.203081in}{2.007691in}%
\pgfsys@useobject{currentmarker}{}%
\end{pgfscope}%
\begin{pgfscope}%
\pgfsys@transformshift{4.283143in}{2.028477in}%
\pgfsys@useobject{currentmarker}{}%
\end{pgfscope}%
\begin{pgfscope}%
\pgfsys@transformshift{4.363205in}{2.048504in}%
\pgfsys@useobject{currentmarker}{}%
\end{pgfscope}%
\begin{pgfscope}%
\pgfsys@transformshift{4.443267in}{2.067036in}%
\pgfsys@useobject{currentmarker}{}%
\end{pgfscope}%
\begin{pgfscope}%
\pgfsys@transformshift{4.523329in}{2.085834in}%
\pgfsys@useobject{currentmarker}{}%
\end{pgfscope}%
\begin{pgfscope}%
\pgfsys@transformshift{4.603391in}{2.101927in}%
\pgfsys@useobject{currentmarker}{}%
\end{pgfscope}%
\begin{pgfscope}%
\pgfsys@transformshift{4.683453in}{2.114926in}%
\pgfsys@useobject{currentmarker}{}%
\end{pgfscope}%
\begin{pgfscope}%
\pgfsys@transformshift{4.763515in}{2.132269in}%
\pgfsys@useobject{currentmarker}{}%
\end{pgfscope}%
\begin{pgfscope}%
\pgfsys@transformshift{4.843577in}{2.148362in}%
\pgfsys@useobject{currentmarker}{}%
\end{pgfscope}%
\end{pgfscope}%
\begin{pgfscope}%
\pgfpathrectangle{\pgfqpoint{0.640323in}{0.527436in}}{\pgfqpoint{4.403409in}{4.235000in}}%
\pgfusepath{clip}%
\pgfsetbuttcap%
\pgfsetroundjoin%
\definecolor{currentfill}{rgb}{0.000000,0.000000,0.000000}%
\pgfsetfillcolor{currentfill}%
\pgfsetfillopacity{0.500000}%
\pgfsetlinewidth{1.003750pt}%
\definecolor{currentstroke}{rgb}{0.000000,0.000000,0.000000}%
\pgfsetstrokecolor{currentstroke}%
\pgfsetstrokeopacity{0.500000}%
\pgfsetdash{}{0pt}%
\pgfsys@defobject{currentmarker}{\pgfqpoint{-0.021960in}{-0.021960in}}{\pgfqpoint{0.021960in}{0.021960in}}{%
\pgfpathmoveto{\pgfqpoint{0.000000in}{-0.021960in}}%
\pgfpathcurveto{\pgfqpoint{0.005824in}{-0.021960in}}{\pgfqpoint{0.011410in}{-0.019646in}}{\pgfqpoint{0.015528in}{-0.015528in}}%
\pgfpathcurveto{\pgfqpoint{0.019646in}{-0.011410in}}{\pgfqpoint{0.021960in}{-0.005824in}}{\pgfqpoint{0.021960in}{0.000000in}}%
\pgfpathcurveto{\pgfqpoint{0.021960in}{0.005824in}}{\pgfqpoint{0.019646in}{0.011410in}}{\pgfqpoint{0.015528in}{0.015528in}}%
\pgfpathcurveto{\pgfqpoint{0.011410in}{0.019646in}}{\pgfqpoint{0.005824in}{0.021960in}}{\pgfqpoint{0.000000in}{0.021960in}}%
\pgfpathcurveto{\pgfqpoint{-0.005824in}{0.021960in}}{\pgfqpoint{-0.011410in}{0.019646in}}{\pgfqpoint{-0.015528in}{0.015528in}}%
\pgfpathcurveto{\pgfqpoint{-0.019646in}{0.011410in}}{\pgfqpoint{-0.021960in}{0.005824in}}{\pgfqpoint{-0.021960in}{0.000000in}}%
\pgfpathcurveto{\pgfqpoint{-0.021960in}{-0.005824in}}{\pgfqpoint{-0.019646in}{-0.011410in}}{\pgfqpoint{-0.015528in}{-0.015528in}}%
\pgfpathcurveto{\pgfqpoint{-0.011410in}{-0.019646in}}{\pgfqpoint{-0.005824in}{-0.021960in}}{\pgfqpoint{0.000000in}{-0.021960in}}%
\pgfpathlineto{\pgfqpoint{0.000000in}{-0.021960in}}%
\pgfpathclose%
\pgfusepath{stroke,fill}%
}%
\begin{pgfscope}%
\pgfsys@transformshift{0.840477in}{0.672709in}%
\pgfsys@useobject{currentmarker}{}%
\end{pgfscope}%
\begin{pgfscope}%
\pgfsys@transformshift{0.920539in}{0.674121in}%
\pgfsys@useobject{currentmarker}{}%
\end{pgfscope}%
\begin{pgfscope}%
\pgfsys@transformshift{1.000601in}{0.675713in}%
\pgfsys@useobject{currentmarker}{}%
\end{pgfscope}%
\begin{pgfscope}%
\pgfsys@transformshift{1.080663in}{0.677454in}%
\pgfsys@useobject{currentmarker}{}%
\end{pgfscope}%
\begin{pgfscope}%
\pgfsys@transformshift{1.160725in}{0.681242in}%
\pgfsys@useobject{currentmarker}{}%
\end{pgfscope}%
\begin{pgfscope}%
\pgfsys@transformshift{1.240787in}{0.686755in}%
\pgfsys@useobject{currentmarker}{}%
\end{pgfscope}%
\begin{pgfscope}%
\pgfsys@transformshift{1.320849in}{0.695784in}%
\pgfsys@useobject{currentmarker}{}%
\end{pgfscope}%
\begin{pgfscope}%
\pgfsys@transformshift{1.400911in}{0.717425in}%
\pgfsys@useobject{currentmarker}{}%
\end{pgfscope}%
\begin{pgfscope}%
\pgfsys@transformshift{1.480973in}{0.869198in}%
\pgfsys@useobject{currentmarker}{}%
\end{pgfscope}%
\begin{pgfscope}%
\pgfsys@transformshift{1.561035in}{1.547600in}%
\pgfsys@useobject{currentmarker}{}%
\end{pgfscope}%
\begin{pgfscope}%
\pgfsys@transformshift{1.641097in}{1.979720in}%
\pgfsys@useobject{currentmarker}{}%
\end{pgfscope}%
\begin{pgfscope}%
\pgfsys@transformshift{1.721159in}{2.285733in}%
\pgfsys@useobject{currentmarker}{}%
\end{pgfscope}%
\begin{pgfscope}%
\pgfsys@transformshift{1.801221in}{2.517012in}%
\pgfsys@useobject{currentmarker}{}%
\end{pgfscope}%
\begin{pgfscope}%
\pgfsys@transformshift{1.881283in}{2.701917in}%
\pgfsys@useobject{currentmarker}{}%
\end{pgfscope}%
\begin{pgfscope}%
\pgfsys@transformshift{1.961345in}{2.863831in}%
\pgfsys@useobject{currentmarker}{}%
\end{pgfscope}%
\begin{pgfscope}%
\pgfsys@transformshift{2.041407in}{2.996946in}%
\pgfsys@useobject{currentmarker}{}%
\end{pgfscope}%
\begin{pgfscope}%
\pgfsys@transformshift{2.121469in}{3.110519in}%
\pgfsys@useobject{currentmarker}{}%
\end{pgfscope}%
\begin{pgfscope}%
\pgfsys@transformshift{2.201531in}{3.216920in}%
\pgfsys@useobject{currentmarker}{}%
\end{pgfscope}%
\begin{pgfscope}%
\pgfsys@transformshift{2.281593in}{3.312146in}%
\pgfsys@useobject{currentmarker}{}%
\end{pgfscope}%
\begin{pgfscope}%
\pgfsys@transformshift{2.361655in}{3.395057in}%
\pgfsys@useobject{currentmarker}{}%
\end{pgfscope}%
\begin{pgfscope}%
\pgfsys@transformshift{2.441717in}{3.471184in}%
\pgfsys@useobject{currentmarker}{}%
\end{pgfscope}%
\begin{pgfscope}%
\pgfsys@transformshift{2.521779in}{3.539586in}%
\pgfsys@useobject{currentmarker}{}%
\end{pgfscope}%
\begin{pgfscope}%
\pgfsys@transformshift{2.601841in}{3.601049in}%
\pgfsys@useobject{currentmarker}{}%
\end{pgfscope}%
\begin{pgfscope}%
\pgfsys@transformshift{2.681903in}{3.658590in}%
\pgfsys@useobject{currentmarker}{}%
\end{pgfscope}%
\begin{pgfscope}%
\pgfsys@transformshift{2.761965in}{3.711705in}%
\pgfsys@useobject{currentmarker}{}%
\end{pgfscope}%
\begin{pgfscope}%
\pgfsys@transformshift{2.842027in}{3.762675in}%
\pgfsys@useobject{currentmarker}{}%
\end{pgfscope}%
\begin{pgfscope}%
\pgfsys@transformshift{2.922089in}{3.804834in}%
\pgfsys@useobject{currentmarker}{}%
\end{pgfscope}%
\begin{pgfscope}%
\pgfsys@transformshift{3.002151in}{3.853441in}%
\pgfsys@useobject{currentmarker}{}%
\end{pgfscope}%
\begin{pgfscope}%
\pgfsys@transformshift{3.082213in}{3.894780in}%
\pgfsys@useobject{currentmarker}{}%
\end{pgfscope}%
\begin{pgfscope}%
\pgfsys@transformshift{3.162275in}{3.930306in}%
\pgfsys@useobject{currentmarker}{}%
\end{pgfscope}%
\begin{pgfscope}%
\pgfsys@transformshift{3.242337in}{3.966099in}%
\pgfsys@useobject{currentmarker}{}%
\end{pgfscope}%
\begin{pgfscope}%
\pgfsys@transformshift{3.322399in}{3.999063in}%
\pgfsys@useobject{currentmarker}{}%
\end{pgfscope}%
\begin{pgfscope}%
\pgfsys@transformshift{3.402461in}{4.029801in}%
\pgfsys@useobject{currentmarker}{}%
\end{pgfscope}%
\begin{pgfscope}%
\pgfsys@transformshift{3.482523in}{4.061052in}%
\pgfsys@useobject{currentmarker}{}%
\end{pgfscope}%
\begin{pgfscope}%
\pgfsys@transformshift{3.562585in}{4.091434in}%
\pgfsys@useobject{currentmarker}{}%
\end{pgfscope}%
\begin{pgfscope}%
\pgfsys@transformshift{3.642647in}{4.116960in}%
\pgfsys@useobject{currentmarker}{}%
\end{pgfscope}%
\begin{pgfscope}%
\pgfsys@transformshift{3.722709in}{4.140547in}%
\pgfsys@useobject{currentmarker}{}%
\end{pgfscope}%
\begin{pgfscope}%
\pgfsys@transformshift{3.802771in}{4.169365in}%
\pgfsys@useobject{currentmarker}{}%
\end{pgfscope}%
\begin{pgfscope}%
\pgfsys@transformshift{3.882833in}{4.192849in}%
\pgfsys@useobject{currentmarker}{}%
\end{pgfscope}%
\begin{pgfscope}%
\pgfsys@transformshift{3.962895in}{4.212733in}%
\pgfsys@useobject{currentmarker}{}%
\end{pgfscope}%
\begin{pgfscope}%
\pgfsys@transformshift{4.042957in}{4.237501in}%
\pgfsys@useobject{currentmarker}{}%
\end{pgfscope}%
\begin{pgfscope}%
\pgfsys@transformshift{4.123019in}{4.255581in}%
\pgfsys@useobject{currentmarker}{}%
\end{pgfscope}%
\begin{pgfscope}%
\pgfsys@transformshift{4.203081in}{4.277153in}%
\pgfsys@useobject{currentmarker}{}%
\end{pgfscope}%
\begin{pgfscope}%
\pgfsys@transformshift{4.283143in}{4.294660in}%
\pgfsys@useobject{currentmarker}{}%
\end{pgfscope}%
\begin{pgfscope}%
\pgfsys@transformshift{4.363205in}{4.311702in}%
\pgfsys@useobject{currentmarker}{}%
\end{pgfscope}%
\begin{pgfscope}%
\pgfsys@transformshift{4.443267in}{4.329359in}%
\pgfsys@useobject{currentmarker}{}%
\end{pgfscope}%
\begin{pgfscope}%
\pgfsys@transformshift{4.523329in}{4.344510in}%
\pgfsys@useobject{currentmarker}{}%
\end{pgfscope}%
\begin{pgfscope}%
\pgfsys@transformshift{4.603391in}{4.363321in}%
\pgfsys@useobject{currentmarker}{}%
\end{pgfscope}%
\begin{pgfscope}%
\pgfsys@transformshift{4.683453in}{4.379947in}%
\pgfsys@useobject{currentmarker}{}%
\end{pgfscope}%
\begin{pgfscope}%
\pgfsys@transformshift{4.763515in}{4.391867in}%
\pgfsys@useobject{currentmarker}{}%
\end{pgfscope}%
\begin{pgfscope}%
\pgfsys@transformshift{4.843577in}{4.408547in}%
\pgfsys@useobject{currentmarker}{}%
\end{pgfscope}%
\end{pgfscope}%
\begin{pgfscope}%
\pgfpathrectangle{\pgfqpoint{0.640323in}{0.527436in}}{\pgfqpoint{4.403409in}{4.235000in}}%
\pgfusepath{clip}%
\pgfsetrectcap%
\pgfsetroundjoin%
\pgfsetlinewidth{0.803000pt}%
\definecolor{currentstroke}{rgb}{0.690196,0.690196,0.690196}%
\pgfsetstrokecolor{currentstroke}%
\pgfsetdash{}{0pt}%
\pgfpathmoveto{\pgfqpoint{0.840477in}{0.527436in}}%
\pgfpathlineto{\pgfqpoint{0.840477in}{4.762436in}}%
\pgfusepath{stroke}%
\end{pgfscope}%
\begin{pgfscope}%
\pgfsetbuttcap%
\pgfsetroundjoin%
\definecolor{currentfill}{rgb}{0.000000,0.000000,0.000000}%
\pgfsetfillcolor{currentfill}%
\pgfsetlinewidth{0.803000pt}%
\definecolor{currentstroke}{rgb}{0.000000,0.000000,0.000000}%
\pgfsetstrokecolor{currentstroke}%
\pgfsetdash{}{0pt}%
\pgfsys@defobject{currentmarker}{\pgfqpoint{0.000000in}{-0.048611in}}{\pgfqpoint{0.000000in}{0.000000in}}{%
\pgfpathmoveto{\pgfqpoint{0.000000in}{0.000000in}}%
\pgfpathlineto{\pgfqpoint{0.000000in}{-0.048611in}}%
\pgfusepath{stroke,fill}%
}%
\begin{pgfscope}%
\pgfsys@transformshift{0.840477in}{0.527436in}%
\pgfsys@useobject{currentmarker}{}%
\end{pgfscope}%
\end{pgfscope}%
\begin{pgfscope}%
\definecolor{textcolor}{rgb}{0.000000,0.000000,0.000000}%
\pgfsetstrokecolor{textcolor}%
\pgfsetfillcolor{textcolor}%
\pgftext[x=0.840477in,y=0.430214in,,top]{\color{textcolor}\sffamily\fontsize{10.000000}{12.000000}\selectfont 0.0}%
\end{pgfscope}%
\begin{pgfscope}%
\pgfpathrectangle{\pgfqpoint{0.640323in}{0.527436in}}{\pgfqpoint{4.403409in}{4.235000in}}%
\pgfusepath{clip}%
\pgfsetrectcap%
\pgfsetroundjoin%
\pgfsetlinewidth{0.803000pt}%
\definecolor{currentstroke}{rgb}{0.690196,0.690196,0.690196}%
\pgfsetstrokecolor{currentstroke}%
\pgfsetdash{}{0pt}%
\pgfpathmoveto{\pgfqpoint{1.641097in}{0.527436in}}%
\pgfpathlineto{\pgfqpoint{1.641097in}{4.762436in}}%
\pgfusepath{stroke}%
\end{pgfscope}%
\begin{pgfscope}%
\pgfsetbuttcap%
\pgfsetroundjoin%
\definecolor{currentfill}{rgb}{0.000000,0.000000,0.000000}%
\pgfsetfillcolor{currentfill}%
\pgfsetlinewidth{0.803000pt}%
\definecolor{currentstroke}{rgb}{0.000000,0.000000,0.000000}%
\pgfsetstrokecolor{currentstroke}%
\pgfsetdash{}{0pt}%
\pgfsys@defobject{currentmarker}{\pgfqpoint{0.000000in}{-0.048611in}}{\pgfqpoint{0.000000in}{0.000000in}}{%
\pgfpathmoveto{\pgfqpoint{0.000000in}{0.000000in}}%
\pgfpathlineto{\pgfqpoint{0.000000in}{-0.048611in}}%
\pgfusepath{stroke,fill}%
}%
\begin{pgfscope}%
\pgfsys@transformshift{1.641097in}{0.527436in}%
\pgfsys@useobject{currentmarker}{}%
\end{pgfscope}%
\end{pgfscope}%
\begin{pgfscope}%
\definecolor{textcolor}{rgb}{0.000000,0.000000,0.000000}%
\pgfsetstrokecolor{textcolor}%
\pgfsetfillcolor{textcolor}%
\pgftext[x=1.641097in,y=0.430214in,,top]{\color{textcolor}\sffamily\fontsize{10.000000}{12.000000}\selectfont 0.2}%
\end{pgfscope}%
\begin{pgfscope}%
\pgfpathrectangle{\pgfqpoint{0.640323in}{0.527436in}}{\pgfqpoint{4.403409in}{4.235000in}}%
\pgfusepath{clip}%
\pgfsetrectcap%
\pgfsetroundjoin%
\pgfsetlinewidth{0.803000pt}%
\definecolor{currentstroke}{rgb}{0.690196,0.690196,0.690196}%
\pgfsetstrokecolor{currentstroke}%
\pgfsetdash{}{0pt}%
\pgfpathmoveto{\pgfqpoint{2.441717in}{0.527436in}}%
\pgfpathlineto{\pgfqpoint{2.441717in}{4.762436in}}%
\pgfusepath{stroke}%
\end{pgfscope}%
\begin{pgfscope}%
\pgfsetbuttcap%
\pgfsetroundjoin%
\definecolor{currentfill}{rgb}{0.000000,0.000000,0.000000}%
\pgfsetfillcolor{currentfill}%
\pgfsetlinewidth{0.803000pt}%
\definecolor{currentstroke}{rgb}{0.000000,0.000000,0.000000}%
\pgfsetstrokecolor{currentstroke}%
\pgfsetdash{}{0pt}%
\pgfsys@defobject{currentmarker}{\pgfqpoint{0.000000in}{-0.048611in}}{\pgfqpoint{0.000000in}{0.000000in}}{%
\pgfpathmoveto{\pgfqpoint{0.000000in}{0.000000in}}%
\pgfpathlineto{\pgfqpoint{0.000000in}{-0.048611in}}%
\pgfusepath{stroke,fill}%
}%
\begin{pgfscope}%
\pgfsys@transformshift{2.441717in}{0.527436in}%
\pgfsys@useobject{currentmarker}{}%
\end{pgfscope}%
\end{pgfscope}%
\begin{pgfscope}%
\definecolor{textcolor}{rgb}{0.000000,0.000000,0.000000}%
\pgfsetstrokecolor{textcolor}%
\pgfsetfillcolor{textcolor}%
\pgftext[x=2.441717in,y=0.430214in,,top]{\color{textcolor}\sffamily\fontsize{10.000000}{12.000000}\selectfont 0.4}%
\end{pgfscope}%
\begin{pgfscope}%
\pgfpathrectangle{\pgfqpoint{0.640323in}{0.527436in}}{\pgfqpoint{4.403409in}{4.235000in}}%
\pgfusepath{clip}%
\pgfsetrectcap%
\pgfsetroundjoin%
\pgfsetlinewidth{0.803000pt}%
\definecolor{currentstroke}{rgb}{0.690196,0.690196,0.690196}%
\pgfsetstrokecolor{currentstroke}%
\pgfsetdash{}{0pt}%
\pgfpathmoveto{\pgfqpoint{3.242337in}{0.527436in}}%
\pgfpathlineto{\pgfqpoint{3.242337in}{4.762436in}}%
\pgfusepath{stroke}%
\end{pgfscope}%
\begin{pgfscope}%
\pgfsetbuttcap%
\pgfsetroundjoin%
\definecolor{currentfill}{rgb}{0.000000,0.000000,0.000000}%
\pgfsetfillcolor{currentfill}%
\pgfsetlinewidth{0.803000pt}%
\definecolor{currentstroke}{rgb}{0.000000,0.000000,0.000000}%
\pgfsetstrokecolor{currentstroke}%
\pgfsetdash{}{0pt}%
\pgfsys@defobject{currentmarker}{\pgfqpoint{0.000000in}{-0.048611in}}{\pgfqpoint{0.000000in}{0.000000in}}{%
\pgfpathmoveto{\pgfqpoint{0.000000in}{0.000000in}}%
\pgfpathlineto{\pgfqpoint{0.000000in}{-0.048611in}}%
\pgfusepath{stroke,fill}%
}%
\begin{pgfscope}%
\pgfsys@transformshift{3.242337in}{0.527436in}%
\pgfsys@useobject{currentmarker}{}%
\end{pgfscope}%
\end{pgfscope}%
\begin{pgfscope}%
\definecolor{textcolor}{rgb}{0.000000,0.000000,0.000000}%
\pgfsetstrokecolor{textcolor}%
\pgfsetfillcolor{textcolor}%
\pgftext[x=3.242337in,y=0.430214in,,top]{\color{textcolor}\sffamily\fontsize{10.000000}{12.000000}\selectfont 0.6}%
\end{pgfscope}%
\begin{pgfscope}%
\pgfpathrectangle{\pgfqpoint{0.640323in}{0.527436in}}{\pgfqpoint{4.403409in}{4.235000in}}%
\pgfusepath{clip}%
\pgfsetrectcap%
\pgfsetroundjoin%
\pgfsetlinewidth{0.803000pt}%
\definecolor{currentstroke}{rgb}{0.690196,0.690196,0.690196}%
\pgfsetstrokecolor{currentstroke}%
\pgfsetdash{}{0pt}%
\pgfpathmoveto{\pgfqpoint{4.042957in}{0.527436in}}%
\pgfpathlineto{\pgfqpoint{4.042957in}{4.762436in}}%
\pgfusepath{stroke}%
\end{pgfscope}%
\begin{pgfscope}%
\pgfsetbuttcap%
\pgfsetroundjoin%
\definecolor{currentfill}{rgb}{0.000000,0.000000,0.000000}%
\pgfsetfillcolor{currentfill}%
\pgfsetlinewidth{0.803000pt}%
\definecolor{currentstroke}{rgb}{0.000000,0.000000,0.000000}%
\pgfsetstrokecolor{currentstroke}%
\pgfsetdash{}{0pt}%
\pgfsys@defobject{currentmarker}{\pgfqpoint{0.000000in}{-0.048611in}}{\pgfqpoint{0.000000in}{0.000000in}}{%
\pgfpathmoveto{\pgfqpoint{0.000000in}{0.000000in}}%
\pgfpathlineto{\pgfqpoint{0.000000in}{-0.048611in}}%
\pgfusepath{stroke,fill}%
}%
\begin{pgfscope}%
\pgfsys@transformshift{4.042957in}{0.527436in}%
\pgfsys@useobject{currentmarker}{}%
\end{pgfscope}%
\end{pgfscope}%
\begin{pgfscope}%
\definecolor{textcolor}{rgb}{0.000000,0.000000,0.000000}%
\pgfsetstrokecolor{textcolor}%
\pgfsetfillcolor{textcolor}%
\pgftext[x=4.042957in,y=0.430214in,,top]{\color{textcolor}\sffamily\fontsize{10.000000}{12.000000}\selectfont 0.8}%
\end{pgfscope}%
\begin{pgfscope}%
\pgfpathrectangle{\pgfqpoint{0.640323in}{0.527436in}}{\pgfqpoint{4.403409in}{4.235000in}}%
\pgfusepath{clip}%
\pgfsetrectcap%
\pgfsetroundjoin%
\pgfsetlinewidth{0.803000pt}%
\definecolor{currentstroke}{rgb}{0.690196,0.690196,0.690196}%
\pgfsetstrokecolor{currentstroke}%
\pgfsetdash{}{0pt}%
\pgfpathmoveto{\pgfqpoint{4.843577in}{0.527436in}}%
\pgfpathlineto{\pgfqpoint{4.843577in}{4.762436in}}%
\pgfusepath{stroke}%
\end{pgfscope}%
\begin{pgfscope}%
\pgfsetbuttcap%
\pgfsetroundjoin%
\definecolor{currentfill}{rgb}{0.000000,0.000000,0.000000}%
\pgfsetfillcolor{currentfill}%
\pgfsetlinewidth{0.803000pt}%
\definecolor{currentstroke}{rgb}{0.000000,0.000000,0.000000}%
\pgfsetstrokecolor{currentstroke}%
\pgfsetdash{}{0pt}%
\pgfsys@defobject{currentmarker}{\pgfqpoint{0.000000in}{-0.048611in}}{\pgfqpoint{0.000000in}{0.000000in}}{%
\pgfpathmoveto{\pgfqpoint{0.000000in}{0.000000in}}%
\pgfpathlineto{\pgfqpoint{0.000000in}{-0.048611in}}%
\pgfusepath{stroke,fill}%
}%
\begin{pgfscope}%
\pgfsys@transformshift{4.843577in}{0.527436in}%
\pgfsys@useobject{currentmarker}{}%
\end{pgfscope}%
\end{pgfscope}%
\begin{pgfscope}%
\definecolor{textcolor}{rgb}{0.000000,0.000000,0.000000}%
\pgfsetstrokecolor{textcolor}%
\pgfsetfillcolor{textcolor}%
\pgftext[x=4.843577in,y=0.430214in,,top]{\color{textcolor}\sffamily\fontsize{10.000000}{12.000000}\selectfont 1.0}%
\end{pgfscope}%
\begin{pgfscope}%
\pgfpathrectangle{\pgfqpoint{0.640323in}{0.527436in}}{\pgfqpoint{4.403409in}{4.235000in}}%
\pgfusepath{clip}%
\pgfsetrectcap%
\pgfsetroundjoin%
\pgfsetlinewidth{0.803000pt}%
\definecolor{currentstroke}{rgb}{0.600000,0.600000,0.600000}%
\pgfsetstrokecolor{currentstroke}%
\pgfsetstrokeopacity{0.200000}%
\pgfsetdash{}{0pt}%
\pgfpathmoveto{\pgfqpoint{1.040632in}{0.527436in}}%
\pgfpathlineto{\pgfqpoint{1.040632in}{4.762436in}}%
\pgfusepath{stroke}%
\end{pgfscope}%
\begin{pgfscope}%
\pgfsetbuttcap%
\pgfsetroundjoin%
\definecolor{currentfill}{rgb}{0.000000,0.000000,0.000000}%
\pgfsetfillcolor{currentfill}%
\pgfsetlinewidth{0.602250pt}%
\definecolor{currentstroke}{rgb}{0.000000,0.000000,0.000000}%
\pgfsetstrokecolor{currentstroke}%
\pgfsetdash{}{0pt}%
\pgfsys@defobject{currentmarker}{\pgfqpoint{0.000000in}{-0.027778in}}{\pgfqpoint{0.000000in}{0.000000in}}{%
\pgfpathmoveto{\pgfqpoint{0.000000in}{0.000000in}}%
\pgfpathlineto{\pgfqpoint{0.000000in}{-0.027778in}}%
\pgfusepath{stroke,fill}%
}%
\begin{pgfscope}%
\pgfsys@transformshift{1.040632in}{0.527436in}%
\pgfsys@useobject{currentmarker}{}%
\end{pgfscope}%
\end{pgfscope}%
\begin{pgfscope}%
\pgfpathrectangle{\pgfqpoint{0.640323in}{0.527436in}}{\pgfqpoint{4.403409in}{4.235000in}}%
\pgfusepath{clip}%
\pgfsetrectcap%
\pgfsetroundjoin%
\pgfsetlinewidth{0.803000pt}%
\definecolor{currentstroke}{rgb}{0.600000,0.600000,0.600000}%
\pgfsetstrokecolor{currentstroke}%
\pgfsetstrokeopacity{0.200000}%
\pgfsetdash{}{0pt}%
\pgfpathmoveto{\pgfqpoint{1.240787in}{0.527436in}}%
\pgfpathlineto{\pgfqpoint{1.240787in}{4.762436in}}%
\pgfusepath{stroke}%
\end{pgfscope}%
\begin{pgfscope}%
\pgfsetbuttcap%
\pgfsetroundjoin%
\definecolor{currentfill}{rgb}{0.000000,0.000000,0.000000}%
\pgfsetfillcolor{currentfill}%
\pgfsetlinewidth{0.602250pt}%
\definecolor{currentstroke}{rgb}{0.000000,0.000000,0.000000}%
\pgfsetstrokecolor{currentstroke}%
\pgfsetdash{}{0pt}%
\pgfsys@defobject{currentmarker}{\pgfqpoint{0.000000in}{-0.027778in}}{\pgfqpoint{0.000000in}{0.000000in}}{%
\pgfpathmoveto{\pgfqpoint{0.000000in}{0.000000in}}%
\pgfpathlineto{\pgfqpoint{0.000000in}{-0.027778in}}%
\pgfusepath{stroke,fill}%
}%
\begin{pgfscope}%
\pgfsys@transformshift{1.240787in}{0.527436in}%
\pgfsys@useobject{currentmarker}{}%
\end{pgfscope}%
\end{pgfscope}%
\begin{pgfscope}%
\pgfpathrectangle{\pgfqpoint{0.640323in}{0.527436in}}{\pgfqpoint{4.403409in}{4.235000in}}%
\pgfusepath{clip}%
\pgfsetrectcap%
\pgfsetroundjoin%
\pgfsetlinewidth{0.803000pt}%
\definecolor{currentstroke}{rgb}{0.600000,0.600000,0.600000}%
\pgfsetstrokecolor{currentstroke}%
\pgfsetstrokeopacity{0.200000}%
\pgfsetdash{}{0pt}%
\pgfpathmoveto{\pgfqpoint{1.440942in}{0.527436in}}%
\pgfpathlineto{\pgfqpoint{1.440942in}{4.762436in}}%
\pgfusepath{stroke}%
\end{pgfscope}%
\begin{pgfscope}%
\pgfsetbuttcap%
\pgfsetroundjoin%
\definecolor{currentfill}{rgb}{0.000000,0.000000,0.000000}%
\pgfsetfillcolor{currentfill}%
\pgfsetlinewidth{0.602250pt}%
\definecolor{currentstroke}{rgb}{0.000000,0.000000,0.000000}%
\pgfsetstrokecolor{currentstroke}%
\pgfsetdash{}{0pt}%
\pgfsys@defobject{currentmarker}{\pgfqpoint{0.000000in}{-0.027778in}}{\pgfqpoint{0.000000in}{0.000000in}}{%
\pgfpathmoveto{\pgfqpoint{0.000000in}{0.000000in}}%
\pgfpathlineto{\pgfqpoint{0.000000in}{-0.027778in}}%
\pgfusepath{stroke,fill}%
}%
\begin{pgfscope}%
\pgfsys@transformshift{1.440942in}{0.527436in}%
\pgfsys@useobject{currentmarker}{}%
\end{pgfscope}%
\end{pgfscope}%
\begin{pgfscope}%
\pgfpathrectangle{\pgfqpoint{0.640323in}{0.527436in}}{\pgfqpoint{4.403409in}{4.235000in}}%
\pgfusepath{clip}%
\pgfsetrectcap%
\pgfsetroundjoin%
\pgfsetlinewidth{0.803000pt}%
\definecolor{currentstroke}{rgb}{0.600000,0.600000,0.600000}%
\pgfsetstrokecolor{currentstroke}%
\pgfsetstrokeopacity{0.200000}%
\pgfsetdash{}{0pt}%
\pgfpathmoveto{\pgfqpoint{1.841252in}{0.527436in}}%
\pgfpathlineto{\pgfqpoint{1.841252in}{4.762436in}}%
\pgfusepath{stroke}%
\end{pgfscope}%
\begin{pgfscope}%
\pgfsetbuttcap%
\pgfsetroundjoin%
\definecolor{currentfill}{rgb}{0.000000,0.000000,0.000000}%
\pgfsetfillcolor{currentfill}%
\pgfsetlinewidth{0.602250pt}%
\definecolor{currentstroke}{rgb}{0.000000,0.000000,0.000000}%
\pgfsetstrokecolor{currentstroke}%
\pgfsetdash{}{0pt}%
\pgfsys@defobject{currentmarker}{\pgfqpoint{0.000000in}{-0.027778in}}{\pgfqpoint{0.000000in}{0.000000in}}{%
\pgfpathmoveto{\pgfqpoint{0.000000in}{0.000000in}}%
\pgfpathlineto{\pgfqpoint{0.000000in}{-0.027778in}}%
\pgfusepath{stroke,fill}%
}%
\begin{pgfscope}%
\pgfsys@transformshift{1.841252in}{0.527436in}%
\pgfsys@useobject{currentmarker}{}%
\end{pgfscope}%
\end{pgfscope}%
\begin{pgfscope}%
\pgfpathrectangle{\pgfqpoint{0.640323in}{0.527436in}}{\pgfqpoint{4.403409in}{4.235000in}}%
\pgfusepath{clip}%
\pgfsetrectcap%
\pgfsetroundjoin%
\pgfsetlinewidth{0.803000pt}%
\definecolor{currentstroke}{rgb}{0.600000,0.600000,0.600000}%
\pgfsetstrokecolor{currentstroke}%
\pgfsetstrokeopacity{0.200000}%
\pgfsetdash{}{0pt}%
\pgfpathmoveto{\pgfqpoint{2.041407in}{0.527436in}}%
\pgfpathlineto{\pgfqpoint{2.041407in}{4.762436in}}%
\pgfusepath{stroke}%
\end{pgfscope}%
\begin{pgfscope}%
\pgfsetbuttcap%
\pgfsetroundjoin%
\definecolor{currentfill}{rgb}{0.000000,0.000000,0.000000}%
\pgfsetfillcolor{currentfill}%
\pgfsetlinewidth{0.602250pt}%
\definecolor{currentstroke}{rgb}{0.000000,0.000000,0.000000}%
\pgfsetstrokecolor{currentstroke}%
\pgfsetdash{}{0pt}%
\pgfsys@defobject{currentmarker}{\pgfqpoint{0.000000in}{-0.027778in}}{\pgfqpoint{0.000000in}{0.000000in}}{%
\pgfpathmoveto{\pgfqpoint{0.000000in}{0.000000in}}%
\pgfpathlineto{\pgfqpoint{0.000000in}{-0.027778in}}%
\pgfusepath{stroke,fill}%
}%
\begin{pgfscope}%
\pgfsys@transformshift{2.041407in}{0.527436in}%
\pgfsys@useobject{currentmarker}{}%
\end{pgfscope}%
\end{pgfscope}%
\begin{pgfscope}%
\pgfpathrectangle{\pgfqpoint{0.640323in}{0.527436in}}{\pgfqpoint{4.403409in}{4.235000in}}%
\pgfusepath{clip}%
\pgfsetrectcap%
\pgfsetroundjoin%
\pgfsetlinewidth{0.803000pt}%
\definecolor{currentstroke}{rgb}{0.600000,0.600000,0.600000}%
\pgfsetstrokecolor{currentstroke}%
\pgfsetstrokeopacity{0.200000}%
\pgfsetdash{}{0pt}%
\pgfpathmoveto{\pgfqpoint{2.241562in}{0.527436in}}%
\pgfpathlineto{\pgfqpoint{2.241562in}{4.762436in}}%
\pgfusepath{stroke}%
\end{pgfscope}%
\begin{pgfscope}%
\pgfsetbuttcap%
\pgfsetroundjoin%
\definecolor{currentfill}{rgb}{0.000000,0.000000,0.000000}%
\pgfsetfillcolor{currentfill}%
\pgfsetlinewidth{0.602250pt}%
\definecolor{currentstroke}{rgb}{0.000000,0.000000,0.000000}%
\pgfsetstrokecolor{currentstroke}%
\pgfsetdash{}{0pt}%
\pgfsys@defobject{currentmarker}{\pgfqpoint{0.000000in}{-0.027778in}}{\pgfqpoint{0.000000in}{0.000000in}}{%
\pgfpathmoveto{\pgfqpoint{0.000000in}{0.000000in}}%
\pgfpathlineto{\pgfqpoint{0.000000in}{-0.027778in}}%
\pgfusepath{stroke,fill}%
}%
\begin{pgfscope}%
\pgfsys@transformshift{2.241562in}{0.527436in}%
\pgfsys@useobject{currentmarker}{}%
\end{pgfscope}%
\end{pgfscope}%
\begin{pgfscope}%
\pgfpathrectangle{\pgfqpoint{0.640323in}{0.527436in}}{\pgfqpoint{4.403409in}{4.235000in}}%
\pgfusepath{clip}%
\pgfsetrectcap%
\pgfsetroundjoin%
\pgfsetlinewidth{0.803000pt}%
\definecolor{currentstroke}{rgb}{0.600000,0.600000,0.600000}%
\pgfsetstrokecolor{currentstroke}%
\pgfsetstrokeopacity{0.200000}%
\pgfsetdash{}{0pt}%
\pgfpathmoveto{\pgfqpoint{2.641872in}{0.527436in}}%
\pgfpathlineto{\pgfqpoint{2.641872in}{4.762436in}}%
\pgfusepath{stroke}%
\end{pgfscope}%
\begin{pgfscope}%
\pgfsetbuttcap%
\pgfsetroundjoin%
\definecolor{currentfill}{rgb}{0.000000,0.000000,0.000000}%
\pgfsetfillcolor{currentfill}%
\pgfsetlinewidth{0.602250pt}%
\definecolor{currentstroke}{rgb}{0.000000,0.000000,0.000000}%
\pgfsetstrokecolor{currentstroke}%
\pgfsetdash{}{0pt}%
\pgfsys@defobject{currentmarker}{\pgfqpoint{0.000000in}{-0.027778in}}{\pgfqpoint{0.000000in}{0.000000in}}{%
\pgfpathmoveto{\pgfqpoint{0.000000in}{0.000000in}}%
\pgfpathlineto{\pgfqpoint{0.000000in}{-0.027778in}}%
\pgfusepath{stroke,fill}%
}%
\begin{pgfscope}%
\pgfsys@transformshift{2.641872in}{0.527436in}%
\pgfsys@useobject{currentmarker}{}%
\end{pgfscope}%
\end{pgfscope}%
\begin{pgfscope}%
\pgfpathrectangle{\pgfqpoint{0.640323in}{0.527436in}}{\pgfqpoint{4.403409in}{4.235000in}}%
\pgfusepath{clip}%
\pgfsetrectcap%
\pgfsetroundjoin%
\pgfsetlinewidth{0.803000pt}%
\definecolor{currentstroke}{rgb}{0.600000,0.600000,0.600000}%
\pgfsetstrokecolor{currentstroke}%
\pgfsetstrokeopacity{0.200000}%
\pgfsetdash{}{0pt}%
\pgfpathmoveto{\pgfqpoint{2.842027in}{0.527436in}}%
\pgfpathlineto{\pgfqpoint{2.842027in}{4.762436in}}%
\pgfusepath{stroke}%
\end{pgfscope}%
\begin{pgfscope}%
\pgfsetbuttcap%
\pgfsetroundjoin%
\definecolor{currentfill}{rgb}{0.000000,0.000000,0.000000}%
\pgfsetfillcolor{currentfill}%
\pgfsetlinewidth{0.602250pt}%
\definecolor{currentstroke}{rgb}{0.000000,0.000000,0.000000}%
\pgfsetstrokecolor{currentstroke}%
\pgfsetdash{}{0pt}%
\pgfsys@defobject{currentmarker}{\pgfqpoint{0.000000in}{-0.027778in}}{\pgfqpoint{0.000000in}{0.000000in}}{%
\pgfpathmoveto{\pgfqpoint{0.000000in}{0.000000in}}%
\pgfpathlineto{\pgfqpoint{0.000000in}{-0.027778in}}%
\pgfusepath{stroke,fill}%
}%
\begin{pgfscope}%
\pgfsys@transformshift{2.842027in}{0.527436in}%
\pgfsys@useobject{currentmarker}{}%
\end{pgfscope}%
\end{pgfscope}%
\begin{pgfscope}%
\pgfpathrectangle{\pgfqpoint{0.640323in}{0.527436in}}{\pgfqpoint{4.403409in}{4.235000in}}%
\pgfusepath{clip}%
\pgfsetrectcap%
\pgfsetroundjoin%
\pgfsetlinewidth{0.803000pt}%
\definecolor{currentstroke}{rgb}{0.600000,0.600000,0.600000}%
\pgfsetstrokecolor{currentstroke}%
\pgfsetstrokeopacity{0.200000}%
\pgfsetdash{}{0pt}%
\pgfpathmoveto{\pgfqpoint{3.042182in}{0.527436in}}%
\pgfpathlineto{\pgfqpoint{3.042182in}{4.762436in}}%
\pgfusepath{stroke}%
\end{pgfscope}%
\begin{pgfscope}%
\pgfsetbuttcap%
\pgfsetroundjoin%
\definecolor{currentfill}{rgb}{0.000000,0.000000,0.000000}%
\pgfsetfillcolor{currentfill}%
\pgfsetlinewidth{0.602250pt}%
\definecolor{currentstroke}{rgb}{0.000000,0.000000,0.000000}%
\pgfsetstrokecolor{currentstroke}%
\pgfsetdash{}{0pt}%
\pgfsys@defobject{currentmarker}{\pgfqpoint{0.000000in}{-0.027778in}}{\pgfqpoint{0.000000in}{0.000000in}}{%
\pgfpathmoveto{\pgfqpoint{0.000000in}{0.000000in}}%
\pgfpathlineto{\pgfqpoint{0.000000in}{-0.027778in}}%
\pgfusepath{stroke,fill}%
}%
\begin{pgfscope}%
\pgfsys@transformshift{3.042182in}{0.527436in}%
\pgfsys@useobject{currentmarker}{}%
\end{pgfscope}%
\end{pgfscope}%
\begin{pgfscope}%
\pgfpathrectangle{\pgfqpoint{0.640323in}{0.527436in}}{\pgfqpoint{4.403409in}{4.235000in}}%
\pgfusepath{clip}%
\pgfsetrectcap%
\pgfsetroundjoin%
\pgfsetlinewidth{0.803000pt}%
\definecolor{currentstroke}{rgb}{0.600000,0.600000,0.600000}%
\pgfsetstrokecolor{currentstroke}%
\pgfsetstrokeopacity{0.200000}%
\pgfsetdash{}{0pt}%
\pgfpathmoveto{\pgfqpoint{3.442492in}{0.527436in}}%
\pgfpathlineto{\pgfqpoint{3.442492in}{4.762436in}}%
\pgfusepath{stroke}%
\end{pgfscope}%
\begin{pgfscope}%
\pgfsetbuttcap%
\pgfsetroundjoin%
\definecolor{currentfill}{rgb}{0.000000,0.000000,0.000000}%
\pgfsetfillcolor{currentfill}%
\pgfsetlinewidth{0.602250pt}%
\definecolor{currentstroke}{rgb}{0.000000,0.000000,0.000000}%
\pgfsetstrokecolor{currentstroke}%
\pgfsetdash{}{0pt}%
\pgfsys@defobject{currentmarker}{\pgfqpoint{0.000000in}{-0.027778in}}{\pgfqpoint{0.000000in}{0.000000in}}{%
\pgfpathmoveto{\pgfqpoint{0.000000in}{0.000000in}}%
\pgfpathlineto{\pgfqpoint{0.000000in}{-0.027778in}}%
\pgfusepath{stroke,fill}%
}%
\begin{pgfscope}%
\pgfsys@transformshift{3.442492in}{0.527436in}%
\pgfsys@useobject{currentmarker}{}%
\end{pgfscope}%
\end{pgfscope}%
\begin{pgfscope}%
\pgfpathrectangle{\pgfqpoint{0.640323in}{0.527436in}}{\pgfqpoint{4.403409in}{4.235000in}}%
\pgfusepath{clip}%
\pgfsetrectcap%
\pgfsetroundjoin%
\pgfsetlinewidth{0.803000pt}%
\definecolor{currentstroke}{rgb}{0.600000,0.600000,0.600000}%
\pgfsetstrokecolor{currentstroke}%
\pgfsetstrokeopacity{0.200000}%
\pgfsetdash{}{0pt}%
\pgfpathmoveto{\pgfqpoint{3.642647in}{0.527436in}}%
\pgfpathlineto{\pgfqpoint{3.642647in}{4.762436in}}%
\pgfusepath{stroke}%
\end{pgfscope}%
\begin{pgfscope}%
\pgfsetbuttcap%
\pgfsetroundjoin%
\definecolor{currentfill}{rgb}{0.000000,0.000000,0.000000}%
\pgfsetfillcolor{currentfill}%
\pgfsetlinewidth{0.602250pt}%
\definecolor{currentstroke}{rgb}{0.000000,0.000000,0.000000}%
\pgfsetstrokecolor{currentstroke}%
\pgfsetdash{}{0pt}%
\pgfsys@defobject{currentmarker}{\pgfqpoint{0.000000in}{-0.027778in}}{\pgfqpoint{0.000000in}{0.000000in}}{%
\pgfpathmoveto{\pgfqpoint{0.000000in}{0.000000in}}%
\pgfpathlineto{\pgfqpoint{0.000000in}{-0.027778in}}%
\pgfusepath{stroke,fill}%
}%
\begin{pgfscope}%
\pgfsys@transformshift{3.642647in}{0.527436in}%
\pgfsys@useobject{currentmarker}{}%
\end{pgfscope}%
\end{pgfscope}%
\begin{pgfscope}%
\pgfpathrectangle{\pgfqpoint{0.640323in}{0.527436in}}{\pgfqpoint{4.403409in}{4.235000in}}%
\pgfusepath{clip}%
\pgfsetrectcap%
\pgfsetroundjoin%
\pgfsetlinewidth{0.803000pt}%
\definecolor{currentstroke}{rgb}{0.600000,0.600000,0.600000}%
\pgfsetstrokecolor{currentstroke}%
\pgfsetstrokeopacity{0.200000}%
\pgfsetdash{}{0pt}%
\pgfpathmoveto{\pgfqpoint{3.842802in}{0.527436in}}%
\pgfpathlineto{\pgfqpoint{3.842802in}{4.762436in}}%
\pgfusepath{stroke}%
\end{pgfscope}%
\begin{pgfscope}%
\pgfsetbuttcap%
\pgfsetroundjoin%
\definecolor{currentfill}{rgb}{0.000000,0.000000,0.000000}%
\pgfsetfillcolor{currentfill}%
\pgfsetlinewidth{0.602250pt}%
\definecolor{currentstroke}{rgb}{0.000000,0.000000,0.000000}%
\pgfsetstrokecolor{currentstroke}%
\pgfsetdash{}{0pt}%
\pgfsys@defobject{currentmarker}{\pgfqpoint{0.000000in}{-0.027778in}}{\pgfqpoint{0.000000in}{0.000000in}}{%
\pgfpathmoveto{\pgfqpoint{0.000000in}{0.000000in}}%
\pgfpathlineto{\pgfqpoint{0.000000in}{-0.027778in}}%
\pgfusepath{stroke,fill}%
}%
\begin{pgfscope}%
\pgfsys@transformshift{3.842802in}{0.527436in}%
\pgfsys@useobject{currentmarker}{}%
\end{pgfscope}%
\end{pgfscope}%
\begin{pgfscope}%
\pgfpathrectangle{\pgfqpoint{0.640323in}{0.527436in}}{\pgfqpoint{4.403409in}{4.235000in}}%
\pgfusepath{clip}%
\pgfsetrectcap%
\pgfsetroundjoin%
\pgfsetlinewidth{0.803000pt}%
\definecolor{currentstroke}{rgb}{0.600000,0.600000,0.600000}%
\pgfsetstrokecolor{currentstroke}%
\pgfsetstrokeopacity{0.200000}%
\pgfsetdash{}{0pt}%
\pgfpathmoveto{\pgfqpoint{4.243112in}{0.527436in}}%
\pgfpathlineto{\pgfqpoint{4.243112in}{4.762436in}}%
\pgfusepath{stroke}%
\end{pgfscope}%
\begin{pgfscope}%
\pgfsetbuttcap%
\pgfsetroundjoin%
\definecolor{currentfill}{rgb}{0.000000,0.000000,0.000000}%
\pgfsetfillcolor{currentfill}%
\pgfsetlinewidth{0.602250pt}%
\definecolor{currentstroke}{rgb}{0.000000,0.000000,0.000000}%
\pgfsetstrokecolor{currentstroke}%
\pgfsetdash{}{0pt}%
\pgfsys@defobject{currentmarker}{\pgfqpoint{0.000000in}{-0.027778in}}{\pgfqpoint{0.000000in}{0.000000in}}{%
\pgfpathmoveto{\pgfqpoint{0.000000in}{0.000000in}}%
\pgfpathlineto{\pgfqpoint{0.000000in}{-0.027778in}}%
\pgfusepath{stroke,fill}%
}%
\begin{pgfscope}%
\pgfsys@transformshift{4.243112in}{0.527436in}%
\pgfsys@useobject{currentmarker}{}%
\end{pgfscope}%
\end{pgfscope}%
\begin{pgfscope}%
\pgfpathrectangle{\pgfqpoint{0.640323in}{0.527436in}}{\pgfqpoint{4.403409in}{4.235000in}}%
\pgfusepath{clip}%
\pgfsetrectcap%
\pgfsetroundjoin%
\pgfsetlinewidth{0.803000pt}%
\definecolor{currentstroke}{rgb}{0.600000,0.600000,0.600000}%
\pgfsetstrokecolor{currentstroke}%
\pgfsetstrokeopacity{0.200000}%
\pgfsetdash{}{0pt}%
\pgfpathmoveto{\pgfqpoint{4.443267in}{0.527436in}}%
\pgfpathlineto{\pgfqpoint{4.443267in}{4.762436in}}%
\pgfusepath{stroke}%
\end{pgfscope}%
\begin{pgfscope}%
\pgfsetbuttcap%
\pgfsetroundjoin%
\definecolor{currentfill}{rgb}{0.000000,0.000000,0.000000}%
\pgfsetfillcolor{currentfill}%
\pgfsetlinewidth{0.602250pt}%
\definecolor{currentstroke}{rgb}{0.000000,0.000000,0.000000}%
\pgfsetstrokecolor{currentstroke}%
\pgfsetdash{}{0pt}%
\pgfsys@defobject{currentmarker}{\pgfqpoint{0.000000in}{-0.027778in}}{\pgfqpoint{0.000000in}{0.000000in}}{%
\pgfpathmoveto{\pgfqpoint{0.000000in}{0.000000in}}%
\pgfpathlineto{\pgfqpoint{0.000000in}{-0.027778in}}%
\pgfusepath{stroke,fill}%
}%
\begin{pgfscope}%
\pgfsys@transformshift{4.443267in}{0.527436in}%
\pgfsys@useobject{currentmarker}{}%
\end{pgfscope}%
\end{pgfscope}%
\begin{pgfscope}%
\pgfpathrectangle{\pgfqpoint{0.640323in}{0.527436in}}{\pgfqpoint{4.403409in}{4.235000in}}%
\pgfusepath{clip}%
\pgfsetrectcap%
\pgfsetroundjoin%
\pgfsetlinewidth{0.803000pt}%
\definecolor{currentstroke}{rgb}{0.600000,0.600000,0.600000}%
\pgfsetstrokecolor{currentstroke}%
\pgfsetstrokeopacity{0.200000}%
\pgfsetdash{}{0pt}%
\pgfpathmoveto{\pgfqpoint{4.643422in}{0.527436in}}%
\pgfpathlineto{\pgfqpoint{4.643422in}{4.762436in}}%
\pgfusepath{stroke}%
\end{pgfscope}%
\begin{pgfscope}%
\pgfsetbuttcap%
\pgfsetroundjoin%
\definecolor{currentfill}{rgb}{0.000000,0.000000,0.000000}%
\pgfsetfillcolor{currentfill}%
\pgfsetlinewidth{0.602250pt}%
\definecolor{currentstroke}{rgb}{0.000000,0.000000,0.000000}%
\pgfsetstrokecolor{currentstroke}%
\pgfsetdash{}{0pt}%
\pgfsys@defobject{currentmarker}{\pgfqpoint{0.000000in}{-0.027778in}}{\pgfqpoint{0.000000in}{0.000000in}}{%
\pgfpathmoveto{\pgfqpoint{0.000000in}{0.000000in}}%
\pgfpathlineto{\pgfqpoint{0.000000in}{-0.027778in}}%
\pgfusepath{stroke,fill}%
}%
\begin{pgfscope}%
\pgfsys@transformshift{4.643422in}{0.527436in}%
\pgfsys@useobject{currentmarker}{}%
\end{pgfscope}%
\end{pgfscope}%
\begin{pgfscope}%
\definecolor{textcolor}{rgb}{0.000000,0.000000,0.000000}%
\pgfsetstrokecolor{textcolor}%
\pgfsetfillcolor{textcolor}%
\pgftext[x=2.842027in,y=0.240245in,,top]{\color{textcolor}\sffamily\fontsize{10.000000}{12.000000}\selectfont turnover probability \(\displaystyle p_1\,(S\rightarrow I\,)\)}%
\end{pgfscope}%
\begin{pgfscope}%
\pgfpathrectangle{\pgfqpoint{0.640323in}{0.527436in}}{\pgfqpoint{4.403409in}{4.235000in}}%
\pgfusepath{clip}%
\pgfsetrectcap%
\pgfsetroundjoin%
\pgfsetlinewidth{0.803000pt}%
\definecolor{currentstroke}{rgb}{0.690196,0.690196,0.690196}%
\pgfsetstrokecolor{currentstroke}%
\pgfsetdash{}{0pt}%
\pgfpathmoveto{\pgfqpoint{0.640323in}{0.664049in}}%
\pgfpathlineto{\pgfqpoint{5.043732in}{0.664049in}}%
\pgfusepath{stroke}%
\end{pgfscope}%
\begin{pgfscope}%
\pgfsetbuttcap%
\pgfsetroundjoin%
\definecolor{currentfill}{rgb}{0.000000,0.000000,0.000000}%
\pgfsetfillcolor{currentfill}%
\pgfsetlinewidth{0.803000pt}%
\definecolor{currentstroke}{rgb}{0.000000,0.000000,0.000000}%
\pgfsetstrokecolor{currentstroke}%
\pgfsetdash{}{0pt}%
\pgfsys@defobject{currentmarker}{\pgfqpoint{-0.048611in}{0.000000in}}{\pgfqpoint{-0.000000in}{0.000000in}}{%
\pgfpathmoveto{\pgfqpoint{-0.000000in}{0.000000in}}%
\pgfpathlineto{\pgfqpoint{-0.048611in}{0.000000in}}%
\pgfusepath{stroke,fill}%
}%
\begin{pgfscope}%
\pgfsys@transformshift{0.640323in}{0.664049in}%
\pgfsys@useobject{currentmarker}{}%
\end{pgfscope}%
\end{pgfscope}%
\begin{pgfscope}%
\definecolor{textcolor}{rgb}{0.000000,0.000000,0.000000}%
\pgfsetstrokecolor{textcolor}%
\pgfsetfillcolor{textcolor}%
\pgftext[x=0.322221in, y=0.611287in, left, base]{\color{textcolor}\sffamily\fontsize{10.000000}{12.000000}\selectfont 0.0}%
\end{pgfscope}%
\begin{pgfscope}%
\pgfpathrectangle{\pgfqpoint{0.640323in}{0.527436in}}{\pgfqpoint{4.403409in}{4.235000in}}%
\pgfusepath{clip}%
\pgfsetrectcap%
\pgfsetroundjoin%
\pgfsetlinewidth{0.803000pt}%
\definecolor{currentstroke}{rgb}{0.690196,0.690196,0.690196}%
\pgfsetstrokecolor{currentstroke}%
\pgfsetdash{}{0pt}%
\pgfpathmoveto{\pgfqpoint{0.640323in}{1.347113in}}%
\pgfpathlineto{\pgfqpoint{5.043732in}{1.347113in}}%
\pgfusepath{stroke}%
\end{pgfscope}%
\begin{pgfscope}%
\pgfsetbuttcap%
\pgfsetroundjoin%
\definecolor{currentfill}{rgb}{0.000000,0.000000,0.000000}%
\pgfsetfillcolor{currentfill}%
\pgfsetlinewidth{0.803000pt}%
\definecolor{currentstroke}{rgb}{0.000000,0.000000,0.000000}%
\pgfsetstrokecolor{currentstroke}%
\pgfsetdash{}{0pt}%
\pgfsys@defobject{currentmarker}{\pgfqpoint{-0.048611in}{0.000000in}}{\pgfqpoint{-0.000000in}{0.000000in}}{%
\pgfpathmoveto{\pgfqpoint{-0.000000in}{0.000000in}}%
\pgfpathlineto{\pgfqpoint{-0.048611in}{0.000000in}}%
\pgfusepath{stroke,fill}%
}%
\begin{pgfscope}%
\pgfsys@transformshift{0.640323in}{1.347113in}%
\pgfsys@useobject{currentmarker}{}%
\end{pgfscope}%
\end{pgfscope}%
\begin{pgfscope}%
\definecolor{textcolor}{rgb}{0.000000,0.000000,0.000000}%
\pgfsetstrokecolor{textcolor}%
\pgfsetfillcolor{textcolor}%
\pgftext[x=0.322221in, y=1.294352in, left, base]{\color{textcolor}\sffamily\fontsize{10.000000}{12.000000}\selectfont 0.1}%
\end{pgfscope}%
\begin{pgfscope}%
\pgfpathrectangle{\pgfqpoint{0.640323in}{0.527436in}}{\pgfqpoint{4.403409in}{4.235000in}}%
\pgfusepath{clip}%
\pgfsetrectcap%
\pgfsetroundjoin%
\pgfsetlinewidth{0.803000pt}%
\definecolor{currentstroke}{rgb}{0.690196,0.690196,0.690196}%
\pgfsetstrokecolor{currentstroke}%
\pgfsetdash{}{0pt}%
\pgfpathmoveto{\pgfqpoint{0.640323in}{2.030178in}}%
\pgfpathlineto{\pgfqpoint{5.043732in}{2.030178in}}%
\pgfusepath{stroke}%
\end{pgfscope}%
\begin{pgfscope}%
\pgfsetbuttcap%
\pgfsetroundjoin%
\definecolor{currentfill}{rgb}{0.000000,0.000000,0.000000}%
\pgfsetfillcolor{currentfill}%
\pgfsetlinewidth{0.803000pt}%
\definecolor{currentstroke}{rgb}{0.000000,0.000000,0.000000}%
\pgfsetstrokecolor{currentstroke}%
\pgfsetdash{}{0pt}%
\pgfsys@defobject{currentmarker}{\pgfqpoint{-0.048611in}{0.000000in}}{\pgfqpoint{-0.000000in}{0.000000in}}{%
\pgfpathmoveto{\pgfqpoint{-0.000000in}{0.000000in}}%
\pgfpathlineto{\pgfqpoint{-0.048611in}{0.000000in}}%
\pgfusepath{stroke,fill}%
}%
\begin{pgfscope}%
\pgfsys@transformshift{0.640323in}{2.030178in}%
\pgfsys@useobject{currentmarker}{}%
\end{pgfscope}%
\end{pgfscope}%
\begin{pgfscope}%
\definecolor{textcolor}{rgb}{0.000000,0.000000,0.000000}%
\pgfsetstrokecolor{textcolor}%
\pgfsetfillcolor{textcolor}%
\pgftext[x=0.322221in, y=1.977416in, left, base]{\color{textcolor}\sffamily\fontsize{10.000000}{12.000000}\selectfont 0.2}%
\end{pgfscope}%
\begin{pgfscope}%
\pgfpathrectangle{\pgfqpoint{0.640323in}{0.527436in}}{\pgfqpoint{4.403409in}{4.235000in}}%
\pgfusepath{clip}%
\pgfsetrectcap%
\pgfsetroundjoin%
\pgfsetlinewidth{0.803000pt}%
\definecolor{currentstroke}{rgb}{0.690196,0.690196,0.690196}%
\pgfsetstrokecolor{currentstroke}%
\pgfsetdash{}{0pt}%
\pgfpathmoveto{\pgfqpoint{0.640323in}{2.713242in}}%
\pgfpathlineto{\pgfqpoint{5.043732in}{2.713242in}}%
\pgfusepath{stroke}%
\end{pgfscope}%
\begin{pgfscope}%
\pgfsetbuttcap%
\pgfsetroundjoin%
\definecolor{currentfill}{rgb}{0.000000,0.000000,0.000000}%
\pgfsetfillcolor{currentfill}%
\pgfsetlinewidth{0.803000pt}%
\definecolor{currentstroke}{rgb}{0.000000,0.000000,0.000000}%
\pgfsetstrokecolor{currentstroke}%
\pgfsetdash{}{0pt}%
\pgfsys@defobject{currentmarker}{\pgfqpoint{-0.048611in}{0.000000in}}{\pgfqpoint{-0.000000in}{0.000000in}}{%
\pgfpathmoveto{\pgfqpoint{-0.000000in}{0.000000in}}%
\pgfpathlineto{\pgfqpoint{-0.048611in}{0.000000in}}%
\pgfusepath{stroke,fill}%
}%
\begin{pgfscope}%
\pgfsys@transformshift{0.640323in}{2.713242in}%
\pgfsys@useobject{currentmarker}{}%
\end{pgfscope}%
\end{pgfscope}%
\begin{pgfscope}%
\definecolor{textcolor}{rgb}{0.000000,0.000000,0.000000}%
\pgfsetstrokecolor{textcolor}%
\pgfsetfillcolor{textcolor}%
\pgftext[x=0.322221in, y=2.660481in, left, base]{\color{textcolor}\sffamily\fontsize{10.000000}{12.000000}\selectfont 0.3}%
\end{pgfscope}%
\begin{pgfscope}%
\pgfpathrectangle{\pgfqpoint{0.640323in}{0.527436in}}{\pgfqpoint{4.403409in}{4.235000in}}%
\pgfusepath{clip}%
\pgfsetrectcap%
\pgfsetroundjoin%
\pgfsetlinewidth{0.803000pt}%
\definecolor{currentstroke}{rgb}{0.690196,0.690196,0.690196}%
\pgfsetstrokecolor{currentstroke}%
\pgfsetdash{}{0pt}%
\pgfpathmoveto{\pgfqpoint{0.640323in}{3.396307in}}%
\pgfpathlineto{\pgfqpoint{5.043732in}{3.396307in}}%
\pgfusepath{stroke}%
\end{pgfscope}%
\begin{pgfscope}%
\pgfsetbuttcap%
\pgfsetroundjoin%
\definecolor{currentfill}{rgb}{0.000000,0.000000,0.000000}%
\pgfsetfillcolor{currentfill}%
\pgfsetlinewidth{0.803000pt}%
\definecolor{currentstroke}{rgb}{0.000000,0.000000,0.000000}%
\pgfsetstrokecolor{currentstroke}%
\pgfsetdash{}{0pt}%
\pgfsys@defobject{currentmarker}{\pgfqpoint{-0.048611in}{0.000000in}}{\pgfqpoint{-0.000000in}{0.000000in}}{%
\pgfpathmoveto{\pgfqpoint{-0.000000in}{0.000000in}}%
\pgfpathlineto{\pgfqpoint{-0.048611in}{0.000000in}}%
\pgfusepath{stroke,fill}%
}%
\begin{pgfscope}%
\pgfsys@transformshift{0.640323in}{3.396307in}%
\pgfsys@useobject{currentmarker}{}%
\end{pgfscope}%
\end{pgfscope}%
\begin{pgfscope}%
\definecolor{textcolor}{rgb}{0.000000,0.000000,0.000000}%
\pgfsetstrokecolor{textcolor}%
\pgfsetfillcolor{textcolor}%
\pgftext[x=0.322221in, y=3.343545in, left, base]{\color{textcolor}\sffamily\fontsize{10.000000}{12.000000}\selectfont 0.4}%
\end{pgfscope}%
\begin{pgfscope}%
\pgfpathrectangle{\pgfqpoint{0.640323in}{0.527436in}}{\pgfqpoint{4.403409in}{4.235000in}}%
\pgfusepath{clip}%
\pgfsetrectcap%
\pgfsetroundjoin%
\pgfsetlinewidth{0.803000pt}%
\definecolor{currentstroke}{rgb}{0.690196,0.690196,0.690196}%
\pgfsetstrokecolor{currentstroke}%
\pgfsetdash{}{0pt}%
\pgfpathmoveto{\pgfqpoint{0.640323in}{4.079371in}}%
\pgfpathlineto{\pgfqpoint{5.043732in}{4.079371in}}%
\pgfusepath{stroke}%
\end{pgfscope}%
\begin{pgfscope}%
\pgfsetbuttcap%
\pgfsetroundjoin%
\definecolor{currentfill}{rgb}{0.000000,0.000000,0.000000}%
\pgfsetfillcolor{currentfill}%
\pgfsetlinewidth{0.803000pt}%
\definecolor{currentstroke}{rgb}{0.000000,0.000000,0.000000}%
\pgfsetstrokecolor{currentstroke}%
\pgfsetdash{}{0pt}%
\pgfsys@defobject{currentmarker}{\pgfqpoint{-0.048611in}{0.000000in}}{\pgfqpoint{-0.000000in}{0.000000in}}{%
\pgfpathmoveto{\pgfqpoint{-0.000000in}{0.000000in}}%
\pgfpathlineto{\pgfqpoint{-0.048611in}{0.000000in}}%
\pgfusepath{stroke,fill}%
}%
\begin{pgfscope}%
\pgfsys@transformshift{0.640323in}{4.079371in}%
\pgfsys@useobject{currentmarker}{}%
\end{pgfscope}%
\end{pgfscope}%
\begin{pgfscope}%
\definecolor{textcolor}{rgb}{0.000000,0.000000,0.000000}%
\pgfsetstrokecolor{textcolor}%
\pgfsetfillcolor{textcolor}%
\pgftext[x=0.322221in, y=4.026610in, left, base]{\color{textcolor}\sffamily\fontsize{10.000000}{12.000000}\selectfont 0.5}%
\end{pgfscope}%
\begin{pgfscope}%
\pgfpathrectangle{\pgfqpoint{0.640323in}{0.527436in}}{\pgfqpoint{4.403409in}{4.235000in}}%
\pgfusepath{clip}%
\pgfsetrectcap%
\pgfsetroundjoin%
\pgfsetlinewidth{0.803000pt}%
\definecolor{currentstroke}{rgb}{0.690196,0.690196,0.690196}%
\pgfsetstrokecolor{currentstroke}%
\pgfsetdash{}{0pt}%
\pgfpathmoveto{\pgfqpoint{0.640323in}{4.762436in}}%
\pgfpathlineto{\pgfqpoint{5.043732in}{4.762436in}}%
\pgfusepath{stroke}%
\end{pgfscope}%
\begin{pgfscope}%
\pgfsetbuttcap%
\pgfsetroundjoin%
\definecolor{currentfill}{rgb}{0.000000,0.000000,0.000000}%
\pgfsetfillcolor{currentfill}%
\pgfsetlinewidth{0.803000pt}%
\definecolor{currentstroke}{rgb}{0.000000,0.000000,0.000000}%
\pgfsetstrokecolor{currentstroke}%
\pgfsetdash{}{0pt}%
\pgfsys@defobject{currentmarker}{\pgfqpoint{-0.048611in}{0.000000in}}{\pgfqpoint{-0.000000in}{0.000000in}}{%
\pgfpathmoveto{\pgfqpoint{-0.000000in}{0.000000in}}%
\pgfpathlineto{\pgfqpoint{-0.048611in}{0.000000in}}%
\pgfusepath{stroke,fill}%
}%
\begin{pgfscope}%
\pgfsys@transformshift{0.640323in}{4.762436in}%
\pgfsys@useobject{currentmarker}{}%
\end{pgfscope}%
\end{pgfscope}%
\begin{pgfscope}%
\definecolor{textcolor}{rgb}{0.000000,0.000000,0.000000}%
\pgfsetstrokecolor{textcolor}%
\pgfsetfillcolor{textcolor}%
\pgftext[x=0.322221in, y=4.709674in, left, base]{\color{textcolor}\sffamily\fontsize{10.000000}{12.000000}\selectfont 0.6}%
\end{pgfscope}%
\begin{pgfscope}%
\pgfpathrectangle{\pgfqpoint{0.640323in}{0.527436in}}{\pgfqpoint{4.403409in}{4.235000in}}%
\pgfusepath{clip}%
\pgfsetrectcap%
\pgfsetroundjoin%
\pgfsetlinewidth{0.803000pt}%
\definecolor{currentstroke}{rgb}{0.600000,0.600000,0.600000}%
\pgfsetstrokecolor{currentstroke}%
\pgfsetstrokeopacity{0.200000}%
\pgfsetdash{}{0pt}%
\pgfpathmoveto{\pgfqpoint{0.640323in}{0.800662in}}%
\pgfpathlineto{\pgfqpoint{5.043732in}{0.800662in}}%
\pgfusepath{stroke}%
\end{pgfscope}%
\begin{pgfscope}%
\pgfsetbuttcap%
\pgfsetroundjoin%
\definecolor{currentfill}{rgb}{0.000000,0.000000,0.000000}%
\pgfsetfillcolor{currentfill}%
\pgfsetlinewidth{0.602250pt}%
\definecolor{currentstroke}{rgb}{0.000000,0.000000,0.000000}%
\pgfsetstrokecolor{currentstroke}%
\pgfsetdash{}{0pt}%
\pgfsys@defobject{currentmarker}{\pgfqpoint{-0.027778in}{0.000000in}}{\pgfqpoint{-0.000000in}{0.000000in}}{%
\pgfpathmoveto{\pgfqpoint{-0.000000in}{0.000000in}}%
\pgfpathlineto{\pgfqpoint{-0.027778in}{0.000000in}}%
\pgfusepath{stroke,fill}%
}%
\begin{pgfscope}%
\pgfsys@transformshift{0.640323in}{0.800662in}%
\pgfsys@useobject{currentmarker}{}%
\end{pgfscope}%
\end{pgfscope}%
\begin{pgfscope}%
\pgfpathrectangle{\pgfqpoint{0.640323in}{0.527436in}}{\pgfqpoint{4.403409in}{4.235000in}}%
\pgfusepath{clip}%
\pgfsetrectcap%
\pgfsetroundjoin%
\pgfsetlinewidth{0.803000pt}%
\definecolor{currentstroke}{rgb}{0.600000,0.600000,0.600000}%
\pgfsetstrokecolor{currentstroke}%
\pgfsetstrokeopacity{0.200000}%
\pgfsetdash{}{0pt}%
\pgfpathmoveto{\pgfqpoint{0.640323in}{0.937275in}}%
\pgfpathlineto{\pgfqpoint{5.043732in}{0.937275in}}%
\pgfusepath{stroke}%
\end{pgfscope}%
\begin{pgfscope}%
\pgfsetbuttcap%
\pgfsetroundjoin%
\definecolor{currentfill}{rgb}{0.000000,0.000000,0.000000}%
\pgfsetfillcolor{currentfill}%
\pgfsetlinewidth{0.602250pt}%
\definecolor{currentstroke}{rgb}{0.000000,0.000000,0.000000}%
\pgfsetstrokecolor{currentstroke}%
\pgfsetdash{}{0pt}%
\pgfsys@defobject{currentmarker}{\pgfqpoint{-0.027778in}{0.000000in}}{\pgfqpoint{-0.000000in}{0.000000in}}{%
\pgfpathmoveto{\pgfqpoint{-0.000000in}{0.000000in}}%
\pgfpathlineto{\pgfqpoint{-0.027778in}{0.000000in}}%
\pgfusepath{stroke,fill}%
}%
\begin{pgfscope}%
\pgfsys@transformshift{0.640323in}{0.937275in}%
\pgfsys@useobject{currentmarker}{}%
\end{pgfscope}%
\end{pgfscope}%
\begin{pgfscope}%
\pgfpathrectangle{\pgfqpoint{0.640323in}{0.527436in}}{\pgfqpoint{4.403409in}{4.235000in}}%
\pgfusepath{clip}%
\pgfsetrectcap%
\pgfsetroundjoin%
\pgfsetlinewidth{0.803000pt}%
\definecolor{currentstroke}{rgb}{0.600000,0.600000,0.600000}%
\pgfsetstrokecolor{currentstroke}%
\pgfsetstrokeopacity{0.200000}%
\pgfsetdash{}{0pt}%
\pgfpathmoveto{\pgfqpoint{0.640323in}{1.073887in}}%
\pgfpathlineto{\pgfqpoint{5.043732in}{1.073887in}}%
\pgfusepath{stroke}%
\end{pgfscope}%
\begin{pgfscope}%
\pgfsetbuttcap%
\pgfsetroundjoin%
\definecolor{currentfill}{rgb}{0.000000,0.000000,0.000000}%
\pgfsetfillcolor{currentfill}%
\pgfsetlinewidth{0.602250pt}%
\definecolor{currentstroke}{rgb}{0.000000,0.000000,0.000000}%
\pgfsetstrokecolor{currentstroke}%
\pgfsetdash{}{0pt}%
\pgfsys@defobject{currentmarker}{\pgfqpoint{-0.027778in}{0.000000in}}{\pgfqpoint{-0.000000in}{0.000000in}}{%
\pgfpathmoveto{\pgfqpoint{-0.000000in}{0.000000in}}%
\pgfpathlineto{\pgfqpoint{-0.027778in}{0.000000in}}%
\pgfusepath{stroke,fill}%
}%
\begin{pgfscope}%
\pgfsys@transformshift{0.640323in}{1.073887in}%
\pgfsys@useobject{currentmarker}{}%
\end{pgfscope}%
\end{pgfscope}%
\begin{pgfscope}%
\pgfpathrectangle{\pgfqpoint{0.640323in}{0.527436in}}{\pgfqpoint{4.403409in}{4.235000in}}%
\pgfusepath{clip}%
\pgfsetrectcap%
\pgfsetroundjoin%
\pgfsetlinewidth{0.803000pt}%
\definecolor{currentstroke}{rgb}{0.600000,0.600000,0.600000}%
\pgfsetstrokecolor{currentstroke}%
\pgfsetstrokeopacity{0.200000}%
\pgfsetdash{}{0pt}%
\pgfpathmoveto{\pgfqpoint{0.640323in}{1.210500in}}%
\pgfpathlineto{\pgfqpoint{5.043732in}{1.210500in}}%
\pgfusepath{stroke}%
\end{pgfscope}%
\begin{pgfscope}%
\pgfsetbuttcap%
\pgfsetroundjoin%
\definecolor{currentfill}{rgb}{0.000000,0.000000,0.000000}%
\pgfsetfillcolor{currentfill}%
\pgfsetlinewidth{0.602250pt}%
\definecolor{currentstroke}{rgb}{0.000000,0.000000,0.000000}%
\pgfsetstrokecolor{currentstroke}%
\pgfsetdash{}{0pt}%
\pgfsys@defobject{currentmarker}{\pgfqpoint{-0.027778in}{0.000000in}}{\pgfqpoint{-0.000000in}{0.000000in}}{%
\pgfpathmoveto{\pgfqpoint{-0.000000in}{0.000000in}}%
\pgfpathlineto{\pgfqpoint{-0.027778in}{0.000000in}}%
\pgfusepath{stroke,fill}%
}%
\begin{pgfscope}%
\pgfsys@transformshift{0.640323in}{1.210500in}%
\pgfsys@useobject{currentmarker}{}%
\end{pgfscope}%
\end{pgfscope}%
\begin{pgfscope}%
\pgfpathrectangle{\pgfqpoint{0.640323in}{0.527436in}}{\pgfqpoint{4.403409in}{4.235000in}}%
\pgfusepath{clip}%
\pgfsetrectcap%
\pgfsetroundjoin%
\pgfsetlinewidth{0.803000pt}%
\definecolor{currentstroke}{rgb}{0.600000,0.600000,0.600000}%
\pgfsetstrokecolor{currentstroke}%
\pgfsetstrokeopacity{0.200000}%
\pgfsetdash{}{0pt}%
\pgfpathmoveto{\pgfqpoint{0.640323in}{1.483726in}}%
\pgfpathlineto{\pgfqpoint{5.043732in}{1.483726in}}%
\pgfusepath{stroke}%
\end{pgfscope}%
\begin{pgfscope}%
\pgfsetbuttcap%
\pgfsetroundjoin%
\definecolor{currentfill}{rgb}{0.000000,0.000000,0.000000}%
\pgfsetfillcolor{currentfill}%
\pgfsetlinewidth{0.602250pt}%
\definecolor{currentstroke}{rgb}{0.000000,0.000000,0.000000}%
\pgfsetstrokecolor{currentstroke}%
\pgfsetdash{}{0pt}%
\pgfsys@defobject{currentmarker}{\pgfqpoint{-0.027778in}{0.000000in}}{\pgfqpoint{-0.000000in}{0.000000in}}{%
\pgfpathmoveto{\pgfqpoint{-0.000000in}{0.000000in}}%
\pgfpathlineto{\pgfqpoint{-0.027778in}{0.000000in}}%
\pgfusepath{stroke,fill}%
}%
\begin{pgfscope}%
\pgfsys@transformshift{0.640323in}{1.483726in}%
\pgfsys@useobject{currentmarker}{}%
\end{pgfscope}%
\end{pgfscope}%
\begin{pgfscope}%
\pgfpathrectangle{\pgfqpoint{0.640323in}{0.527436in}}{\pgfqpoint{4.403409in}{4.235000in}}%
\pgfusepath{clip}%
\pgfsetrectcap%
\pgfsetroundjoin%
\pgfsetlinewidth{0.803000pt}%
\definecolor{currentstroke}{rgb}{0.600000,0.600000,0.600000}%
\pgfsetstrokecolor{currentstroke}%
\pgfsetstrokeopacity{0.200000}%
\pgfsetdash{}{0pt}%
\pgfpathmoveto{\pgfqpoint{0.640323in}{1.620339in}}%
\pgfpathlineto{\pgfqpoint{5.043732in}{1.620339in}}%
\pgfusepath{stroke}%
\end{pgfscope}%
\begin{pgfscope}%
\pgfsetbuttcap%
\pgfsetroundjoin%
\definecolor{currentfill}{rgb}{0.000000,0.000000,0.000000}%
\pgfsetfillcolor{currentfill}%
\pgfsetlinewidth{0.602250pt}%
\definecolor{currentstroke}{rgb}{0.000000,0.000000,0.000000}%
\pgfsetstrokecolor{currentstroke}%
\pgfsetdash{}{0pt}%
\pgfsys@defobject{currentmarker}{\pgfqpoint{-0.027778in}{0.000000in}}{\pgfqpoint{-0.000000in}{0.000000in}}{%
\pgfpathmoveto{\pgfqpoint{-0.000000in}{0.000000in}}%
\pgfpathlineto{\pgfqpoint{-0.027778in}{0.000000in}}%
\pgfusepath{stroke,fill}%
}%
\begin{pgfscope}%
\pgfsys@transformshift{0.640323in}{1.620339in}%
\pgfsys@useobject{currentmarker}{}%
\end{pgfscope}%
\end{pgfscope}%
\begin{pgfscope}%
\pgfpathrectangle{\pgfqpoint{0.640323in}{0.527436in}}{\pgfqpoint{4.403409in}{4.235000in}}%
\pgfusepath{clip}%
\pgfsetrectcap%
\pgfsetroundjoin%
\pgfsetlinewidth{0.803000pt}%
\definecolor{currentstroke}{rgb}{0.600000,0.600000,0.600000}%
\pgfsetstrokecolor{currentstroke}%
\pgfsetstrokeopacity{0.200000}%
\pgfsetdash{}{0pt}%
\pgfpathmoveto{\pgfqpoint{0.640323in}{1.756952in}}%
\pgfpathlineto{\pgfqpoint{5.043732in}{1.756952in}}%
\pgfusepath{stroke}%
\end{pgfscope}%
\begin{pgfscope}%
\pgfsetbuttcap%
\pgfsetroundjoin%
\definecolor{currentfill}{rgb}{0.000000,0.000000,0.000000}%
\pgfsetfillcolor{currentfill}%
\pgfsetlinewidth{0.602250pt}%
\definecolor{currentstroke}{rgb}{0.000000,0.000000,0.000000}%
\pgfsetstrokecolor{currentstroke}%
\pgfsetdash{}{0pt}%
\pgfsys@defobject{currentmarker}{\pgfqpoint{-0.027778in}{0.000000in}}{\pgfqpoint{-0.000000in}{0.000000in}}{%
\pgfpathmoveto{\pgfqpoint{-0.000000in}{0.000000in}}%
\pgfpathlineto{\pgfqpoint{-0.027778in}{0.000000in}}%
\pgfusepath{stroke,fill}%
}%
\begin{pgfscope}%
\pgfsys@transformshift{0.640323in}{1.756952in}%
\pgfsys@useobject{currentmarker}{}%
\end{pgfscope}%
\end{pgfscope}%
\begin{pgfscope}%
\pgfpathrectangle{\pgfqpoint{0.640323in}{0.527436in}}{\pgfqpoint{4.403409in}{4.235000in}}%
\pgfusepath{clip}%
\pgfsetrectcap%
\pgfsetroundjoin%
\pgfsetlinewidth{0.803000pt}%
\definecolor{currentstroke}{rgb}{0.600000,0.600000,0.600000}%
\pgfsetstrokecolor{currentstroke}%
\pgfsetstrokeopacity{0.200000}%
\pgfsetdash{}{0pt}%
\pgfpathmoveto{\pgfqpoint{0.640323in}{1.893565in}}%
\pgfpathlineto{\pgfqpoint{5.043732in}{1.893565in}}%
\pgfusepath{stroke}%
\end{pgfscope}%
\begin{pgfscope}%
\pgfsetbuttcap%
\pgfsetroundjoin%
\definecolor{currentfill}{rgb}{0.000000,0.000000,0.000000}%
\pgfsetfillcolor{currentfill}%
\pgfsetlinewidth{0.602250pt}%
\definecolor{currentstroke}{rgb}{0.000000,0.000000,0.000000}%
\pgfsetstrokecolor{currentstroke}%
\pgfsetdash{}{0pt}%
\pgfsys@defobject{currentmarker}{\pgfqpoint{-0.027778in}{0.000000in}}{\pgfqpoint{-0.000000in}{0.000000in}}{%
\pgfpathmoveto{\pgfqpoint{-0.000000in}{0.000000in}}%
\pgfpathlineto{\pgfqpoint{-0.027778in}{0.000000in}}%
\pgfusepath{stroke,fill}%
}%
\begin{pgfscope}%
\pgfsys@transformshift{0.640323in}{1.893565in}%
\pgfsys@useobject{currentmarker}{}%
\end{pgfscope}%
\end{pgfscope}%
\begin{pgfscope}%
\pgfpathrectangle{\pgfqpoint{0.640323in}{0.527436in}}{\pgfqpoint{4.403409in}{4.235000in}}%
\pgfusepath{clip}%
\pgfsetrectcap%
\pgfsetroundjoin%
\pgfsetlinewidth{0.803000pt}%
\definecolor{currentstroke}{rgb}{0.600000,0.600000,0.600000}%
\pgfsetstrokecolor{currentstroke}%
\pgfsetstrokeopacity{0.200000}%
\pgfsetdash{}{0pt}%
\pgfpathmoveto{\pgfqpoint{0.640323in}{2.166791in}}%
\pgfpathlineto{\pgfqpoint{5.043732in}{2.166791in}}%
\pgfusepath{stroke}%
\end{pgfscope}%
\begin{pgfscope}%
\pgfsetbuttcap%
\pgfsetroundjoin%
\definecolor{currentfill}{rgb}{0.000000,0.000000,0.000000}%
\pgfsetfillcolor{currentfill}%
\pgfsetlinewidth{0.602250pt}%
\definecolor{currentstroke}{rgb}{0.000000,0.000000,0.000000}%
\pgfsetstrokecolor{currentstroke}%
\pgfsetdash{}{0pt}%
\pgfsys@defobject{currentmarker}{\pgfqpoint{-0.027778in}{0.000000in}}{\pgfqpoint{-0.000000in}{0.000000in}}{%
\pgfpathmoveto{\pgfqpoint{-0.000000in}{0.000000in}}%
\pgfpathlineto{\pgfqpoint{-0.027778in}{0.000000in}}%
\pgfusepath{stroke,fill}%
}%
\begin{pgfscope}%
\pgfsys@transformshift{0.640323in}{2.166791in}%
\pgfsys@useobject{currentmarker}{}%
\end{pgfscope}%
\end{pgfscope}%
\begin{pgfscope}%
\pgfpathrectangle{\pgfqpoint{0.640323in}{0.527436in}}{\pgfqpoint{4.403409in}{4.235000in}}%
\pgfusepath{clip}%
\pgfsetrectcap%
\pgfsetroundjoin%
\pgfsetlinewidth{0.803000pt}%
\definecolor{currentstroke}{rgb}{0.600000,0.600000,0.600000}%
\pgfsetstrokecolor{currentstroke}%
\pgfsetstrokeopacity{0.200000}%
\pgfsetdash{}{0pt}%
\pgfpathmoveto{\pgfqpoint{0.640323in}{2.303404in}}%
\pgfpathlineto{\pgfqpoint{5.043732in}{2.303404in}}%
\pgfusepath{stroke}%
\end{pgfscope}%
\begin{pgfscope}%
\pgfsetbuttcap%
\pgfsetroundjoin%
\definecolor{currentfill}{rgb}{0.000000,0.000000,0.000000}%
\pgfsetfillcolor{currentfill}%
\pgfsetlinewidth{0.602250pt}%
\definecolor{currentstroke}{rgb}{0.000000,0.000000,0.000000}%
\pgfsetstrokecolor{currentstroke}%
\pgfsetdash{}{0pt}%
\pgfsys@defobject{currentmarker}{\pgfqpoint{-0.027778in}{0.000000in}}{\pgfqpoint{-0.000000in}{0.000000in}}{%
\pgfpathmoveto{\pgfqpoint{-0.000000in}{0.000000in}}%
\pgfpathlineto{\pgfqpoint{-0.027778in}{0.000000in}}%
\pgfusepath{stroke,fill}%
}%
\begin{pgfscope}%
\pgfsys@transformshift{0.640323in}{2.303404in}%
\pgfsys@useobject{currentmarker}{}%
\end{pgfscope}%
\end{pgfscope}%
\begin{pgfscope}%
\pgfpathrectangle{\pgfqpoint{0.640323in}{0.527436in}}{\pgfqpoint{4.403409in}{4.235000in}}%
\pgfusepath{clip}%
\pgfsetrectcap%
\pgfsetroundjoin%
\pgfsetlinewidth{0.803000pt}%
\definecolor{currentstroke}{rgb}{0.600000,0.600000,0.600000}%
\pgfsetstrokecolor{currentstroke}%
\pgfsetstrokeopacity{0.200000}%
\pgfsetdash{}{0pt}%
\pgfpathmoveto{\pgfqpoint{0.640323in}{2.440016in}}%
\pgfpathlineto{\pgfqpoint{5.043732in}{2.440016in}}%
\pgfusepath{stroke}%
\end{pgfscope}%
\begin{pgfscope}%
\pgfsetbuttcap%
\pgfsetroundjoin%
\definecolor{currentfill}{rgb}{0.000000,0.000000,0.000000}%
\pgfsetfillcolor{currentfill}%
\pgfsetlinewidth{0.602250pt}%
\definecolor{currentstroke}{rgb}{0.000000,0.000000,0.000000}%
\pgfsetstrokecolor{currentstroke}%
\pgfsetdash{}{0pt}%
\pgfsys@defobject{currentmarker}{\pgfqpoint{-0.027778in}{0.000000in}}{\pgfqpoint{-0.000000in}{0.000000in}}{%
\pgfpathmoveto{\pgfqpoint{-0.000000in}{0.000000in}}%
\pgfpathlineto{\pgfqpoint{-0.027778in}{0.000000in}}%
\pgfusepath{stroke,fill}%
}%
\begin{pgfscope}%
\pgfsys@transformshift{0.640323in}{2.440016in}%
\pgfsys@useobject{currentmarker}{}%
\end{pgfscope}%
\end{pgfscope}%
\begin{pgfscope}%
\pgfpathrectangle{\pgfqpoint{0.640323in}{0.527436in}}{\pgfqpoint{4.403409in}{4.235000in}}%
\pgfusepath{clip}%
\pgfsetrectcap%
\pgfsetroundjoin%
\pgfsetlinewidth{0.803000pt}%
\definecolor{currentstroke}{rgb}{0.600000,0.600000,0.600000}%
\pgfsetstrokecolor{currentstroke}%
\pgfsetstrokeopacity{0.200000}%
\pgfsetdash{}{0pt}%
\pgfpathmoveto{\pgfqpoint{0.640323in}{2.576629in}}%
\pgfpathlineto{\pgfqpoint{5.043732in}{2.576629in}}%
\pgfusepath{stroke}%
\end{pgfscope}%
\begin{pgfscope}%
\pgfsetbuttcap%
\pgfsetroundjoin%
\definecolor{currentfill}{rgb}{0.000000,0.000000,0.000000}%
\pgfsetfillcolor{currentfill}%
\pgfsetlinewidth{0.602250pt}%
\definecolor{currentstroke}{rgb}{0.000000,0.000000,0.000000}%
\pgfsetstrokecolor{currentstroke}%
\pgfsetdash{}{0pt}%
\pgfsys@defobject{currentmarker}{\pgfqpoint{-0.027778in}{0.000000in}}{\pgfqpoint{-0.000000in}{0.000000in}}{%
\pgfpathmoveto{\pgfqpoint{-0.000000in}{0.000000in}}%
\pgfpathlineto{\pgfqpoint{-0.027778in}{0.000000in}}%
\pgfusepath{stroke,fill}%
}%
\begin{pgfscope}%
\pgfsys@transformshift{0.640323in}{2.576629in}%
\pgfsys@useobject{currentmarker}{}%
\end{pgfscope}%
\end{pgfscope}%
\begin{pgfscope}%
\pgfpathrectangle{\pgfqpoint{0.640323in}{0.527436in}}{\pgfqpoint{4.403409in}{4.235000in}}%
\pgfusepath{clip}%
\pgfsetrectcap%
\pgfsetroundjoin%
\pgfsetlinewidth{0.803000pt}%
\definecolor{currentstroke}{rgb}{0.600000,0.600000,0.600000}%
\pgfsetstrokecolor{currentstroke}%
\pgfsetstrokeopacity{0.200000}%
\pgfsetdash{}{0pt}%
\pgfpathmoveto{\pgfqpoint{0.640323in}{2.849855in}}%
\pgfpathlineto{\pgfqpoint{5.043732in}{2.849855in}}%
\pgfusepath{stroke}%
\end{pgfscope}%
\begin{pgfscope}%
\pgfsetbuttcap%
\pgfsetroundjoin%
\definecolor{currentfill}{rgb}{0.000000,0.000000,0.000000}%
\pgfsetfillcolor{currentfill}%
\pgfsetlinewidth{0.602250pt}%
\definecolor{currentstroke}{rgb}{0.000000,0.000000,0.000000}%
\pgfsetstrokecolor{currentstroke}%
\pgfsetdash{}{0pt}%
\pgfsys@defobject{currentmarker}{\pgfqpoint{-0.027778in}{0.000000in}}{\pgfqpoint{-0.000000in}{0.000000in}}{%
\pgfpathmoveto{\pgfqpoint{-0.000000in}{0.000000in}}%
\pgfpathlineto{\pgfqpoint{-0.027778in}{0.000000in}}%
\pgfusepath{stroke,fill}%
}%
\begin{pgfscope}%
\pgfsys@transformshift{0.640323in}{2.849855in}%
\pgfsys@useobject{currentmarker}{}%
\end{pgfscope}%
\end{pgfscope}%
\begin{pgfscope}%
\pgfpathrectangle{\pgfqpoint{0.640323in}{0.527436in}}{\pgfqpoint{4.403409in}{4.235000in}}%
\pgfusepath{clip}%
\pgfsetrectcap%
\pgfsetroundjoin%
\pgfsetlinewidth{0.803000pt}%
\definecolor{currentstroke}{rgb}{0.600000,0.600000,0.600000}%
\pgfsetstrokecolor{currentstroke}%
\pgfsetstrokeopacity{0.200000}%
\pgfsetdash{}{0pt}%
\pgfpathmoveto{\pgfqpoint{0.640323in}{2.986468in}}%
\pgfpathlineto{\pgfqpoint{5.043732in}{2.986468in}}%
\pgfusepath{stroke}%
\end{pgfscope}%
\begin{pgfscope}%
\pgfsetbuttcap%
\pgfsetroundjoin%
\definecolor{currentfill}{rgb}{0.000000,0.000000,0.000000}%
\pgfsetfillcolor{currentfill}%
\pgfsetlinewidth{0.602250pt}%
\definecolor{currentstroke}{rgb}{0.000000,0.000000,0.000000}%
\pgfsetstrokecolor{currentstroke}%
\pgfsetdash{}{0pt}%
\pgfsys@defobject{currentmarker}{\pgfqpoint{-0.027778in}{0.000000in}}{\pgfqpoint{-0.000000in}{0.000000in}}{%
\pgfpathmoveto{\pgfqpoint{-0.000000in}{0.000000in}}%
\pgfpathlineto{\pgfqpoint{-0.027778in}{0.000000in}}%
\pgfusepath{stroke,fill}%
}%
\begin{pgfscope}%
\pgfsys@transformshift{0.640323in}{2.986468in}%
\pgfsys@useobject{currentmarker}{}%
\end{pgfscope}%
\end{pgfscope}%
\begin{pgfscope}%
\pgfpathrectangle{\pgfqpoint{0.640323in}{0.527436in}}{\pgfqpoint{4.403409in}{4.235000in}}%
\pgfusepath{clip}%
\pgfsetrectcap%
\pgfsetroundjoin%
\pgfsetlinewidth{0.803000pt}%
\definecolor{currentstroke}{rgb}{0.600000,0.600000,0.600000}%
\pgfsetstrokecolor{currentstroke}%
\pgfsetstrokeopacity{0.200000}%
\pgfsetdash{}{0pt}%
\pgfpathmoveto{\pgfqpoint{0.640323in}{3.123081in}}%
\pgfpathlineto{\pgfqpoint{5.043732in}{3.123081in}}%
\pgfusepath{stroke}%
\end{pgfscope}%
\begin{pgfscope}%
\pgfsetbuttcap%
\pgfsetroundjoin%
\definecolor{currentfill}{rgb}{0.000000,0.000000,0.000000}%
\pgfsetfillcolor{currentfill}%
\pgfsetlinewidth{0.602250pt}%
\definecolor{currentstroke}{rgb}{0.000000,0.000000,0.000000}%
\pgfsetstrokecolor{currentstroke}%
\pgfsetdash{}{0pt}%
\pgfsys@defobject{currentmarker}{\pgfqpoint{-0.027778in}{0.000000in}}{\pgfqpoint{-0.000000in}{0.000000in}}{%
\pgfpathmoveto{\pgfqpoint{-0.000000in}{0.000000in}}%
\pgfpathlineto{\pgfqpoint{-0.027778in}{0.000000in}}%
\pgfusepath{stroke,fill}%
}%
\begin{pgfscope}%
\pgfsys@transformshift{0.640323in}{3.123081in}%
\pgfsys@useobject{currentmarker}{}%
\end{pgfscope}%
\end{pgfscope}%
\begin{pgfscope}%
\pgfpathrectangle{\pgfqpoint{0.640323in}{0.527436in}}{\pgfqpoint{4.403409in}{4.235000in}}%
\pgfusepath{clip}%
\pgfsetrectcap%
\pgfsetroundjoin%
\pgfsetlinewidth{0.803000pt}%
\definecolor{currentstroke}{rgb}{0.600000,0.600000,0.600000}%
\pgfsetstrokecolor{currentstroke}%
\pgfsetstrokeopacity{0.200000}%
\pgfsetdash{}{0pt}%
\pgfpathmoveto{\pgfqpoint{0.640323in}{3.259694in}}%
\pgfpathlineto{\pgfqpoint{5.043732in}{3.259694in}}%
\pgfusepath{stroke}%
\end{pgfscope}%
\begin{pgfscope}%
\pgfsetbuttcap%
\pgfsetroundjoin%
\definecolor{currentfill}{rgb}{0.000000,0.000000,0.000000}%
\pgfsetfillcolor{currentfill}%
\pgfsetlinewidth{0.602250pt}%
\definecolor{currentstroke}{rgb}{0.000000,0.000000,0.000000}%
\pgfsetstrokecolor{currentstroke}%
\pgfsetdash{}{0pt}%
\pgfsys@defobject{currentmarker}{\pgfqpoint{-0.027778in}{0.000000in}}{\pgfqpoint{-0.000000in}{0.000000in}}{%
\pgfpathmoveto{\pgfqpoint{-0.000000in}{0.000000in}}%
\pgfpathlineto{\pgfqpoint{-0.027778in}{0.000000in}}%
\pgfusepath{stroke,fill}%
}%
\begin{pgfscope}%
\pgfsys@transformshift{0.640323in}{3.259694in}%
\pgfsys@useobject{currentmarker}{}%
\end{pgfscope}%
\end{pgfscope}%
\begin{pgfscope}%
\pgfpathrectangle{\pgfqpoint{0.640323in}{0.527436in}}{\pgfqpoint{4.403409in}{4.235000in}}%
\pgfusepath{clip}%
\pgfsetrectcap%
\pgfsetroundjoin%
\pgfsetlinewidth{0.803000pt}%
\definecolor{currentstroke}{rgb}{0.600000,0.600000,0.600000}%
\pgfsetstrokecolor{currentstroke}%
\pgfsetstrokeopacity{0.200000}%
\pgfsetdash{}{0pt}%
\pgfpathmoveto{\pgfqpoint{0.640323in}{3.532920in}}%
\pgfpathlineto{\pgfqpoint{5.043732in}{3.532920in}}%
\pgfusepath{stroke}%
\end{pgfscope}%
\begin{pgfscope}%
\pgfsetbuttcap%
\pgfsetroundjoin%
\definecolor{currentfill}{rgb}{0.000000,0.000000,0.000000}%
\pgfsetfillcolor{currentfill}%
\pgfsetlinewidth{0.602250pt}%
\definecolor{currentstroke}{rgb}{0.000000,0.000000,0.000000}%
\pgfsetstrokecolor{currentstroke}%
\pgfsetdash{}{0pt}%
\pgfsys@defobject{currentmarker}{\pgfqpoint{-0.027778in}{0.000000in}}{\pgfqpoint{-0.000000in}{0.000000in}}{%
\pgfpathmoveto{\pgfqpoint{-0.000000in}{0.000000in}}%
\pgfpathlineto{\pgfqpoint{-0.027778in}{0.000000in}}%
\pgfusepath{stroke,fill}%
}%
\begin{pgfscope}%
\pgfsys@transformshift{0.640323in}{3.532920in}%
\pgfsys@useobject{currentmarker}{}%
\end{pgfscope}%
\end{pgfscope}%
\begin{pgfscope}%
\pgfpathrectangle{\pgfqpoint{0.640323in}{0.527436in}}{\pgfqpoint{4.403409in}{4.235000in}}%
\pgfusepath{clip}%
\pgfsetrectcap%
\pgfsetroundjoin%
\pgfsetlinewidth{0.803000pt}%
\definecolor{currentstroke}{rgb}{0.600000,0.600000,0.600000}%
\pgfsetstrokecolor{currentstroke}%
\pgfsetstrokeopacity{0.200000}%
\pgfsetdash{}{0pt}%
\pgfpathmoveto{\pgfqpoint{0.640323in}{3.669533in}}%
\pgfpathlineto{\pgfqpoint{5.043732in}{3.669533in}}%
\pgfusepath{stroke}%
\end{pgfscope}%
\begin{pgfscope}%
\pgfsetbuttcap%
\pgfsetroundjoin%
\definecolor{currentfill}{rgb}{0.000000,0.000000,0.000000}%
\pgfsetfillcolor{currentfill}%
\pgfsetlinewidth{0.602250pt}%
\definecolor{currentstroke}{rgb}{0.000000,0.000000,0.000000}%
\pgfsetstrokecolor{currentstroke}%
\pgfsetdash{}{0pt}%
\pgfsys@defobject{currentmarker}{\pgfqpoint{-0.027778in}{0.000000in}}{\pgfqpoint{-0.000000in}{0.000000in}}{%
\pgfpathmoveto{\pgfqpoint{-0.000000in}{0.000000in}}%
\pgfpathlineto{\pgfqpoint{-0.027778in}{0.000000in}}%
\pgfusepath{stroke,fill}%
}%
\begin{pgfscope}%
\pgfsys@transformshift{0.640323in}{3.669533in}%
\pgfsys@useobject{currentmarker}{}%
\end{pgfscope}%
\end{pgfscope}%
\begin{pgfscope}%
\pgfpathrectangle{\pgfqpoint{0.640323in}{0.527436in}}{\pgfqpoint{4.403409in}{4.235000in}}%
\pgfusepath{clip}%
\pgfsetrectcap%
\pgfsetroundjoin%
\pgfsetlinewidth{0.803000pt}%
\definecolor{currentstroke}{rgb}{0.600000,0.600000,0.600000}%
\pgfsetstrokecolor{currentstroke}%
\pgfsetstrokeopacity{0.200000}%
\pgfsetdash{}{0pt}%
\pgfpathmoveto{\pgfqpoint{0.640323in}{3.806146in}}%
\pgfpathlineto{\pgfqpoint{5.043732in}{3.806146in}}%
\pgfusepath{stroke}%
\end{pgfscope}%
\begin{pgfscope}%
\pgfsetbuttcap%
\pgfsetroundjoin%
\definecolor{currentfill}{rgb}{0.000000,0.000000,0.000000}%
\pgfsetfillcolor{currentfill}%
\pgfsetlinewidth{0.602250pt}%
\definecolor{currentstroke}{rgb}{0.000000,0.000000,0.000000}%
\pgfsetstrokecolor{currentstroke}%
\pgfsetdash{}{0pt}%
\pgfsys@defobject{currentmarker}{\pgfqpoint{-0.027778in}{0.000000in}}{\pgfqpoint{-0.000000in}{0.000000in}}{%
\pgfpathmoveto{\pgfqpoint{-0.000000in}{0.000000in}}%
\pgfpathlineto{\pgfqpoint{-0.027778in}{0.000000in}}%
\pgfusepath{stroke,fill}%
}%
\begin{pgfscope}%
\pgfsys@transformshift{0.640323in}{3.806146in}%
\pgfsys@useobject{currentmarker}{}%
\end{pgfscope}%
\end{pgfscope}%
\begin{pgfscope}%
\pgfpathrectangle{\pgfqpoint{0.640323in}{0.527436in}}{\pgfqpoint{4.403409in}{4.235000in}}%
\pgfusepath{clip}%
\pgfsetrectcap%
\pgfsetroundjoin%
\pgfsetlinewidth{0.803000pt}%
\definecolor{currentstroke}{rgb}{0.600000,0.600000,0.600000}%
\pgfsetstrokecolor{currentstroke}%
\pgfsetstrokeopacity{0.200000}%
\pgfsetdash{}{0pt}%
\pgfpathmoveto{\pgfqpoint{0.640323in}{3.942758in}}%
\pgfpathlineto{\pgfqpoint{5.043732in}{3.942758in}}%
\pgfusepath{stroke}%
\end{pgfscope}%
\begin{pgfscope}%
\pgfsetbuttcap%
\pgfsetroundjoin%
\definecolor{currentfill}{rgb}{0.000000,0.000000,0.000000}%
\pgfsetfillcolor{currentfill}%
\pgfsetlinewidth{0.602250pt}%
\definecolor{currentstroke}{rgb}{0.000000,0.000000,0.000000}%
\pgfsetstrokecolor{currentstroke}%
\pgfsetdash{}{0pt}%
\pgfsys@defobject{currentmarker}{\pgfqpoint{-0.027778in}{0.000000in}}{\pgfqpoint{-0.000000in}{0.000000in}}{%
\pgfpathmoveto{\pgfqpoint{-0.000000in}{0.000000in}}%
\pgfpathlineto{\pgfqpoint{-0.027778in}{0.000000in}}%
\pgfusepath{stroke,fill}%
}%
\begin{pgfscope}%
\pgfsys@transformshift{0.640323in}{3.942758in}%
\pgfsys@useobject{currentmarker}{}%
\end{pgfscope}%
\end{pgfscope}%
\begin{pgfscope}%
\pgfpathrectangle{\pgfqpoint{0.640323in}{0.527436in}}{\pgfqpoint{4.403409in}{4.235000in}}%
\pgfusepath{clip}%
\pgfsetrectcap%
\pgfsetroundjoin%
\pgfsetlinewidth{0.803000pt}%
\definecolor{currentstroke}{rgb}{0.600000,0.600000,0.600000}%
\pgfsetstrokecolor{currentstroke}%
\pgfsetstrokeopacity{0.200000}%
\pgfsetdash{}{0pt}%
\pgfpathmoveto{\pgfqpoint{0.640323in}{4.215984in}}%
\pgfpathlineto{\pgfqpoint{5.043732in}{4.215984in}}%
\pgfusepath{stroke}%
\end{pgfscope}%
\begin{pgfscope}%
\pgfsetbuttcap%
\pgfsetroundjoin%
\definecolor{currentfill}{rgb}{0.000000,0.000000,0.000000}%
\pgfsetfillcolor{currentfill}%
\pgfsetlinewidth{0.602250pt}%
\definecolor{currentstroke}{rgb}{0.000000,0.000000,0.000000}%
\pgfsetstrokecolor{currentstroke}%
\pgfsetdash{}{0pt}%
\pgfsys@defobject{currentmarker}{\pgfqpoint{-0.027778in}{0.000000in}}{\pgfqpoint{-0.000000in}{0.000000in}}{%
\pgfpathmoveto{\pgfqpoint{-0.000000in}{0.000000in}}%
\pgfpathlineto{\pgfqpoint{-0.027778in}{0.000000in}}%
\pgfusepath{stroke,fill}%
}%
\begin{pgfscope}%
\pgfsys@transformshift{0.640323in}{4.215984in}%
\pgfsys@useobject{currentmarker}{}%
\end{pgfscope}%
\end{pgfscope}%
\begin{pgfscope}%
\pgfpathrectangle{\pgfqpoint{0.640323in}{0.527436in}}{\pgfqpoint{4.403409in}{4.235000in}}%
\pgfusepath{clip}%
\pgfsetrectcap%
\pgfsetroundjoin%
\pgfsetlinewidth{0.803000pt}%
\definecolor{currentstroke}{rgb}{0.600000,0.600000,0.600000}%
\pgfsetstrokecolor{currentstroke}%
\pgfsetstrokeopacity{0.200000}%
\pgfsetdash{}{0pt}%
\pgfpathmoveto{\pgfqpoint{0.640323in}{4.352597in}}%
\pgfpathlineto{\pgfqpoint{5.043732in}{4.352597in}}%
\pgfusepath{stroke}%
\end{pgfscope}%
\begin{pgfscope}%
\pgfsetbuttcap%
\pgfsetroundjoin%
\definecolor{currentfill}{rgb}{0.000000,0.000000,0.000000}%
\pgfsetfillcolor{currentfill}%
\pgfsetlinewidth{0.602250pt}%
\definecolor{currentstroke}{rgb}{0.000000,0.000000,0.000000}%
\pgfsetstrokecolor{currentstroke}%
\pgfsetdash{}{0pt}%
\pgfsys@defobject{currentmarker}{\pgfqpoint{-0.027778in}{0.000000in}}{\pgfqpoint{-0.000000in}{0.000000in}}{%
\pgfpathmoveto{\pgfqpoint{-0.000000in}{0.000000in}}%
\pgfpathlineto{\pgfqpoint{-0.027778in}{0.000000in}}%
\pgfusepath{stroke,fill}%
}%
\begin{pgfscope}%
\pgfsys@transformshift{0.640323in}{4.352597in}%
\pgfsys@useobject{currentmarker}{}%
\end{pgfscope}%
\end{pgfscope}%
\begin{pgfscope}%
\pgfpathrectangle{\pgfqpoint{0.640323in}{0.527436in}}{\pgfqpoint{4.403409in}{4.235000in}}%
\pgfusepath{clip}%
\pgfsetrectcap%
\pgfsetroundjoin%
\pgfsetlinewidth{0.803000pt}%
\definecolor{currentstroke}{rgb}{0.600000,0.600000,0.600000}%
\pgfsetstrokecolor{currentstroke}%
\pgfsetstrokeopacity{0.200000}%
\pgfsetdash{}{0pt}%
\pgfpathmoveto{\pgfqpoint{0.640323in}{4.489210in}}%
\pgfpathlineto{\pgfqpoint{5.043732in}{4.489210in}}%
\pgfusepath{stroke}%
\end{pgfscope}%
\begin{pgfscope}%
\pgfsetbuttcap%
\pgfsetroundjoin%
\definecolor{currentfill}{rgb}{0.000000,0.000000,0.000000}%
\pgfsetfillcolor{currentfill}%
\pgfsetlinewidth{0.602250pt}%
\definecolor{currentstroke}{rgb}{0.000000,0.000000,0.000000}%
\pgfsetstrokecolor{currentstroke}%
\pgfsetdash{}{0pt}%
\pgfsys@defobject{currentmarker}{\pgfqpoint{-0.027778in}{0.000000in}}{\pgfqpoint{-0.000000in}{0.000000in}}{%
\pgfpathmoveto{\pgfqpoint{-0.000000in}{0.000000in}}%
\pgfpathlineto{\pgfqpoint{-0.027778in}{0.000000in}}%
\pgfusepath{stroke,fill}%
}%
\begin{pgfscope}%
\pgfsys@transformshift{0.640323in}{4.489210in}%
\pgfsys@useobject{currentmarker}{}%
\end{pgfscope}%
\end{pgfscope}%
\begin{pgfscope}%
\pgfpathrectangle{\pgfqpoint{0.640323in}{0.527436in}}{\pgfqpoint{4.403409in}{4.235000in}}%
\pgfusepath{clip}%
\pgfsetrectcap%
\pgfsetroundjoin%
\pgfsetlinewidth{0.803000pt}%
\definecolor{currentstroke}{rgb}{0.600000,0.600000,0.600000}%
\pgfsetstrokecolor{currentstroke}%
\pgfsetstrokeopacity{0.200000}%
\pgfsetdash{}{0pt}%
\pgfpathmoveto{\pgfqpoint{0.640323in}{4.625823in}}%
\pgfpathlineto{\pgfqpoint{5.043732in}{4.625823in}}%
\pgfusepath{stroke}%
\end{pgfscope}%
\begin{pgfscope}%
\pgfsetbuttcap%
\pgfsetroundjoin%
\definecolor{currentfill}{rgb}{0.000000,0.000000,0.000000}%
\pgfsetfillcolor{currentfill}%
\pgfsetlinewidth{0.602250pt}%
\definecolor{currentstroke}{rgb}{0.000000,0.000000,0.000000}%
\pgfsetstrokecolor{currentstroke}%
\pgfsetdash{}{0pt}%
\pgfsys@defobject{currentmarker}{\pgfqpoint{-0.027778in}{0.000000in}}{\pgfqpoint{-0.000000in}{0.000000in}}{%
\pgfpathmoveto{\pgfqpoint{-0.000000in}{0.000000in}}%
\pgfpathlineto{\pgfqpoint{-0.027778in}{0.000000in}}%
\pgfusepath{stroke,fill}%
}%
\begin{pgfscope}%
\pgfsys@transformshift{0.640323in}{4.625823in}%
\pgfsys@useobject{currentmarker}{}%
\end{pgfscope}%
\end{pgfscope}%
\begin{pgfscope}%
\definecolor{textcolor}{rgb}{0.000000,0.000000,0.000000}%
\pgfsetstrokecolor{textcolor}%
\pgfsetfillcolor{textcolor}%
\pgftext[x=0.266665in,y=2.644936in,,bottom,rotate=90.000000]{\color{textcolor}\sffamily\fontsize{10.000000}{12.000000}\selectfont avg. infection rate \(\displaystyle \overline{\langle I\rangle}\)}%
\end{pgfscope}%
\begin{pgfscope}%
\pgfpathrectangle{\pgfqpoint{0.640323in}{0.527436in}}{\pgfqpoint{4.403409in}{4.235000in}}%
\pgfusepath{clip}%
\pgfsetbuttcap%
\pgfsetroundjoin%
\pgfsetlinewidth{1.003750pt}%
\definecolor{currentstroke}{rgb}{0.980392,0.164706,0.333333}%
\pgfsetstrokecolor{currentstroke}%
\pgfsetstrokeopacity{0.500000}%
\pgfsetdash{{3.700000pt}{1.600000pt}}{0.000000pt}%
\pgfpathmoveto{\pgfqpoint{0.840477in}{0.673013in}}%
\pgfpathlineto{\pgfqpoint{0.920539in}{0.673804in}}%
\pgfpathlineto{\pgfqpoint{1.000601in}{0.675232in}}%
\pgfpathlineto{\pgfqpoint{1.080663in}{0.676986in}}%
\pgfpathlineto{\pgfqpoint{1.160725in}{0.679653in}}%
\pgfpathlineto{\pgfqpoint{1.240787in}{0.682544in}}%
\pgfpathlineto{\pgfqpoint{1.320849in}{0.688079in}}%
\pgfpathlineto{\pgfqpoint{1.400911in}{0.696104in}}%
\pgfpathlineto{\pgfqpoint{1.480973in}{0.720753in}}%
\pgfpathlineto{\pgfqpoint{1.561035in}{0.852172in}}%
\pgfpathlineto{\pgfqpoint{1.641097in}{1.357469in}}%
\pgfpathlineto{\pgfqpoint{1.721159in}{1.681323in}}%
\pgfpathlineto{\pgfqpoint{1.801221in}{1.927308in}}%
\pgfpathlineto{\pgfqpoint{1.881283in}{2.115752in}}%
\pgfpathlineto{\pgfqpoint{1.961345in}{2.259141in}}%
\pgfpathlineto{\pgfqpoint{2.041407in}{2.386485in}}%
\pgfpathlineto{\pgfqpoint{2.121469in}{2.492223in}}%
\pgfpathlineto{\pgfqpoint{2.201531in}{2.581226in}}%
\pgfpathlineto{\pgfqpoint{2.281593in}{2.662689in}}%
\pgfpathlineto{\pgfqpoint{2.361655in}{2.725387in}}%
\pgfpathlineto{\pgfqpoint{2.441717in}{2.793174in}}%
\pgfpathlineto{\pgfqpoint{2.521779in}{2.852389in}}%
\pgfpathlineto{\pgfqpoint{2.601841in}{2.901495in}}%
\pgfpathlineto{\pgfqpoint{2.681903in}{2.948032in}}%
\pgfpathlineto{\pgfqpoint{2.761965in}{2.987957in}}%
\pgfpathlineto{\pgfqpoint{2.842027in}{3.028347in}}%
\pgfpathlineto{\pgfqpoint{2.922089in}{3.065847in}}%
\pgfpathlineto{\pgfqpoint{3.002151in}{3.096476in}}%
\pgfpathlineto{\pgfqpoint{3.082213in}{3.129522in}}%
\pgfpathlineto{\pgfqpoint{3.162275in}{3.162432in}}%
\pgfpathlineto{\pgfqpoint{3.242337in}{3.190322in}}%
\pgfpathlineto{\pgfqpoint{3.322399in}{3.215814in}}%
\pgfpathlineto{\pgfqpoint{3.402461in}{3.239482in}}%
\pgfpathlineto{\pgfqpoint{3.482523in}{3.262201in}}%
\pgfpathlineto{\pgfqpoint{3.562585in}{3.284243in}}%
\pgfpathlineto{\pgfqpoint{3.642647in}{3.303048in}}%
\pgfpathlineto{\pgfqpoint{3.722709in}{3.324571in}}%
\pgfpathlineto{\pgfqpoint{3.802771in}{3.338342in}}%
\pgfpathlineto{\pgfqpoint{3.882833in}{3.360084in}}%
\pgfpathlineto{\pgfqpoint{3.962895in}{3.374462in}}%
\pgfpathlineto{\pgfqpoint{4.042957in}{3.390665in}}%
\pgfpathlineto{\pgfqpoint{4.123019in}{3.407748in}}%
\pgfpathlineto{\pgfqpoint{4.203081in}{3.418602in}}%
\pgfpathlineto{\pgfqpoint{4.283143in}{3.436041in}}%
\pgfpathlineto{\pgfqpoint{4.363205in}{3.446553in}}%
\pgfpathlineto{\pgfqpoint{4.443267in}{3.460371in}}%
\pgfpathlineto{\pgfqpoint{4.523329in}{3.473951in}}%
\pgfpathlineto{\pgfqpoint{4.603391in}{3.482462in}}%
\pgfpathlineto{\pgfqpoint{4.683453in}{3.492913in}}%
\pgfpathlineto{\pgfqpoint{4.763515in}{3.504894in}}%
\pgfpathlineto{\pgfqpoint{4.843577in}{3.516328in}}%
\pgfusepath{stroke}%
\end{pgfscope}%
\begin{pgfscope}%
\pgfpathrectangle{\pgfqpoint{0.640323in}{0.527436in}}{\pgfqpoint{4.403409in}{4.235000in}}%
\pgfusepath{clip}%
\pgfsetbuttcap%
\pgfsetroundjoin%
\pgfsetlinewidth{1.003750pt}%
\definecolor{currentstroke}{rgb}{0.000000,0.000000,1.000000}%
\pgfsetstrokecolor{currentstroke}%
\pgfsetstrokeopacity{0.500000}%
\pgfsetdash{{3.700000pt}{1.600000pt}}{0.000000pt}%
\pgfpathmoveto{\pgfqpoint{0.840477in}{0.669115in}}%
\pgfpathlineto{\pgfqpoint{0.920539in}{0.669045in}}%
\pgfpathlineto{\pgfqpoint{1.000601in}{0.669505in}}%
\pgfpathlineto{\pgfqpoint{1.080663in}{0.669854in}}%
\pgfpathlineto{\pgfqpoint{1.160725in}{0.670177in}}%
\pgfpathlineto{\pgfqpoint{1.240787in}{0.670457in}}%
\pgfpathlineto{\pgfqpoint{1.320849in}{0.670753in}}%
\pgfpathlineto{\pgfqpoint{1.400911in}{0.670921in}}%
\pgfpathlineto{\pgfqpoint{1.480973in}{0.671593in}}%
\pgfpathlineto{\pgfqpoint{1.561035in}{0.672077in}}%
\pgfpathlineto{\pgfqpoint{1.641097in}{0.672227in}}%
\pgfpathlineto{\pgfqpoint{1.721159in}{0.673004in}}%
\pgfpathlineto{\pgfqpoint{1.801221in}{0.673616in}}%
\pgfpathlineto{\pgfqpoint{1.881283in}{0.674351in}}%
\pgfpathlineto{\pgfqpoint{1.961345in}{0.675458in}}%
\pgfpathlineto{\pgfqpoint{2.041407in}{0.676207in}}%
\pgfpathlineto{\pgfqpoint{2.121469in}{0.677151in}}%
\pgfpathlineto{\pgfqpoint{2.201531in}{0.680067in}}%
\pgfpathlineto{\pgfqpoint{2.281593in}{0.682179in}}%
\pgfpathlineto{\pgfqpoint{2.361655in}{0.688698in}}%
\pgfpathlineto{\pgfqpoint{2.441717in}{0.697883in}}%
\pgfpathlineto{\pgfqpoint{2.521779in}{0.717923in}}%
\pgfpathlineto{\pgfqpoint{2.601841in}{0.849337in}}%
\pgfpathlineto{\pgfqpoint{2.681903in}{0.996383in}}%
\pgfpathlineto{\pgfqpoint{2.761965in}{1.141953in}}%
\pgfpathlineto{\pgfqpoint{2.842027in}{1.265157in}}%
\pgfpathlineto{\pgfqpoint{2.922089in}{1.364012in}}%
\pgfpathlineto{\pgfqpoint{3.002151in}{1.426082in}}%
\pgfpathlineto{\pgfqpoint{3.082213in}{1.497729in}}%
\pgfpathlineto{\pgfqpoint{3.162275in}{1.559157in}}%
\pgfpathlineto{\pgfqpoint{3.242337in}{1.615742in}}%
\pgfpathlineto{\pgfqpoint{3.322399in}{1.665134in}}%
\pgfpathlineto{\pgfqpoint{3.402461in}{1.709916in}}%
\pgfpathlineto{\pgfqpoint{3.482523in}{1.745968in}}%
\pgfpathlineto{\pgfqpoint{3.562585in}{1.780846in}}%
\pgfpathlineto{\pgfqpoint{3.642647in}{1.823161in}}%
\pgfpathlineto{\pgfqpoint{3.722709in}{1.852479in}}%
\pgfpathlineto{\pgfqpoint{3.802771in}{1.885040in}}%
\pgfpathlineto{\pgfqpoint{3.882833in}{1.907383in}}%
\pgfpathlineto{\pgfqpoint{3.962895in}{1.936885in}}%
\pgfpathlineto{\pgfqpoint{4.042957in}{1.961004in}}%
\pgfpathlineto{\pgfqpoint{4.123019in}{1.987746in}}%
\pgfpathlineto{\pgfqpoint{4.203081in}{2.007691in}}%
\pgfpathlineto{\pgfqpoint{4.283143in}{2.028477in}}%
\pgfpathlineto{\pgfqpoint{4.363205in}{2.048504in}}%
\pgfpathlineto{\pgfqpoint{4.443267in}{2.067036in}}%
\pgfpathlineto{\pgfqpoint{4.523329in}{2.085834in}}%
\pgfpathlineto{\pgfqpoint{4.603391in}{2.101927in}}%
\pgfpathlineto{\pgfqpoint{4.683453in}{2.114926in}}%
\pgfpathlineto{\pgfqpoint{4.763515in}{2.132269in}}%
\pgfpathlineto{\pgfqpoint{4.843577in}{2.148362in}}%
\pgfusepath{stroke}%
\end{pgfscope}%
\begin{pgfscope}%
\pgfpathrectangle{\pgfqpoint{0.640323in}{0.527436in}}{\pgfqpoint{4.403409in}{4.235000in}}%
\pgfusepath{clip}%
\pgfsetbuttcap%
\pgfsetroundjoin%
\pgfsetlinewidth{1.003750pt}%
\definecolor{currentstroke}{rgb}{0.000000,0.000000,0.000000}%
\pgfsetstrokecolor{currentstroke}%
\pgfsetstrokeopacity{0.500000}%
\pgfsetdash{{3.700000pt}{1.600000pt}}{0.000000pt}%
\pgfpathmoveto{\pgfqpoint{0.840477in}{0.672709in}}%
\pgfpathlineto{\pgfqpoint{0.920539in}{0.674121in}}%
\pgfpathlineto{\pgfqpoint{1.000601in}{0.675713in}}%
\pgfpathlineto{\pgfqpoint{1.080663in}{0.677454in}}%
\pgfpathlineto{\pgfqpoint{1.160725in}{0.681242in}}%
\pgfpathlineto{\pgfqpoint{1.240787in}{0.686755in}}%
\pgfpathlineto{\pgfqpoint{1.320849in}{0.695784in}}%
\pgfpathlineto{\pgfqpoint{1.400911in}{0.717425in}}%
\pgfpathlineto{\pgfqpoint{1.480973in}{0.869198in}}%
\pgfpathlineto{\pgfqpoint{1.561035in}{1.547600in}}%
\pgfpathlineto{\pgfqpoint{1.641097in}{1.979720in}}%
\pgfpathlineto{\pgfqpoint{1.721159in}{2.285733in}}%
\pgfpathlineto{\pgfqpoint{1.801221in}{2.517012in}}%
\pgfpathlineto{\pgfqpoint{1.881283in}{2.701917in}}%
\pgfpathlineto{\pgfqpoint{1.961345in}{2.863831in}}%
\pgfpathlineto{\pgfqpoint{2.041407in}{2.996946in}}%
\pgfpathlineto{\pgfqpoint{2.121469in}{3.110519in}}%
\pgfpathlineto{\pgfqpoint{2.201531in}{3.216920in}}%
\pgfpathlineto{\pgfqpoint{2.281593in}{3.312146in}}%
\pgfpathlineto{\pgfqpoint{2.361655in}{3.395057in}}%
\pgfpathlineto{\pgfqpoint{2.441717in}{3.471184in}}%
\pgfpathlineto{\pgfqpoint{2.521779in}{3.539586in}}%
\pgfpathlineto{\pgfqpoint{2.601841in}{3.601049in}}%
\pgfpathlineto{\pgfqpoint{2.681903in}{3.658590in}}%
\pgfpathlineto{\pgfqpoint{2.761965in}{3.711705in}}%
\pgfpathlineto{\pgfqpoint{2.842027in}{3.762675in}}%
\pgfpathlineto{\pgfqpoint{2.922089in}{3.804834in}}%
\pgfpathlineto{\pgfqpoint{3.002151in}{3.853441in}}%
\pgfpathlineto{\pgfqpoint{3.082213in}{3.894780in}}%
\pgfpathlineto{\pgfqpoint{3.162275in}{3.930306in}}%
\pgfpathlineto{\pgfqpoint{3.242337in}{3.966099in}}%
\pgfpathlineto{\pgfqpoint{3.322399in}{3.999063in}}%
\pgfpathlineto{\pgfqpoint{3.402461in}{4.029801in}}%
\pgfpathlineto{\pgfqpoint{3.482523in}{4.061052in}}%
\pgfpathlineto{\pgfqpoint{3.562585in}{4.091434in}}%
\pgfpathlineto{\pgfqpoint{3.642647in}{4.116960in}}%
\pgfpathlineto{\pgfqpoint{3.722709in}{4.140547in}}%
\pgfpathlineto{\pgfqpoint{3.802771in}{4.169365in}}%
\pgfpathlineto{\pgfqpoint{3.882833in}{4.192849in}}%
\pgfpathlineto{\pgfqpoint{3.962895in}{4.212733in}}%
\pgfpathlineto{\pgfqpoint{4.042957in}{4.237501in}}%
\pgfpathlineto{\pgfqpoint{4.123019in}{4.255581in}}%
\pgfpathlineto{\pgfqpoint{4.203081in}{4.277153in}}%
\pgfpathlineto{\pgfqpoint{4.283143in}{4.294660in}}%
\pgfpathlineto{\pgfqpoint{4.363205in}{4.311702in}}%
\pgfpathlineto{\pgfqpoint{4.443267in}{4.329359in}}%
\pgfpathlineto{\pgfqpoint{4.523329in}{4.344510in}}%
\pgfpathlineto{\pgfqpoint{4.603391in}{4.363321in}}%
\pgfpathlineto{\pgfqpoint{4.683453in}{4.379947in}}%
\pgfpathlineto{\pgfqpoint{4.763515in}{4.391867in}}%
\pgfpathlineto{\pgfqpoint{4.843577in}{4.408547in}}%
\pgfusepath{stroke}%
\end{pgfscope}%
\begin{pgfscope}%
\pgfsetrectcap%
\pgfsetmiterjoin%
\pgfsetlinewidth{0.803000pt}%
\definecolor{currentstroke}{rgb}{0.000000,0.000000,0.000000}%
\pgfsetstrokecolor{currentstroke}%
\pgfsetdash{}{0pt}%
\pgfpathmoveto{\pgfqpoint{0.640323in}{0.527436in}}%
\pgfpathlineto{\pgfqpoint{0.640323in}{4.762436in}}%
\pgfusepath{stroke}%
\end{pgfscope}%
\begin{pgfscope}%
\pgfsetrectcap%
\pgfsetmiterjoin%
\pgfsetlinewidth{0.803000pt}%
\definecolor{currentstroke}{rgb}{0.000000,0.000000,0.000000}%
\pgfsetstrokecolor{currentstroke}%
\pgfsetdash{}{0pt}%
\pgfpathmoveto{\pgfqpoint{5.043732in}{0.527436in}}%
\pgfpathlineto{\pgfqpoint{5.043732in}{4.762436in}}%
\pgfusepath{stroke}%
\end{pgfscope}%
\begin{pgfscope}%
\pgfsetrectcap%
\pgfsetmiterjoin%
\pgfsetlinewidth{0.803000pt}%
\definecolor{currentstroke}{rgb}{0.000000,0.000000,0.000000}%
\pgfsetstrokecolor{currentstroke}%
\pgfsetdash{}{0pt}%
\pgfpathmoveto{\pgfqpoint{0.640323in}{0.527436in}}%
\pgfpathlineto{\pgfqpoint{5.043732in}{0.527436in}}%
\pgfusepath{stroke}%
\end{pgfscope}%
\begin{pgfscope}%
\pgfsetrectcap%
\pgfsetmiterjoin%
\pgfsetlinewidth{0.803000pt}%
\definecolor{currentstroke}{rgb}{0.000000,0.000000,0.000000}%
\pgfsetstrokecolor{currentstroke}%
\pgfsetdash{}{0pt}%
\pgfpathmoveto{\pgfqpoint{0.640323in}{4.762436in}}%
\pgfpathlineto{\pgfqpoint{5.043732in}{4.762436in}}%
\pgfusepath{stroke}%
\end{pgfscope}%
\begin{pgfscope}%
\definecolor{textcolor}{rgb}{0.000000,0.000000,0.000000}%
\pgfsetstrokecolor{textcolor}%
\pgfsetfillcolor{textcolor}%
\pgftext[x=2.842027in,y=4.845769in,,base]{\color{textcolor}\sffamily\fontsize{12.000000}{14.400000}\selectfont \(\displaystyle \overline{\langle I\rangle}\) over \(\displaystyle p_1\) for \(\displaystyle T=1000\) for \(\displaystyle L=128\)}%
\end{pgfscope}%
\begin{pgfscope}%
\pgfsetbuttcap%
\pgfsetmiterjoin%
\definecolor{currentfill}{rgb}{1.000000,1.000000,1.000000}%
\pgfsetfillcolor{currentfill}%
\pgfsetfillopacity{0.800000}%
\pgfsetlinewidth{1.003750pt}%
\definecolor{currentstroke}{rgb}{0.800000,0.800000,0.800000}%
\pgfsetstrokecolor{currentstroke}%
\pgfsetstrokeopacity{0.800000}%
\pgfsetdash{}{0pt}%
\pgfpathmoveto{\pgfqpoint{0.737545in}{4.039753in}}%
\pgfpathlineto{\pgfqpoint{2.259965in}{4.039753in}}%
\pgfpathquadraticcurveto{\pgfqpoint{2.287743in}{4.039753in}}{\pgfqpoint{2.287743in}{4.067531in}}%
\pgfpathlineto{\pgfqpoint{2.287743in}{4.665214in}}%
\pgfpathquadraticcurveto{\pgfqpoint{2.287743in}{4.692991in}}{\pgfqpoint{2.259965in}{4.692991in}}%
\pgfpathlineto{\pgfqpoint{0.737545in}{4.692991in}}%
\pgfpathquadraticcurveto{\pgfqpoint{0.709767in}{4.692991in}}{\pgfqpoint{0.709767in}{4.665214in}}%
\pgfpathlineto{\pgfqpoint{0.709767in}{4.067531in}}%
\pgfpathquadraticcurveto{\pgfqpoint{0.709767in}{4.039753in}}{\pgfqpoint{0.737545in}{4.039753in}}%
\pgfpathlineto{\pgfqpoint{0.737545in}{4.039753in}}%
\pgfpathclose%
\pgfusepath{stroke,fill}%
\end{pgfscope}%
\begin{pgfscope}%
\pgfsetbuttcap%
\pgfsetroundjoin%
\definecolor{currentfill}{rgb}{0.980392,0.164706,0.333333}%
\pgfsetfillcolor{currentfill}%
\pgfsetfillopacity{0.500000}%
\pgfsetlinewidth{1.003750pt}%
\definecolor{currentstroke}{rgb}{0.980392,0.164706,0.333333}%
\pgfsetstrokecolor{currentstroke}%
\pgfsetstrokeopacity{0.500000}%
\pgfsetdash{}{0pt}%
\pgfsys@defobject{currentmarker}{\pgfqpoint{-0.021960in}{-0.021960in}}{\pgfqpoint{0.021960in}{0.021960in}}{%
\pgfpathmoveto{\pgfqpoint{0.000000in}{-0.021960in}}%
\pgfpathcurveto{\pgfqpoint{0.005824in}{-0.021960in}}{\pgfqpoint{0.011410in}{-0.019646in}}{\pgfqpoint{0.015528in}{-0.015528in}}%
\pgfpathcurveto{\pgfqpoint{0.019646in}{-0.011410in}}{\pgfqpoint{0.021960in}{-0.005824in}}{\pgfqpoint{0.021960in}{0.000000in}}%
\pgfpathcurveto{\pgfqpoint{0.021960in}{0.005824in}}{\pgfqpoint{0.019646in}{0.011410in}}{\pgfqpoint{0.015528in}{0.015528in}}%
\pgfpathcurveto{\pgfqpoint{0.011410in}{0.019646in}}{\pgfqpoint{0.005824in}{0.021960in}}{\pgfqpoint{0.000000in}{0.021960in}}%
\pgfpathcurveto{\pgfqpoint{-0.005824in}{0.021960in}}{\pgfqpoint{-0.011410in}{0.019646in}}{\pgfqpoint{-0.015528in}{0.015528in}}%
\pgfpathcurveto{\pgfqpoint{-0.019646in}{0.011410in}}{\pgfqpoint{-0.021960in}{0.005824in}}{\pgfqpoint{-0.021960in}{0.000000in}}%
\pgfpathcurveto{\pgfqpoint{-0.021960in}{-0.005824in}}{\pgfqpoint{-0.019646in}{-0.011410in}}{\pgfqpoint{-0.015528in}{-0.015528in}}%
\pgfpathcurveto{\pgfqpoint{-0.011410in}{-0.019646in}}{\pgfqpoint{-0.005824in}{-0.021960in}}{\pgfqpoint{0.000000in}{-0.021960in}}%
\pgfpathlineto{\pgfqpoint{0.000000in}{-0.021960in}}%
\pgfpathclose%
\pgfusepath{stroke,fill}%
}%
\begin{pgfscope}%
\pgfsys@transformshift{0.904211in}{4.568371in}%
\pgfsys@useobject{currentmarker}{}%
\end{pgfscope}%
\end{pgfscope}%
\begin{pgfscope}%
\definecolor{textcolor}{rgb}{0.000000,0.000000,0.000000}%
\pgfsetstrokecolor{textcolor}%
\pgfsetfillcolor{textcolor}%
\pgftext[x=1.154211in,y=4.531913in,left,base]{\color{textcolor}\sffamily\fontsize{10.000000}{12.000000}\selectfont \(\displaystyle p_2=0.3\), \(\displaystyle p_3=0.3\)}%
\end{pgfscope}%
\begin{pgfscope}%
\pgfsetbuttcap%
\pgfsetroundjoin%
\definecolor{currentfill}{rgb}{0.000000,0.000000,1.000000}%
\pgfsetfillcolor{currentfill}%
\pgfsetfillopacity{0.500000}%
\pgfsetlinewidth{1.003750pt}%
\definecolor{currentstroke}{rgb}{0.000000,0.000000,1.000000}%
\pgfsetstrokecolor{currentstroke}%
\pgfsetstrokeopacity{0.500000}%
\pgfsetdash{}{0pt}%
\pgfsys@defobject{currentmarker}{\pgfqpoint{-0.021960in}{-0.021960in}}{\pgfqpoint{0.021960in}{0.021960in}}{%
\pgfpathmoveto{\pgfqpoint{0.000000in}{-0.021960in}}%
\pgfpathcurveto{\pgfqpoint{0.005824in}{-0.021960in}}{\pgfqpoint{0.011410in}{-0.019646in}}{\pgfqpoint{0.015528in}{-0.015528in}}%
\pgfpathcurveto{\pgfqpoint{0.019646in}{-0.011410in}}{\pgfqpoint{0.021960in}{-0.005824in}}{\pgfqpoint{0.021960in}{0.000000in}}%
\pgfpathcurveto{\pgfqpoint{0.021960in}{0.005824in}}{\pgfqpoint{0.019646in}{0.011410in}}{\pgfqpoint{0.015528in}{0.015528in}}%
\pgfpathcurveto{\pgfqpoint{0.011410in}{0.019646in}}{\pgfqpoint{0.005824in}{0.021960in}}{\pgfqpoint{0.000000in}{0.021960in}}%
\pgfpathcurveto{\pgfqpoint{-0.005824in}{0.021960in}}{\pgfqpoint{-0.011410in}{0.019646in}}{\pgfqpoint{-0.015528in}{0.015528in}}%
\pgfpathcurveto{\pgfqpoint{-0.019646in}{0.011410in}}{\pgfqpoint{-0.021960in}{0.005824in}}{\pgfqpoint{-0.021960in}{0.000000in}}%
\pgfpathcurveto{\pgfqpoint{-0.021960in}{-0.005824in}}{\pgfqpoint{-0.019646in}{-0.011410in}}{\pgfqpoint{-0.015528in}{-0.015528in}}%
\pgfpathcurveto{\pgfqpoint{-0.011410in}{-0.019646in}}{\pgfqpoint{-0.005824in}{-0.021960in}}{\pgfqpoint{0.000000in}{-0.021960in}}%
\pgfpathlineto{\pgfqpoint{0.000000in}{-0.021960in}}%
\pgfpathclose%
\pgfusepath{stroke,fill}%
}%
\begin{pgfscope}%
\pgfsys@transformshift{0.904211in}{4.364514in}%
\pgfsys@useobject{currentmarker}{}%
\end{pgfscope}%
\end{pgfscope}%
\begin{pgfscope}%
\definecolor{textcolor}{rgb}{0.000000,0.000000,0.000000}%
\pgfsetstrokecolor{textcolor}%
\pgfsetfillcolor{textcolor}%
\pgftext[x=1.154211in,y=4.328056in,left,base]{\color{textcolor}\sffamily\fontsize{10.000000}{12.000000}\selectfont \(\displaystyle p_2=0.6\), \(\displaystyle p_3=0.3\)}%
\end{pgfscope}%
\begin{pgfscope}%
\pgfsetbuttcap%
\pgfsetroundjoin%
\definecolor{currentfill}{rgb}{0.000000,0.000000,0.000000}%
\pgfsetfillcolor{currentfill}%
\pgfsetfillopacity{0.500000}%
\pgfsetlinewidth{1.003750pt}%
\definecolor{currentstroke}{rgb}{0.000000,0.000000,0.000000}%
\pgfsetstrokecolor{currentstroke}%
\pgfsetstrokeopacity{0.500000}%
\pgfsetdash{}{0pt}%
\pgfsys@defobject{currentmarker}{\pgfqpoint{-0.021960in}{-0.021960in}}{\pgfqpoint{0.021960in}{0.021960in}}{%
\pgfpathmoveto{\pgfqpoint{0.000000in}{-0.021960in}}%
\pgfpathcurveto{\pgfqpoint{0.005824in}{-0.021960in}}{\pgfqpoint{0.011410in}{-0.019646in}}{\pgfqpoint{0.015528in}{-0.015528in}}%
\pgfpathcurveto{\pgfqpoint{0.019646in}{-0.011410in}}{\pgfqpoint{0.021960in}{-0.005824in}}{\pgfqpoint{0.021960in}{0.000000in}}%
\pgfpathcurveto{\pgfqpoint{0.021960in}{0.005824in}}{\pgfqpoint{0.019646in}{0.011410in}}{\pgfqpoint{0.015528in}{0.015528in}}%
\pgfpathcurveto{\pgfqpoint{0.011410in}{0.019646in}}{\pgfqpoint{0.005824in}{0.021960in}}{\pgfqpoint{0.000000in}{0.021960in}}%
\pgfpathcurveto{\pgfqpoint{-0.005824in}{0.021960in}}{\pgfqpoint{-0.011410in}{0.019646in}}{\pgfqpoint{-0.015528in}{0.015528in}}%
\pgfpathcurveto{\pgfqpoint{-0.019646in}{0.011410in}}{\pgfqpoint{-0.021960in}{0.005824in}}{\pgfqpoint{-0.021960in}{0.000000in}}%
\pgfpathcurveto{\pgfqpoint{-0.021960in}{-0.005824in}}{\pgfqpoint{-0.019646in}{-0.011410in}}{\pgfqpoint{-0.015528in}{-0.015528in}}%
\pgfpathcurveto{\pgfqpoint{-0.011410in}{-0.019646in}}{\pgfqpoint{-0.005824in}{-0.021960in}}{\pgfqpoint{0.000000in}{-0.021960in}}%
\pgfpathlineto{\pgfqpoint{0.000000in}{-0.021960in}}%
\pgfpathclose%
\pgfusepath{stroke,fill}%
}%
\begin{pgfscope}%
\pgfsys@transformshift{0.904211in}{4.160657in}%
\pgfsys@useobject{currentmarker}{}%
\end{pgfscope}%
\end{pgfscope}%
\begin{pgfscope}%
\definecolor{textcolor}{rgb}{0.000000,0.000000,0.000000}%
\pgfsetstrokecolor{textcolor}%
\pgfsetfillcolor{textcolor}%
\pgftext[x=1.154211in,y=4.124198in,left,base]{\color{textcolor}\sffamily\fontsize{10.000000}{12.000000}\selectfont \(\displaystyle p_2=0.3\), \(\displaystyle p_3=0.6\)}%
\end{pgfscope}%
\begin{pgfscope}%
\pgfsetbuttcap%
\pgfsetmiterjoin%
\definecolor{currentfill}{rgb}{1.000000,1.000000,1.000000}%
\pgfsetfillcolor{currentfill}%
\pgfsetlinewidth{0.000000pt}%
\definecolor{currentstroke}{rgb}{0.000000,0.000000,0.000000}%
\pgfsetstrokecolor{currentstroke}%
\pgfsetstrokeopacity{0.000000}%
\pgfsetdash{}{0pt}%
\pgfpathmoveto{\pgfqpoint{5.924413in}{0.527436in}}%
\pgfpathlineto{\pgfqpoint{10.327822in}{0.527436in}}%
\pgfpathlineto{\pgfqpoint{10.327822in}{4.762436in}}%
\pgfpathlineto{\pgfqpoint{5.924413in}{4.762436in}}%
\pgfpathlineto{\pgfqpoint{5.924413in}{0.527436in}}%
\pgfpathclose%
\pgfusepath{fill}%
\end{pgfscope}%
\begin{pgfscope}%
\pgfpathrectangle{\pgfqpoint{5.924413in}{0.527436in}}{\pgfqpoint{4.403409in}{4.235000in}}%
\pgfusepath{clip}%
\pgfsetbuttcap%
\pgfsetroundjoin%
\definecolor{currentfill}{rgb}{0.980392,0.164706,0.333333}%
\pgfsetfillcolor{currentfill}%
\pgfsetfillopacity{0.500000}%
\pgfsetlinewidth{1.003750pt}%
\definecolor{currentstroke}{rgb}{0.980392,0.164706,0.333333}%
\pgfsetstrokecolor{currentstroke}%
\pgfsetstrokeopacity{0.500000}%
\pgfsetdash{}{0pt}%
\pgfsys@defobject{currentmarker}{\pgfqpoint{-0.021960in}{-0.021960in}}{\pgfqpoint{0.021960in}{0.021960in}}{%
\pgfpathmoveto{\pgfqpoint{0.000000in}{-0.021960in}}%
\pgfpathcurveto{\pgfqpoint{0.005824in}{-0.021960in}}{\pgfqpoint{0.011410in}{-0.019646in}}{\pgfqpoint{0.015528in}{-0.015528in}}%
\pgfpathcurveto{\pgfqpoint{0.019646in}{-0.011410in}}{\pgfqpoint{0.021960in}{-0.005824in}}{\pgfqpoint{0.021960in}{0.000000in}}%
\pgfpathcurveto{\pgfqpoint{0.021960in}{0.005824in}}{\pgfqpoint{0.019646in}{0.011410in}}{\pgfqpoint{0.015528in}{0.015528in}}%
\pgfpathcurveto{\pgfqpoint{0.011410in}{0.019646in}}{\pgfqpoint{0.005824in}{0.021960in}}{\pgfqpoint{0.000000in}{0.021960in}}%
\pgfpathcurveto{\pgfqpoint{-0.005824in}{0.021960in}}{\pgfqpoint{-0.011410in}{0.019646in}}{\pgfqpoint{-0.015528in}{0.015528in}}%
\pgfpathcurveto{\pgfqpoint{-0.019646in}{0.011410in}}{\pgfqpoint{-0.021960in}{0.005824in}}{\pgfqpoint{-0.021960in}{0.000000in}}%
\pgfpathcurveto{\pgfqpoint{-0.021960in}{-0.005824in}}{\pgfqpoint{-0.019646in}{-0.011410in}}{\pgfqpoint{-0.015528in}{-0.015528in}}%
\pgfpathcurveto{\pgfqpoint{-0.011410in}{-0.019646in}}{\pgfqpoint{-0.005824in}{-0.021960in}}{\pgfqpoint{0.000000in}{-0.021960in}}%
\pgfpathlineto{\pgfqpoint{0.000000in}{-0.021960in}}%
\pgfpathclose%
\pgfusepath{stroke,fill}%
}%
\begin{pgfscope}%
\pgfsys@transformshift{6.010755in}{0.962962in}%
\pgfsys@useobject{currentmarker}{}%
\end{pgfscope}%
\begin{pgfscope}%
\pgfsys@transformshift{6.183437in}{0.967422in}%
\pgfsys@useobject{currentmarker}{}%
\end{pgfscope}%
\begin{pgfscope}%
\pgfsys@transformshift{6.356120in}{0.975468in}%
\pgfsys@useobject{currentmarker}{}%
\end{pgfscope}%
\begin{pgfscope}%
\pgfsys@transformshift{6.528803in}{0.985356in}%
\pgfsys@useobject{currentmarker}{}%
\end{pgfscope}%
\begin{pgfscope}%
\pgfsys@transformshift{6.701486in}{1.000386in}%
\pgfsys@useobject{currentmarker}{}%
\end{pgfscope}%
\begin{pgfscope}%
\pgfsys@transformshift{6.874168in}{1.016682in}%
\pgfsys@useobject{currentmarker}{}%
\end{pgfscope}%
\begin{pgfscope}%
\pgfsys@transformshift{7.046851in}{1.047882in}%
\pgfsys@useobject{currentmarker}{}%
\end{pgfscope}%
\begin{pgfscope}%
\pgfsys@transformshift{7.219534in}{1.093109in}%
\pgfsys@useobject{currentmarker}{}%
\end{pgfscope}%
\begin{pgfscope}%
\pgfsys@transformshift{7.392216in}{1.232039in}%
\pgfsys@useobject{currentmarker}{}%
\end{pgfscope}%
\begin{pgfscope}%
\pgfsys@transformshift{7.564899in}{1.972764in}%
\pgfsys@useobject{currentmarker}{}%
\end{pgfscope}%
\begin{pgfscope}%
\pgfsys@transformshift{7.737582in}{4.820802in}%
\pgfsys@useobject{currentmarker}{}%
\end{pgfscope}%
\begin{pgfscope}%
\pgfsys@transformshift{7.910265in}{6.646164in}%
\pgfsys@useobject{currentmarker}{}%
\end{pgfscope}%
\begin{pgfscope}%
\pgfsys@transformshift{8.082947in}{8.032626in}%
\pgfsys@useobject{currentmarker}{}%
\end{pgfscope}%
\begin{pgfscope}%
\pgfsys@transformshift{8.255630in}{9.094764in}%
\pgfsys@useobject{currentmarker}{}%
\end{pgfscope}%
\begin{pgfscope}%
\pgfsys@transformshift{8.428313in}{9.902956in}%
\pgfsys@useobject{currentmarker}{}%
\end{pgfscope}%
\begin{pgfscope}%
\pgfsys@transformshift{8.600995in}{10.620711in}%
\pgfsys@useobject{currentmarker}{}%
\end{pgfscope}%
\begin{pgfscope}%
\pgfsys@transformshift{8.773678in}{11.216691in}%
\pgfsys@useobject{currentmarker}{}%
\end{pgfscope}%
\begin{pgfscope}%
\pgfsys@transformshift{8.946361in}{11.718346in}%
\pgfsys@useobject{currentmarker}{}%
\end{pgfscope}%
\begin{pgfscope}%
\pgfsys@transformshift{9.119044in}{12.177497in}%
\pgfsys@useobject{currentmarker}{}%
\end{pgfscope}%
\begin{pgfscope}%
\pgfsys@transformshift{9.291726in}{12.530889in}%
\pgfsys@useobject{currentmarker}{}%
\end{pgfscope}%
\begin{pgfscope}%
\pgfsys@transformshift{9.464409in}{12.912963in}%
\pgfsys@useobject{currentmarker}{}%
\end{pgfscope}%
\begin{pgfscope}%
\pgfsys@transformshift{9.637092in}{13.246719in}%
\pgfsys@useobject{currentmarker}{}%
\end{pgfscope}%
\begin{pgfscope}%
\pgfsys@transformshift{9.809774in}{13.523496in}%
\pgfsys@useobject{currentmarker}{}%
\end{pgfscope}%
\begin{pgfscope}%
\pgfsys@transformshift{9.982457in}{13.785796in}%
\pgfsys@useobject{currentmarker}{}%
\end{pgfscope}%
\begin{pgfscope}%
\pgfsys@transformshift{10.155140in}{14.010829in}%
\pgfsys@useobject{currentmarker}{}%
\end{pgfscope}%
\begin{pgfscope}%
\pgfsys@transformshift{10.327822in}{14.238479in}%
\pgfsys@useobject{currentmarker}{}%
\end{pgfscope}%
\begin{pgfscope}%
\pgfsys@transformshift{10.500505in}{14.449844in}%
\pgfsys@useobject{currentmarker}{}%
\end{pgfscope}%
\begin{pgfscope}%
\pgfsys@transformshift{10.673188in}{14.622478in}%
\pgfsys@useobject{currentmarker}{}%
\end{pgfscope}%
\begin{pgfscope}%
\pgfsys@transformshift{10.845871in}{14.808741in}%
\pgfsys@useobject{currentmarker}{}%
\end{pgfscope}%
\begin{pgfscope}%
\pgfsys@transformshift{11.018553in}{14.994234in}%
\pgfsys@useobject{currentmarker}{}%
\end{pgfscope}%
\begin{pgfscope}%
\pgfsys@transformshift{11.191236in}{15.151430in}%
\pgfsys@useobject{currentmarker}{}%
\end{pgfscope}%
\begin{pgfscope}%
\pgfsys@transformshift{11.363919in}{15.295112in}%
\pgfsys@useobject{currentmarker}{}%
\end{pgfscope}%
\begin{pgfscope}%
\pgfsys@transformshift{11.536601in}{15.428514in}%
\pgfsys@useobject{currentmarker}{}%
\end{pgfscope}%
\begin{pgfscope}%
\pgfsys@transformshift{11.709284in}{15.556565in}%
\pgfsys@useobject{currentmarker}{}%
\end{pgfscope}%
\begin{pgfscope}%
\pgfsys@transformshift{11.881967in}{15.680805in}%
\pgfsys@useobject{currentmarker}{}%
\end{pgfscope}%
\begin{pgfscope}%
\pgfsys@transformshift{12.054650in}{15.786795in}%
\pgfsys@useobject{currentmarker}{}%
\end{pgfscope}%
\begin{pgfscope}%
\pgfsys@transformshift{12.227332in}{15.908109in}%
\pgfsys@useobject{currentmarker}{}%
\end{pgfscope}%
\begin{pgfscope}%
\pgfsys@transformshift{12.400015in}{15.985725in}%
\pgfsys@useobject{currentmarker}{}%
\end{pgfscope}%
\begin{pgfscope}%
\pgfsys@transformshift{12.572698in}{16.108270in}%
\pgfsys@useobject{currentmarker}{}%
\end{pgfscope}%
\begin{pgfscope}%
\pgfsys@transformshift{12.745380in}{16.189313in}%
\pgfsys@useobject{currentmarker}{}%
\end{pgfscope}%
\begin{pgfscope}%
\pgfsys@transformshift{12.918063in}{16.280635in}%
\pgfsys@useobject{currentmarker}{}%
\end{pgfscope}%
\begin{pgfscope}%
\pgfsys@transformshift{13.090746in}{16.376923in}%
\pgfsys@useobject{currentmarker}{}%
\end{pgfscope}%
\begin{pgfscope}%
\pgfsys@transformshift{13.263429in}{16.438100in}%
\pgfsys@useobject{currentmarker}{}%
\end{pgfscope}%
\begin{pgfscope}%
\pgfsys@transformshift{13.436111in}{16.536390in}%
\pgfsys@useobject{currentmarker}{}%
\end{pgfscope}%
\begin{pgfscope}%
\pgfsys@transformshift{13.608794in}{16.595642in}%
\pgfsys@useobject{currentmarker}{}%
\end{pgfscope}%
\begin{pgfscope}%
\pgfsys@transformshift{13.781477in}{16.673527in}%
\pgfsys@useobject{currentmarker}{}%
\end{pgfscope}%
\begin{pgfscope}%
\pgfsys@transformshift{13.954159in}{16.750065in}%
\pgfsys@useobject{currentmarker}{}%
\end{pgfscope}%
\begin{pgfscope}%
\pgfsys@transformshift{14.126842in}{16.798036in}%
\pgfsys@useobject{currentmarker}{}%
\end{pgfscope}%
\begin{pgfscope}%
\pgfsys@transformshift{14.299525in}{16.856941in}%
\pgfsys@useobject{currentmarker}{}%
\end{pgfscope}%
\begin{pgfscope}%
\pgfsys@transformshift{14.472208in}{16.924470in}%
\pgfsys@useobject{currentmarker}{}%
\end{pgfscope}%
\begin{pgfscope}%
\pgfsys@transformshift{14.644890in}{16.988919in}%
\pgfsys@useobject{currentmarker}{}%
\end{pgfscope}%
\end{pgfscope}%
\begin{pgfscope}%
\pgfpathrectangle{\pgfqpoint{5.924413in}{0.527436in}}{\pgfqpoint{4.403409in}{4.235000in}}%
\pgfusepath{clip}%
\pgfsetbuttcap%
\pgfsetroundjoin%
\definecolor{currentfill}{rgb}{0.000000,0.000000,1.000000}%
\pgfsetfillcolor{currentfill}%
\pgfsetfillopacity{0.500000}%
\pgfsetlinewidth{1.003750pt}%
\definecolor{currentstroke}{rgb}{0.000000,0.000000,1.000000}%
\pgfsetstrokecolor{currentstroke}%
\pgfsetstrokeopacity{0.500000}%
\pgfsetdash{}{0pt}%
\pgfsys@defobject{currentmarker}{\pgfqpoint{-0.021960in}{-0.021960in}}{\pgfqpoint{0.021960in}{0.021960in}}{%
\pgfpathmoveto{\pgfqpoint{0.000000in}{-0.021960in}}%
\pgfpathcurveto{\pgfqpoint{0.005824in}{-0.021960in}}{\pgfqpoint{0.011410in}{-0.019646in}}{\pgfqpoint{0.015528in}{-0.015528in}}%
\pgfpathcurveto{\pgfqpoint{0.019646in}{-0.011410in}}{\pgfqpoint{0.021960in}{-0.005824in}}{\pgfqpoint{0.021960in}{0.000000in}}%
\pgfpathcurveto{\pgfqpoint{0.021960in}{0.005824in}}{\pgfqpoint{0.019646in}{0.011410in}}{\pgfqpoint{0.015528in}{0.015528in}}%
\pgfpathcurveto{\pgfqpoint{0.011410in}{0.019646in}}{\pgfqpoint{0.005824in}{0.021960in}}{\pgfqpoint{0.000000in}{0.021960in}}%
\pgfpathcurveto{\pgfqpoint{-0.005824in}{0.021960in}}{\pgfqpoint{-0.011410in}{0.019646in}}{\pgfqpoint{-0.015528in}{0.015528in}}%
\pgfpathcurveto{\pgfqpoint{-0.019646in}{0.011410in}}{\pgfqpoint{-0.021960in}{0.005824in}}{\pgfqpoint{-0.021960in}{0.000000in}}%
\pgfpathcurveto{\pgfqpoint{-0.021960in}{-0.005824in}}{\pgfqpoint{-0.019646in}{-0.011410in}}{\pgfqpoint{-0.015528in}{-0.015528in}}%
\pgfpathcurveto{\pgfqpoint{-0.011410in}{-0.019646in}}{\pgfqpoint{-0.005824in}{-0.021960in}}{\pgfqpoint{0.000000in}{-0.021960in}}%
\pgfpathlineto{\pgfqpoint{0.000000in}{-0.021960in}}%
\pgfpathclose%
\pgfusepath{stroke,fill}%
}%
\begin{pgfscope}%
\pgfsys@transformshift{6.010755in}{0.940989in}%
\pgfsys@useobject{currentmarker}{}%
\end{pgfscope}%
\begin{pgfscope}%
\pgfsys@transformshift{6.183437in}{0.940599in}%
\pgfsys@useobject{currentmarker}{}%
\end{pgfscope}%
\begin{pgfscope}%
\pgfsys@transformshift{6.356120in}{0.943188in}%
\pgfsys@useobject{currentmarker}{}%
\end{pgfscope}%
\begin{pgfscope}%
\pgfsys@transformshift{6.528803in}{0.945155in}%
\pgfsys@useobject{currentmarker}{}%
\end{pgfscope}%
\begin{pgfscope}%
\pgfsys@transformshift{6.701486in}{0.946979in}%
\pgfsys@useobject{currentmarker}{}%
\end{pgfscope}%
\begin{pgfscope}%
\pgfsys@transformshift{6.874168in}{0.948558in}%
\pgfsys@useobject{currentmarker}{}%
\end{pgfscope}%
\begin{pgfscope}%
\pgfsys@transformshift{7.046851in}{0.950224in}%
\pgfsys@useobject{currentmarker}{}%
\end{pgfscope}%
\begin{pgfscope}%
\pgfsys@transformshift{7.219534in}{0.951171in}%
\pgfsys@useobject{currentmarker}{}%
\end{pgfscope}%
\begin{pgfscope}%
\pgfsys@transformshift{7.392216in}{0.954959in}%
\pgfsys@useobject{currentmarker}{}%
\end{pgfscope}%
\begin{pgfscope}%
\pgfsys@transformshift{7.564899in}{0.957684in}%
\pgfsys@useobject{currentmarker}{}%
\end{pgfscope}%
\begin{pgfscope}%
\pgfsys@transformshift{7.737582in}{0.958533in}%
\pgfsys@useobject{currentmarker}{}%
\end{pgfscope}%
\begin{pgfscope}%
\pgfsys@transformshift{7.910265in}{0.962908in}%
\pgfsys@useobject{currentmarker}{}%
\end{pgfscope}%
\begin{pgfscope}%
\pgfsys@transformshift{8.082947in}{0.966363in}%
\pgfsys@useobject{currentmarker}{}%
\end{pgfscope}%
\begin{pgfscope}%
\pgfsys@transformshift{8.255630in}{0.970505in}%
\pgfsys@useobject{currentmarker}{}%
\end{pgfscope}%
\begin{pgfscope}%
\pgfsys@transformshift{8.428313in}{0.976744in}%
\pgfsys@useobject{currentmarker}{}%
\end{pgfscope}%
\begin{pgfscope}%
\pgfsys@transformshift{8.600995in}{0.980962in}%
\pgfsys@useobject{currentmarker}{}%
\end{pgfscope}%
\begin{pgfscope}%
\pgfsys@transformshift{8.773678in}{0.986287in}%
\pgfsys@useobject{currentmarker}{}%
\end{pgfscope}%
\begin{pgfscope}%
\pgfsys@transformshift{8.946361in}{1.002722in}%
\pgfsys@useobject{currentmarker}{}%
\end{pgfscope}%
\begin{pgfscope}%
\pgfsys@transformshift{9.119044in}{1.014624in}%
\pgfsys@useobject{currentmarker}{}%
\end{pgfscope}%
\begin{pgfscope}%
\pgfsys@transformshift{9.291726in}{1.051371in}%
\pgfsys@useobject{currentmarker}{}%
\end{pgfscope}%
\begin{pgfscope}%
\pgfsys@transformshift{9.464409in}{1.103136in}%
\pgfsys@useobject{currentmarker}{}%
\end{pgfscope}%
\begin{pgfscope}%
\pgfsys@transformshift{9.637092in}{1.216088in}%
\pgfsys@useobject{currentmarker}{}%
\end{pgfscope}%
\begin{pgfscope}%
\pgfsys@transformshift{9.809774in}{1.956787in}%
\pgfsys@useobject{currentmarker}{}%
\end{pgfscope}%
\begin{pgfscope}%
\pgfsys@transformshift{9.982457in}{2.785592in}%
\pgfsys@useobject{currentmarker}{}%
\end{pgfscope}%
\begin{pgfscope}%
\pgfsys@transformshift{10.155140in}{3.606077in}%
\pgfsys@useobject{currentmarker}{}%
\end{pgfscope}%
\begin{pgfscope}%
\pgfsys@transformshift{10.327822in}{4.300501in}%
\pgfsys@useobject{currentmarker}{}%
\end{pgfscope}%
\begin{pgfscope}%
\pgfsys@transformshift{10.500505in}{4.857685in}%
\pgfsys@useobject{currentmarker}{}%
\end{pgfscope}%
\begin{pgfscope}%
\pgfsys@transformshift{10.673188in}{5.207534in}%
\pgfsys@useobject{currentmarker}{}%
\end{pgfscope}%
\begin{pgfscope}%
\pgfsys@transformshift{10.845871in}{5.611361in}%
\pgfsys@useobject{currentmarker}{}%
\end{pgfscope}%
\begin{pgfscope}%
\pgfsys@transformshift{11.018553in}{5.957591in}%
\pgfsys@useobject{currentmarker}{}%
\end{pgfscope}%
\begin{pgfscope}%
\pgfsys@transformshift{11.191236in}{6.276525in}%
\pgfsys@useobject{currentmarker}{}%
\end{pgfscope}%
\begin{pgfscope}%
\pgfsys@transformshift{11.363919in}{6.554919in}%
\pgfsys@useobject{currentmarker}{}%
\end{pgfscope}%
\begin{pgfscope}%
\pgfsys@transformshift{11.536601in}{6.807325in}%
\pgfsys@useobject{currentmarker}{}%
\end{pgfscope}%
\begin{pgfscope}%
\pgfsys@transformshift{11.709284in}{7.010528in}%
\pgfsys@useobject{currentmarker}{}%
\end{pgfscope}%
\begin{pgfscope}%
\pgfsys@transformshift{11.881967in}{7.207109in}%
\pgfsys@useobject{currentmarker}{}%
\end{pgfscope}%
\begin{pgfscope}%
\pgfsys@transformshift{12.054650in}{7.445616in}%
\pgfsys@useobject{currentmarker}{}%
\end{pgfscope}%
\begin{pgfscope}%
\pgfsys@transformshift{12.227332in}{7.610858in}%
\pgfsys@useobject{currentmarker}{}%
\end{pgfscope}%
\begin{pgfscope}%
\pgfsys@transformshift{12.400015in}{7.794388in}%
\pgfsys@useobject{currentmarker}{}%
\end{pgfscope}%
\begin{pgfscope}%
\pgfsys@transformshift{12.572698in}{7.920321in}%
\pgfsys@useobject{currentmarker}{}%
\end{pgfscope}%
\begin{pgfscope}%
\pgfsys@transformshift{12.745380in}{8.086603in}%
\pgfsys@useobject{currentmarker}{}%
\end{pgfscope}%
\begin{pgfscope}%
\pgfsys@transformshift{12.918063in}{8.222546in}%
\pgfsys@useobject{currentmarker}{}%
\end{pgfscope}%
\begin{pgfscope}%
\pgfsys@transformshift{13.090746in}{8.373274in}%
\pgfsys@useobject{currentmarker}{}%
\end{pgfscope}%
\begin{pgfscope}%
\pgfsys@transformshift{13.263429in}{8.485694in}%
\pgfsys@useobject{currentmarker}{}%
\end{pgfscope}%
\begin{pgfscope}%
\pgfsys@transformshift{13.436111in}{8.602849in}%
\pgfsys@useobject{currentmarker}{}%
\end{pgfscope}%
\begin{pgfscope}%
\pgfsys@transformshift{13.608794in}{8.715731in}%
\pgfsys@useobject{currentmarker}{}%
\end{pgfscope}%
\begin{pgfscope}%
\pgfsys@transformshift{13.781477in}{8.820182in}%
\pgfsys@useobject{currentmarker}{}%
\end{pgfscope}%
\begin{pgfscope}%
\pgfsys@transformshift{13.954159in}{8.926134in}%
\pgfsys@useobject{currentmarker}{}%
\end{pgfscope}%
\begin{pgfscope}%
\pgfsys@transformshift{14.126842in}{9.016840in}%
\pgfsys@useobject{currentmarker}{}%
\end{pgfscope}%
\begin{pgfscope}%
\pgfsys@transformshift{14.299525in}{9.090105in}%
\pgfsys@useobject{currentmarker}{}%
\end{pgfscope}%
\begin{pgfscope}%
\pgfsys@transformshift{14.472208in}{9.187857in}%
\pgfsys@useobject{currentmarker}{}%
\end{pgfscope}%
\begin{pgfscope}%
\pgfsys@transformshift{14.644890in}{9.278563in}%
\pgfsys@useobject{currentmarker}{}%
\end{pgfscope}%
\end{pgfscope}%
\begin{pgfscope}%
\pgfpathrectangle{\pgfqpoint{5.924413in}{0.527436in}}{\pgfqpoint{4.403409in}{4.235000in}}%
\pgfusepath{clip}%
\pgfsetbuttcap%
\pgfsetroundjoin%
\definecolor{currentfill}{rgb}{0.000000,0.000000,0.000000}%
\pgfsetfillcolor{currentfill}%
\pgfsetfillopacity{0.500000}%
\pgfsetlinewidth{1.003750pt}%
\definecolor{currentstroke}{rgb}{0.000000,0.000000,0.000000}%
\pgfsetstrokecolor{currentstroke}%
\pgfsetstrokeopacity{0.500000}%
\pgfsetdash{}{0pt}%
\pgfsys@defobject{currentmarker}{\pgfqpoint{-0.021960in}{-0.021960in}}{\pgfqpoint{0.021960in}{0.021960in}}{%
\pgfpathmoveto{\pgfqpoint{0.000000in}{-0.021960in}}%
\pgfpathcurveto{\pgfqpoint{0.005824in}{-0.021960in}}{\pgfqpoint{0.011410in}{-0.019646in}}{\pgfqpoint{0.015528in}{-0.015528in}}%
\pgfpathcurveto{\pgfqpoint{0.019646in}{-0.011410in}}{\pgfqpoint{0.021960in}{-0.005824in}}{\pgfqpoint{0.021960in}{0.000000in}}%
\pgfpathcurveto{\pgfqpoint{0.021960in}{0.005824in}}{\pgfqpoint{0.019646in}{0.011410in}}{\pgfqpoint{0.015528in}{0.015528in}}%
\pgfpathcurveto{\pgfqpoint{0.011410in}{0.019646in}}{\pgfqpoint{0.005824in}{0.021960in}}{\pgfqpoint{0.000000in}{0.021960in}}%
\pgfpathcurveto{\pgfqpoint{-0.005824in}{0.021960in}}{\pgfqpoint{-0.011410in}{0.019646in}}{\pgfqpoint{-0.015528in}{0.015528in}}%
\pgfpathcurveto{\pgfqpoint{-0.019646in}{0.011410in}}{\pgfqpoint{-0.021960in}{0.005824in}}{\pgfqpoint{-0.021960in}{0.000000in}}%
\pgfpathcurveto{\pgfqpoint{-0.021960in}{-0.005824in}}{\pgfqpoint{-0.019646in}{-0.011410in}}{\pgfqpoint{-0.015528in}{-0.015528in}}%
\pgfpathcurveto{\pgfqpoint{-0.011410in}{-0.019646in}}{\pgfqpoint{-0.005824in}{-0.021960in}}{\pgfqpoint{0.000000in}{-0.021960in}}%
\pgfpathlineto{\pgfqpoint{0.000000in}{-0.021960in}}%
\pgfpathclose%
\pgfusepath{stroke,fill}%
}%
\begin{pgfscope}%
\pgfsys@transformshift{6.010755in}{0.961249in}%
\pgfsys@useobject{currentmarker}{}%
\end{pgfscope}%
\begin{pgfscope}%
\pgfsys@transformshift{6.183437in}{0.969208in}%
\pgfsys@useobject{currentmarker}{}%
\end{pgfscope}%
\begin{pgfscope}%
\pgfsys@transformshift{6.356120in}{0.978180in}%
\pgfsys@useobject{currentmarker}{}%
\end{pgfscope}%
\begin{pgfscope}%
\pgfsys@transformshift{6.528803in}{0.987996in}%
\pgfsys@useobject{currentmarker}{}%
\end{pgfscope}%
\begin{pgfscope}%
\pgfsys@transformshift{6.701486in}{1.009346in}%
\pgfsys@useobject{currentmarker}{}%
\end{pgfscope}%
\begin{pgfscope}%
\pgfsys@transformshift{6.874168in}{1.040418in}%
\pgfsys@useobject{currentmarker}{}%
\end{pgfscope}%
\begin{pgfscope}%
\pgfsys@transformshift{7.046851in}{1.091307in}%
\pgfsys@useobject{currentmarker}{}%
\end{pgfscope}%
\begin{pgfscope}%
\pgfsys@transformshift{7.219534in}{1.213285in}%
\pgfsys@useobject{currentmarker}{}%
\end{pgfscope}%
\begin{pgfscope}%
\pgfsys@transformshift{7.392216in}{2.068733in}%
\pgfsys@useobject{currentmarker}{}%
\end{pgfscope}%
\begin{pgfscope}%
\pgfsys@transformshift{7.564899in}{5.892449in}%
\pgfsys@useobject{currentmarker}{}%
\end{pgfscope}%
\begin{pgfscope}%
\pgfsys@transformshift{7.737582in}{8.328036in}%
\pgfsys@useobject{currentmarker}{}%
\end{pgfscope}%
\begin{pgfscope}%
\pgfsys@transformshift{7.910265in}{10.052836in}%
\pgfsys@useobject{currentmarker}{}%
\end{pgfscope}%
\begin{pgfscope}%
\pgfsys@transformshift{8.082947in}{11.356408in}%
\pgfsys@useobject{currentmarker}{}%
\end{pgfscope}%
\begin{pgfscope}%
\pgfsys@transformshift{8.255630in}{12.398603in}%
\pgfsys@useobject{currentmarker}{}%
\end{pgfscope}%
\begin{pgfscope}%
\pgfsys@transformshift{8.428313in}{13.311207in}%
\pgfsys@useobject{currentmarker}{}%
\end{pgfscope}%
\begin{pgfscope}%
\pgfsys@transformshift{8.600995in}{14.061495in}%
\pgfsys@useobject{currentmarker}{}%
\end{pgfscope}%
\begin{pgfscope}%
\pgfsys@transformshift{8.773678in}{14.701634in}%
\pgfsys@useobject{currentmarker}{}%
\end{pgfscope}%
\begin{pgfscope}%
\pgfsys@transformshift{8.946361in}{15.301349in}%
\pgfsys@useobject{currentmarker}{}%
\end{pgfscope}%
\begin{pgfscope}%
\pgfsys@transformshift{9.119044in}{15.838077in}%
\pgfsys@useobject{currentmarker}{}%
\end{pgfscope}%
\begin{pgfscope}%
\pgfsys@transformshift{9.291726in}{16.305390in}%
\pgfsys@useobject{currentmarker}{}%
\end{pgfscope}%
\begin{pgfscope}%
\pgfsys@transformshift{9.464409in}{16.734473in}%
\pgfsys@useobject{currentmarker}{}%
\end{pgfscope}%
\begin{pgfscope}%
\pgfsys@transformshift{9.637092in}{17.120012in}%
\pgfsys@useobject{currentmarker}{}%
\end{pgfscope}%
\begin{pgfscope}%
\pgfsys@transformshift{9.809774in}{17.466435in}%
\pgfsys@useobject{currentmarker}{}%
\end{pgfscope}%
\begin{pgfscope}%
\pgfsys@transformshift{9.982457in}{17.790759in}%
\pgfsys@useobject{currentmarker}{}%
\end{pgfscope}%
\begin{pgfscope}%
\pgfsys@transformshift{10.155140in}{18.090135in}%
\pgfsys@useobject{currentmarker}{}%
\end{pgfscope}%
\begin{pgfscope}%
\pgfsys@transformshift{10.327822in}{18.377422in}%
\pgfsys@useobject{currentmarker}{}%
\end{pgfscope}%
\begin{pgfscope}%
\pgfsys@transformshift{10.500505in}{18.615044in}%
\pgfsys@useobject{currentmarker}{}%
\end{pgfscope}%
\begin{pgfscope}%
\pgfsys@transformshift{10.673188in}{18.889010in}%
\pgfsys@useobject{currentmarker}{}%
\end{pgfscope}%
\begin{pgfscope}%
\pgfsys@transformshift{10.845871in}{19.122012in}%
\pgfsys@useobject{currentmarker}{}%
\end{pgfscope}%
\begin{pgfscope}%
\pgfsys@transformshift{11.018553in}{19.322250in}%
\pgfsys@useobject{currentmarker}{}%
\end{pgfscope}%
\begin{pgfscope}%
\pgfsys@transformshift{11.191236in}{19.523990in}%
\pgfsys@useobject{currentmarker}{}%
\end{pgfscope}%
\begin{pgfscope}%
\pgfsys@transformshift{11.363919in}{19.709791in}%
\pgfsys@useobject{currentmarker}{}%
\end{pgfscope}%
\begin{pgfscope}%
\pgfsys@transformshift{11.536601in}{19.883041in}%
\pgfsys@useobject{currentmarker}{}%
\end{pgfscope}%
\begin{pgfscope}%
\pgfsys@transformshift{11.709284in}{20.059179in}%
\pgfsys@useobject{currentmarker}{}%
\end{pgfscope}%
\begin{pgfscope}%
\pgfsys@transformshift{11.881967in}{20.230427in}%
\pgfsys@useobject{currentmarker}{}%
\end{pgfscope}%
\begin{pgfscope}%
\pgfsys@transformshift{12.054650in}{20.374301in}%
\pgfsys@useobject{currentmarker}{}%
\end{pgfscope}%
\begin{pgfscope}%
\pgfsys@transformshift{12.227332in}{20.507242in}%
\pgfsys@useobject{currentmarker}{}%
\end{pgfscope}%
\begin{pgfscope}%
\pgfsys@transformshift{12.400015in}{20.669673in}%
\pgfsys@useobject{currentmarker}{}%
\end{pgfscope}%
\begin{pgfscope}%
\pgfsys@transformshift{12.572698in}{20.802036in}%
\pgfsys@useobject{currentmarker}{}%
\end{pgfscope}%
\begin{pgfscope}%
\pgfsys@transformshift{12.745380in}{20.914110in}%
\pgfsys@useobject{currentmarker}{}%
\end{pgfscope}%
\begin{pgfscope}%
\pgfsys@transformshift{12.918063in}{21.053711in}%
\pgfsys@useobject{currentmarker}{}%
\end{pgfscope}%
\begin{pgfscope}%
\pgfsys@transformshift{13.090746in}{21.155620in}%
\pgfsys@useobject{currentmarker}{}%
\end{pgfscope}%
\begin{pgfscope}%
\pgfsys@transformshift{13.263429in}{21.277203in}%
\pgfsys@useobject{currentmarker}{}%
\end{pgfscope}%
\begin{pgfscope}%
\pgfsys@transformshift{13.436111in}{21.375879in}%
\pgfsys@useobject{currentmarker}{}%
\end{pgfscope}%
\begin{pgfscope}%
\pgfsys@transformshift{13.608794in}{21.471936in}%
\pgfsys@useobject{currentmarker}{}%
\end{pgfscope}%
\begin{pgfscope}%
\pgfsys@transformshift{13.781477in}{21.571459in}%
\pgfsys@useobject{currentmarker}{}%
\end{pgfscope}%
\begin{pgfscope}%
\pgfsys@transformshift{13.954159in}{21.656852in}%
\pgfsys@useobject{currentmarker}{}%
\end{pgfscope}%
\begin{pgfscope}%
\pgfsys@transformshift{14.126842in}{21.762881in}%
\pgfsys@useobject{currentmarker}{}%
\end{pgfscope}%
\begin{pgfscope}%
\pgfsys@transformshift{14.299525in}{21.856590in}%
\pgfsys@useobject{currentmarker}{}%
\end{pgfscope}%
\begin{pgfscope}%
\pgfsys@transformshift{14.472208in}{21.923772in}%
\pgfsys@useobject{currentmarker}{}%
\end{pgfscope}%
\begin{pgfscope}%
\pgfsys@transformshift{14.644890in}{22.017789in}%
\pgfsys@useobject{currentmarker}{}%
\end{pgfscope}%
\end{pgfscope}%
\begin{pgfscope}%
\pgfpathrectangle{\pgfqpoint{5.924413in}{0.527436in}}{\pgfqpoint{4.403409in}{4.235000in}}%
\pgfusepath{clip}%
\pgfsetrectcap%
\pgfsetroundjoin%
\pgfsetlinewidth{0.803000pt}%
\definecolor{currentstroke}{rgb}{0.690196,0.690196,0.690196}%
\pgfsetstrokecolor{currentstroke}%
\pgfsetdash{}{0pt}%
\pgfpathmoveto{\pgfqpoint{6.010755in}{0.527436in}}%
\pgfpathlineto{\pgfqpoint{6.010755in}{4.762436in}}%
\pgfusepath{stroke}%
\end{pgfscope}%
\begin{pgfscope}%
\pgfsetbuttcap%
\pgfsetroundjoin%
\definecolor{currentfill}{rgb}{0.000000,0.000000,0.000000}%
\pgfsetfillcolor{currentfill}%
\pgfsetlinewidth{0.803000pt}%
\definecolor{currentstroke}{rgb}{0.000000,0.000000,0.000000}%
\pgfsetstrokecolor{currentstroke}%
\pgfsetdash{}{0pt}%
\pgfsys@defobject{currentmarker}{\pgfqpoint{0.000000in}{-0.048611in}}{\pgfqpoint{0.000000in}{0.000000in}}{%
\pgfpathmoveto{\pgfqpoint{0.000000in}{0.000000in}}%
\pgfpathlineto{\pgfqpoint{0.000000in}{-0.048611in}}%
\pgfusepath{stroke,fill}%
}%
\begin{pgfscope}%
\pgfsys@transformshift{6.010755in}{0.527436in}%
\pgfsys@useobject{currentmarker}{}%
\end{pgfscope}%
\end{pgfscope}%
\begin{pgfscope}%
\definecolor{textcolor}{rgb}{0.000000,0.000000,0.000000}%
\pgfsetstrokecolor{textcolor}%
\pgfsetfillcolor{textcolor}%
\pgftext[x=6.010755in,y=0.430214in,,top]{\color{textcolor}\sffamily\fontsize{10.000000}{12.000000}\selectfont 0.0}%
\end{pgfscope}%
\begin{pgfscope}%
\pgfpathrectangle{\pgfqpoint{5.924413in}{0.527436in}}{\pgfqpoint{4.403409in}{4.235000in}}%
\pgfusepath{clip}%
\pgfsetrectcap%
\pgfsetroundjoin%
\pgfsetlinewidth{0.803000pt}%
\definecolor{currentstroke}{rgb}{0.690196,0.690196,0.690196}%
\pgfsetstrokecolor{currentstroke}%
\pgfsetdash{}{0pt}%
\pgfpathmoveto{\pgfqpoint{6.874168in}{0.527436in}}%
\pgfpathlineto{\pgfqpoint{6.874168in}{4.762436in}}%
\pgfusepath{stroke}%
\end{pgfscope}%
\begin{pgfscope}%
\pgfsetbuttcap%
\pgfsetroundjoin%
\definecolor{currentfill}{rgb}{0.000000,0.000000,0.000000}%
\pgfsetfillcolor{currentfill}%
\pgfsetlinewidth{0.803000pt}%
\definecolor{currentstroke}{rgb}{0.000000,0.000000,0.000000}%
\pgfsetstrokecolor{currentstroke}%
\pgfsetdash{}{0pt}%
\pgfsys@defobject{currentmarker}{\pgfqpoint{0.000000in}{-0.048611in}}{\pgfqpoint{0.000000in}{0.000000in}}{%
\pgfpathmoveto{\pgfqpoint{0.000000in}{0.000000in}}%
\pgfpathlineto{\pgfqpoint{0.000000in}{-0.048611in}}%
\pgfusepath{stroke,fill}%
}%
\begin{pgfscope}%
\pgfsys@transformshift{6.874168in}{0.527436in}%
\pgfsys@useobject{currentmarker}{}%
\end{pgfscope}%
\end{pgfscope}%
\begin{pgfscope}%
\definecolor{textcolor}{rgb}{0.000000,0.000000,0.000000}%
\pgfsetstrokecolor{textcolor}%
\pgfsetfillcolor{textcolor}%
\pgftext[x=6.874168in,y=0.430214in,,top]{\color{textcolor}\sffamily\fontsize{10.000000}{12.000000}\selectfont 0.1}%
\end{pgfscope}%
\begin{pgfscope}%
\pgfpathrectangle{\pgfqpoint{5.924413in}{0.527436in}}{\pgfqpoint{4.403409in}{4.235000in}}%
\pgfusepath{clip}%
\pgfsetrectcap%
\pgfsetroundjoin%
\pgfsetlinewidth{0.803000pt}%
\definecolor{currentstroke}{rgb}{0.690196,0.690196,0.690196}%
\pgfsetstrokecolor{currentstroke}%
\pgfsetdash{}{0pt}%
\pgfpathmoveto{\pgfqpoint{7.737582in}{0.527436in}}%
\pgfpathlineto{\pgfqpoint{7.737582in}{4.762436in}}%
\pgfusepath{stroke}%
\end{pgfscope}%
\begin{pgfscope}%
\pgfsetbuttcap%
\pgfsetroundjoin%
\definecolor{currentfill}{rgb}{0.000000,0.000000,0.000000}%
\pgfsetfillcolor{currentfill}%
\pgfsetlinewidth{0.803000pt}%
\definecolor{currentstroke}{rgb}{0.000000,0.000000,0.000000}%
\pgfsetstrokecolor{currentstroke}%
\pgfsetdash{}{0pt}%
\pgfsys@defobject{currentmarker}{\pgfqpoint{0.000000in}{-0.048611in}}{\pgfqpoint{0.000000in}{0.000000in}}{%
\pgfpathmoveto{\pgfqpoint{0.000000in}{0.000000in}}%
\pgfpathlineto{\pgfqpoint{0.000000in}{-0.048611in}}%
\pgfusepath{stroke,fill}%
}%
\begin{pgfscope}%
\pgfsys@transformshift{7.737582in}{0.527436in}%
\pgfsys@useobject{currentmarker}{}%
\end{pgfscope}%
\end{pgfscope}%
\begin{pgfscope}%
\definecolor{textcolor}{rgb}{0.000000,0.000000,0.000000}%
\pgfsetstrokecolor{textcolor}%
\pgfsetfillcolor{textcolor}%
\pgftext[x=7.737582in,y=0.430214in,,top]{\color{textcolor}\sffamily\fontsize{10.000000}{12.000000}\selectfont 0.2}%
\end{pgfscope}%
\begin{pgfscope}%
\pgfpathrectangle{\pgfqpoint{5.924413in}{0.527436in}}{\pgfqpoint{4.403409in}{4.235000in}}%
\pgfusepath{clip}%
\pgfsetrectcap%
\pgfsetroundjoin%
\pgfsetlinewidth{0.803000pt}%
\definecolor{currentstroke}{rgb}{0.690196,0.690196,0.690196}%
\pgfsetstrokecolor{currentstroke}%
\pgfsetdash{}{0pt}%
\pgfpathmoveto{\pgfqpoint{8.600995in}{0.527436in}}%
\pgfpathlineto{\pgfqpoint{8.600995in}{4.762436in}}%
\pgfusepath{stroke}%
\end{pgfscope}%
\begin{pgfscope}%
\pgfsetbuttcap%
\pgfsetroundjoin%
\definecolor{currentfill}{rgb}{0.000000,0.000000,0.000000}%
\pgfsetfillcolor{currentfill}%
\pgfsetlinewidth{0.803000pt}%
\definecolor{currentstroke}{rgb}{0.000000,0.000000,0.000000}%
\pgfsetstrokecolor{currentstroke}%
\pgfsetdash{}{0pt}%
\pgfsys@defobject{currentmarker}{\pgfqpoint{0.000000in}{-0.048611in}}{\pgfqpoint{0.000000in}{0.000000in}}{%
\pgfpathmoveto{\pgfqpoint{0.000000in}{0.000000in}}%
\pgfpathlineto{\pgfqpoint{0.000000in}{-0.048611in}}%
\pgfusepath{stroke,fill}%
}%
\begin{pgfscope}%
\pgfsys@transformshift{8.600995in}{0.527436in}%
\pgfsys@useobject{currentmarker}{}%
\end{pgfscope}%
\end{pgfscope}%
\begin{pgfscope}%
\definecolor{textcolor}{rgb}{0.000000,0.000000,0.000000}%
\pgfsetstrokecolor{textcolor}%
\pgfsetfillcolor{textcolor}%
\pgftext[x=8.600995in,y=0.430214in,,top]{\color{textcolor}\sffamily\fontsize{10.000000}{12.000000}\selectfont 0.3}%
\end{pgfscope}%
\begin{pgfscope}%
\pgfpathrectangle{\pgfqpoint{5.924413in}{0.527436in}}{\pgfqpoint{4.403409in}{4.235000in}}%
\pgfusepath{clip}%
\pgfsetrectcap%
\pgfsetroundjoin%
\pgfsetlinewidth{0.803000pt}%
\definecolor{currentstroke}{rgb}{0.690196,0.690196,0.690196}%
\pgfsetstrokecolor{currentstroke}%
\pgfsetdash{}{0pt}%
\pgfpathmoveto{\pgfqpoint{9.464409in}{0.527436in}}%
\pgfpathlineto{\pgfqpoint{9.464409in}{4.762436in}}%
\pgfusepath{stroke}%
\end{pgfscope}%
\begin{pgfscope}%
\pgfsetbuttcap%
\pgfsetroundjoin%
\definecolor{currentfill}{rgb}{0.000000,0.000000,0.000000}%
\pgfsetfillcolor{currentfill}%
\pgfsetlinewidth{0.803000pt}%
\definecolor{currentstroke}{rgb}{0.000000,0.000000,0.000000}%
\pgfsetstrokecolor{currentstroke}%
\pgfsetdash{}{0pt}%
\pgfsys@defobject{currentmarker}{\pgfqpoint{0.000000in}{-0.048611in}}{\pgfqpoint{0.000000in}{0.000000in}}{%
\pgfpathmoveto{\pgfqpoint{0.000000in}{0.000000in}}%
\pgfpathlineto{\pgfqpoint{0.000000in}{-0.048611in}}%
\pgfusepath{stroke,fill}%
}%
\begin{pgfscope}%
\pgfsys@transformshift{9.464409in}{0.527436in}%
\pgfsys@useobject{currentmarker}{}%
\end{pgfscope}%
\end{pgfscope}%
\begin{pgfscope}%
\definecolor{textcolor}{rgb}{0.000000,0.000000,0.000000}%
\pgfsetstrokecolor{textcolor}%
\pgfsetfillcolor{textcolor}%
\pgftext[x=9.464409in,y=0.430214in,,top]{\color{textcolor}\sffamily\fontsize{10.000000}{12.000000}\selectfont 0.4}%
\end{pgfscope}%
\begin{pgfscope}%
\pgfpathrectangle{\pgfqpoint{5.924413in}{0.527436in}}{\pgfqpoint{4.403409in}{4.235000in}}%
\pgfusepath{clip}%
\pgfsetrectcap%
\pgfsetroundjoin%
\pgfsetlinewidth{0.803000pt}%
\definecolor{currentstroke}{rgb}{0.690196,0.690196,0.690196}%
\pgfsetstrokecolor{currentstroke}%
\pgfsetdash{}{0pt}%
\pgfpathmoveto{\pgfqpoint{10.327822in}{0.527436in}}%
\pgfpathlineto{\pgfqpoint{10.327822in}{4.762436in}}%
\pgfusepath{stroke}%
\end{pgfscope}%
\begin{pgfscope}%
\pgfsetbuttcap%
\pgfsetroundjoin%
\definecolor{currentfill}{rgb}{0.000000,0.000000,0.000000}%
\pgfsetfillcolor{currentfill}%
\pgfsetlinewidth{0.803000pt}%
\definecolor{currentstroke}{rgb}{0.000000,0.000000,0.000000}%
\pgfsetstrokecolor{currentstroke}%
\pgfsetdash{}{0pt}%
\pgfsys@defobject{currentmarker}{\pgfqpoint{0.000000in}{-0.048611in}}{\pgfqpoint{0.000000in}{0.000000in}}{%
\pgfpathmoveto{\pgfqpoint{0.000000in}{0.000000in}}%
\pgfpathlineto{\pgfqpoint{0.000000in}{-0.048611in}}%
\pgfusepath{stroke,fill}%
}%
\begin{pgfscope}%
\pgfsys@transformshift{10.327822in}{0.527436in}%
\pgfsys@useobject{currentmarker}{}%
\end{pgfscope}%
\end{pgfscope}%
\begin{pgfscope}%
\definecolor{textcolor}{rgb}{0.000000,0.000000,0.000000}%
\pgfsetstrokecolor{textcolor}%
\pgfsetfillcolor{textcolor}%
\pgftext[x=10.327822in,y=0.430214in,,top]{\color{textcolor}\sffamily\fontsize{10.000000}{12.000000}\selectfont 0.5}%
\end{pgfscope}%
\begin{pgfscope}%
\pgfpathrectangle{\pgfqpoint{5.924413in}{0.527436in}}{\pgfqpoint{4.403409in}{4.235000in}}%
\pgfusepath{clip}%
\pgfsetrectcap%
\pgfsetroundjoin%
\pgfsetlinewidth{0.803000pt}%
\definecolor{currentstroke}{rgb}{0.600000,0.600000,0.600000}%
\pgfsetstrokecolor{currentstroke}%
\pgfsetstrokeopacity{0.200000}%
\pgfsetdash{}{0pt}%
\pgfpathmoveto{\pgfqpoint{6.183437in}{0.527436in}}%
\pgfpathlineto{\pgfqpoint{6.183437in}{4.762436in}}%
\pgfusepath{stroke}%
\end{pgfscope}%
\begin{pgfscope}%
\pgfsetbuttcap%
\pgfsetroundjoin%
\definecolor{currentfill}{rgb}{0.000000,0.000000,0.000000}%
\pgfsetfillcolor{currentfill}%
\pgfsetlinewidth{0.602250pt}%
\definecolor{currentstroke}{rgb}{0.000000,0.000000,0.000000}%
\pgfsetstrokecolor{currentstroke}%
\pgfsetdash{}{0pt}%
\pgfsys@defobject{currentmarker}{\pgfqpoint{0.000000in}{-0.027778in}}{\pgfqpoint{0.000000in}{0.000000in}}{%
\pgfpathmoveto{\pgfqpoint{0.000000in}{0.000000in}}%
\pgfpathlineto{\pgfqpoint{0.000000in}{-0.027778in}}%
\pgfusepath{stroke,fill}%
}%
\begin{pgfscope}%
\pgfsys@transformshift{6.183437in}{0.527436in}%
\pgfsys@useobject{currentmarker}{}%
\end{pgfscope}%
\end{pgfscope}%
\begin{pgfscope}%
\pgfpathrectangle{\pgfqpoint{5.924413in}{0.527436in}}{\pgfqpoint{4.403409in}{4.235000in}}%
\pgfusepath{clip}%
\pgfsetrectcap%
\pgfsetroundjoin%
\pgfsetlinewidth{0.803000pt}%
\definecolor{currentstroke}{rgb}{0.600000,0.600000,0.600000}%
\pgfsetstrokecolor{currentstroke}%
\pgfsetstrokeopacity{0.200000}%
\pgfsetdash{}{0pt}%
\pgfpathmoveto{\pgfqpoint{6.356120in}{0.527436in}}%
\pgfpathlineto{\pgfqpoint{6.356120in}{4.762436in}}%
\pgfusepath{stroke}%
\end{pgfscope}%
\begin{pgfscope}%
\pgfsetbuttcap%
\pgfsetroundjoin%
\definecolor{currentfill}{rgb}{0.000000,0.000000,0.000000}%
\pgfsetfillcolor{currentfill}%
\pgfsetlinewidth{0.602250pt}%
\definecolor{currentstroke}{rgb}{0.000000,0.000000,0.000000}%
\pgfsetstrokecolor{currentstroke}%
\pgfsetdash{}{0pt}%
\pgfsys@defobject{currentmarker}{\pgfqpoint{0.000000in}{-0.027778in}}{\pgfqpoint{0.000000in}{0.000000in}}{%
\pgfpathmoveto{\pgfqpoint{0.000000in}{0.000000in}}%
\pgfpathlineto{\pgfqpoint{0.000000in}{-0.027778in}}%
\pgfusepath{stroke,fill}%
}%
\begin{pgfscope}%
\pgfsys@transformshift{6.356120in}{0.527436in}%
\pgfsys@useobject{currentmarker}{}%
\end{pgfscope}%
\end{pgfscope}%
\begin{pgfscope}%
\pgfpathrectangle{\pgfqpoint{5.924413in}{0.527436in}}{\pgfqpoint{4.403409in}{4.235000in}}%
\pgfusepath{clip}%
\pgfsetrectcap%
\pgfsetroundjoin%
\pgfsetlinewidth{0.803000pt}%
\definecolor{currentstroke}{rgb}{0.600000,0.600000,0.600000}%
\pgfsetstrokecolor{currentstroke}%
\pgfsetstrokeopacity{0.200000}%
\pgfsetdash{}{0pt}%
\pgfpathmoveto{\pgfqpoint{6.528803in}{0.527436in}}%
\pgfpathlineto{\pgfqpoint{6.528803in}{4.762436in}}%
\pgfusepath{stroke}%
\end{pgfscope}%
\begin{pgfscope}%
\pgfsetbuttcap%
\pgfsetroundjoin%
\definecolor{currentfill}{rgb}{0.000000,0.000000,0.000000}%
\pgfsetfillcolor{currentfill}%
\pgfsetlinewidth{0.602250pt}%
\definecolor{currentstroke}{rgb}{0.000000,0.000000,0.000000}%
\pgfsetstrokecolor{currentstroke}%
\pgfsetdash{}{0pt}%
\pgfsys@defobject{currentmarker}{\pgfqpoint{0.000000in}{-0.027778in}}{\pgfqpoint{0.000000in}{0.000000in}}{%
\pgfpathmoveto{\pgfqpoint{0.000000in}{0.000000in}}%
\pgfpathlineto{\pgfqpoint{0.000000in}{-0.027778in}}%
\pgfusepath{stroke,fill}%
}%
\begin{pgfscope}%
\pgfsys@transformshift{6.528803in}{0.527436in}%
\pgfsys@useobject{currentmarker}{}%
\end{pgfscope}%
\end{pgfscope}%
\begin{pgfscope}%
\pgfpathrectangle{\pgfqpoint{5.924413in}{0.527436in}}{\pgfqpoint{4.403409in}{4.235000in}}%
\pgfusepath{clip}%
\pgfsetrectcap%
\pgfsetroundjoin%
\pgfsetlinewidth{0.803000pt}%
\definecolor{currentstroke}{rgb}{0.600000,0.600000,0.600000}%
\pgfsetstrokecolor{currentstroke}%
\pgfsetstrokeopacity{0.200000}%
\pgfsetdash{}{0pt}%
\pgfpathmoveto{\pgfqpoint{6.701486in}{0.527436in}}%
\pgfpathlineto{\pgfqpoint{6.701486in}{4.762436in}}%
\pgfusepath{stroke}%
\end{pgfscope}%
\begin{pgfscope}%
\pgfsetbuttcap%
\pgfsetroundjoin%
\definecolor{currentfill}{rgb}{0.000000,0.000000,0.000000}%
\pgfsetfillcolor{currentfill}%
\pgfsetlinewidth{0.602250pt}%
\definecolor{currentstroke}{rgb}{0.000000,0.000000,0.000000}%
\pgfsetstrokecolor{currentstroke}%
\pgfsetdash{}{0pt}%
\pgfsys@defobject{currentmarker}{\pgfqpoint{0.000000in}{-0.027778in}}{\pgfqpoint{0.000000in}{0.000000in}}{%
\pgfpathmoveto{\pgfqpoint{0.000000in}{0.000000in}}%
\pgfpathlineto{\pgfqpoint{0.000000in}{-0.027778in}}%
\pgfusepath{stroke,fill}%
}%
\begin{pgfscope}%
\pgfsys@transformshift{6.701486in}{0.527436in}%
\pgfsys@useobject{currentmarker}{}%
\end{pgfscope}%
\end{pgfscope}%
\begin{pgfscope}%
\pgfpathrectangle{\pgfqpoint{5.924413in}{0.527436in}}{\pgfqpoint{4.403409in}{4.235000in}}%
\pgfusepath{clip}%
\pgfsetrectcap%
\pgfsetroundjoin%
\pgfsetlinewidth{0.803000pt}%
\definecolor{currentstroke}{rgb}{0.600000,0.600000,0.600000}%
\pgfsetstrokecolor{currentstroke}%
\pgfsetstrokeopacity{0.200000}%
\pgfsetdash{}{0pt}%
\pgfpathmoveto{\pgfqpoint{7.046851in}{0.527436in}}%
\pgfpathlineto{\pgfqpoint{7.046851in}{4.762436in}}%
\pgfusepath{stroke}%
\end{pgfscope}%
\begin{pgfscope}%
\pgfsetbuttcap%
\pgfsetroundjoin%
\definecolor{currentfill}{rgb}{0.000000,0.000000,0.000000}%
\pgfsetfillcolor{currentfill}%
\pgfsetlinewidth{0.602250pt}%
\definecolor{currentstroke}{rgb}{0.000000,0.000000,0.000000}%
\pgfsetstrokecolor{currentstroke}%
\pgfsetdash{}{0pt}%
\pgfsys@defobject{currentmarker}{\pgfqpoint{0.000000in}{-0.027778in}}{\pgfqpoint{0.000000in}{0.000000in}}{%
\pgfpathmoveto{\pgfqpoint{0.000000in}{0.000000in}}%
\pgfpathlineto{\pgfqpoint{0.000000in}{-0.027778in}}%
\pgfusepath{stroke,fill}%
}%
\begin{pgfscope}%
\pgfsys@transformshift{7.046851in}{0.527436in}%
\pgfsys@useobject{currentmarker}{}%
\end{pgfscope}%
\end{pgfscope}%
\begin{pgfscope}%
\pgfpathrectangle{\pgfqpoint{5.924413in}{0.527436in}}{\pgfqpoint{4.403409in}{4.235000in}}%
\pgfusepath{clip}%
\pgfsetrectcap%
\pgfsetroundjoin%
\pgfsetlinewidth{0.803000pt}%
\definecolor{currentstroke}{rgb}{0.600000,0.600000,0.600000}%
\pgfsetstrokecolor{currentstroke}%
\pgfsetstrokeopacity{0.200000}%
\pgfsetdash{}{0pt}%
\pgfpathmoveto{\pgfqpoint{7.219534in}{0.527436in}}%
\pgfpathlineto{\pgfqpoint{7.219534in}{4.762436in}}%
\pgfusepath{stroke}%
\end{pgfscope}%
\begin{pgfscope}%
\pgfsetbuttcap%
\pgfsetroundjoin%
\definecolor{currentfill}{rgb}{0.000000,0.000000,0.000000}%
\pgfsetfillcolor{currentfill}%
\pgfsetlinewidth{0.602250pt}%
\definecolor{currentstroke}{rgb}{0.000000,0.000000,0.000000}%
\pgfsetstrokecolor{currentstroke}%
\pgfsetdash{}{0pt}%
\pgfsys@defobject{currentmarker}{\pgfqpoint{0.000000in}{-0.027778in}}{\pgfqpoint{0.000000in}{0.000000in}}{%
\pgfpathmoveto{\pgfqpoint{0.000000in}{0.000000in}}%
\pgfpathlineto{\pgfqpoint{0.000000in}{-0.027778in}}%
\pgfusepath{stroke,fill}%
}%
\begin{pgfscope}%
\pgfsys@transformshift{7.219534in}{0.527436in}%
\pgfsys@useobject{currentmarker}{}%
\end{pgfscope}%
\end{pgfscope}%
\begin{pgfscope}%
\pgfpathrectangle{\pgfqpoint{5.924413in}{0.527436in}}{\pgfqpoint{4.403409in}{4.235000in}}%
\pgfusepath{clip}%
\pgfsetrectcap%
\pgfsetroundjoin%
\pgfsetlinewidth{0.803000pt}%
\definecolor{currentstroke}{rgb}{0.600000,0.600000,0.600000}%
\pgfsetstrokecolor{currentstroke}%
\pgfsetstrokeopacity{0.200000}%
\pgfsetdash{}{0pt}%
\pgfpathmoveto{\pgfqpoint{7.392216in}{0.527436in}}%
\pgfpathlineto{\pgfqpoint{7.392216in}{4.762436in}}%
\pgfusepath{stroke}%
\end{pgfscope}%
\begin{pgfscope}%
\pgfsetbuttcap%
\pgfsetroundjoin%
\definecolor{currentfill}{rgb}{0.000000,0.000000,0.000000}%
\pgfsetfillcolor{currentfill}%
\pgfsetlinewidth{0.602250pt}%
\definecolor{currentstroke}{rgb}{0.000000,0.000000,0.000000}%
\pgfsetstrokecolor{currentstroke}%
\pgfsetdash{}{0pt}%
\pgfsys@defobject{currentmarker}{\pgfqpoint{0.000000in}{-0.027778in}}{\pgfqpoint{0.000000in}{0.000000in}}{%
\pgfpathmoveto{\pgfqpoint{0.000000in}{0.000000in}}%
\pgfpathlineto{\pgfqpoint{0.000000in}{-0.027778in}}%
\pgfusepath{stroke,fill}%
}%
\begin{pgfscope}%
\pgfsys@transformshift{7.392216in}{0.527436in}%
\pgfsys@useobject{currentmarker}{}%
\end{pgfscope}%
\end{pgfscope}%
\begin{pgfscope}%
\pgfpathrectangle{\pgfqpoint{5.924413in}{0.527436in}}{\pgfqpoint{4.403409in}{4.235000in}}%
\pgfusepath{clip}%
\pgfsetrectcap%
\pgfsetroundjoin%
\pgfsetlinewidth{0.803000pt}%
\definecolor{currentstroke}{rgb}{0.600000,0.600000,0.600000}%
\pgfsetstrokecolor{currentstroke}%
\pgfsetstrokeopacity{0.200000}%
\pgfsetdash{}{0pt}%
\pgfpathmoveto{\pgfqpoint{7.564899in}{0.527436in}}%
\pgfpathlineto{\pgfqpoint{7.564899in}{4.762436in}}%
\pgfusepath{stroke}%
\end{pgfscope}%
\begin{pgfscope}%
\pgfsetbuttcap%
\pgfsetroundjoin%
\definecolor{currentfill}{rgb}{0.000000,0.000000,0.000000}%
\pgfsetfillcolor{currentfill}%
\pgfsetlinewidth{0.602250pt}%
\definecolor{currentstroke}{rgb}{0.000000,0.000000,0.000000}%
\pgfsetstrokecolor{currentstroke}%
\pgfsetdash{}{0pt}%
\pgfsys@defobject{currentmarker}{\pgfqpoint{0.000000in}{-0.027778in}}{\pgfqpoint{0.000000in}{0.000000in}}{%
\pgfpathmoveto{\pgfqpoint{0.000000in}{0.000000in}}%
\pgfpathlineto{\pgfqpoint{0.000000in}{-0.027778in}}%
\pgfusepath{stroke,fill}%
}%
\begin{pgfscope}%
\pgfsys@transformshift{7.564899in}{0.527436in}%
\pgfsys@useobject{currentmarker}{}%
\end{pgfscope}%
\end{pgfscope}%
\begin{pgfscope}%
\pgfpathrectangle{\pgfqpoint{5.924413in}{0.527436in}}{\pgfqpoint{4.403409in}{4.235000in}}%
\pgfusepath{clip}%
\pgfsetrectcap%
\pgfsetroundjoin%
\pgfsetlinewidth{0.803000pt}%
\definecolor{currentstroke}{rgb}{0.600000,0.600000,0.600000}%
\pgfsetstrokecolor{currentstroke}%
\pgfsetstrokeopacity{0.200000}%
\pgfsetdash{}{0pt}%
\pgfpathmoveto{\pgfqpoint{7.910265in}{0.527436in}}%
\pgfpathlineto{\pgfqpoint{7.910265in}{4.762436in}}%
\pgfusepath{stroke}%
\end{pgfscope}%
\begin{pgfscope}%
\pgfsetbuttcap%
\pgfsetroundjoin%
\definecolor{currentfill}{rgb}{0.000000,0.000000,0.000000}%
\pgfsetfillcolor{currentfill}%
\pgfsetlinewidth{0.602250pt}%
\definecolor{currentstroke}{rgb}{0.000000,0.000000,0.000000}%
\pgfsetstrokecolor{currentstroke}%
\pgfsetdash{}{0pt}%
\pgfsys@defobject{currentmarker}{\pgfqpoint{0.000000in}{-0.027778in}}{\pgfqpoint{0.000000in}{0.000000in}}{%
\pgfpathmoveto{\pgfqpoint{0.000000in}{0.000000in}}%
\pgfpathlineto{\pgfqpoint{0.000000in}{-0.027778in}}%
\pgfusepath{stroke,fill}%
}%
\begin{pgfscope}%
\pgfsys@transformshift{7.910265in}{0.527436in}%
\pgfsys@useobject{currentmarker}{}%
\end{pgfscope}%
\end{pgfscope}%
\begin{pgfscope}%
\pgfpathrectangle{\pgfqpoint{5.924413in}{0.527436in}}{\pgfqpoint{4.403409in}{4.235000in}}%
\pgfusepath{clip}%
\pgfsetrectcap%
\pgfsetroundjoin%
\pgfsetlinewidth{0.803000pt}%
\definecolor{currentstroke}{rgb}{0.600000,0.600000,0.600000}%
\pgfsetstrokecolor{currentstroke}%
\pgfsetstrokeopacity{0.200000}%
\pgfsetdash{}{0pt}%
\pgfpathmoveto{\pgfqpoint{8.082947in}{0.527436in}}%
\pgfpathlineto{\pgfqpoint{8.082947in}{4.762436in}}%
\pgfusepath{stroke}%
\end{pgfscope}%
\begin{pgfscope}%
\pgfsetbuttcap%
\pgfsetroundjoin%
\definecolor{currentfill}{rgb}{0.000000,0.000000,0.000000}%
\pgfsetfillcolor{currentfill}%
\pgfsetlinewidth{0.602250pt}%
\definecolor{currentstroke}{rgb}{0.000000,0.000000,0.000000}%
\pgfsetstrokecolor{currentstroke}%
\pgfsetdash{}{0pt}%
\pgfsys@defobject{currentmarker}{\pgfqpoint{0.000000in}{-0.027778in}}{\pgfqpoint{0.000000in}{0.000000in}}{%
\pgfpathmoveto{\pgfqpoint{0.000000in}{0.000000in}}%
\pgfpathlineto{\pgfqpoint{0.000000in}{-0.027778in}}%
\pgfusepath{stroke,fill}%
}%
\begin{pgfscope}%
\pgfsys@transformshift{8.082947in}{0.527436in}%
\pgfsys@useobject{currentmarker}{}%
\end{pgfscope}%
\end{pgfscope}%
\begin{pgfscope}%
\pgfpathrectangle{\pgfqpoint{5.924413in}{0.527436in}}{\pgfqpoint{4.403409in}{4.235000in}}%
\pgfusepath{clip}%
\pgfsetrectcap%
\pgfsetroundjoin%
\pgfsetlinewidth{0.803000pt}%
\definecolor{currentstroke}{rgb}{0.600000,0.600000,0.600000}%
\pgfsetstrokecolor{currentstroke}%
\pgfsetstrokeopacity{0.200000}%
\pgfsetdash{}{0pt}%
\pgfpathmoveto{\pgfqpoint{8.255630in}{0.527436in}}%
\pgfpathlineto{\pgfqpoint{8.255630in}{4.762436in}}%
\pgfusepath{stroke}%
\end{pgfscope}%
\begin{pgfscope}%
\pgfsetbuttcap%
\pgfsetroundjoin%
\definecolor{currentfill}{rgb}{0.000000,0.000000,0.000000}%
\pgfsetfillcolor{currentfill}%
\pgfsetlinewidth{0.602250pt}%
\definecolor{currentstroke}{rgb}{0.000000,0.000000,0.000000}%
\pgfsetstrokecolor{currentstroke}%
\pgfsetdash{}{0pt}%
\pgfsys@defobject{currentmarker}{\pgfqpoint{0.000000in}{-0.027778in}}{\pgfqpoint{0.000000in}{0.000000in}}{%
\pgfpathmoveto{\pgfqpoint{0.000000in}{0.000000in}}%
\pgfpathlineto{\pgfqpoint{0.000000in}{-0.027778in}}%
\pgfusepath{stroke,fill}%
}%
\begin{pgfscope}%
\pgfsys@transformshift{8.255630in}{0.527436in}%
\pgfsys@useobject{currentmarker}{}%
\end{pgfscope}%
\end{pgfscope}%
\begin{pgfscope}%
\pgfpathrectangle{\pgfqpoint{5.924413in}{0.527436in}}{\pgfqpoint{4.403409in}{4.235000in}}%
\pgfusepath{clip}%
\pgfsetrectcap%
\pgfsetroundjoin%
\pgfsetlinewidth{0.803000pt}%
\definecolor{currentstroke}{rgb}{0.600000,0.600000,0.600000}%
\pgfsetstrokecolor{currentstroke}%
\pgfsetstrokeopacity{0.200000}%
\pgfsetdash{}{0pt}%
\pgfpathmoveto{\pgfqpoint{8.428313in}{0.527436in}}%
\pgfpathlineto{\pgfqpoint{8.428313in}{4.762436in}}%
\pgfusepath{stroke}%
\end{pgfscope}%
\begin{pgfscope}%
\pgfsetbuttcap%
\pgfsetroundjoin%
\definecolor{currentfill}{rgb}{0.000000,0.000000,0.000000}%
\pgfsetfillcolor{currentfill}%
\pgfsetlinewidth{0.602250pt}%
\definecolor{currentstroke}{rgb}{0.000000,0.000000,0.000000}%
\pgfsetstrokecolor{currentstroke}%
\pgfsetdash{}{0pt}%
\pgfsys@defobject{currentmarker}{\pgfqpoint{0.000000in}{-0.027778in}}{\pgfqpoint{0.000000in}{0.000000in}}{%
\pgfpathmoveto{\pgfqpoint{0.000000in}{0.000000in}}%
\pgfpathlineto{\pgfqpoint{0.000000in}{-0.027778in}}%
\pgfusepath{stroke,fill}%
}%
\begin{pgfscope}%
\pgfsys@transformshift{8.428313in}{0.527436in}%
\pgfsys@useobject{currentmarker}{}%
\end{pgfscope}%
\end{pgfscope}%
\begin{pgfscope}%
\pgfpathrectangle{\pgfqpoint{5.924413in}{0.527436in}}{\pgfqpoint{4.403409in}{4.235000in}}%
\pgfusepath{clip}%
\pgfsetrectcap%
\pgfsetroundjoin%
\pgfsetlinewidth{0.803000pt}%
\definecolor{currentstroke}{rgb}{0.600000,0.600000,0.600000}%
\pgfsetstrokecolor{currentstroke}%
\pgfsetstrokeopacity{0.200000}%
\pgfsetdash{}{0pt}%
\pgfpathmoveto{\pgfqpoint{8.773678in}{0.527436in}}%
\pgfpathlineto{\pgfqpoint{8.773678in}{4.762436in}}%
\pgfusepath{stroke}%
\end{pgfscope}%
\begin{pgfscope}%
\pgfsetbuttcap%
\pgfsetroundjoin%
\definecolor{currentfill}{rgb}{0.000000,0.000000,0.000000}%
\pgfsetfillcolor{currentfill}%
\pgfsetlinewidth{0.602250pt}%
\definecolor{currentstroke}{rgb}{0.000000,0.000000,0.000000}%
\pgfsetstrokecolor{currentstroke}%
\pgfsetdash{}{0pt}%
\pgfsys@defobject{currentmarker}{\pgfqpoint{0.000000in}{-0.027778in}}{\pgfqpoint{0.000000in}{0.000000in}}{%
\pgfpathmoveto{\pgfqpoint{0.000000in}{0.000000in}}%
\pgfpathlineto{\pgfqpoint{0.000000in}{-0.027778in}}%
\pgfusepath{stroke,fill}%
}%
\begin{pgfscope}%
\pgfsys@transformshift{8.773678in}{0.527436in}%
\pgfsys@useobject{currentmarker}{}%
\end{pgfscope}%
\end{pgfscope}%
\begin{pgfscope}%
\pgfpathrectangle{\pgfqpoint{5.924413in}{0.527436in}}{\pgfqpoint{4.403409in}{4.235000in}}%
\pgfusepath{clip}%
\pgfsetrectcap%
\pgfsetroundjoin%
\pgfsetlinewidth{0.803000pt}%
\definecolor{currentstroke}{rgb}{0.600000,0.600000,0.600000}%
\pgfsetstrokecolor{currentstroke}%
\pgfsetstrokeopacity{0.200000}%
\pgfsetdash{}{0pt}%
\pgfpathmoveto{\pgfqpoint{8.946361in}{0.527436in}}%
\pgfpathlineto{\pgfqpoint{8.946361in}{4.762436in}}%
\pgfusepath{stroke}%
\end{pgfscope}%
\begin{pgfscope}%
\pgfsetbuttcap%
\pgfsetroundjoin%
\definecolor{currentfill}{rgb}{0.000000,0.000000,0.000000}%
\pgfsetfillcolor{currentfill}%
\pgfsetlinewidth{0.602250pt}%
\definecolor{currentstroke}{rgb}{0.000000,0.000000,0.000000}%
\pgfsetstrokecolor{currentstroke}%
\pgfsetdash{}{0pt}%
\pgfsys@defobject{currentmarker}{\pgfqpoint{0.000000in}{-0.027778in}}{\pgfqpoint{0.000000in}{0.000000in}}{%
\pgfpathmoveto{\pgfqpoint{0.000000in}{0.000000in}}%
\pgfpathlineto{\pgfqpoint{0.000000in}{-0.027778in}}%
\pgfusepath{stroke,fill}%
}%
\begin{pgfscope}%
\pgfsys@transformshift{8.946361in}{0.527436in}%
\pgfsys@useobject{currentmarker}{}%
\end{pgfscope}%
\end{pgfscope}%
\begin{pgfscope}%
\pgfpathrectangle{\pgfqpoint{5.924413in}{0.527436in}}{\pgfqpoint{4.403409in}{4.235000in}}%
\pgfusepath{clip}%
\pgfsetrectcap%
\pgfsetroundjoin%
\pgfsetlinewidth{0.803000pt}%
\definecolor{currentstroke}{rgb}{0.600000,0.600000,0.600000}%
\pgfsetstrokecolor{currentstroke}%
\pgfsetstrokeopacity{0.200000}%
\pgfsetdash{}{0pt}%
\pgfpathmoveto{\pgfqpoint{9.119044in}{0.527436in}}%
\pgfpathlineto{\pgfqpoint{9.119044in}{4.762436in}}%
\pgfusepath{stroke}%
\end{pgfscope}%
\begin{pgfscope}%
\pgfsetbuttcap%
\pgfsetroundjoin%
\definecolor{currentfill}{rgb}{0.000000,0.000000,0.000000}%
\pgfsetfillcolor{currentfill}%
\pgfsetlinewidth{0.602250pt}%
\definecolor{currentstroke}{rgb}{0.000000,0.000000,0.000000}%
\pgfsetstrokecolor{currentstroke}%
\pgfsetdash{}{0pt}%
\pgfsys@defobject{currentmarker}{\pgfqpoint{0.000000in}{-0.027778in}}{\pgfqpoint{0.000000in}{0.000000in}}{%
\pgfpathmoveto{\pgfqpoint{0.000000in}{0.000000in}}%
\pgfpathlineto{\pgfqpoint{0.000000in}{-0.027778in}}%
\pgfusepath{stroke,fill}%
}%
\begin{pgfscope}%
\pgfsys@transformshift{9.119044in}{0.527436in}%
\pgfsys@useobject{currentmarker}{}%
\end{pgfscope}%
\end{pgfscope}%
\begin{pgfscope}%
\pgfpathrectangle{\pgfqpoint{5.924413in}{0.527436in}}{\pgfqpoint{4.403409in}{4.235000in}}%
\pgfusepath{clip}%
\pgfsetrectcap%
\pgfsetroundjoin%
\pgfsetlinewidth{0.803000pt}%
\definecolor{currentstroke}{rgb}{0.600000,0.600000,0.600000}%
\pgfsetstrokecolor{currentstroke}%
\pgfsetstrokeopacity{0.200000}%
\pgfsetdash{}{0pt}%
\pgfpathmoveto{\pgfqpoint{9.291726in}{0.527436in}}%
\pgfpathlineto{\pgfqpoint{9.291726in}{4.762436in}}%
\pgfusepath{stroke}%
\end{pgfscope}%
\begin{pgfscope}%
\pgfsetbuttcap%
\pgfsetroundjoin%
\definecolor{currentfill}{rgb}{0.000000,0.000000,0.000000}%
\pgfsetfillcolor{currentfill}%
\pgfsetlinewidth{0.602250pt}%
\definecolor{currentstroke}{rgb}{0.000000,0.000000,0.000000}%
\pgfsetstrokecolor{currentstroke}%
\pgfsetdash{}{0pt}%
\pgfsys@defobject{currentmarker}{\pgfqpoint{0.000000in}{-0.027778in}}{\pgfqpoint{0.000000in}{0.000000in}}{%
\pgfpathmoveto{\pgfqpoint{0.000000in}{0.000000in}}%
\pgfpathlineto{\pgfqpoint{0.000000in}{-0.027778in}}%
\pgfusepath{stroke,fill}%
}%
\begin{pgfscope}%
\pgfsys@transformshift{9.291726in}{0.527436in}%
\pgfsys@useobject{currentmarker}{}%
\end{pgfscope}%
\end{pgfscope}%
\begin{pgfscope}%
\pgfpathrectangle{\pgfqpoint{5.924413in}{0.527436in}}{\pgfqpoint{4.403409in}{4.235000in}}%
\pgfusepath{clip}%
\pgfsetrectcap%
\pgfsetroundjoin%
\pgfsetlinewidth{0.803000pt}%
\definecolor{currentstroke}{rgb}{0.600000,0.600000,0.600000}%
\pgfsetstrokecolor{currentstroke}%
\pgfsetstrokeopacity{0.200000}%
\pgfsetdash{}{0pt}%
\pgfpathmoveto{\pgfqpoint{9.637092in}{0.527436in}}%
\pgfpathlineto{\pgfqpoint{9.637092in}{4.762436in}}%
\pgfusepath{stroke}%
\end{pgfscope}%
\begin{pgfscope}%
\pgfsetbuttcap%
\pgfsetroundjoin%
\definecolor{currentfill}{rgb}{0.000000,0.000000,0.000000}%
\pgfsetfillcolor{currentfill}%
\pgfsetlinewidth{0.602250pt}%
\definecolor{currentstroke}{rgb}{0.000000,0.000000,0.000000}%
\pgfsetstrokecolor{currentstroke}%
\pgfsetdash{}{0pt}%
\pgfsys@defobject{currentmarker}{\pgfqpoint{0.000000in}{-0.027778in}}{\pgfqpoint{0.000000in}{0.000000in}}{%
\pgfpathmoveto{\pgfqpoint{0.000000in}{0.000000in}}%
\pgfpathlineto{\pgfqpoint{0.000000in}{-0.027778in}}%
\pgfusepath{stroke,fill}%
}%
\begin{pgfscope}%
\pgfsys@transformshift{9.637092in}{0.527436in}%
\pgfsys@useobject{currentmarker}{}%
\end{pgfscope}%
\end{pgfscope}%
\begin{pgfscope}%
\pgfpathrectangle{\pgfqpoint{5.924413in}{0.527436in}}{\pgfqpoint{4.403409in}{4.235000in}}%
\pgfusepath{clip}%
\pgfsetrectcap%
\pgfsetroundjoin%
\pgfsetlinewidth{0.803000pt}%
\definecolor{currentstroke}{rgb}{0.600000,0.600000,0.600000}%
\pgfsetstrokecolor{currentstroke}%
\pgfsetstrokeopacity{0.200000}%
\pgfsetdash{}{0pt}%
\pgfpathmoveto{\pgfqpoint{9.809774in}{0.527436in}}%
\pgfpathlineto{\pgfqpoint{9.809774in}{4.762436in}}%
\pgfusepath{stroke}%
\end{pgfscope}%
\begin{pgfscope}%
\pgfsetbuttcap%
\pgfsetroundjoin%
\definecolor{currentfill}{rgb}{0.000000,0.000000,0.000000}%
\pgfsetfillcolor{currentfill}%
\pgfsetlinewidth{0.602250pt}%
\definecolor{currentstroke}{rgb}{0.000000,0.000000,0.000000}%
\pgfsetstrokecolor{currentstroke}%
\pgfsetdash{}{0pt}%
\pgfsys@defobject{currentmarker}{\pgfqpoint{0.000000in}{-0.027778in}}{\pgfqpoint{0.000000in}{0.000000in}}{%
\pgfpathmoveto{\pgfqpoint{0.000000in}{0.000000in}}%
\pgfpathlineto{\pgfqpoint{0.000000in}{-0.027778in}}%
\pgfusepath{stroke,fill}%
}%
\begin{pgfscope}%
\pgfsys@transformshift{9.809774in}{0.527436in}%
\pgfsys@useobject{currentmarker}{}%
\end{pgfscope}%
\end{pgfscope}%
\begin{pgfscope}%
\pgfpathrectangle{\pgfqpoint{5.924413in}{0.527436in}}{\pgfqpoint{4.403409in}{4.235000in}}%
\pgfusepath{clip}%
\pgfsetrectcap%
\pgfsetroundjoin%
\pgfsetlinewidth{0.803000pt}%
\definecolor{currentstroke}{rgb}{0.600000,0.600000,0.600000}%
\pgfsetstrokecolor{currentstroke}%
\pgfsetstrokeopacity{0.200000}%
\pgfsetdash{}{0pt}%
\pgfpathmoveto{\pgfqpoint{9.982457in}{0.527436in}}%
\pgfpathlineto{\pgfqpoint{9.982457in}{4.762436in}}%
\pgfusepath{stroke}%
\end{pgfscope}%
\begin{pgfscope}%
\pgfsetbuttcap%
\pgfsetroundjoin%
\definecolor{currentfill}{rgb}{0.000000,0.000000,0.000000}%
\pgfsetfillcolor{currentfill}%
\pgfsetlinewidth{0.602250pt}%
\definecolor{currentstroke}{rgb}{0.000000,0.000000,0.000000}%
\pgfsetstrokecolor{currentstroke}%
\pgfsetdash{}{0pt}%
\pgfsys@defobject{currentmarker}{\pgfqpoint{0.000000in}{-0.027778in}}{\pgfqpoint{0.000000in}{0.000000in}}{%
\pgfpathmoveto{\pgfqpoint{0.000000in}{0.000000in}}%
\pgfpathlineto{\pgfqpoint{0.000000in}{-0.027778in}}%
\pgfusepath{stroke,fill}%
}%
\begin{pgfscope}%
\pgfsys@transformshift{9.982457in}{0.527436in}%
\pgfsys@useobject{currentmarker}{}%
\end{pgfscope}%
\end{pgfscope}%
\begin{pgfscope}%
\pgfpathrectangle{\pgfqpoint{5.924413in}{0.527436in}}{\pgfqpoint{4.403409in}{4.235000in}}%
\pgfusepath{clip}%
\pgfsetrectcap%
\pgfsetroundjoin%
\pgfsetlinewidth{0.803000pt}%
\definecolor{currentstroke}{rgb}{0.600000,0.600000,0.600000}%
\pgfsetstrokecolor{currentstroke}%
\pgfsetstrokeopacity{0.200000}%
\pgfsetdash{}{0pt}%
\pgfpathmoveto{\pgfqpoint{10.155140in}{0.527436in}}%
\pgfpathlineto{\pgfqpoint{10.155140in}{4.762436in}}%
\pgfusepath{stroke}%
\end{pgfscope}%
\begin{pgfscope}%
\pgfsetbuttcap%
\pgfsetroundjoin%
\definecolor{currentfill}{rgb}{0.000000,0.000000,0.000000}%
\pgfsetfillcolor{currentfill}%
\pgfsetlinewidth{0.602250pt}%
\definecolor{currentstroke}{rgb}{0.000000,0.000000,0.000000}%
\pgfsetstrokecolor{currentstroke}%
\pgfsetdash{}{0pt}%
\pgfsys@defobject{currentmarker}{\pgfqpoint{0.000000in}{-0.027778in}}{\pgfqpoint{0.000000in}{0.000000in}}{%
\pgfpathmoveto{\pgfqpoint{0.000000in}{0.000000in}}%
\pgfpathlineto{\pgfqpoint{0.000000in}{-0.027778in}}%
\pgfusepath{stroke,fill}%
}%
\begin{pgfscope}%
\pgfsys@transformshift{10.155140in}{0.527436in}%
\pgfsys@useobject{currentmarker}{}%
\end{pgfscope}%
\end{pgfscope}%
\begin{pgfscope}%
\definecolor{textcolor}{rgb}{0.000000,0.000000,0.000000}%
\pgfsetstrokecolor{textcolor}%
\pgfsetfillcolor{textcolor}%
\pgftext[x=8.126118in,y=0.240245in,,top]{\color{textcolor}\sffamily\fontsize{10.000000}{12.000000}\selectfont turnover probability \(\displaystyle p_1\,(S\rightarrow I\,)\)}%
\end{pgfscope}%
\begin{pgfscope}%
\pgfpathrectangle{\pgfqpoint{5.924413in}{0.527436in}}{\pgfqpoint{4.403409in}{4.235000in}}%
\pgfusepath{clip}%
\pgfsetrectcap%
\pgfsetroundjoin%
\pgfsetlinewidth{0.803000pt}%
\definecolor{currentstroke}{rgb}{0.690196,0.690196,0.690196}%
\pgfsetstrokecolor{currentstroke}%
\pgfsetdash{}{0pt}%
\pgfpathmoveto{\pgfqpoint{5.924413in}{0.912436in}}%
\pgfpathlineto{\pgfqpoint{10.327822in}{0.912436in}}%
\pgfusepath{stroke}%
\end{pgfscope}%
\begin{pgfscope}%
\pgfsetbuttcap%
\pgfsetroundjoin%
\definecolor{currentfill}{rgb}{0.000000,0.000000,0.000000}%
\pgfsetfillcolor{currentfill}%
\pgfsetlinewidth{0.803000pt}%
\definecolor{currentstroke}{rgb}{0.000000,0.000000,0.000000}%
\pgfsetstrokecolor{currentstroke}%
\pgfsetdash{}{0pt}%
\pgfsys@defobject{currentmarker}{\pgfqpoint{-0.048611in}{0.000000in}}{\pgfqpoint{-0.000000in}{0.000000in}}{%
\pgfpathmoveto{\pgfqpoint{-0.000000in}{0.000000in}}%
\pgfpathlineto{\pgfqpoint{-0.048611in}{0.000000in}}%
\pgfusepath{stroke,fill}%
}%
\begin{pgfscope}%
\pgfsys@transformshift{5.924413in}{0.912436in}%
\pgfsys@useobject{currentmarker}{}%
\end{pgfscope}%
\end{pgfscope}%
\begin{pgfscope}%
\definecolor{textcolor}{rgb}{0.000000,0.000000,0.000000}%
\pgfsetstrokecolor{textcolor}%
\pgfsetfillcolor{textcolor}%
\pgftext[x=5.517946in, y=0.859674in, left, base]{\color{textcolor}\sffamily\fontsize{10.000000}{12.000000}\selectfont 0.00}%
\end{pgfscope}%
\begin{pgfscope}%
\pgfpathrectangle{\pgfqpoint{5.924413in}{0.527436in}}{\pgfqpoint{4.403409in}{4.235000in}}%
\pgfusepath{clip}%
\pgfsetrectcap%
\pgfsetroundjoin%
\pgfsetlinewidth{0.803000pt}%
\definecolor{currentstroke}{rgb}{0.690196,0.690196,0.690196}%
\pgfsetstrokecolor{currentstroke}%
\pgfsetdash{}{0pt}%
\pgfpathmoveto{\pgfqpoint{5.924413in}{1.682436in}}%
\pgfpathlineto{\pgfqpoint{10.327822in}{1.682436in}}%
\pgfusepath{stroke}%
\end{pgfscope}%
\begin{pgfscope}%
\pgfsetbuttcap%
\pgfsetroundjoin%
\definecolor{currentfill}{rgb}{0.000000,0.000000,0.000000}%
\pgfsetfillcolor{currentfill}%
\pgfsetlinewidth{0.803000pt}%
\definecolor{currentstroke}{rgb}{0.000000,0.000000,0.000000}%
\pgfsetstrokecolor{currentstroke}%
\pgfsetdash{}{0pt}%
\pgfsys@defobject{currentmarker}{\pgfqpoint{-0.048611in}{0.000000in}}{\pgfqpoint{-0.000000in}{0.000000in}}{%
\pgfpathmoveto{\pgfqpoint{-0.000000in}{0.000000in}}%
\pgfpathlineto{\pgfqpoint{-0.048611in}{0.000000in}}%
\pgfusepath{stroke,fill}%
}%
\begin{pgfscope}%
\pgfsys@transformshift{5.924413in}{1.682436in}%
\pgfsys@useobject{currentmarker}{}%
\end{pgfscope}%
\end{pgfscope}%
\begin{pgfscope}%
\definecolor{textcolor}{rgb}{0.000000,0.000000,0.000000}%
\pgfsetstrokecolor{textcolor}%
\pgfsetfillcolor{textcolor}%
\pgftext[x=5.517946in, y=1.629674in, left, base]{\color{textcolor}\sffamily\fontsize{10.000000}{12.000000}\selectfont 0.02}%
\end{pgfscope}%
\begin{pgfscope}%
\pgfpathrectangle{\pgfqpoint{5.924413in}{0.527436in}}{\pgfqpoint{4.403409in}{4.235000in}}%
\pgfusepath{clip}%
\pgfsetrectcap%
\pgfsetroundjoin%
\pgfsetlinewidth{0.803000pt}%
\definecolor{currentstroke}{rgb}{0.690196,0.690196,0.690196}%
\pgfsetstrokecolor{currentstroke}%
\pgfsetdash{}{0pt}%
\pgfpathmoveto{\pgfqpoint{5.924413in}{2.452436in}}%
\pgfpathlineto{\pgfqpoint{10.327822in}{2.452436in}}%
\pgfusepath{stroke}%
\end{pgfscope}%
\begin{pgfscope}%
\pgfsetbuttcap%
\pgfsetroundjoin%
\definecolor{currentfill}{rgb}{0.000000,0.000000,0.000000}%
\pgfsetfillcolor{currentfill}%
\pgfsetlinewidth{0.803000pt}%
\definecolor{currentstroke}{rgb}{0.000000,0.000000,0.000000}%
\pgfsetstrokecolor{currentstroke}%
\pgfsetdash{}{0pt}%
\pgfsys@defobject{currentmarker}{\pgfqpoint{-0.048611in}{0.000000in}}{\pgfqpoint{-0.000000in}{0.000000in}}{%
\pgfpathmoveto{\pgfqpoint{-0.000000in}{0.000000in}}%
\pgfpathlineto{\pgfqpoint{-0.048611in}{0.000000in}}%
\pgfusepath{stroke,fill}%
}%
\begin{pgfscope}%
\pgfsys@transformshift{5.924413in}{2.452436in}%
\pgfsys@useobject{currentmarker}{}%
\end{pgfscope}%
\end{pgfscope}%
\begin{pgfscope}%
\definecolor{textcolor}{rgb}{0.000000,0.000000,0.000000}%
\pgfsetstrokecolor{textcolor}%
\pgfsetfillcolor{textcolor}%
\pgftext[x=5.517946in, y=2.399674in, left, base]{\color{textcolor}\sffamily\fontsize{10.000000}{12.000000}\selectfont 0.04}%
\end{pgfscope}%
\begin{pgfscope}%
\pgfpathrectangle{\pgfqpoint{5.924413in}{0.527436in}}{\pgfqpoint{4.403409in}{4.235000in}}%
\pgfusepath{clip}%
\pgfsetrectcap%
\pgfsetroundjoin%
\pgfsetlinewidth{0.803000pt}%
\definecolor{currentstroke}{rgb}{0.690196,0.690196,0.690196}%
\pgfsetstrokecolor{currentstroke}%
\pgfsetdash{}{0pt}%
\pgfpathmoveto{\pgfqpoint{5.924413in}{3.222436in}}%
\pgfpathlineto{\pgfqpoint{10.327822in}{3.222436in}}%
\pgfusepath{stroke}%
\end{pgfscope}%
\begin{pgfscope}%
\pgfsetbuttcap%
\pgfsetroundjoin%
\definecolor{currentfill}{rgb}{0.000000,0.000000,0.000000}%
\pgfsetfillcolor{currentfill}%
\pgfsetlinewidth{0.803000pt}%
\definecolor{currentstroke}{rgb}{0.000000,0.000000,0.000000}%
\pgfsetstrokecolor{currentstroke}%
\pgfsetdash{}{0pt}%
\pgfsys@defobject{currentmarker}{\pgfqpoint{-0.048611in}{0.000000in}}{\pgfqpoint{-0.000000in}{0.000000in}}{%
\pgfpathmoveto{\pgfqpoint{-0.000000in}{0.000000in}}%
\pgfpathlineto{\pgfqpoint{-0.048611in}{0.000000in}}%
\pgfusepath{stroke,fill}%
}%
\begin{pgfscope}%
\pgfsys@transformshift{5.924413in}{3.222436in}%
\pgfsys@useobject{currentmarker}{}%
\end{pgfscope}%
\end{pgfscope}%
\begin{pgfscope}%
\definecolor{textcolor}{rgb}{0.000000,0.000000,0.000000}%
\pgfsetstrokecolor{textcolor}%
\pgfsetfillcolor{textcolor}%
\pgftext[x=5.517946in, y=3.169674in, left, base]{\color{textcolor}\sffamily\fontsize{10.000000}{12.000000}\selectfont 0.06}%
\end{pgfscope}%
\begin{pgfscope}%
\pgfpathrectangle{\pgfqpoint{5.924413in}{0.527436in}}{\pgfqpoint{4.403409in}{4.235000in}}%
\pgfusepath{clip}%
\pgfsetrectcap%
\pgfsetroundjoin%
\pgfsetlinewidth{0.803000pt}%
\definecolor{currentstroke}{rgb}{0.690196,0.690196,0.690196}%
\pgfsetstrokecolor{currentstroke}%
\pgfsetdash{}{0pt}%
\pgfpathmoveto{\pgfqpoint{5.924413in}{3.992436in}}%
\pgfpathlineto{\pgfqpoint{10.327822in}{3.992436in}}%
\pgfusepath{stroke}%
\end{pgfscope}%
\begin{pgfscope}%
\pgfsetbuttcap%
\pgfsetroundjoin%
\definecolor{currentfill}{rgb}{0.000000,0.000000,0.000000}%
\pgfsetfillcolor{currentfill}%
\pgfsetlinewidth{0.803000pt}%
\definecolor{currentstroke}{rgb}{0.000000,0.000000,0.000000}%
\pgfsetstrokecolor{currentstroke}%
\pgfsetdash{}{0pt}%
\pgfsys@defobject{currentmarker}{\pgfqpoint{-0.048611in}{0.000000in}}{\pgfqpoint{-0.000000in}{0.000000in}}{%
\pgfpathmoveto{\pgfqpoint{-0.000000in}{0.000000in}}%
\pgfpathlineto{\pgfqpoint{-0.048611in}{0.000000in}}%
\pgfusepath{stroke,fill}%
}%
\begin{pgfscope}%
\pgfsys@transformshift{5.924413in}{3.992436in}%
\pgfsys@useobject{currentmarker}{}%
\end{pgfscope}%
\end{pgfscope}%
\begin{pgfscope}%
\definecolor{textcolor}{rgb}{0.000000,0.000000,0.000000}%
\pgfsetstrokecolor{textcolor}%
\pgfsetfillcolor{textcolor}%
\pgftext[x=5.517946in, y=3.939674in, left, base]{\color{textcolor}\sffamily\fontsize{10.000000}{12.000000}\selectfont 0.08}%
\end{pgfscope}%
\begin{pgfscope}%
\pgfpathrectangle{\pgfqpoint{5.924413in}{0.527436in}}{\pgfqpoint{4.403409in}{4.235000in}}%
\pgfusepath{clip}%
\pgfsetrectcap%
\pgfsetroundjoin%
\pgfsetlinewidth{0.803000pt}%
\definecolor{currentstroke}{rgb}{0.690196,0.690196,0.690196}%
\pgfsetstrokecolor{currentstroke}%
\pgfsetdash{}{0pt}%
\pgfpathmoveto{\pgfqpoint{5.924413in}{4.762436in}}%
\pgfpathlineto{\pgfqpoint{10.327822in}{4.762436in}}%
\pgfusepath{stroke}%
\end{pgfscope}%
\begin{pgfscope}%
\pgfsetbuttcap%
\pgfsetroundjoin%
\definecolor{currentfill}{rgb}{0.000000,0.000000,0.000000}%
\pgfsetfillcolor{currentfill}%
\pgfsetlinewidth{0.803000pt}%
\definecolor{currentstroke}{rgb}{0.000000,0.000000,0.000000}%
\pgfsetstrokecolor{currentstroke}%
\pgfsetdash{}{0pt}%
\pgfsys@defobject{currentmarker}{\pgfqpoint{-0.048611in}{0.000000in}}{\pgfqpoint{-0.000000in}{0.000000in}}{%
\pgfpathmoveto{\pgfqpoint{-0.000000in}{0.000000in}}%
\pgfpathlineto{\pgfqpoint{-0.048611in}{0.000000in}}%
\pgfusepath{stroke,fill}%
}%
\begin{pgfscope}%
\pgfsys@transformshift{5.924413in}{4.762436in}%
\pgfsys@useobject{currentmarker}{}%
\end{pgfscope}%
\end{pgfscope}%
\begin{pgfscope}%
\definecolor{textcolor}{rgb}{0.000000,0.000000,0.000000}%
\pgfsetstrokecolor{textcolor}%
\pgfsetfillcolor{textcolor}%
\pgftext[x=5.517946in, y=4.709674in, left, base]{\color{textcolor}\sffamily\fontsize{10.000000}{12.000000}\selectfont 0.10}%
\end{pgfscope}%
\begin{pgfscope}%
\pgfpathrectangle{\pgfqpoint{5.924413in}{0.527436in}}{\pgfqpoint{4.403409in}{4.235000in}}%
\pgfusepath{clip}%
\pgfsetrectcap%
\pgfsetroundjoin%
\pgfsetlinewidth{0.803000pt}%
\definecolor{currentstroke}{rgb}{0.600000,0.600000,0.600000}%
\pgfsetstrokecolor{currentstroke}%
\pgfsetstrokeopacity{0.200000}%
\pgfsetdash{}{0pt}%
\pgfpathmoveto{\pgfqpoint{5.924413in}{0.719936in}}%
\pgfpathlineto{\pgfqpoint{10.327822in}{0.719936in}}%
\pgfusepath{stroke}%
\end{pgfscope}%
\begin{pgfscope}%
\pgfsetbuttcap%
\pgfsetroundjoin%
\definecolor{currentfill}{rgb}{0.000000,0.000000,0.000000}%
\pgfsetfillcolor{currentfill}%
\pgfsetlinewidth{0.602250pt}%
\definecolor{currentstroke}{rgb}{0.000000,0.000000,0.000000}%
\pgfsetstrokecolor{currentstroke}%
\pgfsetdash{}{0pt}%
\pgfsys@defobject{currentmarker}{\pgfqpoint{-0.027778in}{0.000000in}}{\pgfqpoint{-0.000000in}{0.000000in}}{%
\pgfpathmoveto{\pgfqpoint{-0.000000in}{0.000000in}}%
\pgfpathlineto{\pgfqpoint{-0.027778in}{0.000000in}}%
\pgfusepath{stroke,fill}%
}%
\begin{pgfscope}%
\pgfsys@transformshift{5.924413in}{0.719936in}%
\pgfsys@useobject{currentmarker}{}%
\end{pgfscope}%
\end{pgfscope}%
\begin{pgfscope}%
\pgfpathrectangle{\pgfqpoint{5.924413in}{0.527436in}}{\pgfqpoint{4.403409in}{4.235000in}}%
\pgfusepath{clip}%
\pgfsetrectcap%
\pgfsetroundjoin%
\pgfsetlinewidth{0.803000pt}%
\definecolor{currentstroke}{rgb}{0.600000,0.600000,0.600000}%
\pgfsetstrokecolor{currentstroke}%
\pgfsetstrokeopacity{0.200000}%
\pgfsetdash{}{0pt}%
\pgfpathmoveto{\pgfqpoint{5.924413in}{1.104936in}}%
\pgfpathlineto{\pgfqpoint{10.327822in}{1.104936in}}%
\pgfusepath{stroke}%
\end{pgfscope}%
\begin{pgfscope}%
\pgfsetbuttcap%
\pgfsetroundjoin%
\definecolor{currentfill}{rgb}{0.000000,0.000000,0.000000}%
\pgfsetfillcolor{currentfill}%
\pgfsetlinewidth{0.602250pt}%
\definecolor{currentstroke}{rgb}{0.000000,0.000000,0.000000}%
\pgfsetstrokecolor{currentstroke}%
\pgfsetdash{}{0pt}%
\pgfsys@defobject{currentmarker}{\pgfqpoint{-0.027778in}{0.000000in}}{\pgfqpoint{-0.000000in}{0.000000in}}{%
\pgfpathmoveto{\pgfqpoint{-0.000000in}{0.000000in}}%
\pgfpathlineto{\pgfqpoint{-0.027778in}{0.000000in}}%
\pgfusepath{stroke,fill}%
}%
\begin{pgfscope}%
\pgfsys@transformshift{5.924413in}{1.104936in}%
\pgfsys@useobject{currentmarker}{}%
\end{pgfscope}%
\end{pgfscope}%
\begin{pgfscope}%
\pgfpathrectangle{\pgfqpoint{5.924413in}{0.527436in}}{\pgfqpoint{4.403409in}{4.235000in}}%
\pgfusepath{clip}%
\pgfsetrectcap%
\pgfsetroundjoin%
\pgfsetlinewidth{0.803000pt}%
\definecolor{currentstroke}{rgb}{0.600000,0.600000,0.600000}%
\pgfsetstrokecolor{currentstroke}%
\pgfsetstrokeopacity{0.200000}%
\pgfsetdash{}{0pt}%
\pgfpathmoveto{\pgfqpoint{5.924413in}{1.297436in}}%
\pgfpathlineto{\pgfqpoint{10.327822in}{1.297436in}}%
\pgfusepath{stroke}%
\end{pgfscope}%
\begin{pgfscope}%
\pgfsetbuttcap%
\pgfsetroundjoin%
\definecolor{currentfill}{rgb}{0.000000,0.000000,0.000000}%
\pgfsetfillcolor{currentfill}%
\pgfsetlinewidth{0.602250pt}%
\definecolor{currentstroke}{rgb}{0.000000,0.000000,0.000000}%
\pgfsetstrokecolor{currentstroke}%
\pgfsetdash{}{0pt}%
\pgfsys@defobject{currentmarker}{\pgfqpoint{-0.027778in}{0.000000in}}{\pgfqpoint{-0.000000in}{0.000000in}}{%
\pgfpathmoveto{\pgfqpoint{-0.000000in}{0.000000in}}%
\pgfpathlineto{\pgfqpoint{-0.027778in}{0.000000in}}%
\pgfusepath{stroke,fill}%
}%
\begin{pgfscope}%
\pgfsys@transformshift{5.924413in}{1.297436in}%
\pgfsys@useobject{currentmarker}{}%
\end{pgfscope}%
\end{pgfscope}%
\begin{pgfscope}%
\pgfpathrectangle{\pgfqpoint{5.924413in}{0.527436in}}{\pgfqpoint{4.403409in}{4.235000in}}%
\pgfusepath{clip}%
\pgfsetrectcap%
\pgfsetroundjoin%
\pgfsetlinewidth{0.803000pt}%
\definecolor{currentstroke}{rgb}{0.600000,0.600000,0.600000}%
\pgfsetstrokecolor{currentstroke}%
\pgfsetstrokeopacity{0.200000}%
\pgfsetdash{}{0pt}%
\pgfpathmoveto{\pgfqpoint{5.924413in}{1.489936in}}%
\pgfpathlineto{\pgfqpoint{10.327822in}{1.489936in}}%
\pgfusepath{stroke}%
\end{pgfscope}%
\begin{pgfscope}%
\pgfsetbuttcap%
\pgfsetroundjoin%
\definecolor{currentfill}{rgb}{0.000000,0.000000,0.000000}%
\pgfsetfillcolor{currentfill}%
\pgfsetlinewidth{0.602250pt}%
\definecolor{currentstroke}{rgb}{0.000000,0.000000,0.000000}%
\pgfsetstrokecolor{currentstroke}%
\pgfsetdash{}{0pt}%
\pgfsys@defobject{currentmarker}{\pgfqpoint{-0.027778in}{0.000000in}}{\pgfqpoint{-0.000000in}{0.000000in}}{%
\pgfpathmoveto{\pgfqpoint{-0.000000in}{0.000000in}}%
\pgfpathlineto{\pgfqpoint{-0.027778in}{0.000000in}}%
\pgfusepath{stroke,fill}%
}%
\begin{pgfscope}%
\pgfsys@transformshift{5.924413in}{1.489936in}%
\pgfsys@useobject{currentmarker}{}%
\end{pgfscope}%
\end{pgfscope}%
\begin{pgfscope}%
\pgfpathrectangle{\pgfqpoint{5.924413in}{0.527436in}}{\pgfqpoint{4.403409in}{4.235000in}}%
\pgfusepath{clip}%
\pgfsetrectcap%
\pgfsetroundjoin%
\pgfsetlinewidth{0.803000pt}%
\definecolor{currentstroke}{rgb}{0.600000,0.600000,0.600000}%
\pgfsetstrokecolor{currentstroke}%
\pgfsetstrokeopacity{0.200000}%
\pgfsetdash{}{0pt}%
\pgfpathmoveto{\pgfqpoint{5.924413in}{1.874936in}}%
\pgfpathlineto{\pgfqpoint{10.327822in}{1.874936in}}%
\pgfusepath{stroke}%
\end{pgfscope}%
\begin{pgfscope}%
\pgfsetbuttcap%
\pgfsetroundjoin%
\definecolor{currentfill}{rgb}{0.000000,0.000000,0.000000}%
\pgfsetfillcolor{currentfill}%
\pgfsetlinewidth{0.602250pt}%
\definecolor{currentstroke}{rgb}{0.000000,0.000000,0.000000}%
\pgfsetstrokecolor{currentstroke}%
\pgfsetdash{}{0pt}%
\pgfsys@defobject{currentmarker}{\pgfqpoint{-0.027778in}{0.000000in}}{\pgfqpoint{-0.000000in}{0.000000in}}{%
\pgfpathmoveto{\pgfqpoint{-0.000000in}{0.000000in}}%
\pgfpathlineto{\pgfqpoint{-0.027778in}{0.000000in}}%
\pgfusepath{stroke,fill}%
}%
\begin{pgfscope}%
\pgfsys@transformshift{5.924413in}{1.874936in}%
\pgfsys@useobject{currentmarker}{}%
\end{pgfscope}%
\end{pgfscope}%
\begin{pgfscope}%
\pgfpathrectangle{\pgfqpoint{5.924413in}{0.527436in}}{\pgfqpoint{4.403409in}{4.235000in}}%
\pgfusepath{clip}%
\pgfsetrectcap%
\pgfsetroundjoin%
\pgfsetlinewidth{0.803000pt}%
\definecolor{currentstroke}{rgb}{0.600000,0.600000,0.600000}%
\pgfsetstrokecolor{currentstroke}%
\pgfsetstrokeopacity{0.200000}%
\pgfsetdash{}{0pt}%
\pgfpathmoveto{\pgfqpoint{5.924413in}{2.067436in}}%
\pgfpathlineto{\pgfqpoint{10.327822in}{2.067436in}}%
\pgfusepath{stroke}%
\end{pgfscope}%
\begin{pgfscope}%
\pgfsetbuttcap%
\pgfsetroundjoin%
\definecolor{currentfill}{rgb}{0.000000,0.000000,0.000000}%
\pgfsetfillcolor{currentfill}%
\pgfsetlinewidth{0.602250pt}%
\definecolor{currentstroke}{rgb}{0.000000,0.000000,0.000000}%
\pgfsetstrokecolor{currentstroke}%
\pgfsetdash{}{0pt}%
\pgfsys@defobject{currentmarker}{\pgfqpoint{-0.027778in}{0.000000in}}{\pgfqpoint{-0.000000in}{0.000000in}}{%
\pgfpathmoveto{\pgfqpoint{-0.000000in}{0.000000in}}%
\pgfpathlineto{\pgfqpoint{-0.027778in}{0.000000in}}%
\pgfusepath{stroke,fill}%
}%
\begin{pgfscope}%
\pgfsys@transformshift{5.924413in}{2.067436in}%
\pgfsys@useobject{currentmarker}{}%
\end{pgfscope}%
\end{pgfscope}%
\begin{pgfscope}%
\pgfpathrectangle{\pgfqpoint{5.924413in}{0.527436in}}{\pgfqpoint{4.403409in}{4.235000in}}%
\pgfusepath{clip}%
\pgfsetrectcap%
\pgfsetroundjoin%
\pgfsetlinewidth{0.803000pt}%
\definecolor{currentstroke}{rgb}{0.600000,0.600000,0.600000}%
\pgfsetstrokecolor{currentstroke}%
\pgfsetstrokeopacity{0.200000}%
\pgfsetdash{}{0pt}%
\pgfpathmoveto{\pgfqpoint{5.924413in}{2.259936in}}%
\pgfpathlineto{\pgfqpoint{10.327822in}{2.259936in}}%
\pgfusepath{stroke}%
\end{pgfscope}%
\begin{pgfscope}%
\pgfsetbuttcap%
\pgfsetroundjoin%
\definecolor{currentfill}{rgb}{0.000000,0.000000,0.000000}%
\pgfsetfillcolor{currentfill}%
\pgfsetlinewidth{0.602250pt}%
\definecolor{currentstroke}{rgb}{0.000000,0.000000,0.000000}%
\pgfsetstrokecolor{currentstroke}%
\pgfsetdash{}{0pt}%
\pgfsys@defobject{currentmarker}{\pgfqpoint{-0.027778in}{0.000000in}}{\pgfqpoint{-0.000000in}{0.000000in}}{%
\pgfpathmoveto{\pgfqpoint{-0.000000in}{0.000000in}}%
\pgfpathlineto{\pgfqpoint{-0.027778in}{0.000000in}}%
\pgfusepath{stroke,fill}%
}%
\begin{pgfscope}%
\pgfsys@transformshift{5.924413in}{2.259936in}%
\pgfsys@useobject{currentmarker}{}%
\end{pgfscope}%
\end{pgfscope}%
\begin{pgfscope}%
\pgfpathrectangle{\pgfqpoint{5.924413in}{0.527436in}}{\pgfqpoint{4.403409in}{4.235000in}}%
\pgfusepath{clip}%
\pgfsetrectcap%
\pgfsetroundjoin%
\pgfsetlinewidth{0.803000pt}%
\definecolor{currentstroke}{rgb}{0.600000,0.600000,0.600000}%
\pgfsetstrokecolor{currentstroke}%
\pgfsetstrokeopacity{0.200000}%
\pgfsetdash{}{0pt}%
\pgfpathmoveto{\pgfqpoint{5.924413in}{2.644936in}}%
\pgfpathlineto{\pgfqpoint{10.327822in}{2.644936in}}%
\pgfusepath{stroke}%
\end{pgfscope}%
\begin{pgfscope}%
\pgfsetbuttcap%
\pgfsetroundjoin%
\definecolor{currentfill}{rgb}{0.000000,0.000000,0.000000}%
\pgfsetfillcolor{currentfill}%
\pgfsetlinewidth{0.602250pt}%
\definecolor{currentstroke}{rgb}{0.000000,0.000000,0.000000}%
\pgfsetstrokecolor{currentstroke}%
\pgfsetdash{}{0pt}%
\pgfsys@defobject{currentmarker}{\pgfqpoint{-0.027778in}{0.000000in}}{\pgfqpoint{-0.000000in}{0.000000in}}{%
\pgfpathmoveto{\pgfqpoint{-0.000000in}{0.000000in}}%
\pgfpathlineto{\pgfqpoint{-0.027778in}{0.000000in}}%
\pgfusepath{stroke,fill}%
}%
\begin{pgfscope}%
\pgfsys@transformshift{5.924413in}{2.644936in}%
\pgfsys@useobject{currentmarker}{}%
\end{pgfscope}%
\end{pgfscope}%
\begin{pgfscope}%
\pgfpathrectangle{\pgfqpoint{5.924413in}{0.527436in}}{\pgfqpoint{4.403409in}{4.235000in}}%
\pgfusepath{clip}%
\pgfsetrectcap%
\pgfsetroundjoin%
\pgfsetlinewidth{0.803000pt}%
\definecolor{currentstroke}{rgb}{0.600000,0.600000,0.600000}%
\pgfsetstrokecolor{currentstroke}%
\pgfsetstrokeopacity{0.200000}%
\pgfsetdash{}{0pt}%
\pgfpathmoveto{\pgfqpoint{5.924413in}{2.837436in}}%
\pgfpathlineto{\pgfqpoint{10.327822in}{2.837436in}}%
\pgfusepath{stroke}%
\end{pgfscope}%
\begin{pgfscope}%
\pgfsetbuttcap%
\pgfsetroundjoin%
\definecolor{currentfill}{rgb}{0.000000,0.000000,0.000000}%
\pgfsetfillcolor{currentfill}%
\pgfsetlinewidth{0.602250pt}%
\definecolor{currentstroke}{rgb}{0.000000,0.000000,0.000000}%
\pgfsetstrokecolor{currentstroke}%
\pgfsetdash{}{0pt}%
\pgfsys@defobject{currentmarker}{\pgfqpoint{-0.027778in}{0.000000in}}{\pgfqpoint{-0.000000in}{0.000000in}}{%
\pgfpathmoveto{\pgfqpoint{-0.000000in}{0.000000in}}%
\pgfpathlineto{\pgfqpoint{-0.027778in}{0.000000in}}%
\pgfusepath{stroke,fill}%
}%
\begin{pgfscope}%
\pgfsys@transformshift{5.924413in}{2.837436in}%
\pgfsys@useobject{currentmarker}{}%
\end{pgfscope}%
\end{pgfscope}%
\begin{pgfscope}%
\pgfpathrectangle{\pgfqpoint{5.924413in}{0.527436in}}{\pgfqpoint{4.403409in}{4.235000in}}%
\pgfusepath{clip}%
\pgfsetrectcap%
\pgfsetroundjoin%
\pgfsetlinewidth{0.803000pt}%
\definecolor{currentstroke}{rgb}{0.600000,0.600000,0.600000}%
\pgfsetstrokecolor{currentstroke}%
\pgfsetstrokeopacity{0.200000}%
\pgfsetdash{}{0pt}%
\pgfpathmoveto{\pgfqpoint{5.924413in}{3.029936in}}%
\pgfpathlineto{\pgfqpoint{10.327822in}{3.029936in}}%
\pgfusepath{stroke}%
\end{pgfscope}%
\begin{pgfscope}%
\pgfsetbuttcap%
\pgfsetroundjoin%
\definecolor{currentfill}{rgb}{0.000000,0.000000,0.000000}%
\pgfsetfillcolor{currentfill}%
\pgfsetlinewidth{0.602250pt}%
\definecolor{currentstroke}{rgb}{0.000000,0.000000,0.000000}%
\pgfsetstrokecolor{currentstroke}%
\pgfsetdash{}{0pt}%
\pgfsys@defobject{currentmarker}{\pgfqpoint{-0.027778in}{0.000000in}}{\pgfqpoint{-0.000000in}{0.000000in}}{%
\pgfpathmoveto{\pgfqpoint{-0.000000in}{0.000000in}}%
\pgfpathlineto{\pgfqpoint{-0.027778in}{0.000000in}}%
\pgfusepath{stroke,fill}%
}%
\begin{pgfscope}%
\pgfsys@transformshift{5.924413in}{3.029936in}%
\pgfsys@useobject{currentmarker}{}%
\end{pgfscope}%
\end{pgfscope}%
\begin{pgfscope}%
\pgfpathrectangle{\pgfqpoint{5.924413in}{0.527436in}}{\pgfqpoint{4.403409in}{4.235000in}}%
\pgfusepath{clip}%
\pgfsetrectcap%
\pgfsetroundjoin%
\pgfsetlinewidth{0.803000pt}%
\definecolor{currentstroke}{rgb}{0.600000,0.600000,0.600000}%
\pgfsetstrokecolor{currentstroke}%
\pgfsetstrokeopacity{0.200000}%
\pgfsetdash{}{0pt}%
\pgfpathmoveto{\pgfqpoint{5.924413in}{3.414936in}}%
\pgfpathlineto{\pgfqpoint{10.327822in}{3.414936in}}%
\pgfusepath{stroke}%
\end{pgfscope}%
\begin{pgfscope}%
\pgfsetbuttcap%
\pgfsetroundjoin%
\definecolor{currentfill}{rgb}{0.000000,0.000000,0.000000}%
\pgfsetfillcolor{currentfill}%
\pgfsetlinewidth{0.602250pt}%
\definecolor{currentstroke}{rgb}{0.000000,0.000000,0.000000}%
\pgfsetstrokecolor{currentstroke}%
\pgfsetdash{}{0pt}%
\pgfsys@defobject{currentmarker}{\pgfqpoint{-0.027778in}{0.000000in}}{\pgfqpoint{-0.000000in}{0.000000in}}{%
\pgfpathmoveto{\pgfqpoint{-0.000000in}{0.000000in}}%
\pgfpathlineto{\pgfqpoint{-0.027778in}{0.000000in}}%
\pgfusepath{stroke,fill}%
}%
\begin{pgfscope}%
\pgfsys@transformshift{5.924413in}{3.414936in}%
\pgfsys@useobject{currentmarker}{}%
\end{pgfscope}%
\end{pgfscope}%
\begin{pgfscope}%
\pgfpathrectangle{\pgfqpoint{5.924413in}{0.527436in}}{\pgfqpoint{4.403409in}{4.235000in}}%
\pgfusepath{clip}%
\pgfsetrectcap%
\pgfsetroundjoin%
\pgfsetlinewidth{0.803000pt}%
\definecolor{currentstroke}{rgb}{0.600000,0.600000,0.600000}%
\pgfsetstrokecolor{currentstroke}%
\pgfsetstrokeopacity{0.200000}%
\pgfsetdash{}{0pt}%
\pgfpathmoveto{\pgfqpoint{5.924413in}{3.607436in}}%
\pgfpathlineto{\pgfqpoint{10.327822in}{3.607436in}}%
\pgfusepath{stroke}%
\end{pgfscope}%
\begin{pgfscope}%
\pgfsetbuttcap%
\pgfsetroundjoin%
\definecolor{currentfill}{rgb}{0.000000,0.000000,0.000000}%
\pgfsetfillcolor{currentfill}%
\pgfsetlinewidth{0.602250pt}%
\definecolor{currentstroke}{rgb}{0.000000,0.000000,0.000000}%
\pgfsetstrokecolor{currentstroke}%
\pgfsetdash{}{0pt}%
\pgfsys@defobject{currentmarker}{\pgfqpoint{-0.027778in}{0.000000in}}{\pgfqpoint{-0.000000in}{0.000000in}}{%
\pgfpathmoveto{\pgfqpoint{-0.000000in}{0.000000in}}%
\pgfpathlineto{\pgfqpoint{-0.027778in}{0.000000in}}%
\pgfusepath{stroke,fill}%
}%
\begin{pgfscope}%
\pgfsys@transformshift{5.924413in}{3.607436in}%
\pgfsys@useobject{currentmarker}{}%
\end{pgfscope}%
\end{pgfscope}%
\begin{pgfscope}%
\pgfpathrectangle{\pgfqpoint{5.924413in}{0.527436in}}{\pgfqpoint{4.403409in}{4.235000in}}%
\pgfusepath{clip}%
\pgfsetrectcap%
\pgfsetroundjoin%
\pgfsetlinewidth{0.803000pt}%
\definecolor{currentstroke}{rgb}{0.600000,0.600000,0.600000}%
\pgfsetstrokecolor{currentstroke}%
\pgfsetstrokeopacity{0.200000}%
\pgfsetdash{}{0pt}%
\pgfpathmoveto{\pgfqpoint{5.924413in}{3.799936in}}%
\pgfpathlineto{\pgfqpoint{10.327822in}{3.799936in}}%
\pgfusepath{stroke}%
\end{pgfscope}%
\begin{pgfscope}%
\pgfsetbuttcap%
\pgfsetroundjoin%
\definecolor{currentfill}{rgb}{0.000000,0.000000,0.000000}%
\pgfsetfillcolor{currentfill}%
\pgfsetlinewidth{0.602250pt}%
\definecolor{currentstroke}{rgb}{0.000000,0.000000,0.000000}%
\pgfsetstrokecolor{currentstroke}%
\pgfsetdash{}{0pt}%
\pgfsys@defobject{currentmarker}{\pgfqpoint{-0.027778in}{0.000000in}}{\pgfqpoint{-0.000000in}{0.000000in}}{%
\pgfpathmoveto{\pgfqpoint{-0.000000in}{0.000000in}}%
\pgfpathlineto{\pgfqpoint{-0.027778in}{0.000000in}}%
\pgfusepath{stroke,fill}%
}%
\begin{pgfscope}%
\pgfsys@transformshift{5.924413in}{3.799936in}%
\pgfsys@useobject{currentmarker}{}%
\end{pgfscope}%
\end{pgfscope}%
\begin{pgfscope}%
\pgfpathrectangle{\pgfqpoint{5.924413in}{0.527436in}}{\pgfqpoint{4.403409in}{4.235000in}}%
\pgfusepath{clip}%
\pgfsetrectcap%
\pgfsetroundjoin%
\pgfsetlinewidth{0.803000pt}%
\definecolor{currentstroke}{rgb}{0.600000,0.600000,0.600000}%
\pgfsetstrokecolor{currentstroke}%
\pgfsetstrokeopacity{0.200000}%
\pgfsetdash{}{0pt}%
\pgfpathmoveto{\pgfqpoint{5.924413in}{4.184936in}}%
\pgfpathlineto{\pgfqpoint{10.327822in}{4.184936in}}%
\pgfusepath{stroke}%
\end{pgfscope}%
\begin{pgfscope}%
\pgfsetbuttcap%
\pgfsetroundjoin%
\definecolor{currentfill}{rgb}{0.000000,0.000000,0.000000}%
\pgfsetfillcolor{currentfill}%
\pgfsetlinewidth{0.602250pt}%
\definecolor{currentstroke}{rgb}{0.000000,0.000000,0.000000}%
\pgfsetstrokecolor{currentstroke}%
\pgfsetdash{}{0pt}%
\pgfsys@defobject{currentmarker}{\pgfqpoint{-0.027778in}{0.000000in}}{\pgfqpoint{-0.000000in}{0.000000in}}{%
\pgfpathmoveto{\pgfqpoint{-0.000000in}{0.000000in}}%
\pgfpathlineto{\pgfqpoint{-0.027778in}{0.000000in}}%
\pgfusepath{stroke,fill}%
}%
\begin{pgfscope}%
\pgfsys@transformshift{5.924413in}{4.184936in}%
\pgfsys@useobject{currentmarker}{}%
\end{pgfscope}%
\end{pgfscope}%
\begin{pgfscope}%
\pgfpathrectangle{\pgfqpoint{5.924413in}{0.527436in}}{\pgfqpoint{4.403409in}{4.235000in}}%
\pgfusepath{clip}%
\pgfsetrectcap%
\pgfsetroundjoin%
\pgfsetlinewidth{0.803000pt}%
\definecolor{currentstroke}{rgb}{0.600000,0.600000,0.600000}%
\pgfsetstrokecolor{currentstroke}%
\pgfsetstrokeopacity{0.200000}%
\pgfsetdash{}{0pt}%
\pgfpathmoveto{\pgfqpoint{5.924413in}{4.377436in}}%
\pgfpathlineto{\pgfqpoint{10.327822in}{4.377436in}}%
\pgfusepath{stroke}%
\end{pgfscope}%
\begin{pgfscope}%
\pgfsetbuttcap%
\pgfsetroundjoin%
\definecolor{currentfill}{rgb}{0.000000,0.000000,0.000000}%
\pgfsetfillcolor{currentfill}%
\pgfsetlinewidth{0.602250pt}%
\definecolor{currentstroke}{rgb}{0.000000,0.000000,0.000000}%
\pgfsetstrokecolor{currentstroke}%
\pgfsetdash{}{0pt}%
\pgfsys@defobject{currentmarker}{\pgfqpoint{-0.027778in}{0.000000in}}{\pgfqpoint{-0.000000in}{0.000000in}}{%
\pgfpathmoveto{\pgfqpoint{-0.000000in}{0.000000in}}%
\pgfpathlineto{\pgfqpoint{-0.027778in}{0.000000in}}%
\pgfusepath{stroke,fill}%
}%
\begin{pgfscope}%
\pgfsys@transformshift{5.924413in}{4.377436in}%
\pgfsys@useobject{currentmarker}{}%
\end{pgfscope}%
\end{pgfscope}%
\begin{pgfscope}%
\pgfpathrectangle{\pgfqpoint{5.924413in}{0.527436in}}{\pgfqpoint{4.403409in}{4.235000in}}%
\pgfusepath{clip}%
\pgfsetrectcap%
\pgfsetroundjoin%
\pgfsetlinewidth{0.803000pt}%
\definecolor{currentstroke}{rgb}{0.600000,0.600000,0.600000}%
\pgfsetstrokecolor{currentstroke}%
\pgfsetstrokeopacity{0.200000}%
\pgfsetdash{}{0pt}%
\pgfpathmoveto{\pgfqpoint{5.924413in}{4.569936in}}%
\pgfpathlineto{\pgfqpoint{10.327822in}{4.569936in}}%
\pgfusepath{stroke}%
\end{pgfscope}%
\begin{pgfscope}%
\pgfsetbuttcap%
\pgfsetroundjoin%
\definecolor{currentfill}{rgb}{0.000000,0.000000,0.000000}%
\pgfsetfillcolor{currentfill}%
\pgfsetlinewidth{0.602250pt}%
\definecolor{currentstroke}{rgb}{0.000000,0.000000,0.000000}%
\pgfsetstrokecolor{currentstroke}%
\pgfsetdash{}{0pt}%
\pgfsys@defobject{currentmarker}{\pgfqpoint{-0.027778in}{0.000000in}}{\pgfqpoint{-0.000000in}{0.000000in}}{%
\pgfpathmoveto{\pgfqpoint{-0.000000in}{0.000000in}}%
\pgfpathlineto{\pgfqpoint{-0.027778in}{0.000000in}}%
\pgfusepath{stroke,fill}%
}%
\begin{pgfscope}%
\pgfsys@transformshift{5.924413in}{4.569936in}%
\pgfsys@useobject{currentmarker}{}%
\end{pgfscope}%
\end{pgfscope}%
\begin{pgfscope}%
\definecolor{textcolor}{rgb}{0.000000,0.000000,0.000000}%
\pgfsetstrokecolor{textcolor}%
\pgfsetfillcolor{textcolor}%
\pgftext[x=5.462391in,y=2.644936in,,bottom,rotate=90.000000]{\color{textcolor}\sffamily\fontsize{10.000000}{12.000000}\selectfont avg. infection rate \(\displaystyle \overline{\langle I\rangle}\)}%
\end{pgfscope}%
\begin{pgfscope}%
\pgfpathrectangle{\pgfqpoint{5.924413in}{0.527436in}}{\pgfqpoint{4.403409in}{4.235000in}}%
\pgfusepath{clip}%
\pgfsetbuttcap%
\pgfsetroundjoin%
\pgfsetlinewidth{1.003750pt}%
\definecolor{currentstroke}{rgb}{0.980392,0.164706,0.333333}%
\pgfsetstrokecolor{currentstroke}%
\pgfsetstrokeopacity{0.500000}%
\pgfsetdash{{3.700000pt}{1.600000pt}}{0.000000pt}%
\pgfpathmoveto{\pgfqpoint{6.010755in}{0.962962in}}%
\pgfpathlineto{\pgfqpoint{6.183437in}{0.967422in}}%
\pgfpathlineto{\pgfqpoint{6.356120in}{0.975468in}}%
\pgfpathlineto{\pgfqpoint{6.528803in}{0.985356in}}%
\pgfpathlineto{\pgfqpoint{6.701486in}{1.000386in}}%
\pgfpathlineto{\pgfqpoint{6.874168in}{1.016682in}}%
\pgfpathlineto{\pgfqpoint{7.046851in}{1.047882in}}%
\pgfpathlineto{\pgfqpoint{7.219534in}{1.093109in}}%
\pgfpathlineto{\pgfqpoint{7.392216in}{1.232039in}}%
\pgfpathlineto{\pgfqpoint{7.564899in}{1.972764in}}%
\pgfpathlineto{\pgfqpoint{7.734649in}{4.772436in}}%
\pgfusepath{stroke}%
\end{pgfscope}%
\begin{pgfscope}%
\pgfpathrectangle{\pgfqpoint{5.924413in}{0.527436in}}{\pgfqpoint{4.403409in}{4.235000in}}%
\pgfusepath{clip}%
\pgfsetbuttcap%
\pgfsetroundjoin%
\pgfsetlinewidth{1.003750pt}%
\definecolor{currentstroke}{rgb}{0.000000,0.000000,1.000000}%
\pgfsetstrokecolor{currentstroke}%
\pgfsetstrokeopacity{0.500000}%
\pgfsetdash{{3.700000pt}{1.600000pt}}{0.000000pt}%
\pgfpathmoveto{\pgfqpoint{6.010755in}{0.940989in}}%
\pgfpathlineto{\pgfqpoint{6.183437in}{0.940599in}}%
\pgfpathlineto{\pgfqpoint{6.356120in}{0.943188in}}%
\pgfpathlineto{\pgfqpoint{6.528803in}{0.945155in}}%
\pgfpathlineto{\pgfqpoint{6.701486in}{0.946979in}}%
\pgfpathlineto{\pgfqpoint{6.874168in}{0.948558in}}%
\pgfpathlineto{\pgfqpoint{7.046851in}{0.950224in}}%
\pgfpathlineto{\pgfqpoint{7.219534in}{0.951171in}}%
\pgfpathlineto{\pgfqpoint{7.392216in}{0.954959in}}%
\pgfpathlineto{\pgfqpoint{7.564899in}{0.957684in}}%
\pgfpathlineto{\pgfqpoint{7.737582in}{0.958533in}}%
\pgfpathlineto{\pgfqpoint{7.910265in}{0.962908in}}%
\pgfpathlineto{\pgfqpoint{8.082947in}{0.966363in}}%
\pgfpathlineto{\pgfqpoint{8.255630in}{0.970505in}}%
\pgfpathlineto{\pgfqpoint{8.428313in}{0.976744in}}%
\pgfpathlineto{\pgfqpoint{8.600995in}{0.980962in}}%
\pgfpathlineto{\pgfqpoint{8.773678in}{0.986287in}}%
\pgfpathlineto{\pgfqpoint{8.946361in}{1.002722in}}%
\pgfpathlineto{\pgfqpoint{9.119044in}{1.014624in}}%
\pgfpathlineto{\pgfqpoint{9.291726in}{1.051371in}}%
\pgfpathlineto{\pgfqpoint{9.464409in}{1.103136in}}%
\pgfpathlineto{\pgfqpoint{9.637092in}{1.216088in}}%
\pgfpathlineto{\pgfqpoint{9.809774in}{1.956787in}}%
\pgfpathlineto{\pgfqpoint{9.982457in}{2.785592in}}%
\pgfpathlineto{\pgfqpoint{10.155140in}{3.606077in}}%
\pgfpathlineto{\pgfqpoint{10.327822in}{4.300501in}}%
\pgfpathlineto{\pgfqpoint{10.337822in}{4.332768in}}%
\pgfusepath{stroke}%
\end{pgfscope}%
\begin{pgfscope}%
\pgfpathrectangle{\pgfqpoint{5.924413in}{0.527436in}}{\pgfqpoint{4.403409in}{4.235000in}}%
\pgfusepath{clip}%
\pgfsetbuttcap%
\pgfsetroundjoin%
\pgfsetlinewidth{1.003750pt}%
\definecolor{currentstroke}{rgb}{0.000000,0.000000,0.000000}%
\pgfsetstrokecolor{currentstroke}%
\pgfsetstrokeopacity{0.500000}%
\pgfsetdash{{3.700000pt}{1.600000pt}}{0.000000pt}%
\pgfpathmoveto{\pgfqpoint{6.010755in}{0.961249in}}%
\pgfpathlineto{\pgfqpoint{6.183437in}{0.969208in}}%
\pgfpathlineto{\pgfqpoint{6.356120in}{0.978180in}}%
\pgfpathlineto{\pgfqpoint{6.528803in}{0.987996in}}%
\pgfpathlineto{\pgfqpoint{6.701486in}{1.009346in}}%
\pgfpathlineto{\pgfqpoint{6.874168in}{1.040418in}}%
\pgfpathlineto{\pgfqpoint{7.046851in}{1.091307in}}%
\pgfpathlineto{\pgfqpoint{7.219534in}{1.213285in}}%
\pgfpathlineto{\pgfqpoint{7.392216in}{2.068733in}}%
\pgfpathlineto{\pgfqpoint{7.514318in}{4.772436in}}%
\pgfusepath{stroke}%
\end{pgfscope}%
\begin{pgfscope}%
\pgfsetrectcap%
\pgfsetmiterjoin%
\pgfsetlinewidth{0.803000pt}%
\definecolor{currentstroke}{rgb}{0.000000,0.000000,0.000000}%
\pgfsetstrokecolor{currentstroke}%
\pgfsetdash{}{0pt}%
\pgfpathmoveto{\pgfqpoint{5.924413in}{0.527436in}}%
\pgfpathlineto{\pgfqpoint{5.924413in}{4.762436in}}%
\pgfusepath{stroke}%
\end{pgfscope}%
\begin{pgfscope}%
\pgfsetrectcap%
\pgfsetmiterjoin%
\pgfsetlinewidth{0.803000pt}%
\definecolor{currentstroke}{rgb}{0.000000,0.000000,0.000000}%
\pgfsetstrokecolor{currentstroke}%
\pgfsetdash{}{0pt}%
\pgfpathmoveto{\pgfqpoint{10.327822in}{0.527436in}}%
\pgfpathlineto{\pgfqpoint{10.327822in}{4.762436in}}%
\pgfusepath{stroke}%
\end{pgfscope}%
\begin{pgfscope}%
\pgfsetrectcap%
\pgfsetmiterjoin%
\pgfsetlinewidth{0.803000pt}%
\definecolor{currentstroke}{rgb}{0.000000,0.000000,0.000000}%
\pgfsetstrokecolor{currentstroke}%
\pgfsetdash{}{0pt}%
\pgfpathmoveto{\pgfqpoint{5.924413in}{0.527436in}}%
\pgfpathlineto{\pgfqpoint{10.327822in}{0.527436in}}%
\pgfusepath{stroke}%
\end{pgfscope}%
\begin{pgfscope}%
\pgfsetrectcap%
\pgfsetmiterjoin%
\pgfsetlinewidth{0.803000pt}%
\definecolor{currentstroke}{rgb}{0.000000,0.000000,0.000000}%
\pgfsetstrokecolor{currentstroke}%
\pgfsetdash{}{0pt}%
\pgfpathmoveto{\pgfqpoint{5.924413in}{4.762436in}}%
\pgfpathlineto{\pgfqpoint{10.327822in}{4.762436in}}%
\pgfusepath{stroke}%
\end{pgfscope}%
\begin{pgfscope}%
\definecolor{textcolor}{rgb}{0.000000,0.000000,0.000000}%
\pgfsetstrokecolor{textcolor}%
\pgfsetfillcolor{textcolor}%
\pgftext[x=8.126118in,y=4.845769in,,base]{\color{textcolor}\sffamily\fontsize{12.000000}{14.400000}\selectfont \(\displaystyle \overline{\langle I\rangle}\) over \(\displaystyle p_1\) for \(\displaystyle T=1000\) for \(\displaystyle L=128\) (magnified)}%
\end{pgfscope}%
\begin{pgfscope}%
\pgfsetbuttcap%
\pgfsetmiterjoin%
\definecolor{currentfill}{rgb}{1.000000,1.000000,1.000000}%
\pgfsetfillcolor{currentfill}%
\pgfsetfillopacity{0.800000}%
\pgfsetlinewidth{1.003750pt}%
\definecolor{currentstroke}{rgb}{0.800000,0.800000,0.800000}%
\pgfsetstrokecolor{currentstroke}%
\pgfsetstrokeopacity{0.800000}%
\pgfsetdash{}{0pt}%
\pgfpathmoveto{\pgfqpoint{6.021636in}{4.039753in}}%
\pgfpathlineto{\pgfqpoint{7.544056in}{4.039753in}}%
\pgfpathquadraticcurveto{\pgfqpoint{7.571834in}{4.039753in}}{\pgfqpoint{7.571834in}{4.067531in}}%
\pgfpathlineto{\pgfqpoint{7.571834in}{4.665214in}}%
\pgfpathquadraticcurveto{\pgfqpoint{7.571834in}{4.692991in}}{\pgfqpoint{7.544056in}{4.692991in}}%
\pgfpathlineto{\pgfqpoint{6.021636in}{4.692991in}}%
\pgfpathquadraticcurveto{\pgfqpoint{5.993858in}{4.692991in}}{\pgfqpoint{5.993858in}{4.665214in}}%
\pgfpathlineto{\pgfqpoint{5.993858in}{4.067531in}}%
\pgfpathquadraticcurveto{\pgfqpoint{5.993858in}{4.039753in}}{\pgfqpoint{6.021636in}{4.039753in}}%
\pgfpathlineto{\pgfqpoint{6.021636in}{4.039753in}}%
\pgfpathclose%
\pgfusepath{stroke,fill}%
\end{pgfscope}%
\begin{pgfscope}%
\pgfsetbuttcap%
\pgfsetroundjoin%
\definecolor{currentfill}{rgb}{0.980392,0.164706,0.333333}%
\pgfsetfillcolor{currentfill}%
\pgfsetfillopacity{0.500000}%
\pgfsetlinewidth{1.003750pt}%
\definecolor{currentstroke}{rgb}{0.980392,0.164706,0.333333}%
\pgfsetstrokecolor{currentstroke}%
\pgfsetstrokeopacity{0.500000}%
\pgfsetdash{}{0pt}%
\pgfsys@defobject{currentmarker}{\pgfqpoint{-0.021960in}{-0.021960in}}{\pgfqpoint{0.021960in}{0.021960in}}{%
\pgfpathmoveto{\pgfqpoint{0.000000in}{-0.021960in}}%
\pgfpathcurveto{\pgfqpoint{0.005824in}{-0.021960in}}{\pgfqpoint{0.011410in}{-0.019646in}}{\pgfqpoint{0.015528in}{-0.015528in}}%
\pgfpathcurveto{\pgfqpoint{0.019646in}{-0.011410in}}{\pgfqpoint{0.021960in}{-0.005824in}}{\pgfqpoint{0.021960in}{0.000000in}}%
\pgfpathcurveto{\pgfqpoint{0.021960in}{0.005824in}}{\pgfqpoint{0.019646in}{0.011410in}}{\pgfqpoint{0.015528in}{0.015528in}}%
\pgfpathcurveto{\pgfqpoint{0.011410in}{0.019646in}}{\pgfqpoint{0.005824in}{0.021960in}}{\pgfqpoint{0.000000in}{0.021960in}}%
\pgfpathcurveto{\pgfqpoint{-0.005824in}{0.021960in}}{\pgfqpoint{-0.011410in}{0.019646in}}{\pgfqpoint{-0.015528in}{0.015528in}}%
\pgfpathcurveto{\pgfqpoint{-0.019646in}{0.011410in}}{\pgfqpoint{-0.021960in}{0.005824in}}{\pgfqpoint{-0.021960in}{0.000000in}}%
\pgfpathcurveto{\pgfqpoint{-0.021960in}{-0.005824in}}{\pgfqpoint{-0.019646in}{-0.011410in}}{\pgfqpoint{-0.015528in}{-0.015528in}}%
\pgfpathcurveto{\pgfqpoint{-0.011410in}{-0.019646in}}{\pgfqpoint{-0.005824in}{-0.021960in}}{\pgfqpoint{0.000000in}{-0.021960in}}%
\pgfpathlineto{\pgfqpoint{0.000000in}{-0.021960in}}%
\pgfpathclose%
\pgfusepath{stroke,fill}%
}%
\begin{pgfscope}%
\pgfsys@transformshift{6.188302in}{4.568371in}%
\pgfsys@useobject{currentmarker}{}%
\end{pgfscope}%
\end{pgfscope}%
\begin{pgfscope}%
\definecolor{textcolor}{rgb}{0.000000,0.000000,0.000000}%
\pgfsetstrokecolor{textcolor}%
\pgfsetfillcolor{textcolor}%
\pgftext[x=6.438302in,y=4.531913in,left,base]{\color{textcolor}\sffamily\fontsize{10.000000}{12.000000}\selectfont \(\displaystyle p_2=0.3\), \(\displaystyle p_3=0.3\)}%
\end{pgfscope}%
\begin{pgfscope}%
\pgfsetbuttcap%
\pgfsetroundjoin%
\definecolor{currentfill}{rgb}{0.000000,0.000000,1.000000}%
\pgfsetfillcolor{currentfill}%
\pgfsetfillopacity{0.500000}%
\pgfsetlinewidth{1.003750pt}%
\definecolor{currentstroke}{rgb}{0.000000,0.000000,1.000000}%
\pgfsetstrokecolor{currentstroke}%
\pgfsetstrokeopacity{0.500000}%
\pgfsetdash{}{0pt}%
\pgfsys@defobject{currentmarker}{\pgfqpoint{-0.021960in}{-0.021960in}}{\pgfqpoint{0.021960in}{0.021960in}}{%
\pgfpathmoveto{\pgfqpoint{0.000000in}{-0.021960in}}%
\pgfpathcurveto{\pgfqpoint{0.005824in}{-0.021960in}}{\pgfqpoint{0.011410in}{-0.019646in}}{\pgfqpoint{0.015528in}{-0.015528in}}%
\pgfpathcurveto{\pgfqpoint{0.019646in}{-0.011410in}}{\pgfqpoint{0.021960in}{-0.005824in}}{\pgfqpoint{0.021960in}{0.000000in}}%
\pgfpathcurveto{\pgfqpoint{0.021960in}{0.005824in}}{\pgfqpoint{0.019646in}{0.011410in}}{\pgfqpoint{0.015528in}{0.015528in}}%
\pgfpathcurveto{\pgfqpoint{0.011410in}{0.019646in}}{\pgfqpoint{0.005824in}{0.021960in}}{\pgfqpoint{0.000000in}{0.021960in}}%
\pgfpathcurveto{\pgfqpoint{-0.005824in}{0.021960in}}{\pgfqpoint{-0.011410in}{0.019646in}}{\pgfqpoint{-0.015528in}{0.015528in}}%
\pgfpathcurveto{\pgfqpoint{-0.019646in}{0.011410in}}{\pgfqpoint{-0.021960in}{0.005824in}}{\pgfqpoint{-0.021960in}{0.000000in}}%
\pgfpathcurveto{\pgfqpoint{-0.021960in}{-0.005824in}}{\pgfqpoint{-0.019646in}{-0.011410in}}{\pgfqpoint{-0.015528in}{-0.015528in}}%
\pgfpathcurveto{\pgfqpoint{-0.011410in}{-0.019646in}}{\pgfqpoint{-0.005824in}{-0.021960in}}{\pgfqpoint{0.000000in}{-0.021960in}}%
\pgfpathlineto{\pgfqpoint{0.000000in}{-0.021960in}}%
\pgfpathclose%
\pgfusepath{stroke,fill}%
}%
\begin{pgfscope}%
\pgfsys@transformshift{6.188302in}{4.364514in}%
\pgfsys@useobject{currentmarker}{}%
\end{pgfscope}%
\end{pgfscope}%
\begin{pgfscope}%
\definecolor{textcolor}{rgb}{0.000000,0.000000,0.000000}%
\pgfsetstrokecolor{textcolor}%
\pgfsetfillcolor{textcolor}%
\pgftext[x=6.438302in,y=4.328056in,left,base]{\color{textcolor}\sffamily\fontsize{10.000000}{12.000000}\selectfont \(\displaystyle p_2=0.6\), \(\displaystyle p_3=0.3\)}%
\end{pgfscope}%
\begin{pgfscope}%
\pgfsetbuttcap%
\pgfsetroundjoin%
\definecolor{currentfill}{rgb}{0.000000,0.000000,0.000000}%
\pgfsetfillcolor{currentfill}%
\pgfsetfillopacity{0.500000}%
\pgfsetlinewidth{1.003750pt}%
\definecolor{currentstroke}{rgb}{0.000000,0.000000,0.000000}%
\pgfsetstrokecolor{currentstroke}%
\pgfsetstrokeopacity{0.500000}%
\pgfsetdash{}{0pt}%
\pgfsys@defobject{currentmarker}{\pgfqpoint{-0.021960in}{-0.021960in}}{\pgfqpoint{0.021960in}{0.021960in}}{%
\pgfpathmoveto{\pgfqpoint{0.000000in}{-0.021960in}}%
\pgfpathcurveto{\pgfqpoint{0.005824in}{-0.021960in}}{\pgfqpoint{0.011410in}{-0.019646in}}{\pgfqpoint{0.015528in}{-0.015528in}}%
\pgfpathcurveto{\pgfqpoint{0.019646in}{-0.011410in}}{\pgfqpoint{0.021960in}{-0.005824in}}{\pgfqpoint{0.021960in}{0.000000in}}%
\pgfpathcurveto{\pgfqpoint{0.021960in}{0.005824in}}{\pgfqpoint{0.019646in}{0.011410in}}{\pgfqpoint{0.015528in}{0.015528in}}%
\pgfpathcurveto{\pgfqpoint{0.011410in}{0.019646in}}{\pgfqpoint{0.005824in}{0.021960in}}{\pgfqpoint{0.000000in}{0.021960in}}%
\pgfpathcurveto{\pgfqpoint{-0.005824in}{0.021960in}}{\pgfqpoint{-0.011410in}{0.019646in}}{\pgfqpoint{-0.015528in}{0.015528in}}%
\pgfpathcurveto{\pgfqpoint{-0.019646in}{0.011410in}}{\pgfqpoint{-0.021960in}{0.005824in}}{\pgfqpoint{-0.021960in}{0.000000in}}%
\pgfpathcurveto{\pgfqpoint{-0.021960in}{-0.005824in}}{\pgfqpoint{-0.019646in}{-0.011410in}}{\pgfqpoint{-0.015528in}{-0.015528in}}%
\pgfpathcurveto{\pgfqpoint{-0.011410in}{-0.019646in}}{\pgfqpoint{-0.005824in}{-0.021960in}}{\pgfqpoint{0.000000in}{-0.021960in}}%
\pgfpathlineto{\pgfqpoint{0.000000in}{-0.021960in}}%
\pgfpathclose%
\pgfusepath{stroke,fill}%
}%
\begin{pgfscope}%
\pgfsys@transformshift{6.188302in}{4.160657in}%
\pgfsys@useobject{currentmarker}{}%
\end{pgfscope}%
\end{pgfscope}%
\begin{pgfscope}%
\definecolor{textcolor}{rgb}{0.000000,0.000000,0.000000}%
\pgfsetstrokecolor{textcolor}%
\pgfsetfillcolor{textcolor}%
\pgftext[x=6.438302in,y=4.124198in,left,base]{\color{textcolor}\sffamily\fontsize{10.000000}{12.000000}\selectfont \(\displaystyle p_2=0.3\), \(\displaystyle p_3=0.6\)}%
\end{pgfscope}%
\end{pgfpicture}%
\makeatother%
\endgroup%
}
    \caption{Detailed view of the time-averaged infection rates $\overline{\langle I\rangle}$ over $p_1\left(\susceptible\rightarrow\infected\,\right)$ for $L=128$ and $T=1000$ simulation steps. On the left the different
    trends for the respective choices of $p_2\left(\susceptible\rightarrow\infected\,\right)$ and $p_3\left(\susceptible\rightarrow\infected\,\right)$ can be observed, 
    while on the right the critical values of $p_1$ for $\overline{\langle I\rangle}$ approaching zero are visible.}\label{fig:Res_Dis_Avg_Inf_over_p1_L128}
\end{figure}

\subsection{Vaccinated People without Participation in the Spread}

\begin{figure}[ht]
    \centering
    \resizebox{\textwidth}{!}{%% Creator: Matplotlib, PGF backend
%%
%% To include the figure in your LaTeX document, write
%%   \input{<filename>.pgf}
%%
%% Make sure the required packages are loaded in your preamble
%%   \usepackage{pgf}
%%
%% Also ensure that all the required font packages are loaded; for instance,
%% the lmodern package is sometimes necessary when using math font.
%%   \usepackage{lmodern}
%%
%% Figures using additional raster images can only be included by \input if
%% they are in the same directory as the main LaTeX file. For loading figures
%% from other directories you can use the `import` package
%%   \usepackage{import}
%%
%% and then include the figures with
%%   \import{<path to file>}{<filename>.pgf}
%%
%% Matplotlib used the following preamble
%%   
%%   \usepackage{fontspec}
%%   \setmainfont{DejaVuSerif.ttf}[Path=\detokenize{/home/carlo/.local/lib/python3.10/site-packages/matplotlib/mpl-data/fonts/ttf/}]
%%   \setsansfont{DejaVuSans.ttf}[Path=\detokenize{/home/carlo/.local/lib/python3.10/site-packages/matplotlib/mpl-data/fonts/ttf/}]
%%   \setmonofont{DejaVuSansMono.ttf}[Path=\detokenize{/home/carlo/.local/lib/python3.10/site-packages/matplotlib/mpl-data/fonts/ttf/}]
%%   \makeatletter\@ifpackageloaded{underscore}{}{\usepackage[strings]{underscore}}\makeatother
%%
\begingroup%
\makeatletter%
\begin{pgfpicture}%
\pgfpathrectangle{\pgfpointorigin}{\pgfqpoint{10.516188in}{5.092713in}}%
\pgfusepath{use as bounding box, clip}%
\begin{pgfscope}%
\pgfsetbuttcap%
\pgfsetmiterjoin%
\definecolor{currentfill}{rgb}{1.000000,1.000000,1.000000}%
\pgfsetfillcolor{currentfill}%
\pgfsetlinewidth{0.000000pt}%
\definecolor{currentstroke}{rgb}{1.000000,1.000000,1.000000}%
\pgfsetstrokecolor{currentstroke}%
\pgfsetdash{}{0pt}%
\pgfpathmoveto{\pgfqpoint{0.000000in}{0.000000in}}%
\pgfpathlineto{\pgfqpoint{10.516188in}{0.000000in}}%
\pgfpathlineto{\pgfqpoint{10.516188in}{5.092713in}}%
\pgfpathlineto{\pgfqpoint{0.000000in}{5.092713in}}%
\pgfpathlineto{\pgfqpoint{0.000000in}{0.000000in}}%
\pgfpathclose%
\pgfusepath{fill}%
\end{pgfscope}%
\begin{pgfscope}%
\pgfsetbuttcap%
\pgfsetmiterjoin%
\definecolor{currentfill}{rgb}{1.000000,1.000000,1.000000}%
\pgfsetfillcolor{currentfill}%
\pgfsetlinewidth{0.000000pt}%
\definecolor{currentstroke}{rgb}{0.000000,0.000000,0.000000}%
\pgfsetstrokecolor{currentstroke}%
\pgfsetstrokeopacity{0.000000}%
\pgfsetdash{}{0pt}%
\pgfpathmoveto{\pgfqpoint{0.728688in}{0.521603in}}%
\pgfpathlineto{\pgfqpoint{10.416188in}{0.521603in}}%
\pgfpathlineto{\pgfqpoint{10.416188in}{4.756603in}}%
\pgfpathlineto{\pgfqpoint{0.728688in}{4.756603in}}%
\pgfpathlineto{\pgfqpoint{0.728688in}{0.521603in}}%
\pgfpathclose%
\pgfusepath{fill}%
\end{pgfscope}%
\begin{pgfscope}%
\pgfpathrectangle{\pgfqpoint{0.728688in}{0.521603in}}{\pgfqpoint{9.687500in}{4.235000in}}%
\pgfusepath{clip}%
\pgfsetbuttcap%
\pgfsetroundjoin%
\definecolor{currentfill}{rgb}{0.000000,0.000000,1.000000}%
\pgfsetfillcolor{currentfill}%
\pgfsetfillopacity{0.500000}%
\pgfsetlinewidth{1.003750pt}%
\definecolor{currentstroke}{rgb}{0.000000,0.000000,1.000000}%
\pgfsetstrokecolor{currentstroke}%
\pgfsetstrokeopacity{0.500000}%
\pgfsetdash{}{0pt}%
\pgfsys@defobject{currentmarker}{\pgfqpoint{-0.021960in}{-0.021960in}}{\pgfqpoint{0.021960in}{0.021960in}}{%
\pgfpathmoveto{\pgfqpoint{0.000000in}{-0.021960in}}%
\pgfpathcurveto{\pgfqpoint{0.005824in}{-0.021960in}}{\pgfqpoint{0.011410in}{-0.019646in}}{\pgfqpoint{0.015528in}{-0.015528in}}%
\pgfpathcurveto{\pgfqpoint{0.019646in}{-0.011410in}}{\pgfqpoint{0.021960in}{-0.005824in}}{\pgfqpoint{0.021960in}{0.000000in}}%
\pgfpathcurveto{\pgfqpoint{0.021960in}{0.005824in}}{\pgfqpoint{0.019646in}{0.011410in}}{\pgfqpoint{0.015528in}{0.015528in}}%
\pgfpathcurveto{\pgfqpoint{0.011410in}{0.019646in}}{\pgfqpoint{0.005824in}{0.021960in}}{\pgfqpoint{0.000000in}{0.021960in}}%
\pgfpathcurveto{\pgfqpoint{-0.005824in}{0.021960in}}{\pgfqpoint{-0.011410in}{0.019646in}}{\pgfqpoint{-0.015528in}{0.015528in}}%
\pgfpathcurveto{\pgfqpoint{-0.019646in}{0.011410in}}{\pgfqpoint{-0.021960in}{0.005824in}}{\pgfqpoint{-0.021960in}{0.000000in}}%
\pgfpathcurveto{\pgfqpoint{-0.021960in}{-0.005824in}}{\pgfqpoint{-0.019646in}{-0.011410in}}{\pgfqpoint{-0.015528in}{-0.015528in}}%
\pgfpathcurveto{\pgfqpoint{-0.011410in}{-0.019646in}}{\pgfqpoint{-0.005824in}{-0.021960in}}{\pgfqpoint{0.000000in}{-0.021960in}}%
\pgfpathlineto{\pgfqpoint{0.000000in}{-0.021960in}}%
\pgfpathclose%
\pgfusepath{stroke,fill}%
}%
\begin{pgfscope}%
\pgfsys@transformshift{1.169029in}{4.046571in}%
\pgfsys@useobject{currentmarker}{}%
\end{pgfscope}%
\begin{pgfscope}%
\pgfsys@transformshift{1.348760in}{4.207524in}%
\pgfsys@useobject{currentmarker}{}%
\end{pgfscope}%
\begin{pgfscope}%
\pgfsys@transformshift{1.528491in}{3.612032in}%
\pgfsys@useobject{currentmarker}{}%
\end{pgfscope}%
\begin{pgfscope}%
\pgfsys@transformshift{1.708222in}{3.675538in}%
\pgfsys@useobject{currentmarker}{}%
\end{pgfscope}%
\begin{pgfscope}%
\pgfsys@transformshift{1.887953in}{3.140982in}%
\pgfsys@useobject{currentmarker}{}%
\end{pgfscope}%
\begin{pgfscope}%
\pgfsys@transformshift{2.067684in}{3.043000in}%
\pgfsys@useobject{currentmarker}{}%
\end{pgfscope}%
\begin{pgfscope}%
\pgfsys@transformshift{2.247415in}{0.862755in}%
\pgfsys@useobject{currentmarker}{}%
\end{pgfscope}%
\begin{pgfscope}%
\pgfsys@transformshift{2.427146in}{1.019771in}%
\pgfsys@useobject{currentmarker}{}%
\end{pgfscope}%
\begin{pgfscope}%
\pgfsys@transformshift{2.606877in}{0.809341in}%
\pgfsys@useobject{currentmarker}{}%
\end{pgfscope}%
\begin{pgfscope}%
\pgfsys@transformshift{2.786608in}{1.230678in}%
\pgfsys@useobject{currentmarker}{}%
\end{pgfscope}%
\begin{pgfscope}%
\pgfsys@transformshift{2.966339in}{0.982488in}%
\pgfsys@useobject{currentmarker}{}%
\end{pgfscope}%
\begin{pgfscope}%
\pgfsys@transformshift{3.146070in}{0.878766in}%
\pgfsys@useobject{currentmarker}{}%
\end{pgfscope}%
\begin{pgfscope}%
\pgfsys@transformshift{3.325801in}{0.780542in}%
\pgfsys@useobject{currentmarker}{}%
\end{pgfscope}%
\begin{pgfscope}%
\pgfsys@transformshift{3.505532in}{0.772835in}%
\pgfsys@useobject{currentmarker}{}%
\end{pgfscope}%
\begin{pgfscope}%
\pgfsys@transformshift{3.685263in}{0.745829in}%
\pgfsys@useobject{currentmarker}{}%
\end{pgfscope}%
\begin{pgfscope}%
\pgfsys@transformshift{3.864994in}{0.778630in}%
\pgfsys@useobject{currentmarker}{}%
\end{pgfscope}%
\begin{pgfscope}%
\pgfsys@transformshift{4.044725in}{0.752521in}%
\pgfsys@useobject{currentmarker}{}%
\end{pgfscope}%
\begin{pgfscope}%
\pgfsys@transformshift{4.224456in}{0.752342in}%
\pgfsys@useobject{currentmarker}{}%
\end{pgfscope}%
\begin{pgfscope}%
\pgfsys@transformshift{4.404187in}{0.755628in}%
\pgfsys@useobject{currentmarker}{}%
\end{pgfscope}%
\begin{pgfscope}%
\pgfsys@transformshift{4.583918in}{0.731131in}%
\pgfsys@useobject{currentmarker}{}%
\end{pgfscope}%
\begin{pgfscope}%
\pgfsys@transformshift{4.763648in}{0.732625in}%
\pgfsys@useobject{currentmarker}{}%
\end{pgfscope}%
\begin{pgfscope}%
\pgfsys@transformshift{4.943379in}{0.731609in}%
\pgfsys@useobject{currentmarker}{}%
\end{pgfscope}%
\begin{pgfscope}%
\pgfsys@transformshift{5.123110in}{0.734955in}%
\pgfsys@useobject{currentmarker}{}%
\end{pgfscope}%
\begin{pgfscope}%
\pgfsys@transformshift{5.302841in}{0.739974in}%
\pgfsys@useobject{currentmarker}{}%
\end{pgfscope}%
\begin{pgfscope}%
\pgfsys@transformshift{5.482572in}{0.741348in}%
\pgfsys@useobject{currentmarker}{}%
\end{pgfscope}%
\begin{pgfscope}%
\pgfsys@transformshift{5.662303in}{0.727128in}%
\pgfsys@useobject{currentmarker}{}%
\end{pgfscope}%
\begin{pgfscope}%
\pgfsys@transformshift{5.842034in}{0.722886in}%
\pgfsys@useobject{currentmarker}{}%
\end{pgfscope}%
\begin{pgfscope}%
\pgfsys@transformshift{6.021765in}{0.720078in}%
\pgfsys@useobject{currentmarker}{}%
\end{pgfscope}%
\begin{pgfscope}%
\pgfsys@transformshift{6.201496in}{0.726053in}%
\pgfsys@useobject{currentmarker}{}%
\end{pgfscope}%
\begin{pgfscope}%
\pgfsys@transformshift{6.381227in}{0.719301in}%
\pgfsys@useobject{currentmarker}{}%
\end{pgfscope}%
\begin{pgfscope}%
\pgfsys@transformshift{6.560958in}{0.718883in}%
\pgfsys@useobject{currentmarker}{}%
\end{pgfscope}%
\begin{pgfscope}%
\pgfsys@transformshift{6.740689in}{0.728502in}%
\pgfsys@useobject{currentmarker}{}%
\end{pgfscope}%
\begin{pgfscope}%
\pgfsys@transformshift{6.920420in}{0.721811in}%
\pgfsys@useobject{currentmarker}{}%
\end{pgfscope}%
\begin{pgfscope}%
\pgfsys@transformshift{7.100151in}{0.719720in}%
\pgfsys@useobject{currentmarker}{}%
\end{pgfscope}%
\begin{pgfscope}%
\pgfsys@transformshift{7.279882in}{0.719540in}%
\pgfsys@useobject{currentmarker}{}%
\end{pgfscope}%
\begin{pgfscope}%
\pgfsys@transformshift{7.459613in}{0.722647in}%
\pgfsys@useobject{currentmarker}{}%
\end{pgfscope}%
\begin{pgfscope}%
\pgfsys@transformshift{7.639344in}{0.718166in}%
\pgfsys@useobject{currentmarker}{}%
\end{pgfscope}%
\begin{pgfscope}%
\pgfsys@transformshift{7.819075in}{0.719182in}%
\pgfsys@useobject{currentmarker}{}%
\end{pgfscope}%
\begin{pgfscope}%
\pgfsys@transformshift{7.998806in}{0.715717in}%
\pgfsys@useobject{currentmarker}{}%
\end{pgfscope}%
\begin{pgfscope}%
\pgfsys@transformshift{8.178537in}{0.718106in}%
\pgfsys@useobject{currentmarker}{}%
\end{pgfscope}%
\begin{pgfscope}%
\pgfsys@transformshift{8.358268in}{0.718106in}%
\pgfsys@useobject{currentmarker}{}%
\end{pgfscope}%
\begin{pgfscope}%
\pgfsys@transformshift{8.537999in}{0.715955in}%
\pgfsys@useobject{currentmarker}{}%
\end{pgfscope}%
\begin{pgfscope}%
\pgfsys@transformshift{8.717730in}{0.715298in}%
\pgfsys@useobject{currentmarker}{}%
\end{pgfscope}%
\begin{pgfscope}%
\pgfsys@transformshift{8.897461in}{0.715657in}%
\pgfsys@useobject{currentmarker}{}%
\end{pgfscope}%
\begin{pgfscope}%
\pgfsys@transformshift{9.077192in}{0.717808in}%
\pgfsys@useobject{currentmarker}{}%
\end{pgfscope}%
\begin{pgfscope}%
\pgfsys@transformshift{9.256923in}{0.715478in}%
\pgfsys@useobject{currentmarker}{}%
\end{pgfscope}%
\begin{pgfscope}%
\pgfsys@transformshift{9.436654in}{0.714103in}%
\pgfsys@useobject{currentmarker}{}%
\end{pgfscope}%
\begin{pgfscope}%
\pgfsys@transformshift{9.616385in}{0.716135in}%
\pgfsys@useobject{currentmarker}{}%
\end{pgfscope}%
\begin{pgfscope}%
\pgfsys@transformshift{9.796116in}{0.714283in}%
\pgfsys@useobject{currentmarker}{}%
\end{pgfscope}%
\begin{pgfscope}%
\pgfsys@transformshift{9.975847in}{0.714342in}%
\pgfsys@useobject{currentmarker}{}%
\end{pgfscope}%
\end{pgfscope}%
\begin{pgfscope}%
\pgfpathrectangle{\pgfqpoint{0.728688in}{0.521603in}}{\pgfqpoint{9.687500in}{4.235000in}}%
\pgfusepath{clip}%
\pgfsetbuttcap%
\pgfsetroundjoin%
\definecolor{currentfill}{rgb}{0.980392,0.164706,0.333333}%
\pgfsetfillcolor{currentfill}%
\pgfsetfillopacity{0.500000}%
\pgfsetlinewidth{1.003750pt}%
\definecolor{currentstroke}{rgb}{0.980392,0.164706,0.333333}%
\pgfsetstrokecolor{currentstroke}%
\pgfsetstrokeopacity{0.500000}%
\pgfsetdash{}{0pt}%
\pgfsys@defobject{currentmarker}{\pgfqpoint{-0.021960in}{-0.021960in}}{\pgfqpoint{0.021960in}{0.021960in}}{%
\pgfpathmoveto{\pgfqpoint{0.000000in}{-0.021960in}}%
\pgfpathcurveto{\pgfqpoint{0.005824in}{-0.021960in}}{\pgfqpoint{0.011410in}{-0.019646in}}{\pgfqpoint{0.015528in}{-0.015528in}}%
\pgfpathcurveto{\pgfqpoint{0.019646in}{-0.011410in}}{\pgfqpoint{0.021960in}{-0.005824in}}{\pgfqpoint{0.021960in}{0.000000in}}%
\pgfpathcurveto{\pgfqpoint{0.021960in}{0.005824in}}{\pgfqpoint{0.019646in}{0.011410in}}{\pgfqpoint{0.015528in}{0.015528in}}%
\pgfpathcurveto{\pgfqpoint{0.011410in}{0.019646in}}{\pgfqpoint{0.005824in}{0.021960in}}{\pgfqpoint{0.000000in}{0.021960in}}%
\pgfpathcurveto{\pgfqpoint{-0.005824in}{0.021960in}}{\pgfqpoint{-0.011410in}{0.019646in}}{\pgfqpoint{-0.015528in}{0.015528in}}%
\pgfpathcurveto{\pgfqpoint{-0.019646in}{0.011410in}}{\pgfqpoint{-0.021960in}{0.005824in}}{\pgfqpoint{-0.021960in}{0.000000in}}%
\pgfpathcurveto{\pgfqpoint{-0.021960in}{-0.005824in}}{\pgfqpoint{-0.019646in}{-0.011410in}}{\pgfqpoint{-0.015528in}{-0.015528in}}%
\pgfpathcurveto{\pgfqpoint{-0.011410in}{-0.019646in}}{\pgfqpoint{-0.005824in}{-0.021960in}}{\pgfqpoint{0.000000in}{-0.021960in}}%
\pgfpathlineto{\pgfqpoint{0.000000in}{-0.021960in}}%
\pgfpathclose%
\pgfusepath{stroke,fill}%
}%
\begin{pgfscope}%
\pgfsys@transformshift{1.169029in}{4.391817in}%
\pgfsys@useobject{currentmarker}{}%
\end{pgfscope}%
\begin{pgfscope}%
\pgfsys@transformshift{1.348760in}{4.322162in}%
\pgfsys@useobject{currentmarker}{}%
\end{pgfscope}%
\begin{pgfscope}%
\pgfsys@transformshift{1.528491in}{4.026152in}%
\pgfsys@useobject{currentmarker}{}%
\end{pgfscope}%
\begin{pgfscope}%
\pgfsys@transformshift{1.708222in}{3.852214in}%
\pgfsys@useobject{currentmarker}{}%
\end{pgfscope}%
\begin{pgfscope}%
\pgfsys@transformshift{1.887953in}{3.563958in}%
\pgfsys@useobject{currentmarker}{}%
\end{pgfscope}%
\begin{pgfscope}%
\pgfsys@transformshift{2.067684in}{3.194653in}%
\pgfsys@useobject{currentmarker}{}%
\end{pgfscope}%
\begin{pgfscope}%
\pgfsys@transformshift{2.247415in}{2.992969in}%
\pgfsys@useobject{currentmarker}{}%
\end{pgfscope}%
\begin{pgfscope}%
\pgfsys@transformshift{2.427146in}{2.672533in}%
\pgfsys@useobject{currentmarker}{}%
\end{pgfscope}%
\begin{pgfscope}%
\pgfsys@transformshift{2.606877in}{2.406884in}%
\pgfsys@useobject{currentmarker}{}%
\end{pgfscope}%
\begin{pgfscope}%
\pgfsys@transformshift{2.786608in}{2.273233in}%
\pgfsys@useobject{currentmarker}{}%
\end{pgfscope}%
\begin{pgfscope}%
\pgfsys@transformshift{2.966339in}{1.935697in}%
\pgfsys@useobject{currentmarker}{}%
\end{pgfscope}%
\begin{pgfscope}%
\pgfsys@transformshift{3.146070in}{1.472835in}%
\pgfsys@useobject{currentmarker}{}%
\end{pgfscope}%
\begin{pgfscope}%
\pgfsys@transformshift{3.325801in}{0.914585in}%
\pgfsys@useobject{currentmarker}{}%
\end{pgfscope}%
\begin{pgfscope}%
\pgfsys@transformshift{3.505532in}{0.808922in}%
\pgfsys@useobject{currentmarker}{}%
\end{pgfscope}%
\begin{pgfscope}%
\pgfsys@transformshift{3.685263in}{0.784725in}%
\pgfsys@useobject{currentmarker}{}%
\end{pgfscope}%
\begin{pgfscope}%
\pgfsys@transformshift{3.864994in}{0.777092in}%
\pgfsys@useobject{currentmarker}{}%
\end{pgfscope}%
\begin{pgfscope}%
\pgfsys@transformshift{4.044725in}{0.782215in}%
\pgfsys@useobject{currentmarker}{}%
\end{pgfscope}%
\begin{pgfscope}%
\pgfsys@transformshift{4.224456in}{0.762887in}%
\pgfsys@useobject{currentmarker}{}%
\end{pgfscope}%
\begin{pgfscope}%
\pgfsys@transformshift{4.404187in}{0.735836in}%
\pgfsys@useobject{currentmarker}{}%
\end{pgfscope}%
\begin{pgfscope}%
\pgfsys@transformshift{4.583918in}{0.743051in}%
\pgfsys@useobject{currentmarker}{}%
\end{pgfscope}%
\begin{pgfscope}%
\pgfsys@transformshift{4.763648in}{0.742334in}%
\pgfsys@useobject{currentmarker}{}%
\end{pgfscope}%
\begin{pgfscope}%
\pgfsys@transformshift{4.943379in}{0.731923in}%
\pgfsys@useobject{currentmarker}{}%
\end{pgfscope}%
\begin{pgfscope}%
\pgfsys@transformshift{5.123110in}{0.737465in}%
\pgfsys@useobject{currentmarker}{}%
\end{pgfscope}%
\begin{pgfscope}%
\pgfsys@transformshift{5.302841in}{0.734134in}%
\pgfsys@useobject{currentmarker}{}%
\end{pgfscope}%
\begin{pgfscope}%
\pgfsys@transformshift{5.482572in}{0.732909in}%
\pgfsys@useobject{currentmarker}{}%
\end{pgfscope}%
\begin{pgfscope}%
\pgfsys@transformshift{5.662303in}{0.729443in}%
\pgfsys@useobject{currentmarker}{}%
\end{pgfscope}%
\begin{pgfscope}%
\pgfsys@transformshift{5.842034in}{0.725948in}%
\pgfsys@useobject{currentmarker}{}%
\end{pgfscope}%
\begin{pgfscope}%
\pgfsys@transformshift{6.021765in}{0.726098in}%
\pgfsys@useobject{currentmarker}{}%
\end{pgfscope}%
\begin{pgfscope}%
\pgfsys@transformshift{6.201496in}{0.724440in}%
\pgfsys@useobject{currentmarker}{}%
\end{pgfscope}%
\begin{pgfscope}%
\pgfsys@transformshift{6.381227in}{0.726830in}%
\pgfsys@useobject{currentmarker}{}%
\end{pgfscope}%
\begin{pgfscope}%
\pgfsys@transformshift{6.560958in}{0.723230in}%
\pgfsys@useobject{currentmarker}{}%
\end{pgfscope}%
\begin{pgfscope}%
\pgfsys@transformshift{6.740689in}{0.724664in}%
\pgfsys@useobject{currentmarker}{}%
\end{pgfscope}%
\begin{pgfscope}%
\pgfsys@transformshift{6.920420in}{0.720870in}%
\pgfsys@useobject{currentmarker}{}%
\end{pgfscope}%
\begin{pgfscope}%
\pgfsys@transformshift{7.100151in}{0.720586in}%
\pgfsys@useobject{currentmarker}{}%
\end{pgfscope}%
\begin{pgfscope}%
\pgfsys@transformshift{7.279882in}{0.720750in}%
\pgfsys@useobject{currentmarker}{}%
\end{pgfscope}%
\begin{pgfscope}%
\pgfsys@transformshift{7.459613in}{0.721139in}%
\pgfsys@useobject{currentmarker}{}%
\end{pgfscope}%
\begin{pgfscope}%
\pgfsys@transformshift{7.639344in}{0.719003in}%
\pgfsys@useobject{currentmarker}{}%
\end{pgfscope}%
\begin{pgfscope}%
\pgfsys@transformshift{7.819075in}{0.719122in}%
\pgfsys@useobject{currentmarker}{}%
\end{pgfscope}%
\begin{pgfscope}%
\pgfsys@transformshift{7.998806in}{0.717852in}%
\pgfsys@useobject{currentmarker}{}%
\end{pgfscope}%
\begin{pgfscope}%
\pgfsys@transformshift{8.178537in}{0.719167in}%
\pgfsys@useobject{currentmarker}{}%
\end{pgfscope}%
\begin{pgfscope}%
\pgfsys@transformshift{8.358268in}{0.717494in}%
\pgfsys@useobject{currentmarker}{}%
\end{pgfscope}%
\begin{pgfscope}%
\pgfsys@transformshift{8.537999in}{0.716643in}%
\pgfsys@useobject{currentmarker}{}%
\end{pgfscope}%
\begin{pgfscope}%
\pgfsys@transformshift{8.717730in}{0.716404in}%
\pgfsys@useobject{currentmarker}{}%
\end{pgfscope}%
\begin{pgfscope}%
\pgfsys@transformshift{8.897461in}{0.716090in}%
\pgfsys@useobject{currentmarker}{}%
\end{pgfscope}%
\begin{pgfscope}%
\pgfsys@transformshift{9.077192in}{0.716344in}%
\pgfsys@useobject{currentmarker}{}%
\end{pgfscope}%
\begin{pgfscope}%
\pgfsys@transformshift{9.256923in}{0.715089in}%
\pgfsys@useobject{currentmarker}{}%
\end{pgfscope}%
\begin{pgfscope}%
\pgfsys@transformshift{9.436654in}{0.715283in}%
\pgfsys@useobject{currentmarker}{}%
\end{pgfscope}%
\begin{pgfscope}%
\pgfsys@transformshift{9.616385in}{0.714835in}%
\pgfsys@useobject{currentmarker}{}%
\end{pgfscope}%
\begin{pgfscope}%
\pgfsys@transformshift{9.796116in}{0.714566in}%
\pgfsys@useobject{currentmarker}{}%
\end{pgfscope}%
\begin{pgfscope}%
\pgfsys@transformshift{9.975847in}{0.714327in}%
\pgfsys@useobject{currentmarker}{}%
\end{pgfscope}%
\end{pgfscope}%
\begin{pgfscope}%
\pgfpathrectangle{\pgfqpoint{0.728688in}{0.521603in}}{\pgfqpoint{9.687500in}{4.235000in}}%
\pgfusepath{clip}%
\pgfsetbuttcap%
\pgfsetroundjoin%
\definecolor{currentfill}{rgb}{0.239216,0.478431,0.992157}%
\pgfsetfillcolor{currentfill}%
\pgfsetfillopacity{0.500000}%
\pgfsetlinewidth{1.003750pt}%
\definecolor{currentstroke}{rgb}{0.239216,0.478431,0.992157}%
\pgfsetstrokecolor{currentstroke}%
\pgfsetstrokeopacity{0.500000}%
\pgfsetdash{}{0pt}%
\pgfsys@defobject{currentmarker}{\pgfqpoint{-0.021960in}{-0.021960in}}{\pgfqpoint{0.021960in}{0.021960in}}{%
\pgfpathmoveto{\pgfqpoint{0.000000in}{-0.021960in}}%
\pgfpathcurveto{\pgfqpoint{0.005824in}{-0.021960in}}{\pgfqpoint{0.011410in}{-0.019646in}}{\pgfqpoint{0.015528in}{-0.015528in}}%
\pgfpathcurveto{\pgfqpoint{0.019646in}{-0.011410in}}{\pgfqpoint{0.021960in}{-0.005824in}}{\pgfqpoint{0.021960in}{0.000000in}}%
\pgfpathcurveto{\pgfqpoint{0.021960in}{0.005824in}}{\pgfqpoint{0.019646in}{0.011410in}}{\pgfqpoint{0.015528in}{0.015528in}}%
\pgfpathcurveto{\pgfqpoint{0.011410in}{0.019646in}}{\pgfqpoint{0.005824in}{0.021960in}}{\pgfqpoint{0.000000in}{0.021960in}}%
\pgfpathcurveto{\pgfqpoint{-0.005824in}{0.021960in}}{\pgfqpoint{-0.011410in}{0.019646in}}{\pgfqpoint{-0.015528in}{0.015528in}}%
\pgfpathcurveto{\pgfqpoint{-0.019646in}{0.011410in}}{\pgfqpoint{-0.021960in}{0.005824in}}{\pgfqpoint{-0.021960in}{0.000000in}}%
\pgfpathcurveto{\pgfqpoint{-0.021960in}{-0.005824in}}{\pgfqpoint{-0.019646in}{-0.011410in}}{\pgfqpoint{-0.015528in}{-0.015528in}}%
\pgfpathcurveto{\pgfqpoint{-0.011410in}{-0.019646in}}{\pgfqpoint{-0.005824in}{-0.021960in}}{\pgfqpoint{0.000000in}{-0.021960in}}%
\pgfpathlineto{\pgfqpoint{0.000000in}{-0.021960in}}%
\pgfpathclose%
\pgfusepath{stroke,fill}%
}%
\begin{pgfscope}%
\pgfsys@transformshift{1.169029in}{4.508214in}%
\pgfsys@useobject{currentmarker}{}%
\end{pgfscope}%
\begin{pgfscope}%
\pgfsys@transformshift{1.348760in}{4.350030in}%
\pgfsys@useobject{currentmarker}{}%
\end{pgfscope}%
\begin{pgfscope}%
\pgfsys@transformshift{1.528491in}{4.111408in}%
\pgfsys@useobject{currentmarker}{}%
\end{pgfscope}%
\begin{pgfscope}%
\pgfsys@transformshift{1.708222in}{3.898727in}%
\pgfsys@useobject{currentmarker}{}%
\end{pgfscope}%
\begin{pgfscope}%
\pgfsys@transformshift{1.887953in}{3.717998in}%
\pgfsys@useobject{currentmarker}{}%
\end{pgfscope}%
\begin{pgfscope}%
\pgfsys@transformshift{2.067684in}{3.402746in}%
\pgfsys@useobject{currentmarker}{}%
\end{pgfscope}%
\begin{pgfscope}%
\pgfsys@transformshift{2.247415in}{3.179113in}%
\pgfsys@useobject{currentmarker}{}%
\end{pgfscope}%
\begin{pgfscope}%
\pgfsys@transformshift{2.427146in}{2.843213in}%
\pgfsys@useobject{currentmarker}{}%
\end{pgfscope}%
\begin{pgfscope}%
\pgfsys@transformshift{2.606877in}{2.673695in}%
\pgfsys@useobject{currentmarker}{}%
\end{pgfscope}%
\begin{pgfscope}%
\pgfsys@transformshift{2.786608in}{2.320542in}%
\pgfsys@useobject{currentmarker}{}%
\end{pgfscope}%
\begin{pgfscope}%
\pgfsys@transformshift{2.966339in}{1.944972in}%
\pgfsys@useobject{currentmarker}{}%
\end{pgfscope}%
\begin{pgfscope}%
\pgfsys@transformshift{3.146070in}{1.702614in}%
\pgfsys@useobject{currentmarker}{}%
\end{pgfscope}%
\begin{pgfscope}%
\pgfsys@transformshift{3.325801in}{1.103289in}%
\pgfsys@useobject{currentmarker}{}%
\end{pgfscope}%
\begin{pgfscope}%
\pgfsys@transformshift{3.505532in}{1.007478in}%
\pgfsys@useobject{currentmarker}{}%
\end{pgfscope}%
\begin{pgfscope}%
\pgfsys@transformshift{3.685263in}{0.803504in}%
\pgfsys@useobject{currentmarker}{}%
\end{pgfscope}%
\begin{pgfscope}%
\pgfsys@transformshift{3.864994in}{0.846261in}%
\pgfsys@useobject{currentmarker}{}%
\end{pgfscope}%
\begin{pgfscope}%
\pgfsys@transformshift{4.044725in}{0.763596in}%
\pgfsys@useobject{currentmarker}{}%
\end{pgfscope}%
\begin{pgfscope}%
\pgfsys@transformshift{4.224456in}{0.764336in}%
\pgfsys@useobject{currentmarker}{}%
\end{pgfscope}%
\begin{pgfscope}%
\pgfsys@transformshift{4.404187in}{0.759059in}%
\pgfsys@useobject{currentmarker}{}%
\end{pgfscope}%
\begin{pgfscope}%
\pgfsys@transformshift{4.583918in}{0.750605in}%
\pgfsys@useobject{currentmarker}{}%
\end{pgfscope}%
\begin{pgfscope}%
\pgfsys@transformshift{4.763648in}{0.748484in}%
\pgfsys@useobject{currentmarker}{}%
\end{pgfscope}%
\begin{pgfscope}%
\pgfsys@transformshift{4.943379in}{0.741053in}%
\pgfsys@useobject{currentmarker}{}%
\end{pgfscope}%
\begin{pgfscope}%
\pgfsys@transformshift{5.123110in}{0.734007in}%
\pgfsys@useobject{currentmarker}{}%
\end{pgfscope}%
\begin{pgfscope}%
\pgfsys@transformshift{5.302841in}{0.734862in}%
\pgfsys@useobject{currentmarker}{}%
\end{pgfscope}%
\begin{pgfscope}%
\pgfsys@transformshift{5.482572in}{0.729369in}%
\pgfsys@useobject{currentmarker}{}%
\end{pgfscope}%
\begin{pgfscope}%
\pgfsys@transformshift{5.662303in}{0.731867in}%
\pgfsys@useobject{currentmarker}{}%
\end{pgfscope}%
\begin{pgfscope}%
\pgfsys@transformshift{5.842034in}{0.729111in}%
\pgfsys@useobject{currentmarker}{}%
\end{pgfscope}%
\begin{pgfscope}%
\pgfsys@transformshift{6.021765in}{0.727804in}%
\pgfsys@useobject{currentmarker}{}%
\end{pgfscope}%
\begin{pgfscope}%
\pgfsys@transformshift{6.201496in}{0.726949in}%
\pgfsys@useobject{currentmarker}{}%
\end{pgfscope}%
\begin{pgfscope}%
\pgfsys@transformshift{6.381227in}{0.725582in}%
\pgfsys@useobject{currentmarker}{}%
\end{pgfscope}%
\begin{pgfscope}%
\pgfsys@transformshift{6.560958in}{0.724895in}%
\pgfsys@useobject{currentmarker}{}%
\end{pgfscope}%
\begin{pgfscope}%
\pgfsys@transformshift{6.740689in}{0.721590in}%
\pgfsys@useobject{currentmarker}{}%
\end{pgfscope}%
\begin{pgfscope}%
\pgfsys@transformshift{6.920420in}{0.722128in}%
\pgfsys@useobject{currentmarker}{}%
\end{pgfscope}%
\begin{pgfscope}%
\pgfsys@transformshift{7.100151in}{0.722307in}%
\pgfsys@useobject{currentmarker}{}%
\end{pgfscope}%
\begin{pgfscope}%
\pgfsys@transformshift{7.279882in}{0.720369in}%
\pgfsys@useobject{currentmarker}{}%
\end{pgfscope}%
\begin{pgfscope}%
\pgfsys@transformshift{7.459613in}{0.719708in}%
\pgfsys@useobject{currentmarker}{}%
\end{pgfscope}%
\begin{pgfscope}%
\pgfsys@transformshift{7.639344in}{0.719215in}%
\pgfsys@useobject{currentmarker}{}%
\end{pgfscope}%
\begin{pgfscope}%
\pgfsys@transformshift{7.819075in}{0.718719in}%
\pgfsys@useobject{currentmarker}{}%
\end{pgfscope}%
\begin{pgfscope}%
\pgfsys@transformshift{7.998806in}{0.718327in}%
\pgfsys@useobject{currentmarker}{}%
\end{pgfscope}%
\begin{pgfscope}%
\pgfsys@transformshift{8.178537in}{0.717572in}%
\pgfsys@useobject{currentmarker}{}%
\end{pgfscope}%
\begin{pgfscope}%
\pgfsys@transformshift{8.358268in}{0.717240in}%
\pgfsys@useobject{currentmarker}{}%
\end{pgfscope}%
\begin{pgfscope}%
\pgfsys@transformshift{8.537999in}{0.717341in}%
\pgfsys@useobject{currentmarker}{}%
\end{pgfscope}%
\begin{pgfscope}%
\pgfsys@transformshift{8.717730in}{0.716687in}%
\pgfsys@useobject{currentmarker}{}%
\end{pgfscope}%
\begin{pgfscope}%
\pgfsys@transformshift{8.897461in}{0.716187in}%
\pgfsys@useobject{currentmarker}{}%
\end{pgfscope}%
\begin{pgfscope}%
\pgfsys@transformshift{9.077192in}{0.715814in}%
\pgfsys@useobject{currentmarker}{}%
\end{pgfscope}%
\begin{pgfscope}%
\pgfsys@transformshift{9.256923in}{0.715227in}%
\pgfsys@useobject{currentmarker}{}%
\end{pgfscope}%
\begin{pgfscope}%
\pgfsys@transformshift{9.436654in}{0.715160in}%
\pgfsys@useobject{currentmarker}{}%
\end{pgfscope}%
\begin{pgfscope}%
\pgfsys@transformshift{9.616385in}{0.714772in}%
\pgfsys@useobject{currentmarker}{}%
\end{pgfscope}%
\begin{pgfscope}%
\pgfsys@transformshift{9.796116in}{0.714525in}%
\pgfsys@useobject{currentmarker}{}%
\end{pgfscope}%
\begin{pgfscope}%
\pgfsys@transformshift{9.975847in}{0.714305in}%
\pgfsys@useobject{currentmarker}{}%
\end{pgfscope}%
\end{pgfscope}%
\begin{pgfscope}%
\pgfpathrectangle{\pgfqpoint{0.728688in}{0.521603in}}{\pgfqpoint{9.687500in}{4.235000in}}%
\pgfusepath{clip}%
\pgfsetbuttcap%
\pgfsetroundjoin%
\definecolor{currentfill}{rgb}{0.000000,0.000000,0.000000}%
\pgfsetfillcolor{currentfill}%
\pgfsetfillopacity{0.500000}%
\pgfsetlinewidth{1.003750pt}%
\definecolor{currentstroke}{rgb}{0.000000,0.000000,0.000000}%
\pgfsetstrokecolor{currentstroke}%
\pgfsetstrokeopacity{0.500000}%
\pgfsetdash{}{0pt}%
\pgfsys@defobject{currentmarker}{\pgfqpoint{-0.021960in}{-0.021960in}}{\pgfqpoint{0.021960in}{0.021960in}}{%
\pgfpathmoveto{\pgfqpoint{0.000000in}{-0.021960in}}%
\pgfpathcurveto{\pgfqpoint{0.005824in}{-0.021960in}}{\pgfqpoint{0.011410in}{-0.019646in}}{\pgfqpoint{0.015528in}{-0.015528in}}%
\pgfpathcurveto{\pgfqpoint{0.019646in}{-0.011410in}}{\pgfqpoint{0.021960in}{-0.005824in}}{\pgfqpoint{0.021960in}{0.000000in}}%
\pgfpathcurveto{\pgfqpoint{0.021960in}{0.005824in}}{\pgfqpoint{0.019646in}{0.011410in}}{\pgfqpoint{0.015528in}{0.015528in}}%
\pgfpathcurveto{\pgfqpoint{0.011410in}{0.019646in}}{\pgfqpoint{0.005824in}{0.021960in}}{\pgfqpoint{0.000000in}{0.021960in}}%
\pgfpathcurveto{\pgfqpoint{-0.005824in}{0.021960in}}{\pgfqpoint{-0.011410in}{0.019646in}}{\pgfqpoint{-0.015528in}{0.015528in}}%
\pgfpathcurveto{\pgfqpoint{-0.019646in}{0.011410in}}{\pgfqpoint{-0.021960in}{0.005824in}}{\pgfqpoint{-0.021960in}{0.000000in}}%
\pgfpathcurveto{\pgfqpoint{-0.021960in}{-0.005824in}}{\pgfqpoint{-0.019646in}{-0.011410in}}{\pgfqpoint{-0.015528in}{-0.015528in}}%
\pgfpathcurveto{\pgfqpoint{-0.011410in}{-0.019646in}}{\pgfqpoint{-0.005824in}{-0.021960in}}{\pgfqpoint{0.000000in}{-0.021960in}}%
\pgfpathlineto{\pgfqpoint{0.000000in}{-0.021960in}}%
\pgfpathclose%
\pgfusepath{stroke,fill}%
}%
\begin{pgfscope}%
\pgfsys@transformshift{1.169029in}{4.564103in}%
\pgfsys@useobject{currentmarker}{}%
\end{pgfscope}%
\begin{pgfscope}%
\pgfsys@transformshift{1.348760in}{4.375359in}%
\pgfsys@useobject{currentmarker}{}%
\end{pgfscope}%
\begin{pgfscope}%
\pgfsys@transformshift{1.528491in}{4.185422in}%
\pgfsys@useobject{currentmarker}{}%
\end{pgfscope}%
\begin{pgfscope}%
\pgfsys@transformshift{1.708222in}{3.983019in}%
\pgfsys@useobject{currentmarker}{}%
\end{pgfscope}%
\begin{pgfscope}%
\pgfsys@transformshift{1.887953in}{3.760243in}%
\pgfsys@useobject{currentmarker}{}%
\end{pgfscope}%
\begin{pgfscope}%
\pgfsys@transformshift{2.067684in}{3.524466in}%
\pgfsys@useobject{currentmarker}{}%
\end{pgfscope}%
\begin{pgfscope}%
\pgfsys@transformshift{2.247415in}{3.349319in}%
\pgfsys@useobject{currentmarker}{}%
\end{pgfscope}%
\begin{pgfscope}%
\pgfsys@transformshift{2.427146in}{3.014918in}%
\pgfsys@useobject{currentmarker}{}%
\end{pgfscope}%
\begin{pgfscope}%
\pgfsys@transformshift{2.606877in}{2.783607in}%
\pgfsys@useobject{currentmarker}{}%
\end{pgfscope}%
\begin{pgfscope}%
\pgfsys@transformshift{2.786608in}{2.489402in}%
\pgfsys@useobject{currentmarker}{}%
\end{pgfscope}%
\begin{pgfscope}%
\pgfsys@transformshift{2.966339in}{2.053102in}%
\pgfsys@useobject{currentmarker}{}%
\end{pgfscope}%
\begin{pgfscope}%
\pgfsys@transformshift{3.146070in}{1.735807in}%
\pgfsys@useobject{currentmarker}{}%
\end{pgfscope}%
\begin{pgfscope}%
\pgfsys@transformshift{3.325801in}{1.436998in}%
\pgfsys@useobject{currentmarker}{}%
\end{pgfscope}%
\begin{pgfscope}%
\pgfsys@transformshift{3.505532in}{1.112781in}%
\pgfsys@useobject{currentmarker}{}%
\end{pgfscope}%
\begin{pgfscope}%
\pgfsys@transformshift{3.685263in}{0.883268in}%
\pgfsys@useobject{currentmarker}{}%
\end{pgfscope}%
\begin{pgfscope}%
\pgfsys@transformshift{3.864994in}{0.821761in}%
\pgfsys@useobject{currentmarker}{}%
\end{pgfscope}%
\begin{pgfscope}%
\pgfsys@transformshift{4.044725in}{0.778708in}%
\pgfsys@useobject{currentmarker}{}%
\end{pgfscope}%
\begin{pgfscope}%
\pgfsys@transformshift{4.224456in}{0.766316in}%
\pgfsys@useobject{currentmarker}{}%
\end{pgfscope}%
\begin{pgfscope}%
\pgfsys@transformshift{4.404187in}{0.756322in}%
\pgfsys@useobject{currentmarker}{}%
\end{pgfscope}%
\begin{pgfscope}%
\pgfsys@transformshift{4.583918in}{0.752370in}%
\pgfsys@useobject{currentmarker}{}%
\end{pgfscope}%
\begin{pgfscope}%
\pgfsys@transformshift{4.763648in}{0.747130in}%
\pgfsys@useobject{currentmarker}{}%
\end{pgfscope}%
\begin{pgfscope}%
\pgfsys@transformshift{4.943379in}{0.741988in}%
\pgfsys@useobject{currentmarker}{}%
\end{pgfscope}%
\begin{pgfscope}%
\pgfsys@transformshift{5.123110in}{0.736495in}%
\pgfsys@useobject{currentmarker}{}%
\end{pgfscope}%
\begin{pgfscope}%
\pgfsys@transformshift{5.302841in}{0.737631in}%
\pgfsys@useobject{currentmarker}{}%
\end{pgfscope}%
\begin{pgfscope}%
\pgfsys@transformshift{5.482572in}{0.731761in}%
\pgfsys@useobject{currentmarker}{}%
\end{pgfscope}%
\begin{pgfscope}%
\pgfsys@transformshift{5.662303in}{0.732099in}%
\pgfsys@useobject{currentmarker}{}%
\end{pgfscope}%
\begin{pgfscope}%
\pgfsys@transformshift{5.842034in}{0.728507in}%
\pgfsys@useobject{currentmarker}{}%
\end{pgfscope}%
\begin{pgfscope}%
\pgfsys@transformshift{6.021765in}{0.728153in}%
\pgfsys@useobject{currentmarker}{}%
\end{pgfscope}%
\begin{pgfscope}%
\pgfsys@transformshift{6.201496in}{0.726730in}%
\pgfsys@useobject{currentmarker}{}%
\end{pgfscope}%
\begin{pgfscope}%
\pgfsys@transformshift{6.381227in}{0.724779in}%
\pgfsys@useobject{currentmarker}{}%
\end{pgfscope}%
\begin{pgfscope}%
\pgfsys@transformshift{6.560958in}{0.723371in}%
\pgfsys@useobject{currentmarker}{}%
\end{pgfscope}%
\begin{pgfscope}%
\pgfsys@transformshift{6.740689in}{0.722704in}%
\pgfsys@useobject{currentmarker}{}%
\end{pgfscope}%
\begin{pgfscope}%
\pgfsys@transformshift{6.920420in}{0.721829in}%
\pgfsys@useobject{currentmarker}{}%
\end{pgfscope}%
\begin{pgfscope}%
\pgfsys@transformshift{7.100151in}{0.721662in}%
\pgfsys@useobject{currentmarker}{}%
\end{pgfscope}%
\begin{pgfscope}%
\pgfsys@transformshift{7.279882in}{0.720446in}%
\pgfsys@useobject{currentmarker}{}%
\end{pgfscope}%
\begin{pgfscope}%
\pgfsys@transformshift{7.459613in}{0.719978in}%
\pgfsys@useobject{currentmarker}{}%
\end{pgfscope}%
\begin{pgfscope}%
\pgfsys@transformshift{7.639344in}{0.719070in}%
\pgfsys@useobject{currentmarker}{}%
\end{pgfscope}%
\begin{pgfscope}%
\pgfsys@transformshift{7.819075in}{0.719093in}%
\pgfsys@useobject{currentmarker}{}%
\end{pgfscope}%
\begin{pgfscope}%
\pgfsys@transformshift{7.998806in}{0.718690in}%
\pgfsys@useobject{currentmarker}{}%
\end{pgfscope}%
\begin{pgfscope}%
\pgfsys@transformshift{8.178537in}{0.717470in}%
\pgfsys@useobject{currentmarker}{}%
\end{pgfscope}%
\begin{pgfscope}%
\pgfsys@transformshift{8.358268in}{0.717466in}%
\pgfsys@useobject{currentmarker}{}%
\end{pgfscope}%
\begin{pgfscope}%
\pgfsys@transformshift{8.537999in}{0.717053in}%
\pgfsys@useobject{currentmarker}{}%
\end{pgfscope}%
\begin{pgfscope}%
\pgfsys@transformshift{8.717730in}{0.716508in}%
\pgfsys@useobject{currentmarker}{}%
\end{pgfscope}%
\begin{pgfscope}%
\pgfsys@transformshift{8.897461in}{0.715971in}%
\pgfsys@useobject{currentmarker}{}%
\end{pgfscope}%
\begin{pgfscope}%
\pgfsys@transformshift{9.077192in}{0.715778in}%
\pgfsys@useobject{currentmarker}{}%
\end{pgfscope}%
\begin{pgfscope}%
\pgfsys@transformshift{9.256923in}{0.715462in}%
\pgfsys@useobject{currentmarker}{}%
\end{pgfscope}%
\begin{pgfscope}%
\pgfsys@transformshift{9.436654in}{0.715133in}%
\pgfsys@useobject{currentmarker}{}%
\end{pgfscope}%
\begin{pgfscope}%
\pgfsys@transformshift{9.616385in}{0.714836in}%
\pgfsys@useobject{currentmarker}{}%
\end{pgfscope}%
\begin{pgfscope}%
\pgfsys@transformshift{9.796116in}{0.714465in}%
\pgfsys@useobject{currentmarker}{}%
\end{pgfscope}%
\begin{pgfscope}%
\pgfsys@transformshift{9.975847in}{0.714263in}%
\pgfsys@useobject{currentmarker}{}%
\end{pgfscope}%
\end{pgfscope}%
\begin{pgfscope}%
\pgfpathrectangle{\pgfqpoint{0.728688in}{0.521603in}}{\pgfqpoint{9.687500in}{4.235000in}}%
\pgfusepath{clip}%
\pgfsetrectcap%
\pgfsetroundjoin%
\pgfsetlinewidth{0.803000pt}%
\definecolor{currentstroke}{rgb}{0.690196,0.690196,0.690196}%
\pgfsetstrokecolor{currentstroke}%
\pgfsetdash{}{0pt}%
\pgfpathmoveto{\pgfqpoint{1.169029in}{0.521603in}}%
\pgfpathlineto{\pgfqpoint{1.169029in}{4.756603in}}%
\pgfusepath{stroke}%
\end{pgfscope}%
\begin{pgfscope}%
\pgfsetbuttcap%
\pgfsetroundjoin%
\definecolor{currentfill}{rgb}{0.000000,0.000000,0.000000}%
\pgfsetfillcolor{currentfill}%
\pgfsetlinewidth{0.803000pt}%
\definecolor{currentstroke}{rgb}{0.000000,0.000000,0.000000}%
\pgfsetstrokecolor{currentstroke}%
\pgfsetdash{}{0pt}%
\pgfsys@defobject{currentmarker}{\pgfqpoint{0.000000in}{-0.048611in}}{\pgfqpoint{0.000000in}{0.000000in}}{%
\pgfpathmoveto{\pgfqpoint{0.000000in}{0.000000in}}%
\pgfpathlineto{\pgfqpoint{0.000000in}{-0.048611in}}%
\pgfusepath{stroke,fill}%
}%
\begin{pgfscope}%
\pgfsys@transformshift{1.169029in}{0.521603in}%
\pgfsys@useobject{currentmarker}{}%
\end{pgfscope}%
\end{pgfscope}%
\begin{pgfscope}%
\definecolor{textcolor}{rgb}{0.000000,0.000000,0.000000}%
\pgfsetstrokecolor{textcolor}%
\pgfsetfillcolor{textcolor}%
\pgftext[x=1.169029in,y=0.424381in,,top]{\color{textcolor}\sffamily\fontsize{10.000000}{12.000000}\selectfont 0.0}%
\end{pgfscope}%
\begin{pgfscope}%
\pgfpathrectangle{\pgfqpoint{0.728688in}{0.521603in}}{\pgfqpoint{9.687500in}{4.235000in}}%
\pgfusepath{clip}%
\pgfsetrectcap%
\pgfsetroundjoin%
\pgfsetlinewidth{0.803000pt}%
\definecolor{currentstroke}{rgb}{0.690196,0.690196,0.690196}%
\pgfsetstrokecolor{currentstroke}%
\pgfsetdash{}{0pt}%
\pgfpathmoveto{\pgfqpoint{2.966339in}{0.521603in}}%
\pgfpathlineto{\pgfqpoint{2.966339in}{4.756603in}}%
\pgfusepath{stroke}%
\end{pgfscope}%
\begin{pgfscope}%
\pgfsetbuttcap%
\pgfsetroundjoin%
\definecolor{currentfill}{rgb}{0.000000,0.000000,0.000000}%
\pgfsetfillcolor{currentfill}%
\pgfsetlinewidth{0.803000pt}%
\definecolor{currentstroke}{rgb}{0.000000,0.000000,0.000000}%
\pgfsetstrokecolor{currentstroke}%
\pgfsetdash{}{0pt}%
\pgfsys@defobject{currentmarker}{\pgfqpoint{0.000000in}{-0.048611in}}{\pgfqpoint{0.000000in}{0.000000in}}{%
\pgfpathmoveto{\pgfqpoint{0.000000in}{0.000000in}}%
\pgfpathlineto{\pgfqpoint{0.000000in}{-0.048611in}}%
\pgfusepath{stroke,fill}%
}%
\begin{pgfscope}%
\pgfsys@transformshift{2.966339in}{0.521603in}%
\pgfsys@useobject{currentmarker}{}%
\end{pgfscope}%
\end{pgfscope}%
\begin{pgfscope}%
\definecolor{textcolor}{rgb}{0.000000,0.000000,0.000000}%
\pgfsetstrokecolor{textcolor}%
\pgfsetfillcolor{textcolor}%
\pgftext[x=2.966339in,y=0.424381in,,top]{\color{textcolor}\sffamily\fontsize{10.000000}{12.000000}\selectfont 0.2}%
\end{pgfscope}%
\begin{pgfscope}%
\pgfpathrectangle{\pgfqpoint{0.728688in}{0.521603in}}{\pgfqpoint{9.687500in}{4.235000in}}%
\pgfusepath{clip}%
\pgfsetrectcap%
\pgfsetroundjoin%
\pgfsetlinewidth{0.803000pt}%
\definecolor{currentstroke}{rgb}{0.690196,0.690196,0.690196}%
\pgfsetstrokecolor{currentstroke}%
\pgfsetdash{}{0pt}%
\pgfpathmoveto{\pgfqpoint{4.763648in}{0.521603in}}%
\pgfpathlineto{\pgfqpoint{4.763648in}{4.756603in}}%
\pgfusepath{stroke}%
\end{pgfscope}%
\begin{pgfscope}%
\pgfsetbuttcap%
\pgfsetroundjoin%
\definecolor{currentfill}{rgb}{0.000000,0.000000,0.000000}%
\pgfsetfillcolor{currentfill}%
\pgfsetlinewidth{0.803000pt}%
\definecolor{currentstroke}{rgb}{0.000000,0.000000,0.000000}%
\pgfsetstrokecolor{currentstroke}%
\pgfsetdash{}{0pt}%
\pgfsys@defobject{currentmarker}{\pgfqpoint{0.000000in}{-0.048611in}}{\pgfqpoint{0.000000in}{0.000000in}}{%
\pgfpathmoveto{\pgfqpoint{0.000000in}{0.000000in}}%
\pgfpathlineto{\pgfqpoint{0.000000in}{-0.048611in}}%
\pgfusepath{stroke,fill}%
}%
\begin{pgfscope}%
\pgfsys@transformshift{4.763648in}{0.521603in}%
\pgfsys@useobject{currentmarker}{}%
\end{pgfscope}%
\end{pgfscope}%
\begin{pgfscope}%
\definecolor{textcolor}{rgb}{0.000000,0.000000,0.000000}%
\pgfsetstrokecolor{textcolor}%
\pgfsetfillcolor{textcolor}%
\pgftext[x=4.763648in,y=0.424381in,,top]{\color{textcolor}\sffamily\fontsize{10.000000}{12.000000}\selectfont 0.4}%
\end{pgfscope}%
\begin{pgfscope}%
\pgfpathrectangle{\pgfqpoint{0.728688in}{0.521603in}}{\pgfqpoint{9.687500in}{4.235000in}}%
\pgfusepath{clip}%
\pgfsetrectcap%
\pgfsetroundjoin%
\pgfsetlinewidth{0.803000pt}%
\definecolor{currentstroke}{rgb}{0.690196,0.690196,0.690196}%
\pgfsetstrokecolor{currentstroke}%
\pgfsetdash{}{0pt}%
\pgfpathmoveto{\pgfqpoint{6.560958in}{0.521603in}}%
\pgfpathlineto{\pgfqpoint{6.560958in}{4.756603in}}%
\pgfusepath{stroke}%
\end{pgfscope}%
\begin{pgfscope}%
\pgfsetbuttcap%
\pgfsetroundjoin%
\definecolor{currentfill}{rgb}{0.000000,0.000000,0.000000}%
\pgfsetfillcolor{currentfill}%
\pgfsetlinewidth{0.803000pt}%
\definecolor{currentstroke}{rgb}{0.000000,0.000000,0.000000}%
\pgfsetstrokecolor{currentstroke}%
\pgfsetdash{}{0pt}%
\pgfsys@defobject{currentmarker}{\pgfqpoint{0.000000in}{-0.048611in}}{\pgfqpoint{0.000000in}{0.000000in}}{%
\pgfpathmoveto{\pgfqpoint{0.000000in}{0.000000in}}%
\pgfpathlineto{\pgfqpoint{0.000000in}{-0.048611in}}%
\pgfusepath{stroke,fill}%
}%
\begin{pgfscope}%
\pgfsys@transformshift{6.560958in}{0.521603in}%
\pgfsys@useobject{currentmarker}{}%
\end{pgfscope}%
\end{pgfscope}%
\begin{pgfscope}%
\definecolor{textcolor}{rgb}{0.000000,0.000000,0.000000}%
\pgfsetstrokecolor{textcolor}%
\pgfsetfillcolor{textcolor}%
\pgftext[x=6.560958in,y=0.424381in,,top]{\color{textcolor}\sffamily\fontsize{10.000000}{12.000000}\selectfont 0.6}%
\end{pgfscope}%
\begin{pgfscope}%
\pgfpathrectangle{\pgfqpoint{0.728688in}{0.521603in}}{\pgfqpoint{9.687500in}{4.235000in}}%
\pgfusepath{clip}%
\pgfsetrectcap%
\pgfsetroundjoin%
\pgfsetlinewidth{0.803000pt}%
\definecolor{currentstroke}{rgb}{0.690196,0.690196,0.690196}%
\pgfsetstrokecolor{currentstroke}%
\pgfsetdash{}{0pt}%
\pgfpathmoveto{\pgfqpoint{8.358268in}{0.521603in}}%
\pgfpathlineto{\pgfqpoint{8.358268in}{4.756603in}}%
\pgfusepath{stroke}%
\end{pgfscope}%
\begin{pgfscope}%
\pgfsetbuttcap%
\pgfsetroundjoin%
\definecolor{currentfill}{rgb}{0.000000,0.000000,0.000000}%
\pgfsetfillcolor{currentfill}%
\pgfsetlinewidth{0.803000pt}%
\definecolor{currentstroke}{rgb}{0.000000,0.000000,0.000000}%
\pgfsetstrokecolor{currentstroke}%
\pgfsetdash{}{0pt}%
\pgfsys@defobject{currentmarker}{\pgfqpoint{0.000000in}{-0.048611in}}{\pgfqpoint{0.000000in}{0.000000in}}{%
\pgfpathmoveto{\pgfqpoint{0.000000in}{0.000000in}}%
\pgfpathlineto{\pgfqpoint{0.000000in}{-0.048611in}}%
\pgfusepath{stroke,fill}%
}%
\begin{pgfscope}%
\pgfsys@transformshift{8.358268in}{0.521603in}%
\pgfsys@useobject{currentmarker}{}%
\end{pgfscope}%
\end{pgfscope}%
\begin{pgfscope}%
\definecolor{textcolor}{rgb}{0.000000,0.000000,0.000000}%
\pgfsetstrokecolor{textcolor}%
\pgfsetfillcolor{textcolor}%
\pgftext[x=8.358268in,y=0.424381in,,top]{\color{textcolor}\sffamily\fontsize{10.000000}{12.000000}\selectfont 0.8}%
\end{pgfscope}%
\begin{pgfscope}%
\pgfpathrectangle{\pgfqpoint{0.728688in}{0.521603in}}{\pgfqpoint{9.687500in}{4.235000in}}%
\pgfusepath{clip}%
\pgfsetrectcap%
\pgfsetroundjoin%
\pgfsetlinewidth{0.803000pt}%
\definecolor{currentstroke}{rgb}{0.690196,0.690196,0.690196}%
\pgfsetstrokecolor{currentstroke}%
\pgfsetdash{}{0pt}%
\pgfpathmoveto{\pgfqpoint{10.155578in}{0.521603in}}%
\pgfpathlineto{\pgfqpoint{10.155578in}{4.756603in}}%
\pgfusepath{stroke}%
\end{pgfscope}%
\begin{pgfscope}%
\pgfsetbuttcap%
\pgfsetroundjoin%
\definecolor{currentfill}{rgb}{0.000000,0.000000,0.000000}%
\pgfsetfillcolor{currentfill}%
\pgfsetlinewidth{0.803000pt}%
\definecolor{currentstroke}{rgb}{0.000000,0.000000,0.000000}%
\pgfsetstrokecolor{currentstroke}%
\pgfsetdash{}{0pt}%
\pgfsys@defobject{currentmarker}{\pgfqpoint{0.000000in}{-0.048611in}}{\pgfqpoint{0.000000in}{0.000000in}}{%
\pgfpathmoveto{\pgfqpoint{0.000000in}{0.000000in}}%
\pgfpathlineto{\pgfqpoint{0.000000in}{-0.048611in}}%
\pgfusepath{stroke,fill}%
}%
\begin{pgfscope}%
\pgfsys@transformshift{10.155578in}{0.521603in}%
\pgfsys@useobject{currentmarker}{}%
\end{pgfscope}%
\end{pgfscope}%
\begin{pgfscope}%
\definecolor{textcolor}{rgb}{0.000000,0.000000,0.000000}%
\pgfsetstrokecolor{textcolor}%
\pgfsetfillcolor{textcolor}%
\pgftext[x=10.155578in,y=0.424381in,,top]{\color{textcolor}\sffamily\fontsize{10.000000}{12.000000}\selectfont 1.0}%
\end{pgfscope}%
\begin{pgfscope}%
\pgfpathrectangle{\pgfqpoint{0.728688in}{0.521603in}}{\pgfqpoint{9.687500in}{4.235000in}}%
\pgfusepath{clip}%
\pgfsetrectcap%
\pgfsetroundjoin%
\pgfsetlinewidth{0.803000pt}%
\definecolor{currentstroke}{rgb}{0.600000,0.600000,0.600000}%
\pgfsetstrokecolor{currentstroke}%
\pgfsetstrokeopacity{0.200000}%
\pgfsetdash{}{0pt}%
\pgfpathmoveto{\pgfqpoint{1.618356in}{0.521603in}}%
\pgfpathlineto{\pgfqpoint{1.618356in}{4.756603in}}%
\pgfusepath{stroke}%
\end{pgfscope}%
\begin{pgfscope}%
\pgfsetbuttcap%
\pgfsetroundjoin%
\definecolor{currentfill}{rgb}{0.000000,0.000000,0.000000}%
\pgfsetfillcolor{currentfill}%
\pgfsetlinewidth{0.602250pt}%
\definecolor{currentstroke}{rgb}{0.000000,0.000000,0.000000}%
\pgfsetstrokecolor{currentstroke}%
\pgfsetdash{}{0pt}%
\pgfsys@defobject{currentmarker}{\pgfqpoint{0.000000in}{-0.027778in}}{\pgfqpoint{0.000000in}{0.000000in}}{%
\pgfpathmoveto{\pgfqpoint{0.000000in}{0.000000in}}%
\pgfpathlineto{\pgfqpoint{0.000000in}{-0.027778in}}%
\pgfusepath{stroke,fill}%
}%
\begin{pgfscope}%
\pgfsys@transformshift{1.618356in}{0.521603in}%
\pgfsys@useobject{currentmarker}{}%
\end{pgfscope}%
\end{pgfscope}%
\begin{pgfscope}%
\pgfpathrectangle{\pgfqpoint{0.728688in}{0.521603in}}{\pgfqpoint{9.687500in}{4.235000in}}%
\pgfusepath{clip}%
\pgfsetrectcap%
\pgfsetroundjoin%
\pgfsetlinewidth{0.803000pt}%
\definecolor{currentstroke}{rgb}{0.600000,0.600000,0.600000}%
\pgfsetstrokecolor{currentstroke}%
\pgfsetstrokeopacity{0.200000}%
\pgfsetdash{}{0pt}%
\pgfpathmoveto{\pgfqpoint{2.067684in}{0.521603in}}%
\pgfpathlineto{\pgfqpoint{2.067684in}{4.756603in}}%
\pgfusepath{stroke}%
\end{pgfscope}%
\begin{pgfscope}%
\pgfsetbuttcap%
\pgfsetroundjoin%
\definecolor{currentfill}{rgb}{0.000000,0.000000,0.000000}%
\pgfsetfillcolor{currentfill}%
\pgfsetlinewidth{0.602250pt}%
\definecolor{currentstroke}{rgb}{0.000000,0.000000,0.000000}%
\pgfsetstrokecolor{currentstroke}%
\pgfsetdash{}{0pt}%
\pgfsys@defobject{currentmarker}{\pgfqpoint{0.000000in}{-0.027778in}}{\pgfqpoint{0.000000in}{0.000000in}}{%
\pgfpathmoveto{\pgfqpoint{0.000000in}{0.000000in}}%
\pgfpathlineto{\pgfqpoint{0.000000in}{-0.027778in}}%
\pgfusepath{stroke,fill}%
}%
\begin{pgfscope}%
\pgfsys@transformshift{2.067684in}{0.521603in}%
\pgfsys@useobject{currentmarker}{}%
\end{pgfscope}%
\end{pgfscope}%
\begin{pgfscope}%
\pgfpathrectangle{\pgfqpoint{0.728688in}{0.521603in}}{\pgfqpoint{9.687500in}{4.235000in}}%
\pgfusepath{clip}%
\pgfsetrectcap%
\pgfsetroundjoin%
\pgfsetlinewidth{0.803000pt}%
\definecolor{currentstroke}{rgb}{0.600000,0.600000,0.600000}%
\pgfsetstrokecolor{currentstroke}%
\pgfsetstrokeopacity{0.200000}%
\pgfsetdash{}{0pt}%
\pgfpathmoveto{\pgfqpoint{2.517011in}{0.521603in}}%
\pgfpathlineto{\pgfqpoint{2.517011in}{4.756603in}}%
\pgfusepath{stroke}%
\end{pgfscope}%
\begin{pgfscope}%
\pgfsetbuttcap%
\pgfsetroundjoin%
\definecolor{currentfill}{rgb}{0.000000,0.000000,0.000000}%
\pgfsetfillcolor{currentfill}%
\pgfsetlinewidth{0.602250pt}%
\definecolor{currentstroke}{rgb}{0.000000,0.000000,0.000000}%
\pgfsetstrokecolor{currentstroke}%
\pgfsetdash{}{0pt}%
\pgfsys@defobject{currentmarker}{\pgfqpoint{0.000000in}{-0.027778in}}{\pgfqpoint{0.000000in}{0.000000in}}{%
\pgfpathmoveto{\pgfqpoint{0.000000in}{0.000000in}}%
\pgfpathlineto{\pgfqpoint{0.000000in}{-0.027778in}}%
\pgfusepath{stroke,fill}%
}%
\begin{pgfscope}%
\pgfsys@transformshift{2.517011in}{0.521603in}%
\pgfsys@useobject{currentmarker}{}%
\end{pgfscope}%
\end{pgfscope}%
\begin{pgfscope}%
\pgfpathrectangle{\pgfqpoint{0.728688in}{0.521603in}}{\pgfqpoint{9.687500in}{4.235000in}}%
\pgfusepath{clip}%
\pgfsetrectcap%
\pgfsetroundjoin%
\pgfsetlinewidth{0.803000pt}%
\definecolor{currentstroke}{rgb}{0.600000,0.600000,0.600000}%
\pgfsetstrokecolor{currentstroke}%
\pgfsetstrokeopacity{0.200000}%
\pgfsetdash{}{0pt}%
\pgfpathmoveto{\pgfqpoint{3.415666in}{0.521603in}}%
\pgfpathlineto{\pgfqpoint{3.415666in}{4.756603in}}%
\pgfusepath{stroke}%
\end{pgfscope}%
\begin{pgfscope}%
\pgfsetbuttcap%
\pgfsetroundjoin%
\definecolor{currentfill}{rgb}{0.000000,0.000000,0.000000}%
\pgfsetfillcolor{currentfill}%
\pgfsetlinewidth{0.602250pt}%
\definecolor{currentstroke}{rgb}{0.000000,0.000000,0.000000}%
\pgfsetstrokecolor{currentstroke}%
\pgfsetdash{}{0pt}%
\pgfsys@defobject{currentmarker}{\pgfqpoint{0.000000in}{-0.027778in}}{\pgfqpoint{0.000000in}{0.000000in}}{%
\pgfpathmoveto{\pgfqpoint{0.000000in}{0.000000in}}%
\pgfpathlineto{\pgfqpoint{0.000000in}{-0.027778in}}%
\pgfusepath{stroke,fill}%
}%
\begin{pgfscope}%
\pgfsys@transformshift{3.415666in}{0.521603in}%
\pgfsys@useobject{currentmarker}{}%
\end{pgfscope}%
\end{pgfscope}%
\begin{pgfscope}%
\pgfpathrectangle{\pgfqpoint{0.728688in}{0.521603in}}{\pgfqpoint{9.687500in}{4.235000in}}%
\pgfusepath{clip}%
\pgfsetrectcap%
\pgfsetroundjoin%
\pgfsetlinewidth{0.803000pt}%
\definecolor{currentstroke}{rgb}{0.600000,0.600000,0.600000}%
\pgfsetstrokecolor{currentstroke}%
\pgfsetstrokeopacity{0.200000}%
\pgfsetdash{}{0pt}%
\pgfpathmoveto{\pgfqpoint{3.864994in}{0.521603in}}%
\pgfpathlineto{\pgfqpoint{3.864994in}{4.756603in}}%
\pgfusepath{stroke}%
\end{pgfscope}%
\begin{pgfscope}%
\pgfsetbuttcap%
\pgfsetroundjoin%
\definecolor{currentfill}{rgb}{0.000000,0.000000,0.000000}%
\pgfsetfillcolor{currentfill}%
\pgfsetlinewidth{0.602250pt}%
\definecolor{currentstroke}{rgb}{0.000000,0.000000,0.000000}%
\pgfsetstrokecolor{currentstroke}%
\pgfsetdash{}{0pt}%
\pgfsys@defobject{currentmarker}{\pgfqpoint{0.000000in}{-0.027778in}}{\pgfqpoint{0.000000in}{0.000000in}}{%
\pgfpathmoveto{\pgfqpoint{0.000000in}{0.000000in}}%
\pgfpathlineto{\pgfqpoint{0.000000in}{-0.027778in}}%
\pgfusepath{stroke,fill}%
}%
\begin{pgfscope}%
\pgfsys@transformshift{3.864994in}{0.521603in}%
\pgfsys@useobject{currentmarker}{}%
\end{pgfscope}%
\end{pgfscope}%
\begin{pgfscope}%
\pgfpathrectangle{\pgfqpoint{0.728688in}{0.521603in}}{\pgfqpoint{9.687500in}{4.235000in}}%
\pgfusepath{clip}%
\pgfsetrectcap%
\pgfsetroundjoin%
\pgfsetlinewidth{0.803000pt}%
\definecolor{currentstroke}{rgb}{0.600000,0.600000,0.600000}%
\pgfsetstrokecolor{currentstroke}%
\pgfsetstrokeopacity{0.200000}%
\pgfsetdash{}{0pt}%
\pgfpathmoveto{\pgfqpoint{4.314321in}{0.521603in}}%
\pgfpathlineto{\pgfqpoint{4.314321in}{4.756603in}}%
\pgfusepath{stroke}%
\end{pgfscope}%
\begin{pgfscope}%
\pgfsetbuttcap%
\pgfsetroundjoin%
\definecolor{currentfill}{rgb}{0.000000,0.000000,0.000000}%
\pgfsetfillcolor{currentfill}%
\pgfsetlinewidth{0.602250pt}%
\definecolor{currentstroke}{rgb}{0.000000,0.000000,0.000000}%
\pgfsetstrokecolor{currentstroke}%
\pgfsetdash{}{0pt}%
\pgfsys@defobject{currentmarker}{\pgfqpoint{0.000000in}{-0.027778in}}{\pgfqpoint{0.000000in}{0.000000in}}{%
\pgfpathmoveto{\pgfqpoint{0.000000in}{0.000000in}}%
\pgfpathlineto{\pgfqpoint{0.000000in}{-0.027778in}}%
\pgfusepath{stroke,fill}%
}%
\begin{pgfscope}%
\pgfsys@transformshift{4.314321in}{0.521603in}%
\pgfsys@useobject{currentmarker}{}%
\end{pgfscope}%
\end{pgfscope}%
\begin{pgfscope}%
\pgfpathrectangle{\pgfqpoint{0.728688in}{0.521603in}}{\pgfqpoint{9.687500in}{4.235000in}}%
\pgfusepath{clip}%
\pgfsetrectcap%
\pgfsetroundjoin%
\pgfsetlinewidth{0.803000pt}%
\definecolor{currentstroke}{rgb}{0.600000,0.600000,0.600000}%
\pgfsetstrokecolor{currentstroke}%
\pgfsetstrokeopacity{0.200000}%
\pgfsetdash{}{0pt}%
\pgfpathmoveto{\pgfqpoint{5.212976in}{0.521603in}}%
\pgfpathlineto{\pgfqpoint{5.212976in}{4.756603in}}%
\pgfusepath{stroke}%
\end{pgfscope}%
\begin{pgfscope}%
\pgfsetbuttcap%
\pgfsetroundjoin%
\definecolor{currentfill}{rgb}{0.000000,0.000000,0.000000}%
\pgfsetfillcolor{currentfill}%
\pgfsetlinewidth{0.602250pt}%
\definecolor{currentstroke}{rgb}{0.000000,0.000000,0.000000}%
\pgfsetstrokecolor{currentstroke}%
\pgfsetdash{}{0pt}%
\pgfsys@defobject{currentmarker}{\pgfqpoint{0.000000in}{-0.027778in}}{\pgfqpoint{0.000000in}{0.000000in}}{%
\pgfpathmoveto{\pgfqpoint{0.000000in}{0.000000in}}%
\pgfpathlineto{\pgfqpoint{0.000000in}{-0.027778in}}%
\pgfusepath{stroke,fill}%
}%
\begin{pgfscope}%
\pgfsys@transformshift{5.212976in}{0.521603in}%
\pgfsys@useobject{currentmarker}{}%
\end{pgfscope}%
\end{pgfscope}%
\begin{pgfscope}%
\pgfpathrectangle{\pgfqpoint{0.728688in}{0.521603in}}{\pgfqpoint{9.687500in}{4.235000in}}%
\pgfusepath{clip}%
\pgfsetrectcap%
\pgfsetroundjoin%
\pgfsetlinewidth{0.803000pt}%
\definecolor{currentstroke}{rgb}{0.600000,0.600000,0.600000}%
\pgfsetstrokecolor{currentstroke}%
\pgfsetstrokeopacity{0.200000}%
\pgfsetdash{}{0pt}%
\pgfpathmoveto{\pgfqpoint{5.662303in}{0.521603in}}%
\pgfpathlineto{\pgfqpoint{5.662303in}{4.756603in}}%
\pgfusepath{stroke}%
\end{pgfscope}%
\begin{pgfscope}%
\pgfsetbuttcap%
\pgfsetroundjoin%
\definecolor{currentfill}{rgb}{0.000000,0.000000,0.000000}%
\pgfsetfillcolor{currentfill}%
\pgfsetlinewidth{0.602250pt}%
\definecolor{currentstroke}{rgb}{0.000000,0.000000,0.000000}%
\pgfsetstrokecolor{currentstroke}%
\pgfsetdash{}{0pt}%
\pgfsys@defobject{currentmarker}{\pgfqpoint{0.000000in}{-0.027778in}}{\pgfqpoint{0.000000in}{0.000000in}}{%
\pgfpathmoveto{\pgfqpoint{0.000000in}{0.000000in}}%
\pgfpathlineto{\pgfqpoint{0.000000in}{-0.027778in}}%
\pgfusepath{stroke,fill}%
}%
\begin{pgfscope}%
\pgfsys@transformshift{5.662303in}{0.521603in}%
\pgfsys@useobject{currentmarker}{}%
\end{pgfscope}%
\end{pgfscope}%
\begin{pgfscope}%
\pgfpathrectangle{\pgfqpoint{0.728688in}{0.521603in}}{\pgfqpoint{9.687500in}{4.235000in}}%
\pgfusepath{clip}%
\pgfsetrectcap%
\pgfsetroundjoin%
\pgfsetlinewidth{0.803000pt}%
\definecolor{currentstroke}{rgb}{0.600000,0.600000,0.600000}%
\pgfsetstrokecolor{currentstroke}%
\pgfsetstrokeopacity{0.200000}%
\pgfsetdash{}{0pt}%
\pgfpathmoveto{\pgfqpoint{6.111631in}{0.521603in}}%
\pgfpathlineto{\pgfqpoint{6.111631in}{4.756603in}}%
\pgfusepath{stroke}%
\end{pgfscope}%
\begin{pgfscope}%
\pgfsetbuttcap%
\pgfsetroundjoin%
\definecolor{currentfill}{rgb}{0.000000,0.000000,0.000000}%
\pgfsetfillcolor{currentfill}%
\pgfsetlinewidth{0.602250pt}%
\definecolor{currentstroke}{rgb}{0.000000,0.000000,0.000000}%
\pgfsetstrokecolor{currentstroke}%
\pgfsetdash{}{0pt}%
\pgfsys@defobject{currentmarker}{\pgfqpoint{0.000000in}{-0.027778in}}{\pgfqpoint{0.000000in}{0.000000in}}{%
\pgfpathmoveto{\pgfqpoint{0.000000in}{0.000000in}}%
\pgfpathlineto{\pgfqpoint{0.000000in}{-0.027778in}}%
\pgfusepath{stroke,fill}%
}%
\begin{pgfscope}%
\pgfsys@transformshift{6.111631in}{0.521603in}%
\pgfsys@useobject{currentmarker}{}%
\end{pgfscope}%
\end{pgfscope}%
\begin{pgfscope}%
\pgfpathrectangle{\pgfqpoint{0.728688in}{0.521603in}}{\pgfqpoint{9.687500in}{4.235000in}}%
\pgfusepath{clip}%
\pgfsetrectcap%
\pgfsetroundjoin%
\pgfsetlinewidth{0.803000pt}%
\definecolor{currentstroke}{rgb}{0.600000,0.600000,0.600000}%
\pgfsetstrokecolor{currentstroke}%
\pgfsetstrokeopacity{0.200000}%
\pgfsetdash{}{0pt}%
\pgfpathmoveto{\pgfqpoint{7.010286in}{0.521603in}}%
\pgfpathlineto{\pgfqpoint{7.010286in}{4.756603in}}%
\pgfusepath{stroke}%
\end{pgfscope}%
\begin{pgfscope}%
\pgfsetbuttcap%
\pgfsetroundjoin%
\definecolor{currentfill}{rgb}{0.000000,0.000000,0.000000}%
\pgfsetfillcolor{currentfill}%
\pgfsetlinewidth{0.602250pt}%
\definecolor{currentstroke}{rgb}{0.000000,0.000000,0.000000}%
\pgfsetstrokecolor{currentstroke}%
\pgfsetdash{}{0pt}%
\pgfsys@defobject{currentmarker}{\pgfqpoint{0.000000in}{-0.027778in}}{\pgfqpoint{0.000000in}{0.000000in}}{%
\pgfpathmoveto{\pgfqpoint{0.000000in}{0.000000in}}%
\pgfpathlineto{\pgfqpoint{0.000000in}{-0.027778in}}%
\pgfusepath{stroke,fill}%
}%
\begin{pgfscope}%
\pgfsys@transformshift{7.010286in}{0.521603in}%
\pgfsys@useobject{currentmarker}{}%
\end{pgfscope}%
\end{pgfscope}%
\begin{pgfscope}%
\pgfpathrectangle{\pgfqpoint{0.728688in}{0.521603in}}{\pgfqpoint{9.687500in}{4.235000in}}%
\pgfusepath{clip}%
\pgfsetrectcap%
\pgfsetroundjoin%
\pgfsetlinewidth{0.803000pt}%
\definecolor{currentstroke}{rgb}{0.600000,0.600000,0.600000}%
\pgfsetstrokecolor{currentstroke}%
\pgfsetstrokeopacity{0.200000}%
\pgfsetdash{}{0pt}%
\pgfpathmoveto{\pgfqpoint{7.459613in}{0.521603in}}%
\pgfpathlineto{\pgfqpoint{7.459613in}{4.756603in}}%
\pgfusepath{stroke}%
\end{pgfscope}%
\begin{pgfscope}%
\pgfsetbuttcap%
\pgfsetroundjoin%
\definecolor{currentfill}{rgb}{0.000000,0.000000,0.000000}%
\pgfsetfillcolor{currentfill}%
\pgfsetlinewidth{0.602250pt}%
\definecolor{currentstroke}{rgb}{0.000000,0.000000,0.000000}%
\pgfsetstrokecolor{currentstroke}%
\pgfsetdash{}{0pt}%
\pgfsys@defobject{currentmarker}{\pgfqpoint{0.000000in}{-0.027778in}}{\pgfqpoint{0.000000in}{0.000000in}}{%
\pgfpathmoveto{\pgfqpoint{0.000000in}{0.000000in}}%
\pgfpathlineto{\pgfqpoint{0.000000in}{-0.027778in}}%
\pgfusepath{stroke,fill}%
}%
\begin{pgfscope}%
\pgfsys@transformshift{7.459613in}{0.521603in}%
\pgfsys@useobject{currentmarker}{}%
\end{pgfscope}%
\end{pgfscope}%
\begin{pgfscope}%
\pgfpathrectangle{\pgfqpoint{0.728688in}{0.521603in}}{\pgfqpoint{9.687500in}{4.235000in}}%
\pgfusepath{clip}%
\pgfsetrectcap%
\pgfsetroundjoin%
\pgfsetlinewidth{0.803000pt}%
\definecolor{currentstroke}{rgb}{0.600000,0.600000,0.600000}%
\pgfsetstrokecolor{currentstroke}%
\pgfsetstrokeopacity{0.200000}%
\pgfsetdash{}{0pt}%
\pgfpathmoveto{\pgfqpoint{7.908941in}{0.521603in}}%
\pgfpathlineto{\pgfqpoint{7.908941in}{4.756603in}}%
\pgfusepath{stroke}%
\end{pgfscope}%
\begin{pgfscope}%
\pgfsetbuttcap%
\pgfsetroundjoin%
\definecolor{currentfill}{rgb}{0.000000,0.000000,0.000000}%
\pgfsetfillcolor{currentfill}%
\pgfsetlinewidth{0.602250pt}%
\definecolor{currentstroke}{rgb}{0.000000,0.000000,0.000000}%
\pgfsetstrokecolor{currentstroke}%
\pgfsetdash{}{0pt}%
\pgfsys@defobject{currentmarker}{\pgfqpoint{0.000000in}{-0.027778in}}{\pgfqpoint{0.000000in}{0.000000in}}{%
\pgfpathmoveto{\pgfqpoint{0.000000in}{0.000000in}}%
\pgfpathlineto{\pgfqpoint{0.000000in}{-0.027778in}}%
\pgfusepath{stroke,fill}%
}%
\begin{pgfscope}%
\pgfsys@transformshift{7.908941in}{0.521603in}%
\pgfsys@useobject{currentmarker}{}%
\end{pgfscope}%
\end{pgfscope}%
\begin{pgfscope}%
\pgfpathrectangle{\pgfqpoint{0.728688in}{0.521603in}}{\pgfqpoint{9.687500in}{4.235000in}}%
\pgfusepath{clip}%
\pgfsetrectcap%
\pgfsetroundjoin%
\pgfsetlinewidth{0.803000pt}%
\definecolor{currentstroke}{rgb}{0.600000,0.600000,0.600000}%
\pgfsetstrokecolor{currentstroke}%
\pgfsetstrokeopacity{0.200000}%
\pgfsetdash{}{0pt}%
\pgfpathmoveto{\pgfqpoint{8.807596in}{0.521603in}}%
\pgfpathlineto{\pgfqpoint{8.807596in}{4.756603in}}%
\pgfusepath{stroke}%
\end{pgfscope}%
\begin{pgfscope}%
\pgfsetbuttcap%
\pgfsetroundjoin%
\definecolor{currentfill}{rgb}{0.000000,0.000000,0.000000}%
\pgfsetfillcolor{currentfill}%
\pgfsetlinewidth{0.602250pt}%
\definecolor{currentstroke}{rgb}{0.000000,0.000000,0.000000}%
\pgfsetstrokecolor{currentstroke}%
\pgfsetdash{}{0pt}%
\pgfsys@defobject{currentmarker}{\pgfqpoint{0.000000in}{-0.027778in}}{\pgfqpoint{0.000000in}{0.000000in}}{%
\pgfpathmoveto{\pgfqpoint{0.000000in}{0.000000in}}%
\pgfpathlineto{\pgfqpoint{0.000000in}{-0.027778in}}%
\pgfusepath{stroke,fill}%
}%
\begin{pgfscope}%
\pgfsys@transformshift{8.807596in}{0.521603in}%
\pgfsys@useobject{currentmarker}{}%
\end{pgfscope}%
\end{pgfscope}%
\begin{pgfscope}%
\pgfpathrectangle{\pgfqpoint{0.728688in}{0.521603in}}{\pgfqpoint{9.687500in}{4.235000in}}%
\pgfusepath{clip}%
\pgfsetrectcap%
\pgfsetroundjoin%
\pgfsetlinewidth{0.803000pt}%
\definecolor{currentstroke}{rgb}{0.600000,0.600000,0.600000}%
\pgfsetstrokecolor{currentstroke}%
\pgfsetstrokeopacity{0.200000}%
\pgfsetdash{}{0pt}%
\pgfpathmoveto{\pgfqpoint{9.256923in}{0.521603in}}%
\pgfpathlineto{\pgfqpoint{9.256923in}{4.756603in}}%
\pgfusepath{stroke}%
\end{pgfscope}%
\begin{pgfscope}%
\pgfsetbuttcap%
\pgfsetroundjoin%
\definecolor{currentfill}{rgb}{0.000000,0.000000,0.000000}%
\pgfsetfillcolor{currentfill}%
\pgfsetlinewidth{0.602250pt}%
\definecolor{currentstroke}{rgb}{0.000000,0.000000,0.000000}%
\pgfsetstrokecolor{currentstroke}%
\pgfsetdash{}{0pt}%
\pgfsys@defobject{currentmarker}{\pgfqpoint{0.000000in}{-0.027778in}}{\pgfqpoint{0.000000in}{0.000000in}}{%
\pgfpathmoveto{\pgfqpoint{0.000000in}{0.000000in}}%
\pgfpathlineto{\pgfqpoint{0.000000in}{-0.027778in}}%
\pgfusepath{stroke,fill}%
}%
\begin{pgfscope}%
\pgfsys@transformshift{9.256923in}{0.521603in}%
\pgfsys@useobject{currentmarker}{}%
\end{pgfscope}%
\end{pgfscope}%
\begin{pgfscope}%
\pgfpathrectangle{\pgfqpoint{0.728688in}{0.521603in}}{\pgfqpoint{9.687500in}{4.235000in}}%
\pgfusepath{clip}%
\pgfsetrectcap%
\pgfsetroundjoin%
\pgfsetlinewidth{0.803000pt}%
\definecolor{currentstroke}{rgb}{0.600000,0.600000,0.600000}%
\pgfsetstrokecolor{currentstroke}%
\pgfsetstrokeopacity{0.200000}%
\pgfsetdash{}{0pt}%
\pgfpathmoveto{\pgfqpoint{9.706251in}{0.521603in}}%
\pgfpathlineto{\pgfqpoint{9.706251in}{4.756603in}}%
\pgfusepath{stroke}%
\end{pgfscope}%
\begin{pgfscope}%
\pgfsetbuttcap%
\pgfsetroundjoin%
\definecolor{currentfill}{rgb}{0.000000,0.000000,0.000000}%
\pgfsetfillcolor{currentfill}%
\pgfsetlinewidth{0.602250pt}%
\definecolor{currentstroke}{rgb}{0.000000,0.000000,0.000000}%
\pgfsetstrokecolor{currentstroke}%
\pgfsetdash{}{0pt}%
\pgfsys@defobject{currentmarker}{\pgfqpoint{0.000000in}{-0.027778in}}{\pgfqpoint{0.000000in}{0.000000in}}{%
\pgfpathmoveto{\pgfqpoint{0.000000in}{0.000000in}}%
\pgfpathlineto{\pgfqpoint{0.000000in}{-0.027778in}}%
\pgfusepath{stroke,fill}%
}%
\begin{pgfscope}%
\pgfsys@transformshift{9.706251in}{0.521603in}%
\pgfsys@useobject{currentmarker}{}%
\end{pgfscope}%
\end{pgfscope}%
\begin{pgfscope}%
\definecolor{textcolor}{rgb}{0.000000,0.000000,0.000000}%
\pgfsetstrokecolor{textcolor}%
\pgfsetfillcolor{textcolor}%
\pgftext[x=5.572438in,y=0.234413in,,top]{\color{textcolor}\sffamily\fontsize{10.000000}{12.000000}\selectfont probability \(\displaystyle p_4\) of a vaccinated person \(\displaystyle V\) occupying a grid node}%
\end{pgfscope}%
\begin{pgfscope}%
\pgfpathrectangle{\pgfqpoint{0.728688in}{0.521603in}}{\pgfqpoint{9.687500in}{4.235000in}}%
\pgfusepath{clip}%
\pgfsetrectcap%
\pgfsetroundjoin%
\pgfsetlinewidth{0.803000pt}%
\definecolor{currentstroke}{rgb}{0.690196,0.690196,0.690196}%
\pgfsetstrokecolor{currentstroke}%
\pgfsetdash{}{0pt}%
\pgfpathmoveto{\pgfqpoint{0.728688in}{0.713984in}}%
\pgfpathlineto{\pgfqpoint{10.416188in}{0.713984in}}%
\pgfusepath{stroke}%
\end{pgfscope}%
\begin{pgfscope}%
\pgfsetbuttcap%
\pgfsetroundjoin%
\definecolor{currentfill}{rgb}{0.000000,0.000000,0.000000}%
\pgfsetfillcolor{currentfill}%
\pgfsetlinewidth{0.803000pt}%
\definecolor{currentstroke}{rgb}{0.000000,0.000000,0.000000}%
\pgfsetstrokecolor{currentstroke}%
\pgfsetdash{}{0pt}%
\pgfsys@defobject{currentmarker}{\pgfqpoint{-0.048611in}{0.000000in}}{\pgfqpoint{-0.000000in}{0.000000in}}{%
\pgfpathmoveto{\pgfqpoint{-0.000000in}{0.000000in}}%
\pgfpathlineto{\pgfqpoint{-0.048611in}{0.000000in}}%
\pgfusepath{stroke,fill}%
}%
\begin{pgfscope}%
\pgfsys@transformshift{0.728688in}{0.713984in}%
\pgfsys@useobject{currentmarker}{}%
\end{pgfscope}%
\end{pgfscope}%
\begin{pgfscope}%
\definecolor{textcolor}{rgb}{0.000000,0.000000,0.000000}%
\pgfsetstrokecolor{textcolor}%
\pgfsetfillcolor{textcolor}%
\pgftext[x=0.322221in, y=0.661222in, left, base]{\color{textcolor}\sffamily\fontsize{10.000000}{12.000000}\selectfont 0.00}%
\end{pgfscope}%
\begin{pgfscope}%
\pgfpathrectangle{\pgfqpoint{0.728688in}{0.521603in}}{\pgfqpoint{9.687500in}{4.235000in}}%
\pgfusepath{clip}%
\pgfsetrectcap%
\pgfsetroundjoin%
\pgfsetlinewidth{0.803000pt}%
\definecolor{currentstroke}{rgb}{0.690196,0.690196,0.690196}%
\pgfsetstrokecolor{currentstroke}%
\pgfsetdash{}{0pt}%
\pgfpathmoveto{\pgfqpoint{0.728688in}{1.478749in}}%
\pgfpathlineto{\pgfqpoint{10.416188in}{1.478749in}}%
\pgfusepath{stroke}%
\end{pgfscope}%
\begin{pgfscope}%
\pgfsetbuttcap%
\pgfsetroundjoin%
\definecolor{currentfill}{rgb}{0.000000,0.000000,0.000000}%
\pgfsetfillcolor{currentfill}%
\pgfsetlinewidth{0.803000pt}%
\definecolor{currentstroke}{rgb}{0.000000,0.000000,0.000000}%
\pgfsetstrokecolor{currentstroke}%
\pgfsetdash{}{0pt}%
\pgfsys@defobject{currentmarker}{\pgfqpoint{-0.048611in}{0.000000in}}{\pgfqpoint{-0.000000in}{0.000000in}}{%
\pgfpathmoveto{\pgfqpoint{-0.000000in}{0.000000in}}%
\pgfpathlineto{\pgfqpoint{-0.048611in}{0.000000in}}%
\pgfusepath{stroke,fill}%
}%
\begin{pgfscope}%
\pgfsys@transformshift{0.728688in}{1.478749in}%
\pgfsys@useobject{currentmarker}{}%
\end{pgfscope}%
\end{pgfscope}%
\begin{pgfscope}%
\definecolor{textcolor}{rgb}{0.000000,0.000000,0.000000}%
\pgfsetstrokecolor{textcolor}%
\pgfsetfillcolor{textcolor}%
\pgftext[x=0.322221in, y=1.425988in, left, base]{\color{textcolor}\sffamily\fontsize{10.000000}{12.000000}\selectfont 0.05}%
\end{pgfscope}%
\begin{pgfscope}%
\pgfpathrectangle{\pgfqpoint{0.728688in}{0.521603in}}{\pgfqpoint{9.687500in}{4.235000in}}%
\pgfusepath{clip}%
\pgfsetrectcap%
\pgfsetroundjoin%
\pgfsetlinewidth{0.803000pt}%
\definecolor{currentstroke}{rgb}{0.690196,0.690196,0.690196}%
\pgfsetstrokecolor{currentstroke}%
\pgfsetdash{}{0pt}%
\pgfpathmoveto{\pgfqpoint{0.728688in}{2.243515in}}%
\pgfpathlineto{\pgfqpoint{10.416188in}{2.243515in}}%
\pgfusepath{stroke}%
\end{pgfscope}%
\begin{pgfscope}%
\pgfsetbuttcap%
\pgfsetroundjoin%
\definecolor{currentfill}{rgb}{0.000000,0.000000,0.000000}%
\pgfsetfillcolor{currentfill}%
\pgfsetlinewidth{0.803000pt}%
\definecolor{currentstroke}{rgb}{0.000000,0.000000,0.000000}%
\pgfsetstrokecolor{currentstroke}%
\pgfsetdash{}{0pt}%
\pgfsys@defobject{currentmarker}{\pgfqpoint{-0.048611in}{0.000000in}}{\pgfqpoint{-0.000000in}{0.000000in}}{%
\pgfpathmoveto{\pgfqpoint{-0.000000in}{0.000000in}}%
\pgfpathlineto{\pgfqpoint{-0.048611in}{0.000000in}}%
\pgfusepath{stroke,fill}%
}%
\begin{pgfscope}%
\pgfsys@transformshift{0.728688in}{2.243515in}%
\pgfsys@useobject{currentmarker}{}%
\end{pgfscope}%
\end{pgfscope}%
\begin{pgfscope}%
\definecolor{textcolor}{rgb}{0.000000,0.000000,0.000000}%
\pgfsetstrokecolor{textcolor}%
\pgfsetfillcolor{textcolor}%
\pgftext[x=0.322221in, y=2.190753in, left, base]{\color{textcolor}\sffamily\fontsize{10.000000}{12.000000}\selectfont 0.10}%
\end{pgfscope}%
\begin{pgfscope}%
\pgfpathrectangle{\pgfqpoint{0.728688in}{0.521603in}}{\pgfqpoint{9.687500in}{4.235000in}}%
\pgfusepath{clip}%
\pgfsetrectcap%
\pgfsetroundjoin%
\pgfsetlinewidth{0.803000pt}%
\definecolor{currentstroke}{rgb}{0.690196,0.690196,0.690196}%
\pgfsetstrokecolor{currentstroke}%
\pgfsetdash{}{0pt}%
\pgfpathmoveto{\pgfqpoint{0.728688in}{3.008280in}}%
\pgfpathlineto{\pgfqpoint{10.416188in}{3.008280in}}%
\pgfusepath{stroke}%
\end{pgfscope}%
\begin{pgfscope}%
\pgfsetbuttcap%
\pgfsetroundjoin%
\definecolor{currentfill}{rgb}{0.000000,0.000000,0.000000}%
\pgfsetfillcolor{currentfill}%
\pgfsetlinewidth{0.803000pt}%
\definecolor{currentstroke}{rgb}{0.000000,0.000000,0.000000}%
\pgfsetstrokecolor{currentstroke}%
\pgfsetdash{}{0pt}%
\pgfsys@defobject{currentmarker}{\pgfqpoint{-0.048611in}{0.000000in}}{\pgfqpoint{-0.000000in}{0.000000in}}{%
\pgfpathmoveto{\pgfqpoint{-0.000000in}{0.000000in}}%
\pgfpathlineto{\pgfqpoint{-0.048611in}{0.000000in}}%
\pgfusepath{stroke,fill}%
}%
\begin{pgfscope}%
\pgfsys@transformshift{0.728688in}{3.008280in}%
\pgfsys@useobject{currentmarker}{}%
\end{pgfscope}%
\end{pgfscope}%
\begin{pgfscope}%
\definecolor{textcolor}{rgb}{0.000000,0.000000,0.000000}%
\pgfsetstrokecolor{textcolor}%
\pgfsetfillcolor{textcolor}%
\pgftext[x=0.322221in, y=2.955518in, left, base]{\color{textcolor}\sffamily\fontsize{10.000000}{12.000000}\selectfont 0.15}%
\end{pgfscope}%
\begin{pgfscope}%
\pgfpathrectangle{\pgfqpoint{0.728688in}{0.521603in}}{\pgfqpoint{9.687500in}{4.235000in}}%
\pgfusepath{clip}%
\pgfsetrectcap%
\pgfsetroundjoin%
\pgfsetlinewidth{0.803000pt}%
\definecolor{currentstroke}{rgb}{0.690196,0.690196,0.690196}%
\pgfsetstrokecolor{currentstroke}%
\pgfsetdash{}{0pt}%
\pgfpathmoveto{\pgfqpoint{0.728688in}{3.773045in}}%
\pgfpathlineto{\pgfqpoint{10.416188in}{3.773045in}}%
\pgfusepath{stroke}%
\end{pgfscope}%
\begin{pgfscope}%
\pgfsetbuttcap%
\pgfsetroundjoin%
\definecolor{currentfill}{rgb}{0.000000,0.000000,0.000000}%
\pgfsetfillcolor{currentfill}%
\pgfsetlinewidth{0.803000pt}%
\definecolor{currentstroke}{rgb}{0.000000,0.000000,0.000000}%
\pgfsetstrokecolor{currentstroke}%
\pgfsetdash{}{0pt}%
\pgfsys@defobject{currentmarker}{\pgfqpoint{-0.048611in}{0.000000in}}{\pgfqpoint{-0.000000in}{0.000000in}}{%
\pgfpathmoveto{\pgfqpoint{-0.000000in}{0.000000in}}%
\pgfpathlineto{\pgfqpoint{-0.048611in}{0.000000in}}%
\pgfusepath{stroke,fill}%
}%
\begin{pgfscope}%
\pgfsys@transformshift{0.728688in}{3.773045in}%
\pgfsys@useobject{currentmarker}{}%
\end{pgfscope}%
\end{pgfscope}%
\begin{pgfscope}%
\definecolor{textcolor}{rgb}{0.000000,0.000000,0.000000}%
\pgfsetstrokecolor{textcolor}%
\pgfsetfillcolor{textcolor}%
\pgftext[x=0.322221in, y=3.720284in, left, base]{\color{textcolor}\sffamily\fontsize{10.000000}{12.000000}\selectfont 0.20}%
\end{pgfscope}%
\begin{pgfscope}%
\pgfpathrectangle{\pgfqpoint{0.728688in}{0.521603in}}{\pgfqpoint{9.687500in}{4.235000in}}%
\pgfusepath{clip}%
\pgfsetrectcap%
\pgfsetroundjoin%
\pgfsetlinewidth{0.803000pt}%
\definecolor{currentstroke}{rgb}{0.690196,0.690196,0.690196}%
\pgfsetstrokecolor{currentstroke}%
\pgfsetdash{}{0pt}%
\pgfpathmoveto{\pgfqpoint{0.728688in}{4.537811in}}%
\pgfpathlineto{\pgfqpoint{10.416188in}{4.537811in}}%
\pgfusepath{stroke}%
\end{pgfscope}%
\begin{pgfscope}%
\pgfsetbuttcap%
\pgfsetroundjoin%
\definecolor{currentfill}{rgb}{0.000000,0.000000,0.000000}%
\pgfsetfillcolor{currentfill}%
\pgfsetlinewidth{0.803000pt}%
\definecolor{currentstroke}{rgb}{0.000000,0.000000,0.000000}%
\pgfsetstrokecolor{currentstroke}%
\pgfsetdash{}{0pt}%
\pgfsys@defobject{currentmarker}{\pgfqpoint{-0.048611in}{0.000000in}}{\pgfqpoint{-0.000000in}{0.000000in}}{%
\pgfpathmoveto{\pgfqpoint{-0.000000in}{0.000000in}}%
\pgfpathlineto{\pgfqpoint{-0.048611in}{0.000000in}}%
\pgfusepath{stroke,fill}%
}%
\begin{pgfscope}%
\pgfsys@transformshift{0.728688in}{4.537811in}%
\pgfsys@useobject{currentmarker}{}%
\end{pgfscope}%
\end{pgfscope}%
\begin{pgfscope}%
\definecolor{textcolor}{rgb}{0.000000,0.000000,0.000000}%
\pgfsetstrokecolor{textcolor}%
\pgfsetfillcolor{textcolor}%
\pgftext[x=0.322221in, y=4.485049in, left, base]{\color{textcolor}\sffamily\fontsize{10.000000}{12.000000}\selectfont 0.25}%
\end{pgfscope}%
\begin{pgfscope}%
\pgfpathrectangle{\pgfqpoint{0.728688in}{0.521603in}}{\pgfqpoint{9.687500in}{4.235000in}}%
\pgfusepath{clip}%
\pgfsetrectcap%
\pgfsetroundjoin%
\pgfsetlinewidth{0.803000pt}%
\definecolor{currentstroke}{rgb}{0.600000,0.600000,0.600000}%
\pgfsetstrokecolor{currentstroke}%
\pgfsetstrokeopacity{0.200000}%
\pgfsetdash{}{0pt}%
\pgfpathmoveto{\pgfqpoint{0.728688in}{0.561031in}}%
\pgfpathlineto{\pgfqpoint{10.416188in}{0.561031in}}%
\pgfusepath{stroke}%
\end{pgfscope}%
\begin{pgfscope}%
\pgfsetbuttcap%
\pgfsetroundjoin%
\definecolor{currentfill}{rgb}{0.000000,0.000000,0.000000}%
\pgfsetfillcolor{currentfill}%
\pgfsetlinewidth{0.602250pt}%
\definecolor{currentstroke}{rgb}{0.000000,0.000000,0.000000}%
\pgfsetstrokecolor{currentstroke}%
\pgfsetdash{}{0pt}%
\pgfsys@defobject{currentmarker}{\pgfqpoint{-0.027778in}{0.000000in}}{\pgfqpoint{-0.000000in}{0.000000in}}{%
\pgfpathmoveto{\pgfqpoint{-0.000000in}{0.000000in}}%
\pgfpathlineto{\pgfqpoint{-0.027778in}{0.000000in}}%
\pgfusepath{stroke,fill}%
}%
\begin{pgfscope}%
\pgfsys@transformshift{0.728688in}{0.561031in}%
\pgfsys@useobject{currentmarker}{}%
\end{pgfscope}%
\end{pgfscope}%
\begin{pgfscope}%
\pgfpathrectangle{\pgfqpoint{0.728688in}{0.521603in}}{\pgfqpoint{9.687500in}{4.235000in}}%
\pgfusepath{clip}%
\pgfsetrectcap%
\pgfsetroundjoin%
\pgfsetlinewidth{0.803000pt}%
\definecolor{currentstroke}{rgb}{0.600000,0.600000,0.600000}%
\pgfsetstrokecolor{currentstroke}%
\pgfsetstrokeopacity{0.200000}%
\pgfsetdash{}{0pt}%
\pgfpathmoveto{\pgfqpoint{0.728688in}{0.866937in}}%
\pgfpathlineto{\pgfqpoint{10.416188in}{0.866937in}}%
\pgfusepath{stroke}%
\end{pgfscope}%
\begin{pgfscope}%
\pgfsetbuttcap%
\pgfsetroundjoin%
\definecolor{currentfill}{rgb}{0.000000,0.000000,0.000000}%
\pgfsetfillcolor{currentfill}%
\pgfsetlinewidth{0.602250pt}%
\definecolor{currentstroke}{rgb}{0.000000,0.000000,0.000000}%
\pgfsetstrokecolor{currentstroke}%
\pgfsetdash{}{0pt}%
\pgfsys@defobject{currentmarker}{\pgfqpoint{-0.027778in}{0.000000in}}{\pgfqpoint{-0.000000in}{0.000000in}}{%
\pgfpathmoveto{\pgfqpoint{-0.000000in}{0.000000in}}%
\pgfpathlineto{\pgfqpoint{-0.027778in}{0.000000in}}%
\pgfusepath{stroke,fill}%
}%
\begin{pgfscope}%
\pgfsys@transformshift{0.728688in}{0.866937in}%
\pgfsys@useobject{currentmarker}{}%
\end{pgfscope}%
\end{pgfscope}%
\begin{pgfscope}%
\pgfpathrectangle{\pgfqpoint{0.728688in}{0.521603in}}{\pgfqpoint{9.687500in}{4.235000in}}%
\pgfusepath{clip}%
\pgfsetrectcap%
\pgfsetroundjoin%
\pgfsetlinewidth{0.803000pt}%
\definecolor{currentstroke}{rgb}{0.600000,0.600000,0.600000}%
\pgfsetstrokecolor{currentstroke}%
\pgfsetstrokeopacity{0.200000}%
\pgfsetdash{}{0pt}%
\pgfpathmoveto{\pgfqpoint{0.728688in}{1.019890in}}%
\pgfpathlineto{\pgfqpoint{10.416188in}{1.019890in}}%
\pgfusepath{stroke}%
\end{pgfscope}%
\begin{pgfscope}%
\pgfsetbuttcap%
\pgfsetroundjoin%
\definecolor{currentfill}{rgb}{0.000000,0.000000,0.000000}%
\pgfsetfillcolor{currentfill}%
\pgfsetlinewidth{0.602250pt}%
\definecolor{currentstroke}{rgb}{0.000000,0.000000,0.000000}%
\pgfsetstrokecolor{currentstroke}%
\pgfsetdash{}{0pt}%
\pgfsys@defobject{currentmarker}{\pgfqpoint{-0.027778in}{0.000000in}}{\pgfqpoint{-0.000000in}{0.000000in}}{%
\pgfpathmoveto{\pgfqpoint{-0.000000in}{0.000000in}}%
\pgfpathlineto{\pgfqpoint{-0.027778in}{0.000000in}}%
\pgfusepath{stroke,fill}%
}%
\begin{pgfscope}%
\pgfsys@transformshift{0.728688in}{1.019890in}%
\pgfsys@useobject{currentmarker}{}%
\end{pgfscope}%
\end{pgfscope}%
\begin{pgfscope}%
\pgfpathrectangle{\pgfqpoint{0.728688in}{0.521603in}}{\pgfqpoint{9.687500in}{4.235000in}}%
\pgfusepath{clip}%
\pgfsetrectcap%
\pgfsetroundjoin%
\pgfsetlinewidth{0.803000pt}%
\definecolor{currentstroke}{rgb}{0.600000,0.600000,0.600000}%
\pgfsetstrokecolor{currentstroke}%
\pgfsetstrokeopacity{0.200000}%
\pgfsetdash{}{0pt}%
\pgfpathmoveto{\pgfqpoint{0.728688in}{1.172843in}}%
\pgfpathlineto{\pgfqpoint{10.416188in}{1.172843in}}%
\pgfusepath{stroke}%
\end{pgfscope}%
\begin{pgfscope}%
\pgfsetbuttcap%
\pgfsetroundjoin%
\definecolor{currentfill}{rgb}{0.000000,0.000000,0.000000}%
\pgfsetfillcolor{currentfill}%
\pgfsetlinewidth{0.602250pt}%
\definecolor{currentstroke}{rgb}{0.000000,0.000000,0.000000}%
\pgfsetstrokecolor{currentstroke}%
\pgfsetdash{}{0pt}%
\pgfsys@defobject{currentmarker}{\pgfqpoint{-0.027778in}{0.000000in}}{\pgfqpoint{-0.000000in}{0.000000in}}{%
\pgfpathmoveto{\pgfqpoint{-0.000000in}{0.000000in}}%
\pgfpathlineto{\pgfqpoint{-0.027778in}{0.000000in}}%
\pgfusepath{stroke,fill}%
}%
\begin{pgfscope}%
\pgfsys@transformshift{0.728688in}{1.172843in}%
\pgfsys@useobject{currentmarker}{}%
\end{pgfscope}%
\end{pgfscope}%
\begin{pgfscope}%
\pgfpathrectangle{\pgfqpoint{0.728688in}{0.521603in}}{\pgfqpoint{9.687500in}{4.235000in}}%
\pgfusepath{clip}%
\pgfsetrectcap%
\pgfsetroundjoin%
\pgfsetlinewidth{0.803000pt}%
\definecolor{currentstroke}{rgb}{0.600000,0.600000,0.600000}%
\pgfsetstrokecolor{currentstroke}%
\pgfsetstrokeopacity{0.200000}%
\pgfsetdash{}{0pt}%
\pgfpathmoveto{\pgfqpoint{0.728688in}{1.325796in}}%
\pgfpathlineto{\pgfqpoint{10.416188in}{1.325796in}}%
\pgfusepath{stroke}%
\end{pgfscope}%
\begin{pgfscope}%
\pgfsetbuttcap%
\pgfsetroundjoin%
\definecolor{currentfill}{rgb}{0.000000,0.000000,0.000000}%
\pgfsetfillcolor{currentfill}%
\pgfsetlinewidth{0.602250pt}%
\definecolor{currentstroke}{rgb}{0.000000,0.000000,0.000000}%
\pgfsetstrokecolor{currentstroke}%
\pgfsetdash{}{0pt}%
\pgfsys@defobject{currentmarker}{\pgfqpoint{-0.027778in}{0.000000in}}{\pgfqpoint{-0.000000in}{0.000000in}}{%
\pgfpathmoveto{\pgfqpoint{-0.000000in}{0.000000in}}%
\pgfpathlineto{\pgfqpoint{-0.027778in}{0.000000in}}%
\pgfusepath{stroke,fill}%
}%
\begin{pgfscope}%
\pgfsys@transformshift{0.728688in}{1.325796in}%
\pgfsys@useobject{currentmarker}{}%
\end{pgfscope}%
\end{pgfscope}%
\begin{pgfscope}%
\pgfpathrectangle{\pgfqpoint{0.728688in}{0.521603in}}{\pgfqpoint{9.687500in}{4.235000in}}%
\pgfusepath{clip}%
\pgfsetrectcap%
\pgfsetroundjoin%
\pgfsetlinewidth{0.803000pt}%
\definecolor{currentstroke}{rgb}{0.600000,0.600000,0.600000}%
\pgfsetstrokecolor{currentstroke}%
\pgfsetstrokeopacity{0.200000}%
\pgfsetdash{}{0pt}%
\pgfpathmoveto{\pgfqpoint{0.728688in}{1.631702in}}%
\pgfpathlineto{\pgfqpoint{10.416188in}{1.631702in}}%
\pgfusepath{stroke}%
\end{pgfscope}%
\begin{pgfscope}%
\pgfsetbuttcap%
\pgfsetroundjoin%
\definecolor{currentfill}{rgb}{0.000000,0.000000,0.000000}%
\pgfsetfillcolor{currentfill}%
\pgfsetlinewidth{0.602250pt}%
\definecolor{currentstroke}{rgb}{0.000000,0.000000,0.000000}%
\pgfsetstrokecolor{currentstroke}%
\pgfsetdash{}{0pt}%
\pgfsys@defobject{currentmarker}{\pgfqpoint{-0.027778in}{0.000000in}}{\pgfqpoint{-0.000000in}{0.000000in}}{%
\pgfpathmoveto{\pgfqpoint{-0.000000in}{0.000000in}}%
\pgfpathlineto{\pgfqpoint{-0.027778in}{0.000000in}}%
\pgfusepath{stroke,fill}%
}%
\begin{pgfscope}%
\pgfsys@transformshift{0.728688in}{1.631702in}%
\pgfsys@useobject{currentmarker}{}%
\end{pgfscope}%
\end{pgfscope}%
\begin{pgfscope}%
\pgfpathrectangle{\pgfqpoint{0.728688in}{0.521603in}}{\pgfqpoint{9.687500in}{4.235000in}}%
\pgfusepath{clip}%
\pgfsetrectcap%
\pgfsetroundjoin%
\pgfsetlinewidth{0.803000pt}%
\definecolor{currentstroke}{rgb}{0.600000,0.600000,0.600000}%
\pgfsetstrokecolor{currentstroke}%
\pgfsetstrokeopacity{0.200000}%
\pgfsetdash{}{0pt}%
\pgfpathmoveto{\pgfqpoint{0.728688in}{1.784655in}}%
\pgfpathlineto{\pgfqpoint{10.416188in}{1.784655in}}%
\pgfusepath{stroke}%
\end{pgfscope}%
\begin{pgfscope}%
\pgfsetbuttcap%
\pgfsetroundjoin%
\definecolor{currentfill}{rgb}{0.000000,0.000000,0.000000}%
\pgfsetfillcolor{currentfill}%
\pgfsetlinewidth{0.602250pt}%
\definecolor{currentstroke}{rgb}{0.000000,0.000000,0.000000}%
\pgfsetstrokecolor{currentstroke}%
\pgfsetdash{}{0pt}%
\pgfsys@defobject{currentmarker}{\pgfqpoint{-0.027778in}{0.000000in}}{\pgfqpoint{-0.000000in}{0.000000in}}{%
\pgfpathmoveto{\pgfqpoint{-0.000000in}{0.000000in}}%
\pgfpathlineto{\pgfqpoint{-0.027778in}{0.000000in}}%
\pgfusepath{stroke,fill}%
}%
\begin{pgfscope}%
\pgfsys@transformshift{0.728688in}{1.784655in}%
\pgfsys@useobject{currentmarker}{}%
\end{pgfscope}%
\end{pgfscope}%
\begin{pgfscope}%
\pgfpathrectangle{\pgfqpoint{0.728688in}{0.521603in}}{\pgfqpoint{9.687500in}{4.235000in}}%
\pgfusepath{clip}%
\pgfsetrectcap%
\pgfsetroundjoin%
\pgfsetlinewidth{0.803000pt}%
\definecolor{currentstroke}{rgb}{0.600000,0.600000,0.600000}%
\pgfsetstrokecolor{currentstroke}%
\pgfsetstrokeopacity{0.200000}%
\pgfsetdash{}{0pt}%
\pgfpathmoveto{\pgfqpoint{0.728688in}{1.937608in}}%
\pgfpathlineto{\pgfqpoint{10.416188in}{1.937608in}}%
\pgfusepath{stroke}%
\end{pgfscope}%
\begin{pgfscope}%
\pgfsetbuttcap%
\pgfsetroundjoin%
\definecolor{currentfill}{rgb}{0.000000,0.000000,0.000000}%
\pgfsetfillcolor{currentfill}%
\pgfsetlinewidth{0.602250pt}%
\definecolor{currentstroke}{rgb}{0.000000,0.000000,0.000000}%
\pgfsetstrokecolor{currentstroke}%
\pgfsetdash{}{0pt}%
\pgfsys@defobject{currentmarker}{\pgfqpoint{-0.027778in}{0.000000in}}{\pgfqpoint{-0.000000in}{0.000000in}}{%
\pgfpathmoveto{\pgfqpoint{-0.000000in}{0.000000in}}%
\pgfpathlineto{\pgfqpoint{-0.027778in}{0.000000in}}%
\pgfusepath{stroke,fill}%
}%
\begin{pgfscope}%
\pgfsys@transformshift{0.728688in}{1.937608in}%
\pgfsys@useobject{currentmarker}{}%
\end{pgfscope}%
\end{pgfscope}%
\begin{pgfscope}%
\pgfpathrectangle{\pgfqpoint{0.728688in}{0.521603in}}{\pgfqpoint{9.687500in}{4.235000in}}%
\pgfusepath{clip}%
\pgfsetrectcap%
\pgfsetroundjoin%
\pgfsetlinewidth{0.803000pt}%
\definecolor{currentstroke}{rgb}{0.600000,0.600000,0.600000}%
\pgfsetstrokecolor{currentstroke}%
\pgfsetstrokeopacity{0.200000}%
\pgfsetdash{}{0pt}%
\pgfpathmoveto{\pgfqpoint{0.728688in}{2.090562in}}%
\pgfpathlineto{\pgfqpoint{10.416188in}{2.090562in}}%
\pgfusepath{stroke}%
\end{pgfscope}%
\begin{pgfscope}%
\pgfsetbuttcap%
\pgfsetroundjoin%
\definecolor{currentfill}{rgb}{0.000000,0.000000,0.000000}%
\pgfsetfillcolor{currentfill}%
\pgfsetlinewidth{0.602250pt}%
\definecolor{currentstroke}{rgb}{0.000000,0.000000,0.000000}%
\pgfsetstrokecolor{currentstroke}%
\pgfsetdash{}{0pt}%
\pgfsys@defobject{currentmarker}{\pgfqpoint{-0.027778in}{0.000000in}}{\pgfqpoint{-0.000000in}{0.000000in}}{%
\pgfpathmoveto{\pgfqpoint{-0.000000in}{0.000000in}}%
\pgfpathlineto{\pgfqpoint{-0.027778in}{0.000000in}}%
\pgfusepath{stroke,fill}%
}%
\begin{pgfscope}%
\pgfsys@transformshift{0.728688in}{2.090562in}%
\pgfsys@useobject{currentmarker}{}%
\end{pgfscope}%
\end{pgfscope}%
\begin{pgfscope}%
\pgfpathrectangle{\pgfqpoint{0.728688in}{0.521603in}}{\pgfqpoint{9.687500in}{4.235000in}}%
\pgfusepath{clip}%
\pgfsetrectcap%
\pgfsetroundjoin%
\pgfsetlinewidth{0.803000pt}%
\definecolor{currentstroke}{rgb}{0.600000,0.600000,0.600000}%
\pgfsetstrokecolor{currentstroke}%
\pgfsetstrokeopacity{0.200000}%
\pgfsetdash{}{0pt}%
\pgfpathmoveto{\pgfqpoint{0.728688in}{2.396468in}}%
\pgfpathlineto{\pgfqpoint{10.416188in}{2.396468in}}%
\pgfusepath{stroke}%
\end{pgfscope}%
\begin{pgfscope}%
\pgfsetbuttcap%
\pgfsetroundjoin%
\definecolor{currentfill}{rgb}{0.000000,0.000000,0.000000}%
\pgfsetfillcolor{currentfill}%
\pgfsetlinewidth{0.602250pt}%
\definecolor{currentstroke}{rgb}{0.000000,0.000000,0.000000}%
\pgfsetstrokecolor{currentstroke}%
\pgfsetdash{}{0pt}%
\pgfsys@defobject{currentmarker}{\pgfqpoint{-0.027778in}{0.000000in}}{\pgfqpoint{-0.000000in}{0.000000in}}{%
\pgfpathmoveto{\pgfqpoint{-0.000000in}{0.000000in}}%
\pgfpathlineto{\pgfqpoint{-0.027778in}{0.000000in}}%
\pgfusepath{stroke,fill}%
}%
\begin{pgfscope}%
\pgfsys@transformshift{0.728688in}{2.396468in}%
\pgfsys@useobject{currentmarker}{}%
\end{pgfscope}%
\end{pgfscope}%
\begin{pgfscope}%
\pgfpathrectangle{\pgfqpoint{0.728688in}{0.521603in}}{\pgfqpoint{9.687500in}{4.235000in}}%
\pgfusepath{clip}%
\pgfsetrectcap%
\pgfsetroundjoin%
\pgfsetlinewidth{0.803000pt}%
\definecolor{currentstroke}{rgb}{0.600000,0.600000,0.600000}%
\pgfsetstrokecolor{currentstroke}%
\pgfsetstrokeopacity{0.200000}%
\pgfsetdash{}{0pt}%
\pgfpathmoveto{\pgfqpoint{0.728688in}{2.549421in}}%
\pgfpathlineto{\pgfqpoint{10.416188in}{2.549421in}}%
\pgfusepath{stroke}%
\end{pgfscope}%
\begin{pgfscope}%
\pgfsetbuttcap%
\pgfsetroundjoin%
\definecolor{currentfill}{rgb}{0.000000,0.000000,0.000000}%
\pgfsetfillcolor{currentfill}%
\pgfsetlinewidth{0.602250pt}%
\definecolor{currentstroke}{rgb}{0.000000,0.000000,0.000000}%
\pgfsetstrokecolor{currentstroke}%
\pgfsetdash{}{0pt}%
\pgfsys@defobject{currentmarker}{\pgfqpoint{-0.027778in}{0.000000in}}{\pgfqpoint{-0.000000in}{0.000000in}}{%
\pgfpathmoveto{\pgfqpoint{-0.000000in}{0.000000in}}%
\pgfpathlineto{\pgfqpoint{-0.027778in}{0.000000in}}%
\pgfusepath{stroke,fill}%
}%
\begin{pgfscope}%
\pgfsys@transformshift{0.728688in}{2.549421in}%
\pgfsys@useobject{currentmarker}{}%
\end{pgfscope}%
\end{pgfscope}%
\begin{pgfscope}%
\pgfpathrectangle{\pgfqpoint{0.728688in}{0.521603in}}{\pgfqpoint{9.687500in}{4.235000in}}%
\pgfusepath{clip}%
\pgfsetrectcap%
\pgfsetroundjoin%
\pgfsetlinewidth{0.803000pt}%
\definecolor{currentstroke}{rgb}{0.600000,0.600000,0.600000}%
\pgfsetstrokecolor{currentstroke}%
\pgfsetstrokeopacity{0.200000}%
\pgfsetdash{}{0pt}%
\pgfpathmoveto{\pgfqpoint{0.728688in}{2.702374in}}%
\pgfpathlineto{\pgfqpoint{10.416188in}{2.702374in}}%
\pgfusepath{stroke}%
\end{pgfscope}%
\begin{pgfscope}%
\pgfsetbuttcap%
\pgfsetroundjoin%
\definecolor{currentfill}{rgb}{0.000000,0.000000,0.000000}%
\pgfsetfillcolor{currentfill}%
\pgfsetlinewidth{0.602250pt}%
\definecolor{currentstroke}{rgb}{0.000000,0.000000,0.000000}%
\pgfsetstrokecolor{currentstroke}%
\pgfsetdash{}{0pt}%
\pgfsys@defobject{currentmarker}{\pgfqpoint{-0.027778in}{0.000000in}}{\pgfqpoint{-0.000000in}{0.000000in}}{%
\pgfpathmoveto{\pgfqpoint{-0.000000in}{0.000000in}}%
\pgfpathlineto{\pgfqpoint{-0.027778in}{0.000000in}}%
\pgfusepath{stroke,fill}%
}%
\begin{pgfscope}%
\pgfsys@transformshift{0.728688in}{2.702374in}%
\pgfsys@useobject{currentmarker}{}%
\end{pgfscope}%
\end{pgfscope}%
\begin{pgfscope}%
\pgfpathrectangle{\pgfqpoint{0.728688in}{0.521603in}}{\pgfqpoint{9.687500in}{4.235000in}}%
\pgfusepath{clip}%
\pgfsetrectcap%
\pgfsetroundjoin%
\pgfsetlinewidth{0.803000pt}%
\definecolor{currentstroke}{rgb}{0.600000,0.600000,0.600000}%
\pgfsetstrokecolor{currentstroke}%
\pgfsetstrokeopacity{0.200000}%
\pgfsetdash{}{0pt}%
\pgfpathmoveto{\pgfqpoint{0.728688in}{2.855327in}}%
\pgfpathlineto{\pgfqpoint{10.416188in}{2.855327in}}%
\pgfusepath{stroke}%
\end{pgfscope}%
\begin{pgfscope}%
\pgfsetbuttcap%
\pgfsetroundjoin%
\definecolor{currentfill}{rgb}{0.000000,0.000000,0.000000}%
\pgfsetfillcolor{currentfill}%
\pgfsetlinewidth{0.602250pt}%
\definecolor{currentstroke}{rgb}{0.000000,0.000000,0.000000}%
\pgfsetstrokecolor{currentstroke}%
\pgfsetdash{}{0pt}%
\pgfsys@defobject{currentmarker}{\pgfqpoint{-0.027778in}{0.000000in}}{\pgfqpoint{-0.000000in}{0.000000in}}{%
\pgfpathmoveto{\pgfqpoint{-0.000000in}{0.000000in}}%
\pgfpathlineto{\pgfqpoint{-0.027778in}{0.000000in}}%
\pgfusepath{stroke,fill}%
}%
\begin{pgfscope}%
\pgfsys@transformshift{0.728688in}{2.855327in}%
\pgfsys@useobject{currentmarker}{}%
\end{pgfscope}%
\end{pgfscope}%
\begin{pgfscope}%
\pgfpathrectangle{\pgfqpoint{0.728688in}{0.521603in}}{\pgfqpoint{9.687500in}{4.235000in}}%
\pgfusepath{clip}%
\pgfsetrectcap%
\pgfsetroundjoin%
\pgfsetlinewidth{0.803000pt}%
\definecolor{currentstroke}{rgb}{0.600000,0.600000,0.600000}%
\pgfsetstrokecolor{currentstroke}%
\pgfsetstrokeopacity{0.200000}%
\pgfsetdash{}{0pt}%
\pgfpathmoveto{\pgfqpoint{0.728688in}{3.161233in}}%
\pgfpathlineto{\pgfqpoint{10.416188in}{3.161233in}}%
\pgfusepath{stroke}%
\end{pgfscope}%
\begin{pgfscope}%
\pgfsetbuttcap%
\pgfsetroundjoin%
\definecolor{currentfill}{rgb}{0.000000,0.000000,0.000000}%
\pgfsetfillcolor{currentfill}%
\pgfsetlinewidth{0.602250pt}%
\definecolor{currentstroke}{rgb}{0.000000,0.000000,0.000000}%
\pgfsetstrokecolor{currentstroke}%
\pgfsetdash{}{0pt}%
\pgfsys@defobject{currentmarker}{\pgfqpoint{-0.027778in}{0.000000in}}{\pgfqpoint{-0.000000in}{0.000000in}}{%
\pgfpathmoveto{\pgfqpoint{-0.000000in}{0.000000in}}%
\pgfpathlineto{\pgfqpoint{-0.027778in}{0.000000in}}%
\pgfusepath{stroke,fill}%
}%
\begin{pgfscope}%
\pgfsys@transformshift{0.728688in}{3.161233in}%
\pgfsys@useobject{currentmarker}{}%
\end{pgfscope}%
\end{pgfscope}%
\begin{pgfscope}%
\pgfpathrectangle{\pgfqpoint{0.728688in}{0.521603in}}{\pgfqpoint{9.687500in}{4.235000in}}%
\pgfusepath{clip}%
\pgfsetrectcap%
\pgfsetroundjoin%
\pgfsetlinewidth{0.803000pt}%
\definecolor{currentstroke}{rgb}{0.600000,0.600000,0.600000}%
\pgfsetstrokecolor{currentstroke}%
\pgfsetstrokeopacity{0.200000}%
\pgfsetdash{}{0pt}%
\pgfpathmoveto{\pgfqpoint{0.728688in}{3.314186in}}%
\pgfpathlineto{\pgfqpoint{10.416188in}{3.314186in}}%
\pgfusepath{stroke}%
\end{pgfscope}%
\begin{pgfscope}%
\pgfsetbuttcap%
\pgfsetroundjoin%
\definecolor{currentfill}{rgb}{0.000000,0.000000,0.000000}%
\pgfsetfillcolor{currentfill}%
\pgfsetlinewidth{0.602250pt}%
\definecolor{currentstroke}{rgb}{0.000000,0.000000,0.000000}%
\pgfsetstrokecolor{currentstroke}%
\pgfsetdash{}{0pt}%
\pgfsys@defobject{currentmarker}{\pgfqpoint{-0.027778in}{0.000000in}}{\pgfqpoint{-0.000000in}{0.000000in}}{%
\pgfpathmoveto{\pgfqpoint{-0.000000in}{0.000000in}}%
\pgfpathlineto{\pgfqpoint{-0.027778in}{0.000000in}}%
\pgfusepath{stroke,fill}%
}%
\begin{pgfscope}%
\pgfsys@transformshift{0.728688in}{3.314186in}%
\pgfsys@useobject{currentmarker}{}%
\end{pgfscope}%
\end{pgfscope}%
\begin{pgfscope}%
\pgfpathrectangle{\pgfqpoint{0.728688in}{0.521603in}}{\pgfqpoint{9.687500in}{4.235000in}}%
\pgfusepath{clip}%
\pgfsetrectcap%
\pgfsetroundjoin%
\pgfsetlinewidth{0.803000pt}%
\definecolor{currentstroke}{rgb}{0.600000,0.600000,0.600000}%
\pgfsetstrokecolor{currentstroke}%
\pgfsetstrokeopacity{0.200000}%
\pgfsetdash{}{0pt}%
\pgfpathmoveto{\pgfqpoint{0.728688in}{3.467139in}}%
\pgfpathlineto{\pgfqpoint{10.416188in}{3.467139in}}%
\pgfusepath{stroke}%
\end{pgfscope}%
\begin{pgfscope}%
\pgfsetbuttcap%
\pgfsetroundjoin%
\definecolor{currentfill}{rgb}{0.000000,0.000000,0.000000}%
\pgfsetfillcolor{currentfill}%
\pgfsetlinewidth{0.602250pt}%
\definecolor{currentstroke}{rgb}{0.000000,0.000000,0.000000}%
\pgfsetstrokecolor{currentstroke}%
\pgfsetdash{}{0pt}%
\pgfsys@defobject{currentmarker}{\pgfqpoint{-0.027778in}{0.000000in}}{\pgfqpoint{-0.000000in}{0.000000in}}{%
\pgfpathmoveto{\pgfqpoint{-0.000000in}{0.000000in}}%
\pgfpathlineto{\pgfqpoint{-0.027778in}{0.000000in}}%
\pgfusepath{stroke,fill}%
}%
\begin{pgfscope}%
\pgfsys@transformshift{0.728688in}{3.467139in}%
\pgfsys@useobject{currentmarker}{}%
\end{pgfscope}%
\end{pgfscope}%
\begin{pgfscope}%
\pgfpathrectangle{\pgfqpoint{0.728688in}{0.521603in}}{\pgfqpoint{9.687500in}{4.235000in}}%
\pgfusepath{clip}%
\pgfsetrectcap%
\pgfsetroundjoin%
\pgfsetlinewidth{0.803000pt}%
\definecolor{currentstroke}{rgb}{0.600000,0.600000,0.600000}%
\pgfsetstrokecolor{currentstroke}%
\pgfsetstrokeopacity{0.200000}%
\pgfsetdash{}{0pt}%
\pgfpathmoveto{\pgfqpoint{0.728688in}{3.620092in}}%
\pgfpathlineto{\pgfqpoint{10.416188in}{3.620092in}}%
\pgfusepath{stroke}%
\end{pgfscope}%
\begin{pgfscope}%
\pgfsetbuttcap%
\pgfsetroundjoin%
\definecolor{currentfill}{rgb}{0.000000,0.000000,0.000000}%
\pgfsetfillcolor{currentfill}%
\pgfsetlinewidth{0.602250pt}%
\definecolor{currentstroke}{rgb}{0.000000,0.000000,0.000000}%
\pgfsetstrokecolor{currentstroke}%
\pgfsetdash{}{0pt}%
\pgfsys@defobject{currentmarker}{\pgfqpoint{-0.027778in}{0.000000in}}{\pgfqpoint{-0.000000in}{0.000000in}}{%
\pgfpathmoveto{\pgfqpoint{-0.000000in}{0.000000in}}%
\pgfpathlineto{\pgfqpoint{-0.027778in}{0.000000in}}%
\pgfusepath{stroke,fill}%
}%
\begin{pgfscope}%
\pgfsys@transformshift{0.728688in}{3.620092in}%
\pgfsys@useobject{currentmarker}{}%
\end{pgfscope}%
\end{pgfscope}%
\begin{pgfscope}%
\pgfpathrectangle{\pgfqpoint{0.728688in}{0.521603in}}{\pgfqpoint{9.687500in}{4.235000in}}%
\pgfusepath{clip}%
\pgfsetrectcap%
\pgfsetroundjoin%
\pgfsetlinewidth{0.803000pt}%
\definecolor{currentstroke}{rgb}{0.600000,0.600000,0.600000}%
\pgfsetstrokecolor{currentstroke}%
\pgfsetstrokeopacity{0.200000}%
\pgfsetdash{}{0pt}%
\pgfpathmoveto{\pgfqpoint{0.728688in}{3.925998in}}%
\pgfpathlineto{\pgfqpoint{10.416188in}{3.925998in}}%
\pgfusepath{stroke}%
\end{pgfscope}%
\begin{pgfscope}%
\pgfsetbuttcap%
\pgfsetroundjoin%
\definecolor{currentfill}{rgb}{0.000000,0.000000,0.000000}%
\pgfsetfillcolor{currentfill}%
\pgfsetlinewidth{0.602250pt}%
\definecolor{currentstroke}{rgb}{0.000000,0.000000,0.000000}%
\pgfsetstrokecolor{currentstroke}%
\pgfsetdash{}{0pt}%
\pgfsys@defobject{currentmarker}{\pgfqpoint{-0.027778in}{0.000000in}}{\pgfqpoint{-0.000000in}{0.000000in}}{%
\pgfpathmoveto{\pgfqpoint{-0.000000in}{0.000000in}}%
\pgfpathlineto{\pgfqpoint{-0.027778in}{0.000000in}}%
\pgfusepath{stroke,fill}%
}%
\begin{pgfscope}%
\pgfsys@transformshift{0.728688in}{3.925998in}%
\pgfsys@useobject{currentmarker}{}%
\end{pgfscope}%
\end{pgfscope}%
\begin{pgfscope}%
\pgfpathrectangle{\pgfqpoint{0.728688in}{0.521603in}}{\pgfqpoint{9.687500in}{4.235000in}}%
\pgfusepath{clip}%
\pgfsetrectcap%
\pgfsetroundjoin%
\pgfsetlinewidth{0.803000pt}%
\definecolor{currentstroke}{rgb}{0.600000,0.600000,0.600000}%
\pgfsetstrokecolor{currentstroke}%
\pgfsetstrokeopacity{0.200000}%
\pgfsetdash{}{0pt}%
\pgfpathmoveto{\pgfqpoint{0.728688in}{4.078951in}}%
\pgfpathlineto{\pgfqpoint{10.416188in}{4.078951in}}%
\pgfusepath{stroke}%
\end{pgfscope}%
\begin{pgfscope}%
\pgfsetbuttcap%
\pgfsetroundjoin%
\definecolor{currentfill}{rgb}{0.000000,0.000000,0.000000}%
\pgfsetfillcolor{currentfill}%
\pgfsetlinewidth{0.602250pt}%
\definecolor{currentstroke}{rgb}{0.000000,0.000000,0.000000}%
\pgfsetstrokecolor{currentstroke}%
\pgfsetdash{}{0pt}%
\pgfsys@defobject{currentmarker}{\pgfqpoint{-0.027778in}{0.000000in}}{\pgfqpoint{-0.000000in}{0.000000in}}{%
\pgfpathmoveto{\pgfqpoint{-0.000000in}{0.000000in}}%
\pgfpathlineto{\pgfqpoint{-0.027778in}{0.000000in}}%
\pgfusepath{stroke,fill}%
}%
\begin{pgfscope}%
\pgfsys@transformshift{0.728688in}{4.078951in}%
\pgfsys@useobject{currentmarker}{}%
\end{pgfscope}%
\end{pgfscope}%
\begin{pgfscope}%
\pgfpathrectangle{\pgfqpoint{0.728688in}{0.521603in}}{\pgfqpoint{9.687500in}{4.235000in}}%
\pgfusepath{clip}%
\pgfsetrectcap%
\pgfsetroundjoin%
\pgfsetlinewidth{0.803000pt}%
\definecolor{currentstroke}{rgb}{0.600000,0.600000,0.600000}%
\pgfsetstrokecolor{currentstroke}%
\pgfsetstrokeopacity{0.200000}%
\pgfsetdash{}{0pt}%
\pgfpathmoveto{\pgfqpoint{0.728688in}{4.231905in}}%
\pgfpathlineto{\pgfqpoint{10.416188in}{4.231905in}}%
\pgfusepath{stroke}%
\end{pgfscope}%
\begin{pgfscope}%
\pgfsetbuttcap%
\pgfsetroundjoin%
\definecolor{currentfill}{rgb}{0.000000,0.000000,0.000000}%
\pgfsetfillcolor{currentfill}%
\pgfsetlinewidth{0.602250pt}%
\definecolor{currentstroke}{rgb}{0.000000,0.000000,0.000000}%
\pgfsetstrokecolor{currentstroke}%
\pgfsetdash{}{0pt}%
\pgfsys@defobject{currentmarker}{\pgfqpoint{-0.027778in}{0.000000in}}{\pgfqpoint{-0.000000in}{0.000000in}}{%
\pgfpathmoveto{\pgfqpoint{-0.000000in}{0.000000in}}%
\pgfpathlineto{\pgfqpoint{-0.027778in}{0.000000in}}%
\pgfusepath{stroke,fill}%
}%
\begin{pgfscope}%
\pgfsys@transformshift{0.728688in}{4.231905in}%
\pgfsys@useobject{currentmarker}{}%
\end{pgfscope}%
\end{pgfscope}%
\begin{pgfscope}%
\pgfpathrectangle{\pgfqpoint{0.728688in}{0.521603in}}{\pgfqpoint{9.687500in}{4.235000in}}%
\pgfusepath{clip}%
\pgfsetrectcap%
\pgfsetroundjoin%
\pgfsetlinewidth{0.803000pt}%
\definecolor{currentstroke}{rgb}{0.600000,0.600000,0.600000}%
\pgfsetstrokecolor{currentstroke}%
\pgfsetstrokeopacity{0.200000}%
\pgfsetdash{}{0pt}%
\pgfpathmoveto{\pgfqpoint{0.728688in}{4.384858in}}%
\pgfpathlineto{\pgfqpoint{10.416188in}{4.384858in}}%
\pgfusepath{stroke}%
\end{pgfscope}%
\begin{pgfscope}%
\pgfsetbuttcap%
\pgfsetroundjoin%
\definecolor{currentfill}{rgb}{0.000000,0.000000,0.000000}%
\pgfsetfillcolor{currentfill}%
\pgfsetlinewidth{0.602250pt}%
\definecolor{currentstroke}{rgb}{0.000000,0.000000,0.000000}%
\pgfsetstrokecolor{currentstroke}%
\pgfsetdash{}{0pt}%
\pgfsys@defobject{currentmarker}{\pgfqpoint{-0.027778in}{0.000000in}}{\pgfqpoint{-0.000000in}{0.000000in}}{%
\pgfpathmoveto{\pgfqpoint{-0.000000in}{0.000000in}}%
\pgfpathlineto{\pgfqpoint{-0.027778in}{0.000000in}}%
\pgfusepath{stroke,fill}%
}%
\begin{pgfscope}%
\pgfsys@transformshift{0.728688in}{4.384858in}%
\pgfsys@useobject{currentmarker}{}%
\end{pgfscope}%
\end{pgfscope}%
\begin{pgfscope}%
\pgfpathrectangle{\pgfqpoint{0.728688in}{0.521603in}}{\pgfqpoint{9.687500in}{4.235000in}}%
\pgfusepath{clip}%
\pgfsetrectcap%
\pgfsetroundjoin%
\pgfsetlinewidth{0.803000pt}%
\definecolor{currentstroke}{rgb}{0.600000,0.600000,0.600000}%
\pgfsetstrokecolor{currentstroke}%
\pgfsetstrokeopacity{0.200000}%
\pgfsetdash{}{0pt}%
\pgfpathmoveto{\pgfqpoint{0.728688in}{4.690764in}}%
\pgfpathlineto{\pgfqpoint{10.416188in}{4.690764in}}%
\pgfusepath{stroke}%
\end{pgfscope}%
\begin{pgfscope}%
\pgfsetbuttcap%
\pgfsetroundjoin%
\definecolor{currentfill}{rgb}{0.000000,0.000000,0.000000}%
\pgfsetfillcolor{currentfill}%
\pgfsetlinewidth{0.602250pt}%
\definecolor{currentstroke}{rgb}{0.000000,0.000000,0.000000}%
\pgfsetstrokecolor{currentstroke}%
\pgfsetdash{}{0pt}%
\pgfsys@defobject{currentmarker}{\pgfqpoint{-0.027778in}{0.000000in}}{\pgfqpoint{-0.000000in}{0.000000in}}{%
\pgfpathmoveto{\pgfqpoint{-0.000000in}{0.000000in}}%
\pgfpathlineto{\pgfqpoint{-0.027778in}{0.000000in}}%
\pgfusepath{stroke,fill}%
}%
\begin{pgfscope}%
\pgfsys@transformshift{0.728688in}{4.690764in}%
\pgfsys@useobject{currentmarker}{}%
\end{pgfscope}%
\end{pgfscope}%
\begin{pgfscope}%
\definecolor{textcolor}{rgb}{0.000000,0.000000,0.000000}%
\pgfsetstrokecolor{textcolor}%
\pgfsetfillcolor{textcolor}%
\pgftext[x=0.266665in,y=2.639103in,,bottom,rotate=90.000000]{\color{textcolor}\sffamily\fontsize{10.000000}{12.000000}\selectfont avg. infection rate \(\displaystyle \overline{\langle I\rangle}\)}%
\end{pgfscope}%
\begin{pgfscope}%
\pgfpathrectangle{\pgfqpoint{0.728688in}{0.521603in}}{\pgfqpoint{9.687500in}{4.235000in}}%
\pgfusepath{clip}%
\pgfsetbuttcap%
\pgfsetroundjoin%
\pgfsetlinewidth{1.003750pt}%
\definecolor{currentstroke}{rgb}{0.000000,0.000000,1.000000}%
\pgfsetstrokecolor{currentstroke}%
\pgfsetstrokeopacity{0.500000}%
\pgfsetdash{{3.700000pt}{1.600000pt}}{0.000000pt}%
\pgfpathmoveto{\pgfqpoint{1.169029in}{4.046571in}}%
\pgfpathlineto{\pgfqpoint{1.348760in}{4.207524in}}%
\pgfpathlineto{\pgfqpoint{1.528491in}{3.612032in}}%
\pgfpathlineto{\pgfqpoint{1.708222in}{3.675538in}}%
\pgfpathlineto{\pgfqpoint{1.887953in}{3.140982in}}%
\pgfpathlineto{\pgfqpoint{2.067684in}{3.043000in}}%
\pgfpathlineto{\pgfqpoint{2.247415in}{0.862755in}}%
\pgfpathlineto{\pgfqpoint{2.427146in}{1.019771in}}%
\pgfpathlineto{\pgfqpoint{2.606877in}{0.809341in}}%
\pgfpathlineto{\pgfqpoint{2.786608in}{1.230678in}}%
\pgfpathlineto{\pgfqpoint{2.966339in}{0.982488in}}%
\pgfpathlineto{\pgfqpoint{3.146070in}{0.878766in}}%
\pgfpathlineto{\pgfqpoint{3.325801in}{0.780542in}}%
\pgfpathlineto{\pgfqpoint{3.505532in}{0.772835in}}%
\pgfpathlineto{\pgfqpoint{3.685263in}{0.745829in}}%
\pgfpathlineto{\pgfqpoint{3.864994in}{0.778630in}}%
\pgfpathlineto{\pgfqpoint{4.044725in}{0.752521in}}%
\pgfpathlineto{\pgfqpoint{4.224456in}{0.752342in}}%
\pgfpathlineto{\pgfqpoint{4.404187in}{0.755628in}}%
\pgfpathlineto{\pgfqpoint{4.583918in}{0.731131in}}%
\pgfpathlineto{\pgfqpoint{4.763648in}{0.732625in}}%
\pgfpathlineto{\pgfqpoint{4.943379in}{0.731609in}}%
\pgfpathlineto{\pgfqpoint{5.123110in}{0.734955in}}%
\pgfpathlineto{\pgfqpoint{5.302841in}{0.739974in}}%
\pgfpathlineto{\pgfqpoint{5.482572in}{0.741348in}}%
\pgfpathlineto{\pgfqpoint{5.662303in}{0.727128in}}%
\pgfpathlineto{\pgfqpoint{5.842034in}{0.722886in}}%
\pgfpathlineto{\pgfqpoint{6.021765in}{0.720078in}}%
\pgfpathlineto{\pgfqpoint{6.201496in}{0.726053in}}%
\pgfpathlineto{\pgfqpoint{6.381227in}{0.719301in}}%
\pgfpathlineto{\pgfqpoint{6.560958in}{0.718883in}}%
\pgfpathlineto{\pgfqpoint{6.740689in}{0.728502in}}%
\pgfpathlineto{\pgfqpoint{6.920420in}{0.721811in}}%
\pgfpathlineto{\pgfqpoint{7.100151in}{0.719720in}}%
\pgfpathlineto{\pgfqpoint{7.279882in}{0.719540in}}%
\pgfpathlineto{\pgfqpoint{7.459613in}{0.722647in}}%
\pgfpathlineto{\pgfqpoint{7.639344in}{0.718166in}}%
\pgfpathlineto{\pgfqpoint{7.819075in}{0.719182in}}%
\pgfpathlineto{\pgfqpoint{7.998806in}{0.715717in}}%
\pgfpathlineto{\pgfqpoint{8.178537in}{0.718106in}}%
\pgfpathlineto{\pgfqpoint{8.358268in}{0.718106in}}%
\pgfpathlineto{\pgfqpoint{8.537999in}{0.715955in}}%
\pgfpathlineto{\pgfqpoint{8.717730in}{0.715298in}}%
\pgfpathlineto{\pgfqpoint{8.897461in}{0.715657in}}%
\pgfpathlineto{\pgfqpoint{9.077192in}{0.717808in}}%
\pgfpathlineto{\pgfqpoint{9.256923in}{0.715478in}}%
\pgfpathlineto{\pgfqpoint{9.436654in}{0.714103in}}%
\pgfpathlineto{\pgfqpoint{9.616385in}{0.716135in}}%
\pgfpathlineto{\pgfqpoint{9.796116in}{0.714283in}}%
\pgfpathlineto{\pgfqpoint{9.975847in}{0.714342in}}%
\pgfusepath{stroke}%
\end{pgfscope}%
\begin{pgfscope}%
\pgfpathrectangle{\pgfqpoint{0.728688in}{0.521603in}}{\pgfqpoint{9.687500in}{4.235000in}}%
\pgfusepath{clip}%
\pgfsetbuttcap%
\pgfsetroundjoin%
\pgfsetlinewidth{1.003750pt}%
\definecolor{currentstroke}{rgb}{0.980392,0.164706,0.333333}%
\pgfsetstrokecolor{currentstroke}%
\pgfsetstrokeopacity{0.500000}%
\pgfsetdash{{3.700000pt}{1.600000pt}}{0.000000pt}%
\pgfpathmoveto{\pgfqpoint{1.169029in}{4.391817in}}%
\pgfpathlineto{\pgfqpoint{1.348760in}{4.322162in}}%
\pgfpathlineto{\pgfqpoint{1.528491in}{4.026152in}}%
\pgfpathlineto{\pgfqpoint{1.708222in}{3.852214in}}%
\pgfpathlineto{\pgfqpoint{1.887953in}{3.563958in}}%
\pgfpathlineto{\pgfqpoint{2.067684in}{3.194653in}}%
\pgfpathlineto{\pgfqpoint{2.247415in}{2.992969in}}%
\pgfpathlineto{\pgfqpoint{2.427146in}{2.672533in}}%
\pgfpathlineto{\pgfqpoint{2.606877in}{2.406884in}}%
\pgfpathlineto{\pgfqpoint{2.786608in}{2.273233in}}%
\pgfpathlineto{\pgfqpoint{2.966339in}{1.935697in}}%
\pgfpathlineto{\pgfqpoint{3.146070in}{1.472835in}}%
\pgfpathlineto{\pgfqpoint{3.325801in}{0.914585in}}%
\pgfpathlineto{\pgfqpoint{3.505532in}{0.808922in}}%
\pgfpathlineto{\pgfqpoint{3.685263in}{0.784725in}}%
\pgfpathlineto{\pgfqpoint{3.864994in}{0.777092in}}%
\pgfpathlineto{\pgfqpoint{4.044725in}{0.782215in}}%
\pgfpathlineto{\pgfqpoint{4.224456in}{0.762887in}}%
\pgfpathlineto{\pgfqpoint{4.404187in}{0.735836in}}%
\pgfpathlineto{\pgfqpoint{4.583918in}{0.743051in}}%
\pgfpathlineto{\pgfqpoint{4.763648in}{0.742334in}}%
\pgfpathlineto{\pgfqpoint{4.943379in}{0.731923in}}%
\pgfpathlineto{\pgfqpoint{5.123110in}{0.737465in}}%
\pgfpathlineto{\pgfqpoint{5.302841in}{0.734134in}}%
\pgfpathlineto{\pgfqpoint{5.482572in}{0.732909in}}%
\pgfpathlineto{\pgfqpoint{5.662303in}{0.729443in}}%
\pgfpathlineto{\pgfqpoint{5.842034in}{0.725948in}}%
\pgfpathlineto{\pgfqpoint{6.021765in}{0.726098in}}%
\pgfpathlineto{\pgfqpoint{6.201496in}{0.724440in}}%
\pgfpathlineto{\pgfqpoint{6.381227in}{0.726830in}}%
\pgfpathlineto{\pgfqpoint{6.560958in}{0.723230in}}%
\pgfpathlineto{\pgfqpoint{6.740689in}{0.724664in}}%
\pgfpathlineto{\pgfqpoint{6.920420in}{0.720870in}}%
\pgfpathlineto{\pgfqpoint{7.100151in}{0.720586in}}%
\pgfpathlineto{\pgfqpoint{7.279882in}{0.720750in}}%
\pgfpathlineto{\pgfqpoint{7.459613in}{0.721139in}}%
\pgfpathlineto{\pgfqpoint{7.639344in}{0.719003in}}%
\pgfpathlineto{\pgfqpoint{7.819075in}{0.719122in}}%
\pgfpathlineto{\pgfqpoint{7.998806in}{0.717852in}}%
\pgfpathlineto{\pgfqpoint{8.178537in}{0.719167in}}%
\pgfpathlineto{\pgfqpoint{8.358268in}{0.717494in}}%
\pgfpathlineto{\pgfqpoint{8.537999in}{0.716643in}}%
\pgfpathlineto{\pgfqpoint{8.717730in}{0.716404in}}%
\pgfpathlineto{\pgfqpoint{8.897461in}{0.716090in}}%
\pgfpathlineto{\pgfqpoint{9.077192in}{0.716344in}}%
\pgfpathlineto{\pgfqpoint{9.256923in}{0.715089in}}%
\pgfpathlineto{\pgfqpoint{9.436654in}{0.715283in}}%
\pgfpathlineto{\pgfqpoint{9.616385in}{0.714835in}}%
\pgfpathlineto{\pgfqpoint{9.796116in}{0.714566in}}%
\pgfpathlineto{\pgfqpoint{9.975847in}{0.714327in}}%
\pgfusepath{stroke}%
\end{pgfscope}%
\begin{pgfscope}%
\pgfpathrectangle{\pgfqpoint{0.728688in}{0.521603in}}{\pgfqpoint{9.687500in}{4.235000in}}%
\pgfusepath{clip}%
\pgfsetbuttcap%
\pgfsetroundjoin%
\pgfsetlinewidth{1.003750pt}%
\definecolor{currentstroke}{rgb}{0.239216,0.478431,0.992157}%
\pgfsetstrokecolor{currentstroke}%
\pgfsetstrokeopacity{0.500000}%
\pgfsetdash{{3.700000pt}{1.600000pt}}{0.000000pt}%
\pgfpathmoveto{\pgfqpoint{1.169029in}{4.508214in}}%
\pgfpathlineto{\pgfqpoint{1.348760in}{4.350030in}}%
\pgfpathlineto{\pgfqpoint{1.528491in}{4.111408in}}%
\pgfpathlineto{\pgfqpoint{1.708222in}{3.898727in}}%
\pgfpathlineto{\pgfqpoint{1.887953in}{3.717998in}}%
\pgfpathlineto{\pgfqpoint{2.067684in}{3.402746in}}%
\pgfpathlineto{\pgfqpoint{2.247415in}{3.179113in}}%
\pgfpathlineto{\pgfqpoint{2.427146in}{2.843213in}}%
\pgfpathlineto{\pgfqpoint{2.606877in}{2.673695in}}%
\pgfpathlineto{\pgfqpoint{2.786608in}{2.320542in}}%
\pgfpathlineto{\pgfqpoint{2.966339in}{1.944972in}}%
\pgfpathlineto{\pgfqpoint{3.146070in}{1.702614in}}%
\pgfpathlineto{\pgfqpoint{3.325801in}{1.103289in}}%
\pgfpathlineto{\pgfqpoint{3.505532in}{1.007478in}}%
\pgfpathlineto{\pgfqpoint{3.685263in}{0.803504in}}%
\pgfpathlineto{\pgfqpoint{3.864994in}{0.846261in}}%
\pgfpathlineto{\pgfqpoint{4.044725in}{0.763596in}}%
\pgfpathlineto{\pgfqpoint{4.224456in}{0.764336in}}%
\pgfpathlineto{\pgfqpoint{4.404187in}{0.759059in}}%
\pgfpathlineto{\pgfqpoint{4.583918in}{0.750605in}}%
\pgfpathlineto{\pgfqpoint{4.763648in}{0.748484in}}%
\pgfpathlineto{\pgfqpoint{4.943379in}{0.741053in}}%
\pgfpathlineto{\pgfqpoint{5.123110in}{0.734007in}}%
\pgfpathlineto{\pgfqpoint{5.302841in}{0.734862in}}%
\pgfpathlineto{\pgfqpoint{5.482572in}{0.729369in}}%
\pgfpathlineto{\pgfqpoint{5.662303in}{0.731867in}}%
\pgfpathlineto{\pgfqpoint{5.842034in}{0.729111in}}%
\pgfpathlineto{\pgfqpoint{6.021765in}{0.727804in}}%
\pgfpathlineto{\pgfqpoint{6.201496in}{0.726949in}}%
\pgfpathlineto{\pgfqpoint{6.381227in}{0.725582in}}%
\pgfpathlineto{\pgfqpoint{6.560958in}{0.724895in}}%
\pgfpathlineto{\pgfqpoint{6.740689in}{0.721590in}}%
\pgfpathlineto{\pgfqpoint{6.920420in}{0.722128in}}%
\pgfpathlineto{\pgfqpoint{7.100151in}{0.722307in}}%
\pgfpathlineto{\pgfqpoint{7.279882in}{0.720369in}}%
\pgfpathlineto{\pgfqpoint{7.459613in}{0.719708in}}%
\pgfpathlineto{\pgfqpoint{7.639344in}{0.719215in}}%
\pgfpathlineto{\pgfqpoint{7.819075in}{0.718719in}}%
\pgfpathlineto{\pgfqpoint{7.998806in}{0.718327in}}%
\pgfpathlineto{\pgfqpoint{8.178537in}{0.717572in}}%
\pgfpathlineto{\pgfqpoint{8.358268in}{0.717240in}}%
\pgfpathlineto{\pgfqpoint{8.537999in}{0.717341in}}%
\pgfpathlineto{\pgfqpoint{8.717730in}{0.716687in}}%
\pgfpathlineto{\pgfqpoint{8.897461in}{0.716187in}}%
\pgfpathlineto{\pgfqpoint{9.077192in}{0.715814in}}%
\pgfpathlineto{\pgfqpoint{9.256923in}{0.715227in}}%
\pgfpathlineto{\pgfqpoint{9.436654in}{0.715160in}}%
\pgfpathlineto{\pgfqpoint{9.616385in}{0.714772in}}%
\pgfpathlineto{\pgfqpoint{9.796116in}{0.714525in}}%
\pgfpathlineto{\pgfqpoint{9.975847in}{0.714305in}}%
\pgfusepath{stroke}%
\end{pgfscope}%
\begin{pgfscope}%
\pgfpathrectangle{\pgfqpoint{0.728688in}{0.521603in}}{\pgfqpoint{9.687500in}{4.235000in}}%
\pgfusepath{clip}%
\pgfsetbuttcap%
\pgfsetroundjoin%
\pgfsetlinewidth{1.003750pt}%
\definecolor{currentstroke}{rgb}{0.000000,0.000000,0.000000}%
\pgfsetstrokecolor{currentstroke}%
\pgfsetstrokeopacity{0.500000}%
\pgfsetdash{{3.700000pt}{1.600000pt}}{0.000000pt}%
\pgfpathmoveto{\pgfqpoint{1.169029in}{4.564103in}}%
\pgfpathlineto{\pgfqpoint{1.348760in}{4.375359in}}%
\pgfpathlineto{\pgfqpoint{1.528491in}{4.185422in}}%
\pgfpathlineto{\pgfqpoint{1.708222in}{3.983019in}}%
\pgfpathlineto{\pgfqpoint{1.887953in}{3.760243in}}%
\pgfpathlineto{\pgfqpoint{2.067684in}{3.524466in}}%
\pgfpathlineto{\pgfqpoint{2.247415in}{3.349319in}}%
\pgfpathlineto{\pgfqpoint{2.427146in}{3.014918in}}%
\pgfpathlineto{\pgfqpoint{2.606877in}{2.783607in}}%
\pgfpathlineto{\pgfqpoint{2.786608in}{2.489402in}}%
\pgfpathlineto{\pgfqpoint{2.966339in}{2.053102in}}%
\pgfpathlineto{\pgfqpoint{3.146070in}{1.735807in}}%
\pgfpathlineto{\pgfqpoint{3.325801in}{1.436998in}}%
\pgfpathlineto{\pgfqpoint{3.505532in}{1.112781in}}%
\pgfpathlineto{\pgfqpoint{3.685263in}{0.883268in}}%
\pgfpathlineto{\pgfqpoint{3.864994in}{0.821761in}}%
\pgfpathlineto{\pgfqpoint{4.044725in}{0.778708in}}%
\pgfpathlineto{\pgfqpoint{4.224456in}{0.766316in}}%
\pgfpathlineto{\pgfqpoint{4.404187in}{0.756322in}}%
\pgfpathlineto{\pgfqpoint{4.583918in}{0.752370in}}%
\pgfpathlineto{\pgfqpoint{4.763648in}{0.747130in}}%
\pgfpathlineto{\pgfqpoint{4.943379in}{0.741988in}}%
\pgfpathlineto{\pgfqpoint{5.123110in}{0.736495in}}%
\pgfpathlineto{\pgfqpoint{5.302841in}{0.737631in}}%
\pgfpathlineto{\pgfqpoint{5.482572in}{0.731761in}}%
\pgfpathlineto{\pgfqpoint{5.662303in}{0.732099in}}%
\pgfpathlineto{\pgfqpoint{5.842034in}{0.728507in}}%
\pgfpathlineto{\pgfqpoint{6.021765in}{0.728153in}}%
\pgfpathlineto{\pgfqpoint{6.201496in}{0.726730in}}%
\pgfpathlineto{\pgfqpoint{6.381227in}{0.724779in}}%
\pgfpathlineto{\pgfqpoint{6.560958in}{0.723371in}}%
\pgfpathlineto{\pgfqpoint{6.740689in}{0.722704in}}%
\pgfpathlineto{\pgfqpoint{6.920420in}{0.721829in}}%
\pgfpathlineto{\pgfqpoint{7.100151in}{0.721662in}}%
\pgfpathlineto{\pgfqpoint{7.279882in}{0.720446in}}%
\pgfpathlineto{\pgfqpoint{7.459613in}{0.719978in}}%
\pgfpathlineto{\pgfqpoint{7.639344in}{0.719070in}}%
\pgfpathlineto{\pgfqpoint{7.819075in}{0.719093in}}%
\pgfpathlineto{\pgfqpoint{7.998806in}{0.718690in}}%
\pgfpathlineto{\pgfqpoint{8.178537in}{0.717470in}}%
\pgfpathlineto{\pgfqpoint{8.358268in}{0.717466in}}%
\pgfpathlineto{\pgfqpoint{8.537999in}{0.717053in}}%
\pgfpathlineto{\pgfqpoint{8.717730in}{0.716508in}}%
\pgfpathlineto{\pgfqpoint{8.897461in}{0.715971in}}%
\pgfpathlineto{\pgfqpoint{9.077192in}{0.715778in}}%
\pgfpathlineto{\pgfqpoint{9.256923in}{0.715462in}}%
\pgfpathlineto{\pgfqpoint{9.436654in}{0.715133in}}%
\pgfpathlineto{\pgfqpoint{9.616385in}{0.714836in}}%
\pgfpathlineto{\pgfqpoint{9.796116in}{0.714465in}}%
\pgfpathlineto{\pgfqpoint{9.975847in}{0.714263in}}%
\pgfusepath{stroke}%
\end{pgfscope}%
\begin{pgfscope}%
\pgfsetrectcap%
\pgfsetmiterjoin%
\pgfsetlinewidth{0.803000pt}%
\definecolor{currentstroke}{rgb}{0.000000,0.000000,0.000000}%
\pgfsetstrokecolor{currentstroke}%
\pgfsetdash{}{0pt}%
\pgfpathmoveto{\pgfqpoint{0.728688in}{0.521603in}}%
\pgfpathlineto{\pgfqpoint{0.728688in}{4.756603in}}%
\pgfusepath{stroke}%
\end{pgfscope}%
\begin{pgfscope}%
\pgfsetrectcap%
\pgfsetmiterjoin%
\pgfsetlinewidth{0.803000pt}%
\definecolor{currentstroke}{rgb}{0.000000,0.000000,0.000000}%
\pgfsetstrokecolor{currentstroke}%
\pgfsetdash{}{0pt}%
\pgfpathmoveto{\pgfqpoint{10.416188in}{0.521603in}}%
\pgfpathlineto{\pgfqpoint{10.416188in}{4.756603in}}%
\pgfusepath{stroke}%
\end{pgfscope}%
\begin{pgfscope}%
\pgfsetrectcap%
\pgfsetmiterjoin%
\pgfsetlinewidth{0.803000pt}%
\definecolor{currentstroke}{rgb}{0.000000,0.000000,0.000000}%
\pgfsetstrokecolor{currentstroke}%
\pgfsetdash{}{0pt}%
\pgfpathmoveto{\pgfqpoint{0.728688in}{0.521603in}}%
\pgfpathlineto{\pgfqpoint{10.416188in}{0.521603in}}%
\pgfusepath{stroke}%
\end{pgfscope}%
\begin{pgfscope}%
\pgfsetrectcap%
\pgfsetmiterjoin%
\pgfsetlinewidth{0.803000pt}%
\definecolor{currentstroke}{rgb}{0.000000,0.000000,0.000000}%
\pgfsetstrokecolor{currentstroke}%
\pgfsetdash{}{0pt}%
\pgfpathmoveto{\pgfqpoint{0.728688in}{4.756603in}}%
\pgfpathlineto{\pgfqpoint{10.416188in}{4.756603in}}%
\pgfusepath{stroke}%
\end{pgfscope}%
\begin{pgfscope}%
\definecolor{textcolor}{rgb}{0.000000,0.000000,0.000000}%
\pgfsetstrokecolor{textcolor}%
\pgfsetfillcolor{textcolor}%
\pgftext[x=5.572438in,y=4.839937in,,base]{\color{textcolor}\sffamily\fontsize{12.000000}{14.400000}\selectfont \(\displaystyle \overline{\langle I\rangle}\) over \(\displaystyle p_4\) for \(\displaystyle T=1000\) with \(\displaystyle p_1=p_2=p_3=0.5\)}%
\end{pgfscope}%
\begin{pgfscope}%
\pgfsetbuttcap%
\pgfsetmiterjoin%
\definecolor{currentfill}{rgb}{1.000000,1.000000,1.000000}%
\pgfsetfillcolor{currentfill}%
\pgfsetfillopacity{0.800000}%
\pgfsetlinewidth{1.003750pt}%
\definecolor{currentstroke}{rgb}{0.800000,0.800000,0.800000}%
\pgfsetstrokecolor{currentstroke}%
\pgfsetstrokeopacity{0.800000}%
\pgfsetdash{}{0pt}%
\pgfpathmoveto{\pgfqpoint{9.386482in}{3.830063in}}%
\pgfpathlineto{\pgfqpoint{10.318966in}{3.830063in}}%
\pgfpathquadraticcurveto{\pgfqpoint{10.346743in}{3.830063in}}{\pgfqpoint{10.346743in}{3.857841in}}%
\pgfpathlineto{\pgfqpoint{10.346743in}{4.659381in}}%
\pgfpathquadraticcurveto{\pgfqpoint{10.346743in}{4.687159in}}{\pgfqpoint{10.318966in}{4.687159in}}%
\pgfpathlineto{\pgfqpoint{9.386482in}{4.687159in}}%
\pgfpathquadraticcurveto{\pgfqpoint{9.358704in}{4.687159in}}{\pgfqpoint{9.358704in}{4.659381in}}%
\pgfpathlineto{\pgfqpoint{9.358704in}{3.857841in}}%
\pgfpathquadraticcurveto{\pgfqpoint{9.358704in}{3.830063in}}{\pgfqpoint{9.386482in}{3.830063in}}%
\pgfpathlineto{\pgfqpoint{9.386482in}{3.830063in}}%
\pgfpathclose%
\pgfusepath{stroke,fill}%
\end{pgfscope}%
\begin{pgfscope}%
\pgfsetbuttcap%
\pgfsetroundjoin%
\definecolor{currentfill}{rgb}{0.000000,0.000000,1.000000}%
\pgfsetfillcolor{currentfill}%
\pgfsetfillopacity{0.500000}%
\pgfsetlinewidth{1.003750pt}%
\definecolor{currentstroke}{rgb}{0.000000,0.000000,1.000000}%
\pgfsetstrokecolor{currentstroke}%
\pgfsetstrokeopacity{0.500000}%
\pgfsetdash{}{0pt}%
\pgfsys@defobject{currentmarker}{\pgfqpoint{-0.021960in}{-0.021960in}}{\pgfqpoint{0.021960in}{0.021960in}}{%
\pgfpathmoveto{\pgfqpoint{0.000000in}{-0.021960in}}%
\pgfpathcurveto{\pgfqpoint{0.005824in}{-0.021960in}}{\pgfqpoint{0.011410in}{-0.019646in}}{\pgfqpoint{0.015528in}{-0.015528in}}%
\pgfpathcurveto{\pgfqpoint{0.019646in}{-0.011410in}}{\pgfqpoint{0.021960in}{-0.005824in}}{\pgfqpoint{0.021960in}{0.000000in}}%
\pgfpathcurveto{\pgfqpoint{0.021960in}{0.005824in}}{\pgfqpoint{0.019646in}{0.011410in}}{\pgfqpoint{0.015528in}{0.015528in}}%
\pgfpathcurveto{\pgfqpoint{0.011410in}{0.019646in}}{\pgfqpoint{0.005824in}{0.021960in}}{\pgfqpoint{0.000000in}{0.021960in}}%
\pgfpathcurveto{\pgfqpoint{-0.005824in}{0.021960in}}{\pgfqpoint{-0.011410in}{0.019646in}}{\pgfqpoint{-0.015528in}{0.015528in}}%
\pgfpathcurveto{\pgfqpoint{-0.019646in}{0.011410in}}{\pgfqpoint{-0.021960in}{0.005824in}}{\pgfqpoint{-0.021960in}{0.000000in}}%
\pgfpathcurveto{\pgfqpoint{-0.021960in}{-0.005824in}}{\pgfqpoint{-0.019646in}{-0.011410in}}{\pgfqpoint{-0.015528in}{-0.015528in}}%
\pgfpathcurveto{\pgfqpoint{-0.011410in}{-0.019646in}}{\pgfqpoint{-0.005824in}{-0.021960in}}{\pgfqpoint{0.000000in}{-0.021960in}}%
\pgfpathlineto{\pgfqpoint{0.000000in}{-0.021960in}}%
\pgfpathclose%
\pgfusepath{stroke,fill}%
}%
\begin{pgfscope}%
\pgfsys@transformshift{9.553148in}{4.562539in}%
\pgfsys@useobject{currentmarker}{}%
\end{pgfscope}%
\end{pgfscope}%
\begin{pgfscope}%
\definecolor{textcolor}{rgb}{0.000000,0.000000,0.000000}%
\pgfsetstrokecolor{textcolor}%
\pgfsetfillcolor{textcolor}%
\pgftext[x=9.803148in,y=4.526080in,left,base]{\color{textcolor}\sffamily\fontsize{10.000000}{12.000000}\selectfont \(\displaystyle L=16\)}%
\end{pgfscope}%
\begin{pgfscope}%
\pgfsetbuttcap%
\pgfsetroundjoin%
\definecolor{currentfill}{rgb}{0.980392,0.164706,0.333333}%
\pgfsetfillcolor{currentfill}%
\pgfsetfillopacity{0.500000}%
\pgfsetlinewidth{1.003750pt}%
\definecolor{currentstroke}{rgb}{0.980392,0.164706,0.333333}%
\pgfsetstrokecolor{currentstroke}%
\pgfsetstrokeopacity{0.500000}%
\pgfsetdash{}{0pt}%
\pgfsys@defobject{currentmarker}{\pgfqpoint{-0.021960in}{-0.021960in}}{\pgfqpoint{0.021960in}{0.021960in}}{%
\pgfpathmoveto{\pgfqpoint{0.000000in}{-0.021960in}}%
\pgfpathcurveto{\pgfqpoint{0.005824in}{-0.021960in}}{\pgfqpoint{0.011410in}{-0.019646in}}{\pgfqpoint{0.015528in}{-0.015528in}}%
\pgfpathcurveto{\pgfqpoint{0.019646in}{-0.011410in}}{\pgfqpoint{0.021960in}{-0.005824in}}{\pgfqpoint{0.021960in}{0.000000in}}%
\pgfpathcurveto{\pgfqpoint{0.021960in}{0.005824in}}{\pgfqpoint{0.019646in}{0.011410in}}{\pgfqpoint{0.015528in}{0.015528in}}%
\pgfpathcurveto{\pgfqpoint{0.011410in}{0.019646in}}{\pgfqpoint{0.005824in}{0.021960in}}{\pgfqpoint{0.000000in}{0.021960in}}%
\pgfpathcurveto{\pgfqpoint{-0.005824in}{0.021960in}}{\pgfqpoint{-0.011410in}{0.019646in}}{\pgfqpoint{-0.015528in}{0.015528in}}%
\pgfpathcurveto{\pgfqpoint{-0.019646in}{0.011410in}}{\pgfqpoint{-0.021960in}{0.005824in}}{\pgfqpoint{-0.021960in}{0.000000in}}%
\pgfpathcurveto{\pgfqpoint{-0.021960in}{-0.005824in}}{\pgfqpoint{-0.019646in}{-0.011410in}}{\pgfqpoint{-0.015528in}{-0.015528in}}%
\pgfpathcurveto{\pgfqpoint{-0.011410in}{-0.019646in}}{\pgfqpoint{-0.005824in}{-0.021960in}}{\pgfqpoint{0.000000in}{-0.021960in}}%
\pgfpathlineto{\pgfqpoint{0.000000in}{-0.021960in}}%
\pgfpathclose%
\pgfusepath{stroke,fill}%
}%
\begin{pgfscope}%
\pgfsys@transformshift{9.553148in}{4.358681in}%
\pgfsys@useobject{currentmarker}{}%
\end{pgfscope}%
\end{pgfscope}%
\begin{pgfscope}%
\definecolor{textcolor}{rgb}{0.000000,0.000000,0.000000}%
\pgfsetstrokecolor{textcolor}%
\pgfsetfillcolor{textcolor}%
\pgftext[x=9.803148in,y=4.322223in,left,base]{\color{textcolor}\sffamily\fontsize{10.000000}{12.000000}\selectfont \(\displaystyle L=32\)}%
\end{pgfscope}%
\begin{pgfscope}%
\pgfsetbuttcap%
\pgfsetroundjoin%
\definecolor{currentfill}{rgb}{0.239216,0.478431,0.992157}%
\pgfsetfillcolor{currentfill}%
\pgfsetfillopacity{0.500000}%
\pgfsetlinewidth{1.003750pt}%
\definecolor{currentstroke}{rgb}{0.239216,0.478431,0.992157}%
\pgfsetstrokecolor{currentstroke}%
\pgfsetstrokeopacity{0.500000}%
\pgfsetdash{}{0pt}%
\pgfsys@defobject{currentmarker}{\pgfqpoint{-0.021960in}{-0.021960in}}{\pgfqpoint{0.021960in}{0.021960in}}{%
\pgfpathmoveto{\pgfqpoint{0.000000in}{-0.021960in}}%
\pgfpathcurveto{\pgfqpoint{0.005824in}{-0.021960in}}{\pgfqpoint{0.011410in}{-0.019646in}}{\pgfqpoint{0.015528in}{-0.015528in}}%
\pgfpathcurveto{\pgfqpoint{0.019646in}{-0.011410in}}{\pgfqpoint{0.021960in}{-0.005824in}}{\pgfqpoint{0.021960in}{0.000000in}}%
\pgfpathcurveto{\pgfqpoint{0.021960in}{0.005824in}}{\pgfqpoint{0.019646in}{0.011410in}}{\pgfqpoint{0.015528in}{0.015528in}}%
\pgfpathcurveto{\pgfqpoint{0.011410in}{0.019646in}}{\pgfqpoint{0.005824in}{0.021960in}}{\pgfqpoint{0.000000in}{0.021960in}}%
\pgfpathcurveto{\pgfqpoint{-0.005824in}{0.021960in}}{\pgfqpoint{-0.011410in}{0.019646in}}{\pgfqpoint{-0.015528in}{0.015528in}}%
\pgfpathcurveto{\pgfqpoint{-0.019646in}{0.011410in}}{\pgfqpoint{-0.021960in}{0.005824in}}{\pgfqpoint{-0.021960in}{0.000000in}}%
\pgfpathcurveto{\pgfqpoint{-0.021960in}{-0.005824in}}{\pgfqpoint{-0.019646in}{-0.011410in}}{\pgfqpoint{-0.015528in}{-0.015528in}}%
\pgfpathcurveto{\pgfqpoint{-0.011410in}{-0.019646in}}{\pgfqpoint{-0.005824in}{-0.021960in}}{\pgfqpoint{0.000000in}{-0.021960in}}%
\pgfpathlineto{\pgfqpoint{0.000000in}{-0.021960in}}%
\pgfpathclose%
\pgfusepath{stroke,fill}%
}%
\begin{pgfscope}%
\pgfsys@transformshift{9.553148in}{4.154824in}%
\pgfsys@useobject{currentmarker}{}%
\end{pgfscope}%
\end{pgfscope}%
\begin{pgfscope}%
\definecolor{textcolor}{rgb}{0.000000,0.000000,0.000000}%
\pgfsetstrokecolor{textcolor}%
\pgfsetfillcolor{textcolor}%
\pgftext[x=9.803148in,y=4.118366in,left,base]{\color{textcolor}\sffamily\fontsize{10.000000}{12.000000}\selectfont \(\displaystyle L=64\)}%
\end{pgfscope}%
\begin{pgfscope}%
\pgfsetbuttcap%
\pgfsetroundjoin%
\definecolor{currentfill}{rgb}{0.000000,0.000000,0.000000}%
\pgfsetfillcolor{currentfill}%
\pgfsetfillopacity{0.500000}%
\pgfsetlinewidth{1.003750pt}%
\definecolor{currentstroke}{rgb}{0.000000,0.000000,0.000000}%
\pgfsetstrokecolor{currentstroke}%
\pgfsetstrokeopacity{0.500000}%
\pgfsetdash{}{0pt}%
\pgfsys@defobject{currentmarker}{\pgfqpoint{-0.021960in}{-0.021960in}}{\pgfqpoint{0.021960in}{0.021960in}}{%
\pgfpathmoveto{\pgfqpoint{0.000000in}{-0.021960in}}%
\pgfpathcurveto{\pgfqpoint{0.005824in}{-0.021960in}}{\pgfqpoint{0.011410in}{-0.019646in}}{\pgfqpoint{0.015528in}{-0.015528in}}%
\pgfpathcurveto{\pgfqpoint{0.019646in}{-0.011410in}}{\pgfqpoint{0.021960in}{-0.005824in}}{\pgfqpoint{0.021960in}{0.000000in}}%
\pgfpathcurveto{\pgfqpoint{0.021960in}{0.005824in}}{\pgfqpoint{0.019646in}{0.011410in}}{\pgfqpoint{0.015528in}{0.015528in}}%
\pgfpathcurveto{\pgfqpoint{0.011410in}{0.019646in}}{\pgfqpoint{0.005824in}{0.021960in}}{\pgfqpoint{0.000000in}{0.021960in}}%
\pgfpathcurveto{\pgfqpoint{-0.005824in}{0.021960in}}{\pgfqpoint{-0.011410in}{0.019646in}}{\pgfqpoint{-0.015528in}{0.015528in}}%
\pgfpathcurveto{\pgfqpoint{-0.019646in}{0.011410in}}{\pgfqpoint{-0.021960in}{0.005824in}}{\pgfqpoint{-0.021960in}{0.000000in}}%
\pgfpathcurveto{\pgfqpoint{-0.021960in}{-0.005824in}}{\pgfqpoint{-0.019646in}{-0.011410in}}{\pgfqpoint{-0.015528in}{-0.015528in}}%
\pgfpathcurveto{\pgfqpoint{-0.011410in}{-0.019646in}}{\pgfqpoint{-0.005824in}{-0.021960in}}{\pgfqpoint{0.000000in}{-0.021960in}}%
\pgfpathlineto{\pgfqpoint{0.000000in}{-0.021960in}}%
\pgfpathclose%
\pgfusepath{stroke,fill}%
}%
\begin{pgfscope}%
\pgfsys@transformshift{9.553148in}{3.950967in}%
\pgfsys@useobject{currentmarker}{}%
\end{pgfscope}%
\end{pgfscope}%
\begin{pgfscope}%
\definecolor{textcolor}{rgb}{0.000000,0.000000,0.000000}%
\pgfsetstrokecolor{textcolor}%
\pgfsetfillcolor{textcolor}%
\pgftext[x=9.803148in,y=3.914509in,left,base]{\color{textcolor}\sffamily\fontsize{10.000000}{12.000000}\selectfont \(\displaystyle L=128\)}%
\end{pgfscope}%
\end{pgfpicture}%
\makeatother%
\endgroup%
}
    \caption{Display of the time-averaged infection rates $\overline{\langle I\rangle}$ depending on the vaccination rate $p_4$ at the beginning of the simulation and constant turnover probabilities
    $p_1=p_2=p_3=0.5$. The simulation was performed over $T=1000$ simulation steps and th respective grid sizes $L=16$, $L=32$, $L=64$ and $L=128$.}\label{fig:Res_Dis_Avg_Inf_over_p4}
\end{figure}

\subsection{Time Evolution of the Expected Ratio of Infected People}

While previously the average of the ratio of infected individuals has been taken over time, the focus should now be layed on the time development of th einfection rate $\langle I\rangle_t$ for $N=20$ samples.
As grid size $L=64$ was chosen for appropriate balance between running time and accuracy, the number of simulation steps again was set to $T=1000$.


\begin{figure}[ht]
    \centering
    \resizebox{\textwidth}{!}{%% Creator: Matplotlib, PGF backend
%%
%% To include the figure in your LaTeX document, write
%%   \input{<filename>.pgf}
%%
%% Make sure the required packages are loaded in your preamble
%%   \usepackage{pgf}
%%
%% Also ensure that all the required font packages are loaded; for instance,
%% the lmodern package is sometimes necessary when using math font.
%%   \usepackage{lmodern}
%%
%% Figures using additional raster images can only be included by \input if
%% they are in the same directory as the main LaTeX file. For loading figures
%% from other directories you can use the `import` package
%%   \usepackage{import}
%%
%% and then include the figures with
%%   \import{<path to file>}{<filename>.pgf}
%%
%% Matplotlib used the following preamble
%%   
%%   \usepackage{fontspec}
%%   \setmainfont{DejaVuSerif.ttf}[Path=\detokenize{/home/carlo/.local/lib/python3.10/site-packages/matplotlib/mpl-data/fonts/ttf/}]
%%   \setsansfont{DejaVuSans.ttf}[Path=\detokenize{/home/carlo/.local/lib/python3.10/site-packages/matplotlib/mpl-data/fonts/ttf/}]
%%   \setmonofont{DejaVuSansMono.ttf}[Path=\detokenize{/home/carlo/.local/lib/python3.10/site-packages/matplotlib/mpl-data/fonts/ttf/}]
%%   \makeatletter\@ifpackageloaded{underscore}{}{\usepackage[strings]{underscore}}\makeatother
%%
\begingroup%
\makeatletter%
\begin{pgfpicture}%
\pgfpathrectangle{\pgfpointorigin}{\pgfqpoint{10.489768in}{5.066564in}}%
\pgfusepath{use as bounding box, clip}%
\begin{pgfscope}%
\pgfsetbuttcap%
\pgfsetmiterjoin%
\definecolor{currentfill}{rgb}{1.000000,1.000000,1.000000}%
\pgfsetfillcolor{currentfill}%
\pgfsetlinewidth{0.000000pt}%
\definecolor{currentstroke}{rgb}{1.000000,1.000000,1.000000}%
\pgfsetstrokecolor{currentstroke}%
\pgfsetdash{}{0pt}%
\pgfpathmoveto{\pgfqpoint{0.000000in}{0.000000in}}%
\pgfpathlineto{\pgfqpoint{10.489768in}{0.000000in}}%
\pgfpathlineto{\pgfqpoint{10.489768in}{5.066564in}}%
\pgfpathlineto{\pgfqpoint{0.000000in}{5.066564in}}%
\pgfpathlineto{\pgfqpoint{0.000000in}{0.000000in}}%
\pgfpathclose%
\pgfusepath{fill}%
\end{pgfscope}%
\begin{pgfscope}%
\pgfsetbuttcap%
\pgfsetmiterjoin%
\definecolor{currentfill}{rgb}{1.000000,1.000000,1.000000}%
\pgfsetfillcolor{currentfill}%
\pgfsetlinewidth{0.000000pt}%
\definecolor{currentstroke}{rgb}{0.000000,0.000000,0.000000}%
\pgfsetstrokecolor{currentstroke}%
\pgfsetstrokeopacity{0.000000}%
\pgfsetdash{}{0pt}%
\pgfpathmoveto{\pgfqpoint{0.702268in}{0.521603in}}%
\pgfpathlineto{\pgfqpoint{10.389768in}{0.521603in}}%
\pgfpathlineto{\pgfqpoint{10.389768in}{4.756603in}}%
\pgfpathlineto{\pgfqpoint{0.702268in}{4.756603in}}%
\pgfpathlineto{\pgfqpoint{0.702268in}{0.521603in}}%
\pgfpathclose%
\pgfusepath{fill}%
\end{pgfscope}%
\begin{pgfscope}%
\pgfpathrectangle{\pgfqpoint{0.702268in}{0.521603in}}{\pgfqpoint{9.687500in}{4.235000in}}%
\pgfusepath{clip}%
\pgfsetbuttcap%
\pgfsetroundjoin%
\definecolor{currentfill}{rgb}{0.980392,0.164706,0.333333}%
\pgfsetfillcolor{currentfill}%
\pgfsetfillopacity{0.300000}%
\pgfsetlinewidth{1.003750pt}%
\definecolor{currentstroke}{rgb}{0.980392,0.164706,0.333333}%
\pgfsetstrokecolor{currentstroke}%
\pgfsetstrokeopacity{0.300000}%
\pgfsetdash{}{0pt}%
\pgfsys@defobject{currentmarker}{\pgfqpoint{1.142609in}{1.343136in}}{\pgfqpoint{9.949427in}{4.442509in}}{%
\pgfpathmoveto{\pgfqpoint{1.142609in}{4.442509in}}%
\pgfpathlineto{\pgfqpoint{1.142609in}{4.118161in}}%
\pgfpathlineto{\pgfqpoint{1.151416in}{3.262805in}}%
\pgfpathlineto{\pgfqpoint{1.160222in}{2.654104in}}%
\pgfpathlineto{\pgfqpoint{1.169029in}{2.307562in}}%
\pgfpathlineto{\pgfqpoint{1.177836in}{2.078966in}}%
\pgfpathlineto{\pgfqpoint{1.186643in}{1.965450in}}%
\pgfpathlineto{\pgfqpoint{1.195450in}{1.879165in}}%
\pgfpathlineto{\pgfqpoint{1.204257in}{1.754553in}}%
\pgfpathlineto{\pgfqpoint{1.213063in}{1.681754in}}%
\pgfpathlineto{\pgfqpoint{1.221870in}{1.570521in}}%
\pgfpathlineto{\pgfqpoint{1.230677in}{1.622817in}}%
\pgfpathlineto{\pgfqpoint{1.239484in}{1.555334in}}%
\pgfpathlineto{\pgfqpoint{1.248291in}{1.572520in}}%
\pgfpathlineto{\pgfqpoint{1.257097in}{1.562968in}}%
\pgfpathlineto{\pgfqpoint{1.265904in}{1.656024in}}%
\pgfpathlineto{\pgfqpoint{1.274711in}{1.597245in}}%
\pgfpathlineto{\pgfqpoint{1.283518in}{1.586128in}}%
\pgfpathlineto{\pgfqpoint{1.292325in}{1.668647in}}%
\pgfpathlineto{\pgfqpoint{1.301132in}{1.627378in}}%
\pgfpathlineto{\pgfqpoint{1.309938in}{1.649726in}}%
\pgfpathlineto{\pgfqpoint{1.318745in}{1.725211in}}%
\pgfpathlineto{\pgfqpoint{1.327552in}{1.657012in}}%
\pgfpathlineto{\pgfqpoint{1.336359in}{1.620003in}}%
\pgfpathlineto{\pgfqpoint{1.345166in}{1.618311in}}%
\pgfpathlineto{\pgfqpoint{1.353972in}{1.648509in}}%
\pgfpathlineto{\pgfqpoint{1.362779in}{1.604022in}}%
\pgfpathlineto{\pgfqpoint{1.371586in}{1.580885in}}%
\pgfpathlineto{\pgfqpoint{1.380393in}{1.629417in}}%
\pgfpathlineto{\pgfqpoint{1.389200in}{1.627618in}}%
\pgfpathlineto{\pgfqpoint{1.398007in}{1.685250in}}%
\pgfpathlineto{\pgfqpoint{1.406813in}{1.751332in}}%
\pgfpathlineto{\pgfqpoint{1.415620in}{1.732783in}}%
\pgfpathlineto{\pgfqpoint{1.424427in}{1.639546in}}%
\pgfpathlineto{\pgfqpoint{1.433234in}{1.604922in}}%
\pgfpathlineto{\pgfqpoint{1.442041in}{1.617206in}}%
\pgfpathlineto{\pgfqpoint{1.450847in}{1.612931in}}%
\pgfpathlineto{\pgfqpoint{1.459654in}{1.656411in}}%
\pgfpathlineto{\pgfqpoint{1.468461in}{1.553917in}}%
\pgfpathlineto{\pgfqpoint{1.477268in}{1.563643in}}%
\pgfpathlineto{\pgfqpoint{1.486075in}{1.577906in}}%
\pgfpathlineto{\pgfqpoint{1.494882in}{1.702066in}}%
\pgfpathlineto{\pgfqpoint{1.503688in}{1.741787in}}%
\pgfpathlineto{\pgfqpoint{1.512495in}{1.700954in}}%
\pgfpathlineto{\pgfqpoint{1.521302in}{1.767039in}}%
\pgfpathlineto{\pgfqpoint{1.530109in}{1.789814in}}%
\pgfpathlineto{\pgfqpoint{1.538916in}{1.781212in}}%
\pgfpathlineto{\pgfqpoint{1.547722in}{1.732993in}}%
\pgfpathlineto{\pgfqpoint{1.556529in}{1.692672in}}%
\pgfpathlineto{\pgfqpoint{1.565336in}{1.584334in}}%
\pgfpathlineto{\pgfqpoint{1.574143in}{1.657018in}}%
\pgfpathlineto{\pgfqpoint{1.582950in}{1.625931in}}%
\pgfpathlineto{\pgfqpoint{1.591757in}{1.669628in}}%
\pgfpathlineto{\pgfqpoint{1.600563in}{1.588720in}}%
\pgfpathlineto{\pgfqpoint{1.609370in}{1.589880in}}%
\pgfpathlineto{\pgfqpoint{1.618177in}{1.550521in}}%
\pgfpathlineto{\pgfqpoint{1.626984in}{1.514029in}}%
\pgfpathlineto{\pgfqpoint{1.635791in}{1.492306in}}%
\pgfpathlineto{\pgfqpoint{1.644597in}{1.524241in}}%
\pgfpathlineto{\pgfqpoint{1.653404in}{1.517180in}}%
\pgfpathlineto{\pgfqpoint{1.662211in}{1.671275in}}%
\pgfpathlineto{\pgfqpoint{1.671018in}{1.645754in}}%
\pgfpathlineto{\pgfqpoint{1.679825in}{1.611337in}}%
\pgfpathlineto{\pgfqpoint{1.688632in}{1.626230in}}%
\pgfpathlineto{\pgfqpoint{1.697438in}{1.679357in}}%
\pgfpathlineto{\pgfqpoint{1.706245in}{1.757799in}}%
\pgfpathlineto{\pgfqpoint{1.715052in}{1.649659in}}%
\pgfpathlineto{\pgfqpoint{1.723859in}{1.553765in}}%
\pgfpathlineto{\pgfqpoint{1.732666in}{1.587958in}}%
\pgfpathlineto{\pgfqpoint{1.741472in}{1.589043in}}%
\pgfpathlineto{\pgfqpoint{1.750279in}{1.713114in}}%
\pgfpathlineto{\pgfqpoint{1.759086in}{1.660014in}}%
\pgfpathlineto{\pgfqpoint{1.767893in}{1.713343in}}%
\pgfpathlineto{\pgfqpoint{1.776700in}{1.658527in}}%
\pgfpathlineto{\pgfqpoint{1.785507in}{1.674441in}}%
\pgfpathlineto{\pgfqpoint{1.794313in}{1.696096in}}%
\pgfpathlineto{\pgfqpoint{1.803120in}{1.618863in}}%
\pgfpathlineto{\pgfqpoint{1.811927in}{1.615269in}}%
\pgfpathlineto{\pgfqpoint{1.820734in}{1.757368in}}%
\pgfpathlineto{\pgfqpoint{1.829541in}{1.698317in}}%
\pgfpathlineto{\pgfqpoint{1.838347in}{1.715083in}}%
\pgfpathlineto{\pgfqpoint{1.847154in}{1.648800in}}%
\pgfpathlineto{\pgfqpoint{1.855961in}{1.657577in}}%
\pgfpathlineto{\pgfqpoint{1.864768in}{1.643815in}}%
\pgfpathlineto{\pgfqpoint{1.873575in}{1.665195in}}%
\pgfpathlineto{\pgfqpoint{1.882382in}{1.657487in}}%
\pgfpathlineto{\pgfqpoint{1.891188in}{1.705041in}}%
\pgfpathlineto{\pgfqpoint{1.899995in}{1.615531in}}%
\pgfpathlineto{\pgfqpoint{1.908802in}{1.590985in}}%
\pgfpathlineto{\pgfqpoint{1.917609in}{1.594651in}}%
\pgfpathlineto{\pgfqpoint{1.926416in}{1.560930in}}%
\pgfpathlineto{\pgfqpoint{1.935222in}{1.519671in}}%
\pgfpathlineto{\pgfqpoint{1.944029in}{1.618893in}}%
\pgfpathlineto{\pgfqpoint{1.952836in}{1.571573in}}%
\pgfpathlineto{\pgfqpoint{1.961643in}{1.643747in}}%
\pgfpathlineto{\pgfqpoint{1.970450in}{1.755534in}}%
\pgfpathlineto{\pgfqpoint{1.979257in}{1.610999in}}%
\pgfpathlineto{\pgfqpoint{1.988063in}{1.704107in}}%
\pgfpathlineto{\pgfqpoint{1.996870in}{1.612334in}}%
\pgfpathlineto{\pgfqpoint{2.005677in}{1.660299in}}%
\pgfpathlineto{\pgfqpoint{2.014484in}{1.748173in}}%
\pgfpathlineto{\pgfqpoint{2.023291in}{1.610808in}}%
\pgfpathlineto{\pgfqpoint{2.032097in}{1.619788in}}%
\pgfpathlineto{\pgfqpoint{2.040904in}{1.650804in}}%
\pgfpathlineto{\pgfqpoint{2.049711in}{1.658436in}}%
\pgfpathlineto{\pgfqpoint{2.058518in}{1.602202in}}%
\pgfpathlineto{\pgfqpoint{2.067325in}{1.694253in}}%
\pgfpathlineto{\pgfqpoint{2.076132in}{1.702047in}}%
\pgfpathlineto{\pgfqpoint{2.084938in}{1.679275in}}%
\pgfpathlineto{\pgfqpoint{2.093745in}{1.629331in}}%
\pgfpathlineto{\pgfqpoint{2.102552in}{1.705559in}}%
\pgfpathlineto{\pgfqpoint{2.111359in}{1.619921in}}%
\pgfpathlineto{\pgfqpoint{2.120166in}{1.697937in}}%
\pgfpathlineto{\pgfqpoint{2.128972in}{1.639413in}}%
\pgfpathlineto{\pgfqpoint{2.137779in}{1.636405in}}%
\pgfpathlineto{\pgfqpoint{2.146586in}{1.611148in}}%
\pgfpathlineto{\pgfqpoint{2.155393in}{1.630024in}}%
\pgfpathlineto{\pgfqpoint{2.164200in}{1.571443in}}%
\pgfpathlineto{\pgfqpoint{2.173007in}{1.708770in}}%
\pgfpathlineto{\pgfqpoint{2.181813in}{1.669624in}}%
\pgfpathlineto{\pgfqpoint{2.190620in}{1.606803in}}%
\pgfpathlineto{\pgfqpoint{2.199427in}{1.637822in}}%
\pgfpathlineto{\pgfqpoint{2.208234in}{1.588846in}}%
\pgfpathlineto{\pgfqpoint{2.217041in}{1.590445in}}%
\pgfpathlineto{\pgfqpoint{2.225847in}{1.526150in}}%
\pgfpathlineto{\pgfqpoint{2.234654in}{1.541745in}}%
\pgfpathlineto{\pgfqpoint{2.243461in}{1.543390in}}%
\pgfpathlineto{\pgfqpoint{2.252268in}{1.692147in}}%
\pgfpathlineto{\pgfqpoint{2.261075in}{1.711714in}}%
\pgfpathlineto{\pgfqpoint{2.269882in}{1.847254in}}%
\pgfpathlineto{\pgfqpoint{2.278688in}{1.777441in}}%
\pgfpathlineto{\pgfqpoint{2.287495in}{1.722150in}}%
\pgfpathlineto{\pgfqpoint{2.296302in}{1.765495in}}%
\pgfpathlineto{\pgfqpoint{2.305109in}{1.699072in}}%
\pgfpathlineto{\pgfqpoint{2.313916in}{1.662104in}}%
\pgfpathlineto{\pgfqpoint{2.322722in}{1.608146in}}%
\pgfpathlineto{\pgfqpoint{2.331529in}{1.646211in}}%
\pgfpathlineto{\pgfqpoint{2.340336in}{1.661428in}}%
\pgfpathlineto{\pgfqpoint{2.349143in}{1.661049in}}%
\pgfpathlineto{\pgfqpoint{2.357950in}{1.597737in}}%
\pgfpathlineto{\pgfqpoint{2.366757in}{1.548930in}}%
\pgfpathlineto{\pgfqpoint{2.375563in}{1.592174in}}%
\pgfpathlineto{\pgfqpoint{2.384370in}{1.664667in}}%
\pgfpathlineto{\pgfqpoint{2.393177in}{1.615372in}}%
\pgfpathlineto{\pgfqpoint{2.401984in}{1.603403in}}%
\pgfpathlineto{\pgfqpoint{2.410791in}{1.538018in}}%
\pgfpathlineto{\pgfqpoint{2.419597in}{1.616618in}}%
\pgfpathlineto{\pgfqpoint{2.428404in}{1.604981in}}%
\pgfpathlineto{\pgfqpoint{2.437211in}{1.527227in}}%
\pgfpathlineto{\pgfqpoint{2.446018in}{1.560118in}}%
\pgfpathlineto{\pgfqpoint{2.454825in}{1.547066in}}%
\pgfpathlineto{\pgfqpoint{2.463632in}{1.652320in}}%
\pgfpathlineto{\pgfqpoint{2.472438in}{1.684850in}}%
\pgfpathlineto{\pgfqpoint{2.481245in}{1.738982in}}%
\pgfpathlineto{\pgfqpoint{2.490052in}{1.608846in}}%
\pgfpathlineto{\pgfqpoint{2.498859in}{1.647604in}}%
\pgfpathlineto{\pgfqpoint{2.507666in}{1.595131in}}%
\pgfpathlineto{\pgfqpoint{2.516472in}{1.662522in}}%
\pgfpathlineto{\pgfqpoint{2.525279in}{1.696361in}}%
\pgfpathlineto{\pgfqpoint{2.534086in}{1.692370in}}%
\pgfpathlineto{\pgfqpoint{2.542893in}{1.702115in}}%
\pgfpathlineto{\pgfqpoint{2.551700in}{1.708018in}}%
\pgfpathlineto{\pgfqpoint{2.560507in}{1.802293in}}%
\pgfpathlineto{\pgfqpoint{2.569313in}{1.734165in}}%
\pgfpathlineto{\pgfqpoint{2.578120in}{1.717772in}}%
\pgfpathlineto{\pgfqpoint{2.586927in}{1.749532in}}%
\pgfpathlineto{\pgfqpoint{2.595734in}{1.677412in}}%
\pgfpathlineto{\pgfqpoint{2.604541in}{1.637752in}}%
\pgfpathlineto{\pgfqpoint{2.613347in}{1.596703in}}%
\pgfpathlineto{\pgfqpoint{2.622154in}{1.619657in}}%
\pgfpathlineto{\pgfqpoint{2.630961in}{1.579679in}}%
\pgfpathlineto{\pgfqpoint{2.639768in}{1.550808in}}%
\pgfpathlineto{\pgfqpoint{2.648575in}{1.619876in}}%
\pgfpathlineto{\pgfqpoint{2.657382in}{1.655364in}}%
\pgfpathlineto{\pgfqpoint{2.666188in}{1.600543in}}%
\pgfpathlineto{\pgfqpoint{2.674995in}{1.649244in}}%
\pgfpathlineto{\pgfqpoint{2.683802in}{1.626028in}}%
\pgfpathlineto{\pgfqpoint{2.692609in}{1.602445in}}%
\pgfpathlineto{\pgfqpoint{2.701416in}{1.625800in}}%
\pgfpathlineto{\pgfqpoint{2.710222in}{1.644513in}}%
\pgfpathlineto{\pgfqpoint{2.719029in}{1.636556in}}%
\pgfpathlineto{\pgfqpoint{2.727836in}{1.729506in}}%
\pgfpathlineto{\pgfqpoint{2.736643in}{1.817076in}}%
\pgfpathlineto{\pgfqpoint{2.745450in}{1.767641in}}%
\pgfpathlineto{\pgfqpoint{2.754257in}{1.783528in}}%
\pgfpathlineto{\pgfqpoint{2.763063in}{1.719515in}}%
\pgfpathlineto{\pgfqpoint{2.771870in}{1.688441in}}%
\pgfpathlineto{\pgfqpoint{2.780677in}{1.656577in}}%
\pgfpathlineto{\pgfqpoint{2.789484in}{1.644475in}}%
\pgfpathlineto{\pgfqpoint{2.798291in}{1.693525in}}%
\pgfpathlineto{\pgfqpoint{2.807097in}{1.647430in}}%
\pgfpathlineto{\pgfqpoint{2.815904in}{1.692877in}}%
\pgfpathlineto{\pgfqpoint{2.824711in}{1.744598in}}%
\pgfpathlineto{\pgfqpoint{2.833518in}{1.572705in}}%
\pgfpathlineto{\pgfqpoint{2.842325in}{1.651379in}}%
\pgfpathlineto{\pgfqpoint{2.851132in}{1.784127in}}%
\pgfpathlineto{\pgfqpoint{2.859938in}{1.746734in}}%
\pgfpathlineto{\pgfqpoint{2.868745in}{1.593054in}}%
\pgfpathlineto{\pgfqpoint{2.877552in}{1.585262in}}%
\pgfpathlineto{\pgfqpoint{2.886359in}{1.559929in}}%
\pgfpathlineto{\pgfqpoint{2.895166in}{1.614211in}}%
\pgfpathlineto{\pgfqpoint{2.903972in}{1.689145in}}%
\pgfpathlineto{\pgfqpoint{2.912779in}{1.725921in}}%
\pgfpathlineto{\pgfqpoint{2.921586in}{1.692840in}}%
\pgfpathlineto{\pgfqpoint{2.930393in}{1.583258in}}%
\pgfpathlineto{\pgfqpoint{2.939200in}{1.608727in}}%
\pgfpathlineto{\pgfqpoint{2.948007in}{1.663773in}}%
\pgfpathlineto{\pgfqpoint{2.956813in}{1.730727in}}%
\pgfpathlineto{\pgfqpoint{2.965620in}{1.760865in}}%
\pgfpathlineto{\pgfqpoint{2.974427in}{1.812387in}}%
\pgfpathlineto{\pgfqpoint{2.983234in}{1.671521in}}%
\pgfpathlineto{\pgfqpoint{2.992041in}{1.653736in}}%
\pgfpathlineto{\pgfqpoint{3.000847in}{1.591969in}}%
\pgfpathlineto{\pgfqpoint{3.009654in}{1.604433in}}%
\pgfpathlineto{\pgfqpoint{3.018461in}{1.612989in}}%
\pgfpathlineto{\pgfqpoint{3.027268in}{1.625107in}}%
\pgfpathlineto{\pgfqpoint{3.036075in}{1.522493in}}%
\pgfpathlineto{\pgfqpoint{3.044882in}{1.534133in}}%
\pgfpathlineto{\pgfqpoint{3.053688in}{1.514296in}}%
\pgfpathlineto{\pgfqpoint{3.062495in}{1.519114in}}%
\pgfpathlineto{\pgfqpoint{3.071302in}{1.606142in}}%
\pgfpathlineto{\pgfqpoint{3.080109in}{1.595038in}}%
\pgfpathlineto{\pgfqpoint{3.088916in}{1.499999in}}%
\pgfpathlineto{\pgfqpoint{3.097722in}{1.525571in}}%
\pgfpathlineto{\pgfqpoint{3.106529in}{1.486240in}}%
\pgfpathlineto{\pgfqpoint{3.115336in}{1.540241in}}%
\pgfpathlineto{\pgfqpoint{3.124143in}{1.445832in}}%
\pgfpathlineto{\pgfqpoint{3.132950in}{1.448128in}}%
\pgfpathlineto{\pgfqpoint{3.141757in}{1.454945in}}%
\pgfpathlineto{\pgfqpoint{3.150563in}{1.495460in}}%
\pgfpathlineto{\pgfqpoint{3.159370in}{1.449346in}}%
\pgfpathlineto{\pgfqpoint{3.168177in}{1.530864in}}%
\pgfpathlineto{\pgfqpoint{3.176984in}{1.587677in}}%
\pgfpathlineto{\pgfqpoint{3.185791in}{1.640635in}}%
\pgfpathlineto{\pgfqpoint{3.194597in}{1.657338in}}%
\pgfpathlineto{\pgfqpoint{3.203404in}{1.693281in}}%
\pgfpathlineto{\pgfqpoint{3.212211in}{1.692648in}}%
\pgfpathlineto{\pgfqpoint{3.221018in}{1.664457in}}%
\pgfpathlineto{\pgfqpoint{3.229825in}{1.613453in}}%
\pgfpathlineto{\pgfqpoint{3.238632in}{1.731194in}}%
\pgfpathlineto{\pgfqpoint{3.247438in}{1.712012in}}%
\pgfpathlineto{\pgfqpoint{3.256245in}{1.592233in}}%
\pgfpathlineto{\pgfqpoint{3.265052in}{1.613659in}}%
\pgfpathlineto{\pgfqpoint{3.273859in}{1.532941in}}%
\pgfpathlineto{\pgfqpoint{3.282666in}{1.527842in}}%
\pgfpathlineto{\pgfqpoint{3.291472in}{1.539645in}}%
\pgfpathlineto{\pgfqpoint{3.300279in}{1.615174in}}%
\pgfpathlineto{\pgfqpoint{3.309086in}{1.668122in}}%
\pgfpathlineto{\pgfqpoint{3.317893in}{1.712369in}}%
\pgfpathlineto{\pgfqpoint{3.326700in}{1.663050in}}%
\pgfpathlineto{\pgfqpoint{3.335507in}{1.750702in}}%
\pgfpathlineto{\pgfqpoint{3.344313in}{1.817869in}}%
\pgfpathlineto{\pgfqpoint{3.353120in}{1.817418in}}%
\pgfpathlineto{\pgfqpoint{3.361927in}{1.711500in}}%
\pgfpathlineto{\pgfqpoint{3.370734in}{1.699242in}}%
\pgfpathlineto{\pgfqpoint{3.379541in}{1.751472in}}%
\pgfpathlineto{\pgfqpoint{3.388347in}{1.642586in}}%
\pgfpathlineto{\pgfqpoint{3.397154in}{1.645094in}}%
\pgfpathlineto{\pgfqpoint{3.405961in}{1.688962in}}%
\pgfpathlineto{\pgfqpoint{3.414768in}{1.667983in}}%
\pgfpathlineto{\pgfqpoint{3.423575in}{1.690867in}}%
\pgfpathlineto{\pgfqpoint{3.432382in}{1.625205in}}%
\pgfpathlineto{\pgfqpoint{3.441188in}{1.657107in}}%
\pgfpathlineto{\pgfqpoint{3.449995in}{1.573624in}}%
\pgfpathlineto{\pgfqpoint{3.458802in}{1.517217in}}%
\pgfpathlineto{\pgfqpoint{3.467609in}{1.598649in}}%
\pgfpathlineto{\pgfqpoint{3.476416in}{1.586761in}}%
\pgfpathlineto{\pgfqpoint{3.485222in}{1.614934in}}%
\pgfpathlineto{\pgfqpoint{3.494029in}{1.662323in}}%
\pgfpathlineto{\pgfqpoint{3.502836in}{1.621240in}}%
\pgfpathlineto{\pgfqpoint{3.511643in}{1.626538in}}%
\pgfpathlineto{\pgfqpoint{3.520450in}{1.550423in}}%
\pgfpathlineto{\pgfqpoint{3.529257in}{1.690702in}}%
\pgfpathlineto{\pgfqpoint{3.538063in}{1.614437in}}%
\pgfpathlineto{\pgfqpoint{3.546870in}{1.606803in}}%
\pgfpathlineto{\pgfqpoint{3.555677in}{1.672878in}}%
\pgfpathlineto{\pgfqpoint{3.564484in}{1.755574in}}%
\pgfpathlineto{\pgfqpoint{3.573291in}{1.737962in}}%
\pgfpathlineto{\pgfqpoint{3.582097in}{1.723582in}}%
\pgfpathlineto{\pgfqpoint{3.590904in}{1.595623in}}%
\pgfpathlineto{\pgfqpoint{3.599711in}{1.490810in}}%
\pgfpathlineto{\pgfqpoint{3.608518in}{1.527912in}}%
\pgfpathlineto{\pgfqpoint{3.617325in}{1.567516in}}%
\pgfpathlineto{\pgfqpoint{3.626132in}{1.717620in}}%
\pgfpathlineto{\pgfqpoint{3.634938in}{1.662139in}}%
\pgfpathlineto{\pgfqpoint{3.643745in}{1.732109in}}%
\pgfpathlineto{\pgfqpoint{3.652552in}{1.686668in}}%
\pgfpathlineto{\pgfqpoint{3.661359in}{1.722278in}}%
\pgfpathlineto{\pgfqpoint{3.670166in}{1.668724in}}%
\pgfpathlineto{\pgfqpoint{3.678972in}{1.622752in}}%
\pgfpathlineto{\pgfqpoint{3.687779in}{1.578884in}}%
\pgfpathlineto{\pgfqpoint{3.696586in}{1.720242in}}%
\pgfpathlineto{\pgfqpoint{3.705393in}{1.766736in}}%
\pgfpathlineto{\pgfqpoint{3.714200in}{1.682548in}}%
\pgfpathlineto{\pgfqpoint{3.723007in}{1.617847in}}%
\pgfpathlineto{\pgfqpoint{3.731813in}{1.732114in}}%
\pgfpathlineto{\pgfqpoint{3.740620in}{1.802278in}}%
\pgfpathlineto{\pgfqpoint{3.749427in}{1.725162in}}%
\pgfpathlineto{\pgfqpoint{3.758234in}{1.689819in}}%
\pgfpathlineto{\pgfqpoint{3.767041in}{1.701603in}}%
\pgfpathlineto{\pgfqpoint{3.775847in}{1.619834in}}%
\pgfpathlineto{\pgfqpoint{3.784654in}{1.596557in}}%
\pgfpathlineto{\pgfqpoint{3.793461in}{1.590221in}}%
\pgfpathlineto{\pgfqpoint{3.802268in}{1.593557in}}%
\pgfpathlineto{\pgfqpoint{3.811075in}{1.491954in}}%
\pgfpathlineto{\pgfqpoint{3.819882in}{1.560629in}}%
\pgfpathlineto{\pgfqpoint{3.828688in}{1.625501in}}%
\pgfpathlineto{\pgfqpoint{3.837495in}{1.628296in}}%
\pgfpathlineto{\pgfqpoint{3.846302in}{1.597520in}}%
\pgfpathlineto{\pgfqpoint{3.855109in}{1.608028in}}%
\pgfpathlineto{\pgfqpoint{3.863916in}{1.567260in}}%
\pgfpathlineto{\pgfqpoint{3.872722in}{1.574108in}}%
\pgfpathlineto{\pgfqpoint{3.881529in}{1.557883in}}%
\pgfpathlineto{\pgfqpoint{3.890336in}{1.671422in}}%
\pgfpathlineto{\pgfqpoint{3.899143in}{1.654370in}}%
\pgfpathlineto{\pgfqpoint{3.907950in}{1.647791in}}%
\pgfpathlineto{\pgfqpoint{3.916757in}{1.570676in}}%
\pgfpathlineto{\pgfqpoint{3.925563in}{1.723905in}}%
\pgfpathlineto{\pgfqpoint{3.934370in}{1.671060in}}%
\pgfpathlineto{\pgfqpoint{3.943177in}{1.696962in}}%
\pgfpathlineto{\pgfqpoint{3.951984in}{1.630167in}}%
\pgfpathlineto{\pgfqpoint{3.960791in}{1.757195in}}%
\pgfpathlineto{\pgfqpoint{3.969597in}{1.677169in}}%
\pgfpathlineto{\pgfqpoint{3.978404in}{1.672307in}}%
\pgfpathlineto{\pgfqpoint{3.987211in}{1.670516in}}%
\pgfpathlineto{\pgfqpoint{3.996018in}{1.685603in}}%
\pgfpathlineto{\pgfqpoint{4.004825in}{1.729045in}}%
\pgfpathlineto{\pgfqpoint{4.013632in}{1.661532in}}%
\pgfpathlineto{\pgfqpoint{4.022438in}{1.699103in}}%
\pgfpathlineto{\pgfqpoint{4.031245in}{1.699372in}}%
\pgfpathlineto{\pgfqpoint{4.040052in}{1.699121in}}%
\pgfpathlineto{\pgfqpoint{4.048859in}{1.693951in}}%
\pgfpathlineto{\pgfqpoint{4.057666in}{1.698296in}}%
\pgfpathlineto{\pgfqpoint{4.066472in}{1.625259in}}%
\pgfpathlineto{\pgfqpoint{4.075279in}{1.614862in}}%
\pgfpathlineto{\pgfqpoint{4.084086in}{1.564998in}}%
\pgfpathlineto{\pgfqpoint{4.092893in}{1.706547in}}%
\pgfpathlineto{\pgfqpoint{4.101700in}{1.594499in}}%
\pgfpathlineto{\pgfqpoint{4.110507in}{1.627819in}}%
\pgfpathlineto{\pgfqpoint{4.119313in}{1.640087in}}%
\pgfpathlineto{\pgfqpoint{4.128120in}{1.674272in}}%
\pgfpathlineto{\pgfqpoint{4.136927in}{1.505520in}}%
\pgfpathlineto{\pgfqpoint{4.145734in}{1.580770in}}%
\pgfpathlineto{\pgfqpoint{4.154541in}{1.616422in}}%
\pgfpathlineto{\pgfqpoint{4.163347in}{1.562367in}}%
\pgfpathlineto{\pgfqpoint{4.172154in}{1.555991in}}%
\pgfpathlineto{\pgfqpoint{4.180961in}{1.623761in}}%
\pgfpathlineto{\pgfqpoint{4.189768in}{1.580686in}}%
\pgfpathlineto{\pgfqpoint{4.198575in}{1.485475in}}%
\pgfpathlineto{\pgfqpoint{4.207382in}{1.566018in}}%
\pgfpathlineto{\pgfqpoint{4.216188in}{1.654274in}}%
\pgfpathlineto{\pgfqpoint{4.224995in}{1.645119in}}%
\pgfpathlineto{\pgfqpoint{4.233802in}{1.763653in}}%
\pgfpathlineto{\pgfqpoint{4.242609in}{1.742444in}}%
\pgfpathlineto{\pgfqpoint{4.251416in}{1.764910in}}%
\pgfpathlineto{\pgfqpoint{4.260222in}{1.785497in}}%
\pgfpathlineto{\pgfqpoint{4.269029in}{1.725668in}}%
\pgfpathlineto{\pgfqpoint{4.277836in}{1.632321in}}%
\pgfpathlineto{\pgfqpoint{4.286643in}{1.543258in}}%
\pgfpathlineto{\pgfqpoint{4.295450in}{1.467229in}}%
\pgfpathlineto{\pgfqpoint{4.304257in}{1.555438in}}%
\pgfpathlineto{\pgfqpoint{4.313063in}{1.644020in}}%
\pgfpathlineto{\pgfqpoint{4.321870in}{1.629041in}}%
\pgfpathlineto{\pgfqpoint{4.330677in}{1.540663in}}%
\pgfpathlineto{\pgfqpoint{4.339484in}{1.592969in}}%
\pgfpathlineto{\pgfqpoint{4.348291in}{1.634608in}}%
\pgfpathlineto{\pgfqpoint{4.357097in}{1.681643in}}%
\pgfpathlineto{\pgfqpoint{4.365904in}{1.725537in}}%
\pgfpathlineto{\pgfqpoint{4.374711in}{1.764663in}}%
\pgfpathlineto{\pgfqpoint{4.383518in}{1.686103in}}%
\pgfpathlineto{\pgfqpoint{4.392325in}{1.628141in}}%
\pgfpathlineto{\pgfqpoint{4.401132in}{1.625975in}}%
\pgfpathlineto{\pgfqpoint{4.409938in}{1.631289in}}%
\pgfpathlineto{\pgfqpoint{4.418745in}{1.664785in}}%
\pgfpathlineto{\pgfqpoint{4.427552in}{1.761207in}}%
\pgfpathlineto{\pgfqpoint{4.436359in}{1.817386in}}%
\pgfpathlineto{\pgfqpoint{4.445166in}{1.816463in}}%
\pgfpathlineto{\pgfqpoint{4.453972in}{1.720197in}}%
\pgfpathlineto{\pgfqpoint{4.462779in}{1.681527in}}%
\pgfpathlineto{\pgfqpoint{4.471586in}{1.590791in}}%
\pgfpathlineto{\pgfqpoint{4.480393in}{1.671910in}}%
\pgfpathlineto{\pgfqpoint{4.489200in}{1.695052in}}%
\pgfpathlineto{\pgfqpoint{4.498007in}{1.630931in}}%
\pgfpathlineto{\pgfqpoint{4.506813in}{1.611867in}}%
\pgfpathlineto{\pgfqpoint{4.515620in}{1.604866in}}%
\pgfpathlineto{\pgfqpoint{4.524427in}{1.530223in}}%
\pgfpathlineto{\pgfqpoint{4.533234in}{1.476570in}}%
\pgfpathlineto{\pgfqpoint{4.542041in}{1.551825in}}%
\pgfpathlineto{\pgfqpoint{4.550847in}{1.561048in}}%
\pgfpathlineto{\pgfqpoint{4.559654in}{1.530856in}}%
\pgfpathlineto{\pgfqpoint{4.568461in}{1.589968in}}%
\pgfpathlineto{\pgfqpoint{4.577268in}{1.709807in}}%
\pgfpathlineto{\pgfqpoint{4.586075in}{1.706645in}}%
\pgfpathlineto{\pgfqpoint{4.594882in}{1.559842in}}%
\pgfpathlineto{\pgfqpoint{4.603688in}{1.641335in}}%
\pgfpathlineto{\pgfqpoint{4.612495in}{1.566853in}}%
\pgfpathlineto{\pgfqpoint{4.621302in}{1.551536in}}%
\pgfpathlineto{\pgfqpoint{4.630109in}{1.649363in}}%
\pgfpathlineto{\pgfqpoint{4.638916in}{1.644951in}}%
\pgfpathlineto{\pgfqpoint{4.647722in}{1.725247in}}%
\pgfpathlineto{\pgfqpoint{4.656529in}{1.754390in}}%
\pgfpathlineto{\pgfqpoint{4.665336in}{1.638398in}}%
\pgfpathlineto{\pgfqpoint{4.674143in}{1.658160in}}%
\pgfpathlineto{\pgfqpoint{4.682950in}{1.519406in}}%
\pgfpathlineto{\pgfqpoint{4.691757in}{1.516188in}}%
\pgfpathlineto{\pgfqpoint{4.700563in}{1.523401in}}%
\pgfpathlineto{\pgfqpoint{4.709370in}{1.503817in}}%
\pgfpathlineto{\pgfqpoint{4.718177in}{1.343136in}}%
\pgfpathlineto{\pgfqpoint{4.726984in}{1.422504in}}%
\pgfpathlineto{\pgfqpoint{4.735791in}{1.602446in}}%
\pgfpathlineto{\pgfqpoint{4.744597in}{1.576813in}}%
\pgfpathlineto{\pgfqpoint{4.753404in}{1.651400in}}%
\pgfpathlineto{\pgfqpoint{4.762211in}{1.686145in}}%
\pgfpathlineto{\pgfqpoint{4.771018in}{1.643609in}}%
\pgfpathlineto{\pgfqpoint{4.779825in}{1.695469in}}%
\pgfpathlineto{\pgfqpoint{4.788632in}{1.656762in}}%
\pgfpathlineto{\pgfqpoint{4.797438in}{1.617855in}}%
\pgfpathlineto{\pgfqpoint{4.806245in}{1.587525in}}%
\pgfpathlineto{\pgfqpoint{4.815052in}{1.555915in}}%
\pgfpathlineto{\pgfqpoint{4.823859in}{1.537461in}}%
\pgfpathlineto{\pgfqpoint{4.832666in}{1.497177in}}%
\pgfpathlineto{\pgfqpoint{4.841472in}{1.467786in}}%
\pgfpathlineto{\pgfqpoint{4.850279in}{1.554583in}}%
\pgfpathlineto{\pgfqpoint{4.859086in}{1.629476in}}%
\pgfpathlineto{\pgfqpoint{4.867893in}{1.685412in}}%
\pgfpathlineto{\pgfqpoint{4.876700in}{1.630398in}}%
\pgfpathlineto{\pgfqpoint{4.885507in}{1.647125in}}%
\pgfpathlineto{\pgfqpoint{4.894313in}{1.721891in}}%
\pgfpathlineto{\pgfqpoint{4.903120in}{1.683003in}}%
\pgfpathlineto{\pgfqpoint{4.911927in}{1.593504in}}%
\pgfpathlineto{\pgfqpoint{4.920734in}{1.671613in}}%
\pgfpathlineto{\pgfqpoint{4.929541in}{1.666252in}}%
\pgfpathlineto{\pgfqpoint{4.938347in}{1.619477in}}%
\pgfpathlineto{\pgfqpoint{4.947154in}{1.731706in}}%
\pgfpathlineto{\pgfqpoint{4.955961in}{1.670558in}}%
\pgfpathlineto{\pgfqpoint{4.964768in}{1.686106in}}%
\pgfpathlineto{\pgfqpoint{4.973575in}{1.610381in}}%
\pgfpathlineto{\pgfqpoint{4.982382in}{1.558825in}}%
\pgfpathlineto{\pgfqpoint{4.991188in}{1.535842in}}%
\pgfpathlineto{\pgfqpoint{4.999995in}{1.562060in}}%
\pgfpathlineto{\pgfqpoint{5.008802in}{1.682017in}}%
\pgfpathlineto{\pgfqpoint{5.017609in}{1.643047in}}%
\pgfpathlineto{\pgfqpoint{5.026416in}{1.577726in}}%
\pgfpathlineto{\pgfqpoint{5.035222in}{1.587969in}}%
\pgfpathlineto{\pgfqpoint{5.044029in}{1.498102in}}%
\pgfpathlineto{\pgfqpoint{5.052836in}{1.548776in}}%
\pgfpathlineto{\pgfqpoint{5.061643in}{1.528441in}}%
\pgfpathlineto{\pgfqpoint{5.070450in}{1.533481in}}%
\pgfpathlineto{\pgfqpoint{5.079257in}{1.608056in}}%
\pgfpathlineto{\pgfqpoint{5.088063in}{1.608753in}}%
\pgfpathlineto{\pgfqpoint{5.096870in}{1.652220in}}%
\pgfpathlineto{\pgfqpoint{5.105677in}{1.741111in}}%
\pgfpathlineto{\pgfqpoint{5.114484in}{1.717738in}}%
\pgfpathlineto{\pgfqpoint{5.123291in}{1.724437in}}%
\pgfpathlineto{\pgfqpoint{5.132097in}{1.674393in}}%
\pgfpathlineto{\pgfqpoint{5.140904in}{1.685519in}}%
\pgfpathlineto{\pgfqpoint{5.149711in}{1.740892in}}%
\pgfpathlineto{\pgfqpoint{5.158518in}{1.618727in}}%
\pgfpathlineto{\pgfqpoint{5.167325in}{1.612902in}}%
\pgfpathlineto{\pgfqpoint{5.176132in}{1.636290in}}%
\pgfpathlineto{\pgfqpoint{5.184938in}{1.597324in}}%
\pgfpathlineto{\pgfqpoint{5.193745in}{1.698134in}}%
\pgfpathlineto{\pgfqpoint{5.202552in}{1.672683in}}%
\pgfpathlineto{\pgfqpoint{5.211359in}{1.665653in}}%
\pgfpathlineto{\pgfqpoint{5.220166in}{1.705794in}}%
\pgfpathlineto{\pgfqpoint{5.228972in}{1.632678in}}%
\pgfpathlineto{\pgfqpoint{5.237779in}{1.640979in}}%
\pgfpathlineto{\pgfqpoint{5.246586in}{1.654514in}}%
\pgfpathlineto{\pgfqpoint{5.255393in}{1.664516in}}%
\pgfpathlineto{\pgfqpoint{5.264200in}{1.573096in}}%
\pgfpathlineto{\pgfqpoint{5.273007in}{1.608051in}}%
\pgfpathlineto{\pgfqpoint{5.281813in}{1.616503in}}%
\pgfpathlineto{\pgfqpoint{5.290620in}{1.677788in}}%
\pgfpathlineto{\pgfqpoint{5.299427in}{1.602086in}}%
\pgfpathlineto{\pgfqpoint{5.308234in}{1.478276in}}%
\pgfpathlineto{\pgfqpoint{5.317041in}{1.468997in}}%
\pgfpathlineto{\pgfqpoint{5.325847in}{1.579273in}}%
\pgfpathlineto{\pgfqpoint{5.334654in}{1.505225in}}%
\pgfpathlineto{\pgfqpoint{5.343461in}{1.620539in}}%
\pgfpathlineto{\pgfqpoint{5.352268in}{1.601223in}}%
\pgfpathlineto{\pgfqpoint{5.361075in}{1.618280in}}%
\pgfpathlineto{\pgfqpoint{5.369882in}{1.672750in}}%
\pgfpathlineto{\pgfqpoint{5.378688in}{1.667217in}}%
\pgfpathlineto{\pgfqpoint{5.387495in}{1.562338in}}%
\pgfpathlineto{\pgfqpoint{5.396302in}{1.687756in}}%
\pgfpathlineto{\pgfqpoint{5.405109in}{1.620810in}}%
\pgfpathlineto{\pgfqpoint{5.413916in}{1.747520in}}%
\pgfpathlineto{\pgfqpoint{5.422722in}{1.738829in}}%
\pgfpathlineto{\pgfqpoint{5.431529in}{1.741618in}}%
\pgfpathlineto{\pgfqpoint{5.440336in}{1.758908in}}%
\pgfpathlineto{\pgfqpoint{5.449143in}{1.616689in}}%
\pgfpathlineto{\pgfqpoint{5.457950in}{1.548719in}}%
\pgfpathlineto{\pgfqpoint{5.466757in}{1.552329in}}%
\pgfpathlineto{\pgfqpoint{5.475563in}{1.664663in}}%
\pgfpathlineto{\pgfqpoint{5.484370in}{1.652734in}}%
\pgfpathlineto{\pgfqpoint{5.493177in}{1.656762in}}%
\pgfpathlineto{\pgfqpoint{5.501984in}{1.665437in}}%
\pgfpathlineto{\pgfqpoint{5.510791in}{1.669585in}}%
\pgfpathlineto{\pgfqpoint{5.519597in}{1.626706in}}%
\pgfpathlineto{\pgfqpoint{5.528404in}{1.686516in}}%
\pgfpathlineto{\pgfqpoint{5.537211in}{1.596536in}}%
\pgfpathlineto{\pgfqpoint{5.546018in}{1.499156in}}%
\pgfpathlineto{\pgfqpoint{5.554825in}{1.522915in}}%
\pgfpathlineto{\pgfqpoint{5.563632in}{1.591072in}}%
\pgfpathlineto{\pgfqpoint{5.572438in}{1.633592in}}%
\pgfpathlineto{\pgfqpoint{5.581245in}{1.563615in}}%
\pgfpathlineto{\pgfqpoint{5.590052in}{1.551075in}}%
\pgfpathlineto{\pgfqpoint{5.598859in}{1.612780in}}%
\pgfpathlineto{\pgfqpoint{5.607666in}{1.552107in}}%
\pgfpathlineto{\pgfqpoint{5.616472in}{1.572407in}}%
\pgfpathlineto{\pgfqpoint{5.625279in}{1.625091in}}%
\pgfpathlineto{\pgfqpoint{5.634086in}{1.607287in}}%
\pgfpathlineto{\pgfqpoint{5.642893in}{1.548261in}}%
\pgfpathlineto{\pgfqpoint{5.651700in}{1.617240in}}%
\pgfpathlineto{\pgfqpoint{5.660507in}{1.583930in}}%
\pgfpathlineto{\pgfqpoint{5.669313in}{1.586063in}}%
\pgfpathlineto{\pgfqpoint{5.678120in}{1.640171in}}%
\pgfpathlineto{\pgfqpoint{5.686927in}{1.652512in}}%
\pgfpathlineto{\pgfqpoint{5.695734in}{1.623972in}}%
\pgfpathlineto{\pgfqpoint{5.704541in}{1.672419in}}%
\pgfpathlineto{\pgfqpoint{5.713347in}{1.730809in}}%
\pgfpathlineto{\pgfqpoint{5.722154in}{1.697447in}}%
\pgfpathlineto{\pgfqpoint{5.730961in}{1.711224in}}%
\pgfpathlineto{\pgfqpoint{5.739768in}{1.669515in}}%
\pgfpathlineto{\pgfqpoint{5.748575in}{1.589150in}}%
\pgfpathlineto{\pgfqpoint{5.757382in}{1.525284in}}%
\pgfpathlineto{\pgfqpoint{5.766188in}{1.499693in}}%
\pgfpathlineto{\pgfqpoint{5.774995in}{1.580618in}}%
\pgfpathlineto{\pgfqpoint{5.783802in}{1.615330in}}%
\pgfpathlineto{\pgfqpoint{5.792609in}{1.631856in}}%
\pgfpathlineto{\pgfqpoint{5.801416in}{1.688601in}}%
\pgfpathlineto{\pgfqpoint{5.810222in}{1.641794in}}%
\pgfpathlineto{\pgfqpoint{5.819029in}{1.539213in}}%
\pgfpathlineto{\pgfqpoint{5.827836in}{1.592961in}}%
\pgfpathlineto{\pgfqpoint{5.836643in}{1.607095in}}%
\pgfpathlineto{\pgfqpoint{5.845450in}{1.726481in}}%
\pgfpathlineto{\pgfqpoint{5.854257in}{1.734728in}}%
\pgfpathlineto{\pgfqpoint{5.863063in}{1.781032in}}%
\pgfpathlineto{\pgfqpoint{5.871870in}{1.839978in}}%
\pgfpathlineto{\pgfqpoint{5.880677in}{1.849234in}}%
\pgfpathlineto{\pgfqpoint{5.889484in}{1.743811in}}%
\pgfpathlineto{\pgfqpoint{5.898291in}{1.757823in}}%
\pgfpathlineto{\pgfqpoint{5.907097in}{1.630499in}}%
\pgfpathlineto{\pgfqpoint{5.915904in}{1.738475in}}%
\pgfpathlineto{\pgfqpoint{5.924711in}{1.784985in}}%
\pgfpathlineto{\pgfqpoint{5.933518in}{1.820561in}}%
\pgfpathlineto{\pgfqpoint{5.942325in}{1.947734in}}%
\pgfpathlineto{\pgfqpoint{5.951132in}{1.896459in}}%
\pgfpathlineto{\pgfqpoint{5.959938in}{1.821020in}}%
\pgfpathlineto{\pgfqpoint{5.968745in}{1.651331in}}%
\pgfpathlineto{\pgfqpoint{5.977552in}{1.749930in}}%
\pgfpathlineto{\pgfqpoint{5.986359in}{1.706047in}}%
\pgfpathlineto{\pgfqpoint{5.995166in}{1.689278in}}%
\pgfpathlineto{\pgfqpoint{6.003972in}{1.637139in}}%
\pgfpathlineto{\pgfqpoint{6.012779in}{1.601877in}}%
\pgfpathlineto{\pgfqpoint{6.021586in}{1.627253in}}%
\pgfpathlineto{\pgfqpoint{6.030393in}{1.744847in}}%
\pgfpathlineto{\pgfqpoint{6.039200in}{1.713890in}}%
\pgfpathlineto{\pgfqpoint{6.048007in}{1.678638in}}%
\pgfpathlineto{\pgfqpoint{6.056813in}{1.573762in}}%
\pgfpathlineto{\pgfqpoint{6.065620in}{1.509045in}}%
\pgfpathlineto{\pgfqpoint{6.074427in}{1.554684in}}%
\pgfpathlineto{\pgfqpoint{6.083234in}{1.709232in}}%
\pgfpathlineto{\pgfqpoint{6.092041in}{1.616020in}}%
\pgfpathlineto{\pgfqpoint{6.100847in}{1.739846in}}%
\pgfpathlineto{\pgfqpoint{6.109654in}{1.698162in}}%
\pgfpathlineto{\pgfqpoint{6.118461in}{1.725117in}}%
\pgfpathlineto{\pgfqpoint{6.127268in}{1.576365in}}%
\pgfpathlineto{\pgfqpoint{6.136075in}{1.469520in}}%
\pgfpathlineto{\pgfqpoint{6.144882in}{1.494324in}}%
\pgfpathlineto{\pgfqpoint{6.153688in}{1.571230in}}%
\pgfpathlineto{\pgfqpoint{6.162495in}{1.476289in}}%
\pgfpathlineto{\pgfqpoint{6.171302in}{1.626791in}}%
\pgfpathlineto{\pgfqpoint{6.180109in}{1.751418in}}%
\pgfpathlineto{\pgfqpoint{6.188916in}{1.705765in}}%
\pgfpathlineto{\pgfqpoint{6.197722in}{1.657396in}}%
\pgfpathlineto{\pgfqpoint{6.206529in}{1.680943in}}%
\pgfpathlineto{\pgfqpoint{6.215336in}{1.576495in}}%
\pgfpathlineto{\pgfqpoint{6.224143in}{1.647806in}}%
\pgfpathlineto{\pgfqpoint{6.232950in}{1.618464in}}%
\pgfpathlineto{\pgfqpoint{6.241757in}{1.640739in}}%
\pgfpathlineto{\pgfqpoint{6.250563in}{1.572306in}}%
\pgfpathlineto{\pgfqpoint{6.259370in}{1.551092in}}%
\pgfpathlineto{\pgfqpoint{6.268177in}{1.593720in}}%
\pgfpathlineto{\pgfqpoint{6.276984in}{1.630648in}}%
\pgfpathlineto{\pgfqpoint{6.285791in}{1.554268in}}%
\pgfpathlineto{\pgfqpoint{6.294597in}{1.601552in}}%
\pgfpathlineto{\pgfqpoint{6.303404in}{1.652382in}}%
\pgfpathlineto{\pgfqpoint{6.312211in}{1.656107in}}%
\pgfpathlineto{\pgfqpoint{6.321018in}{1.660019in}}%
\pgfpathlineto{\pgfqpoint{6.329825in}{1.652977in}}%
\pgfpathlineto{\pgfqpoint{6.338632in}{1.626048in}}%
\pgfpathlineto{\pgfqpoint{6.347438in}{1.671013in}}%
\pgfpathlineto{\pgfqpoint{6.356245in}{1.619730in}}%
\pgfpathlineto{\pgfqpoint{6.365052in}{1.644208in}}%
\pgfpathlineto{\pgfqpoint{6.373859in}{1.639562in}}%
\pgfpathlineto{\pgfqpoint{6.382666in}{1.643854in}}%
\pgfpathlineto{\pgfqpoint{6.391472in}{1.605934in}}%
\pgfpathlineto{\pgfqpoint{6.400279in}{1.543792in}}%
\pgfpathlineto{\pgfqpoint{6.409086in}{1.522833in}}%
\pgfpathlineto{\pgfqpoint{6.417893in}{1.521512in}}%
\pgfpathlineto{\pgfqpoint{6.426700in}{1.576686in}}%
\pgfpathlineto{\pgfqpoint{6.435507in}{1.670376in}}%
\pgfpathlineto{\pgfqpoint{6.444313in}{1.725171in}}%
\pgfpathlineto{\pgfqpoint{6.453120in}{1.701107in}}%
\pgfpathlineto{\pgfqpoint{6.461927in}{1.661628in}}%
\pgfpathlineto{\pgfqpoint{6.470734in}{1.657960in}}%
\pgfpathlineto{\pgfqpoint{6.479541in}{1.678407in}}%
\pgfpathlineto{\pgfqpoint{6.488347in}{1.852921in}}%
\pgfpathlineto{\pgfqpoint{6.497154in}{1.886927in}}%
\pgfpathlineto{\pgfqpoint{6.505961in}{1.811703in}}%
\pgfpathlineto{\pgfqpoint{6.514768in}{1.816383in}}%
\pgfpathlineto{\pgfqpoint{6.523575in}{1.769502in}}%
\pgfpathlineto{\pgfqpoint{6.532382in}{1.692658in}}%
\pgfpathlineto{\pgfqpoint{6.541188in}{1.731051in}}%
\pgfpathlineto{\pgfqpoint{6.549995in}{1.738503in}}%
\pgfpathlineto{\pgfqpoint{6.558802in}{1.791199in}}%
\pgfpathlineto{\pgfqpoint{6.567609in}{1.737735in}}%
\pgfpathlineto{\pgfqpoint{6.576416in}{1.693959in}}%
\pgfpathlineto{\pgfqpoint{6.585222in}{1.716804in}}%
\pgfpathlineto{\pgfqpoint{6.594029in}{1.706321in}}%
\pgfpathlineto{\pgfqpoint{6.602836in}{1.698588in}}%
\pgfpathlineto{\pgfqpoint{6.611643in}{1.668842in}}%
\pgfpathlineto{\pgfqpoint{6.620450in}{1.717752in}}%
\pgfpathlineto{\pgfqpoint{6.629257in}{1.648056in}}%
\pgfpathlineto{\pgfqpoint{6.638063in}{1.628501in}}%
\pgfpathlineto{\pgfqpoint{6.646870in}{1.711150in}}%
\pgfpathlineto{\pgfqpoint{6.655677in}{1.567134in}}%
\pgfpathlineto{\pgfqpoint{6.664484in}{1.647087in}}%
\pgfpathlineto{\pgfqpoint{6.673291in}{1.557827in}}%
\pgfpathlineto{\pgfqpoint{6.682097in}{1.594828in}}%
\pgfpathlineto{\pgfqpoint{6.690904in}{1.552306in}}%
\pgfpathlineto{\pgfqpoint{6.699711in}{1.516699in}}%
\pgfpathlineto{\pgfqpoint{6.708518in}{1.580523in}}%
\pgfpathlineto{\pgfqpoint{6.717325in}{1.606767in}}%
\pgfpathlineto{\pgfqpoint{6.726132in}{1.701439in}}%
\pgfpathlineto{\pgfqpoint{6.734938in}{1.743360in}}%
\pgfpathlineto{\pgfqpoint{6.743745in}{1.850221in}}%
\pgfpathlineto{\pgfqpoint{6.752552in}{1.840167in}}%
\pgfpathlineto{\pgfqpoint{6.761359in}{1.746058in}}%
\pgfpathlineto{\pgfqpoint{6.770166in}{1.739503in}}%
\pgfpathlineto{\pgfqpoint{6.778972in}{1.635671in}}%
\pgfpathlineto{\pgfqpoint{6.787779in}{1.693786in}}%
\pgfpathlineto{\pgfqpoint{6.796586in}{1.631293in}}%
\pgfpathlineto{\pgfqpoint{6.805393in}{1.528804in}}%
\pgfpathlineto{\pgfqpoint{6.814200in}{1.559793in}}%
\pgfpathlineto{\pgfqpoint{6.823007in}{1.619910in}}%
\pgfpathlineto{\pgfqpoint{6.831813in}{1.636577in}}%
\pgfpathlineto{\pgfqpoint{6.840620in}{1.502102in}}%
\pgfpathlineto{\pgfqpoint{6.849427in}{1.645456in}}%
\pgfpathlineto{\pgfqpoint{6.858234in}{1.681760in}}%
\pgfpathlineto{\pgfqpoint{6.867041in}{1.656512in}}%
\pgfpathlineto{\pgfqpoint{6.875847in}{1.619429in}}%
\pgfpathlineto{\pgfqpoint{6.884654in}{1.614704in}}%
\pgfpathlineto{\pgfqpoint{6.893461in}{1.569582in}}%
\pgfpathlineto{\pgfqpoint{6.902268in}{1.705953in}}%
\pgfpathlineto{\pgfqpoint{6.911075in}{1.726127in}}%
\pgfpathlineto{\pgfqpoint{6.919882in}{1.652310in}}%
\pgfpathlineto{\pgfqpoint{6.928688in}{1.661395in}}%
\pgfpathlineto{\pgfqpoint{6.937495in}{1.675414in}}%
\pgfpathlineto{\pgfqpoint{6.946302in}{1.756494in}}%
\pgfpathlineto{\pgfqpoint{6.955109in}{1.701077in}}%
\pgfpathlineto{\pgfqpoint{6.963916in}{1.771457in}}%
\pgfpathlineto{\pgfqpoint{6.972722in}{1.934022in}}%
\pgfpathlineto{\pgfqpoint{6.981529in}{1.819272in}}%
\pgfpathlineto{\pgfqpoint{6.990336in}{1.738890in}}%
\pgfpathlineto{\pgfqpoint{6.999143in}{1.659989in}}%
\pgfpathlineto{\pgfqpoint{7.007950in}{1.712804in}}%
\pgfpathlineto{\pgfqpoint{7.016757in}{1.633310in}}%
\pgfpathlineto{\pgfqpoint{7.025563in}{1.720501in}}%
\pgfpathlineto{\pgfqpoint{7.034370in}{1.636135in}}%
\pgfpathlineto{\pgfqpoint{7.043177in}{1.556979in}}%
\pgfpathlineto{\pgfqpoint{7.051984in}{1.612854in}}%
\pgfpathlineto{\pgfqpoint{7.060791in}{1.590786in}}%
\pgfpathlineto{\pgfqpoint{7.069597in}{1.535331in}}%
\pgfpathlineto{\pgfqpoint{7.078404in}{1.607525in}}%
\pgfpathlineto{\pgfqpoint{7.087211in}{1.565591in}}%
\pgfpathlineto{\pgfqpoint{7.096018in}{1.650724in}}%
\pgfpathlineto{\pgfqpoint{7.104825in}{1.688788in}}%
\pgfpathlineto{\pgfqpoint{7.113632in}{1.749030in}}%
\pgfpathlineto{\pgfqpoint{7.122438in}{1.739790in}}%
\pgfpathlineto{\pgfqpoint{7.131245in}{1.793839in}}%
\pgfpathlineto{\pgfqpoint{7.140052in}{1.744978in}}%
\pgfpathlineto{\pgfqpoint{7.148859in}{1.744234in}}%
\pgfpathlineto{\pgfqpoint{7.157666in}{1.710201in}}%
\pgfpathlineto{\pgfqpoint{7.166472in}{1.696310in}}%
\pgfpathlineto{\pgfqpoint{7.175279in}{1.648978in}}%
\pgfpathlineto{\pgfqpoint{7.184086in}{1.690738in}}%
\pgfpathlineto{\pgfqpoint{7.192893in}{1.666744in}}%
\pgfpathlineto{\pgfqpoint{7.201700in}{1.669143in}}%
\pgfpathlineto{\pgfqpoint{7.210507in}{1.711196in}}%
\pgfpathlineto{\pgfqpoint{7.219313in}{1.679623in}}%
\pgfpathlineto{\pgfqpoint{7.228120in}{1.617376in}}%
\pgfpathlineto{\pgfqpoint{7.236927in}{1.643957in}}%
\pgfpathlineto{\pgfqpoint{7.245734in}{1.671641in}}%
\pgfpathlineto{\pgfqpoint{7.254541in}{1.643545in}}%
\pgfpathlineto{\pgfqpoint{7.263347in}{1.553582in}}%
\pgfpathlineto{\pgfqpoint{7.272154in}{1.745586in}}%
\pgfpathlineto{\pgfqpoint{7.280961in}{1.656677in}}%
\pgfpathlineto{\pgfqpoint{7.289768in}{1.745806in}}%
\pgfpathlineto{\pgfqpoint{7.298575in}{1.633313in}}%
\pgfpathlineto{\pgfqpoint{7.307382in}{1.530729in}}%
\pgfpathlineto{\pgfqpoint{7.316188in}{1.627325in}}%
\pgfpathlineto{\pgfqpoint{7.324995in}{1.701590in}}%
\pgfpathlineto{\pgfqpoint{7.333802in}{1.678285in}}%
\pgfpathlineto{\pgfqpoint{7.342609in}{1.672953in}}%
\pgfpathlineto{\pgfqpoint{7.351416in}{1.692833in}}%
\pgfpathlineto{\pgfqpoint{7.360222in}{1.614636in}}%
\pgfpathlineto{\pgfqpoint{7.369029in}{1.761657in}}%
\pgfpathlineto{\pgfqpoint{7.377836in}{1.683618in}}%
\pgfpathlineto{\pgfqpoint{7.386643in}{1.742092in}}%
\pgfpathlineto{\pgfqpoint{7.395450in}{1.671787in}}%
\pgfpathlineto{\pgfqpoint{7.404257in}{1.699389in}}%
\pgfpathlineto{\pgfqpoint{7.413063in}{1.691655in}}%
\pgfpathlineto{\pgfqpoint{7.421870in}{1.716071in}}%
\pgfpathlineto{\pgfqpoint{7.430677in}{1.575615in}}%
\pgfpathlineto{\pgfqpoint{7.439484in}{1.582684in}}%
\pgfpathlineto{\pgfqpoint{7.448291in}{1.562591in}}%
\pgfpathlineto{\pgfqpoint{7.457097in}{1.523904in}}%
\pgfpathlineto{\pgfqpoint{7.465904in}{1.487822in}}%
\pgfpathlineto{\pgfqpoint{7.474711in}{1.510785in}}%
\pgfpathlineto{\pgfqpoint{7.483518in}{1.654038in}}%
\pgfpathlineto{\pgfqpoint{7.492325in}{1.646231in}}%
\pgfpathlineto{\pgfqpoint{7.501132in}{1.672110in}}%
\pgfpathlineto{\pgfqpoint{7.509938in}{1.551914in}}%
\pgfpathlineto{\pgfqpoint{7.518745in}{1.548777in}}%
\pgfpathlineto{\pgfqpoint{7.527552in}{1.508770in}}%
\pgfpathlineto{\pgfqpoint{7.536359in}{1.552798in}}%
\pgfpathlineto{\pgfqpoint{7.545166in}{1.563578in}}%
\pgfpathlineto{\pgfqpoint{7.553972in}{1.560199in}}%
\pgfpathlineto{\pgfqpoint{7.562779in}{1.569293in}}%
\pgfpathlineto{\pgfqpoint{7.571586in}{1.580686in}}%
\pgfpathlineto{\pgfqpoint{7.580393in}{1.580326in}}%
\pgfpathlineto{\pgfqpoint{7.589200in}{1.593304in}}%
\pgfpathlineto{\pgfqpoint{7.598007in}{1.595890in}}%
\pgfpathlineto{\pgfqpoint{7.606813in}{1.588762in}}%
\pgfpathlineto{\pgfqpoint{7.615620in}{1.694925in}}%
\pgfpathlineto{\pgfqpoint{7.624427in}{1.734947in}}%
\pgfpathlineto{\pgfqpoint{7.633234in}{1.763736in}}%
\pgfpathlineto{\pgfqpoint{7.642041in}{1.736397in}}%
\pgfpathlineto{\pgfqpoint{7.650847in}{1.639118in}}%
\pgfpathlineto{\pgfqpoint{7.659654in}{1.692871in}}%
\pgfpathlineto{\pgfqpoint{7.668461in}{1.732282in}}%
\pgfpathlineto{\pgfqpoint{7.677268in}{1.697560in}}%
\pgfpathlineto{\pgfqpoint{7.686075in}{1.647146in}}%
\pgfpathlineto{\pgfqpoint{7.694882in}{1.616243in}}%
\pgfpathlineto{\pgfqpoint{7.703688in}{1.591975in}}%
\pgfpathlineto{\pgfqpoint{7.712495in}{1.588011in}}%
\pgfpathlineto{\pgfqpoint{7.721302in}{1.582931in}}%
\pgfpathlineto{\pgfqpoint{7.730109in}{1.599384in}}%
\pgfpathlineto{\pgfqpoint{7.738916in}{1.589422in}}%
\pgfpathlineto{\pgfqpoint{7.747722in}{1.647073in}}%
\pgfpathlineto{\pgfqpoint{7.756529in}{1.648852in}}%
\pgfpathlineto{\pgfqpoint{7.765336in}{1.726438in}}%
\pgfpathlineto{\pgfqpoint{7.774143in}{1.784036in}}%
\pgfpathlineto{\pgfqpoint{7.782950in}{1.753701in}}%
\pgfpathlineto{\pgfqpoint{7.791757in}{1.644113in}}%
\pgfpathlineto{\pgfqpoint{7.800563in}{1.653116in}}%
\pgfpathlineto{\pgfqpoint{7.809370in}{1.562001in}}%
\pgfpathlineto{\pgfqpoint{7.818177in}{1.558954in}}%
\pgfpathlineto{\pgfqpoint{7.826984in}{1.577639in}}%
\pgfpathlineto{\pgfqpoint{7.835791in}{1.643328in}}%
\pgfpathlineto{\pgfqpoint{7.844597in}{1.566130in}}%
\pgfpathlineto{\pgfqpoint{7.853404in}{1.615122in}}%
\pgfpathlineto{\pgfqpoint{7.862211in}{1.505371in}}%
\pgfpathlineto{\pgfqpoint{7.871018in}{1.590596in}}%
\pgfpathlineto{\pgfqpoint{7.879825in}{1.598917in}}%
\pgfpathlineto{\pgfqpoint{7.888632in}{1.565204in}}%
\pgfpathlineto{\pgfqpoint{7.897438in}{1.549763in}}%
\pgfpathlineto{\pgfqpoint{7.906245in}{1.501630in}}%
\pgfpathlineto{\pgfqpoint{7.915052in}{1.468575in}}%
\pgfpathlineto{\pgfqpoint{7.923859in}{1.507361in}}%
\pgfpathlineto{\pgfqpoint{7.932666in}{1.554120in}}%
\pgfpathlineto{\pgfqpoint{7.941472in}{1.497215in}}%
\pgfpathlineto{\pgfqpoint{7.950279in}{1.577467in}}%
\pgfpathlineto{\pgfqpoint{7.959086in}{1.601194in}}%
\pgfpathlineto{\pgfqpoint{7.967893in}{1.634834in}}%
\pgfpathlineto{\pgfqpoint{7.976700in}{1.616909in}}%
\pgfpathlineto{\pgfqpoint{7.985507in}{1.596748in}}%
\pgfpathlineto{\pgfqpoint{7.994313in}{1.637504in}}%
\pgfpathlineto{\pgfqpoint{8.003120in}{1.632326in}}%
\pgfpathlineto{\pgfqpoint{8.011927in}{1.632846in}}%
\pgfpathlineto{\pgfqpoint{8.020734in}{1.441894in}}%
\pgfpathlineto{\pgfqpoint{8.029541in}{1.634938in}}%
\pgfpathlineto{\pgfqpoint{8.038347in}{1.575491in}}%
\pgfpathlineto{\pgfqpoint{8.047154in}{1.581113in}}%
\pgfpathlineto{\pgfqpoint{8.055961in}{1.581110in}}%
\pgfpathlineto{\pgfqpoint{8.064768in}{1.478004in}}%
\pgfpathlineto{\pgfqpoint{8.073575in}{1.487184in}}%
\pgfpathlineto{\pgfqpoint{8.082382in}{1.474580in}}%
\pgfpathlineto{\pgfqpoint{8.091188in}{1.495533in}}%
\pgfpathlineto{\pgfqpoint{8.099995in}{1.503977in}}%
\pgfpathlineto{\pgfqpoint{8.108802in}{1.550563in}}%
\pgfpathlineto{\pgfqpoint{8.117609in}{1.582142in}}%
\pgfpathlineto{\pgfqpoint{8.126416in}{1.678252in}}%
\pgfpathlineto{\pgfqpoint{8.135222in}{1.615174in}}%
\pgfpathlineto{\pgfqpoint{8.144029in}{1.687877in}}%
\pgfpathlineto{\pgfqpoint{8.152836in}{1.609704in}}%
\pgfpathlineto{\pgfqpoint{8.161643in}{1.632959in}}%
\pgfpathlineto{\pgfqpoint{8.170450in}{1.727422in}}%
\pgfpathlineto{\pgfqpoint{8.179257in}{1.691377in}}%
\pgfpathlineto{\pgfqpoint{8.188063in}{1.617005in}}%
\pgfpathlineto{\pgfqpoint{8.196870in}{1.602128in}}%
\pgfpathlineto{\pgfqpoint{8.205677in}{1.567136in}}%
\pgfpathlineto{\pgfqpoint{8.214484in}{1.567863in}}%
\pgfpathlineto{\pgfqpoint{8.223291in}{1.605527in}}%
\pgfpathlineto{\pgfqpoint{8.232097in}{1.651898in}}%
\pgfpathlineto{\pgfqpoint{8.240904in}{1.601326in}}%
\pgfpathlineto{\pgfqpoint{8.249711in}{1.577699in}}%
\pgfpathlineto{\pgfqpoint{8.258518in}{1.580863in}}%
\pgfpathlineto{\pgfqpoint{8.267325in}{1.543263in}}%
\pgfpathlineto{\pgfqpoint{8.276132in}{1.493461in}}%
\pgfpathlineto{\pgfqpoint{8.284938in}{1.512691in}}%
\pgfpathlineto{\pgfqpoint{8.293745in}{1.469261in}}%
\pgfpathlineto{\pgfqpoint{8.302552in}{1.485677in}}%
\pgfpathlineto{\pgfqpoint{8.311359in}{1.535519in}}%
\pgfpathlineto{\pgfqpoint{8.320166in}{1.520956in}}%
\pgfpathlineto{\pgfqpoint{8.328972in}{1.587385in}}%
\pgfpathlineto{\pgfqpoint{8.337779in}{1.498886in}}%
\pgfpathlineto{\pgfqpoint{8.346586in}{1.456432in}}%
\pgfpathlineto{\pgfqpoint{8.355393in}{1.419285in}}%
\pgfpathlineto{\pgfqpoint{8.364200in}{1.487072in}}%
\pgfpathlineto{\pgfqpoint{8.373007in}{1.435681in}}%
\pgfpathlineto{\pgfqpoint{8.381813in}{1.492154in}}%
\pgfpathlineto{\pgfqpoint{8.390620in}{1.603798in}}%
\pgfpathlineto{\pgfqpoint{8.399427in}{1.628467in}}%
\pgfpathlineto{\pgfqpoint{8.408234in}{1.634432in}}%
\pgfpathlineto{\pgfqpoint{8.417041in}{1.527455in}}%
\pgfpathlineto{\pgfqpoint{8.425847in}{1.520624in}}%
\pgfpathlineto{\pgfqpoint{8.434654in}{1.489464in}}%
\pgfpathlineto{\pgfqpoint{8.443461in}{1.568576in}}%
\pgfpathlineto{\pgfqpoint{8.452268in}{1.532123in}}%
\pgfpathlineto{\pgfqpoint{8.461075in}{1.564919in}}%
\pgfpathlineto{\pgfqpoint{8.469882in}{1.625767in}}%
\pgfpathlineto{\pgfqpoint{8.478688in}{1.630086in}}%
\pgfpathlineto{\pgfqpoint{8.487495in}{1.675649in}}%
\pgfpathlineto{\pgfqpoint{8.496302in}{1.726372in}}%
\pgfpathlineto{\pgfqpoint{8.505109in}{1.614898in}}%
\pgfpathlineto{\pgfqpoint{8.513916in}{1.575277in}}%
\pgfpathlineto{\pgfqpoint{8.522722in}{1.651454in}}%
\pgfpathlineto{\pgfqpoint{8.531529in}{1.643306in}}%
\pgfpathlineto{\pgfqpoint{8.540336in}{1.624290in}}%
\pgfpathlineto{\pgfqpoint{8.549143in}{1.636807in}}%
\pgfpathlineto{\pgfqpoint{8.557950in}{1.705162in}}%
\pgfpathlineto{\pgfqpoint{8.566757in}{1.684037in}}%
\pgfpathlineto{\pgfqpoint{8.575563in}{1.637723in}}%
\pgfpathlineto{\pgfqpoint{8.584370in}{1.561090in}}%
\pgfpathlineto{\pgfqpoint{8.593177in}{1.622190in}}%
\pgfpathlineto{\pgfqpoint{8.601984in}{1.638803in}}%
\pgfpathlineto{\pgfqpoint{8.610791in}{1.692831in}}%
\pgfpathlineto{\pgfqpoint{8.619597in}{1.604171in}}%
\pgfpathlineto{\pgfqpoint{8.628404in}{1.694131in}}%
\pgfpathlineto{\pgfqpoint{8.637211in}{1.629080in}}%
\pgfpathlineto{\pgfqpoint{8.646018in}{1.668078in}}%
\pgfpathlineto{\pgfqpoint{8.654825in}{1.603849in}}%
\pgfpathlineto{\pgfqpoint{8.663632in}{1.540210in}}%
\pgfpathlineto{\pgfqpoint{8.672438in}{1.540161in}}%
\pgfpathlineto{\pgfqpoint{8.681245in}{1.585921in}}%
\pgfpathlineto{\pgfqpoint{8.690052in}{1.540132in}}%
\pgfpathlineto{\pgfqpoint{8.698859in}{1.488146in}}%
\pgfpathlineto{\pgfqpoint{8.707666in}{1.487848in}}%
\pgfpathlineto{\pgfqpoint{8.716472in}{1.459209in}}%
\pgfpathlineto{\pgfqpoint{8.725279in}{1.463836in}}%
\pgfpathlineto{\pgfqpoint{8.734086in}{1.490057in}}%
\pgfpathlineto{\pgfqpoint{8.742893in}{1.586670in}}%
\pgfpathlineto{\pgfqpoint{8.751700in}{1.603584in}}%
\pgfpathlineto{\pgfqpoint{8.760507in}{1.668281in}}%
\pgfpathlineto{\pgfqpoint{8.769313in}{1.608728in}}%
\pgfpathlineto{\pgfqpoint{8.778120in}{1.623671in}}%
\pgfpathlineto{\pgfqpoint{8.786927in}{1.690292in}}%
\pgfpathlineto{\pgfqpoint{8.795734in}{1.686004in}}%
\pgfpathlineto{\pgfqpoint{8.804541in}{1.573076in}}%
\pgfpathlineto{\pgfqpoint{8.813347in}{1.555707in}}%
\pgfpathlineto{\pgfqpoint{8.822154in}{1.664354in}}%
\pgfpathlineto{\pgfqpoint{8.830961in}{1.613200in}}%
\pgfpathlineto{\pgfqpoint{8.839768in}{1.652793in}}%
\pgfpathlineto{\pgfqpoint{8.848575in}{1.747734in}}%
\pgfpathlineto{\pgfqpoint{8.857382in}{1.744768in}}%
\pgfpathlineto{\pgfqpoint{8.866188in}{1.759446in}}%
\pgfpathlineto{\pgfqpoint{8.874995in}{1.738531in}}%
\pgfpathlineto{\pgfqpoint{8.883802in}{1.809359in}}%
\pgfpathlineto{\pgfqpoint{8.892609in}{1.786767in}}%
\pgfpathlineto{\pgfqpoint{8.901416in}{1.737631in}}%
\pgfpathlineto{\pgfqpoint{8.910222in}{1.650319in}}%
\pgfpathlineto{\pgfqpoint{8.919029in}{1.614946in}}%
\pgfpathlineto{\pgfqpoint{8.927836in}{1.645014in}}%
\pgfpathlineto{\pgfqpoint{8.936643in}{1.577602in}}%
\pgfpathlineto{\pgfqpoint{8.945450in}{1.690915in}}%
\pgfpathlineto{\pgfqpoint{8.954257in}{1.771160in}}%
\pgfpathlineto{\pgfqpoint{8.963063in}{1.708346in}}%
\pgfpathlineto{\pgfqpoint{8.971870in}{1.623786in}}%
\pgfpathlineto{\pgfqpoint{8.980677in}{1.592998in}}%
\pgfpathlineto{\pgfqpoint{8.989484in}{1.490650in}}%
\pgfpathlineto{\pgfqpoint{8.998291in}{1.531182in}}%
\pgfpathlineto{\pgfqpoint{9.007097in}{1.642570in}}%
\pgfpathlineto{\pgfqpoint{9.015904in}{1.540520in}}%
\pgfpathlineto{\pgfqpoint{9.024711in}{1.671611in}}%
\pgfpathlineto{\pgfqpoint{9.033518in}{1.700891in}}%
\pgfpathlineto{\pgfqpoint{9.042325in}{1.635676in}}%
\pgfpathlineto{\pgfqpoint{9.051132in}{1.571730in}}%
\pgfpathlineto{\pgfqpoint{9.059938in}{1.522670in}}%
\pgfpathlineto{\pgfqpoint{9.068745in}{1.615777in}}%
\pgfpathlineto{\pgfqpoint{9.077552in}{1.558433in}}%
\pgfpathlineto{\pgfqpoint{9.086359in}{1.478436in}}%
\pgfpathlineto{\pgfqpoint{9.095166in}{1.559629in}}%
\pgfpathlineto{\pgfqpoint{9.103972in}{1.554819in}}%
\pgfpathlineto{\pgfqpoint{9.112779in}{1.489593in}}%
\pgfpathlineto{\pgfqpoint{9.121586in}{1.530867in}}%
\pgfpathlineto{\pgfqpoint{9.130393in}{1.563483in}}%
\pgfpathlineto{\pgfqpoint{9.139200in}{1.544657in}}%
\pgfpathlineto{\pgfqpoint{9.148007in}{1.526150in}}%
\pgfpathlineto{\pgfqpoint{9.156813in}{1.574187in}}%
\pgfpathlineto{\pgfqpoint{9.165620in}{1.575958in}}%
\pgfpathlineto{\pgfqpoint{9.174427in}{1.490020in}}%
\pgfpathlineto{\pgfqpoint{9.183234in}{1.527851in}}%
\pgfpathlineto{\pgfqpoint{9.192041in}{1.651213in}}%
\pgfpathlineto{\pgfqpoint{9.200847in}{1.606128in}}%
\pgfpathlineto{\pgfqpoint{9.209654in}{1.685265in}}%
\pgfpathlineto{\pgfqpoint{9.218461in}{1.714295in}}%
\pgfpathlineto{\pgfqpoint{9.227268in}{1.819520in}}%
\pgfpathlineto{\pgfqpoint{9.236075in}{1.751321in}}%
\pgfpathlineto{\pgfqpoint{9.244882in}{1.728032in}}%
\pgfpathlineto{\pgfqpoint{9.253688in}{1.686038in}}%
\pgfpathlineto{\pgfqpoint{9.262495in}{1.626527in}}%
\pgfpathlineto{\pgfqpoint{9.271302in}{1.696175in}}%
\pgfpathlineto{\pgfqpoint{9.280109in}{1.639862in}}%
\pgfpathlineto{\pgfqpoint{9.288916in}{1.694620in}}%
\pgfpathlineto{\pgfqpoint{9.297722in}{1.734985in}}%
\pgfpathlineto{\pgfqpoint{9.306529in}{1.665797in}}%
\pgfpathlineto{\pgfqpoint{9.315336in}{1.655996in}}%
\pgfpathlineto{\pgfqpoint{9.324143in}{1.696734in}}%
\pgfpathlineto{\pgfqpoint{9.332950in}{1.696330in}}%
\pgfpathlineto{\pgfqpoint{9.341757in}{1.702894in}}%
\pgfpathlineto{\pgfqpoint{9.350563in}{1.642755in}}%
\pgfpathlineto{\pgfqpoint{9.359370in}{1.673469in}}%
\pgfpathlineto{\pgfqpoint{9.368177in}{1.660376in}}%
\pgfpathlineto{\pgfqpoint{9.376984in}{1.666596in}}%
\pgfpathlineto{\pgfqpoint{9.385791in}{1.694825in}}%
\pgfpathlineto{\pgfqpoint{9.394597in}{1.596093in}}%
\pgfpathlineto{\pgfqpoint{9.403404in}{1.500106in}}%
\pgfpathlineto{\pgfqpoint{9.412211in}{1.458951in}}%
\pgfpathlineto{\pgfqpoint{9.421018in}{1.477082in}}%
\pgfpathlineto{\pgfqpoint{9.429825in}{1.529144in}}%
\pgfpathlineto{\pgfqpoint{9.438632in}{1.621640in}}%
\pgfpathlineto{\pgfqpoint{9.447438in}{1.606525in}}%
\pgfpathlineto{\pgfqpoint{9.456245in}{1.531760in}}%
\pgfpathlineto{\pgfqpoint{9.465052in}{1.684599in}}%
\pgfpathlineto{\pgfqpoint{9.473859in}{1.661745in}}%
\pgfpathlineto{\pgfqpoint{9.482666in}{1.603039in}}%
\pgfpathlineto{\pgfqpoint{9.491472in}{1.610229in}}%
\pgfpathlineto{\pgfqpoint{9.500279in}{1.681565in}}%
\pgfpathlineto{\pgfqpoint{9.509086in}{1.737169in}}%
\pgfpathlineto{\pgfqpoint{9.517893in}{1.738992in}}%
\pgfpathlineto{\pgfqpoint{9.526700in}{1.732148in}}%
\pgfpathlineto{\pgfqpoint{9.535507in}{1.691697in}}%
\pgfpathlineto{\pgfqpoint{9.544313in}{1.703497in}}%
\pgfpathlineto{\pgfqpoint{9.553120in}{1.619640in}}%
\pgfpathlineto{\pgfqpoint{9.561927in}{1.557307in}}%
\pgfpathlineto{\pgfqpoint{9.570734in}{1.612908in}}%
\pgfpathlineto{\pgfqpoint{9.579541in}{1.659697in}}%
\pgfpathlineto{\pgfqpoint{9.588347in}{1.677201in}}%
\pgfpathlineto{\pgfqpoint{9.597154in}{1.634279in}}%
\pgfpathlineto{\pgfqpoint{9.605961in}{1.640727in}}%
\pgfpathlineto{\pgfqpoint{9.614768in}{1.654190in}}%
\pgfpathlineto{\pgfqpoint{9.623575in}{1.680547in}}%
\pgfpathlineto{\pgfqpoint{9.632382in}{1.695530in}}%
\pgfpathlineto{\pgfqpoint{9.641188in}{1.634263in}}%
\pgfpathlineto{\pgfqpoint{9.649995in}{1.643252in}}%
\pgfpathlineto{\pgfqpoint{9.658802in}{1.650569in}}%
\pgfpathlineto{\pgfqpoint{9.667609in}{1.574746in}}%
\pgfpathlineto{\pgfqpoint{9.676416in}{1.622758in}}%
\pgfpathlineto{\pgfqpoint{9.685222in}{1.625667in}}%
\pgfpathlineto{\pgfqpoint{9.694029in}{1.575944in}}%
\pgfpathlineto{\pgfqpoint{9.702836in}{1.669369in}}%
\pgfpathlineto{\pgfqpoint{9.711643in}{1.624259in}}%
\pgfpathlineto{\pgfqpoint{9.720450in}{1.578468in}}%
\pgfpathlineto{\pgfqpoint{9.729257in}{1.584452in}}%
\pgfpathlineto{\pgfqpoint{9.738063in}{1.608961in}}%
\pgfpathlineto{\pgfqpoint{9.746870in}{1.550482in}}%
\pgfpathlineto{\pgfqpoint{9.755677in}{1.735602in}}%
\pgfpathlineto{\pgfqpoint{9.764484in}{1.530895in}}%
\pgfpathlineto{\pgfqpoint{9.773291in}{1.484081in}}%
\pgfpathlineto{\pgfqpoint{9.782097in}{1.533338in}}%
\pgfpathlineto{\pgfqpoint{9.790904in}{1.565682in}}%
\pgfpathlineto{\pgfqpoint{9.799711in}{1.522043in}}%
\pgfpathlineto{\pgfqpoint{9.808518in}{1.576492in}}%
\pgfpathlineto{\pgfqpoint{9.817325in}{1.560893in}}%
\pgfpathlineto{\pgfqpoint{9.826132in}{1.522246in}}%
\pgfpathlineto{\pgfqpoint{9.834938in}{1.517107in}}%
\pgfpathlineto{\pgfqpoint{9.843745in}{1.634681in}}%
\pgfpathlineto{\pgfqpoint{9.852552in}{1.571218in}}%
\pgfpathlineto{\pgfqpoint{9.861359in}{1.650655in}}%
\pgfpathlineto{\pgfqpoint{9.870166in}{1.609552in}}%
\pgfpathlineto{\pgfqpoint{9.878972in}{1.629691in}}%
\pgfpathlineto{\pgfqpoint{9.887779in}{1.572828in}}%
\pgfpathlineto{\pgfqpoint{9.896586in}{1.672922in}}%
\pgfpathlineto{\pgfqpoint{9.905393in}{1.729746in}}%
\pgfpathlineto{\pgfqpoint{9.914200in}{1.631038in}}%
\pgfpathlineto{\pgfqpoint{9.923007in}{1.613164in}}%
\pgfpathlineto{\pgfqpoint{9.931813in}{1.616284in}}%
\pgfpathlineto{\pgfqpoint{9.940620in}{1.679194in}}%
\pgfpathlineto{\pgfqpoint{9.949427in}{1.642449in}}%
\pgfpathlineto{\pgfqpoint{9.949427in}{2.219953in}}%
\pgfpathlineto{\pgfqpoint{9.949427in}{2.219953in}}%
\pgfpathlineto{\pgfqpoint{9.940620in}{2.083024in}}%
\pgfpathlineto{\pgfqpoint{9.931813in}{2.135646in}}%
\pgfpathlineto{\pgfqpoint{9.923007in}{2.162097in}}%
\pgfpathlineto{\pgfqpoint{9.914200in}{2.190211in}}%
\pgfpathlineto{\pgfqpoint{9.905393in}{2.175888in}}%
\pgfpathlineto{\pgfqpoint{9.896586in}{2.112627in}}%
\pgfpathlineto{\pgfqpoint{9.887779in}{2.019888in}}%
\pgfpathlineto{\pgfqpoint{9.878972in}{2.166820in}}%
\pgfpathlineto{\pgfqpoint{9.870166in}{2.202758in}}%
\pgfpathlineto{\pgfqpoint{9.861359in}{2.204213in}}%
\pgfpathlineto{\pgfqpoint{9.852552in}{2.230129in}}%
\pgfpathlineto{\pgfqpoint{9.843745in}{2.145359in}}%
\pgfpathlineto{\pgfqpoint{9.834938in}{2.146952in}}%
\pgfpathlineto{\pgfqpoint{9.826132in}{2.054672in}}%
\pgfpathlineto{\pgfqpoint{9.817325in}{2.101142in}}%
\pgfpathlineto{\pgfqpoint{9.808518in}{2.183646in}}%
\pgfpathlineto{\pgfqpoint{9.799711in}{2.137912in}}%
\pgfpathlineto{\pgfqpoint{9.790904in}{2.083309in}}%
\pgfpathlineto{\pgfqpoint{9.782097in}{2.179463in}}%
\pgfpathlineto{\pgfqpoint{9.773291in}{2.120273in}}%
\pgfpathlineto{\pgfqpoint{9.764484in}{2.191520in}}%
\pgfpathlineto{\pgfqpoint{9.755677in}{2.174867in}}%
\pgfpathlineto{\pgfqpoint{9.746870in}{2.267337in}}%
\pgfpathlineto{\pgfqpoint{9.738063in}{2.268563in}}%
\pgfpathlineto{\pgfqpoint{9.729257in}{2.255292in}}%
\pgfpathlineto{\pgfqpoint{9.720450in}{2.194713in}}%
\pgfpathlineto{\pgfqpoint{9.711643in}{2.179843in}}%
\pgfpathlineto{\pgfqpoint{9.702836in}{2.250713in}}%
\pgfpathlineto{\pgfqpoint{9.694029in}{2.219894in}}%
\pgfpathlineto{\pgfqpoint{9.685222in}{2.248428in}}%
\pgfpathlineto{\pgfqpoint{9.676416in}{2.241049in}}%
\pgfpathlineto{\pgfqpoint{9.667609in}{2.157282in}}%
\pgfpathlineto{\pgfqpoint{9.658802in}{2.154207in}}%
\pgfpathlineto{\pgfqpoint{9.649995in}{2.086696in}}%
\pgfpathlineto{\pgfqpoint{9.641188in}{2.083373in}}%
\pgfpathlineto{\pgfqpoint{9.632382in}{2.161418in}}%
\pgfpathlineto{\pgfqpoint{9.623575in}{2.118045in}}%
\pgfpathlineto{\pgfqpoint{9.614768in}{2.179427in}}%
\pgfpathlineto{\pgfqpoint{9.605961in}{2.166130in}}%
\pgfpathlineto{\pgfqpoint{9.597154in}{2.193154in}}%
\pgfpathlineto{\pgfqpoint{9.588347in}{2.270317in}}%
\pgfpathlineto{\pgfqpoint{9.579541in}{2.243914in}}%
\pgfpathlineto{\pgfqpoint{9.570734in}{2.238531in}}%
\pgfpathlineto{\pgfqpoint{9.561927in}{2.224138in}}%
\pgfpathlineto{\pgfqpoint{9.553120in}{2.251700in}}%
\pgfpathlineto{\pgfqpoint{9.544313in}{2.291357in}}%
\pgfpathlineto{\pgfqpoint{9.535507in}{2.154232in}}%
\pgfpathlineto{\pgfqpoint{9.526700in}{2.117941in}}%
\pgfpathlineto{\pgfqpoint{9.517893in}{2.117225in}}%
\pgfpathlineto{\pgfqpoint{9.509086in}{2.156153in}}%
\pgfpathlineto{\pgfqpoint{9.500279in}{2.195959in}}%
\pgfpathlineto{\pgfqpoint{9.491472in}{2.230247in}}%
\pgfpathlineto{\pgfqpoint{9.482666in}{2.244296in}}%
\pgfpathlineto{\pgfqpoint{9.473859in}{2.184184in}}%
\pgfpathlineto{\pgfqpoint{9.465052in}{2.089988in}}%
\pgfpathlineto{\pgfqpoint{9.456245in}{2.073999in}}%
\pgfpathlineto{\pgfqpoint{9.447438in}{2.104253in}}%
\pgfpathlineto{\pgfqpoint{9.438632in}{2.175603in}}%
\pgfpathlineto{\pgfqpoint{9.429825in}{2.185738in}}%
\pgfpathlineto{\pgfqpoint{9.421018in}{2.099837in}}%
\pgfpathlineto{\pgfqpoint{9.412211in}{2.202409in}}%
\pgfpathlineto{\pgfqpoint{9.403404in}{2.231247in}}%
\pgfpathlineto{\pgfqpoint{9.394597in}{2.197721in}}%
\pgfpathlineto{\pgfqpoint{9.385791in}{2.222503in}}%
\pgfpathlineto{\pgfqpoint{9.376984in}{2.110014in}}%
\pgfpathlineto{\pgfqpoint{9.368177in}{2.182123in}}%
\pgfpathlineto{\pgfqpoint{9.359370in}{2.211588in}}%
\pgfpathlineto{\pgfqpoint{9.350563in}{2.281432in}}%
\pgfpathlineto{\pgfqpoint{9.341757in}{2.226802in}}%
\pgfpathlineto{\pgfqpoint{9.332950in}{2.236065in}}%
\pgfpathlineto{\pgfqpoint{9.324143in}{2.225372in}}%
\pgfpathlineto{\pgfqpoint{9.315336in}{2.216693in}}%
\pgfpathlineto{\pgfqpoint{9.306529in}{2.235734in}}%
\pgfpathlineto{\pgfqpoint{9.297722in}{2.246827in}}%
\pgfpathlineto{\pgfqpoint{9.288916in}{2.254247in}}%
\pgfpathlineto{\pgfqpoint{9.280109in}{2.280895in}}%
\pgfpathlineto{\pgfqpoint{9.271302in}{2.217723in}}%
\pgfpathlineto{\pgfqpoint{9.262495in}{2.159697in}}%
\pgfpathlineto{\pgfqpoint{9.253688in}{2.281381in}}%
\pgfpathlineto{\pgfqpoint{9.244882in}{2.294989in}}%
\pgfpathlineto{\pgfqpoint{9.236075in}{2.367105in}}%
\pgfpathlineto{\pgfqpoint{9.227268in}{2.390149in}}%
\pgfpathlineto{\pgfqpoint{9.218461in}{2.254474in}}%
\pgfpathlineto{\pgfqpoint{9.209654in}{2.208731in}}%
\pgfpathlineto{\pgfqpoint{9.200847in}{2.261108in}}%
\pgfpathlineto{\pgfqpoint{9.192041in}{2.172790in}}%
\pgfpathlineto{\pgfqpoint{9.183234in}{2.141043in}}%
\pgfpathlineto{\pgfqpoint{9.174427in}{2.140419in}}%
\pgfpathlineto{\pgfqpoint{9.165620in}{2.202733in}}%
\pgfpathlineto{\pgfqpoint{9.156813in}{2.195565in}}%
\pgfpathlineto{\pgfqpoint{9.148007in}{2.245007in}}%
\pgfpathlineto{\pgfqpoint{9.139200in}{2.221665in}}%
\pgfpathlineto{\pgfqpoint{9.130393in}{2.266705in}}%
\pgfpathlineto{\pgfqpoint{9.121586in}{2.262216in}}%
\pgfpathlineto{\pgfqpoint{9.112779in}{2.229392in}}%
\pgfpathlineto{\pgfqpoint{9.103972in}{2.245122in}}%
\pgfpathlineto{\pgfqpoint{9.095166in}{2.228675in}}%
\pgfpathlineto{\pgfqpoint{9.086359in}{2.136262in}}%
\pgfpathlineto{\pgfqpoint{9.077552in}{2.074761in}}%
\pgfpathlineto{\pgfqpoint{9.068745in}{2.143686in}}%
\pgfpathlineto{\pgfqpoint{9.059938in}{2.087193in}}%
\pgfpathlineto{\pgfqpoint{9.051132in}{2.135618in}}%
\pgfpathlineto{\pgfqpoint{9.042325in}{2.128622in}}%
\pgfpathlineto{\pgfqpoint{9.033518in}{2.164265in}}%
\pgfpathlineto{\pgfqpoint{9.024711in}{2.151718in}}%
\pgfpathlineto{\pgfqpoint{9.015904in}{2.187404in}}%
\pgfpathlineto{\pgfqpoint{9.007097in}{2.113464in}}%
\pgfpathlineto{\pgfqpoint{8.998291in}{2.198092in}}%
\pgfpathlineto{\pgfqpoint{8.989484in}{2.139789in}}%
\pgfpathlineto{\pgfqpoint{8.980677in}{2.242643in}}%
\pgfpathlineto{\pgfqpoint{8.971870in}{2.161763in}}%
\pgfpathlineto{\pgfqpoint{8.963063in}{2.165749in}}%
\pgfpathlineto{\pgfqpoint{8.954257in}{2.117327in}}%
\pgfpathlineto{\pgfqpoint{8.945450in}{2.058934in}}%
\pgfpathlineto{\pgfqpoint{8.936643in}{2.066555in}}%
\pgfpathlineto{\pgfqpoint{8.927836in}{2.168701in}}%
\pgfpathlineto{\pgfqpoint{8.919029in}{2.241271in}}%
\pgfpathlineto{\pgfqpoint{8.910222in}{2.297873in}}%
\pgfpathlineto{\pgfqpoint{8.901416in}{2.306641in}}%
\pgfpathlineto{\pgfqpoint{8.892609in}{2.348131in}}%
\pgfpathlineto{\pgfqpoint{8.883802in}{2.244526in}}%
\pgfpathlineto{\pgfqpoint{8.874995in}{2.309845in}}%
\pgfpathlineto{\pgfqpoint{8.866188in}{2.263574in}}%
\pgfpathlineto{\pgfqpoint{8.857382in}{2.232265in}}%
\pgfpathlineto{\pgfqpoint{8.848575in}{2.298618in}}%
\pgfpathlineto{\pgfqpoint{8.839768in}{2.241204in}}%
\pgfpathlineto{\pgfqpoint{8.830961in}{2.206643in}}%
\pgfpathlineto{\pgfqpoint{8.822154in}{2.130808in}}%
\pgfpathlineto{\pgfqpoint{8.813347in}{2.295056in}}%
\pgfpathlineto{\pgfqpoint{8.804541in}{2.279711in}}%
\pgfpathlineto{\pgfqpoint{8.795734in}{2.245042in}}%
\pgfpathlineto{\pgfqpoint{8.786927in}{2.231140in}}%
\pgfpathlineto{\pgfqpoint{8.778120in}{2.192742in}}%
\pgfpathlineto{\pgfqpoint{8.769313in}{2.319563in}}%
\pgfpathlineto{\pgfqpoint{8.760507in}{2.176299in}}%
\pgfpathlineto{\pgfqpoint{8.751700in}{2.240321in}}%
\pgfpathlineto{\pgfqpoint{8.742893in}{2.214677in}}%
\pgfpathlineto{\pgfqpoint{8.734086in}{2.213805in}}%
\pgfpathlineto{\pgfqpoint{8.725279in}{2.204383in}}%
\pgfpathlineto{\pgfqpoint{8.716472in}{2.127997in}}%
\pgfpathlineto{\pgfqpoint{8.707666in}{2.100090in}}%
\pgfpathlineto{\pgfqpoint{8.698859in}{2.123798in}}%
\pgfpathlineto{\pgfqpoint{8.690052in}{2.213878in}}%
\pgfpathlineto{\pgfqpoint{8.681245in}{2.069929in}}%
\pgfpathlineto{\pgfqpoint{8.672438in}{2.069027in}}%
\pgfpathlineto{\pgfqpoint{8.663632in}{2.160222in}}%
\pgfpathlineto{\pgfqpoint{8.654825in}{2.143977in}}%
\pgfpathlineto{\pgfqpoint{8.646018in}{2.154520in}}%
\pgfpathlineto{\pgfqpoint{8.637211in}{2.224439in}}%
\pgfpathlineto{\pgfqpoint{8.628404in}{2.179289in}}%
\pgfpathlineto{\pgfqpoint{8.619597in}{2.142981in}}%
\pgfpathlineto{\pgfqpoint{8.610791in}{2.159283in}}%
\pgfpathlineto{\pgfqpoint{8.601984in}{2.226353in}}%
\pgfpathlineto{\pgfqpoint{8.593177in}{2.124230in}}%
\pgfpathlineto{\pgfqpoint{8.584370in}{2.109152in}}%
\pgfpathlineto{\pgfqpoint{8.575563in}{2.227433in}}%
\pgfpathlineto{\pgfqpoint{8.566757in}{2.255947in}}%
\pgfpathlineto{\pgfqpoint{8.557950in}{2.214921in}}%
\pgfpathlineto{\pgfqpoint{8.549143in}{2.290809in}}%
\pgfpathlineto{\pgfqpoint{8.540336in}{2.188020in}}%
\pgfpathlineto{\pgfqpoint{8.531529in}{2.191660in}}%
\pgfpathlineto{\pgfqpoint{8.522722in}{2.199984in}}%
\pgfpathlineto{\pgfqpoint{8.513916in}{2.260363in}}%
\pgfpathlineto{\pgfqpoint{8.505109in}{2.255767in}}%
\pgfpathlineto{\pgfqpoint{8.496302in}{2.269888in}}%
\pgfpathlineto{\pgfqpoint{8.487495in}{2.230716in}}%
\pgfpathlineto{\pgfqpoint{8.478688in}{2.216574in}}%
\pgfpathlineto{\pgfqpoint{8.469882in}{2.115874in}}%
\pgfpathlineto{\pgfqpoint{8.461075in}{2.163680in}}%
\pgfpathlineto{\pgfqpoint{8.452268in}{2.228745in}}%
\pgfpathlineto{\pgfqpoint{8.443461in}{2.219053in}}%
\pgfpathlineto{\pgfqpoint{8.434654in}{2.140976in}}%
\pgfpathlineto{\pgfqpoint{8.425847in}{2.087216in}}%
\pgfpathlineto{\pgfqpoint{8.417041in}{2.079654in}}%
\pgfpathlineto{\pgfqpoint{8.408234in}{2.062572in}}%
\pgfpathlineto{\pgfqpoint{8.399427in}{2.187272in}}%
\pgfpathlineto{\pgfqpoint{8.390620in}{2.292954in}}%
\pgfpathlineto{\pgfqpoint{8.381813in}{2.150654in}}%
\pgfpathlineto{\pgfqpoint{8.373007in}{2.157709in}}%
\pgfpathlineto{\pgfqpoint{8.364200in}{2.127626in}}%
\pgfpathlineto{\pgfqpoint{8.355393in}{2.118505in}}%
\pgfpathlineto{\pgfqpoint{8.346586in}{2.068315in}}%
\pgfpathlineto{\pgfqpoint{8.337779in}{2.110977in}}%
\pgfpathlineto{\pgfqpoint{8.328972in}{2.085612in}}%
\pgfpathlineto{\pgfqpoint{8.320166in}{2.010706in}}%
\pgfpathlineto{\pgfqpoint{8.311359in}{2.105209in}}%
\pgfpathlineto{\pgfqpoint{8.302552in}{2.255964in}}%
\pgfpathlineto{\pgfqpoint{8.293745in}{2.270300in}}%
\pgfpathlineto{\pgfqpoint{8.284938in}{2.255711in}}%
\pgfpathlineto{\pgfqpoint{8.276132in}{2.112298in}}%
\pgfpathlineto{\pgfqpoint{8.267325in}{2.116691in}}%
\pgfpathlineto{\pgfqpoint{8.258518in}{2.130589in}}%
\pgfpathlineto{\pgfqpoint{8.249711in}{2.105587in}}%
\pgfpathlineto{\pgfqpoint{8.240904in}{2.132051in}}%
\pgfpathlineto{\pgfqpoint{8.232097in}{2.112400in}}%
\pgfpathlineto{\pgfqpoint{8.223291in}{2.061286in}}%
\pgfpathlineto{\pgfqpoint{8.214484in}{2.110644in}}%
\pgfpathlineto{\pgfqpoint{8.205677in}{2.091469in}}%
\pgfpathlineto{\pgfqpoint{8.196870in}{2.078403in}}%
\pgfpathlineto{\pgfqpoint{8.188063in}{2.101981in}}%
\pgfpathlineto{\pgfqpoint{8.179257in}{2.179288in}}%
\pgfpathlineto{\pgfqpoint{8.170450in}{2.254389in}}%
\pgfpathlineto{\pgfqpoint{8.161643in}{2.201333in}}%
\pgfpathlineto{\pgfqpoint{8.152836in}{2.147005in}}%
\pgfpathlineto{\pgfqpoint{8.144029in}{2.105936in}}%
\pgfpathlineto{\pgfqpoint{8.135222in}{2.253412in}}%
\pgfpathlineto{\pgfqpoint{8.126416in}{2.300130in}}%
\pgfpathlineto{\pgfqpoint{8.117609in}{2.329002in}}%
\pgfpathlineto{\pgfqpoint{8.108802in}{2.357151in}}%
\pgfpathlineto{\pgfqpoint{8.099995in}{2.335768in}}%
\pgfpathlineto{\pgfqpoint{8.091188in}{2.210409in}}%
\pgfpathlineto{\pgfqpoint{8.082382in}{2.241651in}}%
\pgfpathlineto{\pgfqpoint{8.073575in}{2.196102in}}%
\pgfpathlineto{\pgfqpoint{8.064768in}{2.214896in}}%
\pgfpathlineto{\pgfqpoint{8.055961in}{2.201684in}}%
\pgfpathlineto{\pgfqpoint{8.047154in}{2.226418in}}%
\pgfpathlineto{\pgfqpoint{8.038347in}{2.188132in}}%
\pgfpathlineto{\pgfqpoint{8.029541in}{2.208967in}}%
\pgfpathlineto{\pgfqpoint{8.020734in}{2.263374in}}%
\pgfpathlineto{\pgfqpoint{8.011927in}{2.289991in}}%
\pgfpathlineto{\pgfqpoint{8.003120in}{2.293210in}}%
\pgfpathlineto{\pgfqpoint{7.994313in}{2.253738in}}%
\pgfpathlineto{\pgfqpoint{7.985507in}{2.208703in}}%
\pgfpathlineto{\pgfqpoint{7.976700in}{2.053334in}}%
\pgfpathlineto{\pgfqpoint{7.967893in}{2.005894in}}%
\pgfpathlineto{\pgfqpoint{7.959086in}{2.020363in}}%
\pgfpathlineto{\pgfqpoint{7.950279in}{1.985059in}}%
\pgfpathlineto{\pgfqpoint{7.941472in}{1.930778in}}%
\pgfpathlineto{\pgfqpoint{7.932666in}{2.077050in}}%
\pgfpathlineto{\pgfqpoint{7.923859in}{2.136796in}}%
\pgfpathlineto{\pgfqpoint{7.915052in}{2.048638in}}%
\pgfpathlineto{\pgfqpoint{7.906245in}{1.992927in}}%
\pgfpathlineto{\pgfqpoint{7.897438in}{2.024400in}}%
\pgfpathlineto{\pgfqpoint{7.888632in}{2.035046in}}%
\pgfpathlineto{\pgfqpoint{7.879825in}{2.123497in}}%
\pgfpathlineto{\pgfqpoint{7.871018in}{2.146942in}}%
\pgfpathlineto{\pgfqpoint{7.862211in}{2.219798in}}%
\pgfpathlineto{\pgfqpoint{7.853404in}{2.164917in}}%
\pgfpathlineto{\pgfqpoint{7.844597in}{2.149425in}}%
\pgfpathlineto{\pgfqpoint{7.835791in}{2.249263in}}%
\pgfpathlineto{\pgfqpoint{7.826984in}{2.140672in}}%
\pgfpathlineto{\pgfqpoint{7.818177in}{2.172399in}}%
\pgfpathlineto{\pgfqpoint{7.809370in}{2.250983in}}%
\pgfpathlineto{\pgfqpoint{7.800563in}{2.229918in}}%
\pgfpathlineto{\pgfqpoint{7.791757in}{2.245049in}}%
\pgfpathlineto{\pgfqpoint{7.782950in}{2.255602in}}%
\pgfpathlineto{\pgfqpoint{7.774143in}{2.223861in}}%
\pgfpathlineto{\pgfqpoint{7.765336in}{2.321938in}}%
\pgfpathlineto{\pgfqpoint{7.756529in}{2.315138in}}%
\pgfpathlineto{\pgfqpoint{7.747722in}{2.338899in}}%
\pgfpathlineto{\pgfqpoint{7.738916in}{2.249648in}}%
\pgfpathlineto{\pgfqpoint{7.730109in}{2.226643in}}%
\pgfpathlineto{\pgfqpoint{7.721302in}{2.143587in}}%
\pgfpathlineto{\pgfqpoint{7.712495in}{2.168023in}}%
\pgfpathlineto{\pgfqpoint{7.703688in}{2.188796in}}%
\pgfpathlineto{\pgfqpoint{7.694882in}{2.165202in}}%
\pgfpathlineto{\pgfqpoint{7.686075in}{2.153526in}}%
\pgfpathlineto{\pgfqpoint{7.677268in}{2.188903in}}%
\pgfpathlineto{\pgfqpoint{7.668461in}{2.234463in}}%
\pgfpathlineto{\pgfqpoint{7.659654in}{2.203880in}}%
\pgfpathlineto{\pgfqpoint{7.650847in}{2.228119in}}%
\pgfpathlineto{\pgfqpoint{7.642041in}{2.237207in}}%
\pgfpathlineto{\pgfqpoint{7.633234in}{2.286720in}}%
\pgfpathlineto{\pgfqpoint{7.624427in}{2.319613in}}%
\pgfpathlineto{\pgfqpoint{7.615620in}{2.185354in}}%
\pgfpathlineto{\pgfqpoint{7.606813in}{2.142592in}}%
\pgfpathlineto{\pgfqpoint{7.598007in}{2.204051in}}%
\pgfpathlineto{\pgfqpoint{7.589200in}{2.195000in}}%
\pgfpathlineto{\pgfqpoint{7.580393in}{2.080304in}}%
\pgfpathlineto{\pgfqpoint{7.571586in}{2.004496in}}%
\pgfpathlineto{\pgfqpoint{7.562779in}{2.157226in}}%
\pgfpathlineto{\pgfqpoint{7.553972in}{2.227430in}}%
\pgfpathlineto{\pgfqpoint{7.545166in}{2.158162in}}%
\pgfpathlineto{\pgfqpoint{7.536359in}{2.195028in}}%
\pgfpathlineto{\pgfqpoint{7.527552in}{2.101093in}}%
\pgfpathlineto{\pgfqpoint{7.518745in}{2.085092in}}%
\pgfpathlineto{\pgfqpoint{7.509938in}{2.079256in}}%
\pgfpathlineto{\pgfqpoint{7.501132in}{2.107255in}}%
\pgfpathlineto{\pgfqpoint{7.492325in}{2.151012in}}%
\pgfpathlineto{\pgfqpoint{7.483518in}{2.127407in}}%
\pgfpathlineto{\pgfqpoint{7.474711in}{2.169071in}}%
\pgfpathlineto{\pgfqpoint{7.465904in}{2.170108in}}%
\pgfpathlineto{\pgfqpoint{7.457097in}{2.227351in}}%
\pgfpathlineto{\pgfqpoint{7.448291in}{2.210589in}}%
\pgfpathlineto{\pgfqpoint{7.439484in}{2.218663in}}%
\pgfpathlineto{\pgfqpoint{7.430677in}{2.173560in}}%
\pgfpathlineto{\pgfqpoint{7.421870in}{2.252698in}}%
\pgfpathlineto{\pgfqpoint{7.413063in}{2.258618in}}%
\pgfpathlineto{\pgfqpoint{7.404257in}{2.118430in}}%
\pgfpathlineto{\pgfqpoint{7.395450in}{2.115842in}}%
\pgfpathlineto{\pgfqpoint{7.386643in}{2.109347in}}%
\pgfpathlineto{\pgfqpoint{7.377836in}{2.090293in}}%
\pgfpathlineto{\pgfqpoint{7.369029in}{2.109009in}}%
\pgfpathlineto{\pgfqpoint{7.360222in}{2.131110in}}%
\pgfpathlineto{\pgfqpoint{7.351416in}{2.297243in}}%
\pgfpathlineto{\pgfqpoint{7.342609in}{2.292386in}}%
\pgfpathlineto{\pgfqpoint{7.333802in}{2.178663in}}%
\pgfpathlineto{\pgfqpoint{7.324995in}{2.151197in}}%
\pgfpathlineto{\pgfqpoint{7.316188in}{2.138997in}}%
\pgfpathlineto{\pgfqpoint{7.307382in}{2.164925in}}%
\pgfpathlineto{\pgfqpoint{7.298575in}{2.198899in}}%
\pgfpathlineto{\pgfqpoint{7.289768in}{2.098099in}}%
\pgfpathlineto{\pgfqpoint{7.280961in}{2.091149in}}%
\pgfpathlineto{\pgfqpoint{7.272154in}{2.280864in}}%
\pgfpathlineto{\pgfqpoint{7.263347in}{2.163323in}}%
\pgfpathlineto{\pgfqpoint{7.254541in}{2.218856in}}%
\pgfpathlineto{\pgfqpoint{7.245734in}{2.160570in}}%
\pgfpathlineto{\pgfqpoint{7.236927in}{2.107973in}}%
\pgfpathlineto{\pgfqpoint{7.228120in}{2.071420in}}%
\pgfpathlineto{\pgfqpoint{7.219313in}{2.118969in}}%
\pgfpathlineto{\pgfqpoint{7.210507in}{2.153286in}}%
\pgfpathlineto{\pgfqpoint{7.201700in}{2.218670in}}%
\pgfpathlineto{\pgfqpoint{7.192893in}{2.300676in}}%
\pgfpathlineto{\pgfqpoint{7.184086in}{2.308951in}}%
\pgfpathlineto{\pgfqpoint{7.175279in}{2.279312in}}%
\pgfpathlineto{\pgfqpoint{7.166472in}{2.342509in}}%
\pgfpathlineto{\pgfqpoint{7.157666in}{2.253114in}}%
\pgfpathlineto{\pgfqpoint{7.148859in}{2.192321in}}%
\pgfpathlineto{\pgfqpoint{7.140052in}{2.198436in}}%
\pgfpathlineto{\pgfqpoint{7.131245in}{2.315647in}}%
\pgfpathlineto{\pgfqpoint{7.122438in}{2.353899in}}%
\pgfpathlineto{\pgfqpoint{7.113632in}{2.346064in}}%
\pgfpathlineto{\pgfqpoint{7.104825in}{2.336257in}}%
\pgfpathlineto{\pgfqpoint{7.096018in}{2.291284in}}%
\pgfpathlineto{\pgfqpoint{7.087211in}{2.353086in}}%
\pgfpathlineto{\pgfqpoint{7.078404in}{2.321440in}}%
\pgfpathlineto{\pgfqpoint{7.069597in}{2.368280in}}%
\pgfpathlineto{\pgfqpoint{7.060791in}{2.226359in}}%
\pgfpathlineto{\pgfqpoint{7.051984in}{2.155548in}}%
\pgfpathlineto{\pgfqpoint{7.043177in}{2.081725in}}%
\pgfpathlineto{\pgfqpoint{7.034370in}{2.167967in}}%
\pgfpathlineto{\pgfqpoint{7.025563in}{2.158373in}}%
\pgfpathlineto{\pgfqpoint{7.016757in}{2.181080in}}%
\pgfpathlineto{\pgfqpoint{7.007950in}{2.255290in}}%
\pgfpathlineto{\pgfqpoint{6.999143in}{2.266952in}}%
\pgfpathlineto{\pgfqpoint{6.990336in}{2.378804in}}%
\pgfpathlineto{\pgfqpoint{6.981529in}{2.301177in}}%
\pgfpathlineto{\pgfqpoint{6.972722in}{2.214593in}}%
\pgfpathlineto{\pgfqpoint{6.963916in}{2.246054in}}%
\pgfpathlineto{\pgfqpoint{6.955109in}{2.270446in}}%
\pgfpathlineto{\pgfqpoint{6.946302in}{2.259668in}}%
\pgfpathlineto{\pgfqpoint{6.937495in}{2.075167in}}%
\pgfpathlineto{\pgfqpoint{6.928688in}{2.174246in}}%
\pgfpathlineto{\pgfqpoint{6.919882in}{2.098945in}}%
\pgfpathlineto{\pgfqpoint{6.911075in}{2.098551in}}%
\pgfpathlineto{\pgfqpoint{6.902268in}{2.203842in}}%
\pgfpathlineto{\pgfqpoint{6.893461in}{2.264709in}}%
\pgfpathlineto{\pgfqpoint{6.884654in}{2.179109in}}%
\pgfpathlineto{\pgfqpoint{6.875847in}{2.199739in}}%
\pgfpathlineto{\pgfqpoint{6.867041in}{2.258061in}}%
\pgfpathlineto{\pgfqpoint{6.858234in}{2.271942in}}%
\pgfpathlineto{\pgfqpoint{6.849427in}{2.248540in}}%
\pgfpathlineto{\pgfqpoint{6.840620in}{2.129743in}}%
\pgfpathlineto{\pgfqpoint{6.831813in}{2.107145in}}%
\pgfpathlineto{\pgfqpoint{6.823007in}{2.068885in}}%
\pgfpathlineto{\pgfqpoint{6.814200in}{2.013022in}}%
\pgfpathlineto{\pgfqpoint{6.805393in}{2.126372in}}%
\pgfpathlineto{\pgfqpoint{6.796586in}{2.125416in}}%
\pgfpathlineto{\pgfqpoint{6.787779in}{2.205720in}}%
\pgfpathlineto{\pgfqpoint{6.778972in}{2.318705in}}%
\pgfpathlineto{\pgfqpoint{6.770166in}{2.365879in}}%
\pgfpathlineto{\pgfqpoint{6.761359in}{2.373716in}}%
\pgfpathlineto{\pgfqpoint{6.752552in}{2.383895in}}%
\pgfpathlineto{\pgfqpoint{6.743745in}{2.401332in}}%
\pgfpathlineto{\pgfqpoint{6.734938in}{2.206238in}}%
\pgfpathlineto{\pgfqpoint{6.726132in}{2.219318in}}%
\pgfpathlineto{\pgfqpoint{6.717325in}{2.130040in}}%
\pgfpathlineto{\pgfqpoint{6.708518in}{2.201597in}}%
\pgfpathlineto{\pgfqpoint{6.699711in}{2.093164in}}%
\pgfpathlineto{\pgfqpoint{6.690904in}{2.178373in}}%
\pgfpathlineto{\pgfqpoint{6.682097in}{2.176329in}}%
\pgfpathlineto{\pgfqpoint{6.673291in}{2.185164in}}%
\pgfpathlineto{\pgfqpoint{6.664484in}{2.247585in}}%
\pgfpathlineto{\pgfqpoint{6.655677in}{2.212231in}}%
\pgfpathlineto{\pgfqpoint{6.646870in}{2.217815in}}%
\pgfpathlineto{\pgfqpoint{6.638063in}{2.077441in}}%
\pgfpathlineto{\pgfqpoint{6.629257in}{2.108653in}}%
\pgfpathlineto{\pgfqpoint{6.620450in}{2.051325in}}%
\pgfpathlineto{\pgfqpoint{6.611643in}{2.067290in}}%
\pgfpathlineto{\pgfqpoint{6.602836in}{2.108268in}}%
\pgfpathlineto{\pgfqpoint{6.594029in}{2.238442in}}%
\pgfpathlineto{\pgfqpoint{6.585222in}{2.144922in}}%
\pgfpathlineto{\pgfqpoint{6.576416in}{2.114246in}}%
\pgfpathlineto{\pgfqpoint{6.567609in}{2.253690in}}%
\pgfpathlineto{\pgfqpoint{6.558802in}{2.221534in}}%
\pgfpathlineto{\pgfqpoint{6.549995in}{2.244039in}}%
\pgfpathlineto{\pgfqpoint{6.541188in}{2.180767in}}%
\pgfpathlineto{\pgfqpoint{6.532382in}{2.303602in}}%
\pgfpathlineto{\pgfqpoint{6.523575in}{2.355782in}}%
\pgfpathlineto{\pgfqpoint{6.514768in}{2.314354in}}%
\pgfpathlineto{\pgfqpoint{6.505961in}{2.432991in}}%
\pgfpathlineto{\pgfqpoint{6.497154in}{2.335111in}}%
\pgfpathlineto{\pgfqpoint{6.488347in}{2.440460in}}%
\pgfpathlineto{\pgfqpoint{6.479541in}{2.340510in}}%
\pgfpathlineto{\pgfqpoint{6.470734in}{2.284049in}}%
\pgfpathlineto{\pgfqpoint{6.461927in}{2.200099in}}%
\pgfpathlineto{\pgfqpoint{6.453120in}{2.164724in}}%
\pgfpathlineto{\pgfqpoint{6.444313in}{2.220267in}}%
\pgfpathlineto{\pgfqpoint{6.435507in}{2.170774in}}%
\pgfpathlineto{\pgfqpoint{6.426700in}{2.292574in}}%
\pgfpathlineto{\pgfqpoint{6.417893in}{2.229069in}}%
\pgfpathlineto{\pgfqpoint{6.409086in}{2.216729in}}%
\pgfpathlineto{\pgfqpoint{6.400279in}{2.266494in}}%
\pgfpathlineto{\pgfqpoint{6.391472in}{2.181021in}}%
\pgfpathlineto{\pgfqpoint{6.382666in}{2.308442in}}%
\pgfpathlineto{\pgfqpoint{6.373859in}{2.272931in}}%
\pgfpathlineto{\pgfqpoint{6.365052in}{2.331419in}}%
\pgfpathlineto{\pgfqpoint{6.356245in}{2.228954in}}%
\pgfpathlineto{\pgfqpoint{6.347438in}{2.210616in}}%
\pgfpathlineto{\pgfqpoint{6.338632in}{2.186262in}}%
\pgfpathlineto{\pgfqpoint{6.329825in}{2.154554in}}%
\pgfpathlineto{\pgfqpoint{6.321018in}{2.201033in}}%
\pgfpathlineto{\pgfqpoint{6.312211in}{2.269429in}}%
\pgfpathlineto{\pgfqpoint{6.303404in}{2.173645in}}%
\pgfpathlineto{\pgfqpoint{6.294597in}{2.274567in}}%
\pgfpathlineto{\pgfqpoint{6.285791in}{2.297845in}}%
\pgfpathlineto{\pgfqpoint{6.276984in}{2.399231in}}%
\pgfpathlineto{\pgfqpoint{6.268177in}{2.305055in}}%
\pgfpathlineto{\pgfqpoint{6.259370in}{2.208372in}}%
\pgfpathlineto{\pgfqpoint{6.250563in}{2.164501in}}%
\pgfpathlineto{\pgfqpoint{6.241757in}{2.089940in}}%
\pgfpathlineto{\pgfqpoint{6.232950in}{2.172594in}}%
\pgfpathlineto{\pgfqpoint{6.224143in}{2.123351in}}%
\pgfpathlineto{\pgfqpoint{6.215336in}{2.143165in}}%
\pgfpathlineto{\pgfqpoint{6.206529in}{2.196581in}}%
\pgfpathlineto{\pgfqpoint{6.197722in}{2.185103in}}%
\pgfpathlineto{\pgfqpoint{6.188916in}{2.158716in}}%
\pgfpathlineto{\pgfqpoint{6.180109in}{2.179628in}}%
\pgfpathlineto{\pgfqpoint{6.171302in}{2.223298in}}%
\pgfpathlineto{\pgfqpoint{6.162495in}{2.174052in}}%
\pgfpathlineto{\pgfqpoint{6.153688in}{2.168332in}}%
\pgfpathlineto{\pgfqpoint{6.144882in}{2.307697in}}%
\pgfpathlineto{\pgfqpoint{6.136075in}{2.163674in}}%
\pgfpathlineto{\pgfqpoint{6.127268in}{2.174215in}}%
\pgfpathlineto{\pgfqpoint{6.118461in}{2.148978in}}%
\pgfpathlineto{\pgfqpoint{6.109654in}{2.165645in}}%
\pgfpathlineto{\pgfqpoint{6.100847in}{2.228922in}}%
\pgfpathlineto{\pgfqpoint{6.092041in}{2.178468in}}%
\pgfpathlineto{\pgfqpoint{6.083234in}{2.161434in}}%
\pgfpathlineto{\pgfqpoint{6.074427in}{2.169136in}}%
\pgfpathlineto{\pgfqpoint{6.065620in}{2.156419in}}%
\pgfpathlineto{\pgfqpoint{6.056813in}{2.191211in}}%
\pgfpathlineto{\pgfqpoint{6.048007in}{2.232506in}}%
\pgfpathlineto{\pgfqpoint{6.039200in}{2.321443in}}%
\pgfpathlineto{\pgfqpoint{6.030393in}{2.276094in}}%
\pgfpathlineto{\pgfqpoint{6.021586in}{2.185057in}}%
\pgfpathlineto{\pgfqpoint{6.012779in}{2.126048in}}%
\pgfpathlineto{\pgfqpoint{6.003972in}{2.204012in}}%
\pgfpathlineto{\pgfqpoint{5.995166in}{2.223215in}}%
\pgfpathlineto{\pgfqpoint{5.986359in}{2.239391in}}%
\pgfpathlineto{\pgfqpoint{5.977552in}{2.188649in}}%
\pgfpathlineto{\pgfqpoint{5.968745in}{2.209721in}}%
\pgfpathlineto{\pgfqpoint{5.959938in}{2.208859in}}%
\pgfpathlineto{\pgfqpoint{5.951132in}{2.277511in}}%
\pgfpathlineto{\pgfqpoint{5.942325in}{2.146629in}}%
\pgfpathlineto{\pgfqpoint{5.933518in}{2.178454in}}%
\pgfpathlineto{\pgfqpoint{5.924711in}{2.239385in}}%
\pgfpathlineto{\pgfqpoint{5.915904in}{2.251600in}}%
\pgfpathlineto{\pgfqpoint{5.907097in}{2.290258in}}%
\pgfpathlineto{\pgfqpoint{5.898291in}{2.314615in}}%
\pgfpathlineto{\pgfqpoint{5.889484in}{2.285394in}}%
\pgfpathlineto{\pgfqpoint{5.880677in}{2.313098in}}%
\pgfpathlineto{\pgfqpoint{5.871870in}{2.362158in}}%
\pgfpathlineto{\pgfqpoint{5.863063in}{2.303043in}}%
\pgfpathlineto{\pgfqpoint{5.854257in}{2.257372in}}%
\pgfpathlineto{\pgfqpoint{5.845450in}{2.187418in}}%
\pgfpathlineto{\pgfqpoint{5.836643in}{2.087210in}}%
\pgfpathlineto{\pgfqpoint{5.827836in}{2.176116in}}%
\pgfpathlineto{\pgfqpoint{5.819029in}{2.205859in}}%
\pgfpathlineto{\pgfqpoint{5.810222in}{2.209645in}}%
\pgfpathlineto{\pgfqpoint{5.801416in}{2.120954in}}%
\pgfpathlineto{\pgfqpoint{5.792609in}{2.050080in}}%
\pgfpathlineto{\pgfqpoint{5.783802in}{2.143459in}}%
\pgfpathlineto{\pgfqpoint{5.774995in}{2.118465in}}%
\pgfpathlineto{\pgfqpoint{5.766188in}{2.123888in}}%
\pgfpathlineto{\pgfqpoint{5.757382in}{2.167615in}}%
\pgfpathlineto{\pgfqpoint{5.748575in}{2.149736in}}%
\pgfpathlineto{\pgfqpoint{5.739768in}{2.126997in}}%
\pgfpathlineto{\pgfqpoint{5.730961in}{2.129252in}}%
\pgfpathlineto{\pgfqpoint{5.722154in}{2.180078in}}%
\pgfpathlineto{\pgfqpoint{5.713347in}{2.179660in}}%
\pgfpathlineto{\pgfqpoint{5.704541in}{2.248338in}}%
\pgfpathlineto{\pgfqpoint{5.695734in}{2.118345in}}%
\pgfpathlineto{\pgfqpoint{5.686927in}{2.174921in}}%
\pgfpathlineto{\pgfqpoint{5.678120in}{2.111759in}}%
\pgfpathlineto{\pgfqpoint{5.669313in}{2.132248in}}%
\pgfpathlineto{\pgfqpoint{5.660507in}{2.197516in}}%
\pgfpathlineto{\pgfqpoint{5.651700in}{2.193046in}}%
\pgfpathlineto{\pgfqpoint{5.642893in}{2.113774in}}%
\pgfpathlineto{\pgfqpoint{5.634086in}{2.073244in}}%
\pgfpathlineto{\pgfqpoint{5.625279in}{2.243495in}}%
\pgfpathlineto{\pgfqpoint{5.616472in}{2.110878in}}%
\pgfpathlineto{\pgfqpoint{5.607666in}{2.155241in}}%
\pgfpathlineto{\pgfqpoint{5.598859in}{2.169339in}}%
\pgfpathlineto{\pgfqpoint{5.590052in}{2.172745in}}%
\pgfpathlineto{\pgfqpoint{5.581245in}{2.264493in}}%
\pgfpathlineto{\pgfqpoint{5.572438in}{2.116258in}}%
\pgfpathlineto{\pgfqpoint{5.563632in}{2.217133in}}%
\pgfpathlineto{\pgfqpoint{5.554825in}{2.241384in}}%
\pgfpathlineto{\pgfqpoint{5.546018in}{2.226013in}}%
\pgfpathlineto{\pgfqpoint{5.537211in}{2.154044in}}%
\pgfpathlineto{\pgfqpoint{5.528404in}{2.088745in}}%
\pgfpathlineto{\pgfqpoint{5.519597in}{2.105997in}}%
\pgfpathlineto{\pgfqpoint{5.510791in}{2.224412in}}%
\pgfpathlineto{\pgfqpoint{5.501984in}{2.226480in}}%
\pgfpathlineto{\pgfqpoint{5.493177in}{2.253032in}}%
\pgfpathlineto{\pgfqpoint{5.484370in}{2.189765in}}%
\pgfpathlineto{\pgfqpoint{5.475563in}{2.177836in}}%
\pgfpathlineto{\pgfqpoint{5.466757in}{2.223607in}}%
\pgfpathlineto{\pgfqpoint{5.457950in}{2.200456in}}%
\pgfpathlineto{\pgfqpoint{5.449143in}{2.145529in}}%
\pgfpathlineto{\pgfqpoint{5.440336in}{2.270971in}}%
\pgfpathlineto{\pgfqpoint{5.431529in}{2.206574in}}%
\pgfpathlineto{\pgfqpoint{5.422722in}{2.223812in}}%
\pgfpathlineto{\pgfqpoint{5.413916in}{2.278930in}}%
\pgfpathlineto{\pgfqpoint{5.405109in}{2.274536in}}%
\pgfpathlineto{\pgfqpoint{5.396302in}{2.217878in}}%
\pgfpathlineto{\pgfqpoint{5.387495in}{2.245867in}}%
\pgfpathlineto{\pgfqpoint{5.378688in}{2.162239in}}%
\pgfpathlineto{\pgfqpoint{5.369882in}{2.213713in}}%
\pgfpathlineto{\pgfqpoint{5.361075in}{2.176883in}}%
\pgfpathlineto{\pgfqpoint{5.352268in}{2.252971in}}%
\pgfpathlineto{\pgfqpoint{5.343461in}{2.157477in}}%
\pgfpathlineto{\pgfqpoint{5.334654in}{2.133479in}}%
\pgfpathlineto{\pgfqpoint{5.325847in}{2.104012in}}%
\pgfpathlineto{\pgfqpoint{5.317041in}{2.103817in}}%
\pgfpathlineto{\pgfqpoint{5.308234in}{2.088354in}}%
\pgfpathlineto{\pgfqpoint{5.299427in}{2.085360in}}%
\pgfpathlineto{\pgfqpoint{5.290620in}{2.229926in}}%
\pgfpathlineto{\pgfqpoint{5.281813in}{2.246629in}}%
\pgfpathlineto{\pgfqpoint{5.273007in}{2.314787in}}%
\pgfpathlineto{\pgfqpoint{5.264200in}{2.215882in}}%
\pgfpathlineto{\pgfqpoint{5.255393in}{2.206150in}}%
\pgfpathlineto{\pgfqpoint{5.246586in}{2.114563in}}%
\pgfpathlineto{\pgfqpoint{5.237779in}{2.128098in}}%
\pgfpathlineto{\pgfqpoint{5.228972in}{2.253055in}}%
\pgfpathlineto{\pgfqpoint{5.220166in}{2.234865in}}%
\pgfpathlineto{\pgfqpoint{5.211359in}{2.335386in}}%
\pgfpathlineto{\pgfqpoint{5.202552in}{2.191798in}}%
\pgfpathlineto{\pgfqpoint{5.193745in}{2.182145in}}%
\pgfpathlineto{\pgfqpoint{5.184938in}{2.272667in}}%
\pgfpathlineto{\pgfqpoint{5.176132in}{2.242584in}}%
\pgfpathlineto{\pgfqpoint{5.167325in}{2.198058in}}%
\pgfpathlineto{\pgfqpoint{5.158518in}{2.192908in}}%
\pgfpathlineto{\pgfqpoint{5.149711in}{2.159964in}}%
\pgfpathlineto{\pgfqpoint{5.140904in}{2.124766in}}%
\pgfpathlineto{\pgfqpoint{5.132097in}{2.072027in}}%
\pgfpathlineto{\pgfqpoint{5.123291in}{2.203853in}}%
\pgfpathlineto{\pgfqpoint{5.114484in}{2.292240in}}%
\pgfpathlineto{\pgfqpoint{5.105677in}{2.246210in}}%
\pgfpathlineto{\pgfqpoint{5.096870in}{2.263703in}}%
\pgfpathlineto{\pgfqpoint{5.088063in}{2.198103in}}%
\pgfpathlineto{\pgfqpoint{5.079257in}{2.277676in}}%
\pgfpathlineto{\pgfqpoint{5.070450in}{2.217774in}}%
\pgfpathlineto{\pgfqpoint{5.061643in}{2.205667in}}%
\pgfpathlineto{\pgfqpoint{5.052836in}{2.252571in}}%
\pgfpathlineto{\pgfqpoint{5.044029in}{2.211270in}}%
\pgfpathlineto{\pgfqpoint{5.035222in}{2.225746in}}%
\pgfpathlineto{\pgfqpoint{5.026416in}{2.230480in}}%
\pgfpathlineto{\pgfqpoint{5.017609in}{2.222783in}}%
\pgfpathlineto{\pgfqpoint{5.008802in}{2.189998in}}%
\pgfpathlineto{\pgfqpoint{4.999995in}{2.204262in}}%
\pgfpathlineto{\pgfqpoint{4.991188in}{2.237339in}}%
\pgfpathlineto{\pgfqpoint{4.982382in}{2.223295in}}%
\pgfpathlineto{\pgfqpoint{4.973575in}{2.235548in}}%
\pgfpathlineto{\pgfqpoint{4.964768in}{2.087075in}}%
\pgfpathlineto{\pgfqpoint{4.955961in}{2.146587in}}%
\pgfpathlineto{\pgfqpoint{4.947154in}{2.242572in}}%
\pgfpathlineto{\pgfqpoint{4.938347in}{2.336249in}}%
\pgfpathlineto{\pgfqpoint{4.929541in}{2.170063in}}%
\pgfpathlineto{\pgfqpoint{4.920734in}{2.170887in}}%
\pgfpathlineto{\pgfqpoint{4.911927in}{2.218806in}}%
\pgfpathlineto{\pgfqpoint{4.903120in}{2.219933in}}%
\pgfpathlineto{\pgfqpoint{4.894313in}{2.244179in}}%
\pgfpathlineto{\pgfqpoint{4.885507in}{2.170019in}}%
\pgfpathlineto{\pgfqpoint{4.876700in}{2.179888in}}%
\pgfpathlineto{\pgfqpoint{4.867893in}{2.188683in}}%
\pgfpathlineto{\pgfqpoint{4.859086in}{2.192447in}}%
\pgfpathlineto{\pgfqpoint{4.850279in}{2.143151in}}%
\pgfpathlineto{\pgfqpoint{4.841472in}{2.148262in}}%
\pgfpathlineto{\pgfqpoint{4.832666in}{2.136692in}}%
\pgfpathlineto{\pgfqpoint{4.823859in}{2.229592in}}%
\pgfpathlineto{\pgfqpoint{4.815052in}{2.209732in}}%
\pgfpathlineto{\pgfqpoint{4.806245in}{2.169915in}}%
\pgfpathlineto{\pgfqpoint{4.797438in}{2.193105in}}%
\pgfpathlineto{\pgfqpoint{4.788632in}{2.243419in}}%
\pgfpathlineto{\pgfqpoint{4.779825in}{2.282238in}}%
\pgfpathlineto{\pgfqpoint{4.771018in}{2.258595in}}%
\pgfpathlineto{\pgfqpoint{4.762211in}{2.204423in}}%
\pgfpathlineto{\pgfqpoint{4.753404in}{2.213081in}}%
\pgfpathlineto{\pgfqpoint{4.744597in}{2.131209in}}%
\pgfpathlineto{\pgfqpoint{4.735791in}{2.011522in}}%
\pgfpathlineto{\pgfqpoint{4.726984in}{1.991772in}}%
\pgfpathlineto{\pgfqpoint{4.718177in}{2.060177in}}%
\pgfpathlineto{\pgfqpoint{4.709370in}{2.108801in}}%
\pgfpathlineto{\pgfqpoint{4.700563in}{2.155106in}}%
\pgfpathlineto{\pgfqpoint{4.691757in}{2.171258in}}%
\pgfpathlineto{\pgfqpoint{4.682950in}{2.101419in}}%
\pgfpathlineto{\pgfqpoint{4.674143in}{2.203566in}}%
\pgfpathlineto{\pgfqpoint{4.665336in}{2.323512in}}%
\pgfpathlineto{\pgfqpoint{4.656529in}{2.238440in}}%
\pgfpathlineto{\pgfqpoint{4.647722in}{2.254540in}}%
\pgfpathlineto{\pgfqpoint{4.638916in}{2.168764in}}%
\pgfpathlineto{\pgfqpoint{4.630109in}{2.138266in}}%
\pgfpathlineto{\pgfqpoint{4.621302in}{2.170204in}}%
\pgfpathlineto{\pgfqpoint{4.612495in}{2.222800in}}%
\pgfpathlineto{\pgfqpoint{4.603688in}{2.247826in}}%
\pgfpathlineto{\pgfqpoint{4.594882in}{2.222277in}}%
\pgfpathlineto{\pgfqpoint{4.586075in}{2.268308in}}%
\pgfpathlineto{\pgfqpoint{4.577268in}{2.242489in}}%
\pgfpathlineto{\pgfqpoint{4.568461in}{2.262146in}}%
\pgfpathlineto{\pgfqpoint{4.559654in}{2.281454in}}%
\pgfpathlineto{\pgfqpoint{4.550847in}{2.204599in}}%
\pgfpathlineto{\pgfqpoint{4.542041in}{2.242663in}}%
\pgfpathlineto{\pgfqpoint{4.533234in}{2.282219in}}%
\pgfpathlineto{\pgfqpoint{4.524427in}{2.225136in}}%
\pgfpathlineto{\pgfqpoint{4.515620in}{2.247922in}}%
\pgfpathlineto{\pgfqpoint{4.506813in}{2.155861in}}%
\pgfpathlineto{\pgfqpoint{4.498007in}{2.104527in}}%
\pgfpathlineto{\pgfqpoint{4.489200in}{2.124791in}}%
\pgfpathlineto{\pgfqpoint{4.480393in}{2.189142in}}%
\pgfpathlineto{\pgfqpoint{4.471586in}{2.255869in}}%
\pgfpathlineto{\pgfqpoint{4.462779in}{2.260481in}}%
\pgfpathlineto{\pgfqpoint{4.453972in}{2.309682in}}%
\pgfpathlineto{\pgfqpoint{4.445166in}{2.265588in}}%
\pgfpathlineto{\pgfqpoint{4.436359in}{2.347701in}}%
\pgfpathlineto{\pgfqpoint{4.427552in}{2.346199in}}%
\pgfpathlineto{\pgfqpoint{4.418745in}{2.353401in}}%
\pgfpathlineto{\pgfqpoint{4.409938in}{2.220150in}}%
\pgfpathlineto{\pgfqpoint{4.401132in}{2.173291in}}%
\pgfpathlineto{\pgfqpoint{4.392325in}{2.266531in}}%
\pgfpathlineto{\pgfqpoint{4.383518in}{2.285420in}}%
\pgfpathlineto{\pgfqpoint{4.374711in}{2.304345in}}%
\pgfpathlineto{\pgfqpoint{4.365904in}{2.158846in}}%
\pgfpathlineto{\pgfqpoint{4.357097in}{2.140954in}}%
\pgfpathlineto{\pgfqpoint{4.348291in}{2.181806in}}%
\pgfpathlineto{\pgfqpoint{4.339484in}{2.138384in}}%
\pgfpathlineto{\pgfqpoint{4.330677in}{2.146783in}}%
\pgfpathlineto{\pgfqpoint{4.321870in}{2.105742in}}%
\pgfpathlineto{\pgfqpoint{4.313063in}{2.258916in}}%
\pgfpathlineto{\pgfqpoint{4.304257in}{2.264406in}}%
\pgfpathlineto{\pgfqpoint{4.295450in}{2.250407in}}%
\pgfpathlineto{\pgfqpoint{4.286643in}{2.250556in}}%
\pgfpathlineto{\pgfqpoint{4.277836in}{2.341283in}}%
\pgfpathlineto{\pgfqpoint{4.269029in}{2.308316in}}%
\pgfpathlineto{\pgfqpoint{4.260222in}{2.308866in}}%
\pgfpathlineto{\pgfqpoint{4.251416in}{2.274583in}}%
\pgfpathlineto{\pgfqpoint{4.242609in}{2.217442in}}%
\pgfpathlineto{\pgfqpoint{4.233802in}{2.193479in}}%
\pgfpathlineto{\pgfqpoint{4.224995in}{2.299644in}}%
\pgfpathlineto{\pgfqpoint{4.216188in}{2.280875in}}%
\pgfpathlineto{\pgfqpoint{4.207382in}{2.216776in}}%
\pgfpathlineto{\pgfqpoint{4.198575in}{2.293215in}}%
\pgfpathlineto{\pgfqpoint{4.189768in}{2.304372in}}%
\pgfpathlineto{\pgfqpoint{4.180961in}{2.207776in}}%
\pgfpathlineto{\pgfqpoint{4.172154in}{2.167829in}}%
\pgfpathlineto{\pgfqpoint{4.163347in}{2.246514in}}%
\pgfpathlineto{\pgfqpoint{4.154541in}{2.217195in}}%
\pgfpathlineto{\pgfqpoint{4.145734in}{2.192411in}}%
\pgfpathlineto{\pgfqpoint{4.136927in}{2.239551in}}%
\pgfpathlineto{\pgfqpoint{4.128120in}{2.225234in}}%
\pgfpathlineto{\pgfqpoint{4.119313in}{2.117353in}}%
\pgfpathlineto{\pgfqpoint{4.110507in}{2.108314in}}%
\pgfpathlineto{\pgfqpoint{4.101700in}{2.092216in}}%
\pgfpathlineto{\pgfqpoint{4.092893in}{2.169572in}}%
\pgfpathlineto{\pgfqpoint{4.084086in}{2.171135in}}%
\pgfpathlineto{\pgfqpoint{4.075279in}{2.182381in}}%
\pgfpathlineto{\pgfqpoint{4.066472in}{2.196664in}}%
\pgfpathlineto{\pgfqpoint{4.057666in}{2.250570in}}%
\pgfpathlineto{\pgfqpoint{4.048859in}{2.324965in}}%
\pgfpathlineto{\pgfqpoint{4.040052in}{2.290281in}}%
\pgfpathlineto{\pgfqpoint{4.031245in}{2.212446in}}%
\pgfpathlineto{\pgfqpoint{4.022438in}{2.253193in}}%
\pgfpathlineto{\pgfqpoint{4.013632in}{2.248262in}}%
\pgfpathlineto{\pgfqpoint{4.004825in}{2.252092in}}%
\pgfpathlineto{\pgfqpoint{3.996018in}{2.127382in}}%
\pgfpathlineto{\pgfqpoint{3.987211in}{2.121892in}}%
\pgfpathlineto{\pgfqpoint{3.978404in}{2.227874in}}%
\pgfpathlineto{\pgfqpoint{3.969597in}{2.216153in}}%
\pgfpathlineto{\pgfqpoint{3.960791in}{2.256212in}}%
\pgfpathlineto{\pgfqpoint{3.951984in}{2.145768in}}%
\pgfpathlineto{\pgfqpoint{3.943177in}{2.243022in}}%
\pgfpathlineto{\pgfqpoint{3.934370in}{2.178298in}}%
\pgfpathlineto{\pgfqpoint{3.925563in}{2.196178in}}%
\pgfpathlineto{\pgfqpoint{3.916757in}{2.138696in}}%
\pgfpathlineto{\pgfqpoint{3.907950in}{2.137083in}}%
\pgfpathlineto{\pgfqpoint{3.899143in}{2.105767in}}%
\pgfpathlineto{\pgfqpoint{3.890336in}{2.233538in}}%
\pgfpathlineto{\pgfqpoint{3.881529in}{2.152163in}}%
\pgfpathlineto{\pgfqpoint{3.872722in}{2.144877in}}%
\pgfpathlineto{\pgfqpoint{3.863916in}{2.232681in}}%
\pgfpathlineto{\pgfqpoint{3.855109in}{2.106179in}}%
\pgfpathlineto{\pgfqpoint{3.846302in}{2.157840in}}%
\pgfpathlineto{\pgfqpoint{3.837495in}{2.046107in}}%
\pgfpathlineto{\pgfqpoint{3.828688in}{2.079092in}}%
\pgfpathlineto{\pgfqpoint{3.819882in}{2.076669in}}%
\pgfpathlineto{\pgfqpoint{3.811075in}{2.095252in}}%
\pgfpathlineto{\pgfqpoint{3.802268in}{2.054760in}}%
\pgfpathlineto{\pgfqpoint{3.793461in}{2.255034in}}%
\pgfpathlineto{\pgfqpoint{3.784654in}{2.200686in}}%
\pgfpathlineto{\pgfqpoint{3.775847in}{2.134851in}}%
\pgfpathlineto{\pgfqpoint{3.767041in}{2.291228in}}%
\pgfpathlineto{\pgfqpoint{3.758234in}{2.303686in}}%
\pgfpathlineto{\pgfqpoint{3.749427in}{2.288245in}}%
\pgfpathlineto{\pgfqpoint{3.740620in}{2.253687in}}%
\pgfpathlineto{\pgfqpoint{3.731813in}{2.243513in}}%
\pgfpathlineto{\pgfqpoint{3.723007in}{2.207506in}}%
\pgfpathlineto{\pgfqpoint{3.714200in}{2.128413in}}%
\pgfpathlineto{\pgfqpoint{3.705393in}{2.287149in}}%
\pgfpathlineto{\pgfqpoint{3.696586in}{2.263650in}}%
\pgfpathlineto{\pgfqpoint{3.687779in}{2.262266in}}%
\pgfpathlineto{\pgfqpoint{3.678972in}{2.234196in}}%
\pgfpathlineto{\pgfqpoint{3.670166in}{2.146341in}}%
\pgfpathlineto{\pgfqpoint{3.661359in}{2.134670in}}%
\pgfpathlineto{\pgfqpoint{3.652552in}{2.161341in}}%
\pgfpathlineto{\pgfqpoint{3.643745in}{2.227777in}}%
\pgfpathlineto{\pgfqpoint{3.634938in}{2.206447in}}%
\pgfpathlineto{\pgfqpoint{3.626132in}{2.256658in}}%
\pgfpathlineto{\pgfqpoint{3.617325in}{2.237260in}}%
\pgfpathlineto{\pgfqpoint{3.608518in}{2.270005in}}%
\pgfpathlineto{\pgfqpoint{3.599711in}{2.337972in}}%
\pgfpathlineto{\pgfqpoint{3.590904in}{2.316870in}}%
\pgfpathlineto{\pgfqpoint{3.582097in}{2.306298in}}%
\pgfpathlineto{\pgfqpoint{3.573291in}{2.270610in}}%
\pgfpathlineto{\pgfqpoint{3.564484in}{2.270876in}}%
\pgfpathlineto{\pgfqpoint{3.555677in}{2.225223in}}%
\pgfpathlineto{\pgfqpoint{3.546870in}{2.130004in}}%
\pgfpathlineto{\pgfqpoint{3.538063in}{2.212996in}}%
\pgfpathlineto{\pgfqpoint{3.529257in}{2.395453in}}%
\pgfpathlineto{\pgfqpoint{3.520450in}{2.185035in}}%
\pgfpathlineto{\pgfqpoint{3.511643in}{2.204998in}}%
\pgfpathlineto{\pgfqpoint{3.502836in}{2.189721in}}%
\pgfpathlineto{\pgfqpoint{3.494029in}{2.121877in}}%
\pgfpathlineto{\pgfqpoint{3.485222in}{2.021015in}}%
\pgfpathlineto{\pgfqpoint{3.476416in}{2.087643in}}%
\pgfpathlineto{\pgfqpoint{3.467609in}{2.064735in}}%
\pgfpathlineto{\pgfqpoint{3.458802in}{2.059026in}}%
\pgfpathlineto{\pgfqpoint{3.449995in}{2.043098in}}%
\pgfpathlineto{\pgfqpoint{3.441188in}{2.126418in}}%
\pgfpathlineto{\pgfqpoint{3.432382in}{2.171307in}}%
\pgfpathlineto{\pgfqpoint{3.423575in}{2.244283in}}%
\pgfpathlineto{\pgfqpoint{3.414768in}{2.250019in}}%
\pgfpathlineto{\pgfqpoint{3.405961in}{2.426034in}}%
\pgfpathlineto{\pgfqpoint{3.397154in}{2.365559in}}%
\pgfpathlineto{\pgfqpoint{3.388347in}{2.404440in}}%
\pgfpathlineto{\pgfqpoint{3.379541in}{2.379265in}}%
\pgfpathlineto{\pgfqpoint{3.370734in}{2.329232in}}%
\pgfpathlineto{\pgfqpoint{3.361927in}{2.397986in}}%
\pgfpathlineto{\pgfqpoint{3.353120in}{2.373699in}}%
\pgfpathlineto{\pgfqpoint{3.344313in}{2.364365in}}%
\pgfpathlineto{\pgfqpoint{3.335507in}{2.391054in}}%
\pgfpathlineto{\pgfqpoint{3.326700in}{2.388755in}}%
\pgfpathlineto{\pgfqpoint{3.317893in}{2.362824in}}%
\pgfpathlineto{\pgfqpoint{3.309086in}{2.216935in}}%
\pgfpathlineto{\pgfqpoint{3.300279in}{2.223222in}}%
\pgfpathlineto{\pgfqpoint{3.291472in}{2.079831in}}%
\pgfpathlineto{\pgfqpoint{3.282666in}{2.237131in}}%
\pgfpathlineto{\pgfqpoint{3.273859in}{2.173001in}}%
\pgfpathlineto{\pgfqpoint{3.265052in}{2.245988in}}%
\pgfpathlineto{\pgfqpoint{3.256245in}{2.196071in}}%
\pgfpathlineto{\pgfqpoint{3.247438in}{2.311009in}}%
\pgfpathlineto{\pgfqpoint{3.238632in}{2.339164in}}%
\pgfpathlineto{\pgfqpoint{3.229825in}{2.321021in}}%
\pgfpathlineto{\pgfqpoint{3.221018in}{2.220601in}}%
\pgfpathlineto{\pgfqpoint{3.212211in}{2.237048in}}%
\pgfpathlineto{\pgfqpoint{3.203404in}{2.219943in}}%
\pgfpathlineto{\pgfqpoint{3.194597in}{2.314185in}}%
\pgfpathlineto{\pgfqpoint{3.185791in}{2.281471in}}%
\pgfpathlineto{\pgfqpoint{3.176984in}{2.242511in}}%
\pgfpathlineto{\pgfqpoint{3.168177in}{2.144214in}}%
\pgfpathlineto{\pgfqpoint{3.159370in}{2.185254in}}%
\pgfpathlineto{\pgfqpoint{3.150563in}{2.086968in}}%
\pgfpathlineto{\pgfqpoint{3.141757in}{2.095887in}}%
\pgfpathlineto{\pgfqpoint{3.132950in}{2.117827in}}%
\pgfpathlineto{\pgfqpoint{3.124143in}{2.251171in}}%
\pgfpathlineto{\pgfqpoint{3.115336in}{2.254921in}}%
\pgfpathlineto{\pgfqpoint{3.106529in}{2.184734in}}%
\pgfpathlineto{\pgfqpoint{3.097722in}{2.041734in}}%
\pgfpathlineto{\pgfqpoint{3.088916in}{2.152422in}}%
\pgfpathlineto{\pgfqpoint{3.080109in}{2.121193in}}%
\pgfpathlineto{\pgfqpoint{3.071302in}{2.103904in}}%
\pgfpathlineto{\pgfqpoint{3.062495in}{2.083890in}}%
\pgfpathlineto{\pgfqpoint{3.053688in}{2.098322in}}%
\pgfpathlineto{\pgfqpoint{3.044882in}{2.142969in}}%
\pgfpathlineto{\pgfqpoint{3.036075in}{2.156014in}}%
\pgfpathlineto{\pgfqpoint{3.027268in}{2.056829in}}%
\pgfpathlineto{\pgfqpoint{3.018461in}{2.085420in}}%
\pgfpathlineto{\pgfqpoint{3.009654in}{2.128945in}}%
\pgfpathlineto{\pgfqpoint{3.000847in}{2.133200in}}%
\pgfpathlineto{\pgfqpoint{2.992041in}{2.110562in}}%
\pgfpathlineto{\pgfqpoint{2.983234in}{2.162096in}}%
\pgfpathlineto{\pgfqpoint{2.974427in}{2.256622in}}%
\pgfpathlineto{\pgfqpoint{2.965620in}{2.249112in}}%
\pgfpathlineto{\pgfqpoint{2.956813in}{2.255920in}}%
\pgfpathlineto{\pgfqpoint{2.948007in}{2.168439in}}%
\pgfpathlineto{\pgfqpoint{2.939200in}{2.154896in}}%
\pgfpathlineto{\pgfqpoint{2.930393in}{2.165917in}}%
\pgfpathlineto{\pgfqpoint{2.921586in}{2.206666in}}%
\pgfpathlineto{\pgfqpoint{2.912779in}{2.243578in}}%
\pgfpathlineto{\pgfqpoint{2.903972in}{2.232287in}}%
\pgfpathlineto{\pgfqpoint{2.895166in}{2.158295in}}%
\pgfpathlineto{\pgfqpoint{2.886359in}{2.125436in}}%
\pgfpathlineto{\pgfqpoint{2.877552in}{2.149521in}}%
\pgfpathlineto{\pgfqpoint{2.868745in}{2.262488in}}%
\pgfpathlineto{\pgfqpoint{2.859938in}{2.300967in}}%
\pgfpathlineto{\pgfqpoint{2.851132in}{2.305458in}}%
\pgfpathlineto{\pgfqpoint{2.842325in}{2.244642in}}%
\pgfpathlineto{\pgfqpoint{2.833518in}{2.182654in}}%
\pgfpathlineto{\pgfqpoint{2.824711in}{2.173404in}}%
\pgfpathlineto{\pgfqpoint{2.815904in}{2.214163in}}%
\pgfpathlineto{\pgfqpoint{2.807097in}{2.176573in}}%
\pgfpathlineto{\pgfqpoint{2.798291in}{2.227907in}}%
\pgfpathlineto{\pgfqpoint{2.789484in}{2.208312in}}%
\pgfpathlineto{\pgfqpoint{2.780677in}{2.278573in}}%
\pgfpathlineto{\pgfqpoint{2.771870in}{2.173285in}}%
\pgfpathlineto{\pgfqpoint{2.763063in}{2.170378in}}%
\pgfpathlineto{\pgfqpoint{2.754257in}{2.168769in}}%
\pgfpathlineto{\pgfqpoint{2.745450in}{2.252624in}}%
\pgfpathlineto{\pgfqpoint{2.736643in}{2.352790in}}%
\pgfpathlineto{\pgfqpoint{2.727836in}{2.228975in}}%
\pgfpathlineto{\pgfqpoint{2.719029in}{2.188796in}}%
\pgfpathlineto{\pgfqpoint{2.710222in}{2.091620in}}%
\pgfpathlineto{\pgfqpoint{2.701416in}{2.170037in}}%
\pgfpathlineto{\pgfqpoint{2.692609in}{2.148136in}}%
\pgfpathlineto{\pgfqpoint{2.683802in}{2.157497in}}%
\pgfpathlineto{\pgfqpoint{2.674995in}{2.178864in}}%
\pgfpathlineto{\pgfqpoint{2.666188in}{2.136995in}}%
\pgfpathlineto{\pgfqpoint{2.657382in}{2.042371in}}%
\pgfpathlineto{\pgfqpoint{2.648575in}{2.105968in}}%
\pgfpathlineto{\pgfqpoint{2.639768in}{2.195613in}}%
\pgfpathlineto{\pgfqpoint{2.630961in}{2.247753in}}%
\pgfpathlineto{\pgfqpoint{2.622154in}{2.266075in}}%
\pgfpathlineto{\pgfqpoint{2.613347in}{2.160737in}}%
\pgfpathlineto{\pgfqpoint{2.604541in}{2.171803in}}%
\pgfpathlineto{\pgfqpoint{2.595734in}{2.169248in}}%
\pgfpathlineto{\pgfqpoint{2.586927in}{2.232280in}}%
\pgfpathlineto{\pgfqpoint{2.578120in}{2.189942in}}%
\pgfpathlineto{\pgfqpoint{2.569313in}{2.201659in}}%
\pgfpathlineto{\pgfqpoint{2.560507in}{2.305114in}}%
\pgfpathlineto{\pgfqpoint{2.551700in}{2.193512in}}%
\pgfpathlineto{\pgfqpoint{2.542893in}{2.110869in}}%
\pgfpathlineto{\pgfqpoint{2.534086in}{2.150860in}}%
\pgfpathlineto{\pgfqpoint{2.525279in}{2.169470in}}%
\pgfpathlineto{\pgfqpoint{2.516472in}{2.226640in}}%
\pgfpathlineto{\pgfqpoint{2.507666in}{2.235056in}}%
\pgfpathlineto{\pgfqpoint{2.498859in}{2.273153in}}%
\pgfpathlineto{\pgfqpoint{2.490052in}{2.197279in}}%
\pgfpathlineto{\pgfqpoint{2.481245in}{2.115886in}}%
\pgfpathlineto{\pgfqpoint{2.472438in}{2.104803in}}%
\pgfpathlineto{\pgfqpoint{2.463632in}{2.095506in}}%
\pgfpathlineto{\pgfqpoint{2.454825in}{2.226114in}}%
\pgfpathlineto{\pgfqpoint{2.446018in}{2.129352in}}%
\pgfpathlineto{\pgfqpoint{2.437211in}{2.127275in}}%
\pgfpathlineto{\pgfqpoint{2.428404in}{2.128396in}}%
\pgfpathlineto{\pgfqpoint{2.419597in}{2.094160in}}%
\pgfpathlineto{\pgfqpoint{2.410791in}{2.038226in}}%
\pgfpathlineto{\pgfqpoint{2.401984in}{2.087472in}}%
\pgfpathlineto{\pgfqpoint{2.393177in}{2.076852in}}%
\pgfpathlineto{\pgfqpoint{2.384370in}{2.236864in}}%
\pgfpathlineto{\pgfqpoint{2.375563in}{2.257915in}}%
\pgfpathlineto{\pgfqpoint{2.366757in}{2.244152in}}%
\pgfpathlineto{\pgfqpoint{2.357950in}{2.257805in}}%
\pgfpathlineto{\pgfqpoint{2.349143in}{2.240481in}}%
\pgfpathlineto{\pgfqpoint{2.340336in}{2.359512in}}%
\pgfpathlineto{\pgfqpoint{2.331529in}{2.201798in}}%
\pgfpathlineto{\pgfqpoint{2.322722in}{2.277586in}}%
\pgfpathlineto{\pgfqpoint{2.313916in}{2.167352in}}%
\pgfpathlineto{\pgfqpoint{2.305109in}{2.178452in}}%
\pgfpathlineto{\pgfqpoint{2.296302in}{2.197145in}}%
\pgfpathlineto{\pgfqpoint{2.287495in}{2.175276in}}%
\pgfpathlineto{\pgfqpoint{2.278688in}{2.267505in}}%
\pgfpathlineto{\pgfqpoint{2.269882in}{2.228613in}}%
\pgfpathlineto{\pgfqpoint{2.261075in}{2.301019in}}%
\pgfpathlineto{\pgfqpoint{2.252268in}{2.268414in}}%
\pgfpathlineto{\pgfqpoint{2.243461in}{2.249018in}}%
\pgfpathlineto{\pgfqpoint{2.234654in}{2.335104in}}%
\pgfpathlineto{\pgfqpoint{2.225847in}{2.222351in}}%
\pgfpathlineto{\pgfqpoint{2.217041in}{2.265772in}}%
\pgfpathlineto{\pgfqpoint{2.208234in}{2.297617in}}%
\pgfpathlineto{\pgfqpoint{2.199427in}{2.241783in}}%
\pgfpathlineto{\pgfqpoint{2.190620in}{2.230918in}}%
\pgfpathlineto{\pgfqpoint{2.181813in}{2.217514in}}%
\pgfpathlineto{\pgfqpoint{2.173007in}{2.199619in}}%
\pgfpathlineto{\pgfqpoint{2.164200in}{2.196284in}}%
\pgfpathlineto{\pgfqpoint{2.155393in}{2.206291in}}%
\pgfpathlineto{\pgfqpoint{2.146586in}{2.213530in}}%
\pgfpathlineto{\pgfqpoint{2.137779in}{2.216383in}}%
\pgfpathlineto{\pgfqpoint{2.128972in}{2.244296in}}%
\pgfpathlineto{\pgfqpoint{2.120166in}{2.134275in}}%
\pgfpathlineto{\pgfqpoint{2.111359in}{2.151911in}}%
\pgfpathlineto{\pgfqpoint{2.102552in}{2.241284in}}%
\pgfpathlineto{\pgfqpoint{2.093745in}{2.187083in}}%
\pgfpathlineto{\pgfqpoint{2.084938in}{2.268917in}}%
\pgfpathlineto{\pgfqpoint{2.076132in}{2.176152in}}%
\pgfpathlineto{\pgfqpoint{2.067325in}{2.069370in}}%
\pgfpathlineto{\pgfqpoint{2.058518in}{2.106495in}}%
\pgfpathlineto{\pgfqpoint{2.049711in}{2.160732in}}%
\pgfpathlineto{\pgfqpoint{2.040904in}{2.144359in}}%
\pgfpathlineto{\pgfqpoint{2.032097in}{2.160252in}}%
\pgfpathlineto{\pgfqpoint{2.023291in}{2.209766in}}%
\pgfpathlineto{\pgfqpoint{2.014484in}{2.220595in}}%
\pgfpathlineto{\pgfqpoint{2.005677in}{2.121146in}}%
\pgfpathlineto{\pgfqpoint{1.996870in}{2.162253in}}%
\pgfpathlineto{\pgfqpoint{1.988063in}{2.277704in}}%
\pgfpathlineto{\pgfqpoint{1.979257in}{2.362604in}}%
\pgfpathlineto{\pgfqpoint{1.970450in}{2.288737in}}%
\pgfpathlineto{\pgfqpoint{1.961643in}{2.311304in}}%
\pgfpathlineto{\pgfqpoint{1.952836in}{2.362228in}}%
\pgfpathlineto{\pgfqpoint{1.944029in}{2.308723in}}%
\pgfpathlineto{\pgfqpoint{1.935222in}{2.241198in}}%
\pgfpathlineto{\pgfqpoint{1.926416in}{2.250030in}}%
\pgfpathlineto{\pgfqpoint{1.917609in}{2.249254in}}%
\pgfpathlineto{\pgfqpoint{1.908802in}{2.250840in}}%
\pgfpathlineto{\pgfqpoint{1.899995in}{2.294938in}}%
\pgfpathlineto{\pgfqpoint{1.891188in}{2.172483in}}%
\pgfpathlineto{\pgfqpoint{1.882382in}{2.233755in}}%
\pgfpathlineto{\pgfqpoint{1.873575in}{2.267200in}}%
\pgfpathlineto{\pgfqpoint{1.864768in}{2.262550in}}%
\pgfpathlineto{\pgfqpoint{1.855961in}{2.171879in}}%
\pgfpathlineto{\pgfqpoint{1.847154in}{2.157326in}}%
\pgfpathlineto{\pgfqpoint{1.838347in}{2.141865in}}%
\pgfpathlineto{\pgfqpoint{1.829541in}{2.153796in}}%
\pgfpathlineto{\pgfqpoint{1.820734in}{2.124935in}}%
\pgfpathlineto{\pgfqpoint{1.811927in}{2.181243in}}%
\pgfpathlineto{\pgfqpoint{1.803120in}{2.128288in}}%
\pgfpathlineto{\pgfqpoint{1.794313in}{2.124478in}}%
\pgfpathlineto{\pgfqpoint{1.785507in}{2.071980in}}%
\pgfpathlineto{\pgfqpoint{1.776700in}{2.231365in}}%
\pgfpathlineto{\pgfqpoint{1.767893in}{2.264364in}}%
\pgfpathlineto{\pgfqpoint{1.759086in}{2.242921in}}%
\pgfpathlineto{\pgfqpoint{1.750279in}{2.298888in}}%
\pgfpathlineto{\pgfqpoint{1.741472in}{2.315242in}}%
\pgfpathlineto{\pgfqpoint{1.732666in}{2.232616in}}%
\pgfpathlineto{\pgfqpoint{1.723859in}{2.264729in}}%
\pgfpathlineto{\pgfqpoint{1.715052in}{2.247093in}}%
\pgfpathlineto{\pgfqpoint{1.706245in}{2.276184in}}%
\pgfpathlineto{\pgfqpoint{1.697438in}{2.250339in}}%
\pgfpathlineto{\pgfqpoint{1.688632in}{2.155890in}}%
\pgfpathlineto{\pgfqpoint{1.679825in}{2.190685in}}%
\pgfpathlineto{\pgfqpoint{1.671018in}{2.244813in}}%
\pgfpathlineto{\pgfqpoint{1.662211in}{2.268035in}}%
\pgfpathlineto{\pgfqpoint{1.653404in}{2.260836in}}%
\pgfpathlineto{\pgfqpoint{1.644597in}{2.151511in}}%
\pgfpathlineto{\pgfqpoint{1.635791in}{2.233538in}}%
\pgfpathlineto{\pgfqpoint{1.626984in}{2.269496in}}%
\pgfpathlineto{\pgfqpoint{1.618177in}{2.216532in}}%
\pgfpathlineto{\pgfqpoint{1.609370in}{2.151762in}}%
\pgfpathlineto{\pgfqpoint{1.600563in}{2.163885in}}%
\pgfpathlineto{\pgfqpoint{1.591757in}{2.267602in}}%
\pgfpathlineto{\pgfqpoint{1.582950in}{2.364819in}}%
\pgfpathlineto{\pgfqpoint{1.574143in}{2.335813in}}%
\pgfpathlineto{\pgfqpoint{1.565336in}{2.267104in}}%
\pgfpathlineto{\pgfqpoint{1.556529in}{2.263053in}}%
\pgfpathlineto{\pgfqpoint{1.547722in}{2.316788in}}%
\pgfpathlineto{\pgfqpoint{1.538916in}{2.384550in}}%
\pgfpathlineto{\pgfqpoint{1.530109in}{2.325857in}}%
\pgfpathlineto{\pgfqpoint{1.521302in}{2.252552in}}%
\pgfpathlineto{\pgfqpoint{1.512495in}{2.199227in}}%
\pgfpathlineto{\pgfqpoint{1.503688in}{2.185154in}}%
\pgfpathlineto{\pgfqpoint{1.494882in}{2.119857in}}%
\pgfpathlineto{\pgfqpoint{1.486075in}{2.163061in}}%
\pgfpathlineto{\pgfqpoint{1.477268in}{2.129257in}}%
\pgfpathlineto{\pgfqpoint{1.468461in}{2.194584in}}%
\pgfpathlineto{\pgfqpoint{1.459654in}{2.214255in}}%
\pgfpathlineto{\pgfqpoint{1.450847in}{2.141754in}}%
\pgfpathlineto{\pgfqpoint{1.442041in}{2.195778in}}%
\pgfpathlineto{\pgfqpoint{1.433234in}{2.070155in}}%
\pgfpathlineto{\pgfqpoint{1.424427in}{2.126776in}}%
\pgfpathlineto{\pgfqpoint{1.415620in}{2.180441in}}%
\pgfpathlineto{\pgfqpoint{1.406813in}{2.200289in}}%
\pgfpathlineto{\pgfqpoint{1.398007in}{2.241691in}}%
\pgfpathlineto{\pgfqpoint{1.389200in}{2.232029in}}%
\pgfpathlineto{\pgfqpoint{1.380393in}{2.202795in}}%
\pgfpathlineto{\pgfqpoint{1.371586in}{2.213603in}}%
\pgfpathlineto{\pgfqpoint{1.362779in}{2.260459in}}%
\pgfpathlineto{\pgfqpoint{1.353972in}{2.244082in}}%
\pgfpathlineto{\pgfqpoint{1.345166in}{2.270177in}}%
\pgfpathlineto{\pgfqpoint{1.336359in}{2.300080in}}%
\pgfpathlineto{\pgfqpoint{1.327552in}{2.327554in}}%
\pgfpathlineto{\pgfqpoint{1.318745in}{2.170135in}}%
\pgfpathlineto{\pgfqpoint{1.309938in}{2.228473in}}%
\pgfpathlineto{\pgfqpoint{1.301132in}{2.058662in}}%
\pgfpathlineto{\pgfqpoint{1.292325in}{2.197183in}}%
\pgfpathlineto{\pgfqpoint{1.283518in}{2.151410in}}%
\pgfpathlineto{\pgfqpoint{1.274711in}{2.178016in}}%
\pgfpathlineto{\pgfqpoint{1.265904in}{2.203622in}}%
\pgfpathlineto{\pgfqpoint{1.257097in}{2.188962in}}%
\pgfpathlineto{\pgfqpoint{1.248291in}{2.244625in}}%
\pgfpathlineto{\pgfqpoint{1.239484in}{2.177369in}}%
\pgfpathlineto{\pgfqpoint{1.230677in}{2.227273in}}%
\pgfpathlineto{\pgfqpoint{1.221870in}{2.252077in}}%
\pgfpathlineto{\pgfqpoint{1.213063in}{2.297359in}}%
\pgfpathlineto{\pgfqpoint{1.204257in}{2.398841in}}%
\pgfpathlineto{\pgfqpoint{1.195450in}{2.370308in}}%
\pgfpathlineto{\pgfqpoint{1.186643in}{2.375998in}}%
\pgfpathlineto{\pgfqpoint{1.177836in}{2.453966in}}%
\pgfpathlineto{\pgfqpoint{1.169029in}{2.797011in}}%
\pgfpathlineto{\pgfqpoint{1.160222in}{3.153887in}}%
\pgfpathlineto{\pgfqpoint{1.151416in}{3.723549in}}%
\pgfpathlineto{\pgfqpoint{1.142609in}{4.442509in}}%
\pgfpathlineto{\pgfqpoint{1.142609in}{4.442509in}}%
\pgfpathclose%
\pgfusepath{stroke,fill}%
}%
\begin{pgfscope}%
\pgfsys@transformshift{0.000000in}{0.000000in}%
\pgfsys@useobject{currentmarker}{}%
\end{pgfscope}%
\end{pgfscope}%
\begin{pgfscope}%
\pgfpathrectangle{\pgfqpoint{0.702268in}{0.521603in}}{\pgfqpoint{9.687500in}{4.235000in}}%
\pgfusepath{clip}%
\pgfsetrectcap%
\pgfsetroundjoin%
\pgfsetlinewidth{0.803000pt}%
\definecolor{currentstroke}{rgb}{0.690196,0.690196,0.690196}%
\pgfsetstrokecolor{currentstroke}%
\pgfsetdash{}{0pt}%
\pgfpathmoveto{\pgfqpoint{1.142609in}{0.521603in}}%
\pgfpathlineto{\pgfqpoint{1.142609in}{4.756603in}}%
\pgfusepath{stroke}%
\end{pgfscope}%
\begin{pgfscope}%
\pgfsetbuttcap%
\pgfsetroundjoin%
\definecolor{currentfill}{rgb}{0.000000,0.000000,0.000000}%
\pgfsetfillcolor{currentfill}%
\pgfsetlinewidth{0.803000pt}%
\definecolor{currentstroke}{rgb}{0.000000,0.000000,0.000000}%
\pgfsetstrokecolor{currentstroke}%
\pgfsetdash{}{0pt}%
\pgfsys@defobject{currentmarker}{\pgfqpoint{0.000000in}{-0.048611in}}{\pgfqpoint{0.000000in}{0.000000in}}{%
\pgfpathmoveto{\pgfqpoint{0.000000in}{0.000000in}}%
\pgfpathlineto{\pgfqpoint{0.000000in}{-0.048611in}}%
\pgfusepath{stroke,fill}%
}%
\begin{pgfscope}%
\pgfsys@transformshift{1.142609in}{0.521603in}%
\pgfsys@useobject{currentmarker}{}%
\end{pgfscope}%
\end{pgfscope}%
\begin{pgfscope}%
\definecolor{textcolor}{rgb}{0.000000,0.000000,0.000000}%
\pgfsetstrokecolor{textcolor}%
\pgfsetfillcolor{textcolor}%
\pgftext[x=1.142609in,y=0.424381in,,top]{\color{textcolor}\sffamily\fontsize{10.000000}{12.000000}\selectfont 0}%
\end{pgfscope}%
\begin{pgfscope}%
\pgfpathrectangle{\pgfqpoint{0.702268in}{0.521603in}}{\pgfqpoint{9.687500in}{4.235000in}}%
\pgfusepath{clip}%
\pgfsetrectcap%
\pgfsetroundjoin%
\pgfsetlinewidth{0.803000pt}%
\definecolor{currentstroke}{rgb}{0.690196,0.690196,0.690196}%
\pgfsetstrokecolor{currentstroke}%
\pgfsetdash{}{0pt}%
\pgfpathmoveto{\pgfqpoint{2.903972in}{0.521603in}}%
\pgfpathlineto{\pgfqpoint{2.903972in}{4.756603in}}%
\pgfusepath{stroke}%
\end{pgfscope}%
\begin{pgfscope}%
\pgfsetbuttcap%
\pgfsetroundjoin%
\definecolor{currentfill}{rgb}{0.000000,0.000000,0.000000}%
\pgfsetfillcolor{currentfill}%
\pgfsetlinewidth{0.803000pt}%
\definecolor{currentstroke}{rgb}{0.000000,0.000000,0.000000}%
\pgfsetstrokecolor{currentstroke}%
\pgfsetdash{}{0pt}%
\pgfsys@defobject{currentmarker}{\pgfqpoint{0.000000in}{-0.048611in}}{\pgfqpoint{0.000000in}{0.000000in}}{%
\pgfpathmoveto{\pgfqpoint{0.000000in}{0.000000in}}%
\pgfpathlineto{\pgfqpoint{0.000000in}{-0.048611in}}%
\pgfusepath{stroke,fill}%
}%
\begin{pgfscope}%
\pgfsys@transformshift{2.903972in}{0.521603in}%
\pgfsys@useobject{currentmarker}{}%
\end{pgfscope}%
\end{pgfscope}%
\begin{pgfscope}%
\definecolor{textcolor}{rgb}{0.000000,0.000000,0.000000}%
\pgfsetstrokecolor{textcolor}%
\pgfsetfillcolor{textcolor}%
\pgftext[x=2.903972in,y=0.424381in,,top]{\color{textcolor}\sffamily\fontsize{10.000000}{12.000000}\selectfont 200}%
\end{pgfscope}%
\begin{pgfscope}%
\pgfpathrectangle{\pgfqpoint{0.702268in}{0.521603in}}{\pgfqpoint{9.687500in}{4.235000in}}%
\pgfusepath{clip}%
\pgfsetrectcap%
\pgfsetroundjoin%
\pgfsetlinewidth{0.803000pt}%
\definecolor{currentstroke}{rgb}{0.690196,0.690196,0.690196}%
\pgfsetstrokecolor{currentstroke}%
\pgfsetdash{}{0pt}%
\pgfpathmoveto{\pgfqpoint{4.665336in}{0.521603in}}%
\pgfpathlineto{\pgfqpoint{4.665336in}{4.756603in}}%
\pgfusepath{stroke}%
\end{pgfscope}%
\begin{pgfscope}%
\pgfsetbuttcap%
\pgfsetroundjoin%
\definecolor{currentfill}{rgb}{0.000000,0.000000,0.000000}%
\pgfsetfillcolor{currentfill}%
\pgfsetlinewidth{0.803000pt}%
\definecolor{currentstroke}{rgb}{0.000000,0.000000,0.000000}%
\pgfsetstrokecolor{currentstroke}%
\pgfsetdash{}{0pt}%
\pgfsys@defobject{currentmarker}{\pgfqpoint{0.000000in}{-0.048611in}}{\pgfqpoint{0.000000in}{0.000000in}}{%
\pgfpathmoveto{\pgfqpoint{0.000000in}{0.000000in}}%
\pgfpathlineto{\pgfqpoint{0.000000in}{-0.048611in}}%
\pgfusepath{stroke,fill}%
}%
\begin{pgfscope}%
\pgfsys@transformshift{4.665336in}{0.521603in}%
\pgfsys@useobject{currentmarker}{}%
\end{pgfscope}%
\end{pgfscope}%
\begin{pgfscope}%
\definecolor{textcolor}{rgb}{0.000000,0.000000,0.000000}%
\pgfsetstrokecolor{textcolor}%
\pgfsetfillcolor{textcolor}%
\pgftext[x=4.665336in,y=0.424381in,,top]{\color{textcolor}\sffamily\fontsize{10.000000}{12.000000}\selectfont 400}%
\end{pgfscope}%
\begin{pgfscope}%
\pgfpathrectangle{\pgfqpoint{0.702268in}{0.521603in}}{\pgfqpoint{9.687500in}{4.235000in}}%
\pgfusepath{clip}%
\pgfsetrectcap%
\pgfsetroundjoin%
\pgfsetlinewidth{0.803000pt}%
\definecolor{currentstroke}{rgb}{0.690196,0.690196,0.690196}%
\pgfsetstrokecolor{currentstroke}%
\pgfsetdash{}{0pt}%
\pgfpathmoveto{\pgfqpoint{6.426700in}{0.521603in}}%
\pgfpathlineto{\pgfqpoint{6.426700in}{4.756603in}}%
\pgfusepath{stroke}%
\end{pgfscope}%
\begin{pgfscope}%
\pgfsetbuttcap%
\pgfsetroundjoin%
\definecolor{currentfill}{rgb}{0.000000,0.000000,0.000000}%
\pgfsetfillcolor{currentfill}%
\pgfsetlinewidth{0.803000pt}%
\definecolor{currentstroke}{rgb}{0.000000,0.000000,0.000000}%
\pgfsetstrokecolor{currentstroke}%
\pgfsetdash{}{0pt}%
\pgfsys@defobject{currentmarker}{\pgfqpoint{0.000000in}{-0.048611in}}{\pgfqpoint{0.000000in}{0.000000in}}{%
\pgfpathmoveto{\pgfqpoint{0.000000in}{0.000000in}}%
\pgfpathlineto{\pgfqpoint{0.000000in}{-0.048611in}}%
\pgfusepath{stroke,fill}%
}%
\begin{pgfscope}%
\pgfsys@transformshift{6.426700in}{0.521603in}%
\pgfsys@useobject{currentmarker}{}%
\end{pgfscope}%
\end{pgfscope}%
\begin{pgfscope}%
\definecolor{textcolor}{rgb}{0.000000,0.000000,0.000000}%
\pgfsetstrokecolor{textcolor}%
\pgfsetfillcolor{textcolor}%
\pgftext[x=6.426700in,y=0.424381in,,top]{\color{textcolor}\sffamily\fontsize{10.000000}{12.000000}\selectfont 600}%
\end{pgfscope}%
\begin{pgfscope}%
\pgfpathrectangle{\pgfqpoint{0.702268in}{0.521603in}}{\pgfqpoint{9.687500in}{4.235000in}}%
\pgfusepath{clip}%
\pgfsetrectcap%
\pgfsetroundjoin%
\pgfsetlinewidth{0.803000pt}%
\definecolor{currentstroke}{rgb}{0.690196,0.690196,0.690196}%
\pgfsetstrokecolor{currentstroke}%
\pgfsetdash{}{0pt}%
\pgfpathmoveto{\pgfqpoint{8.188063in}{0.521603in}}%
\pgfpathlineto{\pgfqpoint{8.188063in}{4.756603in}}%
\pgfusepath{stroke}%
\end{pgfscope}%
\begin{pgfscope}%
\pgfsetbuttcap%
\pgfsetroundjoin%
\definecolor{currentfill}{rgb}{0.000000,0.000000,0.000000}%
\pgfsetfillcolor{currentfill}%
\pgfsetlinewidth{0.803000pt}%
\definecolor{currentstroke}{rgb}{0.000000,0.000000,0.000000}%
\pgfsetstrokecolor{currentstroke}%
\pgfsetdash{}{0pt}%
\pgfsys@defobject{currentmarker}{\pgfqpoint{0.000000in}{-0.048611in}}{\pgfqpoint{0.000000in}{0.000000in}}{%
\pgfpathmoveto{\pgfqpoint{0.000000in}{0.000000in}}%
\pgfpathlineto{\pgfqpoint{0.000000in}{-0.048611in}}%
\pgfusepath{stroke,fill}%
}%
\begin{pgfscope}%
\pgfsys@transformshift{8.188063in}{0.521603in}%
\pgfsys@useobject{currentmarker}{}%
\end{pgfscope}%
\end{pgfscope}%
\begin{pgfscope}%
\definecolor{textcolor}{rgb}{0.000000,0.000000,0.000000}%
\pgfsetstrokecolor{textcolor}%
\pgfsetfillcolor{textcolor}%
\pgftext[x=8.188063in,y=0.424381in,,top]{\color{textcolor}\sffamily\fontsize{10.000000}{12.000000}\selectfont 800}%
\end{pgfscope}%
\begin{pgfscope}%
\pgfpathrectangle{\pgfqpoint{0.702268in}{0.521603in}}{\pgfqpoint{9.687500in}{4.235000in}}%
\pgfusepath{clip}%
\pgfsetrectcap%
\pgfsetroundjoin%
\pgfsetlinewidth{0.803000pt}%
\definecolor{currentstroke}{rgb}{0.690196,0.690196,0.690196}%
\pgfsetstrokecolor{currentstroke}%
\pgfsetdash{}{0pt}%
\pgfpathmoveto{\pgfqpoint{9.949427in}{0.521603in}}%
\pgfpathlineto{\pgfqpoint{9.949427in}{4.756603in}}%
\pgfusepath{stroke}%
\end{pgfscope}%
\begin{pgfscope}%
\pgfsetbuttcap%
\pgfsetroundjoin%
\definecolor{currentfill}{rgb}{0.000000,0.000000,0.000000}%
\pgfsetfillcolor{currentfill}%
\pgfsetlinewidth{0.803000pt}%
\definecolor{currentstroke}{rgb}{0.000000,0.000000,0.000000}%
\pgfsetstrokecolor{currentstroke}%
\pgfsetdash{}{0pt}%
\pgfsys@defobject{currentmarker}{\pgfqpoint{0.000000in}{-0.048611in}}{\pgfqpoint{0.000000in}{0.000000in}}{%
\pgfpathmoveto{\pgfqpoint{0.000000in}{0.000000in}}%
\pgfpathlineto{\pgfqpoint{0.000000in}{-0.048611in}}%
\pgfusepath{stroke,fill}%
}%
\begin{pgfscope}%
\pgfsys@transformshift{9.949427in}{0.521603in}%
\pgfsys@useobject{currentmarker}{}%
\end{pgfscope}%
\end{pgfscope}%
\begin{pgfscope}%
\definecolor{textcolor}{rgb}{0.000000,0.000000,0.000000}%
\pgfsetstrokecolor{textcolor}%
\pgfsetfillcolor{textcolor}%
\pgftext[x=9.949427in,y=0.424381in,,top]{\color{textcolor}\sffamily\fontsize{10.000000}{12.000000}\selectfont 1000}%
\end{pgfscope}%
\begin{pgfscope}%
\pgfpathrectangle{\pgfqpoint{0.702268in}{0.521603in}}{\pgfqpoint{9.687500in}{4.235000in}}%
\pgfusepath{clip}%
\pgfsetrectcap%
\pgfsetroundjoin%
\pgfsetlinewidth{0.803000pt}%
\definecolor{currentstroke}{rgb}{0.600000,0.600000,0.600000}%
\pgfsetstrokecolor{currentstroke}%
\pgfsetstrokeopacity{0.200000}%
\pgfsetdash{}{0pt}%
\pgfpathmoveto{\pgfqpoint{1.582950in}{0.521603in}}%
\pgfpathlineto{\pgfqpoint{1.582950in}{4.756603in}}%
\pgfusepath{stroke}%
\end{pgfscope}%
\begin{pgfscope}%
\pgfsetbuttcap%
\pgfsetroundjoin%
\definecolor{currentfill}{rgb}{0.000000,0.000000,0.000000}%
\pgfsetfillcolor{currentfill}%
\pgfsetlinewidth{0.602250pt}%
\definecolor{currentstroke}{rgb}{0.000000,0.000000,0.000000}%
\pgfsetstrokecolor{currentstroke}%
\pgfsetdash{}{0pt}%
\pgfsys@defobject{currentmarker}{\pgfqpoint{0.000000in}{-0.027778in}}{\pgfqpoint{0.000000in}{0.000000in}}{%
\pgfpathmoveto{\pgfqpoint{0.000000in}{0.000000in}}%
\pgfpathlineto{\pgfqpoint{0.000000in}{-0.027778in}}%
\pgfusepath{stroke,fill}%
}%
\begin{pgfscope}%
\pgfsys@transformshift{1.582950in}{0.521603in}%
\pgfsys@useobject{currentmarker}{}%
\end{pgfscope}%
\end{pgfscope}%
\begin{pgfscope}%
\pgfpathrectangle{\pgfqpoint{0.702268in}{0.521603in}}{\pgfqpoint{9.687500in}{4.235000in}}%
\pgfusepath{clip}%
\pgfsetrectcap%
\pgfsetroundjoin%
\pgfsetlinewidth{0.803000pt}%
\definecolor{currentstroke}{rgb}{0.600000,0.600000,0.600000}%
\pgfsetstrokecolor{currentstroke}%
\pgfsetstrokeopacity{0.200000}%
\pgfsetdash{}{0pt}%
\pgfpathmoveto{\pgfqpoint{2.023291in}{0.521603in}}%
\pgfpathlineto{\pgfqpoint{2.023291in}{4.756603in}}%
\pgfusepath{stroke}%
\end{pgfscope}%
\begin{pgfscope}%
\pgfsetbuttcap%
\pgfsetroundjoin%
\definecolor{currentfill}{rgb}{0.000000,0.000000,0.000000}%
\pgfsetfillcolor{currentfill}%
\pgfsetlinewidth{0.602250pt}%
\definecolor{currentstroke}{rgb}{0.000000,0.000000,0.000000}%
\pgfsetstrokecolor{currentstroke}%
\pgfsetdash{}{0pt}%
\pgfsys@defobject{currentmarker}{\pgfqpoint{0.000000in}{-0.027778in}}{\pgfqpoint{0.000000in}{0.000000in}}{%
\pgfpathmoveto{\pgfqpoint{0.000000in}{0.000000in}}%
\pgfpathlineto{\pgfqpoint{0.000000in}{-0.027778in}}%
\pgfusepath{stroke,fill}%
}%
\begin{pgfscope}%
\pgfsys@transformshift{2.023291in}{0.521603in}%
\pgfsys@useobject{currentmarker}{}%
\end{pgfscope}%
\end{pgfscope}%
\begin{pgfscope}%
\pgfpathrectangle{\pgfqpoint{0.702268in}{0.521603in}}{\pgfqpoint{9.687500in}{4.235000in}}%
\pgfusepath{clip}%
\pgfsetrectcap%
\pgfsetroundjoin%
\pgfsetlinewidth{0.803000pt}%
\definecolor{currentstroke}{rgb}{0.600000,0.600000,0.600000}%
\pgfsetstrokecolor{currentstroke}%
\pgfsetstrokeopacity{0.200000}%
\pgfsetdash{}{0pt}%
\pgfpathmoveto{\pgfqpoint{2.463632in}{0.521603in}}%
\pgfpathlineto{\pgfqpoint{2.463632in}{4.756603in}}%
\pgfusepath{stroke}%
\end{pgfscope}%
\begin{pgfscope}%
\pgfsetbuttcap%
\pgfsetroundjoin%
\definecolor{currentfill}{rgb}{0.000000,0.000000,0.000000}%
\pgfsetfillcolor{currentfill}%
\pgfsetlinewidth{0.602250pt}%
\definecolor{currentstroke}{rgb}{0.000000,0.000000,0.000000}%
\pgfsetstrokecolor{currentstroke}%
\pgfsetdash{}{0pt}%
\pgfsys@defobject{currentmarker}{\pgfqpoint{0.000000in}{-0.027778in}}{\pgfqpoint{0.000000in}{0.000000in}}{%
\pgfpathmoveto{\pgfqpoint{0.000000in}{0.000000in}}%
\pgfpathlineto{\pgfqpoint{0.000000in}{-0.027778in}}%
\pgfusepath{stroke,fill}%
}%
\begin{pgfscope}%
\pgfsys@transformshift{2.463632in}{0.521603in}%
\pgfsys@useobject{currentmarker}{}%
\end{pgfscope}%
\end{pgfscope}%
\begin{pgfscope}%
\pgfpathrectangle{\pgfqpoint{0.702268in}{0.521603in}}{\pgfqpoint{9.687500in}{4.235000in}}%
\pgfusepath{clip}%
\pgfsetrectcap%
\pgfsetroundjoin%
\pgfsetlinewidth{0.803000pt}%
\definecolor{currentstroke}{rgb}{0.600000,0.600000,0.600000}%
\pgfsetstrokecolor{currentstroke}%
\pgfsetstrokeopacity{0.200000}%
\pgfsetdash{}{0pt}%
\pgfpathmoveto{\pgfqpoint{3.344313in}{0.521603in}}%
\pgfpathlineto{\pgfqpoint{3.344313in}{4.756603in}}%
\pgfusepath{stroke}%
\end{pgfscope}%
\begin{pgfscope}%
\pgfsetbuttcap%
\pgfsetroundjoin%
\definecolor{currentfill}{rgb}{0.000000,0.000000,0.000000}%
\pgfsetfillcolor{currentfill}%
\pgfsetlinewidth{0.602250pt}%
\definecolor{currentstroke}{rgb}{0.000000,0.000000,0.000000}%
\pgfsetstrokecolor{currentstroke}%
\pgfsetdash{}{0pt}%
\pgfsys@defobject{currentmarker}{\pgfqpoint{0.000000in}{-0.027778in}}{\pgfqpoint{0.000000in}{0.000000in}}{%
\pgfpathmoveto{\pgfqpoint{0.000000in}{0.000000in}}%
\pgfpathlineto{\pgfqpoint{0.000000in}{-0.027778in}}%
\pgfusepath{stroke,fill}%
}%
\begin{pgfscope}%
\pgfsys@transformshift{3.344313in}{0.521603in}%
\pgfsys@useobject{currentmarker}{}%
\end{pgfscope}%
\end{pgfscope}%
\begin{pgfscope}%
\pgfpathrectangle{\pgfqpoint{0.702268in}{0.521603in}}{\pgfqpoint{9.687500in}{4.235000in}}%
\pgfusepath{clip}%
\pgfsetrectcap%
\pgfsetroundjoin%
\pgfsetlinewidth{0.803000pt}%
\definecolor{currentstroke}{rgb}{0.600000,0.600000,0.600000}%
\pgfsetstrokecolor{currentstroke}%
\pgfsetstrokeopacity{0.200000}%
\pgfsetdash{}{0pt}%
\pgfpathmoveto{\pgfqpoint{3.784654in}{0.521603in}}%
\pgfpathlineto{\pgfqpoint{3.784654in}{4.756603in}}%
\pgfusepath{stroke}%
\end{pgfscope}%
\begin{pgfscope}%
\pgfsetbuttcap%
\pgfsetroundjoin%
\definecolor{currentfill}{rgb}{0.000000,0.000000,0.000000}%
\pgfsetfillcolor{currentfill}%
\pgfsetlinewidth{0.602250pt}%
\definecolor{currentstroke}{rgb}{0.000000,0.000000,0.000000}%
\pgfsetstrokecolor{currentstroke}%
\pgfsetdash{}{0pt}%
\pgfsys@defobject{currentmarker}{\pgfqpoint{0.000000in}{-0.027778in}}{\pgfqpoint{0.000000in}{0.000000in}}{%
\pgfpathmoveto{\pgfqpoint{0.000000in}{0.000000in}}%
\pgfpathlineto{\pgfqpoint{0.000000in}{-0.027778in}}%
\pgfusepath{stroke,fill}%
}%
\begin{pgfscope}%
\pgfsys@transformshift{3.784654in}{0.521603in}%
\pgfsys@useobject{currentmarker}{}%
\end{pgfscope}%
\end{pgfscope}%
\begin{pgfscope}%
\pgfpathrectangle{\pgfqpoint{0.702268in}{0.521603in}}{\pgfqpoint{9.687500in}{4.235000in}}%
\pgfusepath{clip}%
\pgfsetrectcap%
\pgfsetroundjoin%
\pgfsetlinewidth{0.803000pt}%
\definecolor{currentstroke}{rgb}{0.600000,0.600000,0.600000}%
\pgfsetstrokecolor{currentstroke}%
\pgfsetstrokeopacity{0.200000}%
\pgfsetdash{}{0pt}%
\pgfpathmoveto{\pgfqpoint{4.224995in}{0.521603in}}%
\pgfpathlineto{\pgfqpoint{4.224995in}{4.756603in}}%
\pgfusepath{stroke}%
\end{pgfscope}%
\begin{pgfscope}%
\pgfsetbuttcap%
\pgfsetroundjoin%
\definecolor{currentfill}{rgb}{0.000000,0.000000,0.000000}%
\pgfsetfillcolor{currentfill}%
\pgfsetlinewidth{0.602250pt}%
\definecolor{currentstroke}{rgb}{0.000000,0.000000,0.000000}%
\pgfsetstrokecolor{currentstroke}%
\pgfsetdash{}{0pt}%
\pgfsys@defobject{currentmarker}{\pgfqpoint{0.000000in}{-0.027778in}}{\pgfqpoint{0.000000in}{0.000000in}}{%
\pgfpathmoveto{\pgfqpoint{0.000000in}{0.000000in}}%
\pgfpathlineto{\pgfqpoint{0.000000in}{-0.027778in}}%
\pgfusepath{stroke,fill}%
}%
\begin{pgfscope}%
\pgfsys@transformshift{4.224995in}{0.521603in}%
\pgfsys@useobject{currentmarker}{}%
\end{pgfscope}%
\end{pgfscope}%
\begin{pgfscope}%
\pgfpathrectangle{\pgfqpoint{0.702268in}{0.521603in}}{\pgfqpoint{9.687500in}{4.235000in}}%
\pgfusepath{clip}%
\pgfsetrectcap%
\pgfsetroundjoin%
\pgfsetlinewidth{0.803000pt}%
\definecolor{currentstroke}{rgb}{0.600000,0.600000,0.600000}%
\pgfsetstrokecolor{currentstroke}%
\pgfsetstrokeopacity{0.200000}%
\pgfsetdash{}{0pt}%
\pgfpathmoveto{\pgfqpoint{5.105677in}{0.521603in}}%
\pgfpathlineto{\pgfqpoint{5.105677in}{4.756603in}}%
\pgfusepath{stroke}%
\end{pgfscope}%
\begin{pgfscope}%
\pgfsetbuttcap%
\pgfsetroundjoin%
\definecolor{currentfill}{rgb}{0.000000,0.000000,0.000000}%
\pgfsetfillcolor{currentfill}%
\pgfsetlinewidth{0.602250pt}%
\definecolor{currentstroke}{rgb}{0.000000,0.000000,0.000000}%
\pgfsetstrokecolor{currentstroke}%
\pgfsetdash{}{0pt}%
\pgfsys@defobject{currentmarker}{\pgfqpoint{0.000000in}{-0.027778in}}{\pgfqpoint{0.000000in}{0.000000in}}{%
\pgfpathmoveto{\pgfqpoint{0.000000in}{0.000000in}}%
\pgfpathlineto{\pgfqpoint{0.000000in}{-0.027778in}}%
\pgfusepath{stroke,fill}%
}%
\begin{pgfscope}%
\pgfsys@transformshift{5.105677in}{0.521603in}%
\pgfsys@useobject{currentmarker}{}%
\end{pgfscope}%
\end{pgfscope}%
\begin{pgfscope}%
\pgfpathrectangle{\pgfqpoint{0.702268in}{0.521603in}}{\pgfqpoint{9.687500in}{4.235000in}}%
\pgfusepath{clip}%
\pgfsetrectcap%
\pgfsetroundjoin%
\pgfsetlinewidth{0.803000pt}%
\definecolor{currentstroke}{rgb}{0.600000,0.600000,0.600000}%
\pgfsetstrokecolor{currentstroke}%
\pgfsetstrokeopacity{0.200000}%
\pgfsetdash{}{0pt}%
\pgfpathmoveto{\pgfqpoint{5.546018in}{0.521603in}}%
\pgfpathlineto{\pgfqpoint{5.546018in}{4.756603in}}%
\pgfusepath{stroke}%
\end{pgfscope}%
\begin{pgfscope}%
\pgfsetbuttcap%
\pgfsetroundjoin%
\definecolor{currentfill}{rgb}{0.000000,0.000000,0.000000}%
\pgfsetfillcolor{currentfill}%
\pgfsetlinewidth{0.602250pt}%
\definecolor{currentstroke}{rgb}{0.000000,0.000000,0.000000}%
\pgfsetstrokecolor{currentstroke}%
\pgfsetdash{}{0pt}%
\pgfsys@defobject{currentmarker}{\pgfqpoint{0.000000in}{-0.027778in}}{\pgfqpoint{0.000000in}{0.000000in}}{%
\pgfpathmoveto{\pgfqpoint{0.000000in}{0.000000in}}%
\pgfpathlineto{\pgfqpoint{0.000000in}{-0.027778in}}%
\pgfusepath{stroke,fill}%
}%
\begin{pgfscope}%
\pgfsys@transformshift{5.546018in}{0.521603in}%
\pgfsys@useobject{currentmarker}{}%
\end{pgfscope}%
\end{pgfscope}%
\begin{pgfscope}%
\pgfpathrectangle{\pgfqpoint{0.702268in}{0.521603in}}{\pgfqpoint{9.687500in}{4.235000in}}%
\pgfusepath{clip}%
\pgfsetrectcap%
\pgfsetroundjoin%
\pgfsetlinewidth{0.803000pt}%
\definecolor{currentstroke}{rgb}{0.600000,0.600000,0.600000}%
\pgfsetstrokecolor{currentstroke}%
\pgfsetstrokeopacity{0.200000}%
\pgfsetdash{}{0pt}%
\pgfpathmoveto{\pgfqpoint{5.986359in}{0.521603in}}%
\pgfpathlineto{\pgfqpoint{5.986359in}{4.756603in}}%
\pgfusepath{stroke}%
\end{pgfscope}%
\begin{pgfscope}%
\pgfsetbuttcap%
\pgfsetroundjoin%
\definecolor{currentfill}{rgb}{0.000000,0.000000,0.000000}%
\pgfsetfillcolor{currentfill}%
\pgfsetlinewidth{0.602250pt}%
\definecolor{currentstroke}{rgb}{0.000000,0.000000,0.000000}%
\pgfsetstrokecolor{currentstroke}%
\pgfsetdash{}{0pt}%
\pgfsys@defobject{currentmarker}{\pgfqpoint{0.000000in}{-0.027778in}}{\pgfqpoint{0.000000in}{0.000000in}}{%
\pgfpathmoveto{\pgfqpoint{0.000000in}{0.000000in}}%
\pgfpathlineto{\pgfqpoint{0.000000in}{-0.027778in}}%
\pgfusepath{stroke,fill}%
}%
\begin{pgfscope}%
\pgfsys@transformshift{5.986359in}{0.521603in}%
\pgfsys@useobject{currentmarker}{}%
\end{pgfscope}%
\end{pgfscope}%
\begin{pgfscope}%
\pgfpathrectangle{\pgfqpoint{0.702268in}{0.521603in}}{\pgfqpoint{9.687500in}{4.235000in}}%
\pgfusepath{clip}%
\pgfsetrectcap%
\pgfsetroundjoin%
\pgfsetlinewidth{0.803000pt}%
\definecolor{currentstroke}{rgb}{0.600000,0.600000,0.600000}%
\pgfsetstrokecolor{currentstroke}%
\pgfsetstrokeopacity{0.200000}%
\pgfsetdash{}{0pt}%
\pgfpathmoveto{\pgfqpoint{6.867041in}{0.521603in}}%
\pgfpathlineto{\pgfqpoint{6.867041in}{4.756603in}}%
\pgfusepath{stroke}%
\end{pgfscope}%
\begin{pgfscope}%
\pgfsetbuttcap%
\pgfsetroundjoin%
\definecolor{currentfill}{rgb}{0.000000,0.000000,0.000000}%
\pgfsetfillcolor{currentfill}%
\pgfsetlinewidth{0.602250pt}%
\definecolor{currentstroke}{rgb}{0.000000,0.000000,0.000000}%
\pgfsetstrokecolor{currentstroke}%
\pgfsetdash{}{0pt}%
\pgfsys@defobject{currentmarker}{\pgfqpoint{0.000000in}{-0.027778in}}{\pgfqpoint{0.000000in}{0.000000in}}{%
\pgfpathmoveto{\pgfqpoint{0.000000in}{0.000000in}}%
\pgfpathlineto{\pgfqpoint{0.000000in}{-0.027778in}}%
\pgfusepath{stroke,fill}%
}%
\begin{pgfscope}%
\pgfsys@transformshift{6.867041in}{0.521603in}%
\pgfsys@useobject{currentmarker}{}%
\end{pgfscope}%
\end{pgfscope}%
\begin{pgfscope}%
\pgfpathrectangle{\pgfqpoint{0.702268in}{0.521603in}}{\pgfqpoint{9.687500in}{4.235000in}}%
\pgfusepath{clip}%
\pgfsetrectcap%
\pgfsetroundjoin%
\pgfsetlinewidth{0.803000pt}%
\definecolor{currentstroke}{rgb}{0.600000,0.600000,0.600000}%
\pgfsetstrokecolor{currentstroke}%
\pgfsetstrokeopacity{0.200000}%
\pgfsetdash{}{0pt}%
\pgfpathmoveto{\pgfqpoint{7.307382in}{0.521603in}}%
\pgfpathlineto{\pgfqpoint{7.307382in}{4.756603in}}%
\pgfusepath{stroke}%
\end{pgfscope}%
\begin{pgfscope}%
\pgfsetbuttcap%
\pgfsetroundjoin%
\definecolor{currentfill}{rgb}{0.000000,0.000000,0.000000}%
\pgfsetfillcolor{currentfill}%
\pgfsetlinewidth{0.602250pt}%
\definecolor{currentstroke}{rgb}{0.000000,0.000000,0.000000}%
\pgfsetstrokecolor{currentstroke}%
\pgfsetdash{}{0pt}%
\pgfsys@defobject{currentmarker}{\pgfqpoint{0.000000in}{-0.027778in}}{\pgfqpoint{0.000000in}{0.000000in}}{%
\pgfpathmoveto{\pgfqpoint{0.000000in}{0.000000in}}%
\pgfpathlineto{\pgfqpoint{0.000000in}{-0.027778in}}%
\pgfusepath{stroke,fill}%
}%
\begin{pgfscope}%
\pgfsys@transformshift{7.307382in}{0.521603in}%
\pgfsys@useobject{currentmarker}{}%
\end{pgfscope}%
\end{pgfscope}%
\begin{pgfscope}%
\pgfpathrectangle{\pgfqpoint{0.702268in}{0.521603in}}{\pgfqpoint{9.687500in}{4.235000in}}%
\pgfusepath{clip}%
\pgfsetrectcap%
\pgfsetroundjoin%
\pgfsetlinewidth{0.803000pt}%
\definecolor{currentstroke}{rgb}{0.600000,0.600000,0.600000}%
\pgfsetstrokecolor{currentstroke}%
\pgfsetstrokeopacity{0.200000}%
\pgfsetdash{}{0pt}%
\pgfpathmoveto{\pgfqpoint{7.747722in}{0.521603in}}%
\pgfpathlineto{\pgfqpoint{7.747722in}{4.756603in}}%
\pgfusepath{stroke}%
\end{pgfscope}%
\begin{pgfscope}%
\pgfsetbuttcap%
\pgfsetroundjoin%
\definecolor{currentfill}{rgb}{0.000000,0.000000,0.000000}%
\pgfsetfillcolor{currentfill}%
\pgfsetlinewidth{0.602250pt}%
\definecolor{currentstroke}{rgb}{0.000000,0.000000,0.000000}%
\pgfsetstrokecolor{currentstroke}%
\pgfsetdash{}{0pt}%
\pgfsys@defobject{currentmarker}{\pgfqpoint{0.000000in}{-0.027778in}}{\pgfqpoint{0.000000in}{0.000000in}}{%
\pgfpathmoveto{\pgfqpoint{0.000000in}{0.000000in}}%
\pgfpathlineto{\pgfqpoint{0.000000in}{-0.027778in}}%
\pgfusepath{stroke,fill}%
}%
\begin{pgfscope}%
\pgfsys@transformshift{7.747722in}{0.521603in}%
\pgfsys@useobject{currentmarker}{}%
\end{pgfscope}%
\end{pgfscope}%
\begin{pgfscope}%
\pgfpathrectangle{\pgfqpoint{0.702268in}{0.521603in}}{\pgfqpoint{9.687500in}{4.235000in}}%
\pgfusepath{clip}%
\pgfsetrectcap%
\pgfsetroundjoin%
\pgfsetlinewidth{0.803000pt}%
\definecolor{currentstroke}{rgb}{0.600000,0.600000,0.600000}%
\pgfsetstrokecolor{currentstroke}%
\pgfsetstrokeopacity{0.200000}%
\pgfsetdash{}{0pt}%
\pgfpathmoveto{\pgfqpoint{8.628404in}{0.521603in}}%
\pgfpathlineto{\pgfqpoint{8.628404in}{4.756603in}}%
\pgfusepath{stroke}%
\end{pgfscope}%
\begin{pgfscope}%
\pgfsetbuttcap%
\pgfsetroundjoin%
\definecolor{currentfill}{rgb}{0.000000,0.000000,0.000000}%
\pgfsetfillcolor{currentfill}%
\pgfsetlinewidth{0.602250pt}%
\definecolor{currentstroke}{rgb}{0.000000,0.000000,0.000000}%
\pgfsetstrokecolor{currentstroke}%
\pgfsetdash{}{0pt}%
\pgfsys@defobject{currentmarker}{\pgfqpoint{0.000000in}{-0.027778in}}{\pgfqpoint{0.000000in}{0.000000in}}{%
\pgfpathmoveto{\pgfqpoint{0.000000in}{0.000000in}}%
\pgfpathlineto{\pgfqpoint{0.000000in}{-0.027778in}}%
\pgfusepath{stroke,fill}%
}%
\begin{pgfscope}%
\pgfsys@transformshift{8.628404in}{0.521603in}%
\pgfsys@useobject{currentmarker}{}%
\end{pgfscope}%
\end{pgfscope}%
\begin{pgfscope}%
\pgfpathrectangle{\pgfqpoint{0.702268in}{0.521603in}}{\pgfqpoint{9.687500in}{4.235000in}}%
\pgfusepath{clip}%
\pgfsetrectcap%
\pgfsetroundjoin%
\pgfsetlinewidth{0.803000pt}%
\definecolor{currentstroke}{rgb}{0.600000,0.600000,0.600000}%
\pgfsetstrokecolor{currentstroke}%
\pgfsetstrokeopacity{0.200000}%
\pgfsetdash{}{0pt}%
\pgfpathmoveto{\pgfqpoint{9.068745in}{0.521603in}}%
\pgfpathlineto{\pgfqpoint{9.068745in}{4.756603in}}%
\pgfusepath{stroke}%
\end{pgfscope}%
\begin{pgfscope}%
\pgfsetbuttcap%
\pgfsetroundjoin%
\definecolor{currentfill}{rgb}{0.000000,0.000000,0.000000}%
\pgfsetfillcolor{currentfill}%
\pgfsetlinewidth{0.602250pt}%
\definecolor{currentstroke}{rgb}{0.000000,0.000000,0.000000}%
\pgfsetstrokecolor{currentstroke}%
\pgfsetdash{}{0pt}%
\pgfsys@defobject{currentmarker}{\pgfqpoint{0.000000in}{-0.027778in}}{\pgfqpoint{0.000000in}{0.000000in}}{%
\pgfpathmoveto{\pgfqpoint{0.000000in}{0.000000in}}%
\pgfpathlineto{\pgfqpoint{0.000000in}{-0.027778in}}%
\pgfusepath{stroke,fill}%
}%
\begin{pgfscope}%
\pgfsys@transformshift{9.068745in}{0.521603in}%
\pgfsys@useobject{currentmarker}{}%
\end{pgfscope}%
\end{pgfscope}%
\begin{pgfscope}%
\pgfpathrectangle{\pgfqpoint{0.702268in}{0.521603in}}{\pgfqpoint{9.687500in}{4.235000in}}%
\pgfusepath{clip}%
\pgfsetrectcap%
\pgfsetroundjoin%
\pgfsetlinewidth{0.803000pt}%
\definecolor{currentstroke}{rgb}{0.600000,0.600000,0.600000}%
\pgfsetstrokecolor{currentstroke}%
\pgfsetstrokeopacity{0.200000}%
\pgfsetdash{}{0pt}%
\pgfpathmoveto{\pgfqpoint{9.509086in}{0.521603in}}%
\pgfpathlineto{\pgfqpoint{9.509086in}{4.756603in}}%
\pgfusepath{stroke}%
\end{pgfscope}%
\begin{pgfscope}%
\pgfsetbuttcap%
\pgfsetroundjoin%
\definecolor{currentfill}{rgb}{0.000000,0.000000,0.000000}%
\pgfsetfillcolor{currentfill}%
\pgfsetlinewidth{0.602250pt}%
\definecolor{currentstroke}{rgb}{0.000000,0.000000,0.000000}%
\pgfsetstrokecolor{currentstroke}%
\pgfsetdash{}{0pt}%
\pgfsys@defobject{currentmarker}{\pgfqpoint{0.000000in}{-0.027778in}}{\pgfqpoint{0.000000in}{0.000000in}}{%
\pgfpathmoveto{\pgfqpoint{0.000000in}{0.000000in}}%
\pgfpathlineto{\pgfqpoint{0.000000in}{-0.027778in}}%
\pgfusepath{stroke,fill}%
}%
\begin{pgfscope}%
\pgfsys@transformshift{9.509086in}{0.521603in}%
\pgfsys@useobject{currentmarker}{}%
\end{pgfscope}%
\end{pgfscope}%
\begin{pgfscope}%
\pgfpathrectangle{\pgfqpoint{0.702268in}{0.521603in}}{\pgfqpoint{9.687500in}{4.235000in}}%
\pgfusepath{clip}%
\pgfsetrectcap%
\pgfsetroundjoin%
\pgfsetlinewidth{0.803000pt}%
\definecolor{currentstroke}{rgb}{0.600000,0.600000,0.600000}%
\pgfsetstrokecolor{currentstroke}%
\pgfsetstrokeopacity{0.200000}%
\pgfsetdash{}{0pt}%
\pgfpathmoveto{\pgfqpoint{10.389768in}{0.521603in}}%
\pgfpathlineto{\pgfqpoint{10.389768in}{4.756603in}}%
\pgfusepath{stroke}%
\end{pgfscope}%
\begin{pgfscope}%
\pgfsetbuttcap%
\pgfsetroundjoin%
\definecolor{currentfill}{rgb}{0.000000,0.000000,0.000000}%
\pgfsetfillcolor{currentfill}%
\pgfsetlinewidth{0.602250pt}%
\definecolor{currentstroke}{rgb}{0.000000,0.000000,0.000000}%
\pgfsetstrokecolor{currentstroke}%
\pgfsetdash{}{0pt}%
\pgfsys@defobject{currentmarker}{\pgfqpoint{0.000000in}{-0.027778in}}{\pgfqpoint{0.000000in}{0.000000in}}{%
\pgfpathmoveto{\pgfqpoint{0.000000in}{0.000000in}}%
\pgfpathlineto{\pgfqpoint{0.000000in}{-0.027778in}}%
\pgfusepath{stroke,fill}%
}%
\begin{pgfscope}%
\pgfsys@transformshift{10.389768in}{0.521603in}%
\pgfsys@useobject{currentmarker}{}%
\end{pgfscope}%
\end{pgfscope}%
\begin{pgfscope}%
\definecolor{textcolor}{rgb}{0.000000,0.000000,0.000000}%
\pgfsetstrokecolor{textcolor}%
\pgfsetfillcolor{textcolor}%
\pgftext[x=5.546018in,y=0.234413in,,top]{\color{textcolor}\sffamily\fontsize{10.000000}{12.000000}\selectfont time step \(\displaystyle t\)}%
\end{pgfscope}%
\begin{pgfscope}%
\pgfpathrectangle{\pgfqpoint{0.702268in}{0.521603in}}{\pgfqpoint{9.687500in}{4.235000in}}%
\pgfusepath{clip}%
\pgfsetrectcap%
\pgfsetroundjoin%
\pgfsetlinewidth{0.803000pt}%
\definecolor{currentstroke}{rgb}{0.690196,0.690196,0.690196}%
\pgfsetstrokecolor{currentstroke}%
\pgfsetdash{}{0pt}%
\pgfpathmoveto{\pgfqpoint{0.702268in}{0.578221in}}%
\pgfpathlineto{\pgfqpoint{10.389768in}{0.578221in}}%
\pgfusepath{stroke}%
\end{pgfscope}%
\begin{pgfscope}%
\pgfsetbuttcap%
\pgfsetroundjoin%
\definecolor{currentfill}{rgb}{0.000000,0.000000,0.000000}%
\pgfsetfillcolor{currentfill}%
\pgfsetlinewidth{0.803000pt}%
\definecolor{currentstroke}{rgb}{0.000000,0.000000,0.000000}%
\pgfsetstrokecolor{currentstroke}%
\pgfsetdash{}{0pt}%
\pgfsys@defobject{currentmarker}{\pgfqpoint{-0.048611in}{0.000000in}}{\pgfqpoint{-0.000000in}{0.000000in}}{%
\pgfpathmoveto{\pgfqpoint{-0.000000in}{0.000000in}}%
\pgfpathlineto{\pgfqpoint{-0.048611in}{0.000000in}}%
\pgfusepath{stroke,fill}%
}%
\begin{pgfscope}%
\pgfsys@transformshift{0.702268in}{0.578221in}%
\pgfsys@useobject{currentmarker}{}%
\end{pgfscope}%
\end{pgfscope}%
\begin{pgfscope}%
\definecolor{textcolor}{rgb}{0.000000,0.000000,0.000000}%
\pgfsetstrokecolor{textcolor}%
\pgfsetfillcolor{textcolor}%
\pgftext[x=0.295801in, y=0.525459in, left, base]{\color{textcolor}\sffamily\fontsize{10.000000}{12.000000}\selectfont 0.20}%
\end{pgfscope}%
\begin{pgfscope}%
\pgfpathrectangle{\pgfqpoint{0.702268in}{0.521603in}}{\pgfqpoint{9.687500in}{4.235000in}}%
\pgfusepath{clip}%
\pgfsetrectcap%
\pgfsetroundjoin%
\pgfsetlinewidth{0.803000pt}%
\definecolor{currentstroke}{rgb}{0.690196,0.690196,0.690196}%
\pgfsetstrokecolor{currentstroke}%
\pgfsetdash{}{0pt}%
\pgfpathmoveto{\pgfqpoint{0.702268in}{1.140416in}}%
\pgfpathlineto{\pgfqpoint{10.389768in}{1.140416in}}%
\pgfusepath{stroke}%
\end{pgfscope}%
\begin{pgfscope}%
\pgfsetbuttcap%
\pgfsetroundjoin%
\definecolor{currentfill}{rgb}{0.000000,0.000000,0.000000}%
\pgfsetfillcolor{currentfill}%
\pgfsetlinewidth{0.803000pt}%
\definecolor{currentstroke}{rgb}{0.000000,0.000000,0.000000}%
\pgfsetstrokecolor{currentstroke}%
\pgfsetdash{}{0pt}%
\pgfsys@defobject{currentmarker}{\pgfqpoint{-0.048611in}{0.000000in}}{\pgfqpoint{-0.000000in}{0.000000in}}{%
\pgfpathmoveto{\pgfqpoint{-0.000000in}{0.000000in}}%
\pgfpathlineto{\pgfqpoint{-0.048611in}{0.000000in}}%
\pgfusepath{stroke,fill}%
}%
\begin{pgfscope}%
\pgfsys@transformshift{0.702268in}{1.140416in}%
\pgfsys@useobject{currentmarker}{}%
\end{pgfscope}%
\end{pgfscope}%
\begin{pgfscope}%
\definecolor{textcolor}{rgb}{0.000000,0.000000,0.000000}%
\pgfsetstrokecolor{textcolor}%
\pgfsetfillcolor{textcolor}%
\pgftext[x=0.295801in, y=1.087655in, left, base]{\color{textcolor}\sffamily\fontsize{10.000000}{12.000000}\selectfont 0.22}%
\end{pgfscope}%
\begin{pgfscope}%
\pgfpathrectangle{\pgfqpoint{0.702268in}{0.521603in}}{\pgfqpoint{9.687500in}{4.235000in}}%
\pgfusepath{clip}%
\pgfsetrectcap%
\pgfsetroundjoin%
\pgfsetlinewidth{0.803000pt}%
\definecolor{currentstroke}{rgb}{0.690196,0.690196,0.690196}%
\pgfsetstrokecolor{currentstroke}%
\pgfsetdash{}{0pt}%
\pgfpathmoveto{\pgfqpoint{0.702268in}{1.702612in}}%
\pgfpathlineto{\pgfqpoint{10.389768in}{1.702612in}}%
\pgfusepath{stroke}%
\end{pgfscope}%
\begin{pgfscope}%
\pgfsetbuttcap%
\pgfsetroundjoin%
\definecolor{currentfill}{rgb}{0.000000,0.000000,0.000000}%
\pgfsetfillcolor{currentfill}%
\pgfsetlinewidth{0.803000pt}%
\definecolor{currentstroke}{rgb}{0.000000,0.000000,0.000000}%
\pgfsetstrokecolor{currentstroke}%
\pgfsetdash{}{0pt}%
\pgfsys@defobject{currentmarker}{\pgfqpoint{-0.048611in}{0.000000in}}{\pgfqpoint{-0.000000in}{0.000000in}}{%
\pgfpathmoveto{\pgfqpoint{-0.000000in}{0.000000in}}%
\pgfpathlineto{\pgfqpoint{-0.048611in}{0.000000in}}%
\pgfusepath{stroke,fill}%
}%
\begin{pgfscope}%
\pgfsys@transformshift{0.702268in}{1.702612in}%
\pgfsys@useobject{currentmarker}{}%
\end{pgfscope}%
\end{pgfscope}%
\begin{pgfscope}%
\definecolor{textcolor}{rgb}{0.000000,0.000000,0.000000}%
\pgfsetstrokecolor{textcolor}%
\pgfsetfillcolor{textcolor}%
\pgftext[x=0.295801in, y=1.649850in, left, base]{\color{textcolor}\sffamily\fontsize{10.000000}{12.000000}\selectfont 0.24}%
\end{pgfscope}%
\begin{pgfscope}%
\pgfpathrectangle{\pgfqpoint{0.702268in}{0.521603in}}{\pgfqpoint{9.687500in}{4.235000in}}%
\pgfusepath{clip}%
\pgfsetrectcap%
\pgfsetroundjoin%
\pgfsetlinewidth{0.803000pt}%
\definecolor{currentstroke}{rgb}{0.690196,0.690196,0.690196}%
\pgfsetstrokecolor{currentstroke}%
\pgfsetdash{}{0pt}%
\pgfpathmoveto{\pgfqpoint{0.702268in}{2.264808in}}%
\pgfpathlineto{\pgfqpoint{10.389768in}{2.264808in}}%
\pgfusepath{stroke}%
\end{pgfscope}%
\begin{pgfscope}%
\pgfsetbuttcap%
\pgfsetroundjoin%
\definecolor{currentfill}{rgb}{0.000000,0.000000,0.000000}%
\pgfsetfillcolor{currentfill}%
\pgfsetlinewidth{0.803000pt}%
\definecolor{currentstroke}{rgb}{0.000000,0.000000,0.000000}%
\pgfsetstrokecolor{currentstroke}%
\pgfsetdash{}{0pt}%
\pgfsys@defobject{currentmarker}{\pgfqpoint{-0.048611in}{0.000000in}}{\pgfqpoint{-0.000000in}{0.000000in}}{%
\pgfpathmoveto{\pgfqpoint{-0.000000in}{0.000000in}}%
\pgfpathlineto{\pgfqpoint{-0.048611in}{0.000000in}}%
\pgfusepath{stroke,fill}%
}%
\begin{pgfscope}%
\pgfsys@transformshift{0.702268in}{2.264808in}%
\pgfsys@useobject{currentmarker}{}%
\end{pgfscope}%
\end{pgfscope}%
\begin{pgfscope}%
\definecolor{textcolor}{rgb}{0.000000,0.000000,0.000000}%
\pgfsetstrokecolor{textcolor}%
\pgfsetfillcolor{textcolor}%
\pgftext[x=0.295801in, y=2.212046in, left, base]{\color{textcolor}\sffamily\fontsize{10.000000}{12.000000}\selectfont 0.26}%
\end{pgfscope}%
\begin{pgfscope}%
\pgfpathrectangle{\pgfqpoint{0.702268in}{0.521603in}}{\pgfqpoint{9.687500in}{4.235000in}}%
\pgfusepath{clip}%
\pgfsetrectcap%
\pgfsetroundjoin%
\pgfsetlinewidth{0.803000pt}%
\definecolor{currentstroke}{rgb}{0.690196,0.690196,0.690196}%
\pgfsetstrokecolor{currentstroke}%
\pgfsetdash{}{0pt}%
\pgfpathmoveto{\pgfqpoint{0.702268in}{2.827003in}}%
\pgfpathlineto{\pgfqpoint{10.389768in}{2.827003in}}%
\pgfusepath{stroke}%
\end{pgfscope}%
\begin{pgfscope}%
\pgfsetbuttcap%
\pgfsetroundjoin%
\definecolor{currentfill}{rgb}{0.000000,0.000000,0.000000}%
\pgfsetfillcolor{currentfill}%
\pgfsetlinewidth{0.803000pt}%
\definecolor{currentstroke}{rgb}{0.000000,0.000000,0.000000}%
\pgfsetstrokecolor{currentstroke}%
\pgfsetdash{}{0pt}%
\pgfsys@defobject{currentmarker}{\pgfqpoint{-0.048611in}{0.000000in}}{\pgfqpoint{-0.000000in}{0.000000in}}{%
\pgfpathmoveto{\pgfqpoint{-0.000000in}{0.000000in}}%
\pgfpathlineto{\pgfqpoint{-0.048611in}{0.000000in}}%
\pgfusepath{stroke,fill}%
}%
\begin{pgfscope}%
\pgfsys@transformshift{0.702268in}{2.827003in}%
\pgfsys@useobject{currentmarker}{}%
\end{pgfscope}%
\end{pgfscope}%
\begin{pgfscope}%
\definecolor{textcolor}{rgb}{0.000000,0.000000,0.000000}%
\pgfsetstrokecolor{textcolor}%
\pgfsetfillcolor{textcolor}%
\pgftext[x=0.295801in, y=2.774242in, left, base]{\color{textcolor}\sffamily\fontsize{10.000000}{12.000000}\selectfont 0.28}%
\end{pgfscope}%
\begin{pgfscope}%
\pgfpathrectangle{\pgfqpoint{0.702268in}{0.521603in}}{\pgfqpoint{9.687500in}{4.235000in}}%
\pgfusepath{clip}%
\pgfsetrectcap%
\pgfsetroundjoin%
\pgfsetlinewidth{0.803000pt}%
\definecolor{currentstroke}{rgb}{0.690196,0.690196,0.690196}%
\pgfsetstrokecolor{currentstroke}%
\pgfsetdash{}{0pt}%
\pgfpathmoveto{\pgfqpoint{0.702268in}{3.389199in}}%
\pgfpathlineto{\pgfqpoint{10.389768in}{3.389199in}}%
\pgfusepath{stroke}%
\end{pgfscope}%
\begin{pgfscope}%
\pgfsetbuttcap%
\pgfsetroundjoin%
\definecolor{currentfill}{rgb}{0.000000,0.000000,0.000000}%
\pgfsetfillcolor{currentfill}%
\pgfsetlinewidth{0.803000pt}%
\definecolor{currentstroke}{rgb}{0.000000,0.000000,0.000000}%
\pgfsetstrokecolor{currentstroke}%
\pgfsetdash{}{0pt}%
\pgfsys@defobject{currentmarker}{\pgfqpoint{-0.048611in}{0.000000in}}{\pgfqpoint{-0.000000in}{0.000000in}}{%
\pgfpathmoveto{\pgfqpoint{-0.000000in}{0.000000in}}%
\pgfpathlineto{\pgfqpoint{-0.048611in}{0.000000in}}%
\pgfusepath{stroke,fill}%
}%
\begin{pgfscope}%
\pgfsys@transformshift{0.702268in}{3.389199in}%
\pgfsys@useobject{currentmarker}{}%
\end{pgfscope}%
\end{pgfscope}%
\begin{pgfscope}%
\definecolor{textcolor}{rgb}{0.000000,0.000000,0.000000}%
\pgfsetstrokecolor{textcolor}%
\pgfsetfillcolor{textcolor}%
\pgftext[x=0.295801in, y=3.336437in, left, base]{\color{textcolor}\sffamily\fontsize{10.000000}{12.000000}\selectfont 0.30}%
\end{pgfscope}%
\begin{pgfscope}%
\pgfpathrectangle{\pgfqpoint{0.702268in}{0.521603in}}{\pgfqpoint{9.687500in}{4.235000in}}%
\pgfusepath{clip}%
\pgfsetrectcap%
\pgfsetroundjoin%
\pgfsetlinewidth{0.803000pt}%
\definecolor{currentstroke}{rgb}{0.690196,0.690196,0.690196}%
\pgfsetstrokecolor{currentstroke}%
\pgfsetdash{}{0pt}%
\pgfpathmoveto{\pgfqpoint{0.702268in}{3.951394in}}%
\pgfpathlineto{\pgfqpoint{10.389768in}{3.951394in}}%
\pgfusepath{stroke}%
\end{pgfscope}%
\begin{pgfscope}%
\pgfsetbuttcap%
\pgfsetroundjoin%
\definecolor{currentfill}{rgb}{0.000000,0.000000,0.000000}%
\pgfsetfillcolor{currentfill}%
\pgfsetlinewidth{0.803000pt}%
\definecolor{currentstroke}{rgb}{0.000000,0.000000,0.000000}%
\pgfsetstrokecolor{currentstroke}%
\pgfsetdash{}{0pt}%
\pgfsys@defobject{currentmarker}{\pgfqpoint{-0.048611in}{0.000000in}}{\pgfqpoint{-0.000000in}{0.000000in}}{%
\pgfpathmoveto{\pgfqpoint{-0.000000in}{0.000000in}}%
\pgfpathlineto{\pgfqpoint{-0.048611in}{0.000000in}}%
\pgfusepath{stroke,fill}%
}%
\begin{pgfscope}%
\pgfsys@transformshift{0.702268in}{3.951394in}%
\pgfsys@useobject{currentmarker}{}%
\end{pgfscope}%
\end{pgfscope}%
\begin{pgfscope}%
\definecolor{textcolor}{rgb}{0.000000,0.000000,0.000000}%
\pgfsetstrokecolor{textcolor}%
\pgfsetfillcolor{textcolor}%
\pgftext[x=0.295801in, y=3.898633in, left, base]{\color{textcolor}\sffamily\fontsize{10.000000}{12.000000}\selectfont 0.32}%
\end{pgfscope}%
\begin{pgfscope}%
\pgfpathrectangle{\pgfqpoint{0.702268in}{0.521603in}}{\pgfqpoint{9.687500in}{4.235000in}}%
\pgfusepath{clip}%
\pgfsetrectcap%
\pgfsetroundjoin%
\pgfsetlinewidth{0.803000pt}%
\definecolor{currentstroke}{rgb}{0.690196,0.690196,0.690196}%
\pgfsetstrokecolor{currentstroke}%
\pgfsetdash{}{0pt}%
\pgfpathmoveto{\pgfqpoint{0.702268in}{4.513590in}}%
\pgfpathlineto{\pgfqpoint{10.389768in}{4.513590in}}%
\pgfusepath{stroke}%
\end{pgfscope}%
\begin{pgfscope}%
\pgfsetbuttcap%
\pgfsetroundjoin%
\definecolor{currentfill}{rgb}{0.000000,0.000000,0.000000}%
\pgfsetfillcolor{currentfill}%
\pgfsetlinewidth{0.803000pt}%
\definecolor{currentstroke}{rgb}{0.000000,0.000000,0.000000}%
\pgfsetstrokecolor{currentstroke}%
\pgfsetdash{}{0pt}%
\pgfsys@defobject{currentmarker}{\pgfqpoint{-0.048611in}{0.000000in}}{\pgfqpoint{-0.000000in}{0.000000in}}{%
\pgfpathmoveto{\pgfqpoint{-0.000000in}{0.000000in}}%
\pgfpathlineto{\pgfqpoint{-0.048611in}{0.000000in}}%
\pgfusepath{stroke,fill}%
}%
\begin{pgfscope}%
\pgfsys@transformshift{0.702268in}{4.513590in}%
\pgfsys@useobject{currentmarker}{}%
\end{pgfscope}%
\end{pgfscope}%
\begin{pgfscope}%
\definecolor{textcolor}{rgb}{0.000000,0.000000,0.000000}%
\pgfsetstrokecolor{textcolor}%
\pgfsetfillcolor{textcolor}%
\pgftext[x=0.295801in, y=4.460829in, left, base]{\color{textcolor}\sffamily\fontsize{10.000000}{12.000000}\selectfont 0.34}%
\end{pgfscope}%
\begin{pgfscope}%
\pgfpathrectangle{\pgfqpoint{0.702268in}{0.521603in}}{\pgfqpoint{9.687500in}{4.235000in}}%
\pgfusepath{clip}%
\pgfsetrectcap%
\pgfsetroundjoin%
\pgfsetlinewidth{0.803000pt}%
\definecolor{currentstroke}{rgb}{0.600000,0.600000,0.600000}%
\pgfsetstrokecolor{currentstroke}%
\pgfsetstrokeopacity{0.200000}%
\pgfsetdash{}{0pt}%
\pgfpathmoveto{\pgfqpoint{0.702268in}{0.718770in}}%
\pgfpathlineto{\pgfqpoint{10.389768in}{0.718770in}}%
\pgfusepath{stroke}%
\end{pgfscope}%
\begin{pgfscope}%
\pgfsetbuttcap%
\pgfsetroundjoin%
\definecolor{currentfill}{rgb}{0.000000,0.000000,0.000000}%
\pgfsetfillcolor{currentfill}%
\pgfsetlinewidth{0.602250pt}%
\definecolor{currentstroke}{rgb}{0.000000,0.000000,0.000000}%
\pgfsetstrokecolor{currentstroke}%
\pgfsetdash{}{0pt}%
\pgfsys@defobject{currentmarker}{\pgfqpoint{-0.027778in}{0.000000in}}{\pgfqpoint{-0.000000in}{0.000000in}}{%
\pgfpathmoveto{\pgfqpoint{-0.000000in}{0.000000in}}%
\pgfpathlineto{\pgfqpoint{-0.027778in}{0.000000in}}%
\pgfusepath{stroke,fill}%
}%
\begin{pgfscope}%
\pgfsys@transformshift{0.702268in}{0.718770in}%
\pgfsys@useobject{currentmarker}{}%
\end{pgfscope}%
\end{pgfscope}%
\begin{pgfscope}%
\pgfpathrectangle{\pgfqpoint{0.702268in}{0.521603in}}{\pgfqpoint{9.687500in}{4.235000in}}%
\pgfusepath{clip}%
\pgfsetrectcap%
\pgfsetroundjoin%
\pgfsetlinewidth{0.803000pt}%
\definecolor{currentstroke}{rgb}{0.600000,0.600000,0.600000}%
\pgfsetstrokecolor{currentstroke}%
\pgfsetstrokeopacity{0.200000}%
\pgfsetdash{}{0pt}%
\pgfpathmoveto{\pgfqpoint{0.702268in}{0.859318in}}%
\pgfpathlineto{\pgfqpoint{10.389768in}{0.859318in}}%
\pgfusepath{stroke}%
\end{pgfscope}%
\begin{pgfscope}%
\pgfsetbuttcap%
\pgfsetroundjoin%
\definecolor{currentfill}{rgb}{0.000000,0.000000,0.000000}%
\pgfsetfillcolor{currentfill}%
\pgfsetlinewidth{0.602250pt}%
\definecolor{currentstroke}{rgb}{0.000000,0.000000,0.000000}%
\pgfsetstrokecolor{currentstroke}%
\pgfsetdash{}{0pt}%
\pgfsys@defobject{currentmarker}{\pgfqpoint{-0.027778in}{0.000000in}}{\pgfqpoint{-0.000000in}{0.000000in}}{%
\pgfpathmoveto{\pgfqpoint{-0.000000in}{0.000000in}}%
\pgfpathlineto{\pgfqpoint{-0.027778in}{0.000000in}}%
\pgfusepath{stroke,fill}%
}%
\begin{pgfscope}%
\pgfsys@transformshift{0.702268in}{0.859318in}%
\pgfsys@useobject{currentmarker}{}%
\end{pgfscope}%
\end{pgfscope}%
\begin{pgfscope}%
\pgfpathrectangle{\pgfqpoint{0.702268in}{0.521603in}}{\pgfqpoint{9.687500in}{4.235000in}}%
\pgfusepath{clip}%
\pgfsetrectcap%
\pgfsetroundjoin%
\pgfsetlinewidth{0.803000pt}%
\definecolor{currentstroke}{rgb}{0.600000,0.600000,0.600000}%
\pgfsetstrokecolor{currentstroke}%
\pgfsetstrokeopacity{0.200000}%
\pgfsetdash{}{0pt}%
\pgfpathmoveto{\pgfqpoint{0.702268in}{0.999867in}}%
\pgfpathlineto{\pgfqpoint{10.389768in}{0.999867in}}%
\pgfusepath{stroke}%
\end{pgfscope}%
\begin{pgfscope}%
\pgfsetbuttcap%
\pgfsetroundjoin%
\definecolor{currentfill}{rgb}{0.000000,0.000000,0.000000}%
\pgfsetfillcolor{currentfill}%
\pgfsetlinewidth{0.602250pt}%
\definecolor{currentstroke}{rgb}{0.000000,0.000000,0.000000}%
\pgfsetstrokecolor{currentstroke}%
\pgfsetdash{}{0pt}%
\pgfsys@defobject{currentmarker}{\pgfqpoint{-0.027778in}{0.000000in}}{\pgfqpoint{-0.000000in}{0.000000in}}{%
\pgfpathmoveto{\pgfqpoint{-0.000000in}{0.000000in}}%
\pgfpathlineto{\pgfqpoint{-0.027778in}{0.000000in}}%
\pgfusepath{stroke,fill}%
}%
\begin{pgfscope}%
\pgfsys@transformshift{0.702268in}{0.999867in}%
\pgfsys@useobject{currentmarker}{}%
\end{pgfscope}%
\end{pgfscope}%
\begin{pgfscope}%
\pgfpathrectangle{\pgfqpoint{0.702268in}{0.521603in}}{\pgfqpoint{9.687500in}{4.235000in}}%
\pgfusepath{clip}%
\pgfsetrectcap%
\pgfsetroundjoin%
\pgfsetlinewidth{0.803000pt}%
\definecolor{currentstroke}{rgb}{0.600000,0.600000,0.600000}%
\pgfsetstrokecolor{currentstroke}%
\pgfsetstrokeopacity{0.200000}%
\pgfsetdash{}{0pt}%
\pgfpathmoveto{\pgfqpoint{0.702268in}{1.280965in}}%
\pgfpathlineto{\pgfqpoint{10.389768in}{1.280965in}}%
\pgfusepath{stroke}%
\end{pgfscope}%
\begin{pgfscope}%
\pgfsetbuttcap%
\pgfsetroundjoin%
\definecolor{currentfill}{rgb}{0.000000,0.000000,0.000000}%
\pgfsetfillcolor{currentfill}%
\pgfsetlinewidth{0.602250pt}%
\definecolor{currentstroke}{rgb}{0.000000,0.000000,0.000000}%
\pgfsetstrokecolor{currentstroke}%
\pgfsetdash{}{0pt}%
\pgfsys@defobject{currentmarker}{\pgfqpoint{-0.027778in}{0.000000in}}{\pgfqpoint{-0.000000in}{0.000000in}}{%
\pgfpathmoveto{\pgfqpoint{-0.000000in}{0.000000in}}%
\pgfpathlineto{\pgfqpoint{-0.027778in}{0.000000in}}%
\pgfusepath{stroke,fill}%
}%
\begin{pgfscope}%
\pgfsys@transformshift{0.702268in}{1.280965in}%
\pgfsys@useobject{currentmarker}{}%
\end{pgfscope}%
\end{pgfscope}%
\begin{pgfscope}%
\pgfpathrectangle{\pgfqpoint{0.702268in}{0.521603in}}{\pgfqpoint{9.687500in}{4.235000in}}%
\pgfusepath{clip}%
\pgfsetrectcap%
\pgfsetroundjoin%
\pgfsetlinewidth{0.803000pt}%
\definecolor{currentstroke}{rgb}{0.600000,0.600000,0.600000}%
\pgfsetstrokecolor{currentstroke}%
\pgfsetstrokeopacity{0.200000}%
\pgfsetdash{}{0pt}%
\pgfpathmoveto{\pgfqpoint{0.702268in}{1.421514in}}%
\pgfpathlineto{\pgfqpoint{10.389768in}{1.421514in}}%
\pgfusepath{stroke}%
\end{pgfscope}%
\begin{pgfscope}%
\pgfsetbuttcap%
\pgfsetroundjoin%
\definecolor{currentfill}{rgb}{0.000000,0.000000,0.000000}%
\pgfsetfillcolor{currentfill}%
\pgfsetlinewidth{0.602250pt}%
\definecolor{currentstroke}{rgb}{0.000000,0.000000,0.000000}%
\pgfsetstrokecolor{currentstroke}%
\pgfsetdash{}{0pt}%
\pgfsys@defobject{currentmarker}{\pgfqpoint{-0.027778in}{0.000000in}}{\pgfqpoint{-0.000000in}{0.000000in}}{%
\pgfpathmoveto{\pgfqpoint{-0.000000in}{0.000000in}}%
\pgfpathlineto{\pgfqpoint{-0.027778in}{0.000000in}}%
\pgfusepath{stroke,fill}%
}%
\begin{pgfscope}%
\pgfsys@transformshift{0.702268in}{1.421514in}%
\pgfsys@useobject{currentmarker}{}%
\end{pgfscope}%
\end{pgfscope}%
\begin{pgfscope}%
\pgfpathrectangle{\pgfqpoint{0.702268in}{0.521603in}}{\pgfqpoint{9.687500in}{4.235000in}}%
\pgfusepath{clip}%
\pgfsetrectcap%
\pgfsetroundjoin%
\pgfsetlinewidth{0.803000pt}%
\definecolor{currentstroke}{rgb}{0.600000,0.600000,0.600000}%
\pgfsetstrokecolor{currentstroke}%
\pgfsetstrokeopacity{0.200000}%
\pgfsetdash{}{0pt}%
\pgfpathmoveto{\pgfqpoint{0.702268in}{1.562063in}}%
\pgfpathlineto{\pgfqpoint{10.389768in}{1.562063in}}%
\pgfusepath{stroke}%
\end{pgfscope}%
\begin{pgfscope}%
\pgfsetbuttcap%
\pgfsetroundjoin%
\definecolor{currentfill}{rgb}{0.000000,0.000000,0.000000}%
\pgfsetfillcolor{currentfill}%
\pgfsetlinewidth{0.602250pt}%
\definecolor{currentstroke}{rgb}{0.000000,0.000000,0.000000}%
\pgfsetstrokecolor{currentstroke}%
\pgfsetdash{}{0pt}%
\pgfsys@defobject{currentmarker}{\pgfqpoint{-0.027778in}{0.000000in}}{\pgfqpoint{-0.000000in}{0.000000in}}{%
\pgfpathmoveto{\pgfqpoint{-0.000000in}{0.000000in}}%
\pgfpathlineto{\pgfqpoint{-0.027778in}{0.000000in}}%
\pgfusepath{stroke,fill}%
}%
\begin{pgfscope}%
\pgfsys@transformshift{0.702268in}{1.562063in}%
\pgfsys@useobject{currentmarker}{}%
\end{pgfscope}%
\end{pgfscope}%
\begin{pgfscope}%
\pgfpathrectangle{\pgfqpoint{0.702268in}{0.521603in}}{\pgfqpoint{9.687500in}{4.235000in}}%
\pgfusepath{clip}%
\pgfsetrectcap%
\pgfsetroundjoin%
\pgfsetlinewidth{0.803000pt}%
\definecolor{currentstroke}{rgb}{0.600000,0.600000,0.600000}%
\pgfsetstrokecolor{currentstroke}%
\pgfsetstrokeopacity{0.200000}%
\pgfsetdash{}{0pt}%
\pgfpathmoveto{\pgfqpoint{0.702268in}{1.843161in}}%
\pgfpathlineto{\pgfqpoint{10.389768in}{1.843161in}}%
\pgfusepath{stroke}%
\end{pgfscope}%
\begin{pgfscope}%
\pgfsetbuttcap%
\pgfsetroundjoin%
\definecolor{currentfill}{rgb}{0.000000,0.000000,0.000000}%
\pgfsetfillcolor{currentfill}%
\pgfsetlinewidth{0.602250pt}%
\definecolor{currentstroke}{rgb}{0.000000,0.000000,0.000000}%
\pgfsetstrokecolor{currentstroke}%
\pgfsetdash{}{0pt}%
\pgfsys@defobject{currentmarker}{\pgfqpoint{-0.027778in}{0.000000in}}{\pgfqpoint{-0.000000in}{0.000000in}}{%
\pgfpathmoveto{\pgfqpoint{-0.000000in}{0.000000in}}%
\pgfpathlineto{\pgfqpoint{-0.027778in}{0.000000in}}%
\pgfusepath{stroke,fill}%
}%
\begin{pgfscope}%
\pgfsys@transformshift{0.702268in}{1.843161in}%
\pgfsys@useobject{currentmarker}{}%
\end{pgfscope}%
\end{pgfscope}%
\begin{pgfscope}%
\pgfpathrectangle{\pgfqpoint{0.702268in}{0.521603in}}{\pgfqpoint{9.687500in}{4.235000in}}%
\pgfusepath{clip}%
\pgfsetrectcap%
\pgfsetroundjoin%
\pgfsetlinewidth{0.803000pt}%
\definecolor{currentstroke}{rgb}{0.600000,0.600000,0.600000}%
\pgfsetstrokecolor{currentstroke}%
\pgfsetstrokeopacity{0.200000}%
\pgfsetdash{}{0pt}%
\pgfpathmoveto{\pgfqpoint{0.702268in}{1.983710in}}%
\pgfpathlineto{\pgfqpoint{10.389768in}{1.983710in}}%
\pgfusepath{stroke}%
\end{pgfscope}%
\begin{pgfscope}%
\pgfsetbuttcap%
\pgfsetroundjoin%
\definecolor{currentfill}{rgb}{0.000000,0.000000,0.000000}%
\pgfsetfillcolor{currentfill}%
\pgfsetlinewidth{0.602250pt}%
\definecolor{currentstroke}{rgb}{0.000000,0.000000,0.000000}%
\pgfsetstrokecolor{currentstroke}%
\pgfsetdash{}{0pt}%
\pgfsys@defobject{currentmarker}{\pgfqpoint{-0.027778in}{0.000000in}}{\pgfqpoint{-0.000000in}{0.000000in}}{%
\pgfpathmoveto{\pgfqpoint{-0.000000in}{0.000000in}}%
\pgfpathlineto{\pgfqpoint{-0.027778in}{0.000000in}}%
\pgfusepath{stroke,fill}%
}%
\begin{pgfscope}%
\pgfsys@transformshift{0.702268in}{1.983710in}%
\pgfsys@useobject{currentmarker}{}%
\end{pgfscope}%
\end{pgfscope}%
\begin{pgfscope}%
\pgfpathrectangle{\pgfqpoint{0.702268in}{0.521603in}}{\pgfqpoint{9.687500in}{4.235000in}}%
\pgfusepath{clip}%
\pgfsetrectcap%
\pgfsetroundjoin%
\pgfsetlinewidth{0.803000pt}%
\definecolor{currentstroke}{rgb}{0.600000,0.600000,0.600000}%
\pgfsetstrokecolor{currentstroke}%
\pgfsetstrokeopacity{0.200000}%
\pgfsetdash{}{0pt}%
\pgfpathmoveto{\pgfqpoint{0.702268in}{2.124259in}}%
\pgfpathlineto{\pgfqpoint{10.389768in}{2.124259in}}%
\pgfusepath{stroke}%
\end{pgfscope}%
\begin{pgfscope}%
\pgfsetbuttcap%
\pgfsetroundjoin%
\definecolor{currentfill}{rgb}{0.000000,0.000000,0.000000}%
\pgfsetfillcolor{currentfill}%
\pgfsetlinewidth{0.602250pt}%
\definecolor{currentstroke}{rgb}{0.000000,0.000000,0.000000}%
\pgfsetstrokecolor{currentstroke}%
\pgfsetdash{}{0pt}%
\pgfsys@defobject{currentmarker}{\pgfqpoint{-0.027778in}{0.000000in}}{\pgfqpoint{-0.000000in}{0.000000in}}{%
\pgfpathmoveto{\pgfqpoint{-0.000000in}{0.000000in}}%
\pgfpathlineto{\pgfqpoint{-0.027778in}{0.000000in}}%
\pgfusepath{stroke,fill}%
}%
\begin{pgfscope}%
\pgfsys@transformshift{0.702268in}{2.124259in}%
\pgfsys@useobject{currentmarker}{}%
\end{pgfscope}%
\end{pgfscope}%
\begin{pgfscope}%
\pgfpathrectangle{\pgfqpoint{0.702268in}{0.521603in}}{\pgfqpoint{9.687500in}{4.235000in}}%
\pgfusepath{clip}%
\pgfsetrectcap%
\pgfsetroundjoin%
\pgfsetlinewidth{0.803000pt}%
\definecolor{currentstroke}{rgb}{0.600000,0.600000,0.600000}%
\pgfsetstrokecolor{currentstroke}%
\pgfsetstrokeopacity{0.200000}%
\pgfsetdash{}{0pt}%
\pgfpathmoveto{\pgfqpoint{0.702268in}{2.405356in}}%
\pgfpathlineto{\pgfqpoint{10.389768in}{2.405356in}}%
\pgfusepath{stroke}%
\end{pgfscope}%
\begin{pgfscope}%
\pgfsetbuttcap%
\pgfsetroundjoin%
\definecolor{currentfill}{rgb}{0.000000,0.000000,0.000000}%
\pgfsetfillcolor{currentfill}%
\pgfsetlinewidth{0.602250pt}%
\definecolor{currentstroke}{rgb}{0.000000,0.000000,0.000000}%
\pgfsetstrokecolor{currentstroke}%
\pgfsetdash{}{0pt}%
\pgfsys@defobject{currentmarker}{\pgfqpoint{-0.027778in}{0.000000in}}{\pgfqpoint{-0.000000in}{0.000000in}}{%
\pgfpathmoveto{\pgfqpoint{-0.000000in}{0.000000in}}%
\pgfpathlineto{\pgfqpoint{-0.027778in}{0.000000in}}%
\pgfusepath{stroke,fill}%
}%
\begin{pgfscope}%
\pgfsys@transformshift{0.702268in}{2.405356in}%
\pgfsys@useobject{currentmarker}{}%
\end{pgfscope}%
\end{pgfscope}%
\begin{pgfscope}%
\pgfpathrectangle{\pgfqpoint{0.702268in}{0.521603in}}{\pgfqpoint{9.687500in}{4.235000in}}%
\pgfusepath{clip}%
\pgfsetrectcap%
\pgfsetroundjoin%
\pgfsetlinewidth{0.803000pt}%
\definecolor{currentstroke}{rgb}{0.600000,0.600000,0.600000}%
\pgfsetstrokecolor{currentstroke}%
\pgfsetstrokeopacity{0.200000}%
\pgfsetdash{}{0pt}%
\pgfpathmoveto{\pgfqpoint{0.702268in}{2.545905in}}%
\pgfpathlineto{\pgfqpoint{10.389768in}{2.545905in}}%
\pgfusepath{stroke}%
\end{pgfscope}%
\begin{pgfscope}%
\pgfsetbuttcap%
\pgfsetroundjoin%
\definecolor{currentfill}{rgb}{0.000000,0.000000,0.000000}%
\pgfsetfillcolor{currentfill}%
\pgfsetlinewidth{0.602250pt}%
\definecolor{currentstroke}{rgb}{0.000000,0.000000,0.000000}%
\pgfsetstrokecolor{currentstroke}%
\pgfsetdash{}{0pt}%
\pgfsys@defobject{currentmarker}{\pgfqpoint{-0.027778in}{0.000000in}}{\pgfqpoint{-0.000000in}{0.000000in}}{%
\pgfpathmoveto{\pgfqpoint{-0.000000in}{0.000000in}}%
\pgfpathlineto{\pgfqpoint{-0.027778in}{0.000000in}}%
\pgfusepath{stroke,fill}%
}%
\begin{pgfscope}%
\pgfsys@transformshift{0.702268in}{2.545905in}%
\pgfsys@useobject{currentmarker}{}%
\end{pgfscope}%
\end{pgfscope}%
\begin{pgfscope}%
\pgfpathrectangle{\pgfqpoint{0.702268in}{0.521603in}}{\pgfqpoint{9.687500in}{4.235000in}}%
\pgfusepath{clip}%
\pgfsetrectcap%
\pgfsetroundjoin%
\pgfsetlinewidth{0.803000pt}%
\definecolor{currentstroke}{rgb}{0.600000,0.600000,0.600000}%
\pgfsetstrokecolor{currentstroke}%
\pgfsetstrokeopacity{0.200000}%
\pgfsetdash{}{0pt}%
\pgfpathmoveto{\pgfqpoint{0.702268in}{2.686454in}}%
\pgfpathlineto{\pgfqpoint{10.389768in}{2.686454in}}%
\pgfusepath{stroke}%
\end{pgfscope}%
\begin{pgfscope}%
\pgfsetbuttcap%
\pgfsetroundjoin%
\definecolor{currentfill}{rgb}{0.000000,0.000000,0.000000}%
\pgfsetfillcolor{currentfill}%
\pgfsetlinewidth{0.602250pt}%
\definecolor{currentstroke}{rgb}{0.000000,0.000000,0.000000}%
\pgfsetstrokecolor{currentstroke}%
\pgfsetdash{}{0pt}%
\pgfsys@defobject{currentmarker}{\pgfqpoint{-0.027778in}{0.000000in}}{\pgfqpoint{-0.000000in}{0.000000in}}{%
\pgfpathmoveto{\pgfqpoint{-0.000000in}{0.000000in}}%
\pgfpathlineto{\pgfqpoint{-0.027778in}{0.000000in}}%
\pgfusepath{stroke,fill}%
}%
\begin{pgfscope}%
\pgfsys@transformshift{0.702268in}{2.686454in}%
\pgfsys@useobject{currentmarker}{}%
\end{pgfscope}%
\end{pgfscope}%
\begin{pgfscope}%
\pgfpathrectangle{\pgfqpoint{0.702268in}{0.521603in}}{\pgfqpoint{9.687500in}{4.235000in}}%
\pgfusepath{clip}%
\pgfsetrectcap%
\pgfsetroundjoin%
\pgfsetlinewidth{0.803000pt}%
\definecolor{currentstroke}{rgb}{0.600000,0.600000,0.600000}%
\pgfsetstrokecolor{currentstroke}%
\pgfsetstrokeopacity{0.200000}%
\pgfsetdash{}{0pt}%
\pgfpathmoveto{\pgfqpoint{0.702268in}{2.967552in}}%
\pgfpathlineto{\pgfqpoint{10.389768in}{2.967552in}}%
\pgfusepath{stroke}%
\end{pgfscope}%
\begin{pgfscope}%
\pgfsetbuttcap%
\pgfsetroundjoin%
\definecolor{currentfill}{rgb}{0.000000,0.000000,0.000000}%
\pgfsetfillcolor{currentfill}%
\pgfsetlinewidth{0.602250pt}%
\definecolor{currentstroke}{rgb}{0.000000,0.000000,0.000000}%
\pgfsetstrokecolor{currentstroke}%
\pgfsetdash{}{0pt}%
\pgfsys@defobject{currentmarker}{\pgfqpoint{-0.027778in}{0.000000in}}{\pgfqpoint{-0.000000in}{0.000000in}}{%
\pgfpathmoveto{\pgfqpoint{-0.000000in}{0.000000in}}%
\pgfpathlineto{\pgfqpoint{-0.027778in}{0.000000in}}%
\pgfusepath{stroke,fill}%
}%
\begin{pgfscope}%
\pgfsys@transformshift{0.702268in}{2.967552in}%
\pgfsys@useobject{currentmarker}{}%
\end{pgfscope}%
\end{pgfscope}%
\begin{pgfscope}%
\pgfpathrectangle{\pgfqpoint{0.702268in}{0.521603in}}{\pgfqpoint{9.687500in}{4.235000in}}%
\pgfusepath{clip}%
\pgfsetrectcap%
\pgfsetroundjoin%
\pgfsetlinewidth{0.803000pt}%
\definecolor{currentstroke}{rgb}{0.600000,0.600000,0.600000}%
\pgfsetstrokecolor{currentstroke}%
\pgfsetstrokeopacity{0.200000}%
\pgfsetdash{}{0pt}%
\pgfpathmoveto{\pgfqpoint{0.702268in}{3.108101in}}%
\pgfpathlineto{\pgfqpoint{10.389768in}{3.108101in}}%
\pgfusepath{stroke}%
\end{pgfscope}%
\begin{pgfscope}%
\pgfsetbuttcap%
\pgfsetroundjoin%
\definecolor{currentfill}{rgb}{0.000000,0.000000,0.000000}%
\pgfsetfillcolor{currentfill}%
\pgfsetlinewidth{0.602250pt}%
\definecolor{currentstroke}{rgb}{0.000000,0.000000,0.000000}%
\pgfsetstrokecolor{currentstroke}%
\pgfsetdash{}{0pt}%
\pgfsys@defobject{currentmarker}{\pgfqpoint{-0.027778in}{0.000000in}}{\pgfqpoint{-0.000000in}{0.000000in}}{%
\pgfpathmoveto{\pgfqpoint{-0.000000in}{0.000000in}}%
\pgfpathlineto{\pgfqpoint{-0.027778in}{0.000000in}}%
\pgfusepath{stroke,fill}%
}%
\begin{pgfscope}%
\pgfsys@transformshift{0.702268in}{3.108101in}%
\pgfsys@useobject{currentmarker}{}%
\end{pgfscope}%
\end{pgfscope}%
\begin{pgfscope}%
\pgfpathrectangle{\pgfqpoint{0.702268in}{0.521603in}}{\pgfqpoint{9.687500in}{4.235000in}}%
\pgfusepath{clip}%
\pgfsetrectcap%
\pgfsetroundjoin%
\pgfsetlinewidth{0.803000pt}%
\definecolor{currentstroke}{rgb}{0.600000,0.600000,0.600000}%
\pgfsetstrokecolor{currentstroke}%
\pgfsetstrokeopacity{0.200000}%
\pgfsetdash{}{0pt}%
\pgfpathmoveto{\pgfqpoint{0.702268in}{3.248650in}}%
\pgfpathlineto{\pgfqpoint{10.389768in}{3.248650in}}%
\pgfusepath{stroke}%
\end{pgfscope}%
\begin{pgfscope}%
\pgfsetbuttcap%
\pgfsetroundjoin%
\definecolor{currentfill}{rgb}{0.000000,0.000000,0.000000}%
\pgfsetfillcolor{currentfill}%
\pgfsetlinewidth{0.602250pt}%
\definecolor{currentstroke}{rgb}{0.000000,0.000000,0.000000}%
\pgfsetstrokecolor{currentstroke}%
\pgfsetdash{}{0pt}%
\pgfsys@defobject{currentmarker}{\pgfqpoint{-0.027778in}{0.000000in}}{\pgfqpoint{-0.000000in}{0.000000in}}{%
\pgfpathmoveto{\pgfqpoint{-0.000000in}{0.000000in}}%
\pgfpathlineto{\pgfqpoint{-0.027778in}{0.000000in}}%
\pgfusepath{stroke,fill}%
}%
\begin{pgfscope}%
\pgfsys@transformshift{0.702268in}{3.248650in}%
\pgfsys@useobject{currentmarker}{}%
\end{pgfscope}%
\end{pgfscope}%
\begin{pgfscope}%
\pgfpathrectangle{\pgfqpoint{0.702268in}{0.521603in}}{\pgfqpoint{9.687500in}{4.235000in}}%
\pgfusepath{clip}%
\pgfsetrectcap%
\pgfsetroundjoin%
\pgfsetlinewidth{0.803000pt}%
\definecolor{currentstroke}{rgb}{0.600000,0.600000,0.600000}%
\pgfsetstrokecolor{currentstroke}%
\pgfsetstrokeopacity{0.200000}%
\pgfsetdash{}{0pt}%
\pgfpathmoveto{\pgfqpoint{0.702268in}{3.529748in}}%
\pgfpathlineto{\pgfqpoint{10.389768in}{3.529748in}}%
\pgfusepath{stroke}%
\end{pgfscope}%
\begin{pgfscope}%
\pgfsetbuttcap%
\pgfsetroundjoin%
\definecolor{currentfill}{rgb}{0.000000,0.000000,0.000000}%
\pgfsetfillcolor{currentfill}%
\pgfsetlinewidth{0.602250pt}%
\definecolor{currentstroke}{rgb}{0.000000,0.000000,0.000000}%
\pgfsetstrokecolor{currentstroke}%
\pgfsetdash{}{0pt}%
\pgfsys@defobject{currentmarker}{\pgfqpoint{-0.027778in}{0.000000in}}{\pgfqpoint{-0.000000in}{0.000000in}}{%
\pgfpathmoveto{\pgfqpoint{-0.000000in}{0.000000in}}%
\pgfpathlineto{\pgfqpoint{-0.027778in}{0.000000in}}%
\pgfusepath{stroke,fill}%
}%
\begin{pgfscope}%
\pgfsys@transformshift{0.702268in}{3.529748in}%
\pgfsys@useobject{currentmarker}{}%
\end{pgfscope}%
\end{pgfscope}%
\begin{pgfscope}%
\pgfpathrectangle{\pgfqpoint{0.702268in}{0.521603in}}{\pgfqpoint{9.687500in}{4.235000in}}%
\pgfusepath{clip}%
\pgfsetrectcap%
\pgfsetroundjoin%
\pgfsetlinewidth{0.803000pt}%
\definecolor{currentstroke}{rgb}{0.600000,0.600000,0.600000}%
\pgfsetstrokecolor{currentstroke}%
\pgfsetstrokeopacity{0.200000}%
\pgfsetdash{}{0pt}%
\pgfpathmoveto{\pgfqpoint{0.702268in}{3.670297in}}%
\pgfpathlineto{\pgfqpoint{10.389768in}{3.670297in}}%
\pgfusepath{stroke}%
\end{pgfscope}%
\begin{pgfscope}%
\pgfsetbuttcap%
\pgfsetroundjoin%
\definecolor{currentfill}{rgb}{0.000000,0.000000,0.000000}%
\pgfsetfillcolor{currentfill}%
\pgfsetlinewidth{0.602250pt}%
\definecolor{currentstroke}{rgb}{0.000000,0.000000,0.000000}%
\pgfsetstrokecolor{currentstroke}%
\pgfsetdash{}{0pt}%
\pgfsys@defobject{currentmarker}{\pgfqpoint{-0.027778in}{0.000000in}}{\pgfqpoint{-0.000000in}{0.000000in}}{%
\pgfpathmoveto{\pgfqpoint{-0.000000in}{0.000000in}}%
\pgfpathlineto{\pgfqpoint{-0.027778in}{0.000000in}}%
\pgfusepath{stroke,fill}%
}%
\begin{pgfscope}%
\pgfsys@transformshift{0.702268in}{3.670297in}%
\pgfsys@useobject{currentmarker}{}%
\end{pgfscope}%
\end{pgfscope}%
\begin{pgfscope}%
\pgfpathrectangle{\pgfqpoint{0.702268in}{0.521603in}}{\pgfqpoint{9.687500in}{4.235000in}}%
\pgfusepath{clip}%
\pgfsetrectcap%
\pgfsetroundjoin%
\pgfsetlinewidth{0.803000pt}%
\definecolor{currentstroke}{rgb}{0.600000,0.600000,0.600000}%
\pgfsetstrokecolor{currentstroke}%
\pgfsetstrokeopacity{0.200000}%
\pgfsetdash{}{0pt}%
\pgfpathmoveto{\pgfqpoint{0.702268in}{3.810846in}}%
\pgfpathlineto{\pgfqpoint{10.389768in}{3.810846in}}%
\pgfusepath{stroke}%
\end{pgfscope}%
\begin{pgfscope}%
\pgfsetbuttcap%
\pgfsetroundjoin%
\definecolor{currentfill}{rgb}{0.000000,0.000000,0.000000}%
\pgfsetfillcolor{currentfill}%
\pgfsetlinewidth{0.602250pt}%
\definecolor{currentstroke}{rgb}{0.000000,0.000000,0.000000}%
\pgfsetstrokecolor{currentstroke}%
\pgfsetdash{}{0pt}%
\pgfsys@defobject{currentmarker}{\pgfqpoint{-0.027778in}{0.000000in}}{\pgfqpoint{-0.000000in}{0.000000in}}{%
\pgfpathmoveto{\pgfqpoint{-0.000000in}{0.000000in}}%
\pgfpathlineto{\pgfqpoint{-0.027778in}{0.000000in}}%
\pgfusepath{stroke,fill}%
}%
\begin{pgfscope}%
\pgfsys@transformshift{0.702268in}{3.810846in}%
\pgfsys@useobject{currentmarker}{}%
\end{pgfscope}%
\end{pgfscope}%
\begin{pgfscope}%
\pgfpathrectangle{\pgfqpoint{0.702268in}{0.521603in}}{\pgfqpoint{9.687500in}{4.235000in}}%
\pgfusepath{clip}%
\pgfsetrectcap%
\pgfsetroundjoin%
\pgfsetlinewidth{0.803000pt}%
\definecolor{currentstroke}{rgb}{0.600000,0.600000,0.600000}%
\pgfsetstrokecolor{currentstroke}%
\pgfsetstrokeopacity{0.200000}%
\pgfsetdash{}{0pt}%
\pgfpathmoveto{\pgfqpoint{0.702268in}{4.091943in}}%
\pgfpathlineto{\pgfqpoint{10.389768in}{4.091943in}}%
\pgfusepath{stroke}%
\end{pgfscope}%
\begin{pgfscope}%
\pgfsetbuttcap%
\pgfsetroundjoin%
\definecolor{currentfill}{rgb}{0.000000,0.000000,0.000000}%
\pgfsetfillcolor{currentfill}%
\pgfsetlinewidth{0.602250pt}%
\definecolor{currentstroke}{rgb}{0.000000,0.000000,0.000000}%
\pgfsetstrokecolor{currentstroke}%
\pgfsetdash{}{0pt}%
\pgfsys@defobject{currentmarker}{\pgfqpoint{-0.027778in}{0.000000in}}{\pgfqpoint{-0.000000in}{0.000000in}}{%
\pgfpathmoveto{\pgfqpoint{-0.000000in}{0.000000in}}%
\pgfpathlineto{\pgfqpoint{-0.027778in}{0.000000in}}%
\pgfusepath{stroke,fill}%
}%
\begin{pgfscope}%
\pgfsys@transformshift{0.702268in}{4.091943in}%
\pgfsys@useobject{currentmarker}{}%
\end{pgfscope}%
\end{pgfscope}%
\begin{pgfscope}%
\pgfpathrectangle{\pgfqpoint{0.702268in}{0.521603in}}{\pgfqpoint{9.687500in}{4.235000in}}%
\pgfusepath{clip}%
\pgfsetrectcap%
\pgfsetroundjoin%
\pgfsetlinewidth{0.803000pt}%
\definecolor{currentstroke}{rgb}{0.600000,0.600000,0.600000}%
\pgfsetstrokecolor{currentstroke}%
\pgfsetstrokeopacity{0.200000}%
\pgfsetdash{}{0pt}%
\pgfpathmoveto{\pgfqpoint{0.702268in}{4.232492in}}%
\pgfpathlineto{\pgfqpoint{10.389768in}{4.232492in}}%
\pgfusepath{stroke}%
\end{pgfscope}%
\begin{pgfscope}%
\pgfsetbuttcap%
\pgfsetroundjoin%
\definecolor{currentfill}{rgb}{0.000000,0.000000,0.000000}%
\pgfsetfillcolor{currentfill}%
\pgfsetlinewidth{0.602250pt}%
\definecolor{currentstroke}{rgb}{0.000000,0.000000,0.000000}%
\pgfsetstrokecolor{currentstroke}%
\pgfsetdash{}{0pt}%
\pgfsys@defobject{currentmarker}{\pgfqpoint{-0.027778in}{0.000000in}}{\pgfqpoint{-0.000000in}{0.000000in}}{%
\pgfpathmoveto{\pgfqpoint{-0.000000in}{0.000000in}}%
\pgfpathlineto{\pgfqpoint{-0.027778in}{0.000000in}}%
\pgfusepath{stroke,fill}%
}%
\begin{pgfscope}%
\pgfsys@transformshift{0.702268in}{4.232492in}%
\pgfsys@useobject{currentmarker}{}%
\end{pgfscope}%
\end{pgfscope}%
\begin{pgfscope}%
\pgfpathrectangle{\pgfqpoint{0.702268in}{0.521603in}}{\pgfqpoint{9.687500in}{4.235000in}}%
\pgfusepath{clip}%
\pgfsetrectcap%
\pgfsetroundjoin%
\pgfsetlinewidth{0.803000pt}%
\definecolor{currentstroke}{rgb}{0.600000,0.600000,0.600000}%
\pgfsetstrokecolor{currentstroke}%
\pgfsetstrokeopacity{0.200000}%
\pgfsetdash{}{0pt}%
\pgfpathmoveto{\pgfqpoint{0.702268in}{4.373041in}}%
\pgfpathlineto{\pgfqpoint{10.389768in}{4.373041in}}%
\pgfusepath{stroke}%
\end{pgfscope}%
\begin{pgfscope}%
\pgfsetbuttcap%
\pgfsetroundjoin%
\definecolor{currentfill}{rgb}{0.000000,0.000000,0.000000}%
\pgfsetfillcolor{currentfill}%
\pgfsetlinewidth{0.602250pt}%
\definecolor{currentstroke}{rgb}{0.000000,0.000000,0.000000}%
\pgfsetstrokecolor{currentstroke}%
\pgfsetdash{}{0pt}%
\pgfsys@defobject{currentmarker}{\pgfqpoint{-0.027778in}{0.000000in}}{\pgfqpoint{-0.000000in}{0.000000in}}{%
\pgfpathmoveto{\pgfqpoint{-0.000000in}{0.000000in}}%
\pgfpathlineto{\pgfqpoint{-0.027778in}{0.000000in}}%
\pgfusepath{stroke,fill}%
}%
\begin{pgfscope}%
\pgfsys@transformshift{0.702268in}{4.373041in}%
\pgfsys@useobject{currentmarker}{}%
\end{pgfscope}%
\end{pgfscope}%
\begin{pgfscope}%
\pgfpathrectangle{\pgfqpoint{0.702268in}{0.521603in}}{\pgfqpoint{9.687500in}{4.235000in}}%
\pgfusepath{clip}%
\pgfsetrectcap%
\pgfsetroundjoin%
\pgfsetlinewidth{0.803000pt}%
\definecolor{currentstroke}{rgb}{0.600000,0.600000,0.600000}%
\pgfsetstrokecolor{currentstroke}%
\pgfsetstrokeopacity{0.200000}%
\pgfsetdash{}{0pt}%
\pgfpathmoveto{\pgfqpoint{0.702268in}{4.654139in}}%
\pgfpathlineto{\pgfqpoint{10.389768in}{4.654139in}}%
\pgfusepath{stroke}%
\end{pgfscope}%
\begin{pgfscope}%
\pgfsetbuttcap%
\pgfsetroundjoin%
\definecolor{currentfill}{rgb}{0.000000,0.000000,0.000000}%
\pgfsetfillcolor{currentfill}%
\pgfsetlinewidth{0.602250pt}%
\definecolor{currentstroke}{rgb}{0.000000,0.000000,0.000000}%
\pgfsetstrokecolor{currentstroke}%
\pgfsetdash{}{0pt}%
\pgfsys@defobject{currentmarker}{\pgfqpoint{-0.027778in}{0.000000in}}{\pgfqpoint{-0.000000in}{0.000000in}}{%
\pgfpathmoveto{\pgfqpoint{-0.000000in}{0.000000in}}%
\pgfpathlineto{\pgfqpoint{-0.027778in}{0.000000in}}%
\pgfusepath{stroke,fill}%
}%
\begin{pgfscope}%
\pgfsys@transformshift{0.702268in}{4.654139in}%
\pgfsys@useobject{currentmarker}{}%
\end{pgfscope}%
\end{pgfscope}%
\begin{pgfscope}%
\definecolor{textcolor}{rgb}{0.000000,0.000000,0.000000}%
\pgfsetstrokecolor{textcolor}%
\pgfsetfillcolor{textcolor}%
\pgftext[x=0.240245in,y=2.639103in,,bottom,rotate=90.000000]{\color{textcolor}\sffamily\fontsize{10.000000}{12.000000}\selectfont infection rate \(\displaystyle \langle I\rangle_t\)}%
\end{pgfscope}%
\begin{pgfscope}%
\pgfpathrectangle{\pgfqpoint{0.702268in}{0.521603in}}{\pgfqpoint{9.687500in}{4.235000in}}%
\pgfusepath{clip}%
\pgfsetrectcap%
\pgfsetroundjoin%
\pgfsetlinewidth{0.501875pt}%
\definecolor{currentstroke}{rgb}{0.501961,0.501961,0.501961}%
\pgfsetstrokecolor{currentstroke}%
\pgfsetstrokeopacity{0.250000}%
\pgfsetdash{}{0pt}%
\pgfpathmoveto{\pgfqpoint{1.142609in}{4.200363in}}%
\pgfpathlineto{\pgfqpoint{1.151416in}{3.390576in}}%
\pgfpathlineto{\pgfqpoint{1.177836in}{2.031749in}}%
\pgfpathlineto{\pgfqpoint{1.186643in}{1.956246in}}%
\pgfpathlineto{\pgfqpoint{1.195450in}{2.066071in}}%
\pgfpathlineto{\pgfqpoint{1.204257in}{2.120970in}}%
\pgfpathlineto{\pgfqpoint{1.213063in}{2.148405in}}%
\pgfpathlineto{\pgfqpoint{1.221870in}{2.615084in}}%
\pgfpathlineto{\pgfqpoint{1.230677in}{2.738598in}}%
\pgfpathlineto{\pgfqpoint{1.239484in}{2.429784in}}%
\pgfpathlineto{\pgfqpoint{1.248291in}{2.374885in}}%
\pgfpathlineto{\pgfqpoint{1.257097in}{2.230767in}}%
\pgfpathlineto{\pgfqpoint{1.265904in}{2.182727in}}%
\pgfpathlineto{\pgfqpoint{1.274711in}{1.544494in}}%
\pgfpathlineto{\pgfqpoint{1.283518in}{1.674896in}}%
\pgfpathlineto{\pgfqpoint{1.292325in}{1.853308in}}%
\pgfpathlineto{\pgfqpoint{1.301132in}{1.695472in}}%
\pgfpathlineto{\pgfqpoint{1.309938in}{1.695472in}}%
\pgfpathlineto{\pgfqpoint{1.318745in}{1.894489in}}%
\pgfpathlineto{\pgfqpoint{1.327552in}{2.038608in}}%
\pgfpathlineto{\pgfqpoint{1.336359in}{1.647432in}}%
\pgfpathlineto{\pgfqpoint{1.345166in}{1.132742in}}%
\pgfpathlineto{\pgfqpoint{1.353972in}{1.263115in}}%
\pgfpathlineto{\pgfqpoint{1.362779in}{1.242539in}}%
\pgfpathlineto{\pgfqpoint{1.371586in}{1.441556in}}%
\pgfpathlineto{\pgfqpoint{1.380393in}{1.777834in}}%
\pgfpathlineto{\pgfqpoint{1.389200in}{1.606252in}}%
\pgfpathlineto{\pgfqpoint{1.398007in}{2.024891in}}%
\pgfpathlineto{\pgfqpoint{1.406813in}{2.038608in}}%
\pgfpathlineto{\pgfqpoint{1.415620in}{1.716076in}}%
\pgfpathlineto{\pgfqpoint{1.424427in}{1.537636in}}%
\pgfpathlineto{\pgfqpoint{1.433234in}{1.798410in}}%
\pgfpathlineto{\pgfqpoint{1.442041in}{1.510172in}}%
\pgfpathlineto{\pgfqpoint{1.459654in}{2.011173in}}%
\pgfpathlineto{\pgfqpoint{1.468461in}{2.319987in}}%
\pgfpathlineto{\pgfqpoint{1.477268in}{2.313128in}}%
\pgfpathlineto{\pgfqpoint{1.486075in}{2.512146in}}%
\pgfpathlineto{\pgfqpoint{1.494882in}{1.949388in}}%
\pgfpathlineto{\pgfqpoint{1.503688in}{1.832732in}}%
\pgfpathlineto{\pgfqpoint{1.512495in}{1.517031in}}%
\pgfpathlineto{\pgfqpoint{1.521302in}{2.045467in}}%
\pgfpathlineto{\pgfqpoint{1.530109in}{2.024891in}}%
\pgfpathlineto{\pgfqpoint{1.538916in}{1.805269in}}%
\pgfpathlineto{\pgfqpoint{1.547722in}{1.688613in}}%
\pgfpathlineto{\pgfqpoint{1.556529in}{1.784692in}}%
\pgfpathlineto{\pgfqpoint{1.565336in}{2.031749in}}%
\pgfpathlineto{\pgfqpoint{1.574143in}{2.326846in}}%
\pgfpathlineto{\pgfqpoint{1.582950in}{2.457247in}}%
\pgfpathlineto{\pgfqpoint{1.591757in}{2.059213in}}%
\pgfpathlineto{\pgfqpoint{1.600563in}{1.956246in}}%
\pgfpathlineto{\pgfqpoint{1.609370in}{1.908207in}}%
\pgfpathlineto{\pgfqpoint{1.618177in}{1.812128in}}%
\pgfpathlineto{\pgfqpoint{1.626984in}{2.127829in}}%
\pgfpathlineto{\pgfqpoint{1.644597in}{1.825873in}}%
\pgfpathlineto{\pgfqpoint{1.653404in}{1.736653in}}%
\pgfpathlineto{\pgfqpoint{1.662211in}{2.175868in}}%
\pgfpathlineto{\pgfqpoint{1.671018in}{2.155292in}}%
\pgfpathlineto{\pgfqpoint{1.679825in}{2.217049in}}%
\pgfpathlineto{\pgfqpoint{1.688632in}{2.162151in}}%
\pgfpathlineto{\pgfqpoint{1.697438in}{1.901348in}}%
\pgfpathlineto{\pgfqpoint{1.706245in}{2.217049in}}%
\pgfpathlineto{\pgfqpoint{1.715052in}{2.004286in}}%
\pgfpathlineto{\pgfqpoint{1.723859in}{2.271947in}}%
\pgfpathlineto{\pgfqpoint{1.732666in}{2.464106in}}%
\pgfpathlineto{\pgfqpoint{1.741472in}{2.278806in}}%
\pgfpathlineto{\pgfqpoint{1.750279in}{2.573903in}}%
\pgfpathlineto{\pgfqpoint{1.759086in}{2.436643in}}%
\pgfpathlineto{\pgfqpoint{1.767893in}{2.388603in}}%
\pgfpathlineto{\pgfqpoint{1.776700in}{2.292524in}}%
\pgfpathlineto{\pgfqpoint{1.785507in}{1.887630in}}%
\pgfpathlineto{\pgfqpoint{1.794313in}{1.839591in}}%
\pgfpathlineto{\pgfqpoint{1.803120in}{1.963133in}}%
\pgfpathlineto{\pgfqpoint{1.811927in}{2.024891in}}%
\pgfpathlineto{\pgfqpoint{1.820734in}{2.059213in}}%
\pgfpathlineto{\pgfqpoint{1.829541in}{1.873913in}}%
\pgfpathlineto{\pgfqpoint{1.838347in}{1.825873in}}%
\pgfpathlineto{\pgfqpoint{1.847154in}{1.695472in}}%
\pgfpathlineto{\pgfqpoint{1.855961in}{1.702331in}}%
\pgfpathlineto{\pgfqpoint{1.864768in}{1.921953in}}%
\pgfpathlineto{\pgfqpoint{1.873575in}{2.072930in}}%
\pgfpathlineto{\pgfqpoint{1.882382in}{1.825873in}}%
\pgfpathlineto{\pgfqpoint{1.891188in}{1.681754in}}%
\pgfpathlineto{\pgfqpoint{1.899995in}{1.839591in}}%
\pgfpathlineto{\pgfqpoint{1.908802in}{1.825873in}}%
\pgfpathlineto{\pgfqpoint{1.917609in}{1.716076in}}%
\pgfpathlineto{\pgfqpoint{1.926416in}{1.674896in}}%
\pgfpathlineto{\pgfqpoint{1.935222in}{1.935670in}}%
\pgfpathlineto{\pgfqpoint{1.944029in}{2.114111in}}%
\pgfpathlineto{\pgfqpoint{1.952836in}{2.374885in}}%
\pgfpathlineto{\pgfqpoint{1.961643in}{2.169009in}}%
\pgfpathlineto{\pgfqpoint{1.970450in}{2.189586in}}%
\pgfpathlineto{\pgfqpoint{1.979257in}{2.388603in}}%
\pgfpathlineto{\pgfqpoint{1.988063in}{2.340563in}}%
\pgfpathlineto{\pgfqpoint{1.996870in}{2.395462in}}%
\pgfpathlineto{\pgfqpoint{2.014484in}{2.018032in}}%
\pgfpathlineto{\pgfqpoint{2.023291in}{2.107252in}}%
\pgfpathlineto{\pgfqpoint{2.032097in}{1.770975in}}%
\pgfpathlineto{\pgfqpoint{2.040904in}{1.722935in}}%
\pgfpathlineto{\pgfqpoint{2.049711in}{1.990569in}}%
\pgfpathlineto{\pgfqpoint{2.058518in}{1.750370in}}%
\pgfpathlineto{\pgfqpoint{2.067325in}{1.935670in}}%
\pgfpathlineto{\pgfqpoint{2.076132in}{1.819015in}}%
\pgfpathlineto{\pgfqpoint{2.084938in}{1.530777in}}%
\pgfpathlineto{\pgfqpoint{2.093745in}{1.709190in}}%
\pgfpathlineto{\pgfqpoint{2.102552in}{2.093507in}}%
\pgfpathlineto{\pgfqpoint{2.111359in}{2.134687in}}%
\pgfpathlineto{\pgfqpoint{2.120166in}{2.052326in}}%
\pgfpathlineto{\pgfqpoint{2.128972in}{1.942529in}}%
\pgfpathlineto{\pgfqpoint{2.137779in}{1.908207in}}%
\pgfpathlineto{\pgfqpoint{2.146586in}{1.983710in}}%
\pgfpathlineto{\pgfqpoint{2.155393in}{1.894489in}}%
\pgfpathlineto{\pgfqpoint{2.164200in}{2.024891in}}%
\pgfpathlineto{\pgfqpoint{2.173007in}{1.825873in}}%
\pgfpathlineto{\pgfqpoint{2.181813in}{1.757229in}}%
\pgfpathlineto{\pgfqpoint{2.190620in}{1.969992in}}%
\pgfpathlineto{\pgfqpoint{2.199427in}{1.963133in}}%
\pgfpathlineto{\pgfqpoint{2.208234in}{2.313128in}}%
\pgfpathlineto{\pgfqpoint{2.217041in}{2.786638in}}%
\pgfpathlineto{\pgfqpoint{2.225847in}{2.354309in}}%
\pgfpathlineto{\pgfqpoint{2.234654in}{2.374885in}}%
\pgfpathlineto{\pgfqpoint{2.243461in}{2.217049in}}%
\pgfpathlineto{\pgfqpoint{2.252268in}{1.921953in}}%
\pgfpathlineto{\pgfqpoint{2.261075in}{2.114111in}}%
\pgfpathlineto{\pgfqpoint{2.269882in}{1.997427in}}%
\pgfpathlineto{\pgfqpoint{2.278688in}{2.011173in}}%
\pgfpathlineto{\pgfqpoint{2.287495in}{2.004286in}}%
\pgfpathlineto{\pgfqpoint{2.296302in}{2.100365in}}%
\pgfpathlineto{\pgfqpoint{2.305109in}{1.949388in}}%
\pgfpathlineto{\pgfqpoint{2.313916in}{2.038608in}}%
\pgfpathlineto{\pgfqpoint{2.322722in}{2.100365in}}%
\pgfpathlineto{\pgfqpoint{2.331529in}{1.976851in}}%
\pgfpathlineto{\pgfqpoint{2.340336in}{2.011173in}}%
\pgfpathlineto{\pgfqpoint{2.349143in}{1.887630in}}%
\pgfpathlineto{\pgfqpoint{2.357950in}{1.901348in}}%
\pgfpathlineto{\pgfqpoint{2.366757in}{2.189586in}}%
\pgfpathlineto{\pgfqpoint{2.375563in}{2.024891in}}%
\pgfpathlineto{\pgfqpoint{2.384370in}{2.072930in}}%
\pgfpathlineto{\pgfqpoint{2.393177in}{1.757229in}}%
\pgfpathlineto{\pgfqpoint{2.401984in}{2.004286in}}%
\pgfpathlineto{\pgfqpoint{2.410791in}{1.901348in}}%
\pgfpathlineto{\pgfqpoint{2.419597in}{2.182727in}}%
\pgfpathlineto{\pgfqpoint{2.428404in}{2.024891in}}%
\pgfpathlineto{\pgfqpoint{2.437211in}{1.976851in}}%
\pgfpathlineto{\pgfqpoint{2.446018in}{2.100365in}}%
\pgfpathlineto{\pgfqpoint{2.463632in}{1.674896in}}%
\pgfpathlineto{\pgfqpoint{2.472438in}{1.668037in}}%
\pgfpathlineto{\pgfqpoint{2.481245in}{1.640574in}}%
\pgfpathlineto{\pgfqpoint{2.490052in}{1.722935in}}%
\pgfpathlineto{\pgfqpoint{2.498859in}{1.784692in}}%
\pgfpathlineto{\pgfqpoint{2.507666in}{1.661150in}}%
\pgfpathlineto{\pgfqpoint{2.516472in}{1.482737in}}%
\pgfpathlineto{\pgfqpoint{2.542893in}{2.052326in}}%
\pgfpathlineto{\pgfqpoint{2.551700in}{1.894489in}}%
\pgfpathlineto{\pgfqpoint{2.560507in}{1.894489in}}%
\pgfpathlineto{\pgfqpoint{2.569313in}{1.956246in}}%
\pgfpathlineto{\pgfqpoint{2.578120in}{2.251343in}}%
\pgfpathlineto{\pgfqpoint{2.586927in}{1.558212in}}%
\pgfpathlineto{\pgfqpoint{2.604541in}{1.146460in}}%
\pgfpathlineto{\pgfqpoint{2.613347in}{1.187641in}}%
\pgfpathlineto{\pgfqpoint{2.622154in}{1.256257in}}%
\pgfpathlineto{\pgfqpoint{2.630961in}{0.933697in}}%
\pgfpathlineto{\pgfqpoint{2.639768in}{1.064098in}}%
\pgfpathlineto{\pgfqpoint{2.648575in}{1.496455in}}%
\pgfpathlineto{\pgfqpoint{2.657382in}{1.873913in}}%
\pgfpathlineto{\pgfqpoint{2.666188in}{1.963133in}}%
\pgfpathlineto{\pgfqpoint{2.674995in}{2.313128in}}%
\pgfpathlineto{\pgfqpoint{2.683802in}{2.265089in}}%
\pgfpathlineto{\pgfqpoint{2.692609in}{2.271947in}}%
\pgfpathlineto{\pgfqpoint{2.710222in}{1.956246in}}%
\pgfpathlineto{\pgfqpoint{2.719029in}{2.079789in}}%
\pgfpathlineto{\pgfqpoint{2.727836in}{2.114111in}}%
\pgfpathlineto{\pgfqpoint{2.736643in}{2.443501in}}%
\pgfpathlineto{\pgfqpoint{2.745450in}{1.777834in}}%
\pgfpathlineto{\pgfqpoint{2.754257in}{1.915094in}}%
\pgfpathlineto{\pgfqpoint{2.763063in}{1.674896in}}%
\pgfpathlineto{\pgfqpoint{2.771870in}{1.935670in}}%
\pgfpathlineto{\pgfqpoint{2.780677in}{1.709190in}}%
\pgfpathlineto{\pgfqpoint{2.789484in}{1.640574in}}%
\pgfpathlineto{\pgfqpoint{2.798291in}{1.661150in}}%
\pgfpathlineto{\pgfqpoint{2.807097in}{1.496455in}}%
\pgfpathlineto{\pgfqpoint{2.815904in}{1.510172in}}%
\pgfpathlineto{\pgfqpoint{2.824711in}{1.414093in}}%
\pgfpathlineto{\pgfqpoint{2.833518in}{0.988623in}}%
\pgfpathlineto{\pgfqpoint{2.851132in}{1.530777in}}%
\pgfpathlineto{\pgfqpoint{2.859938in}{1.599393in}}%
\pgfpathlineto{\pgfqpoint{2.868745in}{1.503314in}}%
\pgfpathlineto{\pgfqpoint{2.877552in}{1.517031in}}%
\pgfpathlineto{\pgfqpoint{2.886359in}{1.606252in}}%
\pgfpathlineto{\pgfqpoint{2.895166in}{1.661150in}}%
\pgfpathlineto{\pgfqpoint{2.903972in}{1.770975in}}%
\pgfpathlineto{\pgfqpoint{2.912779in}{2.024891in}}%
\pgfpathlineto{\pgfqpoint{2.921586in}{1.853308in}}%
\pgfpathlineto{\pgfqpoint{2.930393in}{2.045467in}}%
\pgfpathlineto{\pgfqpoint{2.939200in}{2.107252in}}%
\pgfpathlineto{\pgfqpoint{2.948007in}{1.928811in}}%
\pgfpathlineto{\pgfqpoint{2.956813in}{2.258230in}}%
\pgfpathlineto{\pgfqpoint{2.965620in}{2.093507in}}%
\pgfpathlineto{\pgfqpoint{2.974427in}{1.976851in}}%
\pgfpathlineto{\pgfqpoint{2.983234in}{1.764116in}}%
\pgfpathlineto{\pgfqpoint{2.992041in}{1.908207in}}%
\pgfpathlineto{\pgfqpoint{3.000847in}{1.702331in}}%
\pgfpathlineto{\pgfqpoint{3.009654in}{1.606252in}}%
\pgfpathlineto{\pgfqpoint{3.018461in}{1.633715in}}%
\pgfpathlineto{\pgfqpoint{3.027268in}{1.544494in}}%
\pgfpathlineto{\pgfqpoint{3.036075in}{1.633715in}}%
\pgfpathlineto{\pgfqpoint{3.044882in}{1.805269in}}%
\pgfpathlineto{\pgfqpoint{3.053688in}{2.018032in}}%
\pgfpathlineto{\pgfqpoint{3.062495in}{1.661150in}}%
\pgfpathlineto{\pgfqpoint{3.071302in}{1.887630in}}%
\pgfpathlineto{\pgfqpoint{3.080109in}{1.969992in}}%
\pgfpathlineto{\pgfqpoint{3.088916in}{1.626856in}}%
\pgfpathlineto{\pgfqpoint{3.097722in}{1.544494in}}%
\pgfpathlineto{\pgfqpoint{3.106529in}{1.887630in}}%
\pgfpathlineto{\pgfqpoint{3.115336in}{2.402349in}}%
\pgfpathlineto{\pgfqpoint{3.124143in}{2.093507in}}%
\pgfpathlineto{\pgfqpoint{3.132950in}{2.038608in}}%
\pgfpathlineto{\pgfqpoint{3.141757in}{1.640574in}}%
\pgfpathlineto{\pgfqpoint{3.150563in}{1.400376in}}%
\pgfpathlineto{\pgfqpoint{3.168177in}{1.517031in}}%
\pgfpathlineto{\pgfqpoint{3.176984in}{1.386658in}}%
\pgfpathlineto{\pgfqpoint{3.185791in}{1.661150in}}%
\pgfpathlineto{\pgfqpoint{3.194597in}{2.107252in}}%
\pgfpathlineto{\pgfqpoint{3.203404in}{2.169009in}}%
\pgfpathlineto{\pgfqpoint{3.212211in}{2.155292in}}%
\pgfpathlineto{\pgfqpoint{3.221018in}{1.825873in}}%
\pgfpathlineto{\pgfqpoint{3.229825in}{2.024891in}}%
\pgfpathlineto{\pgfqpoint{3.238632in}{2.271947in}}%
\pgfpathlineto{\pgfqpoint{3.247438in}{2.148405in}}%
\pgfpathlineto{\pgfqpoint{3.256245in}{2.100365in}}%
\pgfpathlineto{\pgfqpoint{3.265052in}{2.114111in}}%
\pgfpathlineto{\pgfqpoint{3.273859in}{1.784692in}}%
\pgfpathlineto{\pgfqpoint{3.282666in}{1.983710in}}%
\pgfpathlineto{\pgfqpoint{3.291472in}{1.743512in}}%
\pgfpathlineto{\pgfqpoint{3.300279in}{1.976851in}}%
\pgfpathlineto{\pgfqpoint{3.317893in}{2.175868in}}%
\pgfpathlineto{\pgfqpoint{3.326700in}{2.237625in}}%
\pgfpathlineto{\pgfqpoint{3.335507in}{2.278806in}}%
\pgfpathlineto{\pgfqpoint{3.344313in}{2.203303in}}%
\pgfpathlineto{\pgfqpoint{3.353120in}{2.011173in}}%
\pgfpathlineto{\pgfqpoint{3.361927in}{1.983710in}}%
\pgfpathlineto{\pgfqpoint{3.370734in}{1.935670in}}%
\pgfpathlineto{\pgfqpoint{3.379541in}{1.976851in}}%
\pgfpathlineto{\pgfqpoint{3.388347in}{2.093507in}}%
\pgfpathlineto{\pgfqpoint{3.397154in}{2.443501in}}%
\pgfpathlineto{\pgfqpoint{3.405961in}{2.505287in}}%
\pgfpathlineto{\pgfqpoint{3.414768in}{2.319987in}}%
\pgfpathlineto{\pgfqpoint{3.423575in}{2.086648in}}%
\pgfpathlineto{\pgfqpoint{3.432382in}{2.210190in}}%
\pgfpathlineto{\pgfqpoint{3.441188in}{1.887630in}}%
\pgfpathlineto{\pgfqpoint{3.449995in}{2.093507in}}%
\pgfpathlineto{\pgfqpoint{3.458802in}{2.230767in}}%
\pgfpathlineto{\pgfqpoint{3.467609in}{2.217049in}}%
\pgfpathlineto{\pgfqpoint{3.476416in}{1.873913in}}%
\pgfpathlineto{\pgfqpoint{3.485222in}{1.942529in}}%
\pgfpathlineto{\pgfqpoint{3.494029in}{2.409207in}}%
\pgfpathlineto{\pgfqpoint{3.502836in}{2.573903in}}%
\pgfpathlineto{\pgfqpoint{3.511643in}{2.443501in}}%
\pgfpathlineto{\pgfqpoint{3.520450in}{2.519004in}}%
\pgfpathlineto{\pgfqpoint{3.529257in}{2.930756in}}%
\pgfpathlineto{\pgfqpoint{3.538063in}{2.573903in}}%
\pgfpathlineto{\pgfqpoint{3.546870in}{2.409207in}}%
\pgfpathlineto{\pgfqpoint{3.555677in}{2.326846in}}%
\pgfpathlineto{\pgfqpoint{3.564484in}{2.018032in}}%
\pgfpathlineto{\pgfqpoint{3.573291in}{2.059213in}}%
\pgfpathlineto{\pgfqpoint{3.582097in}{2.217049in}}%
\pgfpathlineto{\pgfqpoint{3.590904in}{1.976851in}}%
\pgfpathlineto{\pgfqpoint{3.599711in}{1.921953in}}%
\pgfpathlineto{\pgfqpoint{3.608518in}{1.935670in}}%
\pgfpathlineto{\pgfqpoint{3.617325in}{1.846450in}}%
\pgfpathlineto{\pgfqpoint{3.626132in}{1.791551in}}%
\pgfpathlineto{\pgfqpoint{3.634938in}{1.825873in}}%
\pgfpathlineto{\pgfqpoint{3.643745in}{1.791551in}}%
\pgfpathlineto{\pgfqpoint{3.652552in}{2.100365in}}%
\pgfpathlineto{\pgfqpoint{3.661359in}{2.059213in}}%
\pgfpathlineto{\pgfqpoint{3.670166in}{1.819015in}}%
\pgfpathlineto{\pgfqpoint{3.678972in}{1.846450in}}%
\pgfpathlineto{\pgfqpoint{3.687779in}{1.558212in}}%
\pgfpathlineto{\pgfqpoint{3.696586in}{1.921953in}}%
\pgfpathlineto{\pgfqpoint{3.705393in}{1.887630in}}%
\pgfpathlineto{\pgfqpoint{3.714200in}{1.722935in}}%
\pgfpathlineto{\pgfqpoint{3.723007in}{1.736653in}}%
\pgfpathlineto{\pgfqpoint{3.731813in}{1.770975in}}%
\pgfpathlineto{\pgfqpoint{3.740620in}{1.956246in}}%
\pgfpathlineto{\pgfqpoint{3.749427in}{1.722935in}}%
\pgfpathlineto{\pgfqpoint{3.758234in}{2.031749in}}%
\pgfpathlineto{\pgfqpoint{3.767041in}{1.935670in}}%
\pgfpathlineto{\pgfqpoint{3.775847in}{1.674896in}}%
\pgfpathlineto{\pgfqpoint{3.784654in}{1.777834in}}%
\pgfpathlineto{\pgfqpoint{3.793461in}{1.613110in}}%
\pgfpathlineto{\pgfqpoint{3.802268in}{1.565071in}}%
\pgfpathlineto{\pgfqpoint{3.811075in}{1.372912in}}%
\pgfpathlineto{\pgfqpoint{3.819882in}{1.359195in}}%
\pgfpathlineto{\pgfqpoint{3.828688in}{1.523918in}}%
\pgfpathlineto{\pgfqpoint{3.837495in}{1.496455in}}%
\pgfpathlineto{\pgfqpoint{3.846302in}{1.757229in}}%
\pgfpathlineto{\pgfqpoint{3.855109in}{1.695472in}}%
\pgfpathlineto{\pgfqpoint{3.863916in}{2.011173in}}%
\pgfpathlineto{\pgfqpoint{3.872722in}{1.757229in}}%
\pgfpathlineto{\pgfqpoint{3.881529in}{1.626856in}}%
\pgfpathlineto{\pgfqpoint{3.890336in}{1.729794in}}%
\pgfpathlineto{\pgfqpoint{3.899143in}{1.757229in}}%
\pgfpathlineto{\pgfqpoint{3.907950in}{2.086648in}}%
\pgfpathlineto{\pgfqpoint{3.916757in}{2.018032in}}%
\pgfpathlineto{\pgfqpoint{3.925563in}{2.326846in}}%
\pgfpathlineto{\pgfqpoint{3.934370in}{2.072930in}}%
\pgfpathlineto{\pgfqpoint{3.943177in}{2.409207in}}%
\pgfpathlineto{\pgfqpoint{3.951984in}{2.024891in}}%
\pgfpathlineto{\pgfqpoint{3.960791in}{2.066071in}}%
\pgfpathlineto{\pgfqpoint{3.969597in}{1.770975in}}%
\pgfpathlineto{\pgfqpoint{3.978404in}{1.530777in}}%
\pgfpathlineto{\pgfqpoint{3.987211in}{1.812128in}}%
\pgfpathlineto{\pgfqpoint{3.996018in}{1.928811in}}%
\pgfpathlineto{\pgfqpoint{4.004825in}{1.928811in}}%
\pgfpathlineto{\pgfqpoint{4.013632in}{2.011173in}}%
\pgfpathlineto{\pgfqpoint{4.022438in}{2.196445in}}%
\pgfpathlineto{\pgfqpoint{4.031245in}{2.141546in}}%
\pgfpathlineto{\pgfqpoint{4.040052in}{2.045467in}}%
\pgfpathlineto{\pgfqpoint{4.048859in}{2.004286in}}%
\pgfpathlineto{\pgfqpoint{4.057666in}{1.983710in}}%
\pgfpathlineto{\pgfqpoint{4.066472in}{1.846450in}}%
\pgfpathlineto{\pgfqpoint{4.075279in}{1.915094in}}%
\pgfpathlineto{\pgfqpoint{4.084086in}{1.798410in}}%
\pgfpathlineto{\pgfqpoint{4.092893in}{1.990569in}}%
\pgfpathlineto{\pgfqpoint{4.101700in}{1.839591in}}%
\pgfpathlineto{\pgfqpoint{4.110507in}{1.969992in}}%
\pgfpathlineto{\pgfqpoint{4.119313in}{2.230767in}}%
\pgfpathlineto{\pgfqpoint{4.128120in}{2.196445in}}%
\pgfpathlineto{\pgfqpoint{4.136927in}{2.182727in}}%
\pgfpathlineto{\pgfqpoint{4.145734in}{2.100365in}}%
\pgfpathlineto{\pgfqpoint{4.154541in}{2.422925in}}%
\pgfpathlineto{\pgfqpoint{4.163347in}{2.601366in}}%
\pgfpathlineto{\pgfqpoint{4.172154in}{2.546468in}}%
\pgfpathlineto{\pgfqpoint{4.180961in}{2.278806in}}%
\pgfpathlineto{\pgfqpoint{4.198575in}{2.738598in}}%
\pgfpathlineto{\pgfqpoint{4.207382in}{2.258230in}}%
\pgfpathlineto{\pgfqpoint{4.216188in}{2.477823in}}%
\pgfpathlineto{\pgfqpoint{4.224995in}{2.416066in}}%
\pgfpathlineto{\pgfqpoint{4.233802in}{2.306269in}}%
\pgfpathlineto{\pgfqpoint{4.242609in}{2.100365in}}%
\pgfpathlineto{\pgfqpoint{4.251416in}{2.052326in}}%
\pgfpathlineto{\pgfqpoint{4.260222in}{2.299383in}}%
\pgfpathlineto{\pgfqpoint{4.269029in}{2.210190in}}%
\pgfpathlineto{\pgfqpoint{4.277836in}{2.162151in}}%
\pgfpathlineto{\pgfqpoint{4.286643in}{1.894489in}}%
\pgfpathlineto{\pgfqpoint{4.295450in}{1.853308in}}%
\pgfpathlineto{\pgfqpoint{4.304257in}{2.244484in}}%
\pgfpathlineto{\pgfqpoint{4.321870in}{1.770975in}}%
\pgfpathlineto{\pgfqpoint{4.330677in}{1.764116in}}%
\pgfpathlineto{\pgfqpoint{4.339484in}{1.599393in}}%
\pgfpathlineto{\pgfqpoint{4.348291in}{1.565071in}}%
\pgfpathlineto{\pgfqpoint{4.357097in}{1.819015in}}%
\pgfpathlineto{\pgfqpoint{4.365904in}{2.018032in}}%
\pgfpathlineto{\pgfqpoint{4.374711in}{2.134687in}}%
\pgfpathlineto{\pgfqpoint{4.383518in}{1.969992in}}%
\pgfpathlineto{\pgfqpoint{4.392325in}{1.839591in}}%
\pgfpathlineto{\pgfqpoint{4.401132in}{1.386658in}}%
\pgfpathlineto{\pgfqpoint{4.409938in}{1.619997in}}%
\pgfpathlineto{\pgfqpoint{4.418745in}{1.626856in}}%
\pgfpathlineto{\pgfqpoint{4.436359in}{1.846450in}}%
\pgfpathlineto{\pgfqpoint{4.445166in}{2.052326in}}%
\pgfpathlineto{\pgfqpoint{4.453972in}{2.594507in}}%
\pgfpathlineto{\pgfqpoint{4.462779in}{2.388603in}}%
\pgfpathlineto{\pgfqpoint{4.471586in}{2.285665in}}%
\pgfpathlineto{\pgfqpoint{4.480393in}{2.134687in}}%
\pgfpathlineto{\pgfqpoint{4.489200in}{2.079789in}}%
\pgfpathlineto{\pgfqpoint{4.498007in}{1.997427in}}%
\pgfpathlineto{\pgfqpoint{4.506813in}{2.052326in}}%
\pgfpathlineto{\pgfqpoint{4.515620in}{2.237625in}}%
\pgfpathlineto{\pgfqpoint{4.524427in}{1.997427in}}%
\pgfpathlineto{\pgfqpoint{4.533234in}{2.429784in}}%
\pgfpathlineto{\pgfqpoint{4.542041in}{1.668037in}}%
\pgfpathlineto{\pgfqpoint{4.550847in}{1.468992in}}%
\pgfpathlineto{\pgfqpoint{4.559654in}{1.420952in}}%
\pgfpathlineto{\pgfqpoint{4.568461in}{1.235680in}}%
\pgfpathlineto{\pgfqpoint{4.577268in}{1.530777in}}%
\pgfpathlineto{\pgfqpoint{4.586075in}{1.517031in}}%
\pgfpathlineto{\pgfqpoint{4.594882in}{1.283720in}}%
\pgfpathlineto{\pgfqpoint{4.603688in}{1.654291in}}%
\pgfpathlineto{\pgfqpoint{4.612495in}{1.647432in}}%
\pgfpathlineto{\pgfqpoint{4.621302in}{1.338618in}}%
\pgfpathlineto{\pgfqpoint{4.630109in}{1.736653in}}%
\pgfpathlineto{\pgfqpoint{4.638916in}{1.880772in}}%
\pgfpathlineto{\pgfqpoint{4.647722in}{1.750370in}}%
\pgfpathlineto{\pgfqpoint{4.665336in}{1.990569in}}%
\pgfpathlineto{\pgfqpoint{4.682950in}{1.517031in}}%
\pgfpathlineto{\pgfqpoint{4.691757in}{1.379799in}}%
\pgfpathlineto{\pgfqpoint{4.700563in}{1.668037in}}%
\pgfpathlineto{\pgfqpoint{4.709370in}{1.688613in}}%
\pgfpathlineto{\pgfqpoint{4.718177in}{1.661150in}}%
\pgfpathlineto{\pgfqpoint{4.726984in}{1.846450in}}%
\pgfpathlineto{\pgfqpoint{4.735791in}{1.983710in}}%
\pgfpathlineto{\pgfqpoint{4.744597in}{2.162151in}}%
\pgfpathlineto{\pgfqpoint{4.753404in}{2.066071in}}%
\pgfpathlineto{\pgfqpoint{4.762211in}{2.217049in}}%
\pgfpathlineto{\pgfqpoint{4.771018in}{2.169009in}}%
\pgfpathlineto{\pgfqpoint{4.779825in}{2.189586in}}%
\pgfpathlineto{\pgfqpoint{4.788632in}{2.347422in}}%
\pgfpathlineto{\pgfqpoint{4.797438in}{2.004286in}}%
\pgfpathlineto{\pgfqpoint{4.806245in}{1.921953in}}%
\pgfpathlineto{\pgfqpoint{4.815052in}{2.169009in}}%
\pgfpathlineto{\pgfqpoint{4.823859in}{2.066071in}}%
\pgfpathlineto{\pgfqpoint{4.832666in}{1.949388in}}%
\pgfpathlineto{\pgfqpoint{4.841472in}{1.921953in}}%
\pgfpathlineto{\pgfqpoint{4.850279in}{1.990569in}}%
\pgfpathlineto{\pgfqpoint{4.859086in}{1.853308in}}%
\pgfpathlineto{\pgfqpoint{4.867893in}{1.791551in}}%
\pgfpathlineto{\pgfqpoint{4.876700in}{1.709190in}}%
\pgfpathlineto{\pgfqpoint{4.885507in}{1.805269in}}%
\pgfpathlineto{\pgfqpoint{4.894313in}{1.668037in}}%
\pgfpathlineto{\pgfqpoint{4.903120in}{1.558212in}}%
\pgfpathlineto{\pgfqpoint{4.920734in}{1.626856in}}%
\pgfpathlineto{\pgfqpoint{4.929541in}{1.942529in}}%
\pgfpathlineto{\pgfqpoint{4.938347in}{1.963133in}}%
\pgfpathlineto{\pgfqpoint{4.947154in}{2.169009in}}%
\pgfpathlineto{\pgfqpoint{4.955961in}{2.155292in}}%
\pgfpathlineto{\pgfqpoint{4.964768in}{2.024891in}}%
\pgfpathlineto{\pgfqpoint{4.973575in}{2.189586in}}%
\pgfpathlineto{\pgfqpoint{4.982382in}{2.148405in}}%
\pgfpathlineto{\pgfqpoint{4.991188in}{2.052326in}}%
\pgfpathlineto{\pgfqpoint{4.999995in}{1.894489in}}%
\pgfpathlineto{\pgfqpoint{5.008802in}{1.901348in}}%
\pgfpathlineto{\pgfqpoint{5.017609in}{2.059213in}}%
\pgfpathlineto{\pgfqpoint{5.026416in}{2.079789in}}%
\pgfpathlineto{\pgfqpoint{5.035222in}{1.654291in}}%
\pgfpathlineto{\pgfqpoint{5.044029in}{1.963133in}}%
\pgfpathlineto{\pgfqpoint{5.052836in}{1.956246in}}%
\pgfpathlineto{\pgfqpoint{5.061643in}{1.894489in}}%
\pgfpathlineto{\pgfqpoint{5.070450in}{2.024891in}}%
\pgfpathlineto{\pgfqpoint{5.079257in}{1.702331in}}%
\pgfpathlineto{\pgfqpoint{5.088063in}{1.770975in}}%
\pgfpathlineto{\pgfqpoint{5.096870in}{1.558212in}}%
\pgfpathlineto{\pgfqpoint{5.105677in}{1.921953in}}%
\pgfpathlineto{\pgfqpoint{5.114484in}{1.894489in}}%
\pgfpathlineto{\pgfqpoint{5.123291in}{2.004286in}}%
\pgfpathlineto{\pgfqpoint{5.132097in}{1.729794in}}%
\pgfpathlineto{\pgfqpoint{5.140904in}{1.613110in}}%
\pgfpathlineto{\pgfqpoint{5.149711in}{1.819015in}}%
\pgfpathlineto{\pgfqpoint{5.158518in}{1.716076in}}%
\pgfpathlineto{\pgfqpoint{5.167325in}{1.757229in}}%
\pgfpathlineto{\pgfqpoint{5.176132in}{1.537636in}}%
\pgfpathlineto{\pgfqpoint{5.184938in}{1.949388in}}%
\pgfpathlineto{\pgfqpoint{5.193745in}{1.963133in}}%
\pgfpathlineto{\pgfqpoint{5.202552in}{1.825873in}}%
\pgfpathlineto{\pgfqpoint{5.211359in}{1.709190in}}%
\pgfpathlineto{\pgfqpoint{5.220166in}{2.024891in}}%
\pgfpathlineto{\pgfqpoint{5.228972in}{2.189586in}}%
\pgfpathlineto{\pgfqpoint{5.237779in}{1.976851in}}%
\pgfpathlineto{\pgfqpoint{5.246586in}{1.956246in}}%
\pgfpathlineto{\pgfqpoint{5.255393in}{2.203303in}}%
\pgfpathlineto{\pgfqpoint{5.264200in}{2.148405in}}%
\pgfpathlineto{\pgfqpoint{5.273007in}{1.633715in}}%
\pgfpathlineto{\pgfqpoint{5.281813in}{1.565071in}}%
\pgfpathlineto{\pgfqpoint{5.290620in}{1.565071in}}%
\pgfpathlineto{\pgfqpoint{5.299427in}{1.633715in}}%
\pgfpathlineto{\pgfqpoint{5.308234in}{1.510172in}}%
\pgfpathlineto{\pgfqpoint{5.317041in}{1.709190in}}%
\pgfpathlineto{\pgfqpoint{5.325847in}{1.736653in}}%
\pgfpathlineto{\pgfqpoint{5.334654in}{1.633715in}}%
\pgfpathlineto{\pgfqpoint{5.343461in}{1.729794in}}%
\pgfpathlineto{\pgfqpoint{5.352268in}{1.702331in}}%
\pgfpathlineto{\pgfqpoint{5.361075in}{1.578816in}}%
\pgfpathlineto{\pgfqpoint{5.369882in}{1.681754in}}%
\pgfpathlineto{\pgfqpoint{5.378688in}{1.455274in}}%
\pgfpathlineto{\pgfqpoint{5.387495in}{1.846450in}}%
\pgfpathlineto{\pgfqpoint{5.396302in}{1.798410in}}%
\pgfpathlineto{\pgfqpoint{5.405109in}{1.846450in}}%
\pgfpathlineto{\pgfqpoint{5.413916in}{1.935670in}}%
\pgfpathlineto{\pgfqpoint{5.422722in}{2.045467in}}%
\pgfpathlineto{\pgfqpoint{5.431529in}{1.709190in}}%
\pgfpathlineto{\pgfqpoint{5.440336in}{1.647432in}}%
\pgfpathlineto{\pgfqpoint{5.449143in}{1.606252in}}%
\pgfpathlineto{\pgfqpoint{5.457950in}{1.805269in}}%
\pgfpathlineto{\pgfqpoint{5.466757in}{2.038608in}}%
\pgfpathlineto{\pgfqpoint{5.475563in}{2.100365in}}%
\pgfpathlineto{\pgfqpoint{5.484370in}{1.702331in}}%
\pgfpathlineto{\pgfqpoint{5.493177in}{1.606252in}}%
\pgfpathlineto{\pgfqpoint{5.501984in}{1.606252in}}%
\pgfpathlineto{\pgfqpoint{5.510791in}{1.496455in}}%
\pgfpathlineto{\pgfqpoint{5.519597in}{1.420952in}}%
\pgfpathlineto{\pgfqpoint{5.528404in}{1.571958in}}%
\pgfpathlineto{\pgfqpoint{5.537211in}{1.599393in}}%
\pgfpathlineto{\pgfqpoint{5.546018in}{1.558212in}}%
\pgfpathlineto{\pgfqpoint{5.554825in}{1.647432in}}%
\pgfpathlineto{\pgfqpoint{5.563632in}{1.770975in}}%
\pgfpathlineto{\pgfqpoint{5.572438in}{1.770975in}}%
\pgfpathlineto{\pgfqpoint{5.581245in}{1.517031in}}%
\pgfpathlineto{\pgfqpoint{5.590052in}{1.853308in}}%
\pgfpathlineto{\pgfqpoint{5.598859in}{1.915094in}}%
\pgfpathlineto{\pgfqpoint{5.607666in}{2.134687in}}%
\pgfpathlineto{\pgfqpoint{5.616472in}{2.100365in}}%
\pgfpathlineto{\pgfqpoint{5.625279in}{2.024891in}}%
\pgfpathlineto{\pgfqpoint{5.634086in}{1.633715in}}%
\pgfpathlineto{\pgfqpoint{5.642893in}{2.114111in}}%
\pgfpathlineto{\pgfqpoint{5.651700in}{2.278806in}}%
\pgfpathlineto{\pgfqpoint{5.660507in}{2.244484in}}%
\pgfpathlineto{\pgfqpoint{5.669313in}{2.072930in}}%
\pgfpathlineto{\pgfqpoint{5.678120in}{2.011173in}}%
\pgfpathlineto{\pgfqpoint{5.686927in}{1.770975in}}%
\pgfpathlineto{\pgfqpoint{5.695734in}{1.702331in}}%
\pgfpathlineto{\pgfqpoint{5.704541in}{1.860167in}}%
\pgfpathlineto{\pgfqpoint{5.713347in}{1.921953in}}%
\pgfpathlineto{\pgfqpoint{5.722154in}{2.031749in}}%
\pgfpathlineto{\pgfqpoint{5.730961in}{1.791551in}}%
\pgfpathlineto{\pgfqpoint{5.739768in}{1.887630in}}%
\pgfpathlineto{\pgfqpoint{5.748575in}{2.292524in}}%
\pgfpathlineto{\pgfqpoint{5.757382in}{2.018032in}}%
\pgfpathlineto{\pgfqpoint{5.766188in}{2.148405in}}%
\pgfpathlineto{\pgfqpoint{5.774995in}{1.990569in}}%
\pgfpathlineto{\pgfqpoint{5.783802in}{1.997427in}}%
\pgfpathlineto{\pgfqpoint{5.792609in}{1.812128in}}%
\pgfpathlineto{\pgfqpoint{5.801416in}{2.189586in}}%
\pgfpathlineto{\pgfqpoint{5.810222in}{2.004286in}}%
\pgfpathlineto{\pgfqpoint{5.819029in}{1.867054in}}%
\pgfpathlineto{\pgfqpoint{5.827836in}{1.860167in}}%
\pgfpathlineto{\pgfqpoint{5.836643in}{1.908207in}}%
\pgfpathlineto{\pgfqpoint{5.845450in}{1.901348in}}%
\pgfpathlineto{\pgfqpoint{5.854257in}{2.230767in}}%
\pgfpathlineto{\pgfqpoint{5.863063in}{1.983710in}}%
\pgfpathlineto{\pgfqpoint{5.871870in}{1.887630in}}%
\pgfpathlineto{\pgfqpoint{5.880677in}{1.928811in}}%
\pgfpathlineto{\pgfqpoint{5.889484in}{2.175868in}}%
\pgfpathlineto{\pgfqpoint{5.898291in}{2.347422in}}%
\pgfpathlineto{\pgfqpoint{5.907097in}{2.450388in}}%
\pgfpathlineto{\pgfqpoint{5.915904in}{2.313128in}}%
\pgfpathlineto{\pgfqpoint{5.933518in}{1.969992in}}%
\pgfpathlineto{\pgfqpoint{5.942325in}{1.839591in}}%
\pgfpathlineto{\pgfqpoint{5.951132in}{1.812128in}}%
\pgfpathlineto{\pgfqpoint{5.959938in}{1.867054in}}%
\pgfpathlineto{\pgfqpoint{5.968745in}{2.038608in}}%
\pgfpathlineto{\pgfqpoint{5.977552in}{2.059213in}}%
\pgfpathlineto{\pgfqpoint{5.986359in}{1.770975in}}%
\pgfpathlineto{\pgfqpoint{5.995166in}{1.585675in}}%
\pgfpathlineto{\pgfqpoint{6.003972in}{1.571958in}}%
\pgfpathlineto{\pgfqpoint{6.012779in}{1.537636in}}%
\pgfpathlineto{\pgfqpoint{6.021586in}{1.530777in}}%
\pgfpathlineto{\pgfqpoint{6.030393in}{1.592534in}}%
\pgfpathlineto{\pgfqpoint{6.039200in}{2.107252in}}%
\pgfpathlineto{\pgfqpoint{6.048007in}{2.024891in}}%
\pgfpathlineto{\pgfqpoint{6.056813in}{2.162151in}}%
\pgfpathlineto{\pgfqpoint{6.065620in}{1.928811in}}%
\pgfpathlineto{\pgfqpoint{6.074427in}{2.038608in}}%
\pgfpathlineto{\pgfqpoint{6.083234in}{1.901348in}}%
\pgfpathlineto{\pgfqpoint{6.092041in}{1.606252in}}%
\pgfpathlineto{\pgfqpoint{6.100847in}{1.846450in}}%
\pgfpathlineto{\pgfqpoint{6.109654in}{1.633715in}}%
\pgfpathlineto{\pgfqpoint{6.118461in}{1.551353in}}%
\pgfpathlineto{\pgfqpoint{6.127268in}{1.654291in}}%
\pgfpathlineto{\pgfqpoint{6.136075in}{1.915094in}}%
\pgfpathlineto{\pgfqpoint{6.144882in}{2.114111in}}%
\pgfpathlineto{\pgfqpoint{6.162495in}{1.407234in}}%
\pgfpathlineto{\pgfqpoint{6.180109in}{2.223908in}}%
\pgfpathlineto{\pgfqpoint{6.188916in}{2.031749in}}%
\pgfpathlineto{\pgfqpoint{6.197722in}{1.935670in}}%
\pgfpathlineto{\pgfqpoint{6.206529in}{1.908207in}}%
\pgfpathlineto{\pgfqpoint{6.215336in}{1.475878in}}%
\pgfpathlineto{\pgfqpoint{6.224143in}{1.668037in}}%
\pgfpathlineto{\pgfqpoint{6.232950in}{1.777834in}}%
\pgfpathlineto{\pgfqpoint{6.241757in}{1.770975in}}%
\pgfpathlineto{\pgfqpoint{6.250563in}{1.757229in}}%
\pgfpathlineto{\pgfqpoint{6.259370in}{1.784692in}}%
\pgfpathlineto{\pgfqpoint{6.268177in}{1.867054in}}%
\pgfpathlineto{\pgfqpoint{6.276984in}{1.688613in}}%
\pgfpathlineto{\pgfqpoint{6.285791in}{1.942529in}}%
\pgfpathlineto{\pgfqpoint{6.294597in}{1.887630in}}%
\pgfpathlineto{\pgfqpoint{6.303404in}{2.223908in}}%
\pgfpathlineto{\pgfqpoint{6.312211in}{2.237625in}}%
\pgfpathlineto{\pgfqpoint{6.321018in}{2.024891in}}%
\pgfpathlineto{\pgfqpoint{6.329825in}{2.059213in}}%
\pgfpathlineto{\pgfqpoint{6.338632in}{2.066071in}}%
\pgfpathlineto{\pgfqpoint{6.347438in}{2.313128in}}%
\pgfpathlineto{\pgfqpoint{6.356245in}{2.189586in}}%
\pgfpathlineto{\pgfqpoint{6.365052in}{2.470965in}}%
\pgfpathlineto{\pgfqpoint{6.373859in}{2.162151in}}%
\pgfpathlineto{\pgfqpoint{6.382666in}{2.217049in}}%
\pgfpathlineto{\pgfqpoint{6.391472in}{1.887630in}}%
\pgfpathlineto{\pgfqpoint{6.400279in}{1.915094in}}%
\pgfpathlineto{\pgfqpoint{6.409086in}{1.949388in}}%
\pgfpathlineto{\pgfqpoint{6.417893in}{2.114111in}}%
\pgfpathlineto{\pgfqpoint{6.426700in}{2.422925in}}%
\pgfpathlineto{\pgfqpoint{6.435507in}{2.278806in}}%
\pgfpathlineto{\pgfqpoint{6.444313in}{2.072930in}}%
\pgfpathlineto{\pgfqpoint{6.453120in}{2.018032in}}%
\pgfpathlineto{\pgfqpoint{6.461927in}{1.983710in}}%
\pgfpathlineto{\pgfqpoint{6.470734in}{1.873913in}}%
\pgfpathlineto{\pgfqpoint{6.479541in}{2.114111in}}%
\pgfpathlineto{\pgfqpoint{6.488347in}{2.422925in}}%
\pgfpathlineto{\pgfqpoint{6.497154in}{2.278806in}}%
\pgfpathlineto{\pgfqpoint{6.505961in}{2.368027in}}%
\pgfpathlineto{\pgfqpoint{6.514768in}{2.114111in}}%
\pgfpathlineto{\pgfqpoint{6.523575in}{2.155292in}}%
\pgfpathlineto{\pgfqpoint{6.532382in}{1.983710in}}%
\pgfpathlineto{\pgfqpoint{6.541188in}{1.867054in}}%
\pgfpathlineto{\pgfqpoint{6.549995in}{1.846450in}}%
\pgfpathlineto{\pgfqpoint{6.558802in}{2.182727in}}%
\pgfpathlineto{\pgfqpoint{6.567609in}{1.956246in}}%
\pgfpathlineto{\pgfqpoint{6.576416in}{2.018032in}}%
\pgfpathlineto{\pgfqpoint{6.585222in}{1.990569in}}%
\pgfpathlineto{\pgfqpoint{6.602836in}{1.757229in}}%
\pgfpathlineto{\pgfqpoint{6.611643in}{1.681754in}}%
\pgfpathlineto{\pgfqpoint{6.620450in}{1.592534in}}%
\pgfpathlineto{\pgfqpoint{6.629257in}{1.613110in}}%
\pgfpathlineto{\pgfqpoint{6.638063in}{1.668037in}}%
\pgfpathlineto{\pgfqpoint{6.646870in}{1.674896in}}%
\pgfpathlineto{\pgfqpoint{6.655677in}{1.626856in}}%
\pgfpathlineto{\pgfqpoint{6.664484in}{1.777834in}}%
\pgfpathlineto{\pgfqpoint{6.673291in}{1.805269in}}%
\pgfpathlineto{\pgfqpoint{6.682097in}{1.825873in}}%
\pgfpathlineto{\pgfqpoint{6.690904in}{1.640574in}}%
\pgfpathlineto{\pgfqpoint{6.699711in}{1.647432in}}%
\pgfpathlineto{\pgfqpoint{6.708518in}{1.722935in}}%
\pgfpathlineto{\pgfqpoint{6.734938in}{2.024891in}}%
\pgfpathlineto{\pgfqpoint{6.743745in}{2.203303in}}%
\pgfpathlineto{\pgfqpoint{6.752552in}{2.093507in}}%
\pgfpathlineto{\pgfqpoint{6.761359in}{2.210190in}}%
\pgfpathlineto{\pgfqpoint{6.770166in}{2.189586in}}%
\pgfpathlineto{\pgfqpoint{6.778972in}{1.901348in}}%
\pgfpathlineto{\pgfqpoint{6.787779in}{1.880772in}}%
\pgfpathlineto{\pgfqpoint{6.796586in}{2.093507in}}%
\pgfpathlineto{\pgfqpoint{6.805393in}{1.935670in}}%
\pgfpathlineto{\pgfqpoint{6.814200in}{1.867054in}}%
\pgfpathlineto{\pgfqpoint{6.823007in}{1.750370in}}%
\pgfpathlineto{\pgfqpoint{6.831813in}{1.764116in}}%
\pgfpathlineto{\pgfqpoint{6.840620in}{1.462133in}}%
\pgfpathlineto{\pgfqpoint{6.849427in}{1.860167in}}%
\pgfpathlineto{\pgfqpoint{6.858234in}{2.093507in}}%
\pgfpathlineto{\pgfqpoint{6.867041in}{2.210190in}}%
\pgfpathlineto{\pgfqpoint{6.875847in}{1.915094in}}%
\pgfpathlineto{\pgfqpoint{6.884654in}{1.969992in}}%
\pgfpathlineto{\pgfqpoint{6.893461in}{2.072930in}}%
\pgfpathlineto{\pgfqpoint{6.902268in}{2.024891in}}%
\pgfpathlineto{\pgfqpoint{6.911075in}{2.134687in}}%
\pgfpathlineto{\pgfqpoint{6.919882in}{2.018032in}}%
\pgfpathlineto{\pgfqpoint{6.928688in}{1.832732in}}%
\pgfpathlineto{\pgfqpoint{6.937495in}{1.722935in}}%
\pgfpathlineto{\pgfqpoint{6.946302in}{1.949388in}}%
\pgfpathlineto{\pgfqpoint{6.955109in}{2.107252in}}%
\pgfpathlineto{\pgfqpoint{6.963916in}{2.210190in}}%
\pgfpathlineto{\pgfqpoint{6.972722in}{2.237625in}}%
\pgfpathlineto{\pgfqpoint{6.981529in}{2.169009in}}%
\pgfpathlineto{\pgfqpoint{6.990336in}{2.319987in}}%
\pgfpathlineto{\pgfqpoint{6.999143in}{2.333705in}}%
\pgfpathlineto{\pgfqpoint{7.007950in}{2.285665in}}%
\pgfpathlineto{\pgfqpoint{7.016757in}{2.120970in}}%
\pgfpathlineto{\pgfqpoint{7.025563in}{2.155292in}}%
\pgfpathlineto{\pgfqpoint{7.034370in}{2.120970in}}%
\pgfpathlineto{\pgfqpoint{7.043177in}{2.066071in}}%
\pgfpathlineto{\pgfqpoint{7.051984in}{1.784692in}}%
\pgfpathlineto{\pgfqpoint{7.069597in}{2.251343in}}%
\pgfpathlineto{\pgfqpoint{7.078404in}{2.107252in}}%
\pgfpathlineto{\pgfqpoint{7.087211in}{2.223908in}}%
\pgfpathlineto{\pgfqpoint{7.096018in}{2.114111in}}%
\pgfpathlineto{\pgfqpoint{7.104825in}{2.107252in}}%
\pgfpathlineto{\pgfqpoint{7.113632in}{2.093507in}}%
\pgfpathlineto{\pgfqpoint{7.122438in}{1.956246in}}%
\pgfpathlineto{\pgfqpoint{7.131245in}{1.921953in}}%
\pgfpathlineto{\pgfqpoint{7.140052in}{1.688613in}}%
\pgfpathlineto{\pgfqpoint{7.148859in}{1.990569in}}%
\pgfpathlineto{\pgfqpoint{7.157666in}{1.832732in}}%
\pgfpathlineto{\pgfqpoint{7.166472in}{2.004286in}}%
\pgfpathlineto{\pgfqpoint{7.175279in}{1.729794in}}%
\pgfpathlineto{\pgfqpoint{7.184086in}{1.709190in}}%
\pgfpathlineto{\pgfqpoint{7.192893in}{1.873913in}}%
\pgfpathlineto{\pgfqpoint{7.201700in}{1.901348in}}%
\pgfpathlineto{\pgfqpoint{7.210507in}{1.750370in}}%
\pgfpathlineto{\pgfqpoint{7.219313in}{1.640574in}}%
\pgfpathlineto{\pgfqpoint{7.228120in}{1.269974in}}%
\pgfpathlineto{\pgfqpoint{7.236927in}{1.324873in}}%
\pgfpathlineto{\pgfqpoint{7.245734in}{1.352336in}}%
\pgfpathlineto{\pgfqpoint{7.254541in}{1.867054in}}%
\pgfpathlineto{\pgfqpoint{7.263347in}{1.668037in}}%
\pgfpathlineto{\pgfqpoint{7.272154in}{2.271947in}}%
\pgfpathlineto{\pgfqpoint{7.280961in}{1.990569in}}%
\pgfpathlineto{\pgfqpoint{7.289768in}{1.928811in}}%
\pgfpathlineto{\pgfqpoint{7.298575in}{1.839591in}}%
\pgfpathlineto{\pgfqpoint{7.307382in}{1.592534in}}%
\pgfpathlineto{\pgfqpoint{7.324995in}{2.127829in}}%
\pgfpathlineto{\pgfqpoint{7.333802in}{2.491541in}}%
\pgfpathlineto{\pgfqpoint{7.351416in}{2.361168in}}%
\pgfpathlineto{\pgfqpoint{7.360222in}{2.155292in}}%
\pgfpathlineto{\pgfqpoint{7.369029in}{2.210190in}}%
\pgfpathlineto{\pgfqpoint{7.377836in}{2.169009in}}%
\pgfpathlineto{\pgfqpoint{7.386643in}{2.155292in}}%
\pgfpathlineto{\pgfqpoint{7.395450in}{1.633715in}}%
\pgfpathlineto{\pgfqpoint{7.404257in}{1.777834in}}%
\pgfpathlineto{\pgfqpoint{7.413063in}{1.798410in}}%
\pgfpathlineto{\pgfqpoint{7.421870in}{2.018032in}}%
\pgfpathlineto{\pgfqpoint{7.430677in}{2.011173in}}%
\pgfpathlineto{\pgfqpoint{7.439484in}{2.045467in}}%
\pgfpathlineto{\pgfqpoint{7.448291in}{1.990569in}}%
\pgfpathlineto{\pgfqpoint{7.457097in}{2.141546in}}%
\pgfpathlineto{\pgfqpoint{7.465904in}{2.189586in}}%
\pgfpathlineto{\pgfqpoint{7.474711in}{1.921953in}}%
\pgfpathlineto{\pgfqpoint{7.483518in}{1.894489in}}%
\pgfpathlineto{\pgfqpoint{7.492325in}{2.093507in}}%
\pgfpathlineto{\pgfqpoint{7.501132in}{2.258230in}}%
\pgfpathlineto{\pgfqpoint{7.509938in}{2.100365in}}%
\pgfpathlineto{\pgfqpoint{7.518745in}{2.148405in}}%
\pgfpathlineto{\pgfqpoint{7.527552in}{1.942529in}}%
\pgfpathlineto{\pgfqpoint{7.536359in}{2.210190in}}%
\pgfpathlineto{\pgfqpoint{7.545166in}{1.963133in}}%
\pgfpathlineto{\pgfqpoint{7.553972in}{2.374885in}}%
\pgfpathlineto{\pgfqpoint{7.562779in}{1.983710in}}%
\pgfpathlineto{\pgfqpoint{7.571586in}{2.134687in}}%
\pgfpathlineto{\pgfqpoint{7.580393in}{2.313128in}}%
\pgfpathlineto{\pgfqpoint{7.589200in}{2.319987in}}%
\pgfpathlineto{\pgfqpoint{7.598007in}{2.464106in}}%
\pgfpathlineto{\pgfqpoint{7.606813in}{2.093507in}}%
\pgfpathlineto{\pgfqpoint{7.615620in}{2.182727in}}%
\pgfpathlineto{\pgfqpoint{7.624427in}{1.722935in}}%
\pgfpathlineto{\pgfqpoint{7.633234in}{1.956246in}}%
\pgfpathlineto{\pgfqpoint{7.642041in}{1.908207in}}%
\pgfpathlineto{\pgfqpoint{7.650847in}{1.935670in}}%
\pgfpathlineto{\pgfqpoint{7.659654in}{2.004286in}}%
\pgfpathlineto{\pgfqpoint{7.668461in}{1.976851in}}%
\pgfpathlineto{\pgfqpoint{7.677268in}{2.114111in}}%
\pgfpathlineto{\pgfqpoint{7.686075in}{1.784692in}}%
\pgfpathlineto{\pgfqpoint{7.694882in}{2.011173in}}%
\pgfpathlineto{\pgfqpoint{7.703688in}{2.038608in}}%
\pgfpathlineto{\pgfqpoint{7.712495in}{2.141546in}}%
\pgfpathlineto{\pgfqpoint{7.721302in}{1.894489in}}%
\pgfpathlineto{\pgfqpoint{7.730109in}{2.004286in}}%
\pgfpathlineto{\pgfqpoint{7.738916in}{1.949388in}}%
\pgfpathlineto{\pgfqpoint{7.747722in}{2.024891in}}%
\pgfpathlineto{\pgfqpoint{7.756529in}{2.169009in}}%
\pgfpathlineto{\pgfqpoint{7.765336in}{1.640574in}}%
\pgfpathlineto{\pgfqpoint{7.774143in}{1.571958in}}%
\pgfpathlineto{\pgfqpoint{7.782950in}{1.331760in}}%
\pgfpathlineto{\pgfqpoint{7.791757in}{1.050381in}}%
\pgfpathlineto{\pgfqpoint{7.800563in}{1.029776in}}%
\pgfpathlineto{\pgfqpoint{7.809370in}{1.084703in}}%
\pgfpathlineto{\pgfqpoint{7.818177in}{1.249398in}}%
\pgfpathlineto{\pgfqpoint{7.826984in}{1.276861in}}%
\pgfpathlineto{\pgfqpoint{7.835791in}{1.208217in}}%
\pgfpathlineto{\pgfqpoint{7.844597in}{1.455274in}}%
\pgfpathlineto{\pgfqpoint{7.853404in}{1.770975in}}%
\pgfpathlineto{\pgfqpoint{7.862211in}{1.626856in}}%
\pgfpathlineto{\pgfqpoint{7.871018in}{1.619997in}}%
\pgfpathlineto{\pgfqpoint{7.879825in}{1.805269in}}%
\pgfpathlineto{\pgfqpoint{7.888632in}{1.798410in}}%
\pgfpathlineto{\pgfqpoint{7.897438in}{1.805269in}}%
\pgfpathlineto{\pgfqpoint{7.906245in}{1.695472in}}%
\pgfpathlineto{\pgfqpoint{7.915052in}{1.551353in}}%
\pgfpathlineto{\pgfqpoint{7.923859in}{1.331760in}}%
\pgfpathlineto{\pgfqpoint{7.932666in}{1.640574in}}%
\pgfpathlineto{\pgfqpoint{7.941472in}{1.139601in}}%
\pgfpathlineto{\pgfqpoint{7.959086in}{1.784692in}}%
\pgfpathlineto{\pgfqpoint{7.967893in}{1.915094in}}%
\pgfpathlineto{\pgfqpoint{7.976700in}{2.024891in}}%
\pgfpathlineto{\pgfqpoint{7.985507in}{1.853308in}}%
\pgfpathlineto{\pgfqpoint{7.994313in}{2.368027in}}%
\pgfpathlineto{\pgfqpoint{8.003120in}{2.457247in}}%
\pgfpathlineto{\pgfqpoint{8.011927in}{2.127829in}}%
\pgfpathlineto{\pgfqpoint{8.020734in}{2.196445in}}%
\pgfpathlineto{\pgfqpoint{8.029541in}{2.203303in}}%
\pgfpathlineto{\pgfqpoint{8.038347in}{2.127829in}}%
\pgfpathlineto{\pgfqpoint{8.055961in}{2.429784in}}%
\pgfpathlineto{\pgfqpoint{8.064768in}{2.477823in}}%
\pgfpathlineto{\pgfqpoint{8.073575in}{2.491541in}}%
\pgfpathlineto{\pgfqpoint{8.082382in}{2.608225in}}%
\pgfpathlineto{\pgfqpoint{8.091188in}{2.546468in}}%
\pgfpathlineto{\pgfqpoint{8.099995in}{2.738598in}}%
\pgfpathlineto{\pgfqpoint{8.108802in}{2.711163in}}%
\pgfpathlineto{\pgfqpoint{8.117609in}{2.299383in}}%
\pgfpathlineto{\pgfqpoint{8.126416in}{2.141546in}}%
\pgfpathlineto{\pgfqpoint{8.135222in}{2.079789in}}%
\pgfpathlineto{\pgfqpoint{8.144029in}{2.155292in}}%
\pgfpathlineto{\pgfqpoint{8.152836in}{1.942529in}}%
\pgfpathlineto{\pgfqpoint{8.161643in}{1.853308in}}%
\pgfpathlineto{\pgfqpoint{8.170450in}{1.901348in}}%
\pgfpathlineto{\pgfqpoint{8.179257in}{1.915094in}}%
\pgfpathlineto{\pgfqpoint{8.188063in}{1.860167in}}%
\pgfpathlineto{\pgfqpoint{8.196870in}{1.784692in}}%
\pgfpathlineto{\pgfqpoint{8.205677in}{1.448415in}}%
\pgfpathlineto{\pgfqpoint{8.214484in}{1.647432in}}%
\pgfpathlineto{\pgfqpoint{8.223291in}{2.052326in}}%
\pgfpathlineto{\pgfqpoint{8.232097in}{2.292524in}}%
\pgfpathlineto{\pgfqpoint{8.240904in}{2.203303in}}%
\pgfpathlineto{\pgfqpoint{8.249711in}{2.477823in}}%
\pgfpathlineto{\pgfqpoint{8.258518in}{2.340563in}}%
\pgfpathlineto{\pgfqpoint{8.267325in}{2.059213in}}%
\pgfpathlineto{\pgfqpoint{8.276132in}{2.066071in}}%
\pgfpathlineto{\pgfqpoint{8.284938in}{2.210190in}}%
\pgfpathlineto{\pgfqpoint{8.293745in}{2.038608in}}%
\pgfpathlineto{\pgfqpoint{8.302552in}{2.196445in}}%
\pgfpathlineto{\pgfqpoint{8.311359in}{2.141546in}}%
\pgfpathlineto{\pgfqpoint{8.320166in}{1.825873in}}%
\pgfpathlineto{\pgfqpoint{8.328972in}{1.887630in}}%
\pgfpathlineto{\pgfqpoint{8.337779in}{1.819015in}}%
\pgfpathlineto{\pgfqpoint{8.346586in}{1.942529in}}%
\pgfpathlineto{\pgfqpoint{8.355393in}{2.381744in}}%
\pgfpathlineto{\pgfqpoint{8.364200in}{2.333705in}}%
\pgfpathlineto{\pgfqpoint{8.373007in}{2.265089in}}%
\pgfpathlineto{\pgfqpoint{8.381813in}{2.107252in}}%
\pgfpathlineto{\pgfqpoint{8.390620in}{2.093507in}}%
\pgfpathlineto{\pgfqpoint{8.399427in}{2.018032in}}%
\pgfpathlineto{\pgfqpoint{8.408234in}{1.976851in}}%
\pgfpathlineto{\pgfqpoint{8.417041in}{1.921953in}}%
\pgfpathlineto{\pgfqpoint{8.425847in}{1.969992in}}%
\pgfpathlineto{\pgfqpoint{8.434654in}{1.949388in}}%
\pgfpathlineto{\pgfqpoint{8.443461in}{1.894489in}}%
\pgfpathlineto{\pgfqpoint{8.452268in}{1.825873in}}%
\pgfpathlineto{\pgfqpoint{8.461075in}{1.468992in}}%
\pgfpathlineto{\pgfqpoint{8.469882in}{1.709190in}}%
\pgfpathlineto{\pgfqpoint{8.478688in}{2.107252in}}%
\pgfpathlineto{\pgfqpoint{8.487495in}{2.066071in}}%
\pgfpathlineto{\pgfqpoint{8.496302in}{2.196445in}}%
\pgfpathlineto{\pgfqpoint{8.505109in}{2.223908in}}%
\pgfpathlineto{\pgfqpoint{8.513916in}{2.148405in}}%
\pgfpathlineto{\pgfqpoint{8.522722in}{2.114111in}}%
\pgfpathlineto{\pgfqpoint{8.531529in}{2.210190in}}%
\pgfpathlineto{\pgfqpoint{8.540336in}{1.908207in}}%
\pgfpathlineto{\pgfqpoint{8.549143in}{2.265089in}}%
\pgfpathlineto{\pgfqpoint{8.557950in}{2.059213in}}%
\pgfpathlineto{\pgfqpoint{8.566757in}{1.626856in}}%
\pgfpathlineto{\pgfqpoint{8.575563in}{1.558212in}}%
\pgfpathlineto{\pgfqpoint{8.584370in}{1.228822in}}%
\pgfpathlineto{\pgfqpoint{8.593177in}{1.695472in}}%
\pgfpathlineto{\pgfqpoint{8.601984in}{1.482737in}}%
\pgfpathlineto{\pgfqpoint{8.610791in}{1.482737in}}%
\pgfpathlineto{\pgfqpoint{8.619597in}{1.722935in}}%
\pgfpathlineto{\pgfqpoint{8.628404in}{1.750370in}}%
\pgfpathlineto{\pgfqpoint{8.637211in}{2.258230in}}%
\pgfpathlineto{\pgfqpoint{8.646018in}{2.162151in}}%
\pgfpathlineto{\pgfqpoint{8.654825in}{2.107252in}}%
\pgfpathlineto{\pgfqpoint{8.663632in}{2.333705in}}%
\pgfpathlineto{\pgfqpoint{8.672438in}{2.059213in}}%
\pgfpathlineto{\pgfqpoint{8.681245in}{2.100365in}}%
\pgfpathlineto{\pgfqpoint{8.690052in}{2.011173in}}%
\pgfpathlineto{\pgfqpoint{8.698859in}{1.832732in}}%
\pgfpathlineto{\pgfqpoint{8.707666in}{1.606252in}}%
\pgfpathlineto{\pgfqpoint{8.716472in}{1.537636in}}%
\pgfpathlineto{\pgfqpoint{8.725279in}{1.832732in}}%
\pgfpathlineto{\pgfqpoint{8.734086in}{1.956246in}}%
\pgfpathlineto{\pgfqpoint{8.742893in}{1.626856in}}%
\pgfpathlineto{\pgfqpoint{8.751700in}{1.695472in}}%
\pgfpathlineto{\pgfqpoint{8.769313in}{1.963133in}}%
\pgfpathlineto{\pgfqpoint{8.778120in}{1.983710in}}%
\pgfpathlineto{\pgfqpoint{8.786927in}{2.196445in}}%
\pgfpathlineto{\pgfqpoint{8.795734in}{1.935670in}}%
\pgfpathlineto{\pgfqpoint{8.804541in}{1.407234in}}%
\pgfpathlineto{\pgfqpoint{8.813347in}{1.269974in}}%
\pgfpathlineto{\pgfqpoint{8.822154in}{1.455274in}}%
\pgfpathlineto{\pgfqpoint{8.830961in}{1.709190in}}%
\pgfpathlineto{\pgfqpoint{8.839768in}{1.407234in}}%
\pgfpathlineto{\pgfqpoint{8.848575in}{1.839591in}}%
\pgfpathlineto{\pgfqpoint{8.857382in}{1.887630in}}%
\pgfpathlineto{\pgfqpoint{8.866188in}{1.860167in}}%
\pgfpathlineto{\pgfqpoint{8.874995in}{1.613110in}}%
\pgfpathlineto{\pgfqpoint{8.883802in}{1.812128in}}%
\pgfpathlineto{\pgfqpoint{8.892609in}{1.983710in}}%
\pgfpathlineto{\pgfqpoint{8.901416in}{1.853308in}}%
\pgfpathlineto{\pgfqpoint{8.910222in}{1.619997in}}%
\pgfpathlineto{\pgfqpoint{8.919029in}{1.530777in}}%
\pgfpathlineto{\pgfqpoint{8.927836in}{1.633715in}}%
\pgfpathlineto{\pgfqpoint{8.936643in}{1.331760in}}%
\pgfpathlineto{\pgfqpoint{8.945450in}{1.681754in}}%
\pgfpathlineto{\pgfqpoint{8.954257in}{1.901348in}}%
\pgfpathlineto{\pgfqpoint{8.963063in}{2.024891in}}%
\pgfpathlineto{\pgfqpoint{8.971870in}{1.633715in}}%
\pgfpathlineto{\pgfqpoint{8.980677in}{1.462133in}}%
\pgfpathlineto{\pgfqpoint{8.989484in}{1.702331in}}%
\pgfpathlineto{\pgfqpoint{8.998291in}{1.825873in}}%
\pgfpathlineto{\pgfqpoint{9.007097in}{1.983710in}}%
\pgfpathlineto{\pgfqpoint{9.015904in}{1.949388in}}%
\pgfpathlineto{\pgfqpoint{9.024711in}{2.066071in}}%
\pgfpathlineto{\pgfqpoint{9.033518in}{1.757229in}}%
\pgfpathlineto{\pgfqpoint{9.042325in}{1.400376in}}%
\pgfpathlineto{\pgfqpoint{9.051132in}{1.462133in}}%
\pgfpathlineto{\pgfqpoint{9.059938in}{1.585675in}}%
\pgfpathlineto{\pgfqpoint{9.068745in}{1.757229in}}%
\pgfpathlineto{\pgfqpoint{9.077552in}{1.880772in}}%
\pgfpathlineto{\pgfqpoint{9.086359in}{2.079789in}}%
\pgfpathlineto{\pgfqpoint{9.095166in}{2.011173in}}%
\pgfpathlineto{\pgfqpoint{9.103972in}{1.839591in}}%
\pgfpathlineto{\pgfqpoint{9.112779in}{1.915094in}}%
\pgfpathlineto{\pgfqpoint{9.121586in}{2.285665in}}%
\pgfpathlineto{\pgfqpoint{9.130393in}{2.333705in}}%
\pgfpathlineto{\pgfqpoint{9.139200in}{2.107252in}}%
\pgfpathlineto{\pgfqpoint{9.148007in}{1.990569in}}%
\pgfpathlineto{\pgfqpoint{9.156813in}{2.024891in}}%
\pgfpathlineto{\pgfqpoint{9.165620in}{1.722935in}}%
\pgfpathlineto{\pgfqpoint{9.174427in}{1.578816in}}%
\pgfpathlineto{\pgfqpoint{9.183234in}{1.921953in}}%
\pgfpathlineto{\pgfqpoint{9.200847in}{1.434698in}}%
\pgfpathlineto{\pgfqpoint{9.209654in}{1.551353in}}%
\pgfpathlineto{\pgfqpoint{9.218461in}{1.688613in}}%
\pgfpathlineto{\pgfqpoint{9.227268in}{2.011173in}}%
\pgfpathlineto{\pgfqpoint{9.236075in}{1.702331in}}%
\pgfpathlineto{\pgfqpoint{9.244882in}{2.018032in}}%
\pgfpathlineto{\pgfqpoint{9.253688in}{2.148405in}}%
\pgfpathlineto{\pgfqpoint{9.262495in}{1.983710in}}%
\pgfpathlineto{\pgfqpoint{9.271302in}{2.457247in}}%
\pgfpathlineto{\pgfqpoint{9.280109in}{2.381744in}}%
\pgfpathlineto{\pgfqpoint{9.288916in}{2.011173in}}%
\pgfpathlineto{\pgfqpoint{9.297722in}{1.716076in}}%
\pgfpathlineto{\pgfqpoint{9.306529in}{1.599393in}}%
\pgfpathlineto{\pgfqpoint{9.315336in}{2.265089in}}%
\pgfpathlineto{\pgfqpoint{9.324143in}{1.949388in}}%
\pgfpathlineto{\pgfqpoint{9.332950in}{1.860167in}}%
\pgfpathlineto{\pgfqpoint{9.341757in}{2.203303in}}%
\pgfpathlineto{\pgfqpoint{9.350563in}{2.477823in}}%
\pgfpathlineto{\pgfqpoint{9.359370in}{2.313128in}}%
\pgfpathlineto{\pgfqpoint{9.368177in}{2.107252in}}%
\pgfpathlineto{\pgfqpoint{9.376984in}{1.668037in}}%
\pgfpathlineto{\pgfqpoint{9.385791in}{2.086648in}}%
\pgfpathlineto{\pgfqpoint{9.394597in}{1.956246in}}%
\pgfpathlineto{\pgfqpoint{9.403404in}{1.915094in}}%
\pgfpathlineto{\pgfqpoint{9.412211in}{1.702331in}}%
\pgfpathlineto{\pgfqpoint{9.421018in}{2.189586in}}%
\pgfpathlineto{\pgfqpoint{9.429825in}{2.374885in}}%
\pgfpathlineto{\pgfqpoint{9.438632in}{2.292524in}}%
\pgfpathlineto{\pgfqpoint{9.447438in}{2.230767in}}%
\pgfpathlineto{\pgfqpoint{9.456245in}{2.038608in}}%
\pgfpathlineto{\pgfqpoint{9.465052in}{2.217049in}}%
\pgfpathlineto{\pgfqpoint{9.473859in}{2.450388in}}%
\pgfpathlineto{\pgfqpoint{9.482666in}{2.546468in}}%
\pgfpathlineto{\pgfqpoint{9.491472in}{2.148405in}}%
\pgfpathlineto{\pgfqpoint{9.500279in}{2.100365in}}%
\pgfpathlineto{\pgfqpoint{9.509086in}{2.134687in}}%
\pgfpathlineto{\pgfqpoint{9.517893in}{1.750370in}}%
\pgfpathlineto{\pgfqpoint{9.526700in}{1.873913in}}%
\pgfpathlineto{\pgfqpoint{9.535507in}{2.120970in}}%
\pgfpathlineto{\pgfqpoint{9.544313in}{1.901348in}}%
\pgfpathlineto{\pgfqpoint{9.553120in}{1.949388in}}%
\pgfpathlineto{\pgfqpoint{9.561927in}{1.819015in}}%
\pgfpathlineto{\pgfqpoint{9.570734in}{2.114111in}}%
\pgfpathlineto{\pgfqpoint{9.579541in}{2.086648in}}%
\pgfpathlineto{\pgfqpoint{9.588347in}{1.867054in}}%
\pgfpathlineto{\pgfqpoint{9.597154in}{1.798410in}}%
\pgfpathlineto{\pgfqpoint{9.605961in}{1.702331in}}%
\pgfpathlineto{\pgfqpoint{9.614768in}{1.956246in}}%
\pgfpathlineto{\pgfqpoint{9.623575in}{1.867054in}}%
\pgfpathlineto{\pgfqpoint{9.632382in}{1.880772in}}%
\pgfpathlineto{\pgfqpoint{9.641188in}{1.819015in}}%
\pgfpathlineto{\pgfqpoint{9.649995in}{1.963133in}}%
\pgfpathlineto{\pgfqpoint{9.658802in}{1.709190in}}%
\pgfpathlineto{\pgfqpoint{9.667609in}{1.743512in}}%
\pgfpathlineto{\pgfqpoint{9.676416in}{1.867054in}}%
\pgfpathlineto{\pgfqpoint{9.685222in}{1.777834in}}%
\pgfpathlineto{\pgfqpoint{9.694029in}{1.565071in}}%
\pgfpathlineto{\pgfqpoint{9.702836in}{1.750370in}}%
\pgfpathlineto{\pgfqpoint{9.711643in}{1.558212in}}%
\pgfpathlineto{\pgfqpoint{9.720450in}{1.674896in}}%
\pgfpathlineto{\pgfqpoint{9.729257in}{1.880772in}}%
\pgfpathlineto{\pgfqpoint{9.738063in}{1.805269in}}%
\pgfpathlineto{\pgfqpoint{9.746870in}{1.846450in}}%
\pgfpathlineto{\pgfqpoint{9.755677in}{1.812128in}}%
\pgfpathlineto{\pgfqpoint{9.764484in}{1.887630in}}%
\pgfpathlineto{\pgfqpoint{9.773291in}{1.949388in}}%
\pgfpathlineto{\pgfqpoint{9.782097in}{2.175868in}}%
\pgfpathlineto{\pgfqpoint{9.790904in}{2.004286in}}%
\pgfpathlineto{\pgfqpoint{9.799711in}{1.880772in}}%
\pgfpathlineto{\pgfqpoint{9.808518in}{2.141546in}}%
\pgfpathlineto{\pgfqpoint{9.817325in}{1.832732in}}%
\pgfpathlineto{\pgfqpoint{9.826132in}{2.072930in}}%
\pgfpathlineto{\pgfqpoint{9.834938in}{2.038608in}}%
\pgfpathlineto{\pgfqpoint{9.843745in}{1.784692in}}%
\pgfpathlineto{\pgfqpoint{9.852552in}{1.894489in}}%
\pgfpathlineto{\pgfqpoint{9.861359in}{1.956246in}}%
\pgfpathlineto{\pgfqpoint{9.870166in}{1.997427in}}%
\pgfpathlineto{\pgfqpoint{9.878972in}{2.066071in}}%
\pgfpathlineto{\pgfqpoint{9.887779in}{2.162151in}}%
\pgfpathlineto{\pgfqpoint{9.896586in}{2.402349in}}%
\pgfpathlineto{\pgfqpoint{9.905393in}{2.155292in}}%
\pgfpathlineto{\pgfqpoint{9.914200in}{1.942529in}}%
\pgfpathlineto{\pgfqpoint{9.923007in}{1.867054in}}%
\pgfpathlineto{\pgfqpoint{9.931813in}{1.839591in}}%
\pgfpathlineto{\pgfqpoint{9.940620in}{1.455274in}}%
\pgfpathlineto{\pgfqpoint{9.949427in}{1.647432in}}%
\pgfpathlineto{\pgfqpoint{9.949427in}{1.647432in}}%
\pgfusepath{stroke}%
\end{pgfscope}%
\begin{pgfscope}%
\pgfpathrectangle{\pgfqpoint{0.702268in}{0.521603in}}{\pgfqpoint{9.687500in}{4.235000in}}%
\pgfusepath{clip}%
\pgfsetrectcap%
\pgfsetroundjoin%
\pgfsetlinewidth{0.501875pt}%
\definecolor{currentstroke}{rgb}{0.501961,0.501961,0.501961}%
\pgfsetstrokecolor{currentstroke}%
\pgfsetstrokeopacity{0.250000}%
\pgfsetdash{}{0pt}%
\pgfpathmoveto{\pgfqpoint{1.142609in}{4.200363in}}%
\pgfpathlineto{\pgfqpoint{1.151416in}{3.390576in}}%
\pgfpathlineto{\pgfqpoint{1.177836in}{2.031749in}}%
\pgfpathlineto{\pgfqpoint{1.186643in}{1.956246in}}%
\pgfpathlineto{\pgfqpoint{1.195450in}{2.066071in}}%
\pgfpathlineto{\pgfqpoint{1.204257in}{2.120970in}}%
\pgfpathlineto{\pgfqpoint{1.213063in}{2.148405in}}%
\pgfpathlineto{\pgfqpoint{1.221870in}{2.615084in}}%
\pgfpathlineto{\pgfqpoint{1.230677in}{2.738598in}}%
\pgfpathlineto{\pgfqpoint{1.239484in}{2.429784in}}%
\pgfpathlineto{\pgfqpoint{1.248291in}{2.374885in}}%
\pgfpathlineto{\pgfqpoint{1.257097in}{2.230767in}}%
\pgfpathlineto{\pgfqpoint{1.265904in}{2.182727in}}%
\pgfpathlineto{\pgfqpoint{1.274711in}{1.544494in}}%
\pgfpathlineto{\pgfqpoint{1.283518in}{1.674896in}}%
\pgfpathlineto{\pgfqpoint{1.292325in}{1.853308in}}%
\pgfpathlineto{\pgfqpoint{1.301132in}{1.695472in}}%
\pgfpathlineto{\pgfqpoint{1.309938in}{1.695472in}}%
\pgfpathlineto{\pgfqpoint{1.318745in}{1.894489in}}%
\pgfpathlineto{\pgfqpoint{1.327552in}{2.038608in}}%
\pgfpathlineto{\pgfqpoint{1.336359in}{1.647432in}}%
\pgfpathlineto{\pgfqpoint{1.345166in}{1.132742in}}%
\pgfpathlineto{\pgfqpoint{1.353972in}{1.263115in}}%
\pgfpathlineto{\pgfqpoint{1.362779in}{1.242539in}}%
\pgfpathlineto{\pgfqpoint{1.371586in}{1.441556in}}%
\pgfpathlineto{\pgfqpoint{1.380393in}{1.777834in}}%
\pgfpathlineto{\pgfqpoint{1.389200in}{1.606252in}}%
\pgfpathlineto{\pgfqpoint{1.398007in}{2.024891in}}%
\pgfpathlineto{\pgfqpoint{1.406813in}{2.038608in}}%
\pgfpathlineto{\pgfqpoint{1.415620in}{1.716076in}}%
\pgfpathlineto{\pgfqpoint{1.424427in}{1.537636in}}%
\pgfpathlineto{\pgfqpoint{1.433234in}{1.798410in}}%
\pgfpathlineto{\pgfqpoint{1.442041in}{1.510172in}}%
\pgfpathlineto{\pgfqpoint{1.459654in}{2.011173in}}%
\pgfpathlineto{\pgfqpoint{1.468461in}{2.319987in}}%
\pgfpathlineto{\pgfqpoint{1.477268in}{2.313128in}}%
\pgfpathlineto{\pgfqpoint{1.486075in}{2.512146in}}%
\pgfpathlineto{\pgfqpoint{1.494882in}{1.949388in}}%
\pgfpathlineto{\pgfqpoint{1.503688in}{1.832732in}}%
\pgfpathlineto{\pgfqpoint{1.512495in}{1.517031in}}%
\pgfpathlineto{\pgfqpoint{1.521302in}{2.045467in}}%
\pgfpathlineto{\pgfqpoint{1.530109in}{2.024891in}}%
\pgfpathlineto{\pgfqpoint{1.538916in}{1.805269in}}%
\pgfpathlineto{\pgfqpoint{1.547722in}{1.688613in}}%
\pgfpathlineto{\pgfqpoint{1.556529in}{1.784692in}}%
\pgfpathlineto{\pgfqpoint{1.565336in}{2.031749in}}%
\pgfpathlineto{\pgfqpoint{1.574143in}{2.326846in}}%
\pgfpathlineto{\pgfqpoint{1.582950in}{2.457247in}}%
\pgfpathlineto{\pgfqpoint{1.591757in}{2.059213in}}%
\pgfpathlineto{\pgfqpoint{1.600563in}{1.956246in}}%
\pgfpathlineto{\pgfqpoint{1.609370in}{1.908207in}}%
\pgfpathlineto{\pgfqpoint{1.618177in}{1.812128in}}%
\pgfpathlineto{\pgfqpoint{1.626984in}{2.127829in}}%
\pgfpathlineto{\pgfqpoint{1.644597in}{1.825873in}}%
\pgfpathlineto{\pgfqpoint{1.653404in}{1.736653in}}%
\pgfpathlineto{\pgfqpoint{1.662211in}{2.175868in}}%
\pgfpathlineto{\pgfqpoint{1.671018in}{2.155292in}}%
\pgfpathlineto{\pgfqpoint{1.679825in}{2.217049in}}%
\pgfpathlineto{\pgfqpoint{1.688632in}{2.162151in}}%
\pgfpathlineto{\pgfqpoint{1.697438in}{1.901348in}}%
\pgfpathlineto{\pgfqpoint{1.706245in}{2.217049in}}%
\pgfpathlineto{\pgfqpoint{1.715052in}{2.004286in}}%
\pgfpathlineto{\pgfqpoint{1.723859in}{2.271947in}}%
\pgfpathlineto{\pgfqpoint{1.732666in}{2.464106in}}%
\pgfpathlineto{\pgfqpoint{1.741472in}{2.278806in}}%
\pgfpathlineto{\pgfqpoint{1.750279in}{2.573903in}}%
\pgfpathlineto{\pgfqpoint{1.759086in}{2.436643in}}%
\pgfpathlineto{\pgfqpoint{1.767893in}{2.388603in}}%
\pgfpathlineto{\pgfqpoint{1.776700in}{2.292524in}}%
\pgfpathlineto{\pgfqpoint{1.785507in}{1.887630in}}%
\pgfpathlineto{\pgfqpoint{1.794313in}{1.839591in}}%
\pgfpathlineto{\pgfqpoint{1.803120in}{1.963133in}}%
\pgfpathlineto{\pgfqpoint{1.811927in}{2.024891in}}%
\pgfpathlineto{\pgfqpoint{1.820734in}{2.059213in}}%
\pgfpathlineto{\pgfqpoint{1.829541in}{1.873913in}}%
\pgfpathlineto{\pgfqpoint{1.838347in}{1.825873in}}%
\pgfpathlineto{\pgfqpoint{1.847154in}{1.695472in}}%
\pgfpathlineto{\pgfqpoint{1.855961in}{1.702331in}}%
\pgfpathlineto{\pgfqpoint{1.864768in}{1.921953in}}%
\pgfpathlineto{\pgfqpoint{1.873575in}{2.072930in}}%
\pgfpathlineto{\pgfqpoint{1.882382in}{1.825873in}}%
\pgfpathlineto{\pgfqpoint{1.891188in}{1.681754in}}%
\pgfpathlineto{\pgfqpoint{1.899995in}{1.839591in}}%
\pgfpathlineto{\pgfqpoint{1.908802in}{1.825873in}}%
\pgfpathlineto{\pgfqpoint{1.917609in}{1.716076in}}%
\pgfpathlineto{\pgfqpoint{1.926416in}{1.674896in}}%
\pgfpathlineto{\pgfqpoint{1.935222in}{1.935670in}}%
\pgfpathlineto{\pgfqpoint{1.944029in}{2.114111in}}%
\pgfpathlineto{\pgfqpoint{1.952836in}{2.374885in}}%
\pgfpathlineto{\pgfqpoint{1.961643in}{2.169009in}}%
\pgfpathlineto{\pgfqpoint{1.970450in}{2.189586in}}%
\pgfpathlineto{\pgfqpoint{1.979257in}{2.388603in}}%
\pgfpathlineto{\pgfqpoint{1.988063in}{2.340563in}}%
\pgfpathlineto{\pgfqpoint{1.996870in}{2.395462in}}%
\pgfpathlineto{\pgfqpoint{2.014484in}{2.018032in}}%
\pgfpathlineto{\pgfqpoint{2.023291in}{2.107252in}}%
\pgfpathlineto{\pgfqpoint{2.032097in}{1.770975in}}%
\pgfpathlineto{\pgfqpoint{2.040904in}{1.722935in}}%
\pgfpathlineto{\pgfqpoint{2.049711in}{1.990569in}}%
\pgfpathlineto{\pgfqpoint{2.058518in}{1.750370in}}%
\pgfpathlineto{\pgfqpoint{2.067325in}{1.935670in}}%
\pgfpathlineto{\pgfqpoint{2.076132in}{1.819015in}}%
\pgfpathlineto{\pgfqpoint{2.084938in}{1.530777in}}%
\pgfpathlineto{\pgfqpoint{2.093745in}{1.709190in}}%
\pgfpathlineto{\pgfqpoint{2.102552in}{2.093507in}}%
\pgfpathlineto{\pgfqpoint{2.111359in}{2.134687in}}%
\pgfpathlineto{\pgfqpoint{2.120166in}{2.052326in}}%
\pgfpathlineto{\pgfqpoint{2.128972in}{1.942529in}}%
\pgfpathlineto{\pgfqpoint{2.137779in}{1.908207in}}%
\pgfpathlineto{\pgfqpoint{2.146586in}{1.983710in}}%
\pgfpathlineto{\pgfqpoint{2.155393in}{1.894489in}}%
\pgfpathlineto{\pgfqpoint{2.164200in}{2.024891in}}%
\pgfpathlineto{\pgfqpoint{2.173007in}{1.825873in}}%
\pgfpathlineto{\pgfqpoint{2.181813in}{1.757229in}}%
\pgfpathlineto{\pgfqpoint{2.190620in}{1.969992in}}%
\pgfpathlineto{\pgfqpoint{2.199427in}{1.963133in}}%
\pgfpathlineto{\pgfqpoint{2.208234in}{2.313128in}}%
\pgfpathlineto{\pgfqpoint{2.217041in}{2.786638in}}%
\pgfpathlineto{\pgfqpoint{2.225847in}{2.354309in}}%
\pgfpathlineto{\pgfqpoint{2.234654in}{2.374885in}}%
\pgfpathlineto{\pgfqpoint{2.243461in}{2.217049in}}%
\pgfpathlineto{\pgfqpoint{2.252268in}{1.921953in}}%
\pgfpathlineto{\pgfqpoint{2.261075in}{2.114111in}}%
\pgfpathlineto{\pgfqpoint{2.269882in}{1.997427in}}%
\pgfpathlineto{\pgfqpoint{2.278688in}{2.011173in}}%
\pgfpathlineto{\pgfqpoint{2.287495in}{2.004286in}}%
\pgfpathlineto{\pgfqpoint{2.296302in}{2.100365in}}%
\pgfpathlineto{\pgfqpoint{2.305109in}{1.949388in}}%
\pgfpathlineto{\pgfqpoint{2.313916in}{2.038608in}}%
\pgfpathlineto{\pgfqpoint{2.322722in}{2.100365in}}%
\pgfpathlineto{\pgfqpoint{2.331529in}{1.976851in}}%
\pgfpathlineto{\pgfqpoint{2.340336in}{2.011173in}}%
\pgfpathlineto{\pgfqpoint{2.349143in}{1.887630in}}%
\pgfpathlineto{\pgfqpoint{2.357950in}{1.901348in}}%
\pgfpathlineto{\pgfqpoint{2.366757in}{2.189586in}}%
\pgfpathlineto{\pgfqpoint{2.375563in}{2.024891in}}%
\pgfpathlineto{\pgfqpoint{2.384370in}{2.072930in}}%
\pgfpathlineto{\pgfqpoint{2.393177in}{1.757229in}}%
\pgfpathlineto{\pgfqpoint{2.401984in}{2.004286in}}%
\pgfpathlineto{\pgfqpoint{2.410791in}{1.901348in}}%
\pgfpathlineto{\pgfqpoint{2.419597in}{2.182727in}}%
\pgfpathlineto{\pgfqpoint{2.428404in}{2.024891in}}%
\pgfpathlineto{\pgfqpoint{2.437211in}{1.976851in}}%
\pgfpathlineto{\pgfqpoint{2.446018in}{2.100365in}}%
\pgfpathlineto{\pgfqpoint{2.463632in}{1.674896in}}%
\pgfpathlineto{\pgfqpoint{2.472438in}{1.668037in}}%
\pgfpathlineto{\pgfqpoint{2.481245in}{1.640574in}}%
\pgfpathlineto{\pgfqpoint{2.490052in}{1.722935in}}%
\pgfpathlineto{\pgfqpoint{2.498859in}{1.784692in}}%
\pgfpathlineto{\pgfqpoint{2.507666in}{1.661150in}}%
\pgfpathlineto{\pgfqpoint{2.516472in}{1.482737in}}%
\pgfpathlineto{\pgfqpoint{2.542893in}{2.052326in}}%
\pgfpathlineto{\pgfqpoint{2.551700in}{1.894489in}}%
\pgfpathlineto{\pgfqpoint{2.560507in}{1.894489in}}%
\pgfpathlineto{\pgfqpoint{2.569313in}{1.956246in}}%
\pgfpathlineto{\pgfqpoint{2.578120in}{2.251343in}}%
\pgfpathlineto{\pgfqpoint{2.586927in}{1.558212in}}%
\pgfpathlineto{\pgfqpoint{2.604541in}{1.146460in}}%
\pgfpathlineto{\pgfqpoint{2.613347in}{1.187641in}}%
\pgfpathlineto{\pgfqpoint{2.622154in}{1.256257in}}%
\pgfpathlineto{\pgfqpoint{2.630961in}{0.933697in}}%
\pgfpathlineto{\pgfqpoint{2.639768in}{1.064098in}}%
\pgfpathlineto{\pgfqpoint{2.648575in}{1.496455in}}%
\pgfpathlineto{\pgfqpoint{2.657382in}{1.873913in}}%
\pgfpathlineto{\pgfqpoint{2.666188in}{1.963133in}}%
\pgfpathlineto{\pgfqpoint{2.674995in}{2.313128in}}%
\pgfpathlineto{\pgfqpoint{2.683802in}{2.265089in}}%
\pgfpathlineto{\pgfqpoint{2.692609in}{2.271947in}}%
\pgfpathlineto{\pgfqpoint{2.710222in}{1.956246in}}%
\pgfpathlineto{\pgfqpoint{2.719029in}{2.079789in}}%
\pgfpathlineto{\pgfqpoint{2.727836in}{2.114111in}}%
\pgfpathlineto{\pgfqpoint{2.736643in}{2.443501in}}%
\pgfpathlineto{\pgfqpoint{2.745450in}{1.777834in}}%
\pgfpathlineto{\pgfqpoint{2.754257in}{1.915094in}}%
\pgfpathlineto{\pgfqpoint{2.763063in}{1.674896in}}%
\pgfpathlineto{\pgfqpoint{2.771870in}{1.935670in}}%
\pgfpathlineto{\pgfqpoint{2.780677in}{1.709190in}}%
\pgfpathlineto{\pgfqpoint{2.789484in}{1.640574in}}%
\pgfpathlineto{\pgfqpoint{2.798291in}{1.661150in}}%
\pgfpathlineto{\pgfqpoint{2.807097in}{1.496455in}}%
\pgfpathlineto{\pgfqpoint{2.815904in}{1.510172in}}%
\pgfpathlineto{\pgfqpoint{2.824711in}{1.414093in}}%
\pgfpathlineto{\pgfqpoint{2.833518in}{0.988623in}}%
\pgfpathlineto{\pgfqpoint{2.851132in}{1.530777in}}%
\pgfpathlineto{\pgfqpoint{2.859938in}{1.599393in}}%
\pgfpathlineto{\pgfqpoint{2.868745in}{1.503314in}}%
\pgfpathlineto{\pgfqpoint{2.877552in}{1.517031in}}%
\pgfpathlineto{\pgfqpoint{2.886359in}{1.606252in}}%
\pgfpathlineto{\pgfqpoint{2.895166in}{1.661150in}}%
\pgfpathlineto{\pgfqpoint{2.903972in}{1.770975in}}%
\pgfpathlineto{\pgfqpoint{2.912779in}{2.024891in}}%
\pgfpathlineto{\pgfqpoint{2.921586in}{1.853308in}}%
\pgfpathlineto{\pgfqpoint{2.930393in}{2.045467in}}%
\pgfpathlineto{\pgfqpoint{2.939200in}{2.107252in}}%
\pgfpathlineto{\pgfqpoint{2.948007in}{1.928811in}}%
\pgfpathlineto{\pgfqpoint{2.956813in}{2.258230in}}%
\pgfpathlineto{\pgfqpoint{2.965620in}{2.093507in}}%
\pgfpathlineto{\pgfqpoint{2.974427in}{1.976851in}}%
\pgfpathlineto{\pgfqpoint{2.983234in}{1.764116in}}%
\pgfpathlineto{\pgfqpoint{2.992041in}{1.908207in}}%
\pgfpathlineto{\pgfqpoint{3.000847in}{1.702331in}}%
\pgfpathlineto{\pgfqpoint{3.009654in}{1.606252in}}%
\pgfpathlineto{\pgfqpoint{3.018461in}{1.633715in}}%
\pgfpathlineto{\pgfqpoint{3.027268in}{1.544494in}}%
\pgfpathlineto{\pgfqpoint{3.036075in}{1.633715in}}%
\pgfpathlineto{\pgfqpoint{3.044882in}{1.805269in}}%
\pgfpathlineto{\pgfqpoint{3.053688in}{2.018032in}}%
\pgfpathlineto{\pgfqpoint{3.062495in}{1.661150in}}%
\pgfpathlineto{\pgfqpoint{3.071302in}{1.887630in}}%
\pgfpathlineto{\pgfqpoint{3.080109in}{1.969992in}}%
\pgfpathlineto{\pgfqpoint{3.088916in}{1.626856in}}%
\pgfpathlineto{\pgfqpoint{3.097722in}{1.544494in}}%
\pgfpathlineto{\pgfqpoint{3.106529in}{1.887630in}}%
\pgfpathlineto{\pgfqpoint{3.115336in}{2.402349in}}%
\pgfpathlineto{\pgfqpoint{3.124143in}{2.093507in}}%
\pgfpathlineto{\pgfqpoint{3.132950in}{2.038608in}}%
\pgfpathlineto{\pgfqpoint{3.141757in}{1.640574in}}%
\pgfpathlineto{\pgfqpoint{3.150563in}{1.400376in}}%
\pgfpathlineto{\pgfqpoint{3.168177in}{1.517031in}}%
\pgfpathlineto{\pgfqpoint{3.176984in}{1.386658in}}%
\pgfpathlineto{\pgfqpoint{3.185791in}{1.661150in}}%
\pgfpathlineto{\pgfqpoint{3.194597in}{2.107252in}}%
\pgfpathlineto{\pgfqpoint{3.203404in}{2.169009in}}%
\pgfpathlineto{\pgfqpoint{3.212211in}{2.155292in}}%
\pgfpathlineto{\pgfqpoint{3.221018in}{1.825873in}}%
\pgfpathlineto{\pgfqpoint{3.229825in}{2.024891in}}%
\pgfpathlineto{\pgfqpoint{3.238632in}{2.271947in}}%
\pgfpathlineto{\pgfqpoint{3.247438in}{2.148405in}}%
\pgfpathlineto{\pgfqpoint{3.256245in}{2.100365in}}%
\pgfpathlineto{\pgfqpoint{3.265052in}{2.114111in}}%
\pgfpathlineto{\pgfqpoint{3.273859in}{1.784692in}}%
\pgfpathlineto{\pgfqpoint{3.282666in}{1.983710in}}%
\pgfpathlineto{\pgfqpoint{3.291472in}{1.743512in}}%
\pgfpathlineto{\pgfqpoint{3.300279in}{1.976851in}}%
\pgfpathlineto{\pgfqpoint{3.317893in}{2.175868in}}%
\pgfpathlineto{\pgfqpoint{3.326700in}{2.237625in}}%
\pgfpathlineto{\pgfqpoint{3.335507in}{2.278806in}}%
\pgfpathlineto{\pgfqpoint{3.344313in}{2.203303in}}%
\pgfpathlineto{\pgfqpoint{3.353120in}{2.011173in}}%
\pgfpathlineto{\pgfqpoint{3.361927in}{1.983710in}}%
\pgfpathlineto{\pgfqpoint{3.370734in}{1.935670in}}%
\pgfpathlineto{\pgfqpoint{3.379541in}{1.976851in}}%
\pgfpathlineto{\pgfqpoint{3.388347in}{2.093507in}}%
\pgfpathlineto{\pgfqpoint{3.397154in}{2.443501in}}%
\pgfpathlineto{\pgfqpoint{3.405961in}{2.505287in}}%
\pgfpathlineto{\pgfqpoint{3.414768in}{2.319987in}}%
\pgfpathlineto{\pgfqpoint{3.423575in}{2.086648in}}%
\pgfpathlineto{\pgfqpoint{3.432382in}{2.210190in}}%
\pgfpathlineto{\pgfqpoint{3.441188in}{1.887630in}}%
\pgfpathlineto{\pgfqpoint{3.449995in}{2.093507in}}%
\pgfpathlineto{\pgfqpoint{3.458802in}{2.230767in}}%
\pgfpathlineto{\pgfqpoint{3.467609in}{2.217049in}}%
\pgfpathlineto{\pgfqpoint{3.476416in}{1.873913in}}%
\pgfpathlineto{\pgfqpoint{3.485222in}{1.942529in}}%
\pgfpathlineto{\pgfqpoint{3.494029in}{2.409207in}}%
\pgfpathlineto{\pgfqpoint{3.502836in}{2.573903in}}%
\pgfpathlineto{\pgfqpoint{3.511643in}{2.443501in}}%
\pgfpathlineto{\pgfqpoint{3.520450in}{2.519004in}}%
\pgfpathlineto{\pgfqpoint{3.529257in}{2.930756in}}%
\pgfpathlineto{\pgfqpoint{3.538063in}{2.573903in}}%
\pgfpathlineto{\pgfqpoint{3.546870in}{2.409207in}}%
\pgfpathlineto{\pgfqpoint{3.555677in}{2.326846in}}%
\pgfpathlineto{\pgfqpoint{3.564484in}{2.018032in}}%
\pgfpathlineto{\pgfqpoint{3.573291in}{2.059213in}}%
\pgfpathlineto{\pgfqpoint{3.582097in}{2.217049in}}%
\pgfpathlineto{\pgfqpoint{3.590904in}{1.976851in}}%
\pgfpathlineto{\pgfqpoint{3.599711in}{1.921953in}}%
\pgfpathlineto{\pgfqpoint{3.608518in}{1.935670in}}%
\pgfpathlineto{\pgfqpoint{3.617325in}{1.846450in}}%
\pgfpathlineto{\pgfqpoint{3.626132in}{1.791551in}}%
\pgfpathlineto{\pgfqpoint{3.634938in}{1.825873in}}%
\pgfpathlineto{\pgfqpoint{3.643745in}{1.791551in}}%
\pgfpathlineto{\pgfqpoint{3.652552in}{2.100365in}}%
\pgfpathlineto{\pgfqpoint{3.661359in}{2.059213in}}%
\pgfpathlineto{\pgfqpoint{3.670166in}{1.819015in}}%
\pgfpathlineto{\pgfqpoint{3.678972in}{1.846450in}}%
\pgfpathlineto{\pgfqpoint{3.687779in}{1.558212in}}%
\pgfpathlineto{\pgfqpoint{3.696586in}{1.921953in}}%
\pgfpathlineto{\pgfqpoint{3.705393in}{1.887630in}}%
\pgfpathlineto{\pgfqpoint{3.714200in}{1.722935in}}%
\pgfpathlineto{\pgfqpoint{3.723007in}{1.736653in}}%
\pgfpathlineto{\pgfqpoint{3.731813in}{1.770975in}}%
\pgfpathlineto{\pgfqpoint{3.740620in}{1.956246in}}%
\pgfpathlineto{\pgfqpoint{3.749427in}{1.722935in}}%
\pgfpathlineto{\pgfqpoint{3.758234in}{2.031749in}}%
\pgfpathlineto{\pgfqpoint{3.767041in}{1.935670in}}%
\pgfpathlineto{\pgfqpoint{3.775847in}{1.674896in}}%
\pgfpathlineto{\pgfqpoint{3.784654in}{1.777834in}}%
\pgfpathlineto{\pgfqpoint{3.793461in}{1.613110in}}%
\pgfpathlineto{\pgfqpoint{3.802268in}{1.565071in}}%
\pgfpathlineto{\pgfqpoint{3.811075in}{1.372912in}}%
\pgfpathlineto{\pgfqpoint{3.819882in}{1.359195in}}%
\pgfpathlineto{\pgfqpoint{3.828688in}{1.523918in}}%
\pgfpathlineto{\pgfqpoint{3.837495in}{1.496455in}}%
\pgfpathlineto{\pgfqpoint{3.846302in}{1.757229in}}%
\pgfpathlineto{\pgfqpoint{3.855109in}{1.695472in}}%
\pgfpathlineto{\pgfqpoint{3.863916in}{2.011173in}}%
\pgfpathlineto{\pgfqpoint{3.872722in}{1.757229in}}%
\pgfpathlineto{\pgfqpoint{3.881529in}{1.626856in}}%
\pgfpathlineto{\pgfqpoint{3.890336in}{1.729794in}}%
\pgfpathlineto{\pgfqpoint{3.899143in}{1.757229in}}%
\pgfpathlineto{\pgfqpoint{3.907950in}{2.086648in}}%
\pgfpathlineto{\pgfqpoint{3.916757in}{2.018032in}}%
\pgfpathlineto{\pgfqpoint{3.925563in}{2.326846in}}%
\pgfpathlineto{\pgfqpoint{3.934370in}{2.072930in}}%
\pgfpathlineto{\pgfqpoint{3.943177in}{2.409207in}}%
\pgfpathlineto{\pgfqpoint{3.951984in}{2.024891in}}%
\pgfpathlineto{\pgfqpoint{3.960791in}{2.066071in}}%
\pgfpathlineto{\pgfqpoint{3.969597in}{1.770975in}}%
\pgfpathlineto{\pgfqpoint{3.978404in}{1.530777in}}%
\pgfpathlineto{\pgfqpoint{3.987211in}{1.812128in}}%
\pgfpathlineto{\pgfqpoint{3.996018in}{1.928811in}}%
\pgfpathlineto{\pgfqpoint{4.004825in}{1.928811in}}%
\pgfpathlineto{\pgfqpoint{4.013632in}{2.011173in}}%
\pgfpathlineto{\pgfqpoint{4.022438in}{2.196445in}}%
\pgfpathlineto{\pgfqpoint{4.031245in}{2.141546in}}%
\pgfpathlineto{\pgfqpoint{4.040052in}{2.045467in}}%
\pgfpathlineto{\pgfqpoint{4.048859in}{2.004286in}}%
\pgfpathlineto{\pgfqpoint{4.057666in}{1.983710in}}%
\pgfpathlineto{\pgfqpoint{4.066472in}{1.846450in}}%
\pgfpathlineto{\pgfqpoint{4.075279in}{1.915094in}}%
\pgfpathlineto{\pgfqpoint{4.084086in}{1.798410in}}%
\pgfpathlineto{\pgfqpoint{4.092893in}{1.990569in}}%
\pgfpathlineto{\pgfqpoint{4.101700in}{1.839591in}}%
\pgfpathlineto{\pgfqpoint{4.110507in}{1.969992in}}%
\pgfpathlineto{\pgfqpoint{4.119313in}{2.230767in}}%
\pgfpathlineto{\pgfqpoint{4.128120in}{2.196445in}}%
\pgfpathlineto{\pgfqpoint{4.136927in}{2.182727in}}%
\pgfpathlineto{\pgfqpoint{4.145734in}{2.100365in}}%
\pgfpathlineto{\pgfqpoint{4.154541in}{2.422925in}}%
\pgfpathlineto{\pgfqpoint{4.163347in}{2.601366in}}%
\pgfpathlineto{\pgfqpoint{4.172154in}{2.546468in}}%
\pgfpathlineto{\pgfqpoint{4.180961in}{2.278806in}}%
\pgfpathlineto{\pgfqpoint{4.198575in}{2.738598in}}%
\pgfpathlineto{\pgfqpoint{4.207382in}{2.258230in}}%
\pgfpathlineto{\pgfqpoint{4.216188in}{2.477823in}}%
\pgfpathlineto{\pgfqpoint{4.224995in}{2.416066in}}%
\pgfpathlineto{\pgfqpoint{4.233802in}{2.306269in}}%
\pgfpathlineto{\pgfqpoint{4.242609in}{2.100365in}}%
\pgfpathlineto{\pgfqpoint{4.251416in}{2.052326in}}%
\pgfpathlineto{\pgfqpoint{4.260222in}{2.299383in}}%
\pgfpathlineto{\pgfqpoint{4.269029in}{2.210190in}}%
\pgfpathlineto{\pgfqpoint{4.277836in}{2.162151in}}%
\pgfpathlineto{\pgfqpoint{4.286643in}{1.894489in}}%
\pgfpathlineto{\pgfqpoint{4.295450in}{1.853308in}}%
\pgfpathlineto{\pgfqpoint{4.304257in}{2.244484in}}%
\pgfpathlineto{\pgfqpoint{4.321870in}{1.770975in}}%
\pgfpathlineto{\pgfqpoint{4.330677in}{1.764116in}}%
\pgfpathlineto{\pgfqpoint{4.339484in}{1.599393in}}%
\pgfpathlineto{\pgfqpoint{4.348291in}{1.565071in}}%
\pgfpathlineto{\pgfqpoint{4.357097in}{1.819015in}}%
\pgfpathlineto{\pgfqpoint{4.365904in}{2.018032in}}%
\pgfpathlineto{\pgfqpoint{4.374711in}{2.134687in}}%
\pgfpathlineto{\pgfqpoint{4.383518in}{1.969992in}}%
\pgfpathlineto{\pgfqpoint{4.392325in}{1.839591in}}%
\pgfpathlineto{\pgfqpoint{4.401132in}{1.386658in}}%
\pgfpathlineto{\pgfqpoint{4.409938in}{1.619997in}}%
\pgfpathlineto{\pgfqpoint{4.418745in}{1.626856in}}%
\pgfpathlineto{\pgfqpoint{4.436359in}{1.846450in}}%
\pgfpathlineto{\pgfqpoint{4.445166in}{2.052326in}}%
\pgfpathlineto{\pgfqpoint{4.453972in}{2.594507in}}%
\pgfpathlineto{\pgfqpoint{4.462779in}{2.388603in}}%
\pgfpathlineto{\pgfqpoint{4.471586in}{2.285665in}}%
\pgfpathlineto{\pgfqpoint{4.480393in}{2.134687in}}%
\pgfpathlineto{\pgfqpoint{4.489200in}{2.079789in}}%
\pgfpathlineto{\pgfqpoint{4.498007in}{1.997427in}}%
\pgfpathlineto{\pgfqpoint{4.506813in}{2.052326in}}%
\pgfpathlineto{\pgfqpoint{4.515620in}{2.237625in}}%
\pgfpathlineto{\pgfqpoint{4.524427in}{1.997427in}}%
\pgfpathlineto{\pgfqpoint{4.533234in}{2.429784in}}%
\pgfpathlineto{\pgfqpoint{4.542041in}{1.668037in}}%
\pgfpathlineto{\pgfqpoint{4.550847in}{1.468992in}}%
\pgfpathlineto{\pgfqpoint{4.559654in}{1.420952in}}%
\pgfpathlineto{\pgfqpoint{4.568461in}{1.235680in}}%
\pgfpathlineto{\pgfqpoint{4.577268in}{1.530777in}}%
\pgfpathlineto{\pgfqpoint{4.586075in}{1.517031in}}%
\pgfpathlineto{\pgfqpoint{4.594882in}{1.283720in}}%
\pgfpathlineto{\pgfqpoint{4.603688in}{1.654291in}}%
\pgfpathlineto{\pgfqpoint{4.612495in}{1.647432in}}%
\pgfpathlineto{\pgfqpoint{4.621302in}{1.338618in}}%
\pgfpathlineto{\pgfqpoint{4.630109in}{1.736653in}}%
\pgfpathlineto{\pgfqpoint{4.638916in}{1.880772in}}%
\pgfpathlineto{\pgfqpoint{4.647722in}{1.750370in}}%
\pgfpathlineto{\pgfqpoint{4.665336in}{1.990569in}}%
\pgfpathlineto{\pgfqpoint{4.682950in}{1.517031in}}%
\pgfpathlineto{\pgfqpoint{4.691757in}{1.379799in}}%
\pgfpathlineto{\pgfqpoint{4.700563in}{1.668037in}}%
\pgfpathlineto{\pgfqpoint{4.709370in}{1.688613in}}%
\pgfpathlineto{\pgfqpoint{4.718177in}{1.661150in}}%
\pgfpathlineto{\pgfqpoint{4.726984in}{1.846450in}}%
\pgfpathlineto{\pgfqpoint{4.735791in}{1.983710in}}%
\pgfpathlineto{\pgfqpoint{4.744597in}{2.162151in}}%
\pgfpathlineto{\pgfqpoint{4.753404in}{2.066071in}}%
\pgfpathlineto{\pgfqpoint{4.762211in}{2.217049in}}%
\pgfpathlineto{\pgfqpoint{4.771018in}{2.169009in}}%
\pgfpathlineto{\pgfqpoint{4.779825in}{2.189586in}}%
\pgfpathlineto{\pgfqpoint{4.788632in}{2.347422in}}%
\pgfpathlineto{\pgfqpoint{4.797438in}{2.004286in}}%
\pgfpathlineto{\pgfqpoint{4.806245in}{1.921953in}}%
\pgfpathlineto{\pgfqpoint{4.815052in}{2.169009in}}%
\pgfpathlineto{\pgfqpoint{4.823859in}{2.066071in}}%
\pgfpathlineto{\pgfqpoint{4.832666in}{1.949388in}}%
\pgfpathlineto{\pgfqpoint{4.841472in}{1.921953in}}%
\pgfpathlineto{\pgfqpoint{4.850279in}{1.990569in}}%
\pgfpathlineto{\pgfqpoint{4.859086in}{1.853308in}}%
\pgfpathlineto{\pgfqpoint{4.867893in}{1.791551in}}%
\pgfpathlineto{\pgfqpoint{4.876700in}{1.709190in}}%
\pgfpathlineto{\pgfqpoint{4.885507in}{1.805269in}}%
\pgfpathlineto{\pgfqpoint{4.894313in}{1.668037in}}%
\pgfpathlineto{\pgfqpoint{4.903120in}{1.558212in}}%
\pgfpathlineto{\pgfqpoint{4.920734in}{1.626856in}}%
\pgfpathlineto{\pgfqpoint{4.929541in}{1.942529in}}%
\pgfpathlineto{\pgfqpoint{4.938347in}{1.963133in}}%
\pgfpathlineto{\pgfqpoint{4.947154in}{2.169009in}}%
\pgfpathlineto{\pgfqpoint{4.955961in}{2.155292in}}%
\pgfpathlineto{\pgfqpoint{4.964768in}{2.024891in}}%
\pgfpathlineto{\pgfqpoint{4.973575in}{2.189586in}}%
\pgfpathlineto{\pgfqpoint{4.982382in}{2.148405in}}%
\pgfpathlineto{\pgfqpoint{4.991188in}{2.052326in}}%
\pgfpathlineto{\pgfqpoint{4.999995in}{1.894489in}}%
\pgfpathlineto{\pgfqpoint{5.008802in}{1.901348in}}%
\pgfpathlineto{\pgfqpoint{5.017609in}{2.059213in}}%
\pgfpathlineto{\pgfqpoint{5.026416in}{2.079789in}}%
\pgfpathlineto{\pgfqpoint{5.035222in}{1.654291in}}%
\pgfpathlineto{\pgfqpoint{5.044029in}{1.963133in}}%
\pgfpathlineto{\pgfqpoint{5.052836in}{1.956246in}}%
\pgfpathlineto{\pgfqpoint{5.061643in}{1.894489in}}%
\pgfpathlineto{\pgfqpoint{5.070450in}{2.024891in}}%
\pgfpathlineto{\pgfqpoint{5.079257in}{1.702331in}}%
\pgfpathlineto{\pgfqpoint{5.088063in}{1.770975in}}%
\pgfpathlineto{\pgfqpoint{5.096870in}{1.558212in}}%
\pgfpathlineto{\pgfqpoint{5.105677in}{1.921953in}}%
\pgfpathlineto{\pgfqpoint{5.114484in}{1.894489in}}%
\pgfpathlineto{\pgfqpoint{5.123291in}{2.004286in}}%
\pgfpathlineto{\pgfqpoint{5.132097in}{1.729794in}}%
\pgfpathlineto{\pgfqpoint{5.140904in}{1.613110in}}%
\pgfpathlineto{\pgfqpoint{5.149711in}{1.819015in}}%
\pgfpathlineto{\pgfqpoint{5.158518in}{1.716076in}}%
\pgfpathlineto{\pgfqpoint{5.167325in}{1.757229in}}%
\pgfpathlineto{\pgfqpoint{5.176132in}{1.537636in}}%
\pgfpathlineto{\pgfqpoint{5.184938in}{1.949388in}}%
\pgfpathlineto{\pgfqpoint{5.193745in}{1.963133in}}%
\pgfpathlineto{\pgfqpoint{5.202552in}{1.825873in}}%
\pgfpathlineto{\pgfqpoint{5.211359in}{1.709190in}}%
\pgfpathlineto{\pgfqpoint{5.220166in}{2.024891in}}%
\pgfpathlineto{\pgfqpoint{5.228972in}{2.189586in}}%
\pgfpathlineto{\pgfqpoint{5.237779in}{1.976851in}}%
\pgfpathlineto{\pgfqpoint{5.246586in}{1.956246in}}%
\pgfpathlineto{\pgfqpoint{5.255393in}{2.203303in}}%
\pgfpathlineto{\pgfqpoint{5.264200in}{2.148405in}}%
\pgfpathlineto{\pgfqpoint{5.273007in}{1.633715in}}%
\pgfpathlineto{\pgfqpoint{5.281813in}{1.565071in}}%
\pgfpathlineto{\pgfqpoint{5.290620in}{1.565071in}}%
\pgfpathlineto{\pgfqpoint{5.299427in}{1.633715in}}%
\pgfpathlineto{\pgfqpoint{5.308234in}{1.510172in}}%
\pgfpathlineto{\pgfqpoint{5.317041in}{1.709190in}}%
\pgfpathlineto{\pgfqpoint{5.325847in}{1.736653in}}%
\pgfpathlineto{\pgfqpoint{5.334654in}{1.633715in}}%
\pgfpathlineto{\pgfqpoint{5.343461in}{1.729794in}}%
\pgfpathlineto{\pgfqpoint{5.352268in}{1.702331in}}%
\pgfpathlineto{\pgfqpoint{5.361075in}{1.578816in}}%
\pgfpathlineto{\pgfqpoint{5.369882in}{1.681754in}}%
\pgfpathlineto{\pgfqpoint{5.378688in}{1.455274in}}%
\pgfpathlineto{\pgfqpoint{5.387495in}{1.846450in}}%
\pgfpathlineto{\pgfqpoint{5.396302in}{1.798410in}}%
\pgfpathlineto{\pgfqpoint{5.405109in}{1.846450in}}%
\pgfpathlineto{\pgfqpoint{5.413916in}{1.935670in}}%
\pgfpathlineto{\pgfqpoint{5.422722in}{2.045467in}}%
\pgfpathlineto{\pgfqpoint{5.431529in}{1.709190in}}%
\pgfpathlineto{\pgfqpoint{5.440336in}{1.647432in}}%
\pgfpathlineto{\pgfqpoint{5.449143in}{1.606252in}}%
\pgfpathlineto{\pgfqpoint{5.457950in}{1.805269in}}%
\pgfpathlineto{\pgfqpoint{5.466757in}{2.038608in}}%
\pgfpathlineto{\pgfqpoint{5.475563in}{2.100365in}}%
\pgfpathlineto{\pgfqpoint{5.484370in}{1.702331in}}%
\pgfpathlineto{\pgfqpoint{5.493177in}{1.606252in}}%
\pgfpathlineto{\pgfqpoint{5.501984in}{1.606252in}}%
\pgfpathlineto{\pgfqpoint{5.510791in}{1.496455in}}%
\pgfpathlineto{\pgfqpoint{5.519597in}{1.420952in}}%
\pgfpathlineto{\pgfqpoint{5.528404in}{1.571958in}}%
\pgfpathlineto{\pgfqpoint{5.537211in}{1.599393in}}%
\pgfpathlineto{\pgfqpoint{5.546018in}{1.558212in}}%
\pgfpathlineto{\pgfqpoint{5.554825in}{1.647432in}}%
\pgfpathlineto{\pgfqpoint{5.563632in}{1.770975in}}%
\pgfpathlineto{\pgfqpoint{5.572438in}{1.770975in}}%
\pgfpathlineto{\pgfqpoint{5.581245in}{1.517031in}}%
\pgfpathlineto{\pgfqpoint{5.590052in}{1.853308in}}%
\pgfpathlineto{\pgfqpoint{5.598859in}{1.915094in}}%
\pgfpathlineto{\pgfqpoint{5.607666in}{2.134687in}}%
\pgfpathlineto{\pgfqpoint{5.616472in}{2.100365in}}%
\pgfpathlineto{\pgfqpoint{5.625279in}{2.024891in}}%
\pgfpathlineto{\pgfqpoint{5.634086in}{1.633715in}}%
\pgfpathlineto{\pgfqpoint{5.642893in}{2.114111in}}%
\pgfpathlineto{\pgfqpoint{5.651700in}{2.278806in}}%
\pgfpathlineto{\pgfqpoint{5.660507in}{2.244484in}}%
\pgfpathlineto{\pgfqpoint{5.669313in}{2.072930in}}%
\pgfpathlineto{\pgfqpoint{5.678120in}{2.011173in}}%
\pgfpathlineto{\pgfqpoint{5.686927in}{1.770975in}}%
\pgfpathlineto{\pgfqpoint{5.695734in}{1.702331in}}%
\pgfpathlineto{\pgfqpoint{5.704541in}{1.860167in}}%
\pgfpathlineto{\pgfqpoint{5.713347in}{1.921953in}}%
\pgfpathlineto{\pgfqpoint{5.722154in}{2.031749in}}%
\pgfpathlineto{\pgfqpoint{5.730961in}{1.791551in}}%
\pgfpathlineto{\pgfqpoint{5.739768in}{1.887630in}}%
\pgfpathlineto{\pgfqpoint{5.748575in}{2.292524in}}%
\pgfpathlineto{\pgfqpoint{5.757382in}{2.018032in}}%
\pgfpathlineto{\pgfqpoint{5.766188in}{2.148405in}}%
\pgfpathlineto{\pgfqpoint{5.774995in}{1.990569in}}%
\pgfpathlineto{\pgfqpoint{5.783802in}{1.997427in}}%
\pgfpathlineto{\pgfqpoint{5.792609in}{1.812128in}}%
\pgfpathlineto{\pgfqpoint{5.801416in}{2.189586in}}%
\pgfpathlineto{\pgfqpoint{5.810222in}{2.004286in}}%
\pgfpathlineto{\pgfqpoint{5.819029in}{1.867054in}}%
\pgfpathlineto{\pgfqpoint{5.827836in}{1.860167in}}%
\pgfpathlineto{\pgfqpoint{5.836643in}{1.908207in}}%
\pgfpathlineto{\pgfqpoint{5.845450in}{1.901348in}}%
\pgfpathlineto{\pgfqpoint{5.854257in}{2.230767in}}%
\pgfpathlineto{\pgfqpoint{5.863063in}{1.983710in}}%
\pgfpathlineto{\pgfqpoint{5.871870in}{1.887630in}}%
\pgfpathlineto{\pgfqpoint{5.880677in}{1.928811in}}%
\pgfpathlineto{\pgfqpoint{5.889484in}{2.175868in}}%
\pgfpathlineto{\pgfqpoint{5.898291in}{2.347422in}}%
\pgfpathlineto{\pgfqpoint{5.907097in}{2.450388in}}%
\pgfpathlineto{\pgfqpoint{5.915904in}{2.313128in}}%
\pgfpathlineto{\pgfqpoint{5.933518in}{1.969992in}}%
\pgfpathlineto{\pgfqpoint{5.942325in}{1.839591in}}%
\pgfpathlineto{\pgfqpoint{5.951132in}{1.812128in}}%
\pgfpathlineto{\pgfqpoint{5.959938in}{1.867054in}}%
\pgfpathlineto{\pgfqpoint{5.968745in}{2.038608in}}%
\pgfpathlineto{\pgfqpoint{5.977552in}{2.059213in}}%
\pgfpathlineto{\pgfqpoint{5.986359in}{1.770975in}}%
\pgfpathlineto{\pgfqpoint{5.995166in}{1.585675in}}%
\pgfpathlineto{\pgfqpoint{6.003972in}{1.571958in}}%
\pgfpathlineto{\pgfqpoint{6.012779in}{1.537636in}}%
\pgfpathlineto{\pgfqpoint{6.021586in}{1.530777in}}%
\pgfpathlineto{\pgfqpoint{6.030393in}{1.592534in}}%
\pgfpathlineto{\pgfqpoint{6.039200in}{2.107252in}}%
\pgfpathlineto{\pgfqpoint{6.048007in}{2.024891in}}%
\pgfpathlineto{\pgfqpoint{6.056813in}{2.162151in}}%
\pgfpathlineto{\pgfqpoint{6.065620in}{1.928811in}}%
\pgfpathlineto{\pgfqpoint{6.074427in}{2.038608in}}%
\pgfpathlineto{\pgfqpoint{6.083234in}{1.901348in}}%
\pgfpathlineto{\pgfqpoint{6.092041in}{1.606252in}}%
\pgfpathlineto{\pgfqpoint{6.100847in}{1.846450in}}%
\pgfpathlineto{\pgfqpoint{6.109654in}{1.633715in}}%
\pgfpathlineto{\pgfqpoint{6.118461in}{1.551353in}}%
\pgfpathlineto{\pgfqpoint{6.127268in}{1.654291in}}%
\pgfpathlineto{\pgfqpoint{6.136075in}{1.915094in}}%
\pgfpathlineto{\pgfqpoint{6.144882in}{2.114111in}}%
\pgfpathlineto{\pgfqpoint{6.162495in}{1.407234in}}%
\pgfpathlineto{\pgfqpoint{6.180109in}{2.223908in}}%
\pgfpathlineto{\pgfqpoint{6.188916in}{2.031749in}}%
\pgfpathlineto{\pgfqpoint{6.197722in}{1.935670in}}%
\pgfpathlineto{\pgfqpoint{6.206529in}{1.908207in}}%
\pgfpathlineto{\pgfqpoint{6.215336in}{1.475878in}}%
\pgfpathlineto{\pgfqpoint{6.224143in}{1.668037in}}%
\pgfpathlineto{\pgfqpoint{6.232950in}{1.777834in}}%
\pgfpathlineto{\pgfqpoint{6.241757in}{1.770975in}}%
\pgfpathlineto{\pgfqpoint{6.250563in}{1.757229in}}%
\pgfpathlineto{\pgfqpoint{6.259370in}{1.784692in}}%
\pgfpathlineto{\pgfqpoint{6.268177in}{1.867054in}}%
\pgfpathlineto{\pgfqpoint{6.276984in}{1.688613in}}%
\pgfpathlineto{\pgfqpoint{6.285791in}{1.942529in}}%
\pgfpathlineto{\pgfqpoint{6.294597in}{1.887630in}}%
\pgfpathlineto{\pgfqpoint{6.303404in}{2.223908in}}%
\pgfpathlineto{\pgfqpoint{6.312211in}{2.237625in}}%
\pgfpathlineto{\pgfqpoint{6.321018in}{2.024891in}}%
\pgfpathlineto{\pgfqpoint{6.329825in}{2.059213in}}%
\pgfpathlineto{\pgfqpoint{6.338632in}{2.066071in}}%
\pgfpathlineto{\pgfqpoint{6.347438in}{2.313128in}}%
\pgfpathlineto{\pgfqpoint{6.356245in}{2.189586in}}%
\pgfpathlineto{\pgfqpoint{6.365052in}{2.470965in}}%
\pgfpathlineto{\pgfqpoint{6.373859in}{2.162151in}}%
\pgfpathlineto{\pgfqpoint{6.382666in}{2.217049in}}%
\pgfpathlineto{\pgfqpoint{6.391472in}{1.887630in}}%
\pgfpathlineto{\pgfqpoint{6.400279in}{1.915094in}}%
\pgfpathlineto{\pgfqpoint{6.409086in}{1.949388in}}%
\pgfpathlineto{\pgfqpoint{6.417893in}{2.114111in}}%
\pgfpathlineto{\pgfqpoint{6.426700in}{2.422925in}}%
\pgfpathlineto{\pgfqpoint{6.435507in}{2.278806in}}%
\pgfpathlineto{\pgfqpoint{6.444313in}{2.072930in}}%
\pgfpathlineto{\pgfqpoint{6.453120in}{2.018032in}}%
\pgfpathlineto{\pgfqpoint{6.461927in}{1.983710in}}%
\pgfpathlineto{\pgfqpoint{6.470734in}{1.873913in}}%
\pgfpathlineto{\pgfqpoint{6.479541in}{2.114111in}}%
\pgfpathlineto{\pgfqpoint{6.488347in}{2.422925in}}%
\pgfpathlineto{\pgfqpoint{6.497154in}{2.278806in}}%
\pgfpathlineto{\pgfqpoint{6.505961in}{2.368027in}}%
\pgfpathlineto{\pgfqpoint{6.514768in}{2.114111in}}%
\pgfpathlineto{\pgfqpoint{6.523575in}{2.155292in}}%
\pgfpathlineto{\pgfqpoint{6.532382in}{1.983710in}}%
\pgfpathlineto{\pgfqpoint{6.541188in}{1.867054in}}%
\pgfpathlineto{\pgfqpoint{6.549995in}{1.846450in}}%
\pgfpathlineto{\pgfqpoint{6.558802in}{2.182727in}}%
\pgfpathlineto{\pgfqpoint{6.567609in}{1.956246in}}%
\pgfpathlineto{\pgfqpoint{6.576416in}{2.018032in}}%
\pgfpathlineto{\pgfqpoint{6.585222in}{1.990569in}}%
\pgfpathlineto{\pgfqpoint{6.602836in}{1.757229in}}%
\pgfpathlineto{\pgfqpoint{6.611643in}{1.681754in}}%
\pgfpathlineto{\pgfqpoint{6.620450in}{1.592534in}}%
\pgfpathlineto{\pgfqpoint{6.629257in}{1.613110in}}%
\pgfpathlineto{\pgfqpoint{6.638063in}{1.668037in}}%
\pgfpathlineto{\pgfqpoint{6.646870in}{1.674896in}}%
\pgfpathlineto{\pgfqpoint{6.655677in}{1.626856in}}%
\pgfpathlineto{\pgfqpoint{6.664484in}{1.777834in}}%
\pgfpathlineto{\pgfqpoint{6.673291in}{1.805269in}}%
\pgfpathlineto{\pgfqpoint{6.682097in}{1.825873in}}%
\pgfpathlineto{\pgfqpoint{6.690904in}{1.640574in}}%
\pgfpathlineto{\pgfqpoint{6.699711in}{1.647432in}}%
\pgfpathlineto{\pgfqpoint{6.708518in}{1.722935in}}%
\pgfpathlineto{\pgfqpoint{6.734938in}{2.024891in}}%
\pgfpathlineto{\pgfqpoint{6.743745in}{2.203303in}}%
\pgfpathlineto{\pgfqpoint{6.752552in}{2.093507in}}%
\pgfpathlineto{\pgfqpoint{6.761359in}{2.210190in}}%
\pgfpathlineto{\pgfqpoint{6.770166in}{2.189586in}}%
\pgfpathlineto{\pgfqpoint{6.778972in}{1.901348in}}%
\pgfpathlineto{\pgfqpoint{6.787779in}{1.880772in}}%
\pgfpathlineto{\pgfqpoint{6.796586in}{2.093507in}}%
\pgfpathlineto{\pgfqpoint{6.805393in}{1.935670in}}%
\pgfpathlineto{\pgfqpoint{6.814200in}{1.867054in}}%
\pgfpathlineto{\pgfqpoint{6.823007in}{1.750370in}}%
\pgfpathlineto{\pgfqpoint{6.831813in}{1.764116in}}%
\pgfpathlineto{\pgfqpoint{6.840620in}{1.462133in}}%
\pgfpathlineto{\pgfqpoint{6.849427in}{1.860167in}}%
\pgfpathlineto{\pgfqpoint{6.858234in}{2.093507in}}%
\pgfpathlineto{\pgfqpoint{6.867041in}{2.210190in}}%
\pgfpathlineto{\pgfqpoint{6.875847in}{1.915094in}}%
\pgfpathlineto{\pgfqpoint{6.884654in}{1.969992in}}%
\pgfpathlineto{\pgfqpoint{6.893461in}{2.072930in}}%
\pgfpathlineto{\pgfqpoint{6.902268in}{2.024891in}}%
\pgfpathlineto{\pgfqpoint{6.911075in}{2.134687in}}%
\pgfpathlineto{\pgfqpoint{6.919882in}{2.018032in}}%
\pgfpathlineto{\pgfqpoint{6.928688in}{1.832732in}}%
\pgfpathlineto{\pgfqpoint{6.937495in}{1.722935in}}%
\pgfpathlineto{\pgfqpoint{6.946302in}{1.949388in}}%
\pgfpathlineto{\pgfqpoint{6.955109in}{2.107252in}}%
\pgfpathlineto{\pgfqpoint{6.963916in}{2.210190in}}%
\pgfpathlineto{\pgfqpoint{6.972722in}{2.237625in}}%
\pgfpathlineto{\pgfqpoint{6.981529in}{2.169009in}}%
\pgfpathlineto{\pgfqpoint{6.990336in}{2.319987in}}%
\pgfpathlineto{\pgfqpoint{6.999143in}{2.333705in}}%
\pgfpathlineto{\pgfqpoint{7.007950in}{2.285665in}}%
\pgfpathlineto{\pgfqpoint{7.016757in}{2.120970in}}%
\pgfpathlineto{\pgfqpoint{7.025563in}{2.155292in}}%
\pgfpathlineto{\pgfqpoint{7.034370in}{2.120970in}}%
\pgfpathlineto{\pgfqpoint{7.043177in}{2.066071in}}%
\pgfpathlineto{\pgfqpoint{7.051984in}{1.784692in}}%
\pgfpathlineto{\pgfqpoint{7.069597in}{2.251343in}}%
\pgfpathlineto{\pgfqpoint{7.078404in}{2.107252in}}%
\pgfpathlineto{\pgfqpoint{7.087211in}{2.223908in}}%
\pgfpathlineto{\pgfqpoint{7.096018in}{2.114111in}}%
\pgfpathlineto{\pgfqpoint{7.104825in}{2.107252in}}%
\pgfpathlineto{\pgfqpoint{7.113632in}{2.093507in}}%
\pgfpathlineto{\pgfqpoint{7.122438in}{1.956246in}}%
\pgfpathlineto{\pgfqpoint{7.131245in}{1.921953in}}%
\pgfpathlineto{\pgfqpoint{7.140052in}{1.688613in}}%
\pgfpathlineto{\pgfqpoint{7.148859in}{1.990569in}}%
\pgfpathlineto{\pgfqpoint{7.157666in}{1.832732in}}%
\pgfpathlineto{\pgfqpoint{7.166472in}{2.004286in}}%
\pgfpathlineto{\pgfqpoint{7.175279in}{1.729794in}}%
\pgfpathlineto{\pgfqpoint{7.184086in}{1.709190in}}%
\pgfpathlineto{\pgfqpoint{7.192893in}{1.873913in}}%
\pgfpathlineto{\pgfqpoint{7.201700in}{1.901348in}}%
\pgfpathlineto{\pgfqpoint{7.210507in}{1.750370in}}%
\pgfpathlineto{\pgfqpoint{7.219313in}{1.640574in}}%
\pgfpathlineto{\pgfqpoint{7.228120in}{1.269974in}}%
\pgfpathlineto{\pgfqpoint{7.236927in}{1.324873in}}%
\pgfpathlineto{\pgfqpoint{7.245734in}{1.352336in}}%
\pgfpathlineto{\pgfqpoint{7.254541in}{1.867054in}}%
\pgfpathlineto{\pgfqpoint{7.263347in}{1.668037in}}%
\pgfpathlineto{\pgfqpoint{7.272154in}{2.271947in}}%
\pgfpathlineto{\pgfqpoint{7.280961in}{1.990569in}}%
\pgfpathlineto{\pgfqpoint{7.289768in}{1.928811in}}%
\pgfpathlineto{\pgfqpoint{7.298575in}{1.839591in}}%
\pgfpathlineto{\pgfqpoint{7.307382in}{1.592534in}}%
\pgfpathlineto{\pgfqpoint{7.324995in}{2.127829in}}%
\pgfpathlineto{\pgfqpoint{7.333802in}{2.491541in}}%
\pgfpathlineto{\pgfqpoint{7.351416in}{2.361168in}}%
\pgfpathlineto{\pgfqpoint{7.360222in}{2.155292in}}%
\pgfpathlineto{\pgfqpoint{7.369029in}{2.210190in}}%
\pgfpathlineto{\pgfqpoint{7.377836in}{2.169009in}}%
\pgfpathlineto{\pgfqpoint{7.386643in}{2.155292in}}%
\pgfpathlineto{\pgfqpoint{7.395450in}{1.633715in}}%
\pgfpathlineto{\pgfqpoint{7.404257in}{1.777834in}}%
\pgfpathlineto{\pgfqpoint{7.413063in}{1.798410in}}%
\pgfpathlineto{\pgfqpoint{7.421870in}{2.018032in}}%
\pgfpathlineto{\pgfqpoint{7.430677in}{2.011173in}}%
\pgfpathlineto{\pgfqpoint{7.439484in}{2.045467in}}%
\pgfpathlineto{\pgfqpoint{7.448291in}{1.990569in}}%
\pgfpathlineto{\pgfqpoint{7.457097in}{2.141546in}}%
\pgfpathlineto{\pgfqpoint{7.465904in}{2.189586in}}%
\pgfpathlineto{\pgfqpoint{7.474711in}{1.921953in}}%
\pgfpathlineto{\pgfqpoint{7.483518in}{1.894489in}}%
\pgfpathlineto{\pgfqpoint{7.492325in}{2.093507in}}%
\pgfpathlineto{\pgfqpoint{7.501132in}{2.258230in}}%
\pgfpathlineto{\pgfqpoint{7.509938in}{2.100365in}}%
\pgfpathlineto{\pgfqpoint{7.518745in}{2.148405in}}%
\pgfpathlineto{\pgfqpoint{7.527552in}{1.942529in}}%
\pgfpathlineto{\pgfqpoint{7.536359in}{2.210190in}}%
\pgfpathlineto{\pgfqpoint{7.545166in}{1.963133in}}%
\pgfpathlineto{\pgfqpoint{7.553972in}{2.374885in}}%
\pgfpathlineto{\pgfqpoint{7.562779in}{1.983710in}}%
\pgfpathlineto{\pgfqpoint{7.571586in}{2.134687in}}%
\pgfpathlineto{\pgfqpoint{7.580393in}{2.313128in}}%
\pgfpathlineto{\pgfqpoint{7.589200in}{2.319987in}}%
\pgfpathlineto{\pgfqpoint{7.598007in}{2.464106in}}%
\pgfpathlineto{\pgfqpoint{7.606813in}{2.093507in}}%
\pgfpathlineto{\pgfqpoint{7.615620in}{2.182727in}}%
\pgfpathlineto{\pgfqpoint{7.624427in}{1.722935in}}%
\pgfpathlineto{\pgfqpoint{7.633234in}{1.956246in}}%
\pgfpathlineto{\pgfqpoint{7.642041in}{1.908207in}}%
\pgfpathlineto{\pgfqpoint{7.650847in}{1.935670in}}%
\pgfpathlineto{\pgfqpoint{7.659654in}{2.004286in}}%
\pgfpathlineto{\pgfqpoint{7.668461in}{1.976851in}}%
\pgfpathlineto{\pgfqpoint{7.677268in}{2.114111in}}%
\pgfpathlineto{\pgfqpoint{7.686075in}{1.784692in}}%
\pgfpathlineto{\pgfqpoint{7.694882in}{2.011173in}}%
\pgfpathlineto{\pgfqpoint{7.703688in}{2.038608in}}%
\pgfpathlineto{\pgfqpoint{7.712495in}{2.141546in}}%
\pgfpathlineto{\pgfqpoint{7.721302in}{1.894489in}}%
\pgfpathlineto{\pgfqpoint{7.730109in}{2.004286in}}%
\pgfpathlineto{\pgfqpoint{7.738916in}{1.949388in}}%
\pgfpathlineto{\pgfqpoint{7.747722in}{2.024891in}}%
\pgfpathlineto{\pgfqpoint{7.756529in}{2.169009in}}%
\pgfpathlineto{\pgfqpoint{7.765336in}{1.640574in}}%
\pgfpathlineto{\pgfqpoint{7.774143in}{1.571958in}}%
\pgfpathlineto{\pgfqpoint{7.782950in}{1.331760in}}%
\pgfpathlineto{\pgfqpoint{7.791757in}{1.050381in}}%
\pgfpathlineto{\pgfqpoint{7.800563in}{1.029776in}}%
\pgfpathlineto{\pgfqpoint{7.809370in}{1.084703in}}%
\pgfpathlineto{\pgfqpoint{7.818177in}{1.249398in}}%
\pgfpathlineto{\pgfqpoint{7.826984in}{1.276861in}}%
\pgfpathlineto{\pgfqpoint{7.835791in}{1.208217in}}%
\pgfpathlineto{\pgfqpoint{7.844597in}{1.455274in}}%
\pgfpathlineto{\pgfqpoint{7.853404in}{1.770975in}}%
\pgfpathlineto{\pgfqpoint{7.862211in}{1.626856in}}%
\pgfpathlineto{\pgfqpoint{7.871018in}{1.619997in}}%
\pgfpathlineto{\pgfqpoint{7.879825in}{1.805269in}}%
\pgfpathlineto{\pgfqpoint{7.888632in}{1.798410in}}%
\pgfpathlineto{\pgfqpoint{7.897438in}{1.805269in}}%
\pgfpathlineto{\pgfqpoint{7.906245in}{1.695472in}}%
\pgfpathlineto{\pgfqpoint{7.915052in}{1.551353in}}%
\pgfpathlineto{\pgfqpoint{7.923859in}{1.331760in}}%
\pgfpathlineto{\pgfqpoint{7.932666in}{1.640574in}}%
\pgfpathlineto{\pgfqpoint{7.941472in}{1.139601in}}%
\pgfpathlineto{\pgfqpoint{7.959086in}{1.784692in}}%
\pgfpathlineto{\pgfqpoint{7.967893in}{1.915094in}}%
\pgfpathlineto{\pgfqpoint{7.976700in}{2.024891in}}%
\pgfpathlineto{\pgfqpoint{7.985507in}{1.853308in}}%
\pgfpathlineto{\pgfqpoint{7.994313in}{2.368027in}}%
\pgfpathlineto{\pgfqpoint{8.003120in}{2.457247in}}%
\pgfpathlineto{\pgfqpoint{8.011927in}{2.127829in}}%
\pgfpathlineto{\pgfqpoint{8.020734in}{2.196445in}}%
\pgfpathlineto{\pgfqpoint{8.029541in}{2.203303in}}%
\pgfpathlineto{\pgfqpoint{8.038347in}{2.127829in}}%
\pgfpathlineto{\pgfqpoint{8.055961in}{2.429784in}}%
\pgfpathlineto{\pgfqpoint{8.064768in}{2.477823in}}%
\pgfpathlineto{\pgfqpoint{8.073575in}{2.491541in}}%
\pgfpathlineto{\pgfqpoint{8.082382in}{2.608225in}}%
\pgfpathlineto{\pgfqpoint{8.091188in}{2.546468in}}%
\pgfpathlineto{\pgfqpoint{8.099995in}{2.738598in}}%
\pgfpathlineto{\pgfqpoint{8.108802in}{2.711163in}}%
\pgfpathlineto{\pgfqpoint{8.117609in}{2.299383in}}%
\pgfpathlineto{\pgfqpoint{8.126416in}{2.141546in}}%
\pgfpathlineto{\pgfqpoint{8.135222in}{2.079789in}}%
\pgfpathlineto{\pgfqpoint{8.144029in}{2.155292in}}%
\pgfpathlineto{\pgfqpoint{8.152836in}{1.942529in}}%
\pgfpathlineto{\pgfqpoint{8.161643in}{1.853308in}}%
\pgfpathlineto{\pgfqpoint{8.170450in}{1.901348in}}%
\pgfpathlineto{\pgfqpoint{8.179257in}{1.915094in}}%
\pgfpathlineto{\pgfqpoint{8.188063in}{1.860167in}}%
\pgfpathlineto{\pgfqpoint{8.196870in}{1.784692in}}%
\pgfpathlineto{\pgfqpoint{8.205677in}{1.448415in}}%
\pgfpathlineto{\pgfqpoint{8.214484in}{1.647432in}}%
\pgfpathlineto{\pgfqpoint{8.223291in}{2.052326in}}%
\pgfpathlineto{\pgfqpoint{8.232097in}{2.292524in}}%
\pgfpathlineto{\pgfqpoint{8.240904in}{2.203303in}}%
\pgfpathlineto{\pgfqpoint{8.249711in}{2.477823in}}%
\pgfpathlineto{\pgfqpoint{8.258518in}{2.340563in}}%
\pgfpathlineto{\pgfqpoint{8.267325in}{2.059213in}}%
\pgfpathlineto{\pgfqpoint{8.276132in}{2.066071in}}%
\pgfpathlineto{\pgfqpoint{8.284938in}{2.210190in}}%
\pgfpathlineto{\pgfqpoint{8.293745in}{2.038608in}}%
\pgfpathlineto{\pgfqpoint{8.302552in}{2.196445in}}%
\pgfpathlineto{\pgfqpoint{8.311359in}{2.141546in}}%
\pgfpathlineto{\pgfqpoint{8.320166in}{1.825873in}}%
\pgfpathlineto{\pgfqpoint{8.328972in}{1.887630in}}%
\pgfpathlineto{\pgfqpoint{8.337779in}{1.819015in}}%
\pgfpathlineto{\pgfqpoint{8.346586in}{1.942529in}}%
\pgfpathlineto{\pgfqpoint{8.355393in}{2.381744in}}%
\pgfpathlineto{\pgfqpoint{8.364200in}{2.333705in}}%
\pgfpathlineto{\pgfqpoint{8.373007in}{2.265089in}}%
\pgfpathlineto{\pgfqpoint{8.381813in}{2.107252in}}%
\pgfpathlineto{\pgfqpoint{8.390620in}{2.093507in}}%
\pgfpathlineto{\pgfqpoint{8.399427in}{2.018032in}}%
\pgfpathlineto{\pgfqpoint{8.408234in}{1.976851in}}%
\pgfpathlineto{\pgfqpoint{8.417041in}{1.921953in}}%
\pgfpathlineto{\pgfqpoint{8.425847in}{1.969992in}}%
\pgfpathlineto{\pgfqpoint{8.434654in}{1.949388in}}%
\pgfpathlineto{\pgfqpoint{8.443461in}{1.894489in}}%
\pgfpathlineto{\pgfqpoint{8.452268in}{1.825873in}}%
\pgfpathlineto{\pgfqpoint{8.461075in}{1.468992in}}%
\pgfpathlineto{\pgfqpoint{8.469882in}{1.709190in}}%
\pgfpathlineto{\pgfqpoint{8.478688in}{2.107252in}}%
\pgfpathlineto{\pgfqpoint{8.487495in}{2.066071in}}%
\pgfpathlineto{\pgfqpoint{8.496302in}{2.196445in}}%
\pgfpathlineto{\pgfqpoint{8.505109in}{2.223908in}}%
\pgfpathlineto{\pgfqpoint{8.513916in}{2.148405in}}%
\pgfpathlineto{\pgfqpoint{8.522722in}{2.114111in}}%
\pgfpathlineto{\pgfqpoint{8.531529in}{2.210190in}}%
\pgfpathlineto{\pgfqpoint{8.540336in}{1.908207in}}%
\pgfpathlineto{\pgfqpoint{8.549143in}{2.265089in}}%
\pgfpathlineto{\pgfqpoint{8.557950in}{2.059213in}}%
\pgfpathlineto{\pgfqpoint{8.566757in}{1.626856in}}%
\pgfpathlineto{\pgfqpoint{8.575563in}{1.558212in}}%
\pgfpathlineto{\pgfqpoint{8.584370in}{1.228822in}}%
\pgfpathlineto{\pgfqpoint{8.593177in}{1.695472in}}%
\pgfpathlineto{\pgfqpoint{8.601984in}{1.482737in}}%
\pgfpathlineto{\pgfqpoint{8.610791in}{1.482737in}}%
\pgfpathlineto{\pgfqpoint{8.619597in}{1.722935in}}%
\pgfpathlineto{\pgfqpoint{8.628404in}{1.750370in}}%
\pgfpathlineto{\pgfqpoint{8.637211in}{2.258230in}}%
\pgfpathlineto{\pgfqpoint{8.646018in}{2.162151in}}%
\pgfpathlineto{\pgfqpoint{8.654825in}{2.107252in}}%
\pgfpathlineto{\pgfqpoint{8.663632in}{2.333705in}}%
\pgfpathlineto{\pgfqpoint{8.672438in}{2.059213in}}%
\pgfpathlineto{\pgfqpoint{8.681245in}{2.100365in}}%
\pgfpathlineto{\pgfqpoint{8.690052in}{2.011173in}}%
\pgfpathlineto{\pgfqpoint{8.698859in}{1.832732in}}%
\pgfpathlineto{\pgfqpoint{8.707666in}{1.606252in}}%
\pgfpathlineto{\pgfqpoint{8.716472in}{1.537636in}}%
\pgfpathlineto{\pgfqpoint{8.725279in}{1.832732in}}%
\pgfpathlineto{\pgfqpoint{8.734086in}{1.956246in}}%
\pgfpathlineto{\pgfqpoint{8.742893in}{1.626856in}}%
\pgfpathlineto{\pgfqpoint{8.751700in}{1.695472in}}%
\pgfpathlineto{\pgfqpoint{8.769313in}{1.963133in}}%
\pgfpathlineto{\pgfqpoint{8.778120in}{1.983710in}}%
\pgfpathlineto{\pgfqpoint{8.786927in}{2.196445in}}%
\pgfpathlineto{\pgfqpoint{8.795734in}{1.935670in}}%
\pgfpathlineto{\pgfqpoint{8.804541in}{1.407234in}}%
\pgfpathlineto{\pgfqpoint{8.813347in}{1.269974in}}%
\pgfpathlineto{\pgfqpoint{8.822154in}{1.455274in}}%
\pgfpathlineto{\pgfqpoint{8.830961in}{1.709190in}}%
\pgfpathlineto{\pgfqpoint{8.839768in}{1.407234in}}%
\pgfpathlineto{\pgfqpoint{8.848575in}{1.839591in}}%
\pgfpathlineto{\pgfqpoint{8.857382in}{1.887630in}}%
\pgfpathlineto{\pgfqpoint{8.866188in}{1.860167in}}%
\pgfpathlineto{\pgfqpoint{8.874995in}{1.613110in}}%
\pgfpathlineto{\pgfqpoint{8.883802in}{1.812128in}}%
\pgfpathlineto{\pgfqpoint{8.892609in}{1.983710in}}%
\pgfpathlineto{\pgfqpoint{8.901416in}{1.853308in}}%
\pgfpathlineto{\pgfqpoint{8.910222in}{1.619997in}}%
\pgfpathlineto{\pgfqpoint{8.919029in}{1.530777in}}%
\pgfpathlineto{\pgfqpoint{8.927836in}{1.633715in}}%
\pgfpathlineto{\pgfqpoint{8.936643in}{1.331760in}}%
\pgfpathlineto{\pgfqpoint{8.945450in}{1.681754in}}%
\pgfpathlineto{\pgfqpoint{8.954257in}{1.901348in}}%
\pgfpathlineto{\pgfqpoint{8.963063in}{2.024891in}}%
\pgfpathlineto{\pgfqpoint{8.971870in}{1.633715in}}%
\pgfpathlineto{\pgfqpoint{8.980677in}{1.462133in}}%
\pgfpathlineto{\pgfqpoint{8.989484in}{1.702331in}}%
\pgfpathlineto{\pgfqpoint{8.998291in}{1.825873in}}%
\pgfpathlineto{\pgfqpoint{9.007097in}{1.983710in}}%
\pgfpathlineto{\pgfqpoint{9.015904in}{1.949388in}}%
\pgfpathlineto{\pgfqpoint{9.024711in}{2.066071in}}%
\pgfpathlineto{\pgfqpoint{9.033518in}{1.757229in}}%
\pgfpathlineto{\pgfqpoint{9.042325in}{1.400376in}}%
\pgfpathlineto{\pgfqpoint{9.051132in}{1.462133in}}%
\pgfpathlineto{\pgfqpoint{9.059938in}{1.585675in}}%
\pgfpathlineto{\pgfqpoint{9.068745in}{1.757229in}}%
\pgfpathlineto{\pgfqpoint{9.077552in}{1.880772in}}%
\pgfpathlineto{\pgfqpoint{9.086359in}{2.079789in}}%
\pgfpathlineto{\pgfqpoint{9.095166in}{2.011173in}}%
\pgfpathlineto{\pgfqpoint{9.103972in}{1.839591in}}%
\pgfpathlineto{\pgfqpoint{9.112779in}{1.915094in}}%
\pgfpathlineto{\pgfqpoint{9.121586in}{2.285665in}}%
\pgfpathlineto{\pgfqpoint{9.130393in}{2.333705in}}%
\pgfpathlineto{\pgfqpoint{9.139200in}{2.107252in}}%
\pgfpathlineto{\pgfqpoint{9.148007in}{1.990569in}}%
\pgfpathlineto{\pgfqpoint{9.156813in}{2.024891in}}%
\pgfpathlineto{\pgfqpoint{9.165620in}{1.722935in}}%
\pgfpathlineto{\pgfqpoint{9.174427in}{1.578816in}}%
\pgfpathlineto{\pgfqpoint{9.183234in}{1.921953in}}%
\pgfpathlineto{\pgfqpoint{9.200847in}{1.434698in}}%
\pgfpathlineto{\pgfqpoint{9.209654in}{1.551353in}}%
\pgfpathlineto{\pgfqpoint{9.218461in}{1.688613in}}%
\pgfpathlineto{\pgfqpoint{9.227268in}{2.011173in}}%
\pgfpathlineto{\pgfqpoint{9.236075in}{1.702331in}}%
\pgfpathlineto{\pgfqpoint{9.244882in}{2.018032in}}%
\pgfpathlineto{\pgfqpoint{9.253688in}{2.148405in}}%
\pgfpathlineto{\pgfqpoint{9.262495in}{1.983710in}}%
\pgfpathlineto{\pgfqpoint{9.271302in}{2.457247in}}%
\pgfpathlineto{\pgfqpoint{9.280109in}{2.381744in}}%
\pgfpathlineto{\pgfqpoint{9.288916in}{2.011173in}}%
\pgfpathlineto{\pgfqpoint{9.297722in}{1.716076in}}%
\pgfpathlineto{\pgfqpoint{9.306529in}{1.599393in}}%
\pgfpathlineto{\pgfqpoint{9.315336in}{2.265089in}}%
\pgfpathlineto{\pgfqpoint{9.324143in}{1.949388in}}%
\pgfpathlineto{\pgfqpoint{9.332950in}{1.860167in}}%
\pgfpathlineto{\pgfqpoint{9.341757in}{2.203303in}}%
\pgfpathlineto{\pgfqpoint{9.350563in}{2.477823in}}%
\pgfpathlineto{\pgfqpoint{9.359370in}{2.313128in}}%
\pgfpathlineto{\pgfqpoint{9.368177in}{2.107252in}}%
\pgfpathlineto{\pgfqpoint{9.376984in}{1.668037in}}%
\pgfpathlineto{\pgfqpoint{9.385791in}{2.086648in}}%
\pgfpathlineto{\pgfqpoint{9.394597in}{1.956246in}}%
\pgfpathlineto{\pgfqpoint{9.403404in}{1.915094in}}%
\pgfpathlineto{\pgfqpoint{9.412211in}{1.702331in}}%
\pgfpathlineto{\pgfqpoint{9.421018in}{2.189586in}}%
\pgfpathlineto{\pgfqpoint{9.429825in}{2.374885in}}%
\pgfpathlineto{\pgfqpoint{9.438632in}{2.292524in}}%
\pgfpathlineto{\pgfqpoint{9.447438in}{2.230767in}}%
\pgfpathlineto{\pgfqpoint{9.456245in}{2.038608in}}%
\pgfpathlineto{\pgfqpoint{9.465052in}{2.217049in}}%
\pgfpathlineto{\pgfqpoint{9.473859in}{2.450388in}}%
\pgfpathlineto{\pgfqpoint{9.482666in}{2.546468in}}%
\pgfpathlineto{\pgfqpoint{9.491472in}{2.148405in}}%
\pgfpathlineto{\pgfqpoint{9.500279in}{2.100365in}}%
\pgfpathlineto{\pgfqpoint{9.509086in}{2.134687in}}%
\pgfpathlineto{\pgfqpoint{9.517893in}{1.750370in}}%
\pgfpathlineto{\pgfqpoint{9.526700in}{1.873913in}}%
\pgfpathlineto{\pgfqpoint{9.535507in}{2.120970in}}%
\pgfpathlineto{\pgfqpoint{9.544313in}{1.901348in}}%
\pgfpathlineto{\pgfqpoint{9.553120in}{1.949388in}}%
\pgfpathlineto{\pgfqpoint{9.561927in}{1.819015in}}%
\pgfpathlineto{\pgfqpoint{9.570734in}{2.114111in}}%
\pgfpathlineto{\pgfqpoint{9.579541in}{2.086648in}}%
\pgfpathlineto{\pgfqpoint{9.588347in}{1.867054in}}%
\pgfpathlineto{\pgfqpoint{9.597154in}{1.798410in}}%
\pgfpathlineto{\pgfqpoint{9.605961in}{1.702331in}}%
\pgfpathlineto{\pgfqpoint{9.614768in}{1.956246in}}%
\pgfpathlineto{\pgfqpoint{9.623575in}{1.867054in}}%
\pgfpathlineto{\pgfqpoint{9.632382in}{1.880772in}}%
\pgfpathlineto{\pgfqpoint{9.641188in}{1.819015in}}%
\pgfpathlineto{\pgfqpoint{9.649995in}{1.963133in}}%
\pgfpathlineto{\pgfqpoint{9.658802in}{1.709190in}}%
\pgfpathlineto{\pgfqpoint{9.667609in}{1.743512in}}%
\pgfpathlineto{\pgfqpoint{9.676416in}{1.867054in}}%
\pgfpathlineto{\pgfqpoint{9.685222in}{1.777834in}}%
\pgfpathlineto{\pgfqpoint{9.694029in}{1.565071in}}%
\pgfpathlineto{\pgfqpoint{9.702836in}{1.750370in}}%
\pgfpathlineto{\pgfqpoint{9.711643in}{1.558212in}}%
\pgfpathlineto{\pgfqpoint{9.720450in}{1.674896in}}%
\pgfpathlineto{\pgfqpoint{9.729257in}{1.880772in}}%
\pgfpathlineto{\pgfqpoint{9.738063in}{1.805269in}}%
\pgfpathlineto{\pgfqpoint{9.746870in}{1.846450in}}%
\pgfpathlineto{\pgfqpoint{9.755677in}{1.812128in}}%
\pgfpathlineto{\pgfqpoint{9.764484in}{1.887630in}}%
\pgfpathlineto{\pgfqpoint{9.773291in}{1.949388in}}%
\pgfpathlineto{\pgfqpoint{9.782097in}{2.175868in}}%
\pgfpathlineto{\pgfqpoint{9.790904in}{2.004286in}}%
\pgfpathlineto{\pgfqpoint{9.799711in}{1.880772in}}%
\pgfpathlineto{\pgfqpoint{9.808518in}{2.141546in}}%
\pgfpathlineto{\pgfqpoint{9.817325in}{1.832732in}}%
\pgfpathlineto{\pgfqpoint{9.826132in}{2.072930in}}%
\pgfpathlineto{\pgfqpoint{9.834938in}{2.038608in}}%
\pgfpathlineto{\pgfqpoint{9.843745in}{1.784692in}}%
\pgfpathlineto{\pgfqpoint{9.852552in}{1.894489in}}%
\pgfpathlineto{\pgfqpoint{9.861359in}{1.956246in}}%
\pgfpathlineto{\pgfqpoint{9.870166in}{1.997427in}}%
\pgfpathlineto{\pgfqpoint{9.878972in}{2.066071in}}%
\pgfpathlineto{\pgfqpoint{9.887779in}{2.162151in}}%
\pgfpathlineto{\pgfqpoint{9.896586in}{2.402349in}}%
\pgfpathlineto{\pgfqpoint{9.905393in}{2.155292in}}%
\pgfpathlineto{\pgfqpoint{9.914200in}{1.942529in}}%
\pgfpathlineto{\pgfqpoint{9.923007in}{1.867054in}}%
\pgfpathlineto{\pgfqpoint{9.931813in}{1.839591in}}%
\pgfpathlineto{\pgfqpoint{9.940620in}{1.455274in}}%
\pgfpathlineto{\pgfqpoint{9.949427in}{1.647432in}}%
\pgfpathlineto{\pgfqpoint{9.949427in}{1.647432in}}%
\pgfusepath{stroke}%
\end{pgfscope}%
\begin{pgfscope}%
\pgfpathrectangle{\pgfqpoint{0.702268in}{0.521603in}}{\pgfqpoint{9.687500in}{4.235000in}}%
\pgfusepath{clip}%
\pgfsetrectcap%
\pgfsetroundjoin%
\pgfsetlinewidth{0.501875pt}%
\definecolor{currentstroke}{rgb}{0.501961,0.501961,0.501961}%
\pgfsetstrokecolor{currentstroke}%
\pgfsetstrokeopacity{0.250000}%
\pgfsetdash{}{0pt}%
\pgfpathmoveto{\pgfqpoint{1.142609in}{4.317046in}}%
\pgfpathlineto{\pgfqpoint{1.160222in}{2.944502in}}%
\pgfpathlineto{\pgfqpoint{1.169029in}{2.443501in}}%
\pgfpathlineto{\pgfqpoint{1.177836in}{2.217049in}}%
\pgfpathlineto{\pgfqpoint{1.186643in}{1.825873in}}%
\pgfpathlineto{\pgfqpoint{1.195450in}{2.230767in}}%
\pgfpathlineto{\pgfqpoint{1.204257in}{2.251343in}}%
\pgfpathlineto{\pgfqpoint{1.213063in}{1.626856in}}%
\pgfpathlineto{\pgfqpoint{1.221870in}{1.167036in}}%
\pgfpathlineto{\pgfqpoint{1.230677in}{1.379799in}}%
\pgfpathlineto{\pgfqpoint{1.239484in}{1.194499in}}%
\pgfpathlineto{\pgfqpoint{1.248291in}{1.475878in}}%
\pgfpathlineto{\pgfqpoint{1.257097in}{1.366053in}}%
\pgfpathlineto{\pgfqpoint{1.265904in}{1.901348in}}%
\pgfpathlineto{\pgfqpoint{1.274711in}{1.668037in}}%
\pgfpathlineto{\pgfqpoint{1.283518in}{1.475878in}}%
\pgfpathlineto{\pgfqpoint{1.292325in}{1.743512in}}%
\pgfpathlineto{\pgfqpoint{1.301132in}{1.853308in}}%
\pgfpathlineto{\pgfqpoint{1.309938in}{2.031749in}}%
\pgfpathlineto{\pgfqpoint{1.318745in}{1.743512in}}%
\pgfpathlineto{\pgfqpoint{1.327552in}{2.072930in}}%
\pgfpathlineto{\pgfqpoint{1.336359in}{2.134687in}}%
\pgfpathlineto{\pgfqpoint{1.345166in}{2.127829in}}%
\pgfpathlineto{\pgfqpoint{1.353972in}{1.846450in}}%
\pgfpathlineto{\pgfqpoint{1.362779in}{1.674896in}}%
\pgfpathlineto{\pgfqpoint{1.371586in}{1.592534in}}%
\pgfpathlineto{\pgfqpoint{1.380393in}{1.338618in}}%
\pgfpathlineto{\pgfqpoint{1.389200in}{1.215076in}}%
\pgfpathlineto{\pgfqpoint{1.398007in}{1.304296in}}%
\pgfpathlineto{\pgfqpoint{1.406813in}{1.496455in}}%
\pgfpathlineto{\pgfqpoint{1.415620in}{1.510172in}}%
\pgfpathlineto{\pgfqpoint{1.424427in}{1.777834in}}%
\pgfpathlineto{\pgfqpoint{1.433234in}{1.873913in}}%
\pgfpathlineto{\pgfqpoint{1.442041in}{1.908207in}}%
\pgfpathlineto{\pgfqpoint{1.450847in}{1.915094in}}%
\pgfpathlineto{\pgfqpoint{1.459654in}{2.004286in}}%
\pgfpathlineto{\pgfqpoint{1.468461in}{2.059213in}}%
\pgfpathlineto{\pgfqpoint{1.477268in}{1.709190in}}%
\pgfpathlineto{\pgfqpoint{1.494882in}{2.175868in}}%
\pgfpathlineto{\pgfqpoint{1.503688in}{2.114111in}}%
\pgfpathlineto{\pgfqpoint{1.512495in}{2.313128in}}%
\pgfpathlineto{\pgfqpoint{1.521302in}{2.134687in}}%
\pgfpathlineto{\pgfqpoint{1.530109in}{2.079789in}}%
\pgfpathlineto{\pgfqpoint{1.538916in}{2.127829in}}%
\pgfpathlineto{\pgfqpoint{1.547722in}{1.956246in}}%
\pgfpathlineto{\pgfqpoint{1.556529in}{1.839591in}}%
\pgfpathlineto{\pgfqpoint{1.565336in}{1.551353in}}%
\pgfpathlineto{\pgfqpoint{1.574143in}{1.709190in}}%
\pgfpathlineto{\pgfqpoint{1.582950in}{1.578816in}}%
\pgfpathlineto{\pgfqpoint{1.591757in}{1.784692in}}%
\pgfpathlineto{\pgfqpoint{1.600563in}{1.963133in}}%
\pgfpathlineto{\pgfqpoint{1.618177in}{2.525863in}}%
\pgfpathlineto{\pgfqpoint{1.626984in}{2.470965in}}%
\pgfpathlineto{\pgfqpoint{1.635791in}{2.512146in}}%
\pgfpathlineto{\pgfqpoint{1.644597in}{2.347422in}}%
\pgfpathlineto{\pgfqpoint{1.653404in}{2.422925in}}%
\pgfpathlineto{\pgfqpoint{1.662211in}{2.388603in}}%
\pgfpathlineto{\pgfqpoint{1.671018in}{2.512146in}}%
\pgfpathlineto{\pgfqpoint{1.679825in}{2.271947in}}%
\pgfpathlineto{\pgfqpoint{1.688632in}{2.422925in}}%
\pgfpathlineto{\pgfqpoint{1.697438in}{2.766061in}}%
\pgfpathlineto{\pgfqpoint{1.706245in}{2.429784in}}%
\pgfpathlineto{\pgfqpoint{1.715052in}{1.921953in}}%
\pgfpathlineto{\pgfqpoint{1.723859in}{1.537636in}}%
\pgfpathlineto{\pgfqpoint{1.732666in}{1.558212in}}%
\pgfpathlineto{\pgfqpoint{1.741472in}{1.455274in}}%
\pgfpathlineto{\pgfqpoint{1.750279in}{1.825873in}}%
\pgfpathlineto{\pgfqpoint{1.767893in}{1.825873in}}%
\pgfpathlineto{\pgfqpoint{1.776700in}{1.798410in}}%
\pgfpathlineto{\pgfqpoint{1.785507in}{1.722935in}}%
\pgfpathlineto{\pgfqpoint{1.794313in}{1.805269in}}%
\pgfpathlineto{\pgfqpoint{1.803120in}{1.695472in}}%
\pgfpathlineto{\pgfqpoint{1.811927in}{1.764116in}}%
\pgfpathlineto{\pgfqpoint{1.820734in}{1.880772in}}%
\pgfpathlineto{\pgfqpoint{1.829541in}{1.825873in}}%
\pgfpathlineto{\pgfqpoint{1.838347in}{2.038608in}}%
\pgfpathlineto{\pgfqpoint{1.847154in}{2.203303in}}%
\pgfpathlineto{\pgfqpoint{1.855961in}{2.093507in}}%
\pgfpathlineto{\pgfqpoint{1.864768in}{2.237625in}}%
\pgfpathlineto{\pgfqpoint{1.873575in}{2.477823in}}%
\pgfpathlineto{\pgfqpoint{1.882382in}{2.196445in}}%
\pgfpathlineto{\pgfqpoint{1.891188in}{2.313128in}}%
\pgfpathlineto{\pgfqpoint{1.899995in}{2.381744in}}%
\pgfpathlineto{\pgfqpoint{1.908802in}{2.546468in}}%
\pgfpathlineto{\pgfqpoint{1.917609in}{2.258230in}}%
\pgfpathlineto{\pgfqpoint{1.926416in}{1.846450in}}%
\pgfpathlineto{\pgfqpoint{1.935222in}{1.681754in}}%
\pgfpathlineto{\pgfqpoint{1.944029in}{2.093507in}}%
\pgfpathlineto{\pgfqpoint{1.952836in}{2.265089in}}%
\pgfpathlineto{\pgfqpoint{1.961643in}{1.949388in}}%
\pgfpathlineto{\pgfqpoint{1.970450in}{1.757229in}}%
\pgfpathlineto{\pgfqpoint{1.979257in}{1.523918in}}%
\pgfpathlineto{\pgfqpoint{1.988063in}{1.743512in}}%
\pgfpathlineto{\pgfqpoint{1.996870in}{1.578816in}}%
\pgfpathlineto{\pgfqpoint{2.005677in}{1.448415in}}%
\pgfpathlineto{\pgfqpoint{2.014484in}{1.722935in}}%
\pgfpathlineto{\pgfqpoint{2.023291in}{1.489596in}}%
\pgfpathlineto{\pgfqpoint{2.032097in}{1.770975in}}%
\pgfpathlineto{\pgfqpoint{2.040904in}{2.223908in}}%
\pgfpathlineto{\pgfqpoint{2.049711in}{2.422925in}}%
\pgfpathlineto{\pgfqpoint{2.058518in}{2.210190in}}%
\pgfpathlineto{\pgfqpoint{2.067325in}{2.086648in}}%
\pgfpathlineto{\pgfqpoint{2.076132in}{1.757229in}}%
\pgfpathlineto{\pgfqpoint{2.084938in}{1.839591in}}%
\pgfpathlineto{\pgfqpoint{2.093745in}{2.457247in}}%
\pgfpathlineto{\pgfqpoint{2.102552in}{2.244484in}}%
\pgfpathlineto{\pgfqpoint{2.111359in}{2.079789in}}%
\pgfpathlineto{\pgfqpoint{2.120166in}{2.107252in}}%
\pgfpathlineto{\pgfqpoint{2.128972in}{2.306269in}}%
\pgfpathlineto{\pgfqpoint{2.137779in}{1.976851in}}%
\pgfpathlineto{\pgfqpoint{2.146586in}{1.873913in}}%
\pgfpathlineto{\pgfqpoint{2.155393in}{1.812128in}}%
\pgfpathlineto{\pgfqpoint{2.164200in}{1.565071in}}%
\pgfpathlineto{\pgfqpoint{2.173007in}{1.908207in}}%
\pgfpathlineto{\pgfqpoint{2.181813in}{1.784692in}}%
\pgfpathlineto{\pgfqpoint{2.190620in}{1.530777in}}%
\pgfpathlineto{\pgfqpoint{2.199427in}{1.606252in}}%
\pgfpathlineto{\pgfqpoint{2.208234in}{1.832732in}}%
\pgfpathlineto{\pgfqpoint{2.217041in}{1.791551in}}%
\pgfpathlineto{\pgfqpoint{2.225847in}{1.764116in}}%
\pgfpathlineto{\pgfqpoint{2.234654in}{2.374885in}}%
\pgfpathlineto{\pgfqpoint{2.243461in}{2.512146in}}%
\pgfpathlineto{\pgfqpoint{2.252268in}{2.162151in}}%
\pgfpathlineto{\pgfqpoint{2.261075in}{2.244484in}}%
\pgfpathlineto{\pgfqpoint{2.269882in}{2.107252in}}%
\pgfpathlineto{\pgfqpoint{2.278688in}{2.072930in}}%
\pgfpathlineto{\pgfqpoint{2.287495in}{1.791551in}}%
\pgfpathlineto{\pgfqpoint{2.296302in}{2.079789in}}%
\pgfpathlineto{\pgfqpoint{2.305109in}{1.969992in}}%
\pgfpathlineto{\pgfqpoint{2.313916in}{2.189586in}}%
\pgfpathlineto{\pgfqpoint{2.322722in}{2.374885in}}%
\pgfpathlineto{\pgfqpoint{2.331529in}{2.244484in}}%
\pgfpathlineto{\pgfqpoint{2.340336in}{2.621942in}}%
\pgfpathlineto{\pgfqpoint{2.349143in}{2.203303in}}%
\pgfpathlineto{\pgfqpoint{2.357950in}{2.189586in}}%
\pgfpathlineto{\pgfqpoint{2.366757in}{2.107252in}}%
\pgfpathlineto{\pgfqpoint{2.375563in}{1.681754in}}%
\pgfpathlineto{\pgfqpoint{2.384370in}{1.537636in}}%
\pgfpathlineto{\pgfqpoint{2.393177in}{1.674896in}}%
\pgfpathlineto{\pgfqpoint{2.401984in}{1.839591in}}%
\pgfpathlineto{\pgfqpoint{2.410791in}{1.819015in}}%
\pgfpathlineto{\pgfqpoint{2.419597in}{1.825873in}}%
\pgfpathlineto{\pgfqpoint{2.428404in}{2.175868in}}%
\pgfpathlineto{\pgfqpoint{2.437211in}{1.997427in}}%
\pgfpathlineto{\pgfqpoint{2.446018in}{1.928811in}}%
\pgfpathlineto{\pgfqpoint{2.454825in}{1.825873in}}%
\pgfpathlineto{\pgfqpoint{2.463632in}{2.079789in}}%
\pgfpathlineto{\pgfqpoint{2.472438in}{2.175868in}}%
\pgfpathlineto{\pgfqpoint{2.481245in}{2.319987in}}%
\pgfpathlineto{\pgfqpoint{2.490052in}{2.299383in}}%
\pgfpathlineto{\pgfqpoint{2.498859in}{2.306269in}}%
\pgfpathlineto{\pgfqpoint{2.507666in}{1.654291in}}%
\pgfpathlineto{\pgfqpoint{2.516472in}{1.688613in}}%
\pgfpathlineto{\pgfqpoint{2.525279in}{1.640574in}}%
\pgfpathlineto{\pgfqpoint{2.534086in}{1.661150in}}%
\pgfpathlineto{\pgfqpoint{2.542893in}{2.114111in}}%
\pgfpathlineto{\pgfqpoint{2.551700in}{2.079789in}}%
\pgfpathlineto{\pgfqpoint{2.560507in}{2.107252in}}%
\pgfpathlineto{\pgfqpoint{2.569313in}{2.368027in}}%
\pgfpathlineto{\pgfqpoint{2.578120in}{2.059213in}}%
\pgfpathlineto{\pgfqpoint{2.586927in}{2.285665in}}%
\pgfpathlineto{\pgfqpoint{2.595734in}{2.258230in}}%
\pgfpathlineto{\pgfqpoint{2.604541in}{2.395462in}}%
\pgfpathlineto{\pgfqpoint{2.622154in}{2.237625in}}%
\pgfpathlineto{\pgfqpoint{2.630961in}{2.024891in}}%
\pgfpathlineto{\pgfqpoint{2.639768in}{2.004286in}}%
\pgfpathlineto{\pgfqpoint{2.648575in}{1.949388in}}%
\pgfpathlineto{\pgfqpoint{2.657382in}{1.915094in}}%
\pgfpathlineto{\pgfqpoint{2.666188in}{1.722935in}}%
\pgfpathlineto{\pgfqpoint{2.674995in}{2.045467in}}%
\pgfpathlineto{\pgfqpoint{2.683802in}{1.709190in}}%
\pgfpathlineto{\pgfqpoint{2.701416in}{1.942529in}}%
\pgfpathlineto{\pgfqpoint{2.710222in}{1.921953in}}%
\pgfpathlineto{\pgfqpoint{2.719029in}{1.846450in}}%
\pgfpathlineto{\pgfqpoint{2.727836in}{1.963133in}}%
\pgfpathlineto{\pgfqpoint{2.736643in}{1.777834in}}%
\pgfpathlineto{\pgfqpoint{2.745450in}{1.908207in}}%
\pgfpathlineto{\pgfqpoint{2.754257in}{2.120970in}}%
\pgfpathlineto{\pgfqpoint{2.763063in}{2.120970in}}%
\pgfpathlineto{\pgfqpoint{2.771870in}{2.100365in}}%
\pgfpathlineto{\pgfqpoint{2.780677in}{2.292524in}}%
\pgfpathlineto{\pgfqpoint{2.789484in}{2.457247in}}%
\pgfpathlineto{\pgfqpoint{2.798291in}{2.381744in}}%
\pgfpathlineto{\pgfqpoint{2.807097in}{1.997427in}}%
\pgfpathlineto{\pgfqpoint{2.815904in}{1.819015in}}%
\pgfpathlineto{\pgfqpoint{2.824711in}{1.983710in}}%
\pgfpathlineto{\pgfqpoint{2.833518in}{1.722935in}}%
\pgfpathlineto{\pgfqpoint{2.842325in}{1.695472in}}%
\pgfpathlineto{\pgfqpoint{2.851132in}{2.134687in}}%
\pgfpathlineto{\pgfqpoint{2.859938in}{2.203303in}}%
\pgfpathlineto{\pgfqpoint{2.868745in}{2.031749in}}%
\pgfpathlineto{\pgfqpoint{2.877552in}{2.024891in}}%
\pgfpathlineto{\pgfqpoint{2.886359in}{1.894489in}}%
\pgfpathlineto{\pgfqpoint{2.895166in}{2.079789in}}%
\pgfpathlineto{\pgfqpoint{2.903972in}{2.024891in}}%
\pgfpathlineto{\pgfqpoint{2.912779in}{2.072930in}}%
\pgfpathlineto{\pgfqpoint{2.921586in}{2.189586in}}%
\pgfpathlineto{\pgfqpoint{2.930393in}{2.175868in}}%
\pgfpathlineto{\pgfqpoint{2.939200in}{2.484682in}}%
\pgfpathlineto{\pgfqpoint{2.948007in}{2.285665in}}%
\pgfpathlineto{\pgfqpoint{2.956813in}{2.333705in}}%
\pgfpathlineto{\pgfqpoint{2.965620in}{2.368027in}}%
\pgfpathlineto{\pgfqpoint{2.974427in}{2.251343in}}%
\pgfpathlineto{\pgfqpoint{2.983234in}{1.825873in}}%
\pgfpathlineto{\pgfqpoint{2.992041in}{1.791551in}}%
\pgfpathlineto{\pgfqpoint{3.000847in}{1.681754in}}%
\pgfpathlineto{\pgfqpoint{3.009654in}{1.832732in}}%
\pgfpathlineto{\pgfqpoint{3.018461in}{1.867054in}}%
\pgfpathlineto{\pgfqpoint{3.027268in}{1.757229in}}%
\pgfpathlineto{\pgfqpoint{3.036075in}{2.100365in}}%
\pgfpathlineto{\pgfqpoint{3.044882in}{1.901348in}}%
\pgfpathlineto{\pgfqpoint{3.053688in}{1.585675in}}%
\pgfpathlineto{\pgfqpoint{3.062495in}{1.798410in}}%
\pgfpathlineto{\pgfqpoint{3.080109in}{1.283720in}}%
\pgfpathlineto{\pgfqpoint{3.088916in}{1.221935in}}%
\pgfpathlineto{\pgfqpoint{3.097722in}{1.345477in}}%
\pgfpathlineto{\pgfqpoint{3.106529in}{1.256257in}}%
\pgfpathlineto{\pgfqpoint{3.115336in}{1.201358in}}%
\pgfpathlineto{\pgfqpoint{3.124143in}{1.077816in}}%
\pgfpathlineto{\pgfqpoint{3.132950in}{1.153319in}}%
\pgfpathlineto{\pgfqpoint{3.141757in}{1.414093in}}%
\pgfpathlineto{\pgfqpoint{3.150563in}{1.736653in}}%
\pgfpathlineto{\pgfqpoint{3.159370in}{1.757229in}}%
\pgfpathlineto{\pgfqpoint{3.168177in}{2.120970in}}%
\pgfpathlineto{\pgfqpoint{3.176984in}{2.313128in}}%
\pgfpathlineto{\pgfqpoint{3.185791in}{2.319987in}}%
\pgfpathlineto{\pgfqpoint{3.194597in}{2.395462in}}%
\pgfpathlineto{\pgfqpoint{3.203404in}{2.347422in}}%
\pgfpathlineto{\pgfqpoint{3.212211in}{2.635660in}}%
\pgfpathlineto{\pgfqpoint{3.221018in}{2.649406in}}%
\pgfpathlineto{\pgfqpoint{3.229825in}{2.697445in}}%
\pgfpathlineto{\pgfqpoint{3.238632in}{2.786638in}}%
\pgfpathlineto{\pgfqpoint{3.247438in}{2.663123in}}%
\pgfpathlineto{\pgfqpoint{3.256245in}{2.313128in}}%
\pgfpathlineto{\pgfqpoint{3.265052in}{1.860167in}}%
\pgfpathlineto{\pgfqpoint{3.273859in}{1.297438in}}%
\pgfpathlineto{\pgfqpoint{3.282666in}{1.098420in}}%
\pgfpathlineto{\pgfqpoint{3.291472in}{1.427839in}}%
\pgfpathlineto{\pgfqpoint{3.300279in}{1.819015in}}%
\pgfpathlineto{\pgfqpoint{3.309086in}{1.853308in}}%
\pgfpathlineto{\pgfqpoint{3.317893in}{1.626856in}}%
\pgfpathlineto{\pgfqpoint{3.326700in}{1.722935in}}%
\pgfpathlineto{\pgfqpoint{3.335507in}{1.873913in}}%
\pgfpathlineto{\pgfqpoint{3.344313in}{2.223908in}}%
\pgfpathlineto{\pgfqpoint{3.353120in}{2.340563in}}%
\pgfpathlineto{\pgfqpoint{3.361927in}{1.997427in}}%
\pgfpathlineto{\pgfqpoint{3.370734in}{1.908207in}}%
\pgfpathlineto{\pgfqpoint{3.379541in}{1.908207in}}%
\pgfpathlineto{\pgfqpoint{3.388347in}{2.038608in}}%
\pgfpathlineto{\pgfqpoint{3.405961in}{1.997427in}}%
\pgfpathlineto{\pgfqpoint{3.414768in}{1.750370in}}%
\pgfpathlineto{\pgfqpoint{3.423575in}{1.674896in}}%
\pgfpathlineto{\pgfqpoint{3.432382in}{1.990569in}}%
\pgfpathlineto{\pgfqpoint{3.441188in}{2.237625in}}%
\pgfpathlineto{\pgfqpoint{3.449995in}{2.237625in}}%
\pgfpathlineto{\pgfqpoint{3.458802in}{1.963133in}}%
\pgfpathlineto{\pgfqpoint{3.467609in}{2.203303in}}%
\pgfpathlineto{\pgfqpoint{3.476416in}{1.942529in}}%
\pgfpathlineto{\pgfqpoint{3.485222in}{1.819015in}}%
\pgfpathlineto{\pgfqpoint{3.494029in}{1.613110in}}%
\pgfpathlineto{\pgfqpoint{3.502836in}{1.695472in}}%
\pgfpathlineto{\pgfqpoint{3.511643in}{2.134687in}}%
\pgfpathlineto{\pgfqpoint{3.520450in}{1.839591in}}%
\pgfpathlineto{\pgfqpoint{3.529257in}{2.018032in}}%
\pgfpathlineto{\pgfqpoint{3.538063in}{2.258230in}}%
\pgfpathlineto{\pgfqpoint{3.546870in}{1.901348in}}%
\pgfpathlineto{\pgfqpoint{3.555677in}{2.079789in}}%
\pgfpathlineto{\pgfqpoint{3.564484in}{2.031749in}}%
\pgfpathlineto{\pgfqpoint{3.573291in}{2.093507in}}%
\pgfpathlineto{\pgfqpoint{3.582097in}{1.873913in}}%
\pgfpathlineto{\pgfqpoint{3.590904in}{1.722935in}}%
\pgfpathlineto{\pgfqpoint{3.599711in}{1.846450in}}%
\pgfpathlineto{\pgfqpoint{3.608518in}{1.798410in}}%
\pgfpathlineto{\pgfqpoint{3.617325in}{1.386658in}}%
\pgfpathlineto{\pgfqpoint{3.626132in}{1.846450in}}%
\pgfpathlineto{\pgfqpoint{3.634938in}{1.702331in}}%
\pgfpathlineto{\pgfqpoint{3.643745in}{1.860167in}}%
\pgfpathlineto{\pgfqpoint{3.652552in}{1.880772in}}%
\pgfpathlineto{\pgfqpoint{3.661359in}{2.086648in}}%
\pgfpathlineto{\pgfqpoint{3.670166in}{2.072930in}}%
\pgfpathlineto{\pgfqpoint{3.678972in}{2.230767in}}%
\pgfpathlineto{\pgfqpoint{3.687779in}{2.011173in}}%
\pgfpathlineto{\pgfqpoint{3.696586in}{2.031749in}}%
\pgfpathlineto{\pgfqpoint{3.705393in}{2.388603in}}%
\pgfpathlineto{\pgfqpoint{3.714200in}{2.416066in}}%
\pgfpathlineto{\pgfqpoint{3.723007in}{2.265089in}}%
\pgfpathlineto{\pgfqpoint{3.731813in}{2.237625in}}%
\pgfpathlineto{\pgfqpoint{3.740620in}{1.969992in}}%
\pgfpathlineto{\pgfqpoint{3.749427in}{1.908207in}}%
\pgfpathlineto{\pgfqpoint{3.758234in}{2.368027in}}%
\pgfpathlineto{\pgfqpoint{3.767041in}{2.347422in}}%
\pgfpathlineto{\pgfqpoint{3.775847in}{2.258230in}}%
\pgfpathlineto{\pgfqpoint{3.784654in}{1.908207in}}%
\pgfpathlineto{\pgfqpoint{3.793461in}{2.072930in}}%
\pgfpathlineto{\pgfqpoint{3.802268in}{1.819015in}}%
\pgfpathlineto{\pgfqpoint{3.819882in}{1.578816in}}%
\pgfpathlineto{\pgfqpoint{3.828688in}{1.839591in}}%
\pgfpathlineto{\pgfqpoint{3.837495in}{1.709190in}}%
\pgfpathlineto{\pgfqpoint{3.855109in}{1.256257in}}%
\pgfpathlineto{\pgfqpoint{3.863916in}{1.585675in}}%
\pgfpathlineto{\pgfqpoint{3.872722in}{1.777834in}}%
\pgfpathlineto{\pgfqpoint{3.881529in}{1.716076in}}%
\pgfpathlineto{\pgfqpoint{3.890336in}{1.729794in}}%
\pgfpathlineto{\pgfqpoint{3.899143in}{1.448415in}}%
\pgfpathlineto{\pgfqpoint{3.907950in}{1.228822in}}%
\pgfpathlineto{\pgfqpoint{3.916757in}{1.228822in}}%
\pgfpathlineto{\pgfqpoint{3.925563in}{1.496455in}}%
\pgfpathlineto{\pgfqpoint{3.934370in}{1.372912in}}%
\pgfpathlineto{\pgfqpoint{3.943177in}{1.455274in}}%
\pgfpathlineto{\pgfqpoint{3.951984in}{1.318014in}}%
\pgfpathlineto{\pgfqpoint{3.960791in}{1.688613in}}%
\pgfpathlineto{\pgfqpoint{3.969597in}{1.887630in}}%
\pgfpathlineto{\pgfqpoint{3.978404in}{2.512146in}}%
\pgfpathlineto{\pgfqpoint{3.996018in}{2.148405in}}%
\pgfpathlineto{\pgfqpoint{4.004825in}{2.066071in}}%
\pgfpathlineto{\pgfqpoint{4.013632in}{2.004286in}}%
\pgfpathlineto{\pgfqpoint{4.022438in}{2.169009in}}%
\pgfpathlineto{\pgfqpoint{4.031245in}{2.210190in}}%
\pgfpathlineto{\pgfqpoint{4.040052in}{2.409207in}}%
\pgfpathlineto{\pgfqpoint{4.048859in}{2.436643in}}%
\pgfpathlineto{\pgfqpoint{4.057666in}{2.422925in}}%
\pgfpathlineto{\pgfqpoint{4.066472in}{2.470965in}}%
\pgfpathlineto{\pgfqpoint{4.075279in}{2.429784in}}%
\pgfpathlineto{\pgfqpoint{4.084086in}{2.553326in}}%
\pgfpathlineto{\pgfqpoint{4.092893in}{2.519004in}}%
\pgfpathlineto{\pgfqpoint{4.101700in}{1.935670in}}%
\pgfpathlineto{\pgfqpoint{4.110507in}{1.770975in}}%
\pgfpathlineto{\pgfqpoint{4.128120in}{1.654291in}}%
\pgfpathlineto{\pgfqpoint{4.136927in}{1.537636in}}%
\pgfpathlineto{\pgfqpoint{4.145734in}{1.736653in}}%
\pgfpathlineto{\pgfqpoint{4.154541in}{1.839591in}}%
\pgfpathlineto{\pgfqpoint{4.163347in}{2.004286in}}%
\pgfpathlineto{\pgfqpoint{4.172154in}{2.100365in}}%
\pgfpathlineto{\pgfqpoint{4.180961in}{2.169009in}}%
\pgfpathlineto{\pgfqpoint{4.189768in}{1.949388in}}%
\pgfpathlineto{\pgfqpoint{4.198575in}{1.915094in}}%
\pgfpathlineto{\pgfqpoint{4.207382in}{1.647432in}}%
\pgfpathlineto{\pgfqpoint{4.216188in}{1.681754in}}%
\pgfpathlineto{\pgfqpoint{4.224995in}{1.935670in}}%
\pgfpathlineto{\pgfqpoint{4.233802in}{1.805269in}}%
\pgfpathlineto{\pgfqpoint{4.242609in}{2.299383in}}%
\pgfpathlineto{\pgfqpoint{4.251416in}{2.072930in}}%
\pgfpathlineto{\pgfqpoint{4.260222in}{1.729794in}}%
\pgfpathlineto{\pgfqpoint{4.269029in}{1.839591in}}%
\pgfpathlineto{\pgfqpoint{4.277836in}{2.347422in}}%
\pgfpathlineto{\pgfqpoint{4.286643in}{2.271947in}}%
\pgfpathlineto{\pgfqpoint{4.295450in}{2.079789in}}%
\pgfpathlineto{\pgfqpoint{4.304257in}{1.956246in}}%
\pgfpathlineto{\pgfqpoint{4.313063in}{2.313128in}}%
\pgfpathlineto{\pgfqpoint{4.321870in}{2.223908in}}%
\pgfpathlineto{\pgfqpoint{4.330677in}{2.162151in}}%
\pgfpathlineto{\pgfqpoint{4.339484in}{2.573903in}}%
\pgfpathlineto{\pgfqpoint{4.348291in}{2.018032in}}%
\pgfpathlineto{\pgfqpoint{4.357097in}{1.681754in}}%
\pgfpathlineto{\pgfqpoint{4.365904in}{1.729794in}}%
\pgfpathlineto{\pgfqpoint{4.374711in}{1.743512in}}%
\pgfpathlineto{\pgfqpoint{4.383518in}{1.695472in}}%
\pgfpathlineto{\pgfqpoint{4.392325in}{1.798410in}}%
\pgfpathlineto{\pgfqpoint{4.401132in}{1.722935in}}%
\pgfpathlineto{\pgfqpoint{4.409938in}{2.038608in}}%
\pgfpathlineto{\pgfqpoint{4.418745in}{2.525863in}}%
\pgfpathlineto{\pgfqpoint{4.427552in}{2.127829in}}%
\pgfpathlineto{\pgfqpoint{4.436359in}{2.436643in}}%
\pgfpathlineto{\pgfqpoint{4.445166in}{2.319987in}}%
\pgfpathlineto{\pgfqpoint{4.453972in}{2.285665in}}%
\pgfpathlineto{\pgfqpoint{4.462779in}{2.203303in}}%
\pgfpathlineto{\pgfqpoint{4.471586in}{2.230767in}}%
\pgfpathlineto{\pgfqpoint{4.480393in}{2.045467in}}%
\pgfpathlineto{\pgfqpoint{4.489200in}{1.894489in}}%
\pgfpathlineto{\pgfqpoint{4.498007in}{1.887630in}}%
\pgfpathlineto{\pgfqpoint{4.506813in}{2.120970in}}%
\pgfpathlineto{\pgfqpoint{4.515620in}{2.244484in}}%
\pgfpathlineto{\pgfqpoint{4.524427in}{1.949388in}}%
\pgfpathlineto{\pgfqpoint{4.533234in}{2.237625in}}%
\pgfpathlineto{\pgfqpoint{4.550847in}{1.819015in}}%
\pgfpathlineto{\pgfqpoint{4.559654in}{2.072930in}}%
\pgfpathlineto{\pgfqpoint{4.568461in}{1.805269in}}%
\pgfpathlineto{\pgfqpoint{4.577268in}{2.141546in}}%
\pgfpathlineto{\pgfqpoint{4.586075in}{2.004286in}}%
\pgfpathlineto{\pgfqpoint{4.594882in}{1.619997in}}%
\pgfpathlineto{\pgfqpoint{4.603688in}{1.475878in}}%
\pgfpathlineto{\pgfqpoint{4.612495in}{1.633715in}}%
\pgfpathlineto{\pgfqpoint{4.621302in}{1.558212in}}%
\pgfpathlineto{\pgfqpoint{4.630109in}{2.004286in}}%
\pgfpathlineto{\pgfqpoint{4.638916in}{2.271947in}}%
\pgfpathlineto{\pgfqpoint{4.647722in}{2.347422in}}%
\pgfpathlineto{\pgfqpoint{4.656529in}{2.031749in}}%
\pgfpathlineto{\pgfqpoint{4.665336in}{1.791551in}}%
\pgfpathlineto{\pgfqpoint{4.674143in}{1.839591in}}%
\pgfpathlineto{\pgfqpoint{4.682950in}{2.004286in}}%
\pgfpathlineto{\pgfqpoint{4.691757in}{1.784692in}}%
\pgfpathlineto{\pgfqpoint{4.700563in}{1.873913in}}%
\pgfpathlineto{\pgfqpoint{4.709370in}{1.928811in}}%
\pgfpathlineto{\pgfqpoint{4.718177in}{1.558212in}}%
\pgfpathlineto{\pgfqpoint{4.726984in}{1.743512in}}%
\pgfpathlineto{\pgfqpoint{4.735791in}{1.647432in}}%
\pgfpathlineto{\pgfqpoint{4.744597in}{1.757229in}}%
\pgfpathlineto{\pgfqpoint{4.753404in}{1.825873in}}%
\pgfpathlineto{\pgfqpoint{4.762211in}{1.819015in}}%
\pgfpathlineto{\pgfqpoint{4.771018in}{1.887630in}}%
\pgfpathlineto{\pgfqpoint{4.779825in}{2.066071in}}%
\pgfpathlineto{\pgfqpoint{4.788632in}{2.285665in}}%
\pgfpathlineto{\pgfqpoint{4.797438in}{2.134687in}}%
\pgfpathlineto{\pgfqpoint{4.806245in}{2.045467in}}%
\pgfpathlineto{\pgfqpoint{4.815052in}{1.654291in}}%
\pgfpathlineto{\pgfqpoint{4.823859in}{1.736653in}}%
\pgfpathlineto{\pgfqpoint{4.832666in}{1.853308in}}%
\pgfpathlineto{\pgfqpoint{4.841472in}{1.997427in}}%
\pgfpathlineto{\pgfqpoint{4.850279in}{1.750370in}}%
\pgfpathlineto{\pgfqpoint{4.859086in}{2.175868in}}%
\pgfpathlineto{\pgfqpoint{4.867893in}{2.162151in}}%
\pgfpathlineto{\pgfqpoint{4.876700in}{2.127829in}}%
\pgfpathlineto{\pgfqpoint{4.885507in}{2.086648in}}%
\pgfpathlineto{\pgfqpoint{4.894313in}{2.299383in}}%
\pgfpathlineto{\pgfqpoint{4.903120in}{1.777834in}}%
\pgfpathlineto{\pgfqpoint{4.911927in}{1.798410in}}%
\pgfpathlineto{\pgfqpoint{4.920734in}{1.798410in}}%
\pgfpathlineto{\pgfqpoint{4.938347in}{2.237625in}}%
\pgfpathlineto{\pgfqpoint{4.947154in}{2.223908in}}%
\pgfpathlineto{\pgfqpoint{4.955961in}{2.333705in}}%
\pgfpathlineto{\pgfqpoint{4.964768in}{2.175868in}}%
\pgfpathlineto{\pgfqpoint{4.973575in}{2.299383in}}%
\pgfpathlineto{\pgfqpoint{4.982382in}{2.210190in}}%
\pgfpathlineto{\pgfqpoint{4.991188in}{2.093507in}}%
\pgfpathlineto{\pgfqpoint{4.999995in}{2.148405in}}%
\pgfpathlineto{\pgfqpoint{5.008802in}{2.223908in}}%
\pgfpathlineto{\pgfqpoint{5.017609in}{2.395462in}}%
\pgfpathlineto{\pgfqpoint{5.026416in}{2.079789in}}%
\pgfpathlineto{\pgfqpoint{5.035222in}{2.141546in}}%
\pgfpathlineto{\pgfqpoint{5.044029in}{1.853308in}}%
\pgfpathlineto{\pgfqpoint{5.052836in}{2.162151in}}%
\pgfpathlineto{\pgfqpoint{5.070450in}{1.606252in}}%
\pgfpathlineto{\pgfqpoint{5.079257in}{2.079789in}}%
\pgfpathlineto{\pgfqpoint{5.088063in}{1.853308in}}%
\pgfpathlineto{\pgfqpoint{5.096870in}{1.784692in}}%
\pgfpathlineto{\pgfqpoint{5.105677in}{2.120970in}}%
\pgfpathlineto{\pgfqpoint{5.114484in}{2.306269in}}%
\pgfpathlineto{\pgfqpoint{5.132097in}{1.928811in}}%
\pgfpathlineto{\pgfqpoint{5.140904in}{2.059213in}}%
\pgfpathlineto{\pgfqpoint{5.158518in}{2.189586in}}%
\pgfpathlineto{\pgfqpoint{5.167325in}{2.134687in}}%
\pgfpathlineto{\pgfqpoint{5.176132in}{2.326846in}}%
\pgfpathlineto{\pgfqpoint{5.184938in}{2.258230in}}%
\pgfpathlineto{\pgfqpoint{5.193745in}{2.319987in}}%
\pgfpathlineto{\pgfqpoint{5.202552in}{2.127829in}}%
\pgfpathlineto{\pgfqpoint{5.211359in}{2.484682in}}%
\pgfpathlineto{\pgfqpoint{5.220166in}{2.573903in}}%
\pgfpathlineto{\pgfqpoint{5.228972in}{2.745485in}}%
\pgfpathlineto{\pgfqpoint{5.237779in}{2.223908in}}%
\pgfpathlineto{\pgfqpoint{5.246586in}{1.956246in}}%
\pgfpathlineto{\pgfqpoint{5.255393in}{1.599393in}}%
\pgfpathlineto{\pgfqpoint{5.264200in}{1.318014in}}%
\pgfpathlineto{\pgfqpoint{5.273007in}{1.633715in}}%
\pgfpathlineto{\pgfqpoint{5.281813in}{1.867054in}}%
\pgfpathlineto{\pgfqpoint{5.290620in}{1.908207in}}%
\pgfpathlineto{\pgfqpoint{5.299427in}{2.038608in}}%
\pgfpathlineto{\pgfqpoint{5.308234in}{2.237625in}}%
\pgfpathlineto{\pgfqpoint{5.317041in}{1.949388in}}%
\pgfpathlineto{\pgfqpoint{5.325847in}{2.100365in}}%
\pgfpathlineto{\pgfqpoint{5.334654in}{2.114111in}}%
\pgfpathlineto{\pgfqpoint{5.343461in}{1.805269in}}%
\pgfpathlineto{\pgfqpoint{5.352268in}{1.784692in}}%
\pgfpathlineto{\pgfqpoint{5.361075in}{1.784692in}}%
\pgfpathlineto{\pgfqpoint{5.369882in}{1.915094in}}%
\pgfpathlineto{\pgfqpoint{5.378688in}{1.887630in}}%
\pgfpathlineto{\pgfqpoint{5.387495in}{1.901348in}}%
\pgfpathlineto{\pgfqpoint{5.396302in}{2.127829in}}%
\pgfpathlineto{\pgfqpoint{5.405109in}{1.819015in}}%
\pgfpathlineto{\pgfqpoint{5.413916in}{2.072930in}}%
\pgfpathlineto{\pgfqpoint{5.422722in}{2.189586in}}%
\pgfpathlineto{\pgfqpoint{5.431529in}{2.223908in}}%
\pgfpathlineto{\pgfqpoint{5.440336in}{1.880772in}}%
\pgfpathlineto{\pgfqpoint{5.449143in}{1.784692in}}%
\pgfpathlineto{\pgfqpoint{5.457950in}{1.510172in}}%
\pgfpathlineto{\pgfqpoint{5.475563in}{1.846450in}}%
\pgfpathlineto{\pgfqpoint{5.484370in}{1.963133in}}%
\pgfpathlineto{\pgfqpoint{5.493177in}{1.935670in}}%
\pgfpathlineto{\pgfqpoint{5.501984in}{1.894489in}}%
\pgfpathlineto{\pgfqpoint{5.510791in}{1.757229in}}%
\pgfpathlineto{\pgfqpoint{5.519597in}{1.832732in}}%
\pgfpathlineto{\pgfqpoint{5.528404in}{1.928811in}}%
\pgfpathlineto{\pgfqpoint{5.537211in}{2.004286in}}%
\pgfpathlineto{\pgfqpoint{5.546018in}{2.066071in}}%
\pgfpathlineto{\pgfqpoint{5.554825in}{2.100365in}}%
\pgfpathlineto{\pgfqpoint{5.563632in}{1.784692in}}%
\pgfpathlineto{\pgfqpoint{5.572438in}{1.819015in}}%
\pgfpathlineto{\pgfqpoint{5.581245in}{1.626856in}}%
\pgfpathlineto{\pgfqpoint{5.590052in}{1.777834in}}%
\pgfpathlineto{\pgfqpoint{5.598859in}{2.004286in}}%
\pgfpathlineto{\pgfqpoint{5.607666in}{1.798410in}}%
\pgfpathlineto{\pgfqpoint{5.616472in}{1.647432in}}%
\pgfpathlineto{\pgfqpoint{5.625279in}{1.750370in}}%
\pgfpathlineto{\pgfqpoint{5.642893in}{2.381744in}}%
\pgfpathlineto{\pgfqpoint{5.660507in}{1.654291in}}%
\pgfpathlineto{\pgfqpoint{5.669313in}{1.510172in}}%
\pgfpathlineto{\pgfqpoint{5.678120in}{1.757229in}}%
\pgfpathlineto{\pgfqpoint{5.686927in}{1.812128in}}%
\pgfpathlineto{\pgfqpoint{5.695734in}{1.963133in}}%
\pgfpathlineto{\pgfqpoint{5.704541in}{2.004286in}}%
\pgfpathlineto{\pgfqpoint{5.713347in}{1.901348in}}%
\pgfpathlineto{\pgfqpoint{5.722154in}{2.086648in}}%
\pgfpathlineto{\pgfqpoint{5.730961in}{1.921953in}}%
\pgfpathlineto{\pgfqpoint{5.739768in}{1.791551in}}%
\pgfpathlineto{\pgfqpoint{5.748575in}{1.784692in}}%
\pgfpathlineto{\pgfqpoint{5.757382in}{1.784692in}}%
\pgfpathlineto{\pgfqpoint{5.766188in}{1.503314in}}%
\pgfpathlineto{\pgfqpoint{5.774995in}{1.530777in}}%
\pgfpathlineto{\pgfqpoint{5.783802in}{1.743512in}}%
\pgfpathlineto{\pgfqpoint{5.792609in}{1.825873in}}%
\pgfpathlineto{\pgfqpoint{5.801416in}{1.825873in}}%
\pgfpathlineto{\pgfqpoint{5.810222in}{1.784692in}}%
\pgfpathlineto{\pgfqpoint{5.819029in}{1.695472in}}%
\pgfpathlineto{\pgfqpoint{5.827836in}{1.819015in}}%
\pgfpathlineto{\pgfqpoint{5.836643in}{1.757229in}}%
\pgfpathlineto{\pgfqpoint{5.845450in}{1.839591in}}%
\pgfpathlineto{\pgfqpoint{5.854257in}{2.313128in}}%
\pgfpathlineto{\pgfqpoint{5.863063in}{2.388603in}}%
\pgfpathlineto{\pgfqpoint{5.871870in}{2.374885in}}%
\pgfpathlineto{\pgfqpoint{5.880677in}{2.381744in}}%
\pgfpathlineto{\pgfqpoint{5.889484in}{2.079789in}}%
\pgfpathlineto{\pgfqpoint{5.898291in}{2.114111in}}%
\pgfpathlineto{\pgfqpoint{5.907097in}{1.867054in}}%
\pgfpathlineto{\pgfqpoint{5.915904in}{1.805269in}}%
\pgfpathlineto{\pgfqpoint{5.924711in}{1.654291in}}%
\pgfpathlineto{\pgfqpoint{5.933518in}{1.750370in}}%
\pgfpathlineto{\pgfqpoint{5.942325in}{1.873913in}}%
\pgfpathlineto{\pgfqpoint{5.951132in}{2.018032in}}%
\pgfpathlineto{\pgfqpoint{5.959938in}{1.681754in}}%
\pgfpathlineto{\pgfqpoint{5.968745in}{1.640574in}}%
\pgfpathlineto{\pgfqpoint{5.977552in}{1.647432in}}%
\pgfpathlineto{\pgfqpoint{5.986359in}{1.963133in}}%
\pgfpathlineto{\pgfqpoint{6.003972in}{1.825873in}}%
\pgfpathlineto{\pgfqpoint{6.012779in}{1.928811in}}%
\pgfpathlineto{\pgfqpoint{6.021586in}{1.414093in}}%
\pgfpathlineto{\pgfqpoint{6.030393in}{1.722935in}}%
\pgfpathlineto{\pgfqpoint{6.039200in}{1.544494in}}%
\pgfpathlineto{\pgfqpoint{6.048007in}{1.668037in}}%
\pgfpathlineto{\pgfqpoint{6.056813in}{1.551353in}}%
\pgfpathlineto{\pgfqpoint{6.065620in}{1.475878in}}%
\pgfpathlineto{\pgfqpoint{6.074427in}{1.537636in}}%
\pgfpathlineto{\pgfqpoint{6.083234in}{1.812128in}}%
\pgfpathlineto{\pgfqpoint{6.092041in}{1.613110in}}%
\pgfpathlineto{\pgfqpoint{6.100847in}{2.155292in}}%
\pgfpathlineto{\pgfqpoint{6.109654in}{1.681754in}}%
\pgfpathlineto{\pgfqpoint{6.127268in}{2.182727in}}%
\pgfpathlineto{\pgfqpoint{6.136075in}{2.072930in}}%
\pgfpathlineto{\pgfqpoint{6.144882in}{2.086648in}}%
\pgfpathlineto{\pgfqpoint{6.153688in}{1.880772in}}%
\pgfpathlineto{\pgfqpoint{6.162495in}{1.709190in}}%
\pgfpathlineto{\pgfqpoint{6.171302in}{1.915094in}}%
\pgfpathlineto{\pgfqpoint{6.180109in}{2.155292in}}%
\pgfpathlineto{\pgfqpoint{6.188916in}{2.100365in}}%
\pgfpathlineto{\pgfqpoint{6.197722in}{1.935670in}}%
\pgfpathlineto{\pgfqpoint{6.206529in}{1.860167in}}%
\pgfpathlineto{\pgfqpoint{6.215336in}{1.969992in}}%
\pgfpathlineto{\pgfqpoint{6.224143in}{2.210190in}}%
\pgfpathlineto{\pgfqpoint{6.232950in}{2.271947in}}%
\pgfpathlineto{\pgfqpoint{6.241757in}{2.210190in}}%
\pgfpathlineto{\pgfqpoint{6.250563in}{2.114111in}}%
\pgfpathlineto{\pgfqpoint{6.259370in}{2.120970in}}%
\pgfpathlineto{\pgfqpoint{6.268177in}{1.880772in}}%
\pgfpathlineto{\pgfqpoint{6.276984in}{1.880772in}}%
\pgfpathlineto{\pgfqpoint{6.285791in}{1.530777in}}%
\pgfpathlineto{\pgfqpoint{6.294597in}{1.702331in}}%
\pgfpathlineto{\pgfqpoint{6.303404in}{1.798410in}}%
\pgfpathlineto{\pgfqpoint{6.312211in}{1.585675in}}%
\pgfpathlineto{\pgfqpoint{6.321018in}{1.819015in}}%
\pgfpathlineto{\pgfqpoint{6.329825in}{2.182727in}}%
\pgfpathlineto{\pgfqpoint{6.338632in}{2.409207in}}%
\pgfpathlineto{\pgfqpoint{6.347438in}{2.450388in}}%
\pgfpathlineto{\pgfqpoint{6.356245in}{2.244484in}}%
\pgfpathlineto{\pgfqpoint{6.365052in}{2.210190in}}%
\pgfpathlineto{\pgfqpoint{6.373859in}{2.539581in}}%
\pgfpathlineto{\pgfqpoint{6.382666in}{2.038608in}}%
\pgfpathlineto{\pgfqpoint{6.391472in}{2.217049in}}%
\pgfpathlineto{\pgfqpoint{6.400279in}{2.155292in}}%
\pgfpathlineto{\pgfqpoint{6.409086in}{1.949388in}}%
\pgfpathlineto{\pgfqpoint{6.417893in}{1.592534in}}%
\pgfpathlineto{\pgfqpoint{6.426700in}{1.695472in}}%
\pgfpathlineto{\pgfqpoint{6.435507in}{1.585675in}}%
\pgfpathlineto{\pgfqpoint{6.444313in}{1.942529in}}%
\pgfpathlineto{\pgfqpoint{6.453120in}{1.997427in}}%
\pgfpathlineto{\pgfqpoint{6.461927in}{1.983710in}}%
\pgfpathlineto{\pgfqpoint{6.470734in}{1.942529in}}%
\pgfpathlineto{\pgfqpoint{6.479541in}{1.976851in}}%
\pgfpathlineto{\pgfqpoint{6.488347in}{1.949388in}}%
\pgfpathlineto{\pgfqpoint{6.497154in}{2.175868in}}%
\pgfpathlineto{\pgfqpoint{6.505961in}{2.155292in}}%
\pgfpathlineto{\pgfqpoint{6.514768in}{1.908207in}}%
\pgfpathlineto{\pgfqpoint{6.523575in}{2.024891in}}%
\pgfpathlineto{\pgfqpoint{6.541188in}{1.819015in}}%
\pgfpathlineto{\pgfqpoint{6.549995in}{1.805269in}}%
\pgfpathlineto{\pgfqpoint{6.558802in}{2.024891in}}%
\pgfpathlineto{\pgfqpoint{6.567609in}{2.018032in}}%
\pgfpathlineto{\pgfqpoint{6.576416in}{1.599393in}}%
\pgfpathlineto{\pgfqpoint{6.585222in}{1.839591in}}%
\pgfpathlineto{\pgfqpoint{6.594029in}{1.716076in}}%
\pgfpathlineto{\pgfqpoint{6.602836in}{1.743512in}}%
\pgfpathlineto{\pgfqpoint{6.611643in}{2.038608in}}%
\pgfpathlineto{\pgfqpoint{6.620450in}{1.750370in}}%
\pgfpathlineto{\pgfqpoint{6.629257in}{1.935670in}}%
\pgfpathlineto{\pgfqpoint{6.638063in}{1.791551in}}%
\pgfpathlineto{\pgfqpoint{6.646870in}{1.764116in}}%
\pgfpathlineto{\pgfqpoint{6.655677in}{1.770975in}}%
\pgfpathlineto{\pgfqpoint{6.673291in}{2.532722in}}%
\pgfpathlineto{\pgfqpoint{6.682097in}{2.223908in}}%
\pgfpathlineto{\pgfqpoint{6.690904in}{2.141546in}}%
\pgfpathlineto{\pgfqpoint{6.699711in}{2.018032in}}%
\pgfpathlineto{\pgfqpoint{6.708518in}{1.743512in}}%
\pgfpathlineto{\pgfqpoint{6.717325in}{1.770975in}}%
\pgfpathlineto{\pgfqpoint{6.726132in}{1.716076in}}%
\pgfpathlineto{\pgfqpoint{6.734938in}{1.873913in}}%
\pgfpathlineto{\pgfqpoint{6.743745in}{2.271947in}}%
\pgfpathlineto{\pgfqpoint{6.752552in}{2.333705in}}%
\pgfpathlineto{\pgfqpoint{6.761359in}{2.512146in}}%
\pgfpathlineto{\pgfqpoint{6.770166in}{2.416066in}}%
\pgfpathlineto{\pgfqpoint{6.778972in}{2.347422in}}%
\pgfpathlineto{\pgfqpoint{6.787779in}{1.860167in}}%
\pgfpathlineto{\pgfqpoint{6.796586in}{1.496455in}}%
\pgfpathlineto{\pgfqpoint{6.805393in}{1.757229in}}%
\pgfpathlineto{\pgfqpoint{6.814200in}{1.613110in}}%
\pgfpathlineto{\pgfqpoint{6.831813in}{2.004286in}}%
\pgfpathlineto{\pgfqpoint{6.840620in}{1.702331in}}%
\pgfpathlineto{\pgfqpoint{6.849427in}{1.825873in}}%
\pgfpathlineto{\pgfqpoint{6.858234in}{1.983710in}}%
\pgfpathlineto{\pgfqpoint{6.867041in}{2.086648in}}%
\pgfpathlineto{\pgfqpoint{6.875847in}{2.086648in}}%
\pgfpathlineto{\pgfqpoint{6.884654in}{2.429784in}}%
\pgfpathlineto{\pgfqpoint{6.893461in}{2.004286in}}%
\pgfpathlineto{\pgfqpoint{6.902268in}{2.223908in}}%
\pgfpathlineto{\pgfqpoint{6.911075in}{2.038608in}}%
\pgfpathlineto{\pgfqpoint{6.919882in}{1.757229in}}%
\pgfpathlineto{\pgfqpoint{6.928688in}{1.963133in}}%
\pgfpathlineto{\pgfqpoint{6.937495in}{2.100365in}}%
\pgfpathlineto{\pgfqpoint{6.946302in}{1.867054in}}%
\pgfpathlineto{\pgfqpoint{6.955109in}{1.880772in}}%
\pgfpathlineto{\pgfqpoint{6.963916in}{2.045467in}}%
\pgfpathlineto{\pgfqpoint{6.972722in}{2.148405in}}%
\pgfpathlineto{\pgfqpoint{6.981529in}{2.045467in}}%
\pgfpathlineto{\pgfqpoint{6.990336in}{2.120970in}}%
\pgfpathlineto{\pgfqpoint{6.999143in}{1.736653in}}%
\pgfpathlineto{\pgfqpoint{7.007950in}{1.640574in}}%
\pgfpathlineto{\pgfqpoint{7.016757in}{1.626856in}}%
\pgfpathlineto{\pgfqpoint{7.025563in}{1.846450in}}%
\pgfpathlineto{\pgfqpoint{7.034370in}{2.141546in}}%
\pgfpathlineto{\pgfqpoint{7.043177in}{2.066071in}}%
\pgfpathlineto{\pgfqpoint{7.051984in}{2.182727in}}%
\pgfpathlineto{\pgfqpoint{7.060791in}{2.203303in}}%
\pgfpathlineto{\pgfqpoint{7.069597in}{2.409207in}}%
\pgfpathlineto{\pgfqpoint{7.078404in}{2.059213in}}%
\pgfpathlineto{\pgfqpoint{7.087211in}{2.162151in}}%
\pgfpathlineto{\pgfqpoint{7.096018in}{1.853308in}}%
\pgfpathlineto{\pgfqpoint{7.104825in}{1.825873in}}%
\pgfpathlineto{\pgfqpoint{7.113632in}{1.887630in}}%
\pgfpathlineto{\pgfqpoint{7.122438in}{2.038608in}}%
\pgfpathlineto{\pgfqpoint{7.131245in}{1.908207in}}%
\pgfpathlineto{\pgfqpoint{7.140052in}{1.462133in}}%
\pgfpathlineto{\pgfqpoint{7.148859in}{1.695472in}}%
\pgfpathlineto{\pgfqpoint{7.157666in}{1.997427in}}%
\pgfpathlineto{\pgfqpoint{7.175279in}{2.313128in}}%
\pgfpathlineto{\pgfqpoint{7.184086in}{2.230767in}}%
\pgfpathlineto{\pgfqpoint{7.192893in}{2.066071in}}%
\pgfpathlineto{\pgfqpoint{7.201700in}{2.066071in}}%
\pgfpathlineto{\pgfqpoint{7.210507in}{1.784692in}}%
\pgfpathlineto{\pgfqpoint{7.219313in}{1.764116in}}%
\pgfpathlineto{\pgfqpoint{7.228120in}{1.654291in}}%
\pgfpathlineto{\pgfqpoint{7.236927in}{2.018032in}}%
\pgfpathlineto{\pgfqpoint{7.254541in}{1.839591in}}%
\pgfpathlineto{\pgfqpoint{7.263347in}{1.496455in}}%
\pgfpathlineto{\pgfqpoint{7.272154in}{1.558212in}}%
\pgfpathlineto{\pgfqpoint{7.280961in}{1.805269in}}%
\pgfpathlineto{\pgfqpoint{7.289768in}{1.928811in}}%
\pgfpathlineto{\pgfqpoint{7.298575in}{1.777834in}}%
\pgfpathlineto{\pgfqpoint{7.307382in}{1.997427in}}%
\pgfpathlineto{\pgfqpoint{7.316188in}{1.873913in}}%
\pgfpathlineto{\pgfqpoint{7.324995in}{1.798410in}}%
\pgfpathlineto{\pgfqpoint{7.333802in}{1.860167in}}%
\pgfpathlineto{\pgfqpoint{7.342609in}{1.791551in}}%
\pgfpathlineto{\pgfqpoint{7.351416in}{1.839591in}}%
\pgfpathlineto{\pgfqpoint{7.360222in}{1.983710in}}%
\pgfpathlineto{\pgfqpoint{7.369029in}{1.770975in}}%
\pgfpathlineto{\pgfqpoint{7.377836in}{1.770975in}}%
\pgfpathlineto{\pgfqpoint{7.395450in}{2.107252in}}%
\pgfpathlineto{\pgfqpoint{7.404257in}{1.619997in}}%
\pgfpathlineto{\pgfqpoint{7.413063in}{1.867054in}}%
\pgfpathlineto{\pgfqpoint{7.421870in}{1.990569in}}%
\pgfpathlineto{\pgfqpoint{7.430677in}{2.196445in}}%
\pgfpathlineto{\pgfqpoint{7.439484in}{2.347422in}}%
\pgfpathlineto{\pgfqpoint{7.448291in}{2.162151in}}%
\pgfpathlineto{\pgfqpoint{7.457097in}{2.162151in}}%
\pgfpathlineto{\pgfqpoint{7.465904in}{2.079789in}}%
\pgfpathlineto{\pgfqpoint{7.474711in}{1.956246in}}%
\pgfpathlineto{\pgfqpoint{7.483518in}{2.011173in}}%
\pgfpathlineto{\pgfqpoint{7.492325in}{1.963133in}}%
\pgfpathlineto{\pgfqpoint{7.501132in}{1.729794in}}%
\pgfpathlineto{\pgfqpoint{7.509938in}{1.633715in}}%
\pgfpathlineto{\pgfqpoint{7.518745in}{1.832732in}}%
\pgfpathlineto{\pgfqpoint{7.527552in}{1.681754in}}%
\pgfpathlineto{\pgfqpoint{7.536359in}{1.702331in}}%
\pgfpathlineto{\pgfqpoint{7.545166in}{1.585675in}}%
\pgfpathlineto{\pgfqpoint{7.553972in}{1.681754in}}%
\pgfpathlineto{\pgfqpoint{7.562779in}{1.853308in}}%
\pgfpathlineto{\pgfqpoint{7.571586in}{1.880772in}}%
\pgfpathlineto{\pgfqpoint{7.580393in}{1.963133in}}%
\pgfpathlineto{\pgfqpoint{7.589200in}{1.928811in}}%
\pgfpathlineto{\pgfqpoint{7.598007in}{2.045467in}}%
\pgfpathlineto{\pgfqpoint{7.606813in}{1.956246in}}%
\pgfpathlineto{\pgfqpoint{7.615620in}{1.990569in}}%
\pgfpathlineto{\pgfqpoint{7.624427in}{2.292524in}}%
\pgfpathlineto{\pgfqpoint{7.633234in}{2.217049in}}%
\pgfpathlineto{\pgfqpoint{7.642041in}{1.921953in}}%
\pgfpathlineto{\pgfqpoint{7.650847in}{1.777834in}}%
\pgfpathlineto{\pgfqpoint{7.659654in}{2.134687in}}%
\pgfpathlineto{\pgfqpoint{7.668461in}{1.867054in}}%
\pgfpathlineto{\pgfqpoint{7.677268in}{1.928811in}}%
\pgfpathlineto{\pgfqpoint{7.686075in}{2.258230in}}%
\pgfpathlineto{\pgfqpoint{7.694882in}{2.347422in}}%
\pgfpathlineto{\pgfqpoint{7.703688in}{2.210190in}}%
\pgfpathlineto{\pgfqpoint{7.712495in}{2.024891in}}%
\pgfpathlineto{\pgfqpoint{7.730109in}{2.505287in}}%
\pgfpathlineto{\pgfqpoint{7.738916in}{2.567044in}}%
\pgfpathlineto{\pgfqpoint{7.747722in}{2.711163in}}%
\pgfpathlineto{\pgfqpoint{7.756529in}{2.587620in}}%
\pgfpathlineto{\pgfqpoint{7.765336in}{2.340563in}}%
\pgfpathlineto{\pgfqpoint{7.774143in}{2.162151in}}%
\pgfpathlineto{\pgfqpoint{7.782950in}{2.217049in}}%
\pgfpathlineto{\pgfqpoint{7.800563in}{1.722935in}}%
\pgfpathlineto{\pgfqpoint{7.809370in}{1.729794in}}%
\pgfpathlineto{\pgfqpoint{7.818177in}{1.599393in}}%
\pgfpathlineto{\pgfqpoint{7.826984in}{1.599393in}}%
\pgfpathlineto{\pgfqpoint{7.835791in}{1.647432in}}%
\pgfpathlineto{\pgfqpoint{7.844597in}{1.125855in}}%
\pgfpathlineto{\pgfqpoint{7.853404in}{1.661150in}}%
\pgfpathlineto{\pgfqpoint{7.871018in}{1.777834in}}%
\pgfpathlineto{\pgfqpoint{7.879825in}{1.867054in}}%
\pgfpathlineto{\pgfqpoint{7.888632in}{1.619997in}}%
\pgfpathlineto{\pgfqpoint{7.897438in}{1.475878in}}%
\pgfpathlineto{\pgfqpoint{7.906245in}{1.619997in}}%
\pgfpathlineto{\pgfqpoint{7.915052in}{1.647432in}}%
\pgfpathlineto{\pgfqpoint{7.923859in}{1.420952in}}%
\pgfpathlineto{\pgfqpoint{7.941472in}{1.729794in}}%
\pgfpathlineto{\pgfqpoint{7.950279in}{1.729794in}}%
\pgfpathlineto{\pgfqpoint{7.959086in}{1.551353in}}%
\pgfpathlineto{\pgfqpoint{7.967893in}{1.901348in}}%
\pgfpathlineto{\pgfqpoint{7.976700in}{1.942529in}}%
\pgfpathlineto{\pgfqpoint{7.985507in}{1.963133in}}%
\pgfpathlineto{\pgfqpoint{7.994313in}{1.887630in}}%
\pgfpathlineto{\pgfqpoint{8.003120in}{1.832732in}}%
\pgfpathlineto{\pgfqpoint{8.011927in}{2.203303in}}%
\pgfpathlineto{\pgfqpoint{8.020734in}{2.052326in}}%
\pgfpathlineto{\pgfqpoint{8.029541in}{2.237625in}}%
\pgfpathlineto{\pgfqpoint{8.038347in}{2.018032in}}%
\pgfpathlineto{\pgfqpoint{8.047154in}{1.901348in}}%
\pgfpathlineto{\pgfqpoint{8.055961in}{1.867054in}}%
\pgfpathlineto{\pgfqpoint{8.064768in}{1.819015in}}%
\pgfpathlineto{\pgfqpoint{8.073575in}{1.963133in}}%
\pgfpathlineto{\pgfqpoint{8.082382in}{1.729794in}}%
\pgfpathlineto{\pgfqpoint{8.091188in}{2.018032in}}%
\pgfpathlineto{\pgfqpoint{8.099995in}{2.477823in}}%
\pgfpathlineto{\pgfqpoint{8.108802in}{2.477823in}}%
\pgfpathlineto{\pgfqpoint{8.117609in}{2.134687in}}%
\pgfpathlineto{\pgfqpoint{8.126416in}{2.072930in}}%
\pgfpathlineto{\pgfqpoint{8.135222in}{1.956246in}}%
\pgfpathlineto{\pgfqpoint{8.144029in}{1.860167in}}%
\pgfpathlineto{\pgfqpoint{8.152836in}{1.544494in}}%
\pgfpathlineto{\pgfqpoint{8.161643in}{1.578816in}}%
\pgfpathlineto{\pgfqpoint{8.170450in}{2.079789in}}%
\pgfpathlineto{\pgfqpoint{8.179257in}{1.935670in}}%
\pgfpathlineto{\pgfqpoint{8.188063in}{1.935670in}}%
\pgfpathlineto{\pgfqpoint{8.196870in}{1.716076in}}%
\pgfpathlineto{\pgfqpoint{8.205677in}{1.729794in}}%
\pgfpathlineto{\pgfqpoint{8.214484in}{1.489596in}}%
\pgfpathlineto{\pgfqpoint{8.223291in}{1.523918in}}%
\pgfpathlineto{\pgfqpoint{8.232097in}{1.537636in}}%
\pgfpathlineto{\pgfqpoint{8.240904in}{1.578816in}}%
\pgfpathlineto{\pgfqpoint{8.249711in}{1.736653in}}%
\pgfpathlineto{\pgfqpoint{8.258518in}{1.640574in}}%
\pgfpathlineto{\pgfqpoint{8.267325in}{1.503314in}}%
\pgfpathlineto{\pgfqpoint{8.276132in}{1.221935in}}%
\pgfpathlineto{\pgfqpoint{8.284938in}{1.256257in}}%
\pgfpathlineto{\pgfqpoint{8.293745in}{1.160177in}}%
\pgfpathlineto{\pgfqpoint{8.302552in}{1.269974in}}%
\pgfpathlineto{\pgfqpoint{8.311359in}{1.434698in}}%
\pgfpathlineto{\pgfqpoint{8.320166in}{1.819015in}}%
\pgfpathlineto{\pgfqpoint{8.328972in}{1.928811in}}%
\pgfpathlineto{\pgfqpoint{8.337779in}{1.880772in}}%
\pgfpathlineto{\pgfqpoint{8.346586in}{1.407234in}}%
\pgfpathlineto{\pgfqpoint{8.355393in}{1.386658in}}%
\pgfpathlineto{\pgfqpoint{8.373007in}{1.915094in}}%
\pgfpathlineto{\pgfqpoint{8.381813in}{2.107252in}}%
\pgfpathlineto{\pgfqpoint{8.390620in}{2.436643in}}%
\pgfpathlineto{\pgfqpoint{8.399427in}{2.265089in}}%
\pgfpathlineto{\pgfqpoint{8.408234in}{2.189586in}}%
\pgfpathlineto{\pgfqpoint{8.417041in}{2.134687in}}%
\pgfpathlineto{\pgfqpoint{8.425847in}{1.908207in}}%
\pgfpathlineto{\pgfqpoint{8.434654in}{2.141546in}}%
\pgfpathlineto{\pgfqpoint{8.443461in}{2.525863in}}%
\pgfpathlineto{\pgfqpoint{8.452268in}{2.210190in}}%
\pgfpathlineto{\pgfqpoint{8.461075in}{1.791551in}}%
\pgfpathlineto{\pgfqpoint{8.469882in}{2.052326in}}%
\pgfpathlineto{\pgfqpoint{8.478688in}{2.450388in}}%
\pgfpathlineto{\pgfqpoint{8.487495in}{2.292524in}}%
\pgfpathlineto{\pgfqpoint{8.496302in}{2.292524in}}%
\pgfpathlineto{\pgfqpoint{8.505109in}{2.422925in}}%
\pgfpathlineto{\pgfqpoint{8.513916in}{2.086648in}}%
\pgfpathlineto{\pgfqpoint{8.522722in}{2.114111in}}%
\pgfpathlineto{\pgfqpoint{8.531529in}{1.860167in}}%
\pgfpathlineto{\pgfqpoint{8.540336in}{1.963133in}}%
\pgfpathlineto{\pgfqpoint{8.549143in}{2.347422in}}%
\pgfpathlineto{\pgfqpoint{8.557950in}{1.997427in}}%
\pgfpathlineto{\pgfqpoint{8.566757in}{1.750370in}}%
\pgfpathlineto{\pgfqpoint{8.575563in}{2.011173in}}%
\pgfpathlineto{\pgfqpoint{8.584370in}{1.750370in}}%
\pgfpathlineto{\pgfqpoint{8.593177in}{2.107252in}}%
\pgfpathlineto{\pgfqpoint{8.601984in}{2.230767in}}%
\pgfpathlineto{\pgfqpoint{8.610791in}{2.196445in}}%
\pgfpathlineto{\pgfqpoint{8.619597in}{2.155292in}}%
\pgfpathlineto{\pgfqpoint{8.628404in}{2.155292in}}%
\pgfpathlineto{\pgfqpoint{8.637211in}{2.066071in}}%
\pgfpathlineto{\pgfqpoint{8.646018in}{2.333705in}}%
\pgfpathlineto{\pgfqpoint{8.654825in}{2.271947in}}%
\pgfpathlineto{\pgfqpoint{8.663632in}{1.825873in}}%
\pgfpathlineto{\pgfqpoint{8.672438in}{1.757229in}}%
\pgfpathlineto{\pgfqpoint{8.681245in}{1.915094in}}%
\pgfpathlineto{\pgfqpoint{8.690052in}{1.997427in}}%
\pgfpathlineto{\pgfqpoint{8.698859in}{2.196445in}}%
\pgfpathlineto{\pgfqpoint{8.707666in}{2.313128in}}%
\pgfpathlineto{\pgfqpoint{8.716472in}{2.587620in}}%
\pgfpathlineto{\pgfqpoint{8.725279in}{2.368027in}}%
\pgfpathlineto{\pgfqpoint{8.734086in}{2.340563in}}%
\pgfpathlineto{\pgfqpoint{8.742893in}{2.292524in}}%
\pgfpathlineto{\pgfqpoint{8.751700in}{2.162151in}}%
\pgfpathlineto{\pgfqpoint{8.760507in}{1.839591in}}%
\pgfpathlineto{\pgfqpoint{8.769313in}{2.018032in}}%
\pgfpathlineto{\pgfqpoint{8.778120in}{1.736653in}}%
\pgfpathlineto{\pgfqpoint{8.786927in}{1.867054in}}%
\pgfpathlineto{\pgfqpoint{8.795734in}{1.956246in}}%
\pgfpathlineto{\pgfqpoint{8.804541in}{1.702331in}}%
\pgfpathlineto{\pgfqpoint{8.813347in}{1.791551in}}%
\pgfpathlineto{\pgfqpoint{8.822154in}{1.722935in}}%
\pgfpathlineto{\pgfqpoint{8.830961in}{1.517031in}}%
\pgfpathlineto{\pgfqpoint{8.839768in}{1.674896in}}%
\pgfpathlineto{\pgfqpoint{8.848575in}{1.942529in}}%
\pgfpathlineto{\pgfqpoint{8.857382in}{1.558212in}}%
\pgfpathlineto{\pgfqpoint{8.874995in}{1.949388in}}%
\pgfpathlineto{\pgfqpoint{8.883802in}{1.949388in}}%
\pgfpathlineto{\pgfqpoint{8.892609in}{1.990569in}}%
\pgfpathlineto{\pgfqpoint{8.901416in}{1.894489in}}%
\pgfpathlineto{\pgfqpoint{8.910222in}{1.517031in}}%
\pgfpathlineto{\pgfqpoint{8.919029in}{1.496455in}}%
\pgfpathlineto{\pgfqpoint{8.927836in}{1.736653in}}%
\pgfpathlineto{\pgfqpoint{8.936643in}{1.503314in}}%
\pgfpathlineto{\pgfqpoint{8.945450in}{1.558212in}}%
\pgfpathlineto{\pgfqpoint{8.954257in}{1.791551in}}%
\pgfpathlineto{\pgfqpoint{8.963063in}{1.585675in}}%
\pgfpathlineto{\pgfqpoint{8.971870in}{1.228822in}}%
\pgfpathlineto{\pgfqpoint{8.980677in}{1.338618in}}%
\pgfpathlineto{\pgfqpoint{8.989484in}{1.276861in}}%
\pgfpathlineto{\pgfqpoint{8.998291in}{1.276861in}}%
\pgfpathlineto{\pgfqpoint{9.007097in}{1.702331in}}%
\pgfpathlineto{\pgfqpoint{9.015904in}{1.839591in}}%
\pgfpathlineto{\pgfqpoint{9.024711in}{2.203303in}}%
\pgfpathlineto{\pgfqpoint{9.033518in}{1.894489in}}%
\pgfpathlineto{\pgfqpoint{9.042325in}{1.990569in}}%
\pgfpathlineto{\pgfqpoint{9.051132in}{1.832732in}}%
\pgfpathlineto{\pgfqpoint{9.059938in}{2.079789in}}%
\pgfpathlineto{\pgfqpoint{9.068745in}{2.024891in}}%
\pgfpathlineto{\pgfqpoint{9.077552in}{1.839591in}}%
\pgfpathlineto{\pgfqpoint{9.086359in}{1.716076in}}%
\pgfpathlineto{\pgfqpoint{9.095166in}{1.674896in}}%
\pgfpathlineto{\pgfqpoint{9.103972in}{1.887630in}}%
\pgfpathlineto{\pgfqpoint{9.112779in}{1.770975in}}%
\pgfpathlineto{\pgfqpoint{9.121586in}{1.688613in}}%
\pgfpathlineto{\pgfqpoint{9.130393in}{2.018032in}}%
\pgfpathlineto{\pgfqpoint{9.139200in}{1.819015in}}%
\pgfpathlineto{\pgfqpoint{9.148007in}{1.695472in}}%
\pgfpathlineto{\pgfqpoint{9.156813in}{2.079789in}}%
\pgfpathlineto{\pgfqpoint{9.165620in}{1.873913in}}%
\pgfpathlineto{\pgfqpoint{9.174427in}{1.949388in}}%
\pgfpathlineto{\pgfqpoint{9.192041in}{1.805269in}}%
\pgfpathlineto{\pgfqpoint{9.200847in}{1.812128in}}%
\pgfpathlineto{\pgfqpoint{9.209654in}{1.928811in}}%
\pgfpathlineto{\pgfqpoint{9.227268in}{2.443501in}}%
\pgfpathlineto{\pgfqpoint{9.236075in}{2.265089in}}%
\pgfpathlineto{\pgfqpoint{9.244882in}{1.997427in}}%
\pgfpathlineto{\pgfqpoint{9.253688in}{2.217049in}}%
\pgfpathlineto{\pgfqpoint{9.262495in}{1.990569in}}%
\pgfpathlineto{\pgfqpoint{9.271302in}{1.928811in}}%
\pgfpathlineto{\pgfqpoint{9.280109in}{2.072930in}}%
\pgfpathlineto{\pgfqpoint{9.288916in}{2.271947in}}%
\pgfpathlineto{\pgfqpoint{9.297722in}{1.976851in}}%
\pgfpathlineto{\pgfqpoint{9.306529in}{1.873913in}}%
\pgfpathlineto{\pgfqpoint{9.315336in}{1.668037in}}%
\pgfpathlineto{\pgfqpoint{9.324143in}{2.175868in}}%
\pgfpathlineto{\pgfqpoint{9.332950in}{2.052326in}}%
\pgfpathlineto{\pgfqpoint{9.350563in}{2.299383in}}%
\pgfpathlineto{\pgfqpoint{9.359370in}{2.141546in}}%
\pgfpathlineto{\pgfqpoint{9.368177in}{2.539581in}}%
\pgfpathlineto{\pgfqpoint{9.376984in}{2.162151in}}%
\pgfpathlineto{\pgfqpoint{9.385791in}{2.182727in}}%
\pgfpathlineto{\pgfqpoint{9.394597in}{2.361168in}}%
\pgfpathlineto{\pgfqpoint{9.412211in}{2.141546in}}%
\pgfpathlineto{\pgfqpoint{9.421018in}{2.031749in}}%
\pgfpathlineto{\pgfqpoint{9.429825in}{2.004286in}}%
\pgfpathlineto{\pgfqpoint{9.438632in}{1.736653in}}%
\pgfpathlineto{\pgfqpoint{9.447438in}{1.709190in}}%
\pgfpathlineto{\pgfqpoint{9.456245in}{1.695472in}}%
\pgfpathlineto{\pgfqpoint{9.465052in}{1.846450in}}%
\pgfpathlineto{\pgfqpoint{9.473859in}{1.915094in}}%
\pgfpathlineto{\pgfqpoint{9.482666in}{1.915094in}}%
\pgfpathlineto{\pgfqpoint{9.491472in}{1.976851in}}%
\pgfpathlineto{\pgfqpoint{9.500279in}{2.127829in}}%
\pgfpathlineto{\pgfqpoint{9.509086in}{2.114111in}}%
\pgfpathlineto{\pgfqpoint{9.517893in}{1.915094in}}%
\pgfpathlineto{\pgfqpoint{9.526700in}{1.784692in}}%
\pgfpathlineto{\pgfqpoint{9.535507in}{1.777834in}}%
\pgfpathlineto{\pgfqpoint{9.544313in}{2.045467in}}%
\pgfpathlineto{\pgfqpoint{9.553120in}{1.805269in}}%
\pgfpathlineto{\pgfqpoint{9.561927in}{1.688613in}}%
\pgfpathlineto{\pgfqpoint{9.570734in}{1.366053in}}%
\pgfpathlineto{\pgfqpoint{9.579541in}{1.386658in}}%
\pgfpathlineto{\pgfqpoint{9.588347in}{1.208217in}}%
\pgfpathlineto{\pgfqpoint{9.597154in}{1.242539in}}%
\pgfpathlineto{\pgfqpoint{9.605961in}{1.208217in}}%
\pgfpathlineto{\pgfqpoint{9.614768in}{1.269974in}}%
\pgfpathlineto{\pgfqpoint{9.632382in}{1.674896in}}%
\pgfpathlineto{\pgfqpoint{9.641188in}{1.571958in}}%
\pgfpathlineto{\pgfqpoint{9.649995in}{1.510172in}}%
\pgfpathlineto{\pgfqpoint{9.658802in}{1.812128in}}%
\pgfpathlineto{\pgfqpoint{9.667609in}{1.578816in}}%
\pgfpathlineto{\pgfqpoint{9.676416in}{1.393517in}}%
\pgfpathlineto{\pgfqpoint{9.685222in}{1.688613in}}%
\pgfpathlineto{\pgfqpoint{9.694029in}{1.661150in}}%
\pgfpathlineto{\pgfqpoint{9.702836in}{1.674896in}}%
\pgfpathlineto{\pgfqpoint{9.711643in}{1.640574in}}%
\pgfpathlineto{\pgfqpoint{9.729257in}{1.736653in}}%
\pgfpathlineto{\pgfqpoint{9.738063in}{1.908207in}}%
\pgfpathlineto{\pgfqpoint{9.746870in}{2.114111in}}%
\pgfpathlineto{\pgfqpoint{9.755677in}{2.004286in}}%
\pgfpathlineto{\pgfqpoint{9.764484in}{2.141546in}}%
\pgfpathlineto{\pgfqpoint{9.773291in}{1.647432in}}%
\pgfpathlineto{\pgfqpoint{9.790904in}{1.489596in}}%
\pgfpathlineto{\pgfqpoint{9.799711in}{1.770975in}}%
\pgfpathlineto{\pgfqpoint{9.808518in}{2.162151in}}%
\pgfpathlineto{\pgfqpoint{9.817325in}{1.908207in}}%
\pgfpathlineto{\pgfqpoint{9.826132in}{2.182727in}}%
\pgfpathlineto{\pgfqpoint{9.834938in}{2.326846in}}%
\pgfpathlineto{\pgfqpoint{9.843745in}{2.217049in}}%
\pgfpathlineto{\pgfqpoint{9.852552in}{1.887630in}}%
\pgfpathlineto{\pgfqpoint{9.861359in}{2.038608in}}%
\pgfpathlineto{\pgfqpoint{9.870166in}{2.333705in}}%
\pgfpathlineto{\pgfqpoint{9.887779in}{1.736653in}}%
\pgfpathlineto{\pgfqpoint{9.896586in}{1.695472in}}%
\pgfpathlineto{\pgfqpoint{9.905393in}{1.812128in}}%
\pgfpathlineto{\pgfqpoint{9.914200in}{1.359195in}}%
\pgfpathlineto{\pgfqpoint{9.923007in}{1.448415in}}%
\pgfpathlineto{\pgfqpoint{9.931813in}{1.551353in}}%
\pgfpathlineto{\pgfqpoint{9.940620in}{1.736653in}}%
\pgfpathlineto{\pgfqpoint{9.949427in}{1.825873in}}%
\pgfpathlineto{\pgfqpoint{9.949427in}{1.825873in}}%
\pgfusepath{stroke}%
\end{pgfscope}%
\begin{pgfscope}%
\pgfpathrectangle{\pgfqpoint{0.702268in}{0.521603in}}{\pgfqpoint{9.687500in}{4.235000in}}%
\pgfusepath{clip}%
\pgfsetrectcap%
\pgfsetroundjoin%
\pgfsetlinewidth{0.501875pt}%
\definecolor{currentstroke}{rgb}{0.501961,0.501961,0.501961}%
\pgfsetstrokecolor{currentstroke}%
\pgfsetstrokeopacity{0.250000}%
\pgfsetdash{}{0pt}%
\pgfpathmoveto{\pgfqpoint{1.142609in}{4.557245in}}%
\pgfpathlineto{\pgfqpoint{1.151416in}{3.418011in}}%
\pgfpathlineto{\pgfqpoint{1.160222in}{2.793524in}}%
\pgfpathlineto{\pgfqpoint{1.169029in}{2.285665in}}%
\pgfpathlineto{\pgfqpoint{1.177836in}{2.045467in}}%
\pgfpathlineto{\pgfqpoint{1.186643in}{2.230767in}}%
\pgfpathlineto{\pgfqpoint{1.195450in}{2.237625in}}%
\pgfpathlineto{\pgfqpoint{1.204257in}{2.477823in}}%
\pgfpathlineto{\pgfqpoint{1.213063in}{2.285665in}}%
\pgfpathlineto{\pgfqpoint{1.221870in}{2.251343in}}%
\pgfpathlineto{\pgfqpoint{1.230677in}{2.066071in}}%
\pgfpathlineto{\pgfqpoint{1.239484in}{2.024891in}}%
\pgfpathlineto{\pgfqpoint{1.248291in}{2.066071in}}%
\pgfpathlineto{\pgfqpoint{1.257097in}{2.045467in}}%
\pgfpathlineto{\pgfqpoint{1.283518in}{1.681754in}}%
\pgfpathlineto{\pgfqpoint{1.292325in}{1.983710in}}%
\pgfpathlineto{\pgfqpoint{1.301132in}{1.729794in}}%
\pgfpathlineto{\pgfqpoint{1.309938in}{1.853308in}}%
\pgfpathlineto{\pgfqpoint{1.318745in}{2.031749in}}%
\pgfpathlineto{\pgfqpoint{1.327552in}{2.086648in}}%
\pgfpathlineto{\pgfqpoint{1.336359in}{1.942529in}}%
\pgfpathlineto{\pgfqpoint{1.345166in}{1.990569in}}%
\pgfpathlineto{\pgfqpoint{1.362779in}{2.072930in}}%
\pgfpathlineto{\pgfqpoint{1.371586in}{2.127829in}}%
\pgfpathlineto{\pgfqpoint{1.380393in}{2.326846in}}%
\pgfpathlineto{\pgfqpoint{1.389200in}{2.217049in}}%
\pgfpathlineto{\pgfqpoint{1.398007in}{1.990569in}}%
\pgfpathlineto{\pgfqpoint{1.406813in}{1.969992in}}%
\pgfpathlineto{\pgfqpoint{1.415620in}{1.777834in}}%
\pgfpathlineto{\pgfqpoint{1.424427in}{1.956246in}}%
\pgfpathlineto{\pgfqpoint{1.433234in}{2.004286in}}%
\pgfpathlineto{\pgfqpoint{1.442041in}{1.963133in}}%
\pgfpathlineto{\pgfqpoint{1.450847in}{1.935670in}}%
\pgfpathlineto{\pgfqpoint{1.459654in}{1.880772in}}%
\pgfpathlineto{\pgfqpoint{1.468461in}{1.530777in}}%
\pgfpathlineto{\pgfqpoint{1.477268in}{1.764116in}}%
\pgfpathlineto{\pgfqpoint{1.486075in}{1.819015in}}%
\pgfpathlineto{\pgfqpoint{1.494882in}{2.072930in}}%
\pgfpathlineto{\pgfqpoint{1.503688in}{2.210190in}}%
\pgfpathlineto{\pgfqpoint{1.512495in}{1.956246in}}%
\pgfpathlineto{\pgfqpoint{1.521302in}{1.736653in}}%
\pgfpathlineto{\pgfqpoint{1.530109in}{1.688613in}}%
\pgfpathlineto{\pgfqpoint{1.538916in}{1.606252in}}%
\pgfpathlineto{\pgfqpoint{1.547722in}{1.935670in}}%
\pgfpathlineto{\pgfqpoint{1.556529in}{2.038608in}}%
\pgfpathlineto{\pgfqpoint{1.565336in}{2.203303in}}%
\pgfpathlineto{\pgfqpoint{1.574143in}{2.649406in}}%
\pgfpathlineto{\pgfqpoint{1.582950in}{1.935670in}}%
\pgfpathlineto{\pgfqpoint{1.591757in}{1.777834in}}%
\pgfpathlineto{\pgfqpoint{1.600563in}{1.846450in}}%
\pgfpathlineto{\pgfqpoint{1.609370in}{1.647432in}}%
\pgfpathlineto{\pgfqpoint{1.618177in}{1.750370in}}%
\pgfpathlineto{\pgfqpoint{1.626984in}{1.420952in}}%
\pgfpathlineto{\pgfqpoint{1.635791in}{1.324873in}}%
\pgfpathlineto{\pgfqpoint{1.644597in}{1.482737in}}%
\pgfpathlineto{\pgfqpoint{1.653404in}{1.496455in}}%
\pgfpathlineto{\pgfqpoint{1.662211in}{1.709190in}}%
\pgfpathlineto{\pgfqpoint{1.671018in}{1.846450in}}%
\pgfpathlineto{\pgfqpoint{1.679825in}{1.661150in}}%
\pgfpathlineto{\pgfqpoint{1.688632in}{1.674896in}}%
\pgfpathlineto{\pgfqpoint{1.697438in}{2.038608in}}%
\pgfpathlineto{\pgfqpoint{1.706245in}{2.066071in}}%
\pgfpathlineto{\pgfqpoint{1.715052in}{1.894489in}}%
\pgfpathlineto{\pgfqpoint{1.723859in}{1.908207in}}%
\pgfpathlineto{\pgfqpoint{1.741472in}{2.066071in}}%
\pgfpathlineto{\pgfqpoint{1.750279in}{1.867054in}}%
\pgfpathlineto{\pgfqpoint{1.759086in}{1.846450in}}%
\pgfpathlineto{\pgfqpoint{1.767893in}{2.011173in}}%
\pgfpathlineto{\pgfqpoint{1.776700in}{1.976851in}}%
\pgfpathlineto{\pgfqpoint{1.785507in}{1.873913in}}%
\pgfpathlineto{\pgfqpoint{1.794313in}{1.839591in}}%
\pgfpathlineto{\pgfqpoint{1.803120in}{2.120970in}}%
\pgfpathlineto{\pgfqpoint{1.811927in}{1.949388in}}%
\pgfpathlineto{\pgfqpoint{1.820734in}{1.942529in}}%
\pgfpathlineto{\pgfqpoint{1.829541in}{1.983710in}}%
\pgfpathlineto{\pgfqpoint{1.838347in}{2.127829in}}%
\pgfpathlineto{\pgfqpoint{1.847154in}{1.901348in}}%
\pgfpathlineto{\pgfqpoint{1.855961in}{1.935670in}}%
\pgfpathlineto{\pgfqpoint{1.864768in}{1.819015in}}%
\pgfpathlineto{\pgfqpoint{1.873575in}{1.956246in}}%
\pgfpathlineto{\pgfqpoint{1.882382in}{2.244484in}}%
\pgfpathlineto{\pgfqpoint{1.891188in}{2.292524in}}%
\pgfpathlineto{\pgfqpoint{1.899995in}{1.880772in}}%
\pgfpathlineto{\pgfqpoint{1.908802in}{1.935670in}}%
\pgfpathlineto{\pgfqpoint{1.917609in}{2.031749in}}%
\pgfpathlineto{\pgfqpoint{1.926416in}{1.867054in}}%
\pgfpathlineto{\pgfqpoint{1.935222in}{2.237625in}}%
\pgfpathlineto{\pgfqpoint{1.944029in}{2.553326in}}%
\pgfpathlineto{\pgfqpoint{1.952836in}{2.265089in}}%
\pgfpathlineto{\pgfqpoint{1.961643in}{2.072930in}}%
\pgfpathlineto{\pgfqpoint{1.970450in}{2.265089in}}%
\pgfpathlineto{\pgfqpoint{1.979257in}{1.736653in}}%
\pgfpathlineto{\pgfqpoint{1.988063in}{1.674896in}}%
\pgfpathlineto{\pgfqpoint{1.996870in}{1.640574in}}%
\pgfpathlineto{\pgfqpoint{2.005677in}{1.709190in}}%
\pgfpathlineto{\pgfqpoint{2.014484in}{1.668037in}}%
\pgfpathlineto{\pgfqpoint{2.023291in}{1.674896in}}%
\pgfpathlineto{\pgfqpoint{2.032097in}{1.571958in}}%
\pgfpathlineto{\pgfqpoint{2.040904in}{1.990569in}}%
\pgfpathlineto{\pgfqpoint{2.049711in}{2.066071in}}%
\pgfpathlineto{\pgfqpoint{2.058518in}{2.100365in}}%
\pgfpathlineto{\pgfqpoint{2.067325in}{2.045467in}}%
\pgfpathlineto{\pgfqpoint{2.076132in}{2.045467in}}%
\pgfpathlineto{\pgfqpoint{2.084938in}{2.196445in}}%
\pgfpathlineto{\pgfqpoint{2.102552in}{2.786638in}}%
\pgfpathlineto{\pgfqpoint{2.120166in}{2.306269in}}%
\pgfpathlineto{\pgfqpoint{2.128972in}{2.450388in}}%
\pgfpathlineto{\pgfqpoint{2.137779in}{2.443501in}}%
\pgfpathlineto{\pgfqpoint{2.146586in}{2.361168in}}%
\pgfpathlineto{\pgfqpoint{2.155393in}{2.182727in}}%
\pgfpathlineto{\pgfqpoint{2.164200in}{2.477823in}}%
\pgfpathlineto{\pgfqpoint{2.173007in}{2.299383in}}%
\pgfpathlineto{\pgfqpoint{2.181813in}{2.470965in}}%
\pgfpathlineto{\pgfqpoint{2.190620in}{2.278806in}}%
\pgfpathlineto{\pgfqpoint{2.199427in}{2.457247in}}%
\pgfpathlineto{\pgfqpoint{2.208234in}{2.903321in}}%
\pgfpathlineto{\pgfqpoint{2.217041in}{2.560185in}}%
\pgfpathlineto{\pgfqpoint{2.225847in}{3.047440in}}%
\pgfpathlineto{\pgfqpoint{2.234654in}{2.766061in}}%
\pgfpathlineto{\pgfqpoint{2.243461in}{2.580761in}}%
\pgfpathlineto{\pgfqpoint{2.252268in}{2.519004in}}%
\pgfpathlineto{\pgfqpoint{2.261075in}{2.100365in}}%
\pgfpathlineto{\pgfqpoint{2.269882in}{2.271947in}}%
\pgfpathlineto{\pgfqpoint{2.278688in}{2.347422in}}%
\pgfpathlineto{\pgfqpoint{2.287495in}{2.100365in}}%
\pgfpathlineto{\pgfqpoint{2.296302in}{1.976851in}}%
\pgfpathlineto{\pgfqpoint{2.305109in}{1.990569in}}%
\pgfpathlineto{\pgfqpoint{2.322722in}{2.718022in}}%
\pgfpathlineto{\pgfqpoint{2.331529in}{2.230767in}}%
\pgfpathlineto{\pgfqpoint{2.340336in}{2.477823in}}%
\pgfpathlineto{\pgfqpoint{2.349143in}{2.299383in}}%
\pgfpathlineto{\pgfqpoint{2.357950in}{2.265089in}}%
\pgfpathlineto{\pgfqpoint{2.366757in}{1.915094in}}%
\pgfpathlineto{\pgfqpoint{2.375563in}{1.860167in}}%
\pgfpathlineto{\pgfqpoint{2.384370in}{1.668037in}}%
\pgfpathlineto{\pgfqpoint{2.393177in}{1.345477in}}%
\pgfpathlineto{\pgfqpoint{2.401984in}{1.441556in}}%
\pgfpathlineto{\pgfqpoint{2.410791in}{1.290579in}}%
\pgfpathlineto{\pgfqpoint{2.428404in}{1.825873in}}%
\pgfpathlineto{\pgfqpoint{2.437211in}{1.908207in}}%
\pgfpathlineto{\pgfqpoint{2.446018in}{1.880772in}}%
\pgfpathlineto{\pgfqpoint{2.454825in}{2.079789in}}%
\pgfpathlineto{\pgfqpoint{2.463632in}{1.901348in}}%
\pgfpathlineto{\pgfqpoint{2.472438in}{1.867054in}}%
\pgfpathlineto{\pgfqpoint{2.481245in}{1.963133in}}%
\pgfpathlineto{\pgfqpoint{2.490052in}{1.839591in}}%
\pgfpathlineto{\pgfqpoint{2.498859in}{2.251343in}}%
\pgfpathlineto{\pgfqpoint{2.507666in}{2.395462in}}%
\pgfpathlineto{\pgfqpoint{2.516472in}{2.470965in}}%
\pgfpathlineto{\pgfqpoint{2.525279in}{2.024891in}}%
\pgfpathlineto{\pgfqpoint{2.534086in}{1.915094in}}%
\pgfpathlineto{\pgfqpoint{2.542893in}{2.251343in}}%
\pgfpathlineto{\pgfqpoint{2.551700in}{1.976851in}}%
\pgfpathlineto{\pgfqpoint{2.560507in}{2.210190in}}%
\pgfpathlineto{\pgfqpoint{2.569313in}{2.045467in}}%
\pgfpathlineto{\pgfqpoint{2.578120in}{2.100365in}}%
\pgfpathlineto{\pgfqpoint{2.586927in}{1.997427in}}%
\pgfpathlineto{\pgfqpoint{2.595734in}{1.832732in}}%
\pgfpathlineto{\pgfqpoint{2.604541in}{1.750370in}}%
\pgfpathlineto{\pgfqpoint{2.613347in}{1.523918in}}%
\pgfpathlineto{\pgfqpoint{2.622154in}{1.908207in}}%
\pgfpathlineto{\pgfqpoint{2.630961in}{1.915094in}}%
\pgfpathlineto{\pgfqpoint{2.639768in}{1.942529in}}%
\pgfpathlineto{\pgfqpoint{2.648575in}{1.770975in}}%
\pgfpathlineto{\pgfqpoint{2.657382in}{1.702331in}}%
\pgfpathlineto{\pgfqpoint{2.666188in}{1.716076in}}%
\pgfpathlineto{\pgfqpoint{2.674995in}{1.764116in}}%
\pgfpathlineto{\pgfqpoint{2.683802in}{1.839591in}}%
\pgfpathlineto{\pgfqpoint{2.692609in}{1.846450in}}%
\pgfpathlineto{\pgfqpoint{2.701416in}{1.880772in}}%
\pgfpathlineto{\pgfqpoint{2.710222in}{2.072930in}}%
\pgfpathlineto{\pgfqpoint{2.719029in}{2.066071in}}%
\pgfpathlineto{\pgfqpoint{2.727836in}{2.038608in}}%
\pgfpathlineto{\pgfqpoint{2.736643in}{1.908207in}}%
\pgfpathlineto{\pgfqpoint{2.745450in}{1.619997in}}%
\pgfpathlineto{\pgfqpoint{2.754257in}{1.702331in}}%
\pgfpathlineto{\pgfqpoint{2.763063in}{1.736653in}}%
\pgfpathlineto{\pgfqpoint{2.780677in}{1.853308in}}%
\pgfpathlineto{\pgfqpoint{2.789484in}{1.846450in}}%
\pgfpathlineto{\pgfqpoint{2.798291in}{2.072930in}}%
\pgfpathlineto{\pgfqpoint{2.807097in}{2.169009in}}%
\pgfpathlineto{\pgfqpoint{2.815904in}{1.709190in}}%
\pgfpathlineto{\pgfqpoint{2.824711in}{1.784692in}}%
\pgfpathlineto{\pgfqpoint{2.833518in}{1.805269in}}%
\pgfpathlineto{\pgfqpoint{2.842325in}{1.599393in}}%
\pgfpathlineto{\pgfqpoint{2.851132in}{1.784692in}}%
\pgfpathlineto{\pgfqpoint{2.859938in}{1.709190in}}%
\pgfpathlineto{\pgfqpoint{2.868745in}{1.311155in}}%
\pgfpathlineto{\pgfqpoint{2.877552in}{1.654291in}}%
\pgfpathlineto{\pgfqpoint{2.886359in}{1.647432in}}%
\pgfpathlineto{\pgfqpoint{2.895166in}{1.949388in}}%
\pgfpathlineto{\pgfqpoint{2.903972in}{1.688613in}}%
\pgfpathlineto{\pgfqpoint{2.912779in}{1.681754in}}%
\pgfpathlineto{\pgfqpoint{2.921586in}{1.736653in}}%
\pgfpathlineto{\pgfqpoint{2.930393in}{1.523918in}}%
\pgfpathlineto{\pgfqpoint{2.939200in}{1.565071in}}%
\pgfpathlineto{\pgfqpoint{2.948007in}{1.530777in}}%
\pgfpathlineto{\pgfqpoint{2.965620in}{2.155292in}}%
\pgfpathlineto{\pgfqpoint{2.974427in}{2.258230in}}%
\pgfpathlineto{\pgfqpoint{2.983234in}{2.079789in}}%
\pgfpathlineto{\pgfqpoint{2.992041in}{1.661150in}}%
\pgfpathlineto{\pgfqpoint{3.000847in}{1.997427in}}%
\pgfpathlineto{\pgfqpoint{3.009654in}{1.633715in}}%
\pgfpathlineto{\pgfqpoint{3.018461in}{1.441556in}}%
\pgfpathlineto{\pgfqpoint{3.027268in}{1.571958in}}%
\pgfpathlineto{\pgfqpoint{3.036075in}{1.468992in}}%
\pgfpathlineto{\pgfqpoint{3.044882in}{1.475878in}}%
\pgfpathlineto{\pgfqpoint{3.053688in}{1.434698in}}%
\pgfpathlineto{\pgfqpoint{3.062495in}{1.462133in}}%
\pgfpathlineto{\pgfqpoint{3.071302in}{1.558212in}}%
\pgfpathlineto{\pgfqpoint{3.080109in}{1.599393in}}%
\pgfpathlineto{\pgfqpoint{3.088916in}{1.928811in}}%
\pgfpathlineto{\pgfqpoint{3.097722in}{1.825873in}}%
\pgfpathlineto{\pgfqpoint{3.106529in}{2.237625in}}%
\pgfpathlineto{\pgfqpoint{3.124143in}{2.011173in}}%
\pgfpathlineto{\pgfqpoint{3.132950in}{2.162151in}}%
\pgfpathlineto{\pgfqpoint{3.141757in}{1.722935in}}%
\pgfpathlineto{\pgfqpoint{3.150563in}{1.812128in}}%
\pgfpathlineto{\pgfqpoint{3.159370in}{1.819015in}}%
\pgfpathlineto{\pgfqpoint{3.168177in}{1.983710in}}%
\pgfpathlineto{\pgfqpoint{3.176984in}{2.381744in}}%
\pgfpathlineto{\pgfqpoint{3.185791in}{2.368027in}}%
\pgfpathlineto{\pgfqpoint{3.194597in}{2.477823in}}%
\pgfpathlineto{\pgfqpoint{3.203404in}{2.361168in}}%
\pgfpathlineto{\pgfqpoint{3.212211in}{2.217049in}}%
\pgfpathlineto{\pgfqpoint{3.221018in}{1.853308in}}%
\pgfpathlineto{\pgfqpoint{3.229825in}{1.400376in}}%
\pgfpathlineto{\pgfqpoint{3.238632in}{1.448415in}}%
\pgfpathlineto{\pgfqpoint{3.247438in}{1.585675in}}%
\pgfpathlineto{\pgfqpoint{3.256245in}{1.297438in}}%
\pgfpathlineto{\pgfqpoint{3.265052in}{1.517031in}}%
\pgfpathlineto{\pgfqpoint{3.273859in}{1.510172in}}%
\pgfpathlineto{\pgfqpoint{3.282666in}{1.379799in}}%
\pgfpathlineto{\pgfqpoint{3.300279in}{1.887630in}}%
\pgfpathlineto{\pgfqpoint{3.309086in}{2.038608in}}%
\pgfpathlineto{\pgfqpoint{3.317893in}{2.038608in}}%
\pgfpathlineto{\pgfqpoint{3.326700in}{1.867054in}}%
\pgfpathlineto{\pgfqpoint{3.335507in}{2.038608in}}%
\pgfpathlineto{\pgfqpoint{3.344313in}{1.743512in}}%
\pgfpathlineto{\pgfqpoint{3.353120in}{2.100365in}}%
\pgfpathlineto{\pgfqpoint{3.361927in}{1.990569in}}%
\pgfpathlineto{\pgfqpoint{3.370734in}{1.983710in}}%
\pgfpathlineto{\pgfqpoint{3.388347in}{2.134687in}}%
\pgfpathlineto{\pgfqpoint{3.397154in}{2.004286in}}%
\pgfpathlineto{\pgfqpoint{3.405961in}{1.976851in}}%
\pgfpathlineto{\pgfqpoint{3.414768in}{2.354309in}}%
\pgfpathlineto{\pgfqpoint{3.423575in}{2.319987in}}%
\pgfpathlineto{\pgfqpoint{3.432382in}{1.894489in}}%
\pgfpathlineto{\pgfqpoint{3.441188in}{2.134687in}}%
\pgfpathlineto{\pgfqpoint{3.449995in}{1.935670in}}%
\pgfpathlineto{\pgfqpoint{3.458802in}{1.571958in}}%
\pgfpathlineto{\pgfqpoint{3.467609in}{1.860167in}}%
\pgfpathlineto{\pgfqpoint{3.476416in}{2.004286in}}%
\pgfpathlineto{\pgfqpoint{3.485222in}{1.702331in}}%
\pgfpathlineto{\pgfqpoint{3.494029in}{1.571958in}}%
\pgfpathlineto{\pgfqpoint{3.502836in}{1.565071in}}%
\pgfpathlineto{\pgfqpoint{3.511643in}{1.503314in}}%
\pgfpathlineto{\pgfqpoint{3.520450in}{1.242539in}}%
\pgfpathlineto{\pgfqpoint{3.529257in}{1.338618in}}%
\pgfpathlineto{\pgfqpoint{3.538063in}{1.338618in}}%
\pgfpathlineto{\pgfqpoint{3.546870in}{1.956246in}}%
\pgfpathlineto{\pgfqpoint{3.555677in}{2.169009in}}%
\pgfpathlineto{\pgfqpoint{3.564484in}{2.024891in}}%
\pgfpathlineto{\pgfqpoint{3.573291in}{2.162151in}}%
\pgfpathlineto{\pgfqpoint{3.582097in}{1.928811in}}%
\pgfpathlineto{\pgfqpoint{3.590904in}{1.352336in}}%
\pgfpathlineto{\pgfqpoint{3.599711in}{1.180782in}}%
\pgfpathlineto{\pgfqpoint{3.608518in}{1.496455in}}%
\pgfpathlineto{\pgfqpoint{3.617325in}{1.853308in}}%
\pgfpathlineto{\pgfqpoint{3.626132in}{2.141546in}}%
\pgfpathlineto{\pgfqpoint{3.634938in}{2.155292in}}%
\pgfpathlineto{\pgfqpoint{3.643745in}{1.976851in}}%
\pgfpathlineto{\pgfqpoint{3.661359in}{1.729794in}}%
\pgfpathlineto{\pgfqpoint{3.670166in}{1.647432in}}%
\pgfpathlineto{\pgfqpoint{3.678972in}{1.544494in}}%
\pgfpathlineto{\pgfqpoint{3.687779in}{1.455274in}}%
\pgfpathlineto{\pgfqpoint{3.696586in}{1.805269in}}%
\pgfpathlineto{\pgfqpoint{3.705393in}{2.086648in}}%
\pgfpathlineto{\pgfqpoint{3.714200in}{1.887630in}}%
\pgfpathlineto{\pgfqpoint{3.723007in}{2.127829in}}%
\pgfpathlineto{\pgfqpoint{3.731813in}{1.942529in}}%
\pgfpathlineto{\pgfqpoint{3.740620in}{1.976851in}}%
\pgfpathlineto{\pgfqpoint{3.749427in}{1.873913in}}%
\pgfpathlineto{\pgfqpoint{3.758234in}{2.100365in}}%
\pgfpathlineto{\pgfqpoint{3.767041in}{2.429784in}}%
\pgfpathlineto{\pgfqpoint{3.775847in}{2.368027in}}%
\pgfpathlineto{\pgfqpoint{3.784654in}{2.169009in}}%
\pgfpathlineto{\pgfqpoint{3.793461in}{1.867054in}}%
\pgfpathlineto{\pgfqpoint{3.802268in}{1.976851in}}%
\pgfpathlineto{\pgfqpoint{3.811075in}{1.743512in}}%
\pgfpathlineto{\pgfqpoint{3.819882in}{1.716076in}}%
\pgfpathlineto{\pgfqpoint{3.828688in}{1.784692in}}%
\pgfpathlineto{\pgfqpoint{3.837495in}{1.928811in}}%
\pgfpathlineto{\pgfqpoint{3.846302in}{1.757229in}}%
\pgfpathlineto{\pgfqpoint{3.855109in}{1.743512in}}%
\pgfpathlineto{\pgfqpoint{3.863916in}{1.867054in}}%
\pgfpathlineto{\pgfqpoint{3.872722in}{2.203303in}}%
\pgfpathlineto{\pgfqpoint{3.881529in}{2.120970in}}%
\pgfpathlineto{\pgfqpoint{3.899143in}{1.791551in}}%
\pgfpathlineto{\pgfqpoint{3.907950in}{1.901348in}}%
\pgfpathlineto{\pgfqpoint{3.916757in}{1.530777in}}%
\pgfpathlineto{\pgfqpoint{3.925563in}{1.825873in}}%
\pgfpathlineto{\pgfqpoint{3.934370in}{1.729794in}}%
\pgfpathlineto{\pgfqpoint{3.943177in}{1.935670in}}%
\pgfpathlineto{\pgfqpoint{3.951984in}{1.825873in}}%
\pgfpathlineto{\pgfqpoint{3.960791in}{2.086648in}}%
\pgfpathlineto{\pgfqpoint{3.969597in}{2.292524in}}%
\pgfpathlineto{\pgfqpoint{3.978404in}{1.956246in}}%
\pgfpathlineto{\pgfqpoint{3.987211in}{1.949388in}}%
\pgfpathlineto{\pgfqpoint{3.996018in}{1.770975in}}%
\pgfpathlineto{\pgfqpoint{4.004825in}{1.969992in}}%
\pgfpathlineto{\pgfqpoint{4.022438in}{1.887630in}}%
\pgfpathlineto{\pgfqpoint{4.031245in}{1.853308in}}%
\pgfpathlineto{\pgfqpoint{4.040052in}{2.045467in}}%
\pgfpathlineto{\pgfqpoint{4.048859in}{2.292524in}}%
\pgfpathlineto{\pgfqpoint{4.057666in}{2.251343in}}%
\pgfpathlineto{\pgfqpoint{4.066472in}{2.182727in}}%
\pgfpathlineto{\pgfqpoint{4.075279in}{2.038608in}}%
\pgfpathlineto{\pgfqpoint{4.084086in}{1.976851in}}%
\pgfpathlineto{\pgfqpoint{4.092893in}{2.134687in}}%
\pgfpathlineto{\pgfqpoint{4.101700in}{1.757229in}}%
\pgfpathlineto{\pgfqpoint{4.110507in}{2.114111in}}%
\pgfpathlineto{\pgfqpoint{4.119313in}{1.949388in}}%
\pgfpathlineto{\pgfqpoint{4.128120in}{1.983710in}}%
\pgfpathlineto{\pgfqpoint{4.136927in}{2.217049in}}%
\pgfpathlineto{\pgfqpoint{4.145734in}{2.258230in}}%
\pgfpathlineto{\pgfqpoint{4.154541in}{1.853308in}}%
\pgfpathlineto{\pgfqpoint{4.163347in}{1.647432in}}%
\pgfpathlineto{\pgfqpoint{4.172154in}{1.716076in}}%
\pgfpathlineto{\pgfqpoint{4.189768in}{2.148405in}}%
\pgfpathlineto{\pgfqpoint{4.198575in}{1.942529in}}%
\pgfpathlineto{\pgfqpoint{4.207382in}{1.613110in}}%
\pgfpathlineto{\pgfqpoint{4.216188in}{1.770975in}}%
\pgfpathlineto{\pgfqpoint{4.224995in}{1.702331in}}%
\pgfpathlineto{\pgfqpoint{4.233802in}{1.736653in}}%
\pgfpathlineto{\pgfqpoint{4.242609in}{1.935670in}}%
\pgfpathlineto{\pgfqpoint{4.251416in}{1.942529in}}%
\pgfpathlineto{\pgfqpoint{4.269029in}{2.354309in}}%
\pgfpathlineto{\pgfqpoint{4.277836in}{2.580761in}}%
\pgfpathlineto{\pgfqpoint{4.286643in}{2.505287in}}%
\pgfpathlineto{\pgfqpoint{4.295450in}{2.244484in}}%
\pgfpathlineto{\pgfqpoint{4.304257in}{2.271947in}}%
\pgfpathlineto{\pgfqpoint{4.313063in}{1.915094in}}%
\pgfpathlineto{\pgfqpoint{4.321870in}{2.162151in}}%
\pgfpathlineto{\pgfqpoint{4.330677in}{2.011173in}}%
\pgfpathlineto{\pgfqpoint{4.339484in}{2.292524in}}%
\pgfpathlineto{\pgfqpoint{4.348291in}{2.086648in}}%
\pgfpathlineto{\pgfqpoint{4.357097in}{2.251343in}}%
\pgfpathlineto{\pgfqpoint{4.365904in}{2.258230in}}%
\pgfpathlineto{\pgfqpoint{4.374711in}{2.663123in}}%
\pgfpathlineto{\pgfqpoint{4.383518in}{2.347422in}}%
\pgfpathlineto{\pgfqpoint{4.392325in}{2.306269in}}%
\pgfpathlineto{\pgfqpoint{4.401132in}{2.127829in}}%
\pgfpathlineto{\pgfqpoint{4.409938in}{1.894489in}}%
\pgfpathlineto{\pgfqpoint{4.418745in}{2.072930in}}%
\pgfpathlineto{\pgfqpoint{4.427552in}{2.004286in}}%
\pgfpathlineto{\pgfqpoint{4.436359in}{1.661150in}}%
\pgfpathlineto{\pgfqpoint{4.445166in}{1.606252in}}%
\pgfpathlineto{\pgfqpoint{4.453972in}{1.125855in}}%
\pgfpathlineto{\pgfqpoint{4.462779in}{1.420952in}}%
\pgfpathlineto{\pgfqpoint{4.471586in}{1.455274in}}%
\pgfpathlineto{\pgfqpoint{4.480393in}{1.530777in}}%
\pgfpathlineto{\pgfqpoint{4.489200in}{1.764116in}}%
\pgfpathlineto{\pgfqpoint{4.506813in}{1.475878in}}%
\pgfpathlineto{\pgfqpoint{4.515620in}{1.571958in}}%
\pgfpathlineto{\pgfqpoint{4.524427in}{1.722935in}}%
\pgfpathlineto{\pgfqpoint{4.533234in}{1.770975in}}%
\pgfpathlineto{\pgfqpoint{4.542041in}{1.901348in}}%
\pgfpathlineto{\pgfqpoint{4.550847in}{2.285665in}}%
\pgfpathlineto{\pgfqpoint{4.559654in}{2.532722in}}%
\pgfpathlineto{\pgfqpoint{4.577268in}{2.477823in}}%
\pgfpathlineto{\pgfqpoint{4.586075in}{2.100365in}}%
\pgfpathlineto{\pgfqpoint{4.594882in}{2.127829in}}%
\pgfpathlineto{\pgfqpoint{4.603688in}{2.093507in}}%
\pgfpathlineto{\pgfqpoint{4.612495in}{2.175868in}}%
\pgfpathlineto{\pgfqpoint{4.630109in}{1.798410in}}%
\pgfpathlineto{\pgfqpoint{4.638916in}{1.908207in}}%
\pgfpathlineto{\pgfqpoint{4.647722in}{1.770975in}}%
\pgfpathlineto{\pgfqpoint{4.656529in}{1.983710in}}%
\pgfpathlineto{\pgfqpoint{4.665336in}{2.052326in}}%
\pgfpathlineto{\pgfqpoint{4.674143in}{1.846450in}}%
\pgfpathlineto{\pgfqpoint{4.682950in}{2.052326in}}%
\pgfpathlineto{\pgfqpoint{4.691757in}{2.319987in}}%
\pgfpathlineto{\pgfqpoint{4.700563in}{2.018032in}}%
\pgfpathlineto{\pgfqpoint{4.718177in}{2.134687in}}%
\pgfpathlineto{\pgfqpoint{4.726984in}{1.791551in}}%
\pgfpathlineto{\pgfqpoint{4.735791in}{2.196445in}}%
\pgfpathlineto{\pgfqpoint{4.744597in}{2.237625in}}%
\pgfpathlineto{\pgfqpoint{4.753404in}{1.784692in}}%
\pgfpathlineto{\pgfqpoint{4.762211in}{1.887630in}}%
\pgfpathlineto{\pgfqpoint{4.771018in}{2.292524in}}%
\pgfpathlineto{\pgfqpoint{4.779825in}{2.333705in}}%
\pgfpathlineto{\pgfqpoint{4.788632in}{2.223908in}}%
\pgfpathlineto{\pgfqpoint{4.797438in}{1.668037in}}%
\pgfpathlineto{\pgfqpoint{4.806245in}{2.175868in}}%
\pgfpathlineto{\pgfqpoint{4.815052in}{2.265089in}}%
\pgfpathlineto{\pgfqpoint{4.823859in}{2.381744in}}%
\pgfpathlineto{\pgfqpoint{4.832666in}{2.155292in}}%
\pgfpathlineto{\pgfqpoint{4.841472in}{2.306269in}}%
\pgfpathlineto{\pgfqpoint{4.850279in}{2.196445in}}%
\pgfpathlineto{\pgfqpoint{4.859086in}{1.825873in}}%
\pgfpathlineto{\pgfqpoint{4.867893in}{1.908207in}}%
\pgfpathlineto{\pgfqpoint{4.876700in}{2.134687in}}%
\pgfpathlineto{\pgfqpoint{4.885507in}{1.997427in}}%
\pgfpathlineto{\pgfqpoint{4.894313in}{2.223908in}}%
\pgfpathlineto{\pgfqpoint{4.903120in}{2.395462in}}%
\pgfpathlineto{\pgfqpoint{4.911927in}{2.107252in}}%
\pgfpathlineto{\pgfqpoint{4.920734in}{1.695472in}}%
\pgfpathlineto{\pgfqpoint{4.929541in}{1.770975in}}%
\pgfpathlineto{\pgfqpoint{4.938347in}{1.633715in}}%
\pgfpathlineto{\pgfqpoint{4.947154in}{1.757229in}}%
\pgfpathlineto{\pgfqpoint{4.955961in}{2.011173in}}%
\pgfpathlineto{\pgfqpoint{4.964768in}{2.100365in}}%
\pgfpathlineto{\pgfqpoint{4.973575in}{2.347422in}}%
\pgfpathlineto{\pgfqpoint{4.982382in}{2.127829in}}%
\pgfpathlineto{\pgfqpoint{4.991188in}{1.949388in}}%
\pgfpathlineto{\pgfqpoint{4.999995in}{2.011173in}}%
\pgfpathlineto{\pgfqpoint{5.008802in}{1.805269in}}%
\pgfpathlineto{\pgfqpoint{5.017609in}{1.722935in}}%
\pgfpathlineto{\pgfqpoint{5.026416in}{1.585675in}}%
\pgfpathlineto{\pgfqpoint{5.035222in}{1.318014in}}%
\pgfpathlineto{\pgfqpoint{5.044029in}{1.091561in}}%
\pgfpathlineto{\pgfqpoint{5.052836in}{1.448415in}}%
\pgfpathlineto{\pgfqpoint{5.061643in}{1.359195in}}%
\pgfpathlineto{\pgfqpoint{5.070450in}{1.565071in}}%
\pgfpathlineto{\pgfqpoint{5.079257in}{1.640574in}}%
\pgfpathlineto{\pgfqpoint{5.088063in}{1.887630in}}%
\pgfpathlineto{\pgfqpoint{5.096870in}{1.839591in}}%
\pgfpathlineto{\pgfqpoint{5.105677in}{1.359195in}}%
\pgfpathlineto{\pgfqpoint{5.114484in}{1.784692in}}%
\pgfpathlineto{\pgfqpoint{5.123291in}{1.626856in}}%
\pgfpathlineto{\pgfqpoint{5.132097in}{1.592534in}}%
\pgfpathlineto{\pgfqpoint{5.140904in}{1.819015in}}%
\pgfpathlineto{\pgfqpoint{5.149711in}{1.613110in}}%
\pgfpathlineto{\pgfqpoint{5.158518in}{1.901348in}}%
\pgfpathlineto{\pgfqpoint{5.167325in}{1.846450in}}%
\pgfpathlineto{\pgfqpoint{5.176132in}{1.661150in}}%
\pgfpathlineto{\pgfqpoint{5.184938in}{1.654291in}}%
\pgfpathlineto{\pgfqpoint{5.193745in}{1.523918in}}%
\pgfpathlineto{\pgfqpoint{5.202552in}{1.647432in}}%
\pgfpathlineto{\pgfqpoint{5.211359in}{1.551353in}}%
\pgfpathlineto{\pgfqpoint{5.220166in}{1.517031in}}%
\pgfpathlineto{\pgfqpoint{5.228972in}{1.311155in}}%
\pgfpathlineto{\pgfqpoint{5.237779in}{1.510172in}}%
\pgfpathlineto{\pgfqpoint{5.246586in}{1.633715in}}%
\pgfpathlineto{\pgfqpoint{5.255393in}{1.640574in}}%
\pgfpathlineto{\pgfqpoint{5.264200in}{1.695472in}}%
\pgfpathlineto{\pgfqpoint{5.273007in}{1.619997in}}%
\pgfpathlineto{\pgfqpoint{5.290620in}{2.196445in}}%
\pgfpathlineto{\pgfqpoint{5.299427in}{1.812128in}}%
\pgfpathlineto{\pgfqpoint{5.308234in}{1.880772in}}%
\pgfpathlineto{\pgfqpoint{5.317041in}{1.839591in}}%
\pgfpathlineto{\pgfqpoint{5.325847in}{1.969992in}}%
\pgfpathlineto{\pgfqpoint{5.334654in}{1.571958in}}%
\pgfpathlineto{\pgfqpoint{5.343461in}{1.832732in}}%
\pgfpathlineto{\pgfqpoint{5.352268in}{1.873913in}}%
\pgfpathlineto{\pgfqpoint{5.361075in}{1.661150in}}%
\pgfpathlineto{\pgfqpoint{5.369882in}{2.086648in}}%
\pgfpathlineto{\pgfqpoint{5.378688in}{2.210190in}}%
\pgfpathlineto{\pgfqpoint{5.387495in}{2.237625in}}%
\pgfpathlineto{\pgfqpoint{5.396302in}{2.340563in}}%
\pgfpathlineto{\pgfqpoint{5.405109in}{2.368027in}}%
\pgfpathlineto{\pgfqpoint{5.413916in}{2.388603in}}%
\pgfpathlineto{\pgfqpoint{5.422722in}{2.024891in}}%
\pgfpathlineto{\pgfqpoint{5.431529in}{1.901348in}}%
\pgfpathlineto{\pgfqpoint{5.440336in}{1.915094in}}%
\pgfpathlineto{\pgfqpoint{5.449143in}{1.420952in}}%
\pgfpathlineto{\pgfqpoint{5.457950in}{1.585675in}}%
\pgfpathlineto{\pgfqpoint{5.466757in}{1.716076in}}%
\pgfpathlineto{\pgfqpoint{5.475563in}{1.331760in}}%
\pgfpathlineto{\pgfqpoint{5.484370in}{1.537636in}}%
\pgfpathlineto{\pgfqpoint{5.493177in}{1.633715in}}%
\pgfpathlineto{\pgfqpoint{5.501984in}{1.976851in}}%
\pgfpathlineto{\pgfqpoint{5.510791in}{1.805269in}}%
\pgfpathlineto{\pgfqpoint{5.519597in}{2.079789in}}%
\pgfpathlineto{\pgfqpoint{5.528404in}{2.134687in}}%
\pgfpathlineto{\pgfqpoint{5.537211in}{2.059213in}}%
\pgfpathlineto{\pgfqpoint{5.546018in}{2.567044in}}%
\pgfpathlineto{\pgfqpoint{5.554825in}{2.464106in}}%
\pgfpathlineto{\pgfqpoint{5.563632in}{2.553326in}}%
\pgfpathlineto{\pgfqpoint{5.572438in}{2.457247in}}%
\pgfpathlineto{\pgfqpoint{5.581245in}{2.525863in}}%
\pgfpathlineto{\pgfqpoint{5.590052in}{2.519004in}}%
\pgfpathlineto{\pgfqpoint{5.598859in}{2.491541in}}%
\pgfpathlineto{\pgfqpoint{5.607666in}{2.175868in}}%
\pgfpathlineto{\pgfqpoint{5.616472in}{2.120970in}}%
\pgfpathlineto{\pgfqpoint{5.625279in}{1.709190in}}%
\pgfpathlineto{\pgfqpoint{5.634086in}{1.654291in}}%
\pgfpathlineto{\pgfqpoint{5.642893in}{1.702331in}}%
\pgfpathlineto{\pgfqpoint{5.651700in}{2.127829in}}%
\pgfpathlineto{\pgfqpoint{5.660507in}{2.134687in}}%
\pgfpathlineto{\pgfqpoint{5.669313in}{2.189586in}}%
\pgfpathlineto{\pgfqpoint{5.678120in}{2.031749in}}%
\pgfpathlineto{\pgfqpoint{5.686927in}{1.503314in}}%
\pgfpathlineto{\pgfqpoint{5.695734in}{1.805269in}}%
\pgfpathlineto{\pgfqpoint{5.704541in}{1.942529in}}%
\pgfpathlineto{\pgfqpoint{5.713347in}{2.155292in}}%
\pgfpathlineto{\pgfqpoint{5.722154in}{2.278806in}}%
\pgfpathlineto{\pgfqpoint{5.730961in}{2.141546in}}%
\pgfpathlineto{\pgfqpoint{5.739768in}{2.148405in}}%
\pgfpathlineto{\pgfqpoint{5.748575in}{2.011173in}}%
\pgfpathlineto{\pgfqpoint{5.757382in}{2.580761in}}%
\pgfpathlineto{\pgfqpoint{5.774995in}{2.155292in}}%
\pgfpathlineto{\pgfqpoint{5.783802in}{2.052326in}}%
\pgfpathlineto{\pgfqpoint{5.792609in}{1.887630in}}%
\pgfpathlineto{\pgfqpoint{5.801416in}{1.832732in}}%
\pgfpathlineto{\pgfqpoint{5.810222in}{1.908207in}}%
\pgfpathlineto{\pgfqpoint{5.819029in}{1.716076in}}%
\pgfpathlineto{\pgfqpoint{5.827836in}{1.860167in}}%
\pgfpathlineto{\pgfqpoint{5.836643in}{2.251343in}}%
\pgfpathlineto{\pgfqpoint{5.845450in}{2.052326in}}%
\pgfpathlineto{\pgfqpoint{5.854257in}{1.812128in}}%
\pgfpathlineto{\pgfqpoint{5.863063in}{1.819015in}}%
\pgfpathlineto{\pgfqpoint{5.871870in}{2.120970in}}%
\pgfpathlineto{\pgfqpoint{5.880677in}{2.134687in}}%
\pgfpathlineto{\pgfqpoint{5.889484in}{2.127829in}}%
\pgfpathlineto{\pgfqpoint{5.898291in}{2.031749in}}%
\pgfpathlineto{\pgfqpoint{5.907097in}{2.079789in}}%
\pgfpathlineto{\pgfqpoint{5.924711in}{1.956246in}}%
\pgfpathlineto{\pgfqpoint{5.933518in}{1.798410in}}%
\pgfpathlineto{\pgfqpoint{5.942325in}{2.162151in}}%
\pgfpathlineto{\pgfqpoint{5.951132in}{2.265089in}}%
\pgfpathlineto{\pgfqpoint{5.959938in}{2.024891in}}%
\pgfpathlineto{\pgfqpoint{5.968745in}{2.306269in}}%
\pgfpathlineto{\pgfqpoint{5.977552in}{2.100365in}}%
\pgfpathlineto{\pgfqpoint{5.986359in}{2.127829in}}%
\pgfpathlineto{\pgfqpoint{5.995166in}{2.271947in}}%
\pgfpathlineto{\pgfqpoint{6.003972in}{2.333705in}}%
\pgfpathlineto{\pgfqpoint{6.012779in}{1.867054in}}%
\pgfpathlineto{\pgfqpoint{6.021586in}{1.873913in}}%
\pgfpathlineto{\pgfqpoint{6.030393in}{1.873913in}}%
\pgfpathlineto{\pgfqpoint{6.039200in}{2.196445in}}%
\pgfpathlineto{\pgfqpoint{6.048007in}{1.894489in}}%
\pgfpathlineto{\pgfqpoint{6.056813in}{2.107252in}}%
\pgfpathlineto{\pgfqpoint{6.065620in}{1.757229in}}%
\pgfpathlineto{\pgfqpoint{6.074427in}{1.565071in}}%
\pgfpathlineto{\pgfqpoint{6.083234in}{1.880772in}}%
\pgfpathlineto{\pgfqpoint{6.092041in}{2.251343in}}%
\pgfpathlineto{\pgfqpoint{6.100847in}{1.976851in}}%
\pgfpathlineto{\pgfqpoint{6.109654in}{1.853308in}}%
\pgfpathlineto{\pgfqpoint{6.118461in}{2.038608in}}%
\pgfpathlineto{\pgfqpoint{6.127268in}{2.024891in}}%
\pgfpathlineto{\pgfqpoint{6.136075in}{1.956246in}}%
\pgfpathlineto{\pgfqpoint{6.144882in}{2.265089in}}%
\pgfpathlineto{\pgfqpoint{6.153688in}{2.141546in}}%
\pgfpathlineto{\pgfqpoint{6.162495in}{2.086648in}}%
\pgfpathlineto{\pgfqpoint{6.171302in}{1.997427in}}%
\pgfpathlineto{\pgfqpoint{6.180109in}{1.695472in}}%
\pgfpathlineto{\pgfqpoint{6.188916in}{2.086648in}}%
\pgfpathlineto{\pgfqpoint{6.197722in}{2.210190in}}%
\pgfpathlineto{\pgfqpoint{6.206529in}{2.086648in}}%
\pgfpathlineto{\pgfqpoint{6.215336in}{1.688613in}}%
\pgfpathlineto{\pgfqpoint{6.224143in}{1.366053in}}%
\pgfpathlineto{\pgfqpoint{6.232950in}{1.565071in}}%
\pgfpathlineto{\pgfqpoint{6.241757in}{1.819015in}}%
\pgfpathlineto{\pgfqpoint{6.250563in}{2.024891in}}%
\pgfpathlineto{\pgfqpoint{6.259370in}{1.983710in}}%
\pgfpathlineto{\pgfqpoint{6.268177in}{2.265089in}}%
\pgfpathlineto{\pgfqpoint{6.276984in}{2.470965in}}%
\pgfpathlineto{\pgfqpoint{6.285791in}{2.217049in}}%
\pgfpathlineto{\pgfqpoint{6.294597in}{2.278806in}}%
\pgfpathlineto{\pgfqpoint{6.303404in}{2.127829in}}%
\pgfpathlineto{\pgfqpoint{6.312211in}{2.182727in}}%
\pgfpathlineto{\pgfqpoint{6.321018in}{2.072930in}}%
\pgfpathlineto{\pgfqpoint{6.329825in}{2.093507in}}%
\pgfpathlineto{\pgfqpoint{6.338632in}{1.819015in}}%
\pgfpathlineto{\pgfqpoint{6.347438in}{2.004286in}}%
\pgfpathlineto{\pgfqpoint{6.356245in}{1.949388in}}%
\pgfpathlineto{\pgfqpoint{6.365052in}{2.223908in}}%
\pgfpathlineto{\pgfqpoint{6.373859in}{2.306269in}}%
\pgfpathlineto{\pgfqpoint{6.382666in}{2.175868in}}%
\pgfpathlineto{\pgfqpoint{6.391472in}{2.004286in}}%
\pgfpathlineto{\pgfqpoint{6.400279in}{2.107252in}}%
\pgfpathlineto{\pgfqpoint{6.409086in}{2.024891in}}%
\pgfpathlineto{\pgfqpoint{6.417893in}{1.695472in}}%
\pgfpathlineto{\pgfqpoint{6.435507in}{1.956246in}}%
\pgfpathlineto{\pgfqpoint{6.453120in}{2.100365in}}%
\pgfpathlineto{\pgfqpoint{6.461927in}{2.354309in}}%
\pgfpathlineto{\pgfqpoint{6.470734in}{2.779779in}}%
\pgfpathlineto{\pgfqpoint{6.479541in}{2.223908in}}%
\pgfpathlineto{\pgfqpoint{6.488347in}{2.519004in}}%
\pgfpathlineto{\pgfqpoint{6.497154in}{2.182727in}}%
\pgfpathlineto{\pgfqpoint{6.505961in}{2.093507in}}%
\pgfpathlineto{\pgfqpoint{6.514768in}{2.127829in}}%
\pgfpathlineto{\pgfqpoint{6.523575in}{2.066071in}}%
\pgfpathlineto{\pgfqpoint{6.532382in}{2.505287in}}%
\pgfpathlineto{\pgfqpoint{6.541188in}{2.210190in}}%
\pgfpathlineto{\pgfqpoint{6.549995in}{2.114111in}}%
\pgfpathlineto{\pgfqpoint{6.558802in}{2.066071in}}%
\pgfpathlineto{\pgfqpoint{6.567609in}{2.196445in}}%
\pgfpathlineto{\pgfqpoint{6.576416in}{2.093507in}}%
\pgfpathlineto{\pgfqpoint{6.585222in}{2.169009in}}%
\pgfpathlineto{\pgfqpoint{6.594029in}{2.127829in}}%
\pgfpathlineto{\pgfqpoint{6.602836in}{1.908207in}}%
\pgfpathlineto{\pgfqpoint{6.611643in}{1.983710in}}%
\pgfpathlineto{\pgfqpoint{6.620450in}{2.258230in}}%
\pgfpathlineto{\pgfqpoint{6.629257in}{2.134687in}}%
\pgfpathlineto{\pgfqpoint{6.638063in}{1.956246in}}%
\pgfpathlineto{\pgfqpoint{6.646870in}{2.148405in}}%
\pgfpathlineto{\pgfqpoint{6.655677in}{1.969992in}}%
\pgfpathlineto{\pgfqpoint{6.664484in}{2.155292in}}%
\pgfpathlineto{\pgfqpoint{6.673291in}{1.915094in}}%
\pgfpathlineto{\pgfqpoint{6.682097in}{2.011173in}}%
\pgfpathlineto{\pgfqpoint{6.690904in}{1.661150in}}%
\pgfpathlineto{\pgfqpoint{6.699711in}{1.544494in}}%
\pgfpathlineto{\pgfqpoint{6.708518in}{1.757229in}}%
\pgfpathlineto{\pgfqpoint{6.717325in}{1.921953in}}%
\pgfpathlineto{\pgfqpoint{6.726132in}{1.880772in}}%
\pgfpathlineto{\pgfqpoint{6.734938in}{1.990569in}}%
\pgfpathlineto{\pgfqpoint{6.743745in}{1.963133in}}%
\pgfpathlineto{\pgfqpoint{6.752552in}{2.024891in}}%
\pgfpathlineto{\pgfqpoint{6.761359in}{1.839591in}}%
\pgfpathlineto{\pgfqpoint{6.770166in}{1.722935in}}%
\pgfpathlineto{\pgfqpoint{6.787779in}{1.853308in}}%
\pgfpathlineto{\pgfqpoint{6.796586in}{1.873913in}}%
\pgfpathlineto{\pgfqpoint{6.805393in}{1.791551in}}%
\pgfpathlineto{\pgfqpoint{6.814200in}{1.661150in}}%
\pgfpathlineto{\pgfqpoint{6.823007in}{1.722935in}}%
\pgfpathlineto{\pgfqpoint{6.831813in}{1.743512in}}%
\pgfpathlineto{\pgfqpoint{6.840620in}{1.345477in}}%
\pgfpathlineto{\pgfqpoint{6.849427in}{1.269974in}}%
\pgfpathlineto{\pgfqpoint{6.858234in}{1.619997in}}%
\pgfpathlineto{\pgfqpoint{6.867041in}{1.427839in}}%
\pgfpathlineto{\pgfqpoint{6.875847in}{1.805269in}}%
\pgfpathlineto{\pgfqpoint{6.884654in}{1.901348in}}%
\pgfpathlineto{\pgfqpoint{6.893461in}{2.203303in}}%
\pgfpathlineto{\pgfqpoint{6.902268in}{2.120970in}}%
\pgfpathlineto{\pgfqpoint{6.911075in}{1.928811in}}%
\pgfpathlineto{\pgfqpoint{6.919882in}{1.901348in}}%
\pgfpathlineto{\pgfqpoint{6.928688in}{1.915094in}}%
\pgfpathlineto{\pgfqpoint{6.937495in}{2.196445in}}%
\pgfpathlineto{\pgfqpoint{6.946302in}{2.381744in}}%
\pgfpathlineto{\pgfqpoint{6.955109in}{2.354309in}}%
\pgfpathlineto{\pgfqpoint{6.963916in}{2.422925in}}%
\pgfpathlineto{\pgfqpoint{6.972722in}{2.223908in}}%
\pgfpathlineto{\pgfqpoint{6.981529in}{2.120970in}}%
\pgfpathlineto{\pgfqpoint{6.990336in}{1.949388in}}%
\pgfpathlineto{\pgfqpoint{6.999143in}{1.654291in}}%
\pgfpathlineto{\pgfqpoint{7.007950in}{1.819015in}}%
\pgfpathlineto{\pgfqpoint{7.016757in}{1.647432in}}%
\pgfpathlineto{\pgfqpoint{7.025563in}{1.942529in}}%
\pgfpathlineto{\pgfqpoint{7.034370in}{2.162151in}}%
\pgfpathlineto{\pgfqpoint{7.043177in}{2.066071in}}%
\pgfpathlineto{\pgfqpoint{7.051984in}{2.127829in}}%
\pgfpathlineto{\pgfqpoint{7.060791in}{2.134687in}}%
\pgfpathlineto{\pgfqpoint{7.069597in}{2.237625in}}%
\pgfpathlineto{\pgfqpoint{7.078404in}{2.265089in}}%
\pgfpathlineto{\pgfqpoint{7.087211in}{2.313128in}}%
\pgfpathlineto{\pgfqpoint{7.096018in}{2.285665in}}%
\pgfpathlineto{\pgfqpoint{7.104825in}{2.038608in}}%
\pgfpathlineto{\pgfqpoint{7.113632in}{2.072930in}}%
\pgfpathlineto{\pgfqpoint{7.122438in}{2.230767in}}%
\pgfpathlineto{\pgfqpoint{7.131245in}{2.278806in}}%
\pgfpathlineto{\pgfqpoint{7.140052in}{2.114111in}}%
\pgfpathlineto{\pgfqpoint{7.148859in}{2.107252in}}%
\pgfpathlineto{\pgfqpoint{7.157666in}{2.175868in}}%
\pgfpathlineto{\pgfqpoint{7.166472in}{2.361168in}}%
\pgfpathlineto{\pgfqpoint{7.175279in}{1.832732in}}%
\pgfpathlineto{\pgfqpoint{7.184086in}{1.901348in}}%
\pgfpathlineto{\pgfqpoint{7.192893in}{1.681754in}}%
\pgfpathlineto{\pgfqpoint{7.201700in}{1.606252in}}%
\pgfpathlineto{\pgfqpoint{7.210507in}{1.668037in}}%
\pgfpathlineto{\pgfqpoint{7.219313in}{1.777834in}}%
\pgfpathlineto{\pgfqpoint{7.228120in}{1.729794in}}%
\pgfpathlineto{\pgfqpoint{7.236927in}{1.805269in}}%
\pgfpathlineto{\pgfqpoint{7.245734in}{1.997427in}}%
\pgfpathlineto{\pgfqpoint{7.254541in}{1.832732in}}%
\pgfpathlineto{\pgfqpoint{7.263347in}{1.757229in}}%
\pgfpathlineto{\pgfqpoint{7.272154in}{1.592534in}}%
\pgfpathlineto{\pgfqpoint{7.280961in}{1.757229in}}%
\pgfpathlineto{\pgfqpoint{7.289768in}{1.592534in}}%
\pgfpathlineto{\pgfqpoint{7.298575in}{1.242539in}}%
\pgfpathlineto{\pgfqpoint{7.307382in}{1.002341in}}%
\pgfpathlineto{\pgfqpoint{7.324995in}{1.654291in}}%
\pgfpathlineto{\pgfqpoint{7.333802in}{1.640574in}}%
\pgfpathlineto{\pgfqpoint{7.342609in}{1.290579in}}%
\pgfpathlineto{\pgfqpoint{7.351416in}{1.448415in}}%
\pgfpathlineto{\pgfqpoint{7.360222in}{1.386658in}}%
\pgfpathlineto{\pgfqpoint{7.369029in}{1.784692in}}%
\pgfpathlineto{\pgfqpoint{7.377836in}{1.722935in}}%
\pgfpathlineto{\pgfqpoint{7.386643in}{1.510172in}}%
\pgfpathlineto{\pgfqpoint{7.395450in}{1.517031in}}%
\pgfpathlineto{\pgfqpoint{7.404257in}{1.716076in}}%
\pgfpathlineto{\pgfqpoint{7.413063in}{2.196445in}}%
\pgfpathlineto{\pgfqpoint{7.421870in}{1.928811in}}%
\pgfpathlineto{\pgfqpoint{7.430677in}{1.613110in}}%
\pgfpathlineto{\pgfqpoint{7.439484in}{1.688613in}}%
\pgfpathlineto{\pgfqpoint{7.448291in}{1.997427in}}%
\pgfpathlineto{\pgfqpoint{7.457097in}{1.983710in}}%
\pgfpathlineto{\pgfqpoint{7.465904in}{2.107252in}}%
\pgfpathlineto{\pgfqpoint{7.474711in}{2.374885in}}%
\pgfpathlineto{\pgfqpoint{7.483518in}{2.120970in}}%
\pgfpathlineto{\pgfqpoint{7.492325in}{2.107252in}}%
\pgfpathlineto{\pgfqpoint{7.501132in}{1.908207in}}%
\pgfpathlineto{\pgfqpoint{7.509938in}{1.956246in}}%
\pgfpathlineto{\pgfqpoint{7.518745in}{2.086648in}}%
\pgfpathlineto{\pgfqpoint{7.527552in}{1.949388in}}%
\pgfpathlineto{\pgfqpoint{7.536359in}{1.935670in}}%
\pgfpathlineto{\pgfqpoint{7.545166in}{2.230767in}}%
\pgfpathlineto{\pgfqpoint{7.553972in}{2.292524in}}%
\pgfpathlineto{\pgfqpoint{7.562779in}{1.908207in}}%
\pgfpathlineto{\pgfqpoint{7.571586in}{2.107252in}}%
\pgfpathlineto{\pgfqpoint{7.580393in}{1.626856in}}%
\pgfpathlineto{\pgfqpoint{7.589200in}{1.798410in}}%
\pgfpathlineto{\pgfqpoint{7.598007in}{1.695472in}}%
\pgfpathlineto{\pgfqpoint{7.606813in}{2.079789in}}%
\pgfpathlineto{\pgfqpoint{7.615620in}{2.114111in}}%
\pgfpathlineto{\pgfqpoint{7.624427in}{2.450388in}}%
\pgfpathlineto{\pgfqpoint{7.633234in}{2.368027in}}%
\pgfpathlineto{\pgfqpoint{7.642041in}{2.313128in}}%
\pgfpathlineto{\pgfqpoint{7.650847in}{2.217049in}}%
\pgfpathlineto{\pgfqpoint{7.659654in}{2.175868in}}%
\pgfpathlineto{\pgfqpoint{7.668461in}{2.340563in}}%
\pgfpathlineto{\pgfqpoint{7.677268in}{2.450388in}}%
\pgfpathlineto{\pgfqpoint{7.686075in}{2.319987in}}%
\pgfpathlineto{\pgfqpoint{7.694882in}{2.265089in}}%
\pgfpathlineto{\pgfqpoint{7.703688in}{2.024891in}}%
\pgfpathlineto{\pgfqpoint{7.712495in}{1.956246in}}%
\pgfpathlineto{\pgfqpoint{7.721302in}{1.983710in}}%
\pgfpathlineto{\pgfqpoint{7.730109in}{2.038608in}}%
\pgfpathlineto{\pgfqpoint{7.738916in}{2.038608in}}%
\pgfpathlineto{\pgfqpoint{7.747722in}{1.894489in}}%
\pgfpathlineto{\pgfqpoint{7.756529in}{2.024891in}}%
\pgfpathlineto{\pgfqpoint{7.765336in}{1.908207in}}%
\pgfpathlineto{\pgfqpoint{7.774143in}{2.052326in}}%
\pgfpathlineto{\pgfqpoint{7.782950in}{1.935670in}}%
\pgfpathlineto{\pgfqpoint{7.791757in}{1.736653in}}%
\pgfpathlineto{\pgfqpoint{7.800563in}{1.935670in}}%
\pgfpathlineto{\pgfqpoint{7.809370in}{2.066071in}}%
\pgfpathlineto{\pgfqpoint{7.818177in}{1.969992in}}%
\pgfpathlineto{\pgfqpoint{7.826984in}{2.340563in}}%
\pgfpathlineto{\pgfqpoint{7.835791in}{2.422925in}}%
\pgfpathlineto{\pgfqpoint{7.844597in}{1.853308in}}%
\pgfpathlineto{\pgfqpoint{7.853404in}{2.196445in}}%
\pgfpathlineto{\pgfqpoint{7.862211in}{2.443501in}}%
\pgfpathlineto{\pgfqpoint{7.871018in}{1.873913in}}%
\pgfpathlineto{\pgfqpoint{7.879825in}{1.846450in}}%
\pgfpathlineto{\pgfqpoint{7.888632in}{1.709190in}}%
\pgfpathlineto{\pgfqpoint{7.897438in}{1.750370in}}%
\pgfpathlineto{\pgfqpoint{7.915052in}{1.606252in}}%
\pgfpathlineto{\pgfqpoint{7.923859in}{1.853308in}}%
\pgfpathlineto{\pgfqpoint{7.932666in}{1.983710in}}%
\pgfpathlineto{\pgfqpoint{7.941472in}{1.928811in}}%
\pgfpathlineto{\pgfqpoint{7.950279in}{2.011173in}}%
\pgfpathlineto{\pgfqpoint{7.959086in}{2.361168in}}%
\pgfpathlineto{\pgfqpoint{7.967893in}{2.155292in}}%
\pgfpathlineto{\pgfqpoint{7.976700in}{2.306269in}}%
\pgfpathlineto{\pgfqpoint{7.985507in}{1.832732in}}%
\pgfpathlineto{\pgfqpoint{7.994313in}{1.770975in}}%
\pgfpathlineto{\pgfqpoint{8.003120in}{1.640574in}}%
\pgfpathlineto{\pgfqpoint{8.011927in}{1.688613in}}%
\pgfpathlineto{\pgfqpoint{8.020734in}{1.626856in}}%
\pgfpathlineto{\pgfqpoint{8.029541in}{1.757229in}}%
\pgfpathlineto{\pgfqpoint{8.038347in}{1.853308in}}%
\pgfpathlineto{\pgfqpoint{8.047154in}{1.798410in}}%
\pgfpathlineto{\pgfqpoint{8.055961in}{1.764116in}}%
\pgfpathlineto{\pgfqpoint{8.064768in}{1.688613in}}%
\pgfpathlineto{\pgfqpoint{8.073575in}{1.407234in}}%
\pgfpathlineto{\pgfqpoint{8.082382in}{1.585675in}}%
\pgfpathlineto{\pgfqpoint{8.091188in}{1.585675in}}%
\pgfpathlineto{\pgfqpoint{8.099995in}{1.661150in}}%
\pgfpathlineto{\pgfqpoint{8.108802in}{1.366053in}}%
\pgfpathlineto{\pgfqpoint{8.117609in}{1.345477in}}%
\pgfpathlineto{\pgfqpoint{8.126416in}{1.736653in}}%
\pgfpathlineto{\pgfqpoint{8.135222in}{1.661150in}}%
\pgfpathlineto{\pgfqpoint{8.144029in}{1.908207in}}%
\pgfpathlineto{\pgfqpoint{8.152836in}{2.072930in}}%
\pgfpathlineto{\pgfqpoint{8.161643in}{2.553326in}}%
\pgfpathlineto{\pgfqpoint{8.170450in}{2.450388in}}%
\pgfpathlineto{\pgfqpoint{8.179257in}{2.141546in}}%
\pgfpathlineto{\pgfqpoint{8.188063in}{1.963133in}}%
\pgfpathlineto{\pgfqpoint{8.196870in}{1.894489in}}%
\pgfpathlineto{\pgfqpoint{8.205677in}{2.072930in}}%
\pgfpathlineto{\pgfqpoint{8.214484in}{2.004286in}}%
\pgfpathlineto{\pgfqpoint{8.223291in}{2.169009in}}%
\pgfpathlineto{\pgfqpoint{8.232097in}{1.990569in}}%
\pgfpathlineto{\pgfqpoint{8.240904in}{1.839591in}}%
\pgfpathlineto{\pgfqpoint{8.249711in}{1.414093in}}%
\pgfpathlineto{\pgfqpoint{8.258518in}{1.167036in}}%
\pgfpathlineto{\pgfqpoint{8.267325in}{1.029776in}}%
\pgfpathlineto{\pgfqpoint{8.276132in}{1.084703in}}%
\pgfpathlineto{\pgfqpoint{8.284938in}{0.981737in}}%
\pgfpathlineto{\pgfqpoint{8.293745in}{1.132742in}}%
\pgfpathlineto{\pgfqpoint{8.302552in}{1.510172in}}%
\pgfpathlineto{\pgfqpoint{8.311359in}{1.517031in}}%
\pgfpathlineto{\pgfqpoint{8.320166in}{1.318014in}}%
\pgfpathlineto{\pgfqpoint{8.328972in}{1.475878in}}%
\pgfpathlineto{\pgfqpoint{8.337779in}{1.434698in}}%
\pgfpathlineto{\pgfqpoint{8.346586in}{1.372912in}}%
\pgfpathlineto{\pgfqpoint{8.355393in}{1.269974in}}%
\pgfpathlineto{\pgfqpoint{8.364200in}{1.496455in}}%
\pgfpathlineto{\pgfqpoint{8.373007in}{1.153319in}}%
\pgfpathlineto{\pgfqpoint{8.381813in}{1.372912in}}%
\pgfpathlineto{\pgfqpoint{8.390620in}{1.523918in}}%
\pgfpathlineto{\pgfqpoint{8.399427in}{1.496455in}}%
\pgfpathlineto{\pgfqpoint{8.408234in}{1.619997in}}%
\pgfpathlineto{\pgfqpoint{8.417041in}{1.345477in}}%
\pgfpathlineto{\pgfqpoint{8.425847in}{1.441556in}}%
\pgfpathlineto{\pgfqpoint{8.434654in}{1.304296in}}%
\pgfpathlineto{\pgfqpoint{8.443461in}{1.359195in}}%
\pgfpathlineto{\pgfqpoint{8.452268in}{1.256257in}}%
\pgfpathlineto{\pgfqpoint{8.461075in}{1.544494in}}%
\pgfpathlineto{\pgfqpoint{8.469882in}{1.688613in}}%
\pgfpathlineto{\pgfqpoint{8.478688in}{1.565071in}}%
\pgfpathlineto{\pgfqpoint{8.487495in}{1.468992in}}%
\pgfpathlineto{\pgfqpoint{8.496302in}{1.613110in}}%
\pgfpathlineto{\pgfqpoint{8.505109in}{1.269974in}}%
\pgfpathlineto{\pgfqpoint{8.513916in}{1.414093in}}%
\pgfpathlineto{\pgfqpoint{8.522722in}{1.633715in}}%
\pgfpathlineto{\pgfqpoint{8.531529in}{2.107252in}}%
\pgfpathlineto{\pgfqpoint{8.540336in}{2.018032in}}%
\pgfpathlineto{\pgfqpoint{8.549143in}{2.127829in}}%
\pgfpathlineto{\pgfqpoint{8.557950in}{2.196445in}}%
\pgfpathlineto{\pgfqpoint{8.566757in}{1.860167in}}%
\pgfpathlineto{\pgfqpoint{8.575563in}{1.825873in}}%
\pgfpathlineto{\pgfqpoint{8.584370in}{1.867054in}}%
\pgfpathlineto{\pgfqpoint{8.593177in}{1.750370in}}%
\pgfpathlineto{\pgfqpoint{8.601984in}{1.757229in}}%
\pgfpathlineto{\pgfqpoint{8.610791in}{2.059213in}}%
\pgfpathlineto{\pgfqpoint{8.619597in}{2.045467in}}%
\pgfpathlineto{\pgfqpoint{8.628404in}{2.278806in}}%
\pgfpathlineto{\pgfqpoint{8.637211in}{2.271947in}}%
\pgfpathlineto{\pgfqpoint{8.646018in}{2.134687in}}%
\pgfpathlineto{\pgfqpoint{8.654825in}{2.086648in}}%
\pgfpathlineto{\pgfqpoint{8.663632in}{1.523918in}}%
\pgfpathlineto{\pgfqpoint{8.672438in}{1.407234in}}%
\pgfpathlineto{\pgfqpoint{8.681245in}{1.523918in}}%
\pgfpathlineto{\pgfqpoint{8.690052in}{1.503314in}}%
\pgfpathlineto{\pgfqpoint{8.698859in}{1.722935in}}%
\pgfpathlineto{\pgfqpoint{8.707666in}{1.668037in}}%
\pgfpathlineto{\pgfqpoint{8.716472in}{1.654291in}}%
\pgfpathlineto{\pgfqpoint{8.725279in}{1.585675in}}%
\pgfpathlineto{\pgfqpoint{8.734086in}{1.592534in}}%
\pgfpathlineto{\pgfqpoint{8.742893in}{1.976851in}}%
\pgfpathlineto{\pgfqpoint{8.751700in}{2.011173in}}%
\pgfpathlineto{\pgfqpoint{8.760507in}{2.134687in}}%
\pgfpathlineto{\pgfqpoint{8.769313in}{2.203303in}}%
\pgfpathlineto{\pgfqpoint{8.778120in}{1.901348in}}%
\pgfpathlineto{\pgfqpoint{8.786927in}{1.976851in}}%
\pgfpathlineto{\pgfqpoint{8.795734in}{2.333705in}}%
\pgfpathlineto{\pgfqpoint{8.804541in}{2.093507in}}%
\pgfpathlineto{\pgfqpoint{8.813347in}{2.169009in}}%
\pgfpathlineto{\pgfqpoint{8.822154in}{2.148405in}}%
\pgfpathlineto{\pgfqpoint{8.830961in}{2.134687in}}%
\pgfpathlineto{\pgfqpoint{8.848575in}{2.175868in}}%
\pgfpathlineto{\pgfqpoint{8.866188in}{1.832732in}}%
\pgfpathlineto{\pgfqpoint{8.874995in}{1.860167in}}%
\pgfpathlineto{\pgfqpoint{8.883802in}{1.867054in}}%
\pgfpathlineto{\pgfqpoint{8.892609in}{1.976851in}}%
\pgfpathlineto{\pgfqpoint{8.901416in}{1.997427in}}%
\pgfpathlineto{\pgfqpoint{8.910222in}{1.942529in}}%
\pgfpathlineto{\pgfqpoint{8.919029in}{1.750370in}}%
\pgfpathlineto{\pgfqpoint{8.927836in}{1.777834in}}%
\pgfpathlineto{\pgfqpoint{8.936643in}{2.011173in}}%
\pgfpathlineto{\pgfqpoint{8.945450in}{1.819015in}}%
\pgfpathlineto{\pgfqpoint{8.954257in}{1.716076in}}%
\pgfpathlineto{\pgfqpoint{8.963063in}{1.592534in}}%
\pgfpathlineto{\pgfqpoint{8.971870in}{1.647432in}}%
\pgfpathlineto{\pgfqpoint{8.980677in}{1.619997in}}%
\pgfpathlineto{\pgfqpoint{8.989484in}{1.441556in}}%
\pgfpathlineto{\pgfqpoint{8.998291in}{1.160177in}}%
\pgfpathlineto{\pgfqpoint{9.007097in}{1.441556in}}%
\pgfpathlineto{\pgfqpoint{9.015904in}{1.523918in}}%
\pgfpathlineto{\pgfqpoint{9.024711in}{1.812128in}}%
\pgfpathlineto{\pgfqpoint{9.033518in}{2.024891in}}%
\pgfpathlineto{\pgfqpoint{9.042325in}{1.942529in}}%
\pgfpathlineto{\pgfqpoint{9.051132in}{2.114111in}}%
\pgfpathlineto{\pgfqpoint{9.059938in}{2.340563in}}%
\pgfpathlineto{\pgfqpoint{9.068745in}{2.265089in}}%
\pgfpathlineto{\pgfqpoint{9.077552in}{2.162151in}}%
\pgfpathlineto{\pgfqpoint{9.086359in}{1.873913in}}%
\pgfpathlineto{\pgfqpoint{9.095166in}{1.791551in}}%
\pgfpathlineto{\pgfqpoint{9.103972in}{1.640574in}}%
\pgfpathlineto{\pgfqpoint{9.112779in}{1.599393in}}%
\pgfpathlineto{\pgfqpoint{9.121586in}{1.633715in}}%
\pgfpathlineto{\pgfqpoint{9.130393in}{1.441556in}}%
\pgfpathlineto{\pgfqpoint{9.139200in}{1.379799in}}%
\pgfpathlineto{\pgfqpoint{9.148007in}{1.146460in}}%
\pgfpathlineto{\pgfqpoint{9.156813in}{1.091561in}}%
\pgfpathlineto{\pgfqpoint{9.165620in}{1.153319in}}%
\pgfpathlineto{\pgfqpoint{9.174427in}{1.173895in}}%
\pgfpathlineto{\pgfqpoint{9.200847in}{2.079789in}}%
\pgfpathlineto{\pgfqpoint{9.209654in}{2.244484in}}%
\pgfpathlineto{\pgfqpoint{9.218461in}{2.086648in}}%
\pgfpathlineto{\pgfqpoint{9.227268in}{2.038608in}}%
\pgfpathlineto{\pgfqpoint{9.236075in}{2.141546in}}%
\pgfpathlineto{\pgfqpoint{9.244882in}{1.784692in}}%
\pgfpathlineto{\pgfqpoint{9.253688in}{1.757229in}}%
\pgfpathlineto{\pgfqpoint{9.271302in}{2.388603in}}%
\pgfpathlineto{\pgfqpoint{9.280109in}{2.539581in}}%
\pgfpathlineto{\pgfqpoint{9.288916in}{2.738598in}}%
\pgfpathlineto{\pgfqpoint{9.297722in}{2.621942in}}%
\pgfpathlineto{\pgfqpoint{9.315336in}{2.464106in}}%
\pgfpathlineto{\pgfqpoint{9.324143in}{2.072930in}}%
\pgfpathlineto{\pgfqpoint{9.332950in}{2.368027in}}%
\pgfpathlineto{\pgfqpoint{9.341757in}{2.169009in}}%
\pgfpathlineto{\pgfqpoint{9.350563in}{2.066071in}}%
\pgfpathlineto{\pgfqpoint{9.359370in}{1.770975in}}%
\pgfpathlineto{\pgfqpoint{9.368177in}{1.757229in}}%
\pgfpathlineto{\pgfqpoint{9.376984in}{1.853308in}}%
\pgfpathlineto{\pgfqpoint{9.385791in}{2.169009in}}%
\pgfpathlineto{\pgfqpoint{9.394597in}{1.832732in}}%
\pgfpathlineto{\pgfqpoint{9.403404in}{1.551353in}}%
\pgfpathlineto{\pgfqpoint{9.412211in}{1.873913in}}%
\pgfpathlineto{\pgfqpoint{9.421018in}{1.935670in}}%
\pgfpathlineto{\pgfqpoint{9.429825in}{1.585675in}}%
\pgfpathlineto{\pgfqpoint{9.438632in}{1.839591in}}%
\pgfpathlineto{\pgfqpoint{9.447438in}{1.716076in}}%
\pgfpathlineto{\pgfqpoint{9.456245in}{1.633715in}}%
\pgfpathlineto{\pgfqpoint{9.465052in}{1.942529in}}%
\pgfpathlineto{\pgfqpoint{9.473859in}{1.805269in}}%
\pgfpathlineto{\pgfqpoint{9.482666in}{1.757229in}}%
\pgfpathlineto{\pgfqpoint{9.491472in}{1.585675in}}%
\pgfpathlineto{\pgfqpoint{9.500279in}{2.072930in}}%
\pgfpathlineto{\pgfqpoint{9.509086in}{2.271947in}}%
\pgfpathlineto{\pgfqpoint{9.517893in}{2.004286in}}%
\pgfpathlineto{\pgfqpoint{9.526700in}{1.860167in}}%
\pgfpathlineto{\pgfqpoint{9.535507in}{2.052326in}}%
\pgfpathlineto{\pgfqpoint{9.544313in}{2.031749in}}%
\pgfpathlineto{\pgfqpoint{9.553120in}{1.956246in}}%
\pgfpathlineto{\pgfqpoint{9.561927in}{1.976851in}}%
\pgfpathlineto{\pgfqpoint{9.570734in}{1.915094in}}%
\pgfpathlineto{\pgfqpoint{9.579541in}{2.388603in}}%
\pgfpathlineto{\pgfqpoint{9.588347in}{2.251343in}}%
\pgfpathlineto{\pgfqpoint{9.597154in}{1.983710in}}%
\pgfpathlineto{\pgfqpoint{9.605961in}{1.928811in}}%
\pgfpathlineto{\pgfqpoint{9.614768in}{2.038608in}}%
\pgfpathlineto{\pgfqpoint{9.623575in}{2.093507in}}%
\pgfpathlineto{\pgfqpoint{9.632382in}{2.079789in}}%
\pgfpathlineto{\pgfqpoint{9.641188in}{1.716076in}}%
\pgfpathlineto{\pgfqpoint{9.649995in}{1.894489in}}%
\pgfpathlineto{\pgfqpoint{9.658802in}{1.544494in}}%
\pgfpathlineto{\pgfqpoint{9.667609in}{1.318014in}}%
\pgfpathlineto{\pgfqpoint{9.676416in}{1.400376in}}%
\pgfpathlineto{\pgfqpoint{9.685222in}{1.585675in}}%
\pgfpathlineto{\pgfqpoint{9.694029in}{1.517031in}}%
\pgfpathlineto{\pgfqpoint{9.702836in}{1.873913in}}%
\pgfpathlineto{\pgfqpoint{9.711643in}{1.983710in}}%
\pgfpathlineto{\pgfqpoint{9.720450in}{2.018032in}}%
\pgfpathlineto{\pgfqpoint{9.729257in}{2.278806in}}%
\pgfpathlineto{\pgfqpoint{9.738063in}{2.031749in}}%
\pgfpathlineto{\pgfqpoint{9.746870in}{1.825873in}}%
\pgfpathlineto{\pgfqpoint{9.755677in}{2.024891in}}%
\pgfpathlineto{\pgfqpoint{9.764484in}{2.086648in}}%
\pgfpathlineto{\pgfqpoint{9.773291in}{1.551353in}}%
\pgfpathlineto{\pgfqpoint{9.782097in}{1.681754in}}%
\pgfpathlineto{\pgfqpoint{9.790904in}{1.668037in}}%
\pgfpathlineto{\pgfqpoint{9.799711in}{2.340563in}}%
\pgfpathlineto{\pgfqpoint{9.808518in}{2.031749in}}%
\pgfpathlineto{\pgfqpoint{9.817325in}{1.894489in}}%
\pgfpathlineto{\pgfqpoint{9.826132in}{1.455274in}}%
\pgfpathlineto{\pgfqpoint{9.834938in}{1.530777in}}%
\pgfpathlineto{\pgfqpoint{9.843745in}{1.345477in}}%
\pgfpathlineto{\pgfqpoint{9.852552in}{0.981737in}}%
\pgfpathlineto{\pgfqpoint{9.861359in}{1.235680in}}%
\pgfpathlineto{\pgfqpoint{9.870166in}{1.105279in}}%
\pgfpathlineto{\pgfqpoint{9.878972in}{1.510172in}}%
\pgfpathlineto{\pgfqpoint{9.887779in}{1.551353in}}%
\pgfpathlineto{\pgfqpoint{9.896586in}{2.114111in}}%
\pgfpathlineto{\pgfqpoint{9.905393in}{2.258230in}}%
\pgfpathlineto{\pgfqpoint{9.923007in}{2.361168in}}%
\pgfpathlineto{\pgfqpoint{9.931813in}{2.244484in}}%
\pgfpathlineto{\pgfqpoint{9.949427in}{1.969992in}}%
\pgfpathlineto{\pgfqpoint{9.949427in}{1.969992in}}%
\pgfusepath{stroke}%
\end{pgfscope}%
\begin{pgfscope}%
\pgfpathrectangle{\pgfqpoint{0.702268in}{0.521603in}}{\pgfqpoint{9.687500in}{4.235000in}}%
\pgfusepath{clip}%
\pgfsetrectcap%
\pgfsetroundjoin%
\pgfsetlinewidth{0.501875pt}%
\definecolor{currentstroke}{rgb}{0.501961,0.501961,0.501961}%
\pgfsetstrokecolor{currentstroke}%
\pgfsetstrokeopacity{0.250000}%
\pgfsetdash{}{0pt}%
\pgfpathmoveto{\pgfqpoint{1.142609in}{4.433702in}}%
\pgfpathlineto{\pgfqpoint{1.151416in}{3.884690in}}%
\pgfpathlineto{\pgfqpoint{1.169029in}{2.594507in}}%
\pgfpathlineto{\pgfqpoint{1.177836in}{2.244484in}}%
\pgfpathlineto{\pgfqpoint{1.186643in}{1.942529in}}%
\pgfpathlineto{\pgfqpoint{1.195450in}{1.908207in}}%
\pgfpathlineto{\pgfqpoint{1.204257in}{1.963133in}}%
\pgfpathlineto{\pgfqpoint{1.213063in}{1.976851in}}%
\pgfpathlineto{\pgfqpoint{1.221870in}{1.887630in}}%
\pgfpathlineto{\pgfqpoint{1.230677in}{1.942529in}}%
\pgfpathlineto{\pgfqpoint{1.239484in}{1.880772in}}%
\pgfpathlineto{\pgfqpoint{1.248291in}{2.326846in}}%
\pgfpathlineto{\pgfqpoint{1.257097in}{2.429784in}}%
\pgfpathlineto{\pgfqpoint{1.265904in}{2.285665in}}%
\pgfpathlineto{\pgfqpoint{1.274711in}{2.292524in}}%
\pgfpathlineto{\pgfqpoint{1.283518in}{2.333705in}}%
\pgfpathlineto{\pgfqpoint{1.292325in}{2.422925in}}%
\pgfpathlineto{\pgfqpoint{1.301132in}{2.175868in}}%
\pgfpathlineto{\pgfqpoint{1.309938in}{2.340563in}}%
\pgfpathlineto{\pgfqpoint{1.318745in}{2.162151in}}%
\pgfpathlineto{\pgfqpoint{1.327552in}{2.285665in}}%
\pgfpathlineto{\pgfqpoint{1.336359in}{2.429784in}}%
\pgfpathlineto{\pgfqpoint{1.345166in}{2.361168in}}%
\pgfpathlineto{\pgfqpoint{1.353972in}{2.086648in}}%
\pgfpathlineto{\pgfqpoint{1.362779in}{1.729794in}}%
\pgfpathlineto{\pgfqpoint{1.371586in}{1.997427in}}%
\pgfpathlineto{\pgfqpoint{1.380393in}{1.558212in}}%
\pgfpathlineto{\pgfqpoint{1.389200in}{1.551353in}}%
\pgfpathlineto{\pgfqpoint{1.398007in}{1.661150in}}%
\pgfpathlineto{\pgfqpoint{1.406813in}{2.004286in}}%
\pgfpathlineto{\pgfqpoint{1.415620in}{2.086648in}}%
\pgfpathlineto{\pgfqpoint{1.424427in}{1.908207in}}%
\pgfpathlineto{\pgfqpoint{1.433234in}{1.860167in}}%
\pgfpathlineto{\pgfqpoint{1.442041in}{2.223908in}}%
\pgfpathlineto{\pgfqpoint{1.450847in}{2.374885in}}%
\pgfpathlineto{\pgfqpoint{1.459654in}{2.422925in}}%
\pgfpathlineto{\pgfqpoint{1.468461in}{2.155292in}}%
\pgfpathlineto{\pgfqpoint{1.477268in}{2.182727in}}%
\pgfpathlineto{\pgfqpoint{1.486075in}{2.230767in}}%
\pgfpathlineto{\pgfqpoint{1.494882in}{2.409207in}}%
\pgfpathlineto{\pgfqpoint{1.503688in}{2.464106in}}%
\pgfpathlineto{\pgfqpoint{1.512495in}{2.237625in}}%
\pgfpathlineto{\pgfqpoint{1.521302in}{2.313128in}}%
\pgfpathlineto{\pgfqpoint{1.530109in}{2.470965in}}%
\pgfpathlineto{\pgfqpoint{1.538916in}{2.148405in}}%
\pgfpathlineto{\pgfqpoint{1.547722in}{2.155292in}}%
\pgfpathlineto{\pgfqpoint{1.556529in}{1.674896in}}%
\pgfpathlineto{\pgfqpoint{1.565336in}{1.510172in}}%
\pgfpathlineto{\pgfqpoint{1.574143in}{1.503314in}}%
\pgfpathlineto{\pgfqpoint{1.582950in}{1.468992in}}%
\pgfpathlineto{\pgfqpoint{1.591757in}{1.805269in}}%
\pgfpathlineto{\pgfqpoint{1.600563in}{1.585675in}}%
\pgfpathlineto{\pgfqpoint{1.609370in}{1.633715in}}%
\pgfpathlineto{\pgfqpoint{1.618177in}{1.606252in}}%
\pgfpathlineto{\pgfqpoint{1.626984in}{1.366053in}}%
\pgfpathlineto{\pgfqpoint{1.635791in}{1.963133in}}%
\pgfpathlineto{\pgfqpoint{1.644597in}{1.921953in}}%
\pgfpathlineto{\pgfqpoint{1.653404in}{2.024891in}}%
\pgfpathlineto{\pgfqpoint{1.662211in}{2.024891in}}%
\pgfpathlineto{\pgfqpoint{1.671018in}{1.942529in}}%
\pgfpathlineto{\pgfqpoint{1.679825in}{2.203303in}}%
\pgfpathlineto{\pgfqpoint{1.688632in}{2.038608in}}%
\pgfpathlineto{\pgfqpoint{1.697438in}{2.258230in}}%
\pgfpathlineto{\pgfqpoint{1.706245in}{2.525863in}}%
\pgfpathlineto{\pgfqpoint{1.715052in}{2.546468in}}%
\pgfpathlineto{\pgfqpoint{1.723859in}{2.491541in}}%
\pgfpathlineto{\pgfqpoint{1.732666in}{2.251343in}}%
\pgfpathlineto{\pgfqpoint{1.741472in}{2.107252in}}%
\pgfpathlineto{\pgfqpoint{1.750279in}{2.203303in}}%
\pgfpathlineto{\pgfqpoint{1.776700in}{1.523918in}}%
\pgfpathlineto{\pgfqpoint{1.785507in}{1.764116in}}%
\pgfpathlineto{\pgfqpoint{1.803120in}{1.661150in}}%
\pgfpathlineto{\pgfqpoint{1.811927in}{1.468992in}}%
\pgfpathlineto{\pgfqpoint{1.820734in}{1.729794in}}%
\pgfpathlineto{\pgfqpoint{1.829541in}{1.764116in}}%
\pgfpathlineto{\pgfqpoint{1.838347in}{2.045467in}}%
\pgfpathlineto{\pgfqpoint{1.847154in}{1.681754in}}%
\pgfpathlineto{\pgfqpoint{1.855961in}{1.839591in}}%
\pgfpathlineto{\pgfqpoint{1.864768in}{2.278806in}}%
\pgfpathlineto{\pgfqpoint{1.873575in}{2.271947in}}%
\pgfpathlineto{\pgfqpoint{1.882382in}{2.148405in}}%
\pgfpathlineto{\pgfqpoint{1.891188in}{2.107252in}}%
\pgfpathlineto{\pgfqpoint{1.899995in}{2.299383in}}%
\pgfpathlineto{\pgfqpoint{1.908802in}{2.107252in}}%
\pgfpathlineto{\pgfqpoint{1.917609in}{1.990569in}}%
\pgfpathlineto{\pgfqpoint{1.926416in}{2.011173in}}%
\pgfpathlineto{\pgfqpoint{1.935222in}{1.764116in}}%
\pgfpathlineto{\pgfqpoint{1.944029in}{2.182727in}}%
\pgfpathlineto{\pgfqpoint{1.952836in}{2.244484in}}%
\pgfpathlineto{\pgfqpoint{1.961643in}{1.976851in}}%
\pgfpathlineto{\pgfqpoint{1.970450in}{1.928811in}}%
\pgfpathlineto{\pgfqpoint{1.979257in}{1.668037in}}%
\pgfpathlineto{\pgfqpoint{1.988063in}{1.668037in}}%
\pgfpathlineto{\pgfqpoint{1.996870in}{1.853308in}}%
\pgfpathlineto{\pgfqpoint{2.005677in}{1.825873in}}%
\pgfpathlineto{\pgfqpoint{2.014484in}{2.313128in}}%
\pgfpathlineto{\pgfqpoint{2.023291in}{2.230767in}}%
\pgfpathlineto{\pgfqpoint{2.032097in}{2.333705in}}%
\pgfpathlineto{\pgfqpoint{2.040904in}{2.162151in}}%
\pgfpathlineto{\pgfqpoint{2.049711in}{2.059213in}}%
\pgfpathlineto{\pgfqpoint{2.058518in}{2.127829in}}%
\pgfpathlineto{\pgfqpoint{2.067325in}{1.990569in}}%
\pgfpathlineto{\pgfqpoint{2.076132in}{1.915094in}}%
\pgfpathlineto{\pgfqpoint{2.084938in}{2.093507in}}%
\pgfpathlineto{\pgfqpoint{2.093745in}{1.777834in}}%
\pgfpathlineto{\pgfqpoint{2.102552in}{1.887630in}}%
\pgfpathlineto{\pgfqpoint{2.111359in}{1.647432in}}%
\pgfpathlineto{\pgfqpoint{2.120166in}{1.736653in}}%
\pgfpathlineto{\pgfqpoint{2.128972in}{1.674896in}}%
\pgfpathlineto{\pgfqpoint{2.137779in}{1.716076in}}%
\pgfpathlineto{\pgfqpoint{2.146586in}{1.921953in}}%
\pgfpathlineto{\pgfqpoint{2.155393in}{1.983710in}}%
\pgfpathlineto{\pgfqpoint{2.164200in}{1.819015in}}%
\pgfpathlineto{\pgfqpoint{2.173007in}{2.011173in}}%
\pgfpathlineto{\pgfqpoint{2.181813in}{1.956246in}}%
\pgfpathlineto{\pgfqpoint{2.190620in}{2.169009in}}%
\pgfpathlineto{\pgfqpoint{2.199427in}{2.512146in}}%
\pgfpathlineto{\pgfqpoint{2.208234in}{2.608225in}}%
\pgfpathlineto{\pgfqpoint{2.217041in}{2.093507in}}%
\pgfpathlineto{\pgfqpoint{2.225847in}{1.997427in}}%
\pgfpathlineto{\pgfqpoint{2.234654in}{2.292524in}}%
\pgfpathlineto{\pgfqpoint{2.243461in}{1.777834in}}%
\pgfpathlineto{\pgfqpoint{2.252268in}{2.011173in}}%
\pgfpathlineto{\pgfqpoint{2.261075in}{1.908207in}}%
\pgfpathlineto{\pgfqpoint{2.269882in}{2.038608in}}%
\pgfpathlineto{\pgfqpoint{2.278688in}{1.963133in}}%
\pgfpathlineto{\pgfqpoint{2.287495in}{2.024891in}}%
\pgfpathlineto{\pgfqpoint{2.296302in}{2.141546in}}%
\pgfpathlineto{\pgfqpoint{2.305109in}{2.162151in}}%
\pgfpathlineto{\pgfqpoint{2.313916in}{2.210190in}}%
\pgfpathlineto{\pgfqpoint{2.322722in}{2.532722in}}%
\pgfpathlineto{\pgfqpoint{2.331529in}{2.244484in}}%
\pgfpathlineto{\pgfqpoint{2.340336in}{2.368027in}}%
\pgfpathlineto{\pgfqpoint{2.349143in}{1.942529in}}%
\pgfpathlineto{\pgfqpoint{2.357950in}{1.338618in}}%
\pgfpathlineto{\pgfqpoint{2.366757in}{1.379799in}}%
\pgfpathlineto{\pgfqpoint{2.375563in}{1.475878in}}%
\pgfpathlineto{\pgfqpoint{2.384370in}{1.366053in}}%
\pgfpathlineto{\pgfqpoint{2.393177in}{1.688613in}}%
\pgfpathlineto{\pgfqpoint{2.401984in}{1.812128in}}%
\pgfpathlineto{\pgfqpoint{2.410791in}{1.873913in}}%
\pgfpathlineto{\pgfqpoint{2.419597in}{2.107252in}}%
\pgfpathlineto{\pgfqpoint{2.428404in}{1.599393in}}%
\pgfpathlineto{\pgfqpoint{2.437211in}{1.571958in}}%
\pgfpathlineto{\pgfqpoint{2.446018in}{1.969992in}}%
\pgfpathlineto{\pgfqpoint{2.454825in}{2.175868in}}%
\pgfpathlineto{\pgfqpoint{2.463632in}{1.880772in}}%
\pgfpathlineto{\pgfqpoint{2.472438in}{1.757229in}}%
\pgfpathlineto{\pgfqpoint{2.481245in}{1.668037in}}%
\pgfpathlineto{\pgfqpoint{2.490052in}{1.695472in}}%
\pgfpathlineto{\pgfqpoint{2.498859in}{1.695472in}}%
\pgfpathlineto{\pgfqpoint{2.516472in}{2.038608in}}%
\pgfpathlineto{\pgfqpoint{2.525279in}{2.155292in}}%
\pgfpathlineto{\pgfqpoint{2.534086in}{2.107252in}}%
\pgfpathlineto{\pgfqpoint{2.542893in}{1.770975in}}%
\pgfpathlineto{\pgfqpoint{2.551700in}{1.935670in}}%
\pgfpathlineto{\pgfqpoint{2.560507in}{2.072930in}}%
\pgfpathlineto{\pgfqpoint{2.569313in}{1.915094in}}%
\pgfpathlineto{\pgfqpoint{2.578120in}{1.942529in}}%
\pgfpathlineto{\pgfqpoint{2.586927in}{1.805269in}}%
\pgfpathlineto{\pgfqpoint{2.595734in}{1.825873in}}%
\pgfpathlineto{\pgfqpoint{2.604541in}{1.661150in}}%
\pgfpathlineto{\pgfqpoint{2.613347in}{1.887630in}}%
\pgfpathlineto{\pgfqpoint{2.622154in}{1.908207in}}%
\pgfpathlineto{\pgfqpoint{2.630961in}{1.798410in}}%
\pgfpathlineto{\pgfqpoint{2.639768in}{1.468992in}}%
\pgfpathlineto{\pgfqpoint{2.648575in}{1.544494in}}%
\pgfpathlineto{\pgfqpoint{2.657382in}{1.468992in}}%
\pgfpathlineto{\pgfqpoint{2.666188in}{1.592534in}}%
\pgfpathlineto{\pgfqpoint{2.674995in}{1.798410in}}%
\pgfpathlineto{\pgfqpoint{2.683802in}{1.825873in}}%
\pgfpathlineto{\pgfqpoint{2.692609in}{1.585675in}}%
\pgfpathlineto{\pgfqpoint{2.701416in}{1.695472in}}%
\pgfpathlineto{\pgfqpoint{2.710222in}{1.674896in}}%
\pgfpathlineto{\pgfqpoint{2.719029in}{1.867054in}}%
\pgfpathlineto{\pgfqpoint{2.727836in}{1.942529in}}%
\pgfpathlineto{\pgfqpoint{2.736643in}{1.983710in}}%
\pgfpathlineto{\pgfqpoint{2.745450in}{1.764116in}}%
\pgfpathlineto{\pgfqpoint{2.754257in}{1.647432in}}%
\pgfpathlineto{\pgfqpoint{2.771870in}{1.805269in}}%
\pgfpathlineto{\pgfqpoint{2.780677in}{2.052326in}}%
\pgfpathlineto{\pgfqpoint{2.789484in}{2.079789in}}%
\pgfpathlineto{\pgfqpoint{2.798291in}{2.052326in}}%
\pgfpathlineto{\pgfqpoint{2.807097in}{2.127829in}}%
\pgfpathlineto{\pgfqpoint{2.815904in}{2.306269in}}%
\pgfpathlineto{\pgfqpoint{2.824711in}{2.093507in}}%
\pgfpathlineto{\pgfqpoint{2.833518in}{2.011173in}}%
\pgfpathlineto{\pgfqpoint{2.842325in}{2.134687in}}%
\pgfpathlineto{\pgfqpoint{2.851132in}{2.319987in}}%
\pgfpathlineto{\pgfqpoint{2.859938in}{2.374885in}}%
\pgfpathlineto{\pgfqpoint{2.877552in}{1.894489in}}%
\pgfpathlineto{\pgfqpoint{2.886359in}{2.114111in}}%
\pgfpathlineto{\pgfqpoint{2.895166in}{1.956246in}}%
\pgfpathlineto{\pgfqpoint{2.903972in}{1.921953in}}%
\pgfpathlineto{\pgfqpoint{2.912779in}{1.819015in}}%
\pgfpathlineto{\pgfqpoint{2.921586in}{1.647432in}}%
\pgfpathlineto{\pgfqpoint{2.930393in}{1.668037in}}%
\pgfpathlineto{\pgfqpoint{2.939200in}{1.585675in}}%
\pgfpathlineto{\pgfqpoint{2.948007in}{1.640574in}}%
\pgfpathlineto{\pgfqpoint{2.956813in}{1.434698in}}%
\pgfpathlineto{\pgfqpoint{2.974427in}{2.004286in}}%
\pgfpathlineto{\pgfqpoint{2.992041in}{1.901348in}}%
\pgfpathlineto{\pgfqpoint{3.000847in}{1.860167in}}%
\pgfpathlineto{\pgfqpoint{3.009654in}{1.736653in}}%
\pgfpathlineto{\pgfqpoint{3.018461in}{1.880772in}}%
\pgfpathlineto{\pgfqpoint{3.027268in}{1.990569in}}%
\pgfpathlineto{\pgfqpoint{3.036075in}{2.258230in}}%
\pgfpathlineto{\pgfqpoint{3.044882in}{2.223908in}}%
\pgfpathlineto{\pgfqpoint{3.053688in}{2.086648in}}%
\pgfpathlineto{\pgfqpoint{3.062495in}{1.860167in}}%
\pgfpathlineto{\pgfqpoint{3.071302in}{2.086648in}}%
\pgfpathlineto{\pgfqpoint{3.080109in}{2.244484in}}%
\pgfpathlineto{\pgfqpoint{3.088916in}{2.278806in}}%
\pgfpathlineto{\pgfqpoint{3.097722in}{1.839591in}}%
\pgfpathlineto{\pgfqpoint{3.106529in}{1.510172in}}%
\pgfpathlineto{\pgfqpoint{3.115336in}{1.791551in}}%
\pgfpathlineto{\pgfqpoint{3.124143in}{1.702331in}}%
\pgfpathlineto{\pgfqpoint{3.132950in}{1.558212in}}%
\pgfpathlineto{\pgfqpoint{3.150563in}{1.901348in}}%
\pgfpathlineto{\pgfqpoint{3.159370in}{1.867054in}}%
\pgfpathlineto{\pgfqpoint{3.168177in}{2.052326in}}%
\pgfpathlineto{\pgfqpoint{3.176984in}{2.409207in}}%
\pgfpathlineto{\pgfqpoint{3.185791in}{2.354309in}}%
\pgfpathlineto{\pgfqpoint{3.194597in}{2.244484in}}%
\pgfpathlineto{\pgfqpoint{3.203404in}{1.880772in}}%
\pgfpathlineto{\pgfqpoint{3.212211in}{1.908207in}}%
\pgfpathlineto{\pgfqpoint{3.221018in}{1.963133in}}%
\pgfpathlineto{\pgfqpoint{3.229825in}{1.880772in}}%
\pgfpathlineto{\pgfqpoint{3.238632in}{2.079789in}}%
\pgfpathlineto{\pgfqpoint{3.247438in}{2.127829in}}%
\pgfpathlineto{\pgfqpoint{3.256245in}{1.619997in}}%
\pgfpathlineto{\pgfqpoint{3.265052in}{1.997427in}}%
\pgfpathlineto{\pgfqpoint{3.273859in}{2.079789in}}%
\pgfpathlineto{\pgfqpoint{3.282666in}{2.127829in}}%
\pgfpathlineto{\pgfqpoint{3.291472in}{2.038608in}}%
\pgfpathlineto{\pgfqpoint{3.300279in}{1.832732in}}%
\pgfpathlineto{\pgfqpoint{3.309086in}{1.935670in}}%
\pgfpathlineto{\pgfqpoint{3.317893in}{2.086648in}}%
\pgfpathlineto{\pgfqpoint{3.326700in}{2.169009in}}%
\pgfpathlineto{\pgfqpoint{3.335507in}{1.935670in}}%
\pgfpathlineto{\pgfqpoint{3.344313in}{2.271947in}}%
\pgfpathlineto{\pgfqpoint{3.361927in}{1.963133in}}%
\pgfpathlineto{\pgfqpoint{3.370734in}{1.647432in}}%
\pgfpathlineto{\pgfqpoint{3.379541in}{1.695472in}}%
\pgfpathlineto{\pgfqpoint{3.388347in}{1.475878in}}%
\pgfpathlineto{\pgfqpoint{3.397154in}{1.873913in}}%
\pgfpathlineto{\pgfqpoint{3.405961in}{1.997427in}}%
\pgfpathlineto{\pgfqpoint{3.414768in}{2.045467in}}%
\pgfpathlineto{\pgfqpoint{3.423575in}{1.846450in}}%
\pgfpathlineto{\pgfqpoint{3.432382in}{1.695472in}}%
\pgfpathlineto{\pgfqpoint{3.441188in}{1.915094in}}%
\pgfpathlineto{\pgfqpoint{3.449995in}{1.976851in}}%
\pgfpathlineto{\pgfqpoint{3.458802in}{2.059213in}}%
\pgfpathlineto{\pgfqpoint{3.467609in}{1.921953in}}%
\pgfpathlineto{\pgfqpoint{3.476416in}{1.915094in}}%
\pgfpathlineto{\pgfqpoint{3.485222in}{1.819015in}}%
\pgfpathlineto{\pgfqpoint{3.502836in}{2.244484in}}%
\pgfpathlineto{\pgfqpoint{3.511643in}{2.271947in}}%
\pgfpathlineto{\pgfqpoint{3.520450in}{2.045467in}}%
\pgfpathlineto{\pgfqpoint{3.529257in}{2.388603in}}%
\pgfpathlineto{\pgfqpoint{3.538063in}{1.915094in}}%
\pgfpathlineto{\pgfqpoint{3.546870in}{1.805269in}}%
\pgfpathlineto{\pgfqpoint{3.555677in}{1.997427in}}%
\pgfpathlineto{\pgfqpoint{3.564484in}{1.894489in}}%
\pgfpathlineto{\pgfqpoint{3.573291in}{1.901348in}}%
\pgfpathlineto{\pgfqpoint{3.582097in}{1.894489in}}%
\pgfpathlineto{\pgfqpoint{3.590904in}{1.791551in}}%
\pgfpathlineto{\pgfqpoint{3.599711in}{1.585675in}}%
\pgfpathlineto{\pgfqpoint{3.608518in}{1.702331in}}%
\pgfpathlineto{\pgfqpoint{3.617325in}{1.585675in}}%
\pgfpathlineto{\pgfqpoint{3.626132in}{1.709190in}}%
\pgfpathlineto{\pgfqpoint{3.634938in}{1.386658in}}%
\pgfpathlineto{\pgfqpoint{3.643745in}{1.901348in}}%
\pgfpathlineto{\pgfqpoint{3.652552in}{1.551353in}}%
\pgfpathlineto{\pgfqpoint{3.661359in}{1.558212in}}%
\pgfpathlineto{\pgfqpoint{3.670166in}{1.901348in}}%
\pgfpathlineto{\pgfqpoint{3.678972in}{2.004286in}}%
\pgfpathlineto{\pgfqpoint{3.696586in}{2.251343in}}%
\pgfpathlineto{\pgfqpoint{3.705393in}{2.155292in}}%
\pgfpathlineto{\pgfqpoint{3.714200in}{2.114111in}}%
\pgfpathlineto{\pgfqpoint{3.723007in}{2.319987in}}%
\pgfpathlineto{\pgfqpoint{3.731813in}{2.457247in}}%
\pgfpathlineto{\pgfqpoint{3.740620in}{2.148405in}}%
\pgfpathlineto{\pgfqpoint{3.749427in}{2.086648in}}%
\pgfpathlineto{\pgfqpoint{3.758234in}{1.860167in}}%
\pgfpathlineto{\pgfqpoint{3.767041in}{2.086648in}}%
\pgfpathlineto{\pgfqpoint{3.775847in}{1.990569in}}%
\pgfpathlineto{\pgfqpoint{3.784654in}{1.942529in}}%
\pgfpathlineto{\pgfqpoint{3.793461in}{1.942529in}}%
\pgfpathlineto{\pgfqpoint{3.802268in}{1.935670in}}%
\pgfpathlineto{\pgfqpoint{3.811075in}{1.915094in}}%
\pgfpathlineto{\pgfqpoint{3.819882in}{1.764116in}}%
\pgfpathlineto{\pgfqpoint{3.828688in}{1.784692in}}%
\pgfpathlineto{\pgfqpoint{3.837495in}{2.004286in}}%
\pgfpathlineto{\pgfqpoint{3.846302in}{1.949388in}}%
\pgfpathlineto{\pgfqpoint{3.855109in}{1.475878in}}%
\pgfpathlineto{\pgfqpoint{3.863916in}{1.112138in}}%
\pgfpathlineto{\pgfqpoint{3.872722in}{1.263115in}}%
\pgfpathlineto{\pgfqpoint{3.881529in}{1.537636in}}%
\pgfpathlineto{\pgfqpoint{3.890336in}{1.517031in}}%
\pgfpathlineto{\pgfqpoint{3.899143in}{1.722935in}}%
\pgfpathlineto{\pgfqpoint{3.907950in}{1.640574in}}%
\pgfpathlineto{\pgfqpoint{3.916757in}{1.599393in}}%
\pgfpathlineto{\pgfqpoint{3.925563in}{1.928811in}}%
\pgfpathlineto{\pgfqpoint{3.934370in}{2.011173in}}%
\pgfpathlineto{\pgfqpoint{3.943177in}{2.258230in}}%
\pgfpathlineto{\pgfqpoint{3.951984in}{1.990569in}}%
\pgfpathlineto{\pgfqpoint{3.960791in}{2.011173in}}%
\pgfpathlineto{\pgfqpoint{3.969597in}{2.066071in}}%
\pgfpathlineto{\pgfqpoint{3.978404in}{2.038608in}}%
\pgfpathlineto{\pgfqpoint{3.987211in}{1.709190in}}%
\pgfpathlineto{\pgfqpoint{3.996018in}{2.148405in}}%
\pgfpathlineto{\pgfqpoint{4.013632in}{2.450388in}}%
\pgfpathlineto{\pgfqpoint{4.022438in}{2.450388in}}%
\pgfpathlineto{\pgfqpoint{4.031245in}{2.120970in}}%
\pgfpathlineto{\pgfqpoint{4.040052in}{2.127829in}}%
\pgfpathlineto{\pgfqpoint{4.048859in}{2.072930in}}%
\pgfpathlineto{\pgfqpoint{4.066472in}{2.031749in}}%
\pgfpathlineto{\pgfqpoint{4.075279in}{1.654291in}}%
\pgfpathlineto{\pgfqpoint{4.084086in}{1.654291in}}%
\pgfpathlineto{\pgfqpoint{4.092893in}{2.100365in}}%
\pgfpathlineto{\pgfqpoint{4.101700in}{2.155292in}}%
\pgfpathlineto{\pgfqpoint{4.110507in}{2.230767in}}%
\pgfpathlineto{\pgfqpoint{4.119313in}{1.736653in}}%
\pgfpathlineto{\pgfqpoint{4.128120in}{1.908207in}}%
\pgfpathlineto{\pgfqpoint{4.136927in}{1.750370in}}%
\pgfpathlineto{\pgfqpoint{4.145734in}{1.963133in}}%
\pgfpathlineto{\pgfqpoint{4.154541in}{2.086648in}}%
\pgfpathlineto{\pgfqpoint{4.163347in}{2.134687in}}%
\pgfpathlineto{\pgfqpoint{4.172154in}{1.722935in}}%
\pgfpathlineto{\pgfqpoint{4.189768in}{2.141546in}}%
\pgfpathlineto{\pgfqpoint{4.198575in}{1.949388in}}%
\pgfpathlineto{\pgfqpoint{4.207382in}{2.059213in}}%
\pgfpathlineto{\pgfqpoint{4.216188in}{2.196445in}}%
\pgfpathlineto{\pgfqpoint{4.224995in}{2.313128in}}%
\pgfpathlineto{\pgfqpoint{4.233802in}{2.127829in}}%
\pgfpathlineto{\pgfqpoint{4.242609in}{1.668037in}}%
\pgfpathlineto{\pgfqpoint{4.251416in}{1.832732in}}%
\pgfpathlineto{\pgfqpoint{4.260222in}{1.777834in}}%
\pgfpathlineto{\pgfqpoint{4.269029in}{1.846450in}}%
\pgfpathlineto{\pgfqpoint{4.277836in}{1.935670in}}%
\pgfpathlineto{\pgfqpoint{4.286643in}{1.770975in}}%
\pgfpathlineto{\pgfqpoint{4.295450in}{1.736653in}}%
\pgfpathlineto{\pgfqpoint{4.304257in}{1.661150in}}%
\pgfpathlineto{\pgfqpoint{4.313063in}{1.668037in}}%
\pgfpathlineto{\pgfqpoint{4.321870in}{1.702331in}}%
\pgfpathlineto{\pgfqpoint{4.330677in}{1.585675in}}%
\pgfpathlineto{\pgfqpoint{4.339484in}{1.784692in}}%
\pgfpathlineto{\pgfqpoint{4.348291in}{2.141546in}}%
\pgfpathlineto{\pgfqpoint{4.365904in}{1.681754in}}%
\pgfpathlineto{\pgfqpoint{4.374711in}{1.695472in}}%
\pgfpathlineto{\pgfqpoint{4.383518in}{1.812128in}}%
\pgfpathlineto{\pgfqpoint{4.392325in}{1.688613in}}%
\pgfpathlineto{\pgfqpoint{4.401132in}{1.393517in}}%
\pgfpathlineto{\pgfqpoint{4.409938in}{1.304296in}}%
\pgfpathlineto{\pgfqpoint{4.418745in}{1.462133in}}%
\pgfpathlineto{\pgfqpoint{4.427552in}{1.839591in}}%
\pgfpathlineto{\pgfqpoint{4.436359in}{2.148405in}}%
\pgfpathlineto{\pgfqpoint{4.445166in}{2.285665in}}%
\pgfpathlineto{\pgfqpoint{4.462779in}{1.482737in}}%
\pgfpathlineto{\pgfqpoint{4.471586in}{1.462133in}}%
\pgfpathlineto{\pgfqpoint{4.480393in}{1.640574in}}%
\pgfpathlineto{\pgfqpoint{4.489200in}{1.956246in}}%
\pgfpathlineto{\pgfqpoint{4.498007in}{1.400376in}}%
\pgfpathlineto{\pgfqpoint{4.506813in}{1.571958in}}%
\pgfpathlineto{\pgfqpoint{4.515620in}{2.004286in}}%
\pgfpathlineto{\pgfqpoint{4.524427in}{1.949388in}}%
\pgfpathlineto{\pgfqpoint{4.533234in}{1.606252in}}%
\pgfpathlineto{\pgfqpoint{4.542041in}{1.592534in}}%
\pgfpathlineto{\pgfqpoint{4.550847in}{1.963133in}}%
\pgfpathlineto{\pgfqpoint{4.559654in}{2.169009in}}%
\pgfpathlineto{\pgfqpoint{4.568461in}{2.258230in}}%
\pgfpathlineto{\pgfqpoint{4.577268in}{2.120970in}}%
\pgfpathlineto{\pgfqpoint{4.586075in}{2.120970in}}%
\pgfpathlineto{\pgfqpoint{4.594882in}{2.079789in}}%
\pgfpathlineto{\pgfqpoint{4.603688in}{2.086648in}}%
\pgfpathlineto{\pgfqpoint{4.612495in}{1.661150in}}%
\pgfpathlineto{\pgfqpoint{4.621302in}{1.551353in}}%
\pgfpathlineto{\pgfqpoint{4.630109in}{1.366053in}}%
\pgfpathlineto{\pgfqpoint{4.638916in}{1.414093in}}%
\pgfpathlineto{\pgfqpoint{4.647722in}{1.468992in}}%
\pgfpathlineto{\pgfqpoint{4.656529in}{1.654291in}}%
\pgfpathlineto{\pgfqpoint{4.665336in}{1.510172in}}%
\pgfpathlineto{\pgfqpoint{4.674143in}{1.558212in}}%
\pgfpathlineto{\pgfqpoint{4.682950in}{1.427839in}}%
\pgfpathlineto{\pgfqpoint{4.691757in}{1.709190in}}%
\pgfpathlineto{\pgfqpoint{4.700563in}{1.722935in}}%
\pgfpathlineto{\pgfqpoint{4.709370in}{1.551353in}}%
\pgfpathlineto{\pgfqpoint{4.718177in}{1.118997in}}%
\pgfpathlineto{\pgfqpoint{4.726984in}{1.263115in}}%
\pgfpathlineto{\pgfqpoint{4.735791in}{1.565071in}}%
\pgfpathlineto{\pgfqpoint{4.744597in}{1.654291in}}%
\pgfpathlineto{\pgfqpoint{4.753404in}{1.647432in}}%
\pgfpathlineto{\pgfqpoint{4.762211in}{1.475878in}}%
\pgfpathlineto{\pgfqpoint{4.771018in}{1.475878in}}%
\pgfpathlineto{\pgfqpoint{4.779825in}{1.537636in}}%
\pgfpathlineto{\pgfqpoint{4.788632in}{1.489596in}}%
\pgfpathlineto{\pgfqpoint{4.797438in}{1.654291in}}%
\pgfpathlineto{\pgfqpoint{4.806245in}{1.935670in}}%
\pgfpathlineto{\pgfqpoint{4.815052in}{1.764116in}}%
\pgfpathlineto{\pgfqpoint{4.823859in}{1.427839in}}%
\pgfpathlineto{\pgfqpoint{4.832666in}{1.359195in}}%
\pgfpathlineto{\pgfqpoint{4.841472in}{1.318014in}}%
\pgfpathlineto{\pgfqpoint{4.850279in}{1.194499in}}%
\pgfpathlineto{\pgfqpoint{4.859086in}{1.228822in}}%
\pgfpathlineto{\pgfqpoint{4.867893in}{1.215076in}}%
\pgfpathlineto{\pgfqpoint{4.876700in}{0.940584in}}%
\pgfpathlineto{\pgfqpoint{4.894313in}{1.537636in}}%
\pgfpathlineto{\pgfqpoint{4.903120in}{2.031749in}}%
\pgfpathlineto{\pgfqpoint{4.911927in}{2.059213in}}%
\pgfpathlineto{\pgfqpoint{4.920734in}{2.155292in}}%
\pgfpathlineto{\pgfqpoint{4.929541in}{1.825873in}}%
\pgfpathlineto{\pgfqpoint{4.938347in}{1.359195in}}%
\pgfpathlineto{\pgfqpoint{4.947154in}{1.592534in}}%
\pgfpathlineto{\pgfqpoint{4.955961in}{1.729794in}}%
\pgfpathlineto{\pgfqpoint{4.964768in}{1.825873in}}%
\pgfpathlineto{\pgfqpoint{4.973575in}{2.086648in}}%
\pgfpathlineto{\pgfqpoint{4.982382in}{1.894489in}}%
\pgfpathlineto{\pgfqpoint{4.991188in}{1.935670in}}%
\pgfpathlineto{\pgfqpoint{4.999995in}{1.983710in}}%
\pgfpathlineto{\pgfqpoint{5.008802in}{2.148405in}}%
\pgfpathlineto{\pgfqpoint{5.017609in}{2.237625in}}%
\pgfpathlineto{\pgfqpoint{5.026416in}{2.018032in}}%
\pgfpathlineto{\pgfqpoint{5.035222in}{1.997427in}}%
\pgfpathlineto{\pgfqpoint{5.044029in}{2.079789in}}%
\pgfpathlineto{\pgfqpoint{5.052836in}{1.887630in}}%
\pgfpathlineto{\pgfqpoint{5.061643in}{2.011173in}}%
\pgfpathlineto{\pgfqpoint{5.070450in}{1.743512in}}%
\pgfpathlineto{\pgfqpoint{5.079257in}{1.757229in}}%
\pgfpathlineto{\pgfqpoint{5.088063in}{1.963133in}}%
\pgfpathlineto{\pgfqpoint{5.096870in}{2.059213in}}%
\pgfpathlineto{\pgfqpoint{5.105677in}{2.210190in}}%
\pgfpathlineto{\pgfqpoint{5.114484in}{1.928811in}}%
\pgfpathlineto{\pgfqpoint{5.123291in}{2.018032in}}%
\pgfpathlineto{\pgfqpoint{5.132097in}{2.251343in}}%
\pgfpathlineto{\pgfqpoint{5.140904in}{1.764116in}}%
\pgfpathlineto{\pgfqpoint{5.149711in}{2.169009in}}%
\pgfpathlineto{\pgfqpoint{5.158518in}{2.182727in}}%
\pgfpathlineto{\pgfqpoint{5.167325in}{2.319987in}}%
\pgfpathlineto{\pgfqpoint{5.176132in}{2.498428in}}%
\pgfpathlineto{\pgfqpoint{5.184938in}{2.306269in}}%
\pgfpathlineto{\pgfqpoint{5.193745in}{2.189586in}}%
\pgfpathlineto{\pgfqpoint{5.202552in}{2.148405in}}%
\pgfpathlineto{\pgfqpoint{5.211359in}{2.354309in}}%
\pgfpathlineto{\pgfqpoint{5.220166in}{1.887630in}}%
\pgfpathlineto{\pgfqpoint{5.228972in}{1.798410in}}%
\pgfpathlineto{\pgfqpoint{5.237779in}{1.647432in}}%
\pgfpathlineto{\pgfqpoint{5.246586in}{1.743512in}}%
\pgfpathlineto{\pgfqpoint{5.255393in}{1.695472in}}%
\pgfpathlineto{\pgfqpoint{5.264200in}{1.661150in}}%
\pgfpathlineto{\pgfqpoint{5.273007in}{1.640574in}}%
\pgfpathlineto{\pgfqpoint{5.281813in}{1.942529in}}%
\pgfpathlineto{\pgfqpoint{5.290620in}{1.633715in}}%
\pgfpathlineto{\pgfqpoint{5.299427in}{1.585675in}}%
\pgfpathlineto{\pgfqpoint{5.308234in}{1.283720in}}%
\pgfpathlineto{\pgfqpoint{5.317041in}{1.366053in}}%
\pgfpathlineto{\pgfqpoint{5.325847in}{1.921953in}}%
\pgfpathlineto{\pgfqpoint{5.334654in}{2.182727in}}%
\pgfpathlineto{\pgfqpoint{5.343461in}{2.114111in}}%
\pgfpathlineto{\pgfqpoint{5.352268in}{2.477823in}}%
\pgfpathlineto{\pgfqpoint{5.361075in}{2.381744in}}%
\pgfpathlineto{\pgfqpoint{5.369882in}{2.203303in}}%
\pgfpathlineto{\pgfqpoint{5.378688in}{1.928811in}}%
\pgfpathlineto{\pgfqpoint{5.387495in}{1.812128in}}%
\pgfpathlineto{\pgfqpoint{5.396302in}{1.846450in}}%
\pgfpathlineto{\pgfqpoint{5.405109in}{1.750370in}}%
\pgfpathlineto{\pgfqpoint{5.413916in}{1.798410in}}%
\pgfpathlineto{\pgfqpoint{5.422722in}{1.530777in}}%
\pgfpathlineto{\pgfqpoint{5.431529in}{1.812128in}}%
\pgfpathlineto{\pgfqpoint{5.440336in}{1.949388in}}%
\pgfpathlineto{\pgfqpoint{5.449143in}{1.873913in}}%
\pgfpathlineto{\pgfqpoint{5.457950in}{1.496455in}}%
\pgfpathlineto{\pgfqpoint{5.466757in}{1.537636in}}%
\pgfpathlineto{\pgfqpoint{5.475563in}{1.695472in}}%
\pgfpathlineto{\pgfqpoint{5.484370in}{1.599393in}}%
\pgfpathlineto{\pgfqpoint{5.493177in}{1.585675in}}%
\pgfpathlineto{\pgfqpoint{5.501984in}{1.599393in}}%
\pgfpathlineto{\pgfqpoint{5.510791in}{1.661150in}}%
\pgfpathlineto{\pgfqpoint{5.519597in}{1.571958in}}%
\pgfpathlineto{\pgfqpoint{5.537211in}{1.935670in}}%
\pgfpathlineto{\pgfqpoint{5.546018in}{1.819015in}}%
\pgfpathlineto{\pgfqpoint{5.554825in}{1.949388in}}%
\pgfpathlineto{\pgfqpoint{5.563632in}{1.592534in}}%
\pgfpathlineto{\pgfqpoint{5.572438in}{1.770975in}}%
\pgfpathlineto{\pgfqpoint{5.581245in}{2.340563in}}%
\pgfpathlineto{\pgfqpoint{5.590052in}{2.484682in}}%
\pgfpathlineto{\pgfqpoint{5.598859in}{2.299383in}}%
\pgfpathlineto{\pgfqpoint{5.607666in}{2.319987in}}%
\pgfpathlineto{\pgfqpoint{5.616472in}{2.120970in}}%
\pgfpathlineto{\pgfqpoint{5.625279in}{2.230767in}}%
\pgfpathlineto{\pgfqpoint{5.634086in}{1.764116in}}%
\pgfpathlineto{\pgfqpoint{5.642893in}{1.825873in}}%
\pgfpathlineto{\pgfqpoint{5.651700in}{1.770975in}}%
\pgfpathlineto{\pgfqpoint{5.660507in}{1.551353in}}%
\pgfpathlineto{\pgfqpoint{5.669313in}{1.688613in}}%
\pgfpathlineto{\pgfqpoint{5.678120in}{2.066071in}}%
\pgfpathlineto{\pgfqpoint{5.704541in}{1.578816in}}%
\pgfpathlineto{\pgfqpoint{5.713347in}{1.613110in}}%
\pgfpathlineto{\pgfqpoint{5.722154in}{1.688613in}}%
\pgfpathlineto{\pgfqpoint{5.730961in}{2.114111in}}%
\pgfpathlineto{\pgfqpoint{5.739768in}{1.935670in}}%
\pgfpathlineto{\pgfqpoint{5.748575in}{2.052326in}}%
\pgfpathlineto{\pgfqpoint{5.757382in}{1.963133in}}%
\pgfpathlineto{\pgfqpoint{5.766188in}{1.983710in}}%
\pgfpathlineto{\pgfqpoint{5.774995in}{1.935670in}}%
\pgfpathlineto{\pgfqpoint{5.783802in}{2.086648in}}%
\pgfpathlineto{\pgfqpoint{5.792609in}{2.045467in}}%
\pgfpathlineto{\pgfqpoint{5.801416in}{2.278806in}}%
\pgfpathlineto{\pgfqpoint{5.810222in}{2.395462in}}%
\pgfpathlineto{\pgfqpoint{5.819029in}{2.196445in}}%
\pgfpathlineto{\pgfqpoint{5.827836in}{2.141546in}}%
\pgfpathlineto{\pgfqpoint{5.836643in}{2.038608in}}%
\pgfpathlineto{\pgfqpoint{5.845450in}{2.175868in}}%
\pgfpathlineto{\pgfqpoint{5.854257in}{2.079789in}}%
\pgfpathlineto{\pgfqpoint{5.863063in}{2.299383in}}%
\pgfpathlineto{\pgfqpoint{5.871870in}{2.326846in}}%
\pgfpathlineto{\pgfqpoint{5.880677in}{1.983710in}}%
\pgfpathlineto{\pgfqpoint{5.889484in}{1.846450in}}%
\pgfpathlineto{\pgfqpoint{5.898291in}{2.107252in}}%
\pgfpathlineto{\pgfqpoint{5.907097in}{2.196445in}}%
\pgfpathlineto{\pgfqpoint{5.915904in}{2.319987in}}%
\pgfpathlineto{\pgfqpoint{5.933518in}{2.004286in}}%
\pgfpathlineto{\pgfqpoint{5.942325in}{2.093507in}}%
\pgfpathlineto{\pgfqpoint{5.951132in}{2.141546in}}%
\pgfpathlineto{\pgfqpoint{5.959938in}{2.120970in}}%
\pgfpathlineto{\pgfqpoint{5.968745in}{2.093507in}}%
\pgfpathlineto{\pgfqpoint{5.977552in}{2.045467in}}%
\pgfpathlineto{\pgfqpoint{5.986359in}{2.402349in}}%
\pgfpathlineto{\pgfqpoint{5.995166in}{2.340563in}}%
\pgfpathlineto{\pgfqpoint{6.003972in}{1.880772in}}%
\pgfpathlineto{\pgfqpoint{6.021586in}{1.825873in}}%
\pgfpathlineto{\pgfqpoint{6.030393in}{1.894489in}}%
\pgfpathlineto{\pgfqpoint{6.039200in}{1.942529in}}%
\pgfpathlineto{\pgfqpoint{6.048007in}{1.681754in}}%
\pgfpathlineto{\pgfqpoint{6.056813in}{1.750370in}}%
\pgfpathlineto{\pgfqpoint{6.065620in}{1.297438in}}%
\pgfpathlineto{\pgfqpoint{6.074427in}{1.743512in}}%
\pgfpathlineto{\pgfqpoint{6.083234in}{2.086648in}}%
\pgfpathlineto{\pgfqpoint{6.092041in}{2.230767in}}%
\pgfpathlineto{\pgfqpoint{6.100847in}{2.532722in}}%
\pgfpathlineto{\pgfqpoint{6.109654in}{2.148405in}}%
\pgfpathlineto{\pgfqpoint{6.118461in}{2.045467in}}%
\pgfpathlineto{\pgfqpoint{6.127268in}{1.880772in}}%
\pgfpathlineto{\pgfqpoint{6.136075in}{1.613110in}}%
\pgfpathlineto{\pgfqpoint{6.144882in}{1.263115in}}%
\pgfpathlineto{\pgfqpoint{6.153688in}{1.702331in}}%
\pgfpathlineto{\pgfqpoint{6.162495in}{1.887630in}}%
\pgfpathlineto{\pgfqpoint{6.171302in}{1.915094in}}%
\pgfpathlineto{\pgfqpoint{6.180109in}{1.976851in}}%
\pgfpathlineto{\pgfqpoint{6.188916in}{2.107252in}}%
\pgfpathlineto{\pgfqpoint{6.197722in}{2.333705in}}%
\pgfpathlineto{\pgfqpoint{6.206529in}{2.093507in}}%
\pgfpathlineto{\pgfqpoint{6.215336in}{1.482737in}}%
\pgfpathlineto{\pgfqpoint{6.224143in}{1.606252in}}%
\pgfpathlineto{\pgfqpoint{6.232950in}{1.221935in}}%
\pgfpathlineto{\pgfqpoint{6.241757in}{1.393517in}}%
\pgfpathlineto{\pgfqpoint{6.250563in}{1.414093in}}%
\pgfpathlineto{\pgfqpoint{6.259370in}{1.201358in}}%
\pgfpathlineto{\pgfqpoint{6.268177in}{1.311155in}}%
\pgfpathlineto{\pgfqpoint{6.276984in}{1.496455in}}%
\pgfpathlineto{\pgfqpoint{6.285791in}{1.510172in}}%
\pgfpathlineto{\pgfqpoint{6.294597in}{1.647432in}}%
\pgfpathlineto{\pgfqpoint{6.303404in}{1.613110in}}%
\pgfpathlineto{\pgfqpoint{6.312211in}{1.517031in}}%
\pgfpathlineto{\pgfqpoint{6.321018in}{1.434698in}}%
\pgfpathlineto{\pgfqpoint{6.329825in}{1.558212in}}%
\pgfpathlineto{\pgfqpoint{6.338632in}{1.331760in}}%
\pgfpathlineto{\pgfqpoint{6.347438in}{1.468992in}}%
\pgfpathlineto{\pgfqpoint{6.356245in}{2.011173in}}%
\pgfpathlineto{\pgfqpoint{6.365052in}{2.004286in}}%
\pgfpathlineto{\pgfqpoint{6.373859in}{1.969992in}}%
\pgfpathlineto{\pgfqpoint{6.382666in}{2.107252in}}%
\pgfpathlineto{\pgfqpoint{6.391472in}{1.887630in}}%
\pgfpathlineto{\pgfqpoint{6.400279in}{2.018032in}}%
\pgfpathlineto{\pgfqpoint{6.409086in}{1.764116in}}%
\pgfpathlineto{\pgfqpoint{6.417893in}{1.770975in}}%
\pgfpathlineto{\pgfqpoint{6.426700in}{1.668037in}}%
\pgfpathlineto{\pgfqpoint{6.435507in}{2.093507in}}%
\pgfpathlineto{\pgfqpoint{6.444313in}{2.086648in}}%
\pgfpathlineto{\pgfqpoint{6.453120in}{2.018032in}}%
\pgfpathlineto{\pgfqpoint{6.461927in}{1.894489in}}%
\pgfpathlineto{\pgfqpoint{6.470734in}{2.196445in}}%
\pgfpathlineto{\pgfqpoint{6.479541in}{2.567044in}}%
\pgfpathlineto{\pgfqpoint{6.488347in}{2.608225in}}%
\pgfpathlineto{\pgfqpoint{6.497154in}{2.422925in}}%
\pgfpathlineto{\pgfqpoint{6.505961in}{2.834677in}}%
\pgfpathlineto{\pgfqpoint{6.514768in}{2.532722in}}%
\pgfpathlineto{\pgfqpoint{6.523575in}{2.402349in}}%
\pgfpathlineto{\pgfqpoint{6.532382in}{2.155292in}}%
\pgfpathlineto{\pgfqpoint{6.541188in}{2.052326in}}%
\pgfpathlineto{\pgfqpoint{6.549995in}{1.915094in}}%
\pgfpathlineto{\pgfqpoint{6.558802in}{2.038608in}}%
\pgfpathlineto{\pgfqpoint{6.567609in}{2.004286in}}%
\pgfpathlineto{\pgfqpoint{6.576416in}{1.887630in}}%
\pgfpathlineto{\pgfqpoint{6.585222in}{1.997427in}}%
\pgfpathlineto{\pgfqpoint{6.594029in}{1.887630in}}%
\pgfpathlineto{\pgfqpoint{6.602836in}{1.901348in}}%
\pgfpathlineto{\pgfqpoint{6.611643in}{2.018032in}}%
\pgfpathlineto{\pgfqpoint{6.620450in}{1.702331in}}%
\pgfpathlineto{\pgfqpoint{6.629257in}{1.510172in}}%
\pgfpathlineto{\pgfqpoint{6.638063in}{1.468992in}}%
\pgfpathlineto{\pgfqpoint{6.646870in}{1.475878in}}%
\pgfpathlineto{\pgfqpoint{6.655677in}{1.434698in}}%
\pgfpathlineto{\pgfqpoint{6.664484in}{1.372912in}}%
\pgfpathlineto{\pgfqpoint{6.673291in}{1.681754in}}%
\pgfpathlineto{\pgfqpoint{6.682097in}{1.894489in}}%
\pgfpathlineto{\pgfqpoint{6.690904in}{1.915094in}}%
\pgfpathlineto{\pgfqpoint{6.699711in}{2.162151in}}%
\pgfpathlineto{\pgfqpoint{6.708518in}{2.210190in}}%
\pgfpathlineto{\pgfqpoint{6.717325in}{1.770975in}}%
\pgfpathlineto{\pgfqpoint{6.726132in}{1.901348in}}%
\pgfpathlineto{\pgfqpoint{6.734938in}{1.819015in}}%
\pgfpathlineto{\pgfqpoint{6.743745in}{1.880772in}}%
\pgfpathlineto{\pgfqpoint{6.752552in}{1.956246in}}%
\pgfpathlineto{\pgfqpoint{6.761359in}{1.352336in}}%
\pgfpathlineto{\pgfqpoint{6.770166in}{1.386658in}}%
\pgfpathlineto{\pgfqpoint{6.778972in}{1.462133in}}%
\pgfpathlineto{\pgfqpoint{6.787779in}{1.386658in}}%
\pgfpathlineto{\pgfqpoint{6.796586in}{1.592534in}}%
\pgfpathlineto{\pgfqpoint{6.805393in}{1.256257in}}%
\pgfpathlineto{\pgfqpoint{6.814200in}{1.400376in}}%
\pgfpathlineto{\pgfqpoint{6.823007in}{1.468992in}}%
\pgfpathlineto{\pgfqpoint{6.831813in}{1.517031in}}%
\pgfpathlineto{\pgfqpoint{6.840620in}{1.496455in}}%
\pgfpathlineto{\pgfqpoint{6.858234in}{1.901348in}}%
\pgfpathlineto{\pgfqpoint{6.867041in}{1.928811in}}%
\pgfpathlineto{\pgfqpoint{6.875847in}{1.551353in}}%
\pgfpathlineto{\pgfqpoint{6.884654in}{1.846450in}}%
\pgfpathlineto{\pgfqpoint{6.893461in}{2.086648in}}%
\pgfpathlineto{\pgfqpoint{6.902268in}{1.798410in}}%
\pgfpathlineto{\pgfqpoint{6.911075in}{2.114111in}}%
\pgfpathlineto{\pgfqpoint{6.928688in}{1.777834in}}%
\pgfpathlineto{\pgfqpoint{6.946302in}{1.935670in}}%
\pgfpathlineto{\pgfqpoint{6.955109in}{1.997427in}}%
\pgfpathlineto{\pgfqpoint{6.963916in}{1.805269in}}%
\pgfpathlineto{\pgfqpoint{6.972722in}{2.052326in}}%
\pgfpathlineto{\pgfqpoint{6.981529in}{2.086648in}}%
\pgfpathlineto{\pgfqpoint{6.990336in}{2.155292in}}%
\pgfpathlineto{\pgfqpoint{6.999143in}{2.271947in}}%
\pgfpathlineto{\pgfqpoint{7.007950in}{2.120970in}}%
\pgfpathlineto{\pgfqpoint{7.016757in}{2.079789in}}%
\pgfpathlineto{\pgfqpoint{7.025563in}{1.764116in}}%
\pgfpathlineto{\pgfqpoint{7.034370in}{1.565071in}}%
\pgfpathlineto{\pgfqpoint{7.043177in}{1.393517in}}%
\pgfpathlineto{\pgfqpoint{7.051984in}{1.908207in}}%
\pgfpathlineto{\pgfqpoint{7.060791in}{1.942529in}}%
\pgfpathlineto{\pgfqpoint{7.069597in}{1.633715in}}%
\pgfpathlineto{\pgfqpoint{7.078404in}{1.558212in}}%
\pgfpathlineto{\pgfqpoint{7.087211in}{1.921953in}}%
\pgfpathlineto{\pgfqpoint{7.096018in}{1.743512in}}%
\pgfpathlineto{\pgfqpoint{7.104825in}{1.812128in}}%
\pgfpathlineto{\pgfqpoint{7.113632in}{1.475878in}}%
\pgfpathlineto{\pgfqpoint{7.122438in}{1.873913in}}%
\pgfpathlineto{\pgfqpoint{7.131245in}{1.860167in}}%
\pgfpathlineto{\pgfqpoint{7.140052in}{2.086648in}}%
\pgfpathlineto{\pgfqpoint{7.148859in}{1.928811in}}%
\pgfpathlineto{\pgfqpoint{7.157666in}{1.921953in}}%
\pgfpathlineto{\pgfqpoint{7.166472in}{2.175868in}}%
\pgfpathlineto{\pgfqpoint{7.175279in}{2.347422in}}%
\pgfpathlineto{\pgfqpoint{7.184086in}{2.107252in}}%
\pgfpathlineto{\pgfqpoint{7.192893in}{2.354309in}}%
\pgfpathlineto{\pgfqpoint{7.201700in}{2.038608in}}%
\pgfpathlineto{\pgfqpoint{7.210507in}{1.798410in}}%
\pgfpathlineto{\pgfqpoint{7.219313in}{1.921953in}}%
\pgfpathlineto{\pgfqpoint{7.228120in}{1.915094in}}%
\pgfpathlineto{\pgfqpoint{7.236927in}{1.819015in}}%
\pgfpathlineto{\pgfqpoint{7.245734in}{1.880772in}}%
\pgfpathlineto{\pgfqpoint{7.254541in}{1.880772in}}%
\pgfpathlineto{\pgfqpoint{7.263347in}{1.963133in}}%
\pgfpathlineto{\pgfqpoint{7.272154in}{2.313128in}}%
\pgfpathlineto{\pgfqpoint{7.280961in}{2.100365in}}%
\pgfpathlineto{\pgfqpoint{7.289768in}{2.004286in}}%
\pgfpathlineto{\pgfqpoint{7.298575in}{1.928811in}}%
\pgfpathlineto{\pgfqpoint{7.307382in}{1.887630in}}%
\pgfpathlineto{\pgfqpoint{7.316188in}{1.702331in}}%
\pgfpathlineto{\pgfqpoint{7.324995in}{1.750370in}}%
\pgfpathlineto{\pgfqpoint{7.333802in}{1.894489in}}%
\pgfpathlineto{\pgfqpoint{7.342609in}{1.750370in}}%
\pgfpathlineto{\pgfqpoint{7.351416in}{1.963133in}}%
\pgfpathlineto{\pgfqpoint{7.360222in}{1.777834in}}%
\pgfpathlineto{\pgfqpoint{7.377836in}{2.162151in}}%
\pgfpathlineto{\pgfqpoint{7.395450in}{2.175868in}}%
\pgfpathlineto{\pgfqpoint{7.404257in}{2.361168in}}%
\pgfpathlineto{\pgfqpoint{7.413063in}{2.127829in}}%
\pgfpathlineto{\pgfqpoint{7.421870in}{1.668037in}}%
\pgfpathlineto{\pgfqpoint{7.430677in}{1.366053in}}%
\pgfpathlineto{\pgfqpoint{7.439484in}{1.448415in}}%
\pgfpathlineto{\pgfqpoint{7.448291in}{1.482737in}}%
\pgfpathlineto{\pgfqpoint{7.457097in}{1.352336in}}%
\pgfpathlineto{\pgfqpoint{7.465904in}{1.249398in}}%
\pgfpathlineto{\pgfqpoint{7.474711in}{1.613110in}}%
\pgfpathlineto{\pgfqpoint{7.483518in}{1.530777in}}%
\pgfpathlineto{\pgfqpoint{7.492325in}{1.798410in}}%
\pgfpathlineto{\pgfqpoint{7.501132in}{1.819015in}}%
\pgfpathlineto{\pgfqpoint{7.509938in}{1.983710in}}%
\pgfpathlineto{\pgfqpoint{7.518745in}{2.203303in}}%
\pgfpathlineto{\pgfqpoint{7.536359in}{2.491541in}}%
\pgfpathlineto{\pgfqpoint{7.545166in}{2.203303in}}%
\pgfpathlineto{\pgfqpoint{7.553972in}{2.505287in}}%
\pgfpathlineto{\pgfqpoint{7.562779in}{2.038608in}}%
\pgfpathlineto{\pgfqpoint{7.571586in}{1.880772in}}%
\pgfpathlineto{\pgfqpoint{7.580393in}{1.702331in}}%
\pgfpathlineto{\pgfqpoint{7.589200in}{1.592534in}}%
\pgfpathlineto{\pgfqpoint{7.598007in}{1.791551in}}%
\pgfpathlineto{\pgfqpoint{7.606813in}{1.722935in}}%
\pgfpathlineto{\pgfqpoint{7.633234in}{2.594507in}}%
\pgfpathlineto{\pgfqpoint{7.642041in}{2.498428in}}%
\pgfpathlineto{\pgfqpoint{7.659654in}{2.148405in}}%
\pgfpathlineto{\pgfqpoint{7.668461in}{1.908207in}}%
\pgfpathlineto{\pgfqpoint{7.677268in}{1.908207in}}%
\pgfpathlineto{\pgfqpoint{7.686075in}{1.571958in}}%
\pgfpathlineto{\pgfqpoint{7.694882in}{1.414093in}}%
\pgfpathlineto{\pgfqpoint{7.703688in}{1.688613in}}%
\pgfpathlineto{\pgfqpoint{7.712495in}{1.784692in}}%
\pgfpathlineto{\pgfqpoint{7.721302in}{1.894489in}}%
\pgfpathlineto{\pgfqpoint{7.730109in}{2.038608in}}%
\pgfpathlineto{\pgfqpoint{7.738916in}{2.162151in}}%
\pgfpathlineto{\pgfqpoint{7.747722in}{2.024891in}}%
\pgfpathlineto{\pgfqpoint{7.756529in}{1.599393in}}%
\pgfpathlineto{\pgfqpoint{7.765336in}{1.867054in}}%
\pgfpathlineto{\pgfqpoint{7.774143in}{1.935670in}}%
\pgfpathlineto{\pgfqpoint{7.791757in}{1.702331in}}%
\pgfpathlineto{\pgfqpoint{7.800563in}{1.832732in}}%
\pgfpathlineto{\pgfqpoint{7.809370in}{1.633715in}}%
\pgfpathlineto{\pgfqpoint{7.818177in}{1.510172in}}%
\pgfpathlineto{\pgfqpoint{7.826984in}{1.880772in}}%
\pgfpathlineto{\pgfqpoint{7.835791in}{2.299383in}}%
\pgfpathlineto{\pgfqpoint{7.844597in}{2.326846in}}%
\pgfpathlineto{\pgfqpoint{7.853404in}{2.519004in}}%
\pgfpathlineto{\pgfqpoint{7.862211in}{2.361168in}}%
\pgfpathlineto{\pgfqpoint{7.871018in}{2.038608in}}%
\pgfpathlineto{\pgfqpoint{7.879825in}{2.196445in}}%
\pgfpathlineto{\pgfqpoint{7.888632in}{1.894489in}}%
\pgfpathlineto{\pgfqpoint{7.897438in}{1.681754in}}%
\pgfpathlineto{\pgfqpoint{7.906245in}{1.654291in}}%
\pgfpathlineto{\pgfqpoint{7.915052in}{1.860167in}}%
\pgfpathlineto{\pgfqpoint{7.923859in}{1.990569in}}%
\pgfpathlineto{\pgfqpoint{7.932666in}{1.846450in}}%
\pgfpathlineto{\pgfqpoint{7.941472in}{1.839591in}}%
\pgfpathlineto{\pgfqpoint{7.950279in}{1.860167in}}%
\pgfpathlineto{\pgfqpoint{7.959086in}{1.510172in}}%
\pgfpathlineto{\pgfqpoint{7.967893in}{1.441556in}}%
\pgfpathlineto{\pgfqpoint{7.976700in}{1.462133in}}%
\pgfpathlineto{\pgfqpoint{7.985507in}{1.448415in}}%
\pgfpathlineto{\pgfqpoint{7.994313in}{1.407234in}}%
\pgfpathlineto{\pgfqpoint{8.003120in}{1.468992in}}%
\pgfpathlineto{\pgfqpoint{8.011927in}{1.942529in}}%
\pgfpathlineto{\pgfqpoint{8.020734in}{1.599393in}}%
\pgfpathlineto{\pgfqpoint{8.029541in}{1.935670in}}%
\pgfpathlineto{\pgfqpoint{8.038347in}{1.592534in}}%
\pgfpathlineto{\pgfqpoint{8.047154in}{1.956246in}}%
\pgfpathlineto{\pgfqpoint{8.055961in}{1.695472in}}%
\pgfpathlineto{\pgfqpoint{8.064768in}{1.867054in}}%
\pgfpathlineto{\pgfqpoint{8.073575in}{1.860167in}}%
\pgfpathlineto{\pgfqpoint{8.082382in}{1.887630in}}%
\pgfpathlineto{\pgfqpoint{8.091188in}{1.770975in}}%
\pgfpathlineto{\pgfqpoint{8.108802in}{1.990569in}}%
\pgfpathlineto{\pgfqpoint{8.117609in}{2.011173in}}%
\pgfpathlineto{\pgfqpoint{8.126416in}{2.470965in}}%
\pgfpathlineto{\pgfqpoint{8.135222in}{2.251343in}}%
\pgfpathlineto{\pgfqpoint{8.144029in}{2.086648in}}%
\pgfpathlineto{\pgfqpoint{8.152836in}{1.764116in}}%
\pgfpathlineto{\pgfqpoint{8.161643in}{1.736653in}}%
\pgfpathlineto{\pgfqpoint{8.179257in}{1.427839in}}%
\pgfpathlineto{\pgfqpoint{8.188063in}{1.304296in}}%
\pgfpathlineto{\pgfqpoint{8.196870in}{1.324873in}}%
\pgfpathlineto{\pgfqpoint{8.205677in}{1.372912in}}%
\pgfpathlineto{\pgfqpoint{8.214484in}{1.359195in}}%
\pgfpathlineto{\pgfqpoint{8.223291in}{1.606252in}}%
\pgfpathlineto{\pgfqpoint{8.232097in}{1.770975in}}%
\pgfpathlineto{\pgfqpoint{8.240904in}{1.709190in}}%
\pgfpathlineto{\pgfqpoint{8.249711in}{1.908207in}}%
\pgfpathlineto{\pgfqpoint{8.258518in}{2.031749in}}%
\pgfpathlineto{\pgfqpoint{8.267325in}{2.079789in}}%
\pgfpathlineto{\pgfqpoint{8.276132in}{2.086648in}}%
\pgfpathlineto{\pgfqpoint{8.284938in}{2.011173in}}%
\pgfpathlineto{\pgfqpoint{8.293745in}{1.949388in}}%
\pgfpathlineto{\pgfqpoint{8.302552in}{1.997427in}}%
\pgfpathlineto{\pgfqpoint{8.311359in}{2.079789in}}%
\pgfpathlineto{\pgfqpoint{8.320166in}{1.997427in}}%
\pgfpathlineto{\pgfqpoint{8.328972in}{2.251343in}}%
\pgfpathlineto{\pgfqpoint{8.346586in}{1.908207in}}%
\pgfpathlineto{\pgfqpoint{8.355393in}{2.217049in}}%
\pgfpathlineto{\pgfqpoint{8.364200in}{2.148405in}}%
\pgfpathlineto{\pgfqpoint{8.373007in}{2.114111in}}%
\pgfpathlineto{\pgfqpoint{8.381813in}{2.182727in}}%
\pgfpathlineto{\pgfqpoint{8.390620in}{2.443501in}}%
\pgfpathlineto{\pgfqpoint{8.399427in}{2.402349in}}%
\pgfpathlineto{\pgfqpoint{8.408234in}{2.086648in}}%
\pgfpathlineto{\pgfqpoint{8.417041in}{2.052326in}}%
\pgfpathlineto{\pgfqpoint{8.425847in}{1.915094in}}%
\pgfpathlineto{\pgfqpoint{8.443461in}{1.551353in}}%
\pgfpathlineto{\pgfqpoint{8.452268in}{1.585675in}}%
\pgfpathlineto{\pgfqpoint{8.461075in}{1.819015in}}%
\pgfpathlineto{\pgfqpoint{8.469882in}{1.928811in}}%
\pgfpathlineto{\pgfqpoint{8.478688in}{1.915094in}}%
\pgfpathlineto{\pgfqpoint{8.487495in}{2.024891in}}%
\pgfpathlineto{\pgfqpoint{8.496302in}{2.018032in}}%
\pgfpathlineto{\pgfqpoint{8.505109in}{2.004286in}}%
\pgfpathlineto{\pgfqpoint{8.513916in}{1.983710in}}%
\pgfpathlineto{\pgfqpoint{8.522722in}{2.031749in}}%
\pgfpathlineto{\pgfqpoint{8.531529in}{1.935670in}}%
\pgfpathlineto{\pgfqpoint{8.540336in}{1.880772in}}%
\pgfpathlineto{\pgfqpoint{8.549143in}{2.038608in}}%
\pgfpathlineto{\pgfqpoint{8.557950in}{1.963133in}}%
\pgfpathlineto{\pgfqpoint{8.566757in}{2.107252in}}%
\pgfpathlineto{\pgfqpoint{8.575563in}{1.867054in}}%
\pgfpathlineto{\pgfqpoint{8.584370in}{2.052326in}}%
\pgfpathlineto{\pgfqpoint{8.593177in}{1.613110in}}%
\pgfpathlineto{\pgfqpoint{8.601984in}{1.722935in}}%
\pgfpathlineto{\pgfqpoint{8.610791in}{1.784692in}}%
\pgfpathlineto{\pgfqpoint{8.619597in}{1.969992in}}%
\pgfpathlineto{\pgfqpoint{8.628404in}{1.880772in}}%
\pgfpathlineto{\pgfqpoint{8.637211in}{2.148405in}}%
\pgfpathlineto{\pgfqpoint{8.646018in}{1.942529in}}%
\pgfpathlineto{\pgfqpoint{8.654825in}{1.921953in}}%
\pgfpathlineto{\pgfqpoint{8.672438in}{1.468992in}}%
\pgfpathlineto{\pgfqpoint{8.681245in}{1.537636in}}%
\pgfpathlineto{\pgfqpoint{8.690052in}{1.633715in}}%
\pgfpathlineto{\pgfqpoint{8.698859in}{1.345477in}}%
\pgfpathlineto{\pgfqpoint{8.707666in}{1.510172in}}%
\pgfpathlineto{\pgfqpoint{8.716472in}{1.256257in}}%
\pgfpathlineto{\pgfqpoint{8.725279in}{1.475878in}}%
\pgfpathlineto{\pgfqpoint{8.742893in}{1.167036in}}%
\pgfpathlineto{\pgfqpoint{8.760507in}{1.585675in}}%
\pgfpathlineto{\pgfqpoint{8.769313in}{1.681754in}}%
\pgfpathlineto{\pgfqpoint{8.778120in}{1.661150in}}%
\pgfpathlineto{\pgfqpoint{8.795734in}{1.496455in}}%
\pgfpathlineto{\pgfqpoint{8.804541in}{1.571958in}}%
\pgfpathlineto{\pgfqpoint{8.813347in}{1.757229in}}%
\pgfpathlineto{\pgfqpoint{8.822154in}{1.990569in}}%
\pgfpathlineto{\pgfqpoint{8.830961in}{2.079789in}}%
\pgfpathlineto{\pgfqpoint{8.839768in}{1.887630in}}%
\pgfpathlineto{\pgfqpoint{8.848575in}{1.798410in}}%
\pgfpathlineto{\pgfqpoint{8.857382in}{1.956246in}}%
\pgfpathlineto{\pgfqpoint{8.866188in}{1.969992in}}%
\pgfpathlineto{\pgfqpoint{8.874995in}{2.086648in}}%
\pgfpathlineto{\pgfqpoint{8.883802in}{1.963133in}}%
\pgfpathlineto{\pgfqpoint{8.892609in}{2.004286in}}%
\pgfpathlineto{\pgfqpoint{8.901416in}{1.681754in}}%
\pgfpathlineto{\pgfqpoint{8.910222in}{2.018032in}}%
\pgfpathlineto{\pgfqpoint{8.919029in}{2.217049in}}%
\pgfpathlineto{\pgfqpoint{8.927836in}{2.244484in}}%
\pgfpathlineto{\pgfqpoint{8.936643in}{1.976851in}}%
\pgfpathlineto{\pgfqpoint{8.945450in}{1.901348in}}%
\pgfpathlineto{\pgfqpoint{8.954257in}{1.750370in}}%
\pgfpathlineto{\pgfqpoint{8.963063in}{2.093507in}}%
\pgfpathlineto{\pgfqpoint{8.971870in}{2.189586in}}%
\pgfpathlineto{\pgfqpoint{8.980677in}{2.162151in}}%
\pgfpathlineto{\pgfqpoint{8.989484in}{2.059213in}}%
\pgfpathlineto{\pgfqpoint{8.998291in}{2.148405in}}%
\pgfpathlineto{\pgfqpoint{9.007097in}{1.757229in}}%
\pgfpathlineto{\pgfqpoint{9.015904in}{1.503314in}}%
\pgfpathlineto{\pgfqpoint{9.024711in}{1.523918in}}%
\pgfpathlineto{\pgfqpoint{9.033518in}{1.846450in}}%
\pgfpathlineto{\pgfqpoint{9.042325in}{1.729794in}}%
\pgfpathlineto{\pgfqpoint{9.051132in}{2.011173in}}%
\pgfpathlineto{\pgfqpoint{9.059938in}{2.100365in}}%
\pgfpathlineto{\pgfqpoint{9.068745in}{2.368027in}}%
\pgfpathlineto{\pgfqpoint{9.077552in}{1.921953in}}%
\pgfpathlineto{\pgfqpoint{9.086359in}{1.716076in}}%
\pgfpathlineto{\pgfqpoint{9.095166in}{1.750370in}}%
\pgfpathlineto{\pgfqpoint{9.103972in}{1.846450in}}%
\pgfpathlineto{\pgfqpoint{9.112779in}{1.359195in}}%
\pgfpathlineto{\pgfqpoint{9.121586in}{1.551353in}}%
\pgfpathlineto{\pgfqpoint{9.130393in}{1.537636in}}%
\pgfpathlineto{\pgfqpoint{9.139200in}{1.606252in}}%
\pgfpathlineto{\pgfqpoint{9.148007in}{1.887630in}}%
\pgfpathlineto{\pgfqpoint{9.165620in}{2.251343in}}%
\pgfpathlineto{\pgfqpoint{9.174427in}{2.086648in}}%
\pgfpathlineto{\pgfqpoint{9.183234in}{1.592534in}}%
\pgfpathlineto{\pgfqpoint{9.192041in}{1.770975in}}%
\pgfpathlineto{\pgfqpoint{9.200847in}{1.510172in}}%
\pgfpathlineto{\pgfqpoint{9.209654in}{1.523918in}}%
\pgfpathlineto{\pgfqpoint{9.218461in}{1.455274in}}%
\pgfpathlineto{\pgfqpoint{9.227268in}{1.695472in}}%
\pgfpathlineto{\pgfqpoint{9.236075in}{1.462133in}}%
\pgfpathlineto{\pgfqpoint{9.244882in}{1.523918in}}%
\pgfpathlineto{\pgfqpoint{9.253688in}{1.764116in}}%
\pgfpathlineto{\pgfqpoint{9.262495in}{1.674896in}}%
\pgfpathlineto{\pgfqpoint{9.271302in}{1.880772in}}%
\pgfpathlineto{\pgfqpoint{9.280109in}{1.997427in}}%
\pgfpathlineto{\pgfqpoint{9.288916in}{1.901348in}}%
\pgfpathlineto{\pgfqpoint{9.297722in}{1.976851in}}%
\pgfpathlineto{\pgfqpoint{9.306529in}{1.969992in}}%
\pgfpathlineto{\pgfqpoint{9.315336in}{1.729794in}}%
\pgfpathlineto{\pgfqpoint{9.324143in}{1.674896in}}%
\pgfpathlineto{\pgfqpoint{9.332950in}{1.764116in}}%
\pgfpathlineto{\pgfqpoint{9.341757in}{2.024891in}}%
\pgfpathlineto{\pgfqpoint{9.350563in}{2.093507in}}%
\pgfpathlineto{\pgfqpoint{9.359370in}{2.251343in}}%
\pgfpathlineto{\pgfqpoint{9.368177in}{1.901348in}}%
\pgfpathlineto{\pgfqpoint{9.376984in}{2.038608in}}%
\pgfpathlineto{\pgfqpoint{9.385791in}{1.949388in}}%
\pgfpathlineto{\pgfqpoint{9.394597in}{2.052326in}}%
\pgfpathlineto{\pgfqpoint{9.412211in}{1.695472in}}%
\pgfpathlineto{\pgfqpoint{9.421018in}{1.551353in}}%
\pgfpathlineto{\pgfqpoint{9.429825in}{1.791551in}}%
\pgfpathlineto{\pgfqpoint{9.438632in}{1.969992in}}%
\pgfpathlineto{\pgfqpoint{9.447438in}{1.805269in}}%
\pgfpathlineto{\pgfqpoint{9.456245in}{1.510172in}}%
\pgfpathlineto{\pgfqpoint{9.465052in}{1.777834in}}%
\pgfpathlineto{\pgfqpoint{9.473859in}{2.093507in}}%
\pgfpathlineto{\pgfqpoint{9.491472in}{2.470965in}}%
\pgfpathlineto{\pgfqpoint{9.500279in}{2.182727in}}%
\pgfpathlineto{\pgfqpoint{9.509086in}{2.251343in}}%
\pgfpathlineto{\pgfqpoint{9.517893in}{2.148405in}}%
\pgfpathlineto{\pgfqpoint{9.526700in}{2.011173in}}%
\pgfpathlineto{\pgfqpoint{9.535507in}{1.983710in}}%
\pgfpathlineto{\pgfqpoint{9.544313in}{1.990569in}}%
\pgfpathlineto{\pgfqpoint{9.553120in}{1.819015in}}%
\pgfpathlineto{\pgfqpoint{9.561927in}{1.709190in}}%
\pgfpathlineto{\pgfqpoint{9.570734in}{1.894489in}}%
\pgfpathlineto{\pgfqpoint{9.579541in}{2.018032in}}%
\pgfpathlineto{\pgfqpoint{9.588347in}{1.880772in}}%
\pgfpathlineto{\pgfqpoint{9.597154in}{1.695472in}}%
\pgfpathlineto{\pgfqpoint{9.614768in}{1.935670in}}%
\pgfpathlineto{\pgfqpoint{9.623575in}{2.251343in}}%
\pgfpathlineto{\pgfqpoint{9.632382in}{2.155292in}}%
\pgfpathlineto{\pgfqpoint{9.641188in}{2.093507in}}%
\pgfpathlineto{\pgfqpoint{9.649995in}{1.523918in}}%
\pgfpathlineto{\pgfqpoint{9.658802in}{1.784692in}}%
\pgfpathlineto{\pgfqpoint{9.667609in}{2.169009in}}%
\pgfpathlineto{\pgfqpoint{9.676416in}{2.141546in}}%
\pgfpathlineto{\pgfqpoint{9.685222in}{1.935670in}}%
\pgfpathlineto{\pgfqpoint{9.694029in}{2.086648in}}%
\pgfpathlineto{\pgfqpoint{9.702836in}{2.395462in}}%
\pgfpathlineto{\pgfqpoint{9.711643in}{1.997427in}}%
\pgfpathlineto{\pgfqpoint{9.720450in}{1.942529in}}%
\pgfpathlineto{\pgfqpoint{9.729257in}{2.066071in}}%
\pgfpathlineto{\pgfqpoint{9.738063in}{2.237625in}}%
\pgfpathlineto{\pgfqpoint{9.746870in}{2.100365in}}%
\pgfpathlineto{\pgfqpoint{9.755677in}{2.024891in}}%
\pgfpathlineto{\pgfqpoint{9.764484in}{1.880772in}}%
\pgfpathlineto{\pgfqpoint{9.773291in}{2.052326in}}%
\pgfpathlineto{\pgfqpoint{9.782097in}{1.873913in}}%
\pgfpathlineto{\pgfqpoint{9.790904in}{1.860167in}}%
\pgfpathlineto{\pgfqpoint{9.799711in}{1.736653in}}%
\pgfpathlineto{\pgfqpoint{9.808518in}{1.510172in}}%
\pgfpathlineto{\pgfqpoint{9.817325in}{1.448415in}}%
\pgfpathlineto{\pgfqpoint{9.826132in}{1.448415in}}%
\pgfpathlineto{\pgfqpoint{9.834938in}{1.482737in}}%
\pgfpathlineto{\pgfqpoint{9.843745in}{2.018032in}}%
\pgfpathlineto{\pgfqpoint{9.852552in}{2.155292in}}%
\pgfpathlineto{\pgfqpoint{9.861359in}{2.107252in}}%
\pgfpathlineto{\pgfqpoint{9.870166in}{1.997427in}}%
\pgfpathlineto{\pgfqpoint{9.878972in}{1.757229in}}%
\pgfpathlineto{\pgfqpoint{9.887779in}{1.825873in}}%
\pgfpathlineto{\pgfqpoint{9.896586in}{2.024891in}}%
\pgfpathlineto{\pgfqpoint{9.905393in}{2.024891in}}%
\pgfpathlineto{\pgfqpoint{9.914200in}{1.969992in}}%
\pgfpathlineto{\pgfqpoint{9.923007in}{1.757229in}}%
\pgfpathlineto{\pgfqpoint{9.931813in}{1.770975in}}%
\pgfpathlineto{\pgfqpoint{9.940620in}{1.750370in}}%
\pgfpathlineto{\pgfqpoint{9.949427in}{2.079789in}}%
\pgfpathlineto{\pgfqpoint{9.949427in}{2.079789in}}%
\pgfusepath{stroke}%
\end{pgfscope}%
\begin{pgfscope}%
\pgfpathrectangle{\pgfqpoint{0.702268in}{0.521603in}}{\pgfqpoint{9.687500in}{4.235000in}}%
\pgfusepath{clip}%
\pgfsetrectcap%
\pgfsetroundjoin%
\pgfsetlinewidth{0.501875pt}%
\definecolor{currentstroke}{rgb}{0.501961,0.501961,0.501961}%
\pgfsetstrokecolor{currentstroke}%
\pgfsetstrokeopacity{0.250000}%
\pgfsetdash{}{0pt}%
\pgfpathmoveto{\pgfqpoint{1.142609in}{4.227826in}}%
\pgfpathlineto{\pgfqpoint{1.151416in}{3.383717in}}%
\pgfpathlineto{\pgfqpoint{1.160222in}{2.676841in}}%
\pgfpathlineto{\pgfqpoint{1.169029in}{2.189586in}}%
\pgfpathlineto{\pgfqpoint{1.177836in}{2.114111in}}%
\pgfpathlineto{\pgfqpoint{1.186643in}{2.436643in}}%
\pgfpathlineto{\pgfqpoint{1.195450in}{2.066071in}}%
\pgfpathlineto{\pgfqpoint{1.204257in}{2.031749in}}%
\pgfpathlineto{\pgfqpoint{1.213063in}{2.114111in}}%
\pgfpathlineto{\pgfqpoint{1.221870in}{2.031749in}}%
\pgfpathlineto{\pgfqpoint{1.230677in}{2.031749in}}%
\pgfpathlineto{\pgfqpoint{1.239484in}{1.853308in}}%
\pgfpathlineto{\pgfqpoint{1.248291in}{2.059213in}}%
\pgfpathlineto{\pgfqpoint{1.257097in}{1.921953in}}%
\pgfpathlineto{\pgfqpoint{1.274711in}{1.688613in}}%
\pgfpathlineto{\pgfqpoint{1.283518in}{1.764116in}}%
\pgfpathlineto{\pgfqpoint{1.292325in}{1.455274in}}%
\pgfpathlineto{\pgfqpoint{1.301132in}{1.619997in}}%
\pgfpathlineto{\pgfqpoint{1.309938in}{1.352336in}}%
\pgfpathlineto{\pgfqpoint{1.318745in}{1.517031in}}%
\pgfpathlineto{\pgfqpoint{1.327552in}{1.599393in}}%
\pgfpathlineto{\pgfqpoint{1.336359in}{1.585675in}}%
\pgfpathlineto{\pgfqpoint{1.345166in}{1.551353in}}%
\pgfpathlineto{\pgfqpoint{1.353972in}{1.997427in}}%
\pgfpathlineto{\pgfqpoint{1.362779in}{2.169009in}}%
\pgfpathlineto{\pgfqpoint{1.371586in}{2.155292in}}%
\pgfpathlineto{\pgfqpoint{1.380393in}{2.182727in}}%
\pgfpathlineto{\pgfqpoint{1.389200in}{2.024891in}}%
\pgfpathlineto{\pgfqpoint{1.398007in}{2.258230in}}%
\pgfpathlineto{\pgfqpoint{1.406813in}{2.127829in}}%
\pgfpathlineto{\pgfqpoint{1.415620in}{2.114111in}}%
\pgfpathlineto{\pgfqpoint{1.424427in}{2.079789in}}%
\pgfpathlineto{\pgfqpoint{1.433234in}{1.942529in}}%
\pgfpathlineto{\pgfqpoint{1.450847in}{2.230767in}}%
\pgfpathlineto{\pgfqpoint{1.459654in}{2.580761in}}%
\pgfpathlineto{\pgfqpoint{1.468461in}{2.333705in}}%
\pgfpathlineto{\pgfqpoint{1.477268in}{2.134687in}}%
\pgfpathlineto{\pgfqpoint{1.486075in}{2.251343in}}%
\pgfpathlineto{\pgfqpoint{1.503688in}{1.695472in}}%
\pgfpathlineto{\pgfqpoint{1.512495in}{1.578816in}}%
\pgfpathlineto{\pgfqpoint{1.521302in}{1.537636in}}%
\pgfpathlineto{\pgfqpoint{1.530109in}{1.709190in}}%
\pgfpathlineto{\pgfqpoint{1.538916in}{1.942529in}}%
\pgfpathlineto{\pgfqpoint{1.547722in}{2.024891in}}%
\pgfpathlineto{\pgfqpoint{1.556529in}{1.983710in}}%
\pgfpathlineto{\pgfqpoint{1.565336in}{2.011173in}}%
\pgfpathlineto{\pgfqpoint{1.574143in}{1.681754in}}%
\pgfpathlineto{\pgfqpoint{1.582950in}{1.647432in}}%
\pgfpathlineto{\pgfqpoint{1.591757in}{1.750370in}}%
\pgfpathlineto{\pgfqpoint{1.600563in}{1.565071in}}%
\pgfpathlineto{\pgfqpoint{1.609370in}{1.558212in}}%
\pgfpathlineto{\pgfqpoint{1.618177in}{1.242539in}}%
\pgfpathlineto{\pgfqpoint{1.626984in}{1.688613in}}%
\pgfpathlineto{\pgfqpoint{1.635791in}{1.544494in}}%
\pgfpathlineto{\pgfqpoint{1.644597in}{1.462133in}}%
\pgfpathlineto{\pgfqpoint{1.653404in}{1.345477in}}%
\pgfpathlineto{\pgfqpoint{1.662211in}{1.352336in}}%
\pgfpathlineto{\pgfqpoint{1.671018in}{1.468992in}}%
\pgfpathlineto{\pgfqpoint{1.679825in}{1.619997in}}%
\pgfpathlineto{\pgfqpoint{1.688632in}{1.674896in}}%
\pgfpathlineto{\pgfqpoint{1.697438in}{1.757229in}}%
\pgfpathlineto{\pgfqpoint{1.706245in}{1.997427in}}%
\pgfpathlineto{\pgfqpoint{1.723859in}{1.839591in}}%
\pgfpathlineto{\pgfqpoint{1.732666in}{1.441556in}}%
\pgfpathlineto{\pgfqpoint{1.741472in}{1.661150in}}%
\pgfpathlineto{\pgfqpoint{1.750279in}{1.558212in}}%
\pgfpathlineto{\pgfqpoint{1.759086in}{1.427839in}}%
\pgfpathlineto{\pgfqpoint{1.776700in}{2.306269in}}%
\pgfpathlineto{\pgfqpoint{1.785507in}{2.210190in}}%
\pgfpathlineto{\pgfqpoint{1.794313in}{2.299383in}}%
\pgfpathlineto{\pgfqpoint{1.803120in}{2.148405in}}%
\pgfpathlineto{\pgfqpoint{1.811927in}{2.155292in}}%
\pgfpathlineto{\pgfqpoint{1.820734in}{2.217049in}}%
\pgfpathlineto{\pgfqpoint{1.829541in}{2.038608in}}%
\pgfpathlineto{\pgfqpoint{1.838347in}{1.963133in}}%
\pgfpathlineto{\pgfqpoint{1.847154in}{1.846450in}}%
\pgfpathlineto{\pgfqpoint{1.855961in}{1.819015in}}%
\pgfpathlineto{\pgfqpoint{1.864768in}{1.997427in}}%
\pgfpathlineto{\pgfqpoint{1.873575in}{2.100365in}}%
\pgfpathlineto{\pgfqpoint{1.882382in}{2.251343in}}%
\pgfpathlineto{\pgfqpoint{1.891188in}{2.079789in}}%
\pgfpathlineto{\pgfqpoint{1.899995in}{2.443501in}}%
\pgfpathlineto{\pgfqpoint{1.908802in}{2.237625in}}%
\pgfpathlineto{\pgfqpoint{1.917609in}{2.100365in}}%
\pgfpathlineto{\pgfqpoint{1.926416in}{2.381744in}}%
\pgfpathlineto{\pgfqpoint{1.935222in}{2.546468in}}%
\pgfpathlineto{\pgfqpoint{1.944029in}{2.217049in}}%
\pgfpathlineto{\pgfqpoint{1.952836in}{2.422925in}}%
\pgfpathlineto{\pgfqpoint{1.961643in}{2.299383in}}%
\pgfpathlineto{\pgfqpoint{1.970450in}{2.306269in}}%
\pgfpathlineto{\pgfqpoint{1.979257in}{2.484682in}}%
\pgfpathlineto{\pgfqpoint{1.988063in}{2.223908in}}%
\pgfpathlineto{\pgfqpoint{1.996870in}{2.162151in}}%
\pgfpathlineto{\pgfqpoint{2.005677in}{2.127829in}}%
\pgfpathlineto{\pgfqpoint{2.014484in}{2.416066in}}%
\pgfpathlineto{\pgfqpoint{2.023291in}{2.285665in}}%
\pgfpathlineto{\pgfqpoint{2.032097in}{2.244484in}}%
\pgfpathlineto{\pgfqpoint{2.040904in}{2.182727in}}%
\pgfpathlineto{\pgfqpoint{2.049711in}{2.265089in}}%
\pgfpathlineto{\pgfqpoint{2.058518in}{2.100365in}}%
\pgfpathlineto{\pgfqpoint{2.067325in}{2.011173in}}%
\pgfpathlineto{\pgfqpoint{2.076132in}{2.217049in}}%
\pgfpathlineto{\pgfqpoint{2.084938in}{2.361168in}}%
\pgfpathlineto{\pgfqpoint{2.093745in}{2.093507in}}%
\pgfpathlineto{\pgfqpoint{2.102552in}{2.313128in}}%
\pgfpathlineto{\pgfqpoint{2.111359in}{2.271947in}}%
\pgfpathlineto{\pgfqpoint{2.120166in}{2.107252in}}%
\pgfpathlineto{\pgfqpoint{2.128972in}{1.867054in}}%
\pgfpathlineto{\pgfqpoint{2.137779in}{1.578816in}}%
\pgfpathlineto{\pgfqpoint{2.146586in}{1.242539in}}%
\pgfpathlineto{\pgfqpoint{2.155393in}{1.681754in}}%
\pgfpathlineto{\pgfqpoint{2.164200in}{1.702331in}}%
\pgfpathlineto{\pgfqpoint{2.173007in}{2.066071in}}%
\pgfpathlineto{\pgfqpoint{2.181813in}{1.949388in}}%
\pgfpathlineto{\pgfqpoint{2.190620in}{2.004286in}}%
\pgfpathlineto{\pgfqpoint{2.199427in}{2.278806in}}%
\pgfpathlineto{\pgfqpoint{2.208234in}{2.052326in}}%
\pgfpathlineto{\pgfqpoint{2.217041in}{1.791551in}}%
\pgfpathlineto{\pgfqpoint{2.225847in}{1.819015in}}%
\pgfpathlineto{\pgfqpoint{2.234654in}{1.695472in}}%
\pgfpathlineto{\pgfqpoint{2.243461in}{1.414093in}}%
\pgfpathlineto{\pgfqpoint{2.261075in}{1.901348in}}%
\pgfpathlineto{\pgfqpoint{2.269882in}{1.764116in}}%
\pgfpathlineto{\pgfqpoint{2.278688in}{1.839591in}}%
\pgfpathlineto{\pgfqpoint{2.287495in}{1.757229in}}%
\pgfpathlineto{\pgfqpoint{2.296302in}{1.853308in}}%
\pgfpathlineto{\pgfqpoint{2.305109in}{2.189586in}}%
\pgfpathlineto{\pgfqpoint{2.313916in}{2.107252in}}%
\pgfpathlineto{\pgfqpoint{2.322722in}{1.468992in}}%
\pgfpathlineto{\pgfqpoint{2.331529in}{1.722935in}}%
\pgfpathlineto{\pgfqpoint{2.340336in}{1.764116in}}%
\pgfpathlineto{\pgfqpoint{2.349143in}{1.867054in}}%
\pgfpathlineto{\pgfqpoint{2.357950in}{1.709190in}}%
\pgfpathlineto{\pgfqpoint{2.366757in}{1.613110in}}%
\pgfpathlineto{\pgfqpoint{2.375563in}{1.674896in}}%
\pgfpathlineto{\pgfqpoint{2.384370in}{1.908207in}}%
\pgfpathlineto{\pgfqpoint{2.393177in}{1.873913in}}%
\pgfpathlineto{\pgfqpoint{2.401984in}{1.359195in}}%
\pgfpathlineto{\pgfqpoint{2.410791in}{1.269974in}}%
\pgfpathlineto{\pgfqpoint{2.419597in}{1.215076in}}%
\pgfpathlineto{\pgfqpoint{2.428404in}{1.318014in}}%
\pgfpathlineto{\pgfqpoint{2.437211in}{1.393517in}}%
\pgfpathlineto{\pgfqpoint{2.446018in}{1.146460in}}%
\pgfpathlineto{\pgfqpoint{2.454825in}{1.077816in}}%
\pgfpathlineto{\pgfqpoint{2.463632in}{1.468992in}}%
\pgfpathlineto{\pgfqpoint{2.472438in}{1.647432in}}%
\pgfpathlineto{\pgfqpoint{2.481245in}{1.695472in}}%
\pgfpathlineto{\pgfqpoint{2.490052in}{1.599393in}}%
\pgfpathlineto{\pgfqpoint{2.498859in}{1.757229in}}%
\pgfpathlineto{\pgfqpoint{2.507666in}{1.716076in}}%
\pgfpathlineto{\pgfqpoint{2.516472in}{1.846450in}}%
\pgfpathlineto{\pgfqpoint{2.525279in}{1.757229in}}%
\pgfpathlineto{\pgfqpoint{2.534086in}{1.633715in}}%
\pgfpathlineto{\pgfqpoint{2.551700in}{1.976851in}}%
\pgfpathlineto{\pgfqpoint{2.560507in}{2.299383in}}%
\pgfpathlineto{\pgfqpoint{2.569313in}{2.072930in}}%
\pgfpathlineto{\pgfqpoint{2.578120in}{1.880772in}}%
\pgfpathlineto{\pgfqpoint{2.586927in}{1.784692in}}%
\pgfpathlineto{\pgfqpoint{2.595734in}{2.045467in}}%
\pgfpathlineto{\pgfqpoint{2.604541in}{2.155292in}}%
\pgfpathlineto{\pgfqpoint{2.613347in}{2.395462in}}%
\pgfpathlineto{\pgfqpoint{2.622154in}{2.319987in}}%
\pgfpathlineto{\pgfqpoint{2.630961in}{2.072930in}}%
\pgfpathlineto{\pgfqpoint{2.639768in}{1.969992in}}%
\pgfpathlineto{\pgfqpoint{2.648575in}{1.901348in}}%
\pgfpathlineto{\pgfqpoint{2.657382in}{1.873913in}}%
\pgfpathlineto{\pgfqpoint{2.666188in}{2.175868in}}%
\pgfpathlineto{\pgfqpoint{2.674995in}{2.031749in}}%
\pgfpathlineto{\pgfqpoint{2.683802in}{2.354309in}}%
\pgfpathlineto{\pgfqpoint{2.692609in}{2.134687in}}%
\pgfpathlineto{\pgfqpoint{2.701416in}{2.223908in}}%
\pgfpathlineto{\pgfqpoint{2.710222in}{1.846450in}}%
\pgfpathlineto{\pgfqpoint{2.719029in}{2.182727in}}%
\pgfpathlineto{\pgfqpoint{2.727836in}{2.395462in}}%
\pgfpathlineto{\pgfqpoint{2.736643in}{2.182727in}}%
\pgfpathlineto{\pgfqpoint{2.745450in}{2.285665in}}%
\pgfpathlineto{\pgfqpoint{2.754257in}{2.038608in}}%
\pgfpathlineto{\pgfqpoint{2.763063in}{1.873913in}}%
\pgfpathlineto{\pgfqpoint{2.771870in}{1.736653in}}%
\pgfpathlineto{\pgfqpoint{2.780677in}{1.695472in}}%
\pgfpathlineto{\pgfqpoint{2.789484in}{1.475878in}}%
\pgfpathlineto{\pgfqpoint{2.798291in}{1.523918in}}%
\pgfpathlineto{\pgfqpoint{2.807097in}{1.386658in}}%
\pgfpathlineto{\pgfqpoint{2.815904in}{1.716076in}}%
\pgfpathlineto{\pgfqpoint{2.824711in}{1.805269in}}%
\pgfpathlineto{\pgfqpoint{2.833518in}{1.777834in}}%
\pgfpathlineto{\pgfqpoint{2.842325in}{1.963133in}}%
\pgfpathlineto{\pgfqpoint{2.851132in}{2.354309in}}%
\pgfpathlineto{\pgfqpoint{2.859938in}{2.306269in}}%
\pgfpathlineto{\pgfqpoint{2.868745in}{2.004286in}}%
\pgfpathlineto{\pgfqpoint{2.877552in}{1.832732in}}%
\pgfpathlineto{\pgfqpoint{2.886359in}{1.743512in}}%
\pgfpathlineto{\pgfqpoint{2.903972in}{1.496455in}}%
\pgfpathlineto{\pgfqpoint{2.912779in}{1.901348in}}%
\pgfpathlineto{\pgfqpoint{2.921586in}{1.805269in}}%
\pgfpathlineto{\pgfqpoint{2.930393in}{1.695472in}}%
\pgfpathlineto{\pgfqpoint{2.939200in}{1.860167in}}%
\pgfpathlineto{\pgfqpoint{2.948007in}{2.059213in}}%
\pgfpathlineto{\pgfqpoint{2.956813in}{1.805269in}}%
\pgfpathlineto{\pgfqpoint{2.965620in}{1.791551in}}%
\pgfpathlineto{\pgfqpoint{2.983234in}{1.263115in}}%
\pgfpathlineto{\pgfqpoint{2.992041in}{1.613110in}}%
\pgfpathlineto{\pgfqpoint{3.000847in}{1.716076in}}%
\pgfpathlineto{\pgfqpoint{3.009654in}{1.640574in}}%
\pgfpathlineto{\pgfqpoint{3.018461in}{1.908207in}}%
\pgfpathlineto{\pgfqpoint{3.027268in}{2.292524in}}%
\pgfpathlineto{\pgfqpoint{3.036075in}{2.443501in}}%
\pgfpathlineto{\pgfqpoint{3.044882in}{2.738598in}}%
\pgfpathlineto{\pgfqpoint{3.071302in}{2.038608in}}%
\pgfpathlineto{\pgfqpoint{3.080109in}{2.038608in}}%
\pgfpathlineto{\pgfqpoint{3.088916in}{2.381744in}}%
\pgfpathlineto{\pgfqpoint{3.097722in}{2.052326in}}%
\pgfpathlineto{\pgfqpoint{3.106529in}{2.059213in}}%
\pgfpathlineto{\pgfqpoint{3.115336in}{2.072930in}}%
\pgfpathlineto{\pgfqpoint{3.124143in}{1.805269in}}%
\pgfpathlineto{\pgfqpoint{3.132950in}{1.290579in}}%
\pgfpathlineto{\pgfqpoint{3.141757in}{1.379799in}}%
\pgfpathlineto{\pgfqpoint{3.150563in}{1.386658in}}%
\pgfpathlineto{\pgfqpoint{3.159370in}{1.523918in}}%
\pgfpathlineto{\pgfqpoint{3.168177in}{1.448415in}}%
\pgfpathlineto{\pgfqpoint{3.176984in}{1.791551in}}%
\pgfpathlineto{\pgfqpoint{3.185791in}{1.743512in}}%
\pgfpathlineto{\pgfqpoint{3.194597in}{2.230767in}}%
\pgfpathlineto{\pgfqpoint{3.203404in}{1.997427in}}%
\pgfpathlineto{\pgfqpoint{3.212211in}{2.066071in}}%
\pgfpathlineto{\pgfqpoint{3.221018in}{2.237625in}}%
\pgfpathlineto{\pgfqpoint{3.229825in}{2.148405in}}%
\pgfpathlineto{\pgfqpoint{3.238632in}{2.072930in}}%
\pgfpathlineto{\pgfqpoint{3.247438in}{1.949388in}}%
\pgfpathlineto{\pgfqpoint{3.256245in}{1.915094in}}%
\pgfpathlineto{\pgfqpoint{3.265052in}{2.114111in}}%
\pgfpathlineto{\pgfqpoint{3.273859in}{2.155292in}}%
\pgfpathlineto{\pgfqpoint{3.282666in}{2.278806in}}%
\pgfpathlineto{\pgfqpoint{3.291472in}{1.990569in}}%
\pgfpathlineto{\pgfqpoint{3.300279in}{2.024891in}}%
\pgfpathlineto{\pgfqpoint{3.309086in}{1.798410in}}%
\pgfpathlineto{\pgfqpoint{3.317893in}{2.326846in}}%
\pgfpathlineto{\pgfqpoint{3.326700in}{2.464106in}}%
\pgfpathlineto{\pgfqpoint{3.335507in}{2.402349in}}%
\pgfpathlineto{\pgfqpoint{3.344313in}{2.251343in}}%
\pgfpathlineto{\pgfqpoint{3.353120in}{2.457247in}}%
\pgfpathlineto{\pgfqpoint{3.361927in}{2.381744in}}%
\pgfpathlineto{\pgfqpoint{3.370734in}{2.457247in}}%
\pgfpathlineto{\pgfqpoint{3.379541in}{2.395462in}}%
\pgfpathlineto{\pgfqpoint{3.388347in}{2.175868in}}%
\pgfpathlineto{\pgfqpoint{3.397154in}{2.402349in}}%
\pgfpathlineto{\pgfqpoint{3.405961in}{2.292524in}}%
\pgfpathlineto{\pgfqpoint{3.414768in}{1.784692in}}%
\pgfpathlineto{\pgfqpoint{3.423575in}{1.489596in}}%
\pgfpathlineto{\pgfqpoint{3.432382in}{1.297438in}}%
\pgfpathlineto{\pgfqpoint{3.441188in}{1.819015in}}%
\pgfpathlineto{\pgfqpoint{3.449995in}{1.798410in}}%
\pgfpathlineto{\pgfqpoint{3.458802in}{1.928811in}}%
\pgfpathlineto{\pgfqpoint{3.467609in}{1.942529in}}%
\pgfpathlineto{\pgfqpoint{3.476416in}{2.031749in}}%
\pgfpathlineto{\pgfqpoint{3.485222in}{1.880772in}}%
\pgfpathlineto{\pgfqpoint{3.494029in}{2.093507in}}%
\pgfpathlineto{\pgfqpoint{3.502836in}{2.169009in}}%
\pgfpathlineto{\pgfqpoint{3.511643in}{1.908207in}}%
\pgfpathlineto{\pgfqpoint{3.529257in}{2.313128in}}%
\pgfpathlineto{\pgfqpoint{3.546870in}{1.887630in}}%
\pgfpathlineto{\pgfqpoint{3.555677in}{2.299383in}}%
\pgfpathlineto{\pgfqpoint{3.564484in}{2.347422in}}%
\pgfpathlineto{\pgfqpoint{3.573291in}{2.265089in}}%
\pgfpathlineto{\pgfqpoint{3.582097in}{2.223908in}}%
\pgfpathlineto{\pgfqpoint{3.590904in}{2.141546in}}%
\pgfpathlineto{\pgfqpoint{3.599711in}{2.491541in}}%
\pgfpathlineto{\pgfqpoint{3.608518in}{2.162151in}}%
\pgfpathlineto{\pgfqpoint{3.617325in}{2.093507in}}%
\pgfpathlineto{\pgfqpoint{3.626132in}{2.217049in}}%
\pgfpathlineto{\pgfqpoint{3.634938in}{1.928811in}}%
\pgfpathlineto{\pgfqpoint{3.643745in}{1.915094in}}%
\pgfpathlineto{\pgfqpoint{3.652552in}{1.805269in}}%
\pgfpathlineto{\pgfqpoint{3.661359in}{1.750370in}}%
\pgfpathlineto{\pgfqpoint{3.670166in}{1.619997in}}%
\pgfpathlineto{\pgfqpoint{3.678972in}{1.537636in}}%
\pgfpathlineto{\pgfqpoint{3.687779in}{1.311155in}}%
\pgfpathlineto{\pgfqpoint{3.705393in}{1.578816in}}%
\pgfpathlineto{\pgfqpoint{3.714200in}{1.400376in}}%
\pgfpathlineto{\pgfqpoint{3.723007in}{1.462133in}}%
\pgfpathlineto{\pgfqpoint{3.731813in}{1.798410in}}%
\pgfpathlineto{\pgfqpoint{3.740620in}{1.750370in}}%
\pgfpathlineto{\pgfqpoint{3.749427in}{1.613110in}}%
\pgfpathlineto{\pgfqpoint{3.758234in}{1.407234in}}%
\pgfpathlineto{\pgfqpoint{3.767041in}{1.352336in}}%
\pgfpathlineto{\pgfqpoint{3.775847in}{1.599393in}}%
\pgfpathlineto{\pgfqpoint{3.784654in}{1.489596in}}%
\pgfpathlineto{\pgfqpoint{3.793461in}{1.647432in}}%
\pgfpathlineto{\pgfqpoint{3.802268in}{1.942529in}}%
\pgfpathlineto{\pgfqpoint{3.811075in}{1.764116in}}%
\pgfpathlineto{\pgfqpoint{3.819882in}{1.928811in}}%
\pgfpathlineto{\pgfqpoint{3.828688in}{2.230767in}}%
\pgfpathlineto{\pgfqpoint{3.837495in}{2.313128in}}%
\pgfpathlineto{\pgfqpoint{3.855109in}{1.619997in}}%
\pgfpathlineto{\pgfqpoint{3.863916in}{1.908207in}}%
\pgfpathlineto{\pgfqpoint{3.872722in}{1.805269in}}%
\pgfpathlineto{\pgfqpoint{3.881529in}{1.750370in}}%
\pgfpathlineto{\pgfqpoint{3.899143in}{1.681754in}}%
\pgfpathlineto{\pgfqpoint{3.907950in}{2.107252in}}%
\pgfpathlineto{\pgfqpoint{3.916757in}{2.038608in}}%
\pgfpathlineto{\pgfqpoint{3.925563in}{2.265089in}}%
\pgfpathlineto{\pgfqpoint{3.934370in}{2.230767in}}%
\pgfpathlineto{\pgfqpoint{3.943177in}{2.553326in}}%
\pgfpathlineto{\pgfqpoint{3.951984in}{2.313128in}}%
\pgfpathlineto{\pgfqpoint{3.960791in}{2.244484in}}%
\pgfpathlineto{\pgfqpoint{3.969597in}{2.354309in}}%
\pgfpathlineto{\pgfqpoint{3.978404in}{2.004286in}}%
\pgfpathlineto{\pgfqpoint{3.987211in}{1.709190in}}%
\pgfpathlineto{\pgfqpoint{3.996018in}{1.880772in}}%
\pgfpathlineto{\pgfqpoint{4.004825in}{1.853308in}}%
\pgfpathlineto{\pgfqpoint{4.013632in}{1.558212in}}%
\pgfpathlineto{\pgfqpoint{4.022438in}{1.517031in}}%
\pgfpathlineto{\pgfqpoint{4.031245in}{1.969992in}}%
\pgfpathlineto{\pgfqpoint{4.040052in}{2.162151in}}%
\pgfpathlineto{\pgfqpoint{4.048859in}{2.120970in}}%
\pgfpathlineto{\pgfqpoint{4.057666in}{1.942529in}}%
\pgfpathlineto{\pgfqpoint{4.066472in}{1.647432in}}%
\pgfpathlineto{\pgfqpoint{4.075279in}{1.736653in}}%
\pgfpathlineto{\pgfqpoint{4.084086in}{1.640574in}}%
\pgfpathlineto{\pgfqpoint{4.092893in}{1.668037in}}%
\pgfpathlineto{\pgfqpoint{4.101700in}{1.668037in}}%
\pgfpathlineto{\pgfqpoint{4.110507in}{1.764116in}}%
\pgfpathlineto{\pgfqpoint{4.119313in}{1.722935in}}%
\pgfpathlineto{\pgfqpoint{4.128120in}{1.770975in}}%
\pgfpathlineto{\pgfqpoint{4.136927in}{1.565071in}}%
\pgfpathlineto{\pgfqpoint{4.145734in}{1.880772in}}%
\pgfpathlineto{\pgfqpoint{4.154541in}{2.114111in}}%
\pgfpathlineto{\pgfqpoint{4.163347in}{2.052326in}}%
\pgfpathlineto{\pgfqpoint{4.172154in}{1.867054in}}%
\pgfpathlineto{\pgfqpoint{4.180961in}{2.169009in}}%
\pgfpathlineto{\pgfqpoint{4.189768in}{2.388603in}}%
\pgfpathlineto{\pgfqpoint{4.198575in}{2.381744in}}%
\pgfpathlineto{\pgfqpoint{4.207382in}{2.649406in}}%
\pgfpathlineto{\pgfqpoint{4.216188in}{2.381744in}}%
\pgfpathlineto{\pgfqpoint{4.224995in}{2.045467in}}%
\pgfpathlineto{\pgfqpoint{4.233802in}{1.976851in}}%
\pgfpathlineto{\pgfqpoint{4.242609in}{2.045467in}}%
\pgfpathlineto{\pgfqpoint{4.251416in}{2.079789in}}%
\pgfpathlineto{\pgfqpoint{4.260222in}{2.223908in}}%
\pgfpathlineto{\pgfqpoint{4.269029in}{1.867054in}}%
\pgfpathlineto{\pgfqpoint{4.277836in}{1.716076in}}%
\pgfpathlineto{\pgfqpoint{4.286643in}{1.825873in}}%
\pgfpathlineto{\pgfqpoint{4.295450in}{1.777834in}}%
\pgfpathlineto{\pgfqpoint{4.304257in}{2.093507in}}%
\pgfpathlineto{\pgfqpoint{4.313063in}{2.155292in}}%
\pgfpathlineto{\pgfqpoint{4.321870in}{1.880772in}}%
\pgfpathlineto{\pgfqpoint{4.330677in}{2.148405in}}%
\pgfpathlineto{\pgfqpoint{4.339484in}{1.963133in}}%
\pgfpathlineto{\pgfqpoint{4.348291in}{1.709190in}}%
\pgfpathlineto{\pgfqpoint{4.357097in}{1.791551in}}%
\pgfpathlineto{\pgfqpoint{4.365904in}{1.846450in}}%
\pgfpathlineto{\pgfqpoint{4.374711in}{1.915094in}}%
\pgfpathlineto{\pgfqpoint{4.392325in}{1.551353in}}%
\pgfpathlineto{\pgfqpoint{4.401132in}{2.114111in}}%
\pgfpathlineto{\pgfqpoint{4.409938in}{1.860167in}}%
\pgfpathlineto{\pgfqpoint{4.418745in}{2.018032in}}%
\pgfpathlineto{\pgfqpoint{4.427552in}{1.997427in}}%
\pgfpathlineto{\pgfqpoint{4.436359in}{2.162151in}}%
\pgfpathlineto{\pgfqpoint{4.445166in}{1.887630in}}%
\pgfpathlineto{\pgfqpoint{4.453972in}{2.120970in}}%
\pgfpathlineto{\pgfqpoint{4.462779in}{2.162151in}}%
\pgfpathlineto{\pgfqpoint{4.471586in}{2.258230in}}%
\pgfpathlineto{\pgfqpoint{4.480393in}{2.306269in}}%
\pgfpathlineto{\pgfqpoint{4.489200in}{1.901348in}}%
\pgfpathlineto{\pgfqpoint{4.498007in}{1.750370in}}%
\pgfpathlineto{\pgfqpoint{4.506813in}{2.127829in}}%
\pgfpathlineto{\pgfqpoint{4.515620in}{1.976851in}}%
\pgfpathlineto{\pgfqpoint{4.524427in}{2.141546in}}%
\pgfpathlineto{\pgfqpoint{4.542041in}{1.990569in}}%
\pgfpathlineto{\pgfqpoint{4.550847in}{1.784692in}}%
\pgfpathlineto{\pgfqpoint{4.559654in}{1.695472in}}%
\pgfpathlineto{\pgfqpoint{4.568461in}{1.592534in}}%
\pgfpathlineto{\pgfqpoint{4.577268in}{1.935670in}}%
\pgfpathlineto{\pgfqpoint{4.594882in}{1.606252in}}%
\pgfpathlineto{\pgfqpoint{4.603688in}{1.764116in}}%
\pgfpathlineto{\pgfqpoint{4.612495in}{1.565071in}}%
\pgfpathlineto{\pgfqpoint{4.621302in}{1.565071in}}%
\pgfpathlineto{\pgfqpoint{4.630109in}{1.901348in}}%
\pgfpathlineto{\pgfqpoint{4.638916in}{1.537636in}}%
\pgfpathlineto{\pgfqpoint{4.647722in}{1.702331in}}%
\pgfpathlineto{\pgfqpoint{4.656529in}{1.743512in}}%
\pgfpathlineto{\pgfqpoint{4.665336in}{1.448415in}}%
\pgfpathlineto{\pgfqpoint{4.674143in}{1.482737in}}%
\pgfpathlineto{\pgfqpoint{4.682950in}{1.400376in}}%
\pgfpathlineto{\pgfqpoint{4.691757in}{1.407234in}}%
\pgfpathlineto{\pgfqpoint{4.700563in}{1.070957in}}%
\pgfpathlineto{\pgfqpoint{4.709370in}{1.125855in}}%
\pgfpathlineto{\pgfqpoint{4.718177in}{1.420952in}}%
\pgfpathlineto{\pgfqpoint{4.726984in}{1.468992in}}%
\pgfpathlineto{\pgfqpoint{4.735791in}{1.613110in}}%
\pgfpathlineto{\pgfqpoint{4.744597in}{1.599393in}}%
\pgfpathlineto{\pgfqpoint{4.753404in}{1.729794in}}%
\pgfpathlineto{\pgfqpoint{4.762211in}{1.908207in}}%
\pgfpathlineto{\pgfqpoint{4.771018in}{2.011173in}}%
\pgfpathlineto{\pgfqpoint{4.779825in}{2.086648in}}%
\pgfpathlineto{\pgfqpoint{4.788632in}{2.072930in}}%
\pgfpathlineto{\pgfqpoint{4.797438in}{2.011173in}}%
\pgfpathlineto{\pgfqpoint{4.806245in}{1.716076in}}%
\pgfpathlineto{\pgfqpoint{4.815052in}{1.496455in}}%
\pgfpathlineto{\pgfqpoint{4.823859in}{1.359195in}}%
\pgfpathlineto{\pgfqpoint{4.832666in}{1.661150in}}%
\pgfpathlineto{\pgfqpoint{4.841472in}{1.400376in}}%
\pgfpathlineto{\pgfqpoint{4.850279in}{1.372912in}}%
\pgfpathlineto{\pgfqpoint{4.859086in}{1.530777in}}%
\pgfpathlineto{\pgfqpoint{4.867893in}{1.921953in}}%
\pgfpathlineto{\pgfqpoint{4.876700in}{2.031749in}}%
\pgfpathlineto{\pgfqpoint{4.885507in}{1.716076in}}%
\pgfpathlineto{\pgfqpoint{4.894313in}{1.805269in}}%
\pgfpathlineto{\pgfqpoint{4.903120in}{1.551353in}}%
\pgfpathlineto{\pgfqpoint{4.911927in}{1.626856in}}%
\pgfpathlineto{\pgfqpoint{4.920734in}{1.736653in}}%
\pgfpathlineto{\pgfqpoint{4.929541in}{2.210190in}}%
\pgfpathlineto{\pgfqpoint{4.938347in}{2.470965in}}%
\pgfpathlineto{\pgfqpoint{4.947154in}{2.093507in}}%
\pgfpathlineto{\pgfqpoint{4.955961in}{1.880772in}}%
\pgfpathlineto{\pgfqpoint{4.964768in}{1.812128in}}%
\pgfpathlineto{\pgfqpoint{4.973575in}{2.223908in}}%
\pgfpathlineto{\pgfqpoint{4.982382in}{2.560185in}}%
\pgfpathlineto{\pgfqpoint{4.991188in}{2.621942in}}%
\pgfpathlineto{\pgfqpoint{4.999995in}{2.477823in}}%
\pgfpathlineto{\pgfqpoint{5.017609in}{1.853308in}}%
\pgfpathlineto{\pgfqpoint{5.035222in}{2.258230in}}%
\pgfpathlineto{\pgfqpoint{5.044029in}{2.203303in}}%
\pgfpathlineto{\pgfqpoint{5.052836in}{2.429784in}}%
\pgfpathlineto{\pgfqpoint{5.061643in}{2.155292in}}%
\pgfpathlineto{\pgfqpoint{5.070450in}{2.162151in}}%
\pgfpathlineto{\pgfqpoint{5.079257in}{2.278806in}}%
\pgfpathlineto{\pgfqpoint{5.088063in}{2.477823in}}%
\pgfpathlineto{\pgfqpoint{5.096870in}{2.416066in}}%
\pgfpathlineto{\pgfqpoint{5.105677in}{2.340563in}}%
\pgfpathlineto{\pgfqpoint{5.114484in}{2.169009in}}%
\pgfpathlineto{\pgfqpoint{5.123291in}{2.429784in}}%
\pgfpathlineto{\pgfqpoint{5.132097in}{2.052326in}}%
\pgfpathlineto{\pgfqpoint{5.140904in}{1.963133in}}%
\pgfpathlineto{\pgfqpoint{5.149711in}{2.223908in}}%
\pgfpathlineto{\pgfqpoint{5.158518in}{2.038608in}}%
\pgfpathlineto{\pgfqpoint{5.167325in}{1.949388in}}%
\pgfpathlineto{\pgfqpoint{5.176132in}{1.812128in}}%
\pgfpathlineto{\pgfqpoint{5.184938in}{1.441556in}}%
\pgfpathlineto{\pgfqpoint{5.202552in}{1.853308in}}%
\pgfpathlineto{\pgfqpoint{5.211359in}{2.340563in}}%
\pgfpathlineto{\pgfqpoint{5.228972in}{1.969992in}}%
\pgfpathlineto{\pgfqpoint{5.237779in}{2.107252in}}%
\pgfpathlineto{\pgfqpoint{5.246586in}{2.162151in}}%
\pgfpathlineto{\pgfqpoint{5.255393in}{1.942529in}}%
\pgfpathlineto{\pgfqpoint{5.273007in}{2.326846in}}%
\pgfpathlineto{\pgfqpoint{5.281813in}{2.086648in}}%
\pgfpathlineto{\pgfqpoint{5.290620in}{2.162151in}}%
\pgfpathlineto{\pgfqpoint{5.299427in}{1.729794in}}%
\pgfpathlineto{\pgfqpoint{5.308234in}{1.613110in}}%
\pgfpathlineto{\pgfqpoint{5.317041in}{1.201358in}}%
\pgfpathlineto{\pgfqpoint{5.325847in}{1.544494in}}%
\pgfpathlineto{\pgfqpoint{5.334654in}{1.571958in}}%
\pgfpathlineto{\pgfqpoint{5.343461in}{1.503314in}}%
\pgfpathlineto{\pgfqpoint{5.352268in}{1.578816in}}%
\pgfpathlineto{\pgfqpoint{5.361075in}{1.722935in}}%
\pgfpathlineto{\pgfqpoint{5.369882in}{1.571958in}}%
\pgfpathlineto{\pgfqpoint{5.378688in}{1.846450in}}%
\pgfpathlineto{\pgfqpoint{5.387495in}{1.702331in}}%
\pgfpathlineto{\pgfqpoint{5.396302in}{1.668037in}}%
\pgfpathlineto{\pgfqpoint{5.405109in}{1.311155in}}%
\pgfpathlineto{\pgfqpoint{5.413916in}{1.578816in}}%
\pgfpathlineto{\pgfqpoint{5.422722in}{1.599393in}}%
\pgfpathlineto{\pgfqpoint{5.431529in}{1.819015in}}%
\pgfpathlineto{\pgfqpoint{5.440336in}{2.114111in}}%
\pgfpathlineto{\pgfqpoint{5.449143in}{2.093507in}}%
\pgfpathlineto{\pgfqpoint{5.457950in}{2.333705in}}%
\pgfpathlineto{\pgfqpoint{5.466757in}{2.368027in}}%
\pgfpathlineto{\pgfqpoint{5.475563in}{2.354309in}}%
\pgfpathlineto{\pgfqpoint{5.484370in}{2.546468in}}%
\pgfpathlineto{\pgfqpoint{5.493177in}{2.587620in}}%
\pgfpathlineto{\pgfqpoint{5.501984in}{2.189586in}}%
\pgfpathlineto{\pgfqpoint{5.510791in}{1.942529in}}%
\pgfpathlineto{\pgfqpoint{5.519597in}{2.114111in}}%
\pgfpathlineto{\pgfqpoint{5.528404in}{1.887630in}}%
\pgfpathlineto{\pgfqpoint{5.537211in}{1.777834in}}%
\pgfpathlineto{\pgfqpoint{5.546018in}{1.599393in}}%
\pgfpathlineto{\pgfqpoint{5.554825in}{1.640574in}}%
\pgfpathlineto{\pgfqpoint{5.563632in}{2.175868in}}%
\pgfpathlineto{\pgfqpoint{5.572438in}{1.757229in}}%
\pgfpathlineto{\pgfqpoint{5.581245in}{1.517031in}}%
\pgfpathlineto{\pgfqpoint{5.590052in}{1.475878in}}%
\pgfpathlineto{\pgfqpoint{5.607666in}{1.702331in}}%
\pgfpathlineto{\pgfqpoint{5.616472in}{2.031749in}}%
\pgfpathlineto{\pgfqpoint{5.625279in}{2.052326in}}%
\pgfpathlineto{\pgfqpoint{5.634086in}{1.880772in}}%
\pgfpathlineto{\pgfqpoint{5.642893in}{1.743512in}}%
\pgfpathlineto{\pgfqpoint{5.651700in}{1.736653in}}%
\pgfpathlineto{\pgfqpoint{5.660507in}{1.997427in}}%
\pgfpathlineto{\pgfqpoint{5.669313in}{2.210190in}}%
\pgfpathlineto{\pgfqpoint{5.678120in}{1.812128in}}%
\pgfpathlineto{\pgfqpoint{5.686927in}{1.825873in}}%
\pgfpathlineto{\pgfqpoint{5.695734in}{1.969992in}}%
\pgfpathlineto{\pgfqpoint{5.704541in}{2.271947in}}%
\pgfpathlineto{\pgfqpoint{5.713347in}{2.210190in}}%
\pgfpathlineto{\pgfqpoint{5.722154in}{1.963133in}}%
\pgfpathlineto{\pgfqpoint{5.730961in}{2.148405in}}%
\pgfpathlineto{\pgfqpoint{5.748575in}{1.311155in}}%
\pgfpathlineto{\pgfqpoint{5.766188in}{1.551353in}}%
\pgfpathlineto{\pgfqpoint{5.774995in}{2.024891in}}%
\pgfpathlineto{\pgfqpoint{5.783802in}{1.770975in}}%
\pgfpathlineto{\pgfqpoint{5.792609in}{1.805269in}}%
\pgfpathlineto{\pgfqpoint{5.801416in}{1.764116in}}%
\pgfpathlineto{\pgfqpoint{5.810222in}{2.114111in}}%
\pgfpathlineto{\pgfqpoint{5.827836in}{1.887630in}}%
\pgfpathlineto{\pgfqpoint{5.836643in}{1.551353in}}%
\pgfpathlineto{\pgfqpoint{5.845450in}{2.011173in}}%
\pgfpathlineto{\pgfqpoint{5.854257in}{1.942529in}}%
\pgfpathlineto{\pgfqpoint{5.863063in}{2.470965in}}%
\pgfpathlineto{\pgfqpoint{5.871870in}{2.635660in}}%
\pgfpathlineto{\pgfqpoint{5.880677in}{2.642547in}}%
\pgfpathlineto{\pgfqpoint{5.889484in}{2.464106in}}%
\pgfpathlineto{\pgfqpoint{5.898291in}{2.031749in}}%
\pgfpathlineto{\pgfqpoint{5.907097in}{1.798410in}}%
\pgfpathlineto{\pgfqpoint{5.915904in}{1.887630in}}%
\pgfpathlineto{\pgfqpoint{5.924711in}{2.052326in}}%
\pgfpathlineto{\pgfqpoint{5.933518in}{2.333705in}}%
\pgfpathlineto{\pgfqpoint{5.942325in}{2.210190in}}%
\pgfpathlineto{\pgfqpoint{5.951132in}{2.299383in}}%
\pgfpathlineto{\pgfqpoint{5.959938in}{1.928811in}}%
\pgfpathlineto{\pgfqpoint{5.968745in}{2.066071in}}%
\pgfpathlineto{\pgfqpoint{5.986359in}{1.901348in}}%
\pgfpathlineto{\pgfqpoint{5.995166in}{2.114111in}}%
\pgfpathlineto{\pgfqpoint{6.003972in}{2.127829in}}%
\pgfpathlineto{\pgfqpoint{6.012779in}{1.839591in}}%
\pgfpathlineto{\pgfqpoint{6.021586in}{2.052326in}}%
\pgfpathlineto{\pgfqpoint{6.030393in}{2.093507in}}%
\pgfpathlineto{\pgfqpoint{6.039200in}{1.530777in}}%
\pgfpathlineto{\pgfqpoint{6.048007in}{1.510172in}}%
\pgfpathlineto{\pgfqpoint{6.056813in}{1.510172in}}%
\pgfpathlineto{\pgfqpoint{6.065620in}{1.606252in}}%
\pgfpathlineto{\pgfqpoint{6.074427in}{1.908207in}}%
\pgfpathlineto{\pgfqpoint{6.083234in}{1.846450in}}%
\pgfpathlineto{\pgfqpoint{6.100847in}{1.688613in}}%
\pgfpathlineto{\pgfqpoint{6.109654in}{1.963133in}}%
\pgfpathlineto{\pgfqpoint{6.118461in}{1.901348in}}%
\pgfpathlineto{\pgfqpoint{6.127268in}{1.777834in}}%
\pgfpathlineto{\pgfqpoint{6.136075in}{1.894489in}}%
\pgfpathlineto{\pgfqpoint{6.144882in}{2.223908in}}%
\pgfpathlineto{\pgfqpoint{6.153688in}{2.100365in}}%
\pgfpathlineto{\pgfqpoint{6.162495in}{2.265089in}}%
\pgfpathlineto{\pgfqpoint{6.171302in}{2.066071in}}%
\pgfpathlineto{\pgfqpoint{6.180109in}{2.120970in}}%
\pgfpathlineto{\pgfqpoint{6.188916in}{2.072930in}}%
\pgfpathlineto{\pgfqpoint{6.197722in}{1.750370in}}%
\pgfpathlineto{\pgfqpoint{6.206529in}{2.361168in}}%
\pgfpathlineto{\pgfqpoint{6.215336in}{2.470965in}}%
\pgfpathlineto{\pgfqpoint{6.224143in}{2.230767in}}%
\pgfpathlineto{\pgfqpoint{6.232950in}{2.114111in}}%
\pgfpathlineto{\pgfqpoint{6.241757in}{1.764116in}}%
\pgfpathlineto{\pgfqpoint{6.250563in}{1.661150in}}%
\pgfpathlineto{\pgfqpoint{6.259370in}{1.654291in}}%
\pgfpathlineto{\pgfqpoint{6.268177in}{1.963133in}}%
\pgfpathlineto{\pgfqpoint{6.276984in}{2.052326in}}%
\pgfpathlineto{\pgfqpoint{6.285791in}{1.969992in}}%
\pgfpathlineto{\pgfqpoint{6.294597in}{1.695472in}}%
\pgfpathlineto{\pgfqpoint{6.303404in}{1.681754in}}%
\pgfpathlineto{\pgfqpoint{6.312211in}{2.018032in}}%
\pgfpathlineto{\pgfqpoint{6.321018in}{2.031749in}}%
\pgfpathlineto{\pgfqpoint{6.329825in}{2.018032in}}%
\pgfpathlineto{\pgfqpoint{6.338632in}{2.079789in}}%
\pgfpathlineto{\pgfqpoint{6.347438in}{1.873913in}}%
\pgfpathlineto{\pgfqpoint{6.356245in}{1.963133in}}%
\pgfpathlineto{\pgfqpoint{6.365052in}{1.812128in}}%
\pgfpathlineto{\pgfqpoint{6.373859in}{1.709190in}}%
\pgfpathlineto{\pgfqpoint{6.382666in}{1.839591in}}%
\pgfpathlineto{\pgfqpoint{6.391472in}{1.784692in}}%
\pgfpathlineto{\pgfqpoint{6.400279in}{1.969992in}}%
\pgfpathlineto{\pgfqpoint{6.409086in}{2.244484in}}%
\pgfpathlineto{\pgfqpoint{6.417893in}{2.223908in}}%
\pgfpathlineto{\pgfqpoint{6.426700in}{2.114111in}}%
\pgfpathlineto{\pgfqpoint{6.435507in}{2.237625in}}%
\pgfpathlineto{\pgfqpoint{6.444313in}{2.230767in}}%
\pgfpathlineto{\pgfqpoint{6.453120in}{2.100365in}}%
\pgfpathlineto{\pgfqpoint{6.461927in}{2.066071in}}%
\pgfpathlineto{\pgfqpoint{6.470734in}{1.798410in}}%
\pgfpathlineto{\pgfqpoint{6.479541in}{1.853308in}}%
\pgfpathlineto{\pgfqpoint{6.488347in}{1.770975in}}%
\pgfpathlineto{\pgfqpoint{6.497154in}{1.825873in}}%
\pgfpathlineto{\pgfqpoint{6.505961in}{1.681754in}}%
\pgfpathlineto{\pgfqpoint{6.514768in}{2.107252in}}%
\pgfpathlineto{\pgfqpoint{6.523575in}{2.072930in}}%
\pgfpathlineto{\pgfqpoint{6.532382in}{2.223908in}}%
\pgfpathlineto{\pgfqpoint{6.549995in}{1.743512in}}%
\pgfpathlineto{\pgfqpoint{6.558802in}{1.674896in}}%
\pgfpathlineto{\pgfqpoint{6.567609in}{1.551353in}}%
\pgfpathlineto{\pgfqpoint{6.576416in}{1.722935in}}%
\pgfpathlineto{\pgfqpoint{6.585222in}{1.695472in}}%
\pgfpathlineto{\pgfqpoint{6.594029in}{1.963133in}}%
\pgfpathlineto{\pgfqpoint{6.602836in}{1.860167in}}%
\pgfpathlineto{\pgfqpoint{6.611643in}{1.619997in}}%
\pgfpathlineto{\pgfqpoint{6.620450in}{1.867054in}}%
\pgfpathlineto{\pgfqpoint{6.638063in}{1.592534in}}%
\pgfpathlineto{\pgfqpoint{6.646870in}{1.825873in}}%
\pgfpathlineto{\pgfqpoint{6.655677in}{1.372912in}}%
\pgfpathlineto{\pgfqpoint{6.664484in}{1.455274in}}%
\pgfpathlineto{\pgfqpoint{6.673291in}{1.400376in}}%
\pgfpathlineto{\pgfqpoint{6.682097in}{1.393517in}}%
\pgfpathlineto{\pgfqpoint{6.690904in}{1.743512in}}%
\pgfpathlineto{\pgfqpoint{6.699711in}{1.867054in}}%
\pgfpathlineto{\pgfqpoint{6.708518in}{1.928811in}}%
\pgfpathlineto{\pgfqpoint{6.717325in}{2.045467in}}%
\pgfpathlineto{\pgfqpoint{6.726132in}{2.141546in}}%
\pgfpathlineto{\pgfqpoint{6.734938in}{1.969992in}}%
\pgfpathlineto{\pgfqpoint{6.743745in}{2.271947in}}%
\pgfpathlineto{\pgfqpoint{6.752552in}{2.011173in}}%
\pgfpathlineto{\pgfqpoint{6.761359in}{2.271947in}}%
\pgfpathlineto{\pgfqpoint{6.770166in}{1.901348in}}%
\pgfpathlineto{\pgfqpoint{6.778972in}{1.633715in}}%
\pgfpathlineto{\pgfqpoint{6.787779in}{2.079789in}}%
\pgfpathlineto{\pgfqpoint{6.796586in}{1.997427in}}%
\pgfpathlineto{\pgfqpoint{6.805393in}{2.155292in}}%
\pgfpathlineto{\pgfqpoint{6.814200in}{2.169009in}}%
\pgfpathlineto{\pgfqpoint{6.823007in}{2.066071in}}%
\pgfpathlineto{\pgfqpoint{6.831813in}{1.901348in}}%
\pgfpathlineto{\pgfqpoint{6.840620in}{2.354309in}}%
\pgfpathlineto{\pgfqpoint{6.849427in}{2.388603in}}%
\pgfpathlineto{\pgfqpoint{6.858234in}{1.997427in}}%
\pgfpathlineto{\pgfqpoint{6.867041in}{2.299383in}}%
\pgfpathlineto{\pgfqpoint{6.875847in}{2.169009in}}%
\pgfpathlineto{\pgfqpoint{6.884654in}{2.114111in}}%
\pgfpathlineto{\pgfqpoint{6.893461in}{2.381744in}}%
\pgfpathlineto{\pgfqpoint{6.902268in}{2.265089in}}%
\pgfpathlineto{\pgfqpoint{6.911075in}{2.127829in}}%
\pgfpathlineto{\pgfqpoint{6.919882in}{2.038608in}}%
\pgfpathlineto{\pgfqpoint{6.928688in}{2.072930in}}%
\pgfpathlineto{\pgfqpoint{6.937495in}{2.086648in}}%
\pgfpathlineto{\pgfqpoint{6.946302in}{1.812128in}}%
\pgfpathlineto{\pgfqpoint{6.955109in}{1.716076in}}%
\pgfpathlineto{\pgfqpoint{6.963916in}{1.702331in}}%
\pgfpathlineto{\pgfqpoint{6.972722in}{1.976851in}}%
\pgfpathlineto{\pgfqpoint{6.981529in}{2.059213in}}%
\pgfpathlineto{\pgfqpoint{6.990336in}{1.921953in}}%
\pgfpathlineto{\pgfqpoint{6.999143in}{1.997427in}}%
\pgfpathlineto{\pgfqpoint{7.007950in}{1.942529in}}%
\pgfpathlineto{\pgfqpoint{7.016757in}{1.860167in}}%
\pgfpathlineto{\pgfqpoint{7.025563in}{1.942529in}}%
\pgfpathlineto{\pgfqpoint{7.034370in}{1.331760in}}%
\pgfpathlineto{\pgfqpoint{7.043177in}{1.297438in}}%
\pgfpathlineto{\pgfqpoint{7.051984in}{1.722935in}}%
\pgfpathlineto{\pgfqpoint{7.060791in}{1.729794in}}%
\pgfpathlineto{\pgfqpoint{7.069597in}{2.155292in}}%
\pgfpathlineto{\pgfqpoint{7.078404in}{2.271947in}}%
\pgfpathlineto{\pgfqpoint{7.087211in}{2.292524in}}%
\pgfpathlineto{\pgfqpoint{7.096018in}{1.894489in}}%
\pgfpathlineto{\pgfqpoint{7.104825in}{1.983710in}}%
\pgfpathlineto{\pgfqpoint{7.113632in}{1.853308in}}%
\pgfpathlineto{\pgfqpoint{7.122438in}{1.619997in}}%
\pgfpathlineto{\pgfqpoint{7.131245in}{1.743512in}}%
\pgfpathlineto{\pgfqpoint{7.140052in}{1.839591in}}%
\pgfpathlineto{\pgfqpoint{7.148859in}{2.450388in}}%
\pgfpathlineto{\pgfqpoint{7.157666in}{2.333705in}}%
\pgfpathlineto{\pgfqpoint{7.166472in}{2.704304in}}%
\pgfpathlineto{\pgfqpoint{7.175279in}{2.381744in}}%
\pgfpathlineto{\pgfqpoint{7.184086in}{2.299383in}}%
\pgfpathlineto{\pgfqpoint{7.192893in}{1.969992in}}%
\pgfpathlineto{\pgfqpoint{7.201700in}{2.169009in}}%
\pgfpathlineto{\pgfqpoint{7.210507in}{2.086648in}}%
\pgfpathlineto{\pgfqpoint{7.219313in}{2.299383in}}%
\pgfpathlineto{\pgfqpoint{7.228120in}{1.873913in}}%
\pgfpathlineto{\pgfqpoint{7.236927in}{1.928811in}}%
\pgfpathlineto{\pgfqpoint{7.245734in}{2.004286in}}%
\pgfpathlineto{\pgfqpoint{7.254541in}{2.024891in}}%
\pgfpathlineto{\pgfqpoint{7.263347in}{1.681754in}}%
\pgfpathlineto{\pgfqpoint{7.272154in}{1.770975in}}%
\pgfpathlineto{\pgfqpoint{7.280961in}{1.537636in}}%
\pgfpathlineto{\pgfqpoint{7.289768in}{1.619997in}}%
\pgfpathlineto{\pgfqpoint{7.298575in}{1.853308in}}%
\pgfpathlineto{\pgfqpoint{7.307382in}{1.743512in}}%
\pgfpathlineto{\pgfqpoint{7.316188in}{1.722935in}}%
\pgfpathlineto{\pgfqpoint{7.324995in}{1.894489in}}%
\pgfpathlineto{\pgfqpoint{7.333802in}{1.894489in}}%
\pgfpathlineto{\pgfqpoint{7.342609in}{2.258230in}}%
\pgfpathlineto{\pgfqpoint{7.351416in}{2.278806in}}%
\pgfpathlineto{\pgfqpoint{7.360222in}{2.265089in}}%
\pgfpathlineto{\pgfqpoint{7.369029in}{2.141546in}}%
\pgfpathlineto{\pgfqpoint{7.377836in}{2.278806in}}%
\pgfpathlineto{\pgfqpoint{7.386643in}{2.093507in}}%
\pgfpathlineto{\pgfqpoint{7.395450in}{1.976851in}}%
\pgfpathlineto{\pgfqpoint{7.404257in}{1.805269in}}%
\pgfpathlineto{\pgfqpoint{7.413063in}{2.265089in}}%
\pgfpathlineto{\pgfqpoint{7.421870in}{2.217049in}}%
\pgfpathlineto{\pgfqpoint{7.430677in}{2.251343in}}%
\pgfpathlineto{\pgfqpoint{7.439484in}{2.134687in}}%
\pgfpathlineto{\pgfqpoint{7.448291in}{2.196445in}}%
\pgfpathlineto{\pgfqpoint{7.457097in}{2.374885in}}%
\pgfpathlineto{\pgfqpoint{7.465904in}{2.642547in}}%
\pgfpathlineto{\pgfqpoint{7.474711in}{2.409207in}}%
\pgfpathlineto{\pgfqpoint{7.483518in}{2.326846in}}%
\pgfpathlineto{\pgfqpoint{7.492325in}{2.402349in}}%
\pgfpathlineto{\pgfqpoint{7.501132in}{1.942529in}}%
\pgfpathlineto{\pgfqpoint{7.509938in}{1.894489in}}%
\pgfpathlineto{\pgfqpoint{7.518745in}{1.709190in}}%
\pgfpathlineto{\pgfqpoint{7.527552in}{1.448415in}}%
\pgfpathlineto{\pgfqpoint{7.536359in}{1.613110in}}%
\pgfpathlineto{\pgfqpoint{7.545166in}{1.544494in}}%
\pgfpathlineto{\pgfqpoint{7.553972in}{1.969992in}}%
\pgfpathlineto{\pgfqpoint{7.562779in}{2.278806in}}%
\pgfpathlineto{\pgfqpoint{7.571586in}{1.894489in}}%
\pgfpathlineto{\pgfqpoint{7.580393in}{2.031749in}}%
\pgfpathlineto{\pgfqpoint{7.589200in}{2.306269in}}%
\pgfpathlineto{\pgfqpoint{7.598007in}{2.381744in}}%
\pgfpathlineto{\pgfqpoint{7.606813in}{2.258230in}}%
\pgfpathlineto{\pgfqpoint{7.615620in}{2.402349in}}%
\pgfpathlineto{\pgfqpoint{7.624427in}{2.066071in}}%
\pgfpathlineto{\pgfqpoint{7.633234in}{2.018032in}}%
\pgfpathlineto{\pgfqpoint{7.650847in}{2.409207in}}%
\pgfpathlineto{\pgfqpoint{7.659654in}{2.100365in}}%
\pgfpathlineto{\pgfqpoint{7.668461in}{2.395462in}}%
\pgfpathlineto{\pgfqpoint{7.677268in}{2.223908in}}%
\pgfpathlineto{\pgfqpoint{7.686075in}{2.251343in}}%
\pgfpathlineto{\pgfqpoint{7.694882in}{2.093507in}}%
\pgfpathlineto{\pgfqpoint{7.703688in}{2.155292in}}%
\pgfpathlineto{\pgfqpoint{7.712495in}{1.908207in}}%
\pgfpathlineto{\pgfqpoint{7.721302in}{2.011173in}}%
\pgfpathlineto{\pgfqpoint{7.730109in}{2.093507in}}%
\pgfpathlineto{\pgfqpoint{7.738916in}{1.935670in}}%
\pgfpathlineto{\pgfqpoint{7.747722in}{2.004286in}}%
\pgfpathlineto{\pgfqpoint{7.756529in}{1.832732in}}%
\pgfpathlineto{\pgfqpoint{7.765336in}{1.908207in}}%
\pgfpathlineto{\pgfqpoint{7.774143in}{1.791551in}}%
\pgfpathlineto{\pgfqpoint{7.782950in}{2.024891in}}%
\pgfpathlineto{\pgfqpoint{7.791757in}{1.983710in}}%
\pgfpathlineto{\pgfqpoint{7.800563in}{2.299383in}}%
\pgfpathlineto{\pgfqpoint{7.809370in}{2.189586in}}%
\pgfpathlineto{\pgfqpoint{7.818177in}{1.935670in}}%
\pgfpathlineto{\pgfqpoint{7.826984in}{1.825873in}}%
\pgfpathlineto{\pgfqpoint{7.835791in}{2.182727in}}%
\pgfpathlineto{\pgfqpoint{7.844597in}{2.086648in}}%
\pgfpathlineto{\pgfqpoint{7.853404in}{1.791551in}}%
\pgfpathlineto{\pgfqpoint{7.862211in}{1.770975in}}%
\pgfpathlineto{\pgfqpoint{7.871018in}{1.764116in}}%
\pgfpathlineto{\pgfqpoint{7.879825in}{1.729794in}}%
\pgfpathlineto{\pgfqpoint{7.888632in}{1.791551in}}%
\pgfpathlineto{\pgfqpoint{7.897438in}{1.812128in}}%
\pgfpathlineto{\pgfqpoint{7.906245in}{2.093507in}}%
\pgfpathlineto{\pgfqpoint{7.915052in}{1.983710in}}%
\pgfpathlineto{\pgfqpoint{7.923859in}{1.661150in}}%
\pgfpathlineto{\pgfqpoint{7.932666in}{1.592534in}}%
\pgfpathlineto{\pgfqpoint{7.941472in}{1.626856in}}%
\pgfpathlineto{\pgfqpoint{7.950279in}{1.798410in}}%
\pgfpathlineto{\pgfqpoint{7.959086in}{1.901348in}}%
\pgfpathlineto{\pgfqpoint{7.967893in}{1.983710in}}%
\pgfpathlineto{\pgfqpoint{7.976700in}{1.695472in}}%
\pgfpathlineto{\pgfqpoint{7.985507in}{2.203303in}}%
\pgfpathlineto{\pgfqpoint{7.994313in}{2.169009in}}%
\pgfpathlineto{\pgfqpoint{8.003120in}{2.011173in}}%
\pgfpathlineto{\pgfqpoint{8.011927in}{1.640574in}}%
\pgfpathlineto{\pgfqpoint{8.020734in}{1.523918in}}%
\pgfpathlineto{\pgfqpoint{8.029541in}{1.956246in}}%
\pgfpathlineto{\pgfqpoint{8.038347in}{2.107252in}}%
\pgfpathlineto{\pgfqpoint{8.047154in}{1.949388in}}%
\pgfpathlineto{\pgfqpoint{8.055961in}{1.915094in}}%
\pgfpathlineto{\pgfqpoint{8.064768in}{1.860167in}}%
\pgfpathlineto{\pgfqpoint{8.073575in}{2.045467in}}%
\pgfpathlineto{\pgfqpoint{8.082382in}{1.949388in}}%
\pgfpathlineto{\pgfqpoint{8.099995in}{1.626856in}}%
\pgfpathlineto{\pgfqpoint{8.108802in}{1.743512in}}%
\pgfpathlineto{\pgfqpoint{8.117609in}{1.921953in}}%
\pgfpathlineto{\pgfqpoint{8.126416in}{1.915094in}}%
\pgfpathlineto{\pgfqpoint{8.135222in}{1.764116in}}%
\pgfpathlineto{\pgfqpoint{8.144029in}{1.640574in}}%
\pgfpathlineto{\pgfqpoint{8.152836in}{1.805269in}}%
\pgfpathlineto{\pgfqpoint{8.161643in}{1.798410in}}%
\pgfpathlineto{\pgfqpoint{8.170450in}{2.004286in}}%
\pgfpathlineto{\pgfqpoint{8.179257in}{2.093507in}}%
\pgfpathlineto{\pgfqpoint{8.188063in}{2.024891in}}%
\pgfpathlineto{\pgfqpoint{8.196870in}{2.223908in}}%
\pgfpathlineto{\pgfqpoint{8.205677in}{2.313128in}}%
\pgfpathlineto{\pgfqpoint{8.214484in}{2.141546in}}%
\pgfpathlineto{\pgfqpoint{8.223291in}{2.120970in}}%
\pgfpathlineto{\pgfqpoint{8.232097in}{2.155292in}}%
\pgfpathlineto{\pgfqpoint{8.240904in}{2.155292in}}%
\pgfpathlineto{\pgfqpoint{8.249711in}{1.976851in}}%
\pgfpathlineto{\pgfqpoint{8.258518in}{1.908207in}}%
\pgfpathlineto{\pgfqpoint{8.267325in}{2.011173in}}%
\pgfpathlineto{\pgfqpoint{8.276132in}{1.798410in}}%
\pgfpathlineto{\pgfqpoint{8.284938in}{1.819015in}}%
\pgfpathlineto{\pgfqpoint{8.293745in}{1.688613in}}%
\pgfpathlineto{\pgfqpoint{8.302552in}{1.784692in}}%
\pgfpathlineto{\pgfqpoint{8.311359in}{1.908207in}}%
\pgfpathlineto{\pgfqpoint{8.320166in}{1.805269in}}%
\pgfpathlineto{\pgfqpoint{8.328972in}{1.592534in}}%
\pgfpathlineto{\pgfqpoint{8.337779in}{1.256257in}}%
\pgfpathlineto{\pgfqpoint{8.346586in}{1.297438in}}%
\pgfpathlineto{\pgfqpoint{8.355393in}{1.263115in}}%
\pgfpathlineto{\pgfqpoint{8.364200in}{1.194499in}}%
\pgfpathlineto{\pgfqpoint{8.373007in}{1.606252in}}%
\pgfpathlineto{\pgfqpoint{8.381813in}{1.770975in}}%
\pgfpathlineto{\pgfqpoint{8.390620in}{1.709190in}}%
\pgfpathlineto{\pgfqpoint{8.399427in}{1.578816in}}%
\pgfpathlineto{\pgfqpoint{8.408234in}{1.661150in}}%
\pgfpathlineto{\pgfqpoint{8.417041in}{1.496455in}}%
\pgfpathlineto{\pgfqpoint{8.425847in}{1.832732in}}%
\pgfpathlineto{\pgfqpoint{8.434654in}{1.921953in}}%
\pgfpathlineto{\pgfqpoint{8.443461in}{2.093507in}}%
\pgfpathlineto{\pgfqpoint{8.452268in}{1.908207in}}%
\pgfpathlineto{\pgfqpoint{8.461075in}{2.066071in}}%
\pgfpathlineto{\pgfqpoint{8.469882in}{1.709190in}}%
\pgfpathlineto{\pgfqpoint{8.478688in}{2.052326in}}%
\pgfpathlineto{\pgfqpoint{8.487495in}{2.217049in}}%
\pgfpathlineto{\pgfqpoint{8.496302in}{2.299383in}}%
\pgfpathlineto{\pgfqpoint{8.505109in}{2.182727in}}%
\pgfpathlineto{\pgfqpoint{8.513916in}{2.333705in}}%
\pgfpathlineto{\pgfqpoint{8.522722in}{2.018032in}}%
\pgfpathlineto{\pgfqpoint{8.531529in}{1.867054in}}%
\pgfpathlineto{\pgfqpoint{8.540336in}{1.976851in}}%
\pgfpathlineto{\pgfqpoint{8.549143in}{1.983710in}}%
\pgfpathlineto{\pgfqpoint{8.557950in}{2.079789in}}%
\pgfpathlineto{\pgfqpoint{8.566757in}{2.031749in}}%
\pgfpathlineto{\pgfqpoint{8.575563in}{1.935670in}}%
\pgfpathlineto{\pgfqpoint{8.584370in}{1.860167in}}%
\pgfpathlineto{\pgfqpoint{8.593177in}{2.100365in}}%
\pgfpathlineto{\pgfqpoint{8.601984in}{2.052326in}}%
\pgfpathlineto{\pgfqpoint{8.610791in}{1.949388in}}%
\pgfpathlineto{\pgfqpoint{8.619597in}{1.956246in}}%
\pgfpathlineto{\pgfqpoint{8.637211in}{1.510172in}}%
\pgfpathlineto{\pgfqpoint{8.646018in}{1.812128in}}%
\pgfpathlineto{\pgfqpoint{8.654825in}{2.210190in}}%
\pgfpathlineto{\pgfqpoint{8.663632in}{2.162151in}}%
\pgfpathlineto{\pgfqpoint{8.672438in}{2.340563in}}%
\pgfpathlineto{\pgfqpoint{8.681245in}{2.251343in}}%
\pgfpathlineto{\pgfqpoint{8.690052in}{2.230767in}}%
\pgfpathlineto{\pgfqpoint{8.698859in}{1.894489in}}%
\pgfpathlineto{\pgfqpoint{8.707666in}{1.949388in}}%
\pgfpathlineto{\pgfqpoint{8.716472in}{1.853308in}}%
\pgfpathlineto{\pgfqpoint{8.725279in}{1.695472in}}%
\pgfpathlineto{\pgfqpoint{8.734086in}{1.770975in}}%
\pgfpathlineto{\pgfqpoint{8.742893in}{1.860167in}}%
\pgfpathlineto{\pgfqpoint{8.751700in}{1.784692in}}%
\pgfpathlineto{\pgfqpoint{8.760507in}{1.894489in}}%
\pgfpathlineto{\pgfqpoint{8.769313in}{1.784692in}}%
\pgfpathlineto{\pgfqpoint{8.778120in}{1.757229in}}%
\pgfpathlineto{\pgfqpoint{8.786927in}{1.688613in}}%
\pgfpathlineto{\pgfqpoint{8.795734in}{1.853308in}}%
\pgfpathlineto{\pgfqpoint{8.804541in}{1.873913in}}%
\pgfpathlineto{\pgfqpoint{8.813347in}{1.873913in}}%
\pgfpathlineto{\pgfqpoint{8.822154in}{1.736653in}}%
\pgfpathlineto{\pgfqpoint{8.830961in}{1.565071in}}%
\pgfpathlineto{\pgfqpoint{8.848575in}{1.798410in}}%
\pgfpathlineto{\pgfqpoint{8.857382in}{1.997427in}}%
\pgfpathlineto{\pgfqpoint{8.866188in}{1.867054in}}%
\pgfpathlineto{\pgfqpoint{8.874995in}{1.921953in}}%
\pgfpathlineto{\pgfqpoint{8.883802in}{2.182727in}}%
\pgfpathlineto{\pgfqpoint{8.892609in}{2.539581in}}%
\pgfpathlineto{\pgfqpoint{8.901416in}{2.018032in}}%
\pgfpathlineto{\pgfqpoint{8.910222in}{1.681754in}}%
\pgfpathlineto{\pgfqpoint{8.919029in}{1.832732in}}%
\pgfpathlineto{\pgfqpoint{8.927836in}{1.935670in}}%
\pgfpathlineto{\pgfqpoint{8.936643in}{2.004286in}}%
\pgfpathlineto{\pgfqpoint{8.945450in}{1.976851in}}%
\pgfpathlineto{\pgfqpoint{8.954257in}{1.839591in}}%
\pgfpathlineto{\pgfqpoint{8.963063in}{1.880772in}}%
\pgfpathlineto{\pgfqpoint{8.971870in}{2.031749in}}%
\pgfpathlineto{\pgfqpoint{8.980677in}{2.292524in}}%
\pgfpathlineto{\pgfqpoint{8.989484in}{2.333705in}}%
\pgfpathlineto{\pgfqpoint{8.998291in}{2.024891in}}%
\pgfpathlineto{\pgfqpoint{9.007097in}{2.326846in}}%
\pgfpathlineto{\pgfqpoint{9.015904in}{2.553326in}}%
\pgfpathlineto{\pgfqpoint{9.024711in}{2.169009in}}%
\pgfpathlineto{\pgfqpoint{9.033518in}{2.580761in}}%
\pgfpathlineto{\pgfqpoint{9.042325in}{2.169009in}}%
\pgfpathlineto{\pgfqpoint{9.051132in}{1.619997in}}%
\pgfpathlineto{\pgfqpoint{9.059938in}{1.619997in}}%
\pgfpathlineto{\pgfqpoint{9.068745in}{1.928811in}}%
\pgfpathlineto{\pgfqpoint{9.077552in}{1.681754in}}%
\pgfpathlineto{\pgfqpoint{9.086359in}{2.018032in}}%
\pgfpathlineto{\pgfqpoint{9.095166in}{1.915094in}}%
\pgfpathlineto{\pgfqpoint{9.103972in}{1.784692in}}%
\pgfpathlineto{\pgfqpoint{9.112779in}{2.052326in}}%
\pgfpathlineto{\pgfqpoint{9.121586in}{2.217049in}}%
\pgfpathlineto{\pgfqpoint{9.130393in}{1.956246in}}%
\pgfpathlineto{\pgfqpoint{9.139200in}{1.901348in}}%
\pgfpathlineto{\pgfqpoint{9.148007in}{2.066071in}}%
\pgfpathlineto{\pgfqpoint{9.156813in}{1.997427in}}%
\pgfpathlineto{\pgfqpoint{9.165620in}{2.086648in}}%
\pgfpathlineto{\pgfqpoint{9.174427in}{1.832732in}}%
\pgfpathlineto{\pgfqpoint{9.183234in}{1.537636in}}%
\pgfpathlineto{\pgfqpoint{9.192041in}{1.729794in}}%
\pgfpathlineto{\pgfqpoint{9.200847in}{1.668037in}}%
\pgfpathlineto{\pgfqpoint{9.209654in}{1.743512in}}%
\pgfpathlineto{\pgfqpoint{9.218461in}{2.045467in}}%
\pgfpathlineto{\pgfqpoint{9.227268in}{1.935670in}}%
\pgfpathlineto{\pgfqpoint{9.236075in}{2.244484in}}%
\pgfpathlineto{\pgfqpoint{9.244882in}{1.873913in}}%
\pgfpathlineto{\pgfqpoint{9.253688in}{1.819015in}}%
\pgfpathlineto{\pgfqpoint{9.262495in}{2.120970in}}%
\pgfpathlineto{\pgfqpoint{9.271302in}{1.997427in}}%
\pgfpathlineto{\pgfqpoint{9.280109in}{2.093507in}}%
\pgfpathlineto{\pgfqpoint{9.288916in}{2.011173in}}%
\pgfpathlineto{\pgfqpoint{9.297722in}{2.374885in}}%
\pgfpathlineto{\pgfqpoint{9.306529in}{2.031749in}}%
\pgfpathlineto{\pgfqpoint{9.315336in}{1.935670in}}%
\pgfpathlineto{\pgfqpoint{9.324143in}{2.237625in}}%
\pgfpathlineto{\pgfqpoint{9.332950in}{2.306269in}}%
\pgfpathlineto{\pgfqpoint{9.341757in}{2.244484in}}%
\pgfpathlineto{\pgfqpoint{9.350563in}{1.798410in}}%
\pgfpathlineto{\pgfqpoint{9.359370in}{1.674896in}}%
\pgfpathlineto{\pgfqpoint{9.368177in}{1.633715in}}%
\pgfpathlineto{\pgfqpoint{9.376984in}{1.537636in}}%
\pgfpathlineto{\pgfqpoint{9.385791in}{2.052326in}}%
\pgfpathlineto{\pgfqpoint{9.394597in}{1.681754in}}%
\pgfpathlineto{\pgfqpoint{9.403404in}{1.839591in}}%
\pgfpathlineto{\pgfqpoint{9.412211in}{1.812128in}}%
\pgfpathlineto{\pgfqpoint{9.421018in}{1.475878in}}%
\pgfpathlineto{\pgfqpoint{9.429825in}{1.969992in}}%
\pgfpathlineto{\pgfqpoint{9.438632in}{1.997427in}}%
\pgfpathlineto{\pgfqpoint{9.447438in}{1.846450in}}%
\pgfpathlineto{\pgfqpoint{9.456245in}{1.640574in}}%
\pgfpathlineto{\pgfqpoint{9.465052in}{1.647432in}}%
\pgfpathlineto{\pgfqpoint{9.473859in}{1.407234in}}%
\pgfpathlineto{\pgfqpoint{9.482666in}{1.311155in}}%
\pgfpathlineto{\pgfqpoint{9.491472in}{1.407234in}}%
\pgfpathlineto{\pgfqpoint{9.500279in}{1.420952in}}%
\pgfpathlineto{\pgfqpoint{9.509086in}{1.462133in}}%
\pgfpathlineto{\pgfqpoint{9.535507in}{1.860167in}}%
\pgfpathlineto{\pgfqpoint{9.544313in}{1.764116in}}%
\pgfpathlineto{\pgfqpoint{9.553120in}{1.716076in}}%
\pgfpathlineto{\pgfqpoint{9.561927in}{1.599393in}}%
\pgfpathlineto{\pgfqpoint{9.570734in}{1.647432in}}%
\pgfpathlineto{\pgfqpoint{9.579541in}{1.832732in}}%
\pgfpathlineto{\pgfqpoint{9.588347in}{1.921953in}}%
\pgfpathlineto{\pgfqpoint{9.597154in}{2.148405in}}%
\pgfpathlineto{\pgfqpoint{9.605961in}{1.908207in}}%
\pgfpathlineto{\pgfqpoint{9.614768in}{2.175868in}}%
\pgfpathlineto{\pgfqpoint{9.623575in}{1.908207in}}%
\pgfpathlineto{\pgfqpoint{9.632382in}{2.141546in}}%
\pgfpathlineto{\pgfqpoint{9.641188in}{1.949388in}}%
\pgfpathlineto{\pgfqpoint{9.649995in}{2.395462in}}%
\pgfpathlineto{\pgfqpoint{9.658802in}{2.484682in}}%
\pgfpathlineto{\pgfqpoint{9.667609in}{1.928811in}}%
\pgfpathlineto{\pgfqpoint{9.676416in}{1.517031in}}%
\pgfpathlineto{\pgfqpoint{9.685222in}{1.414093in}}%
\pgfpathlineto{\pgfqpoint{9.694029in}{1.167036in}}%
\pgfpathlineto{\pgfqpoint{9.702836in}{1.118997in}}%
\pgfpathlineto{\pgfqpoint{9.711643in}{1.304296in}}%
\pgfpathlineto{\pgfqpoint{9.720450in}{1.324873in}}%
\pgfpathlineto{\pgfqpoint{9.729257in}{1.571958in}}%
\pgfpathlineto{\pgfqpoint{9.738063in}{1.386658in}}%
\pgfpathlineto{\pgfqpoint{9.746870in}{1.455274in}}%
\pgfpathlineto{\pgfqpoint{9.755677in}{1.736653in}}%
\pgfpathlineto{\pgfqpoint{9.764484in}{1.372912in}}%
\pgfpathlineto{\pgfqpoint{9.773291in}{1.256257in}}%
\pgfpathlineto{\pgfqpoint{9.782097in}{1.414093in}}%
\pgfpathlineto{\pgfqpoint{9.790904in}{1.290579in}}%
\pgfpathlineto{\pgfqpoint{9.799711in}{1.324873in}}%
\pgfpathlineto{\pgfqpoint{9.808518in}{1.530777in}}%
\pgfpathlineto{\pgfqpoint{9.817325in}{1.867054in}}%
\pgfpathlineto{\pgfqpoint{9.826132in}{1.592534in}}%
\pgfpathlineto{\pgfqpoint{9.834938in}{1.928811in}}%
\pgfpathlineto{\pgfqpoint{9.843745in}{1.976851in}}%
\pgfpathlineto{\pgfqpoint{9.852552in}{2.230767in}}%
\pgfpathlineto{\pgfqpoint{9.861359in}{2.031749in}}%
\pgfpathlineto{\pgfqpoint{9.870166in}{2.134687in}}%
\pgfpathlineto{\pgfqpoint{9.878972in}{1.956246in}}%
\pgfpathlineto{\pgfqpoint{9.887779in}{1.702331in}}%
\pgfpathlineto{\pgfqpoint{9.896586in}{1.585675in}}%
\pgfpathlineto{\pgfqpoint{9.905393in}{1.784692in}}%
\pgfpathlineto{\pgfqpoint{9.914200in}{1.928811in}}%
\pgfpathlineto{\pgfqpoint{9.923007in}{2.052326in}}%
\pgfpathlineto{\pgfqpoint{9.931813in}{1.915094in}}%
\pgfpathlineto{\pgfqpoint{9.940620in}{1.921953in}}%
\pgfpathlineto{\pgfqpoint{9.949427in}{1.668037in}}%
\pgfpathlineto{\pgfqpoint{9.949427in}{1.668037in}}%
\pgfusepath{stroke}%
\end{pgfscope}%
\begin{pgfscope}%
\pgfpathrectangle{\pgfqpoint{0.702268in}{0.521603in}}{\pgfqpoint{9.687500in}{4.235000in}}%
\pgfusepath{clip}%
\pgfsetrectcap%
\pgfsetroundjoin%
\pgfsetlinewidth{0.501875pt}%
\definecolor{currentstroke}{rgb}{0.501961,0.501961,0.501961}%
\pgfsetstrokecolor{currentstroke}%
\pgfsetstrokeopacity{0.250000}%
\pgfsetdash{}{0pt}%
\pgfpathmoveto{\pgfqpoint{1.142609in}{4.413126in}}%
\pgfpathlineto{\pgfqpoint{1.151416in}{3.589593in}}%
\pgfpathlineto{\pgfqpoint{1.160222in}{2.573903in}}%
\pgfpathlineto{\pgfqpoint{1.169029in}{2.464106in}}%
\pgfpathlineto{\pgfqpoint{1.177836in}{2.210190in}}%
\pgfpathlineto{\pgfqpoint{1.186643in}{2.512146in}}%
\pgfpathlineto{\pgfqpoint{1.195450in}{2.354309in}}%
\pgfpathlineto{\pgfqpoint{1.204257in}{2.340563in}}%
\pgfpathlineto{\pgfqpoint{1.213063in}{1.757229in}}%
\pgfpathlineto{\pgfqpoint{1.221870in}{1.681754in}}%
\pgfpathlineto{\pgfqpoint{1.230677in}{1.846450in}}%
\pgfpathlineto{\pgfqpoint{1.239484in}{1.791551in}}%
\pgfpathlineto{\pgfqpoint{1.248291in}{2.114111in}}%
\pgfpathlineto{\pgfqpoint{1.257097in}{1.887630in}}%
\pgfpathlineto{\pgfqpoint{1.265904in}{1.626856in}}%
\pgfpathlineto{\pgfqpoint{1.274711in}{2.162151in}}%
\pgfpathlineto{\pgfqpoint{1.283518in}{2.072930in}}%
\pgfpathlineto{\pgfqpoint{1.292325in}{1.880772in}}%
\pgfpathlineto{\pgfqpoint{1.301132in}{1.613110in}}%
\pgfpathlineto{\pgfqpoint{1.309938in}{1.867054in}}%
\pgfpathlineto{\pgfqpoint{1.318745in}{1.963133in}}%
\pgfpathlineto{\pgfqpoint{1.327552in}{2.265089in}}%
\pgfpathlineto{\pgfqpoint{1.336359in}{2.505287in}}%
\pgfpathlineto{\pgfqpoint{1.345166in}{2.347422in}}%
\pgfpathlineto{\pgfqpoint{1.353972in}{2.354309in}}%
\pgfpathlineto{\pgfqpoint{1.371586in}{2.608225in}}%
\pgfpathlineto{\pgfqpoint{1.380393in}{2.066071in}}%
\pgfpathlineto{\pgfqpoint{1.389200in}{1.777834in}}%
\pgfpathlineto{\pgfqpoint{1.398007in}{1.619997in}}%
\pgfpathlineto{\pgfqpoint{1.406813in}{1.956246in}}%
\pgfpathlineto{\pgfqpoint{1.415620in}{1.805269in}}%
\pgfpathlineto{\pgfqpoint{1.424427in}{2.024891in}}%
\pgfpathlineto{\pgfqpoint{1.433234in}{2.052326in}}%
\pgfpathlineto{\pgfqpoint{1.450847in}{1.908207in}}%
\pgfpathlineto{\pgfqpoint{1.459654in}{1.832732in}}%
\pgfpathlineto{\pgfqpoint{1.468461in}{1.702331in}}%
\pgfpathlineto{\pgfqpoint{1.477268in}{1.853308in}}%
\pgfpathlineto{\pgfqpoint{1.486075in}{1.839591in}}%
\pgfpathlineto{\pgfqpoint{1.503688in}{2.120970in}}%
\pgfpathlineto{\pgfqpoint{1.512495in}{1.544494in}}%
\pgfpathlineto{\pgfqpoint{1.521302in}{1.716076in}}%
\pgfpathlineto{\pgfqpoint{1.530109in}{1.613110in}}%
\pgfpathlineto{\pgfqpoint{1.538916in}{1.640574in}}%
\pgfpathlineto{\pgfqpoint{1.547722in}{1.290579in}}%
\pgfpathlineto{\pgfqpoint{1.556529in}{1.414093in}}%
\pgfpathlineto{\pgfqpoint{1.565336in}{1.722935in}}%
\pgfpathlineto{\pgfqpoint{1.574143in}{1.668037in}}%
\pgfpathlineto{\pgfqpoint{1.582950in}{1.825873in}}%
\pgfpathlineto{\pgfqpoint{1.600563in}{1.537636in}}%
\pgfpathlineto{\pgfqpoint{1.609370in}{1.462133in}}%
\pgfpathlineto{\pgfqpoint{1.618177in}{1.729794in}}%
\pgfpathlineto{\pgfqpoint{1.626984in}{1.867054in}}%
\pgfpathlineto{\pgfqpoint{1.635791in}{1.784692in}}%
\pgfpathlineto{\pgfqpoint{1.644597in}{1.825873in}}%
\pgfpathlineto{\pgfqpoint{1.662211in}{2.546468in}}%
\pgfpathlineto{\pgfqpoint{1.671018in}{2.079789in}}%
\pgfpathlineto{\pgfqpoint{1.679825in}{2.100365in}}%
\pgfpathlineto{\pgfqpoint{1.688632in}{1.963133in}}%
\pgfpathlineto{\pgfqpoint{1.697438in}{1.969992in}}%
\pgfpathlineto{\pgfqpoint{1.706245in}{2.031749in}}%
\pgfpathlineto{\pgfqpoint{1.715052in}{2.079789in}}%
\pgfpathlineto{\pgfqpoint{1.723859in}{2.374885in}}%
\pgfpathlineto{\pgfqpoint{1.732666in}{2.038608in}}%
\pgfpathlineto{\pgfqpoint{1.741472in}{1.860167in}}%
\pgfpathlineto{\pgfqpoint{1.750279in}{1.784692in}}%
\pgfpathlineto{\pgfqpoint{1.759086in}{1.764116in}}%
\pgfpathlineto{\pgfqpoint{1.767893in}{1.832732in}}%
\pgfpathlineto{\pgfqpoint{1.776700in}{1.777834in}}%
\pgfpathlineto{\pgfqpoint{1.785507in}{1.695472in}}%
\pgfpathlineto{\pgfqpoint{1.794313in}{1.812128in}}%
\pgfpathlineto{\pgfqpoint{1.803120in}{1.777834in}}%
\pgfpathlineto{\pgfqpoint{1.811927in}{1.510172in}}%
\pgfpathlineto{\pgfqpoint{1.820734in}{1.846450in}}%
\pgfpathlineto{\pgfqpoint{1.829541in}{1.716076in}}%
\pgfpathlineto{\pgfqpoint{1.838347in}{1.427839in}}%
\pgfpathlineto{\pgfqpoint{1.847154in}{1.805269in}}%
\pgfpathlineto{\pgfqpoint{1.855961in}{1.853308in}}%
\pgfpathlineto{\pgfqpoint{1.864768in}{1.853308in}}%
\pgfpathlineto{\pgfqpoint{1.873575in}{1.777834in}}%
\pgfpathlineto{\pgfqpoint{1.891188in}{1.544494in}}%
\pgfpathlineto{\pgfqpoint{1.899995in}{1.173895in}}%
\pgfpathlineto{\pgfqpoint{1.908802in}{1.043522in}}%
\pgfpathlineto{\pgfqpoint{1.917609in}{1.146460in}}%
\pgfpathlineto{\pgfqpoint{1.926416in}{1.070957in}}%
\pgfpathlineto{\pgfqpoint{1.935222in}{0.974878in}}%
\pgfpathlineto{\pgfqpoint{1.944029in}{0.947443in}}%
\pgfpathlineto{\pgfqpoint{1.952836in}{1.029776in}}%
\pgfpathlineto{\pgfqpoint{1.961643in}{1.201358in}}%
\pgfpathlineto{\pgfqpoint{1.970450in}{1.558212in}}%
\pgfpathlineto{\pgfqpoint{1.979257in}{2.018032in}}%
\pgfpathlineto{\pgfqpoint{1.988063in}{1.709190in}}%
\pgfpathlineto{\pgfqpoint{1.996870in}{1.448415in}}%
\pgfpathlineto{\pgfqpoint{2.005677in}{1.722935in}}%
\pgfpathlineto{\pgfqpoint{2.014484in}{1.791551in}}%
\pgfpathlineto{\pgfqpoint{2.023291in}{1.915094in}}%
\pgfpathlineto{\pgfqpoint{2.032097in}{1.750370in}}%
\pgfpathlineto{\pgfqpoint{2.040904in}{1.386658in}}%
\pgfpathlineto{\pgfqpoint{2.049711in}{1.812128in}}%
\pgfpathlineto{\pgfqpoint{2.058518in}{1.640574in}}%
\pgfpathlineto{\pgfqpoint{2.067325in}{1.819015in}}%
\pgfpathlineto{\pgfqpoint{2.076132in}{1.867054in}}%
\pgfpathlineto{\pgfqpoint{2.084938in}{1.963133in}}%
\pgfpathlineto{\pgfqpoint{2.093745in}{1.860167in}}%
\pgfpathlineto{\pgfqpoint{2.102552in}{1.613110in}}%
\pgfpathlineto{\pgfqpoint{2.111359in}{1.489596in}}%
\pgfpathlineto{\pgfqpoint{2.120166in}{1.674896in}}%
\pgfpathlineto{\pgfqpoint{2.128972in}{1.688613in}}%
\pgfpathlineto{\pgfqpoint{2.137779in}{1.537636in}}%
\pgfpathlineto{\pgfqpoint{2.146586in}{1.489596in}}%
\pgfpathlineto{\pgfqpoint{2.155393in}{1.832732in}}%
\pgfpathlineto{\pgfqpoint{2.164200in}{1.510172in}}%
\pgfpathlineto{\pgfqpoint{2.173007in}{1.805269in}}%
\pgfpathlineto{\pgfqpoint{2.181813in}{1.654291in}}%
\pgfpathlineto{\pgfqpoint{2.190620in}{1.674896in}}%
\pgfpathlineto{\pgfqpoint{2.199427in}{1.764116in}}%
\pgfpathlineto{\pgfqpoint{2.217041in}{2.175868in}}%
\pgfpathlineto{\pgfqpoint{2.225847in}{1.681754in}}%
\pgfpathlineto{\pgfqpoint{2.243461in}{1.256257in}}%
\pgfpathlineto{\pgfqpoint{2.252268in}{1.297438in}}%
\pgfpathlineto{\pgfqpoint{2.261075in}{1.530777in}}%
\pgfpathlineto{\pgfqpoint{2.269882in}{2.045467in}}%
\pgfpathlineto{\pgfqpoint{2.278688in}{2.230767in}}%
\pgfpathlineto{\pgfqpoint{2.287495in}{2.251343in}}%
\pgfpathlineto{\pgfqpoint{2.296302in}{1.928811in}}%
\pgfpathlineto{\pgfqpoint{2.305109in}{2.230767in}}%
\pgfpathlineto{\pgfqpoint{2.313916in}{1.777834in}}%
\pgfpathlineto{\pgfqpoint{2.322722in}{1.935670in}}%
\pgfpathlineto{\pgfqpoint{2.331529in}{2.169009in}}%
\pgfpathlineto{\pgfqpoint{2.340336in}{2.107252in}}%
\pgfpathlineto{\pgfqpoint{2.349143in}{2.306269in}}%
\pgfpathlineto{\pgfqpoint{2.357950in}{2.567044in}}%
\pgfpathlineto{\pgfqpoint{2.366757in}{2.230767in}}%
\pgfpathlineto{\pgfqpoint{2.375563in}{2.114111in}}%
\pgfpathlineto{\pgfqpoint{2.384370in}{2.086648in}}%
\pgfpathlineto{\pgfqpoint{2.401984in}{1.942529in}}%
\pgfpathlineto{\pgfqpoint{2.410791in}{1.729794in}}%
\pgfpathlineto{\pgfqpoint{2.419597in}{1.867054in}}%
\pgfpathlineto{\pgfqpoint{2.428404in}{2.079789in}}%
\pgfpathlineto{\pgfqpoint{2.437211in}{2.120970in}}%
\pgfpathlineto{\pgfqpoint{2.446018in}{1.935670in}}%
\pgfpathlineto{\pgfqpoint{2.454825in}{1.915094in}}%
\pgfpathlineto{\pgfqpoint{2.463632in}{1.674896in}}%
\pgfpathlineto{\pgfqpoint{2.472438in}{1.510172in}}%
\pgfpathlineto{\pgfqpoint{2.481245in}{1.873913in}}%
\pgfpathlineto{\pgfqpoint{2.490052in}{1.949388in}}%
\pgfpathlineto{\pgfqpoint{2.498859in}{1.921953in}}%
\pgfpathlineto{\pgfqpoint{2.507666in}{1.633715in}}%
\pgfpathlineto{\pgfqpoint{2.516472in}{1.819015in}}%
\pgfpathlineto{\pgfqpoint{2.525279in}{1.798410in}}%
\pgfpathlineto{\pgfqpoint{2.534086in}{1.956246in}}%
\pgfpathlineto{\pgfqpoint{2.542893in}{1.654291in}}%
\pgfpathlineto{\pgfqpoint{2.551700in}{1.613110in}}%
\pgfpathlineto{\pgfqpoint{2.560507in}{1.626856in}}%
\pgfpathlineto{\pgfqpoint{2.569313in}{1.668037in}}%
\pgfpathlineto{\pgfqpoint{2.586927in}{1.770975in}}%
\pgfpathlineto{\pgfqpoint{2.595734in}{1.764116in}}%
\pgfpathlineto{\pgfqpoint{2.604541in}{2.011173in}}%
\pgfpathlineto{\pgfqpoint{2.613347in}{1.873913in}}%
\pgfpathlineto{\pgfqpoint{2.622154in}{2.018032in}}%
\pgfpathlineto{\pgfqpoint{2.630961in}{2.114111in}}%
\pgfpathlineto{\pgfqpoint{2.639768in}{2.278806in}}%
\pgfpathlineto{\pgfqpoint{2.648575in}{2.059213in}}%
\pgfpathlineto{\pgfqpoint{2.657382in}{1.963133in}}%
\pgfpathlineto{\pgfqpoint{2.666188in}{1.894489in}}%
\pgfpathlineto{\pgfqpoint{2.674995in}{2.045467in}}%
\pgfpathlineto{\pgfqpoint{2.683802in}{2.018032in}}%
\pgfpathlineto{\pgfqpoint{2.692609in}{2.470965in}}%
\pgfpathlineto{\pgfqpoint{2.701416in}{2.553326in}}%
\pgfpathlineto{\pgfqpoint{2.710222in}{2.244484in}}%
\pgfpathlineto{\pgfqpoint{2.719029in}{2.663123in}}%
\pgfpathlineto{\pgfqpoint{2.727836in}{2.340563in}}%
\pgfpathlineto{\pgfqpoint{2.736643in}{2.519004in}}%
\pgfpathlineto{\pgfqpoint{2.745450in}{2.299383in}}%
\pgfpathlineto{\pgfqpoint{2.754257in}{2.038608in}}%
\pgfpathlineto{\pgfqpoint{2.763063in}{2.381744in}}%
\pgfpathlineto{\pgfqpoint{2.771870in}{2.525863in}}%
\pgfpathlineto{\pgfqpoint{2.780677in}{2.223908in}}%
\pgfpathlineto{\pgfqpoint{2.789484in}{2.278806in}}%
\pgfpathlineto{\pgfqpoint{2.798291in}{2.072930in}}%
\pgfpathlineto{\pgfqpoint{2.807097in}{2.031749in}}%
\pgfpathlineto{\pgfqpoint{2.815904in}{2.175868in}}%
\pgfpathlineto{\pgfqpoint{2.824711in}{2.072930in}}%
\pgfpathlineto{\pgfqpoint{2.833518in}{2.422925in}}%
\pgfpathlineto{\pgfqpoint{2.842325in}{2.134687in}}%
\pgfpathlineto{\pgfqpoint{2.851132in}{1.887630in}}%
\pgfpathlineto{\pgfqpoint{2.859938in}{1.990569in}}%
\pgfpathlineto{\pgfqpoint{2.868745in}{2.024891in}}%
\pgfpathlineto{\pgfqpoint{2.877552in}{1.716076in}}%
\pgfpathlineto{\pgfqpoint{2.886359in}{1.681754in}}%
\pgfpathlineto{\pgfqpoint{2.895166in}{1.640574in}}%
\pgfpathlineto{\pgfqpoint{2.903972in}{1.901348in}}%
\pgfpathlineto{\pgfqpoint{2.912779in}{1.661150in}}%
\pgfpathlineto{\pgfqpoint{2.921586in}{1.558212in}}%
\pgfpathlineto{\pgfqpoint{2.930393in}{1.517031in}}%
\pgfpathlineto{\pgfqpoint{2.948007in}{2.031749in}}%
\pgfpathlineto{\pgfqpoint{2.965620in}{2.210190in}}%
\pgfpathlineto{\pgfqpoint{2.974427in}{1.880772in}}%
\pgfpathlineto{\pgfqpoint{2.983234in}{1.860167in}}%
\pgfpathlineto{\pgfqpoint{2.992041in}{1.942529in}}%
\pgfpathlineto{\pgfqpoint{3.000847in}{1.976851in}}%
\pgfpathlineto{\pgfqpoint{3.009654in}{1.935670in}}%
\pgfpathlineto{\pgfqpoint{3.018461in}{2.038608in}}%
\pgfpathlineto{\pgfqpoint{3.027268in}{1.839591in}}%
\pgfpathlineto{\pgfqpoint{3.036075in}{1.537636in}}%
\pgfpathlineto{\pgfqpoint{3.044882in}{1.633715in}}%
\pgfpathlineto{\pgfqpoint{3.053688in}{1.599393in}}%
\pgfpathlineto{\pgfqpoint{3.062495in}{1.489596in}}%
\pgfpathlineto{\pgfqpoint{3.071302in}{1.647432in}}%
\pgfpathlineto{\pgfqpoint{3.080109in}{1.736653in}}%
\pgfpathlineto{\pgfqpoint{3.088916in}{2.038608in}}%
\pgfpathlineto{\pgfqpoint{3.097722in}{2.072930in}}%
\pgfpathlineto{\pgfqpoint{3.106529in}{2.553326in}}%
\pgfpathlineto{\pgfqpoint{3.115336in}{2.223908in}}%
\pgfpathlineto{\pgfqpoint{3.124143in}{2.059213in}}%
\pgfpathlineto{\pgfqpoint{3.132950in}{2.141546in}}%
\pgfpathlineto{\pgfqpoint{3.141757in}{2.134687in}}%
\pgfpathlineto{\pgfqpoint{3.150563in}{2.182727in}}%
\pgfpathlineto{\pgfqpoint{3.159370in}{2.656264in}}%
\pgfpathlineto{\pgfqpoint{3.168177in}{2.223908in}}%
\pgfpathlineto{\pgfqpoint{3.176984in}{1.969992in}}%
\pgfpathlineto{\pgfqpoint{3.185791in}{2.347422in}}%
\pgfpathlineto{\pgfqpoint{3.194597in}{2.271947in}}%
\pgfpathlineto{\pgfqpoint{3.203404in}{2.148405in}}%
\pgfpathlineto{\pgfqpoint{3.212211in}{2.189586in}}%
\pgfpathlineto{\pgfqpoint{3.221018in}{2.004286in}}%
\pgfpathlineto{\pgfqpoint{3.229825in}{2.230767in}}%
\pgfpathlineto{\pgfqpoint{3.238632in}{2.237625in}}%
\pgfpathlineto{\pgfqpoint{3.247438in}{2.326846in}}%
\pgfpathlineto{\pgfqpoint{3.256245in}{2.230767in}}%
\pgfpathlineto{\pgfqpoint{3.265052in}{2.573903in}}%
\pgfpathlineto{\pgfqpoint{3.273859in}{2.484682in}}%
\pgfpathlineto{\pgfqpoint{3.282666in}{2.718022in}}%
\pgfpathlineto{\pgfqpoint{3.300279in}{2.134687in}}%
\pgfpathlineto{\pgfqpoint{3.309086in}{2.134687in}}%
\pgfpathlineto{\pgfqpoint{3.317893in}{2.169009in}}%
\pgfpathlineto{\pgfqpoint{3.326700in}{2.223908in}}%
\pgfpathlineto{\pgfqpoint{3.335507in}{2.326846in}}%
\pgfpathlineto{\pgfqpoint{3.344313in}{1.860167in}}%
\pgfpathlineto{\pgfqpoint{3.353120in}{1.668037in}}%
\pgfpathlineto{\pgfqpoint{3.361927in}{1.633715in}}%
\pgfpathlineto{\pgfqpoint{3.370734in}{1.935670in}}%
\pgfpathlineto{\pgfqpoint{3.379541in}{2.079789in}}%
\pgfpathlineto{\pgfqpoint{3.388347in}{1.942529in}}%
\pgfpathlineto{\pgfqpoint{3.397154in}{1.901348in}}%
\pgfpathlineto{\pgfqpoint{3.405961in}{2.162151in}}%
\pgfpathlineto{\pgfqpoint{3.423575in}{1.832732in}}%
\pgfpathlineto{\pgfqpoint{3.432382in}{1.592534in}}%
\pgfpathlineto{\pgfqpoint{3.441188in}{1.722935in}}%
\pgfpathlineto{\pgfqpoint{3.449995in}{1.654291in}}%
\pgfpathlineto{\pgfqpoint{3.458802in}{1.517031in}}%
\pgfpathlineto{\pgfqpoint{3.467609in}{1.551353in}}%
\pgfpathlineto{\pgfqpoint{3.476416in}{1.846450in}}%
\pgfpathlineto{\pgfqpoint{3.485222in}{1.942529in}}%
\pgfpathlineto{\pgfqpoint{3.494029in}{1.887630in}}%
\pgfpathlineto{\pgfqpoint{3.502836in}{2.038608in}}%
\pgfpathlineto{\pgfqpoint{3.511643in}{1.729794in}}%
\pgfpathlineto{\pgfqpoint{3.520450in}{1.681754in}}%
\pgfpathlineto{\pgfqpoint{3.529257in}{1.997427in}}%
\pgfpathlineto{\pgfqpoint{3.538063in}{1.798410in}}%
\pgfpathlineto{\pgfqpoint{3.546870in}{1.812128in}}%
\pgfpathlineto{\pgfqpoint{3.555677in}{1.894489in}}%
\pgfpathlineto{\pgfqpoint{3.564484in}{2.354309in}}%
\pgfpathlineto{\pgfqpoint{3.573291in}{2.519004in}}%
\pgfpathlineto{\pgfqpoint{3.582097in}{2.834677in}}%
\pgfpathlineto{\pgfqpoint{3.590904in}{2.766061in}}%
\pgfpathlineto{\pgfqpoint{3.599711in}{2.766061in}}%
\pgfpathlineto{\pgfqpoint{3.617325in}{2.313128in}}%
\pgfpathlineto{\pgfqpoint{3.626132in}{2.450388in}}%
\pgfpathlineto{\pgfqpoint{3.634938in}{2.313128in}}%
\pgfpathlineto{\pgfqpoint{3.643745in}{2.093507in}}%
\pgfpathlineto{\pgfqpoint{3.652552in}{2.134687in}}%
\pgfpathlineto{\pgfqpoint{3.661359in}{1.764116in}}%
\pgfpathlineto{\pgfqpoint{3.670166in}{1.695472in}}%
\pgfpathlineto{\pgfqpoint{3.678972in}{1.880772in}}%
\pgfpathlineto{\pgfqpoint{3.687779in}{1.654291in}}%
\pgfpathlineto{\pgfqpoint{3.696586in}{1.770975in}}%
\pgfpathlineto{\pgfqpoint{3.705393in}{1.812128in}}%
\pgfpathlineto{\pgfqpoint{3.714200in}{1.764116in}}%
\pgfpathlineto{\pgfqpoint{3.723007in}{2.018032in}}%
\pgfpathlineto{\pgfqpoint{3.731813in}{2.306269in}}%
\pgfpathlineto{\pgfqpoint{3.740620in}{2.155292in}}%
\pgfpathlineto{\pgfqpoint{3.749427in}{2.162151in}}%
\pgfpathlineto{\pgfqpoint{3.758234in}{1.812128in}}%
\pgfpathlineto{\pgfqpoint{3.767041in}{1.722935in}}%
\pgfpathlineto{\pgfqpoint{3.775847in}{1.606252in}}%
\pgfpathlineto{\pgfqpoint{3.784654in}{1.523918in}}%
\pgfpathlineto{\pgfqpoint{3.793461in}{1.668037in}}%
\pgfpathlineto{\pgfqpoint{3.802268in}{1.997427in}}%
\pgfpathlineto{\pgfqpoint{3.811075in}{2.402349in}}%
\pgfpathlineto{\pgfqpoint{3.819882in}{2.258230in}}%
\pgfpathlineto{\pgfqpoint{3.828688in}{2.285665in}}%
\pgfpathlineto{\pgfqpoint{3.837495in}{1.832732in}}%
\pgfpathlineto{\pgfqpoint{3.846302in}{1.764116in}}%
\pgfpathlineto{\pgfqpoint{3.855109in}{1.921953in}}%
\pgfpathlineto{\pgfqpoint{3.863916in}{1.935670in}}%
\pgfpathlineto{\pgfqpoint{3.872722in}{1.578816in}}%
\pgfpathlineto{\pgfqpoint{3.881529in}{1.331760in}}%
\pgfpathlineto{\pgfqpoint{3.890336in}{1.798410in}}%
\pgfpathlineto{\pgfqpoint{3.899143in}{2.079789in}}%
\pgfpathlineto{\pgfqpoint{3.907950in}{2.196445in}}%
\pgfpathlineto{\pgfqpoint{3.916757in}{2.162151in}}%
\pgfpathlineto{\pgfqpoint{3.925563in}{2.169009in}}%
\pgfpathlineto{\pgfqpoint{3.934370in}{2.244484in}}%
\pgfpathlineto{\pgfqpoint{3.943177in}{2.258230in}}%
\pgfpathlineto{\pgfqpoint{3.951984in}{1.709190in}}%
\pgfpathlineto{\pgfqpoint{3.960791in}{1.860167in}}%
\pgfpathlineto{\pgfqpoint{3.969597in}{1.647432in}}%
\pgfpathlineto{\pgfqpoint{3.978404in}{1.750370in}}%
\pgfpathlineto{\pgfqpoint{3.987211in}{1.695472in}}%
\pgfpathlineto{\pgfqpoint{3.996018in}{1.880772in}}%
\pgfpathlineto{\pgfqpoint{4.004825in}{1.894489in}}%
\pgfpathlineto{\pgfqpoint{4.013632in}{1.997427in}}%
\pgfpathlineto{\pgfqpoint{4.022438in}{1.928811in}}%
\pgfpathlineto{\pgfqpoint{4.031245in}{1.963133in}}%
\pgfpathlineto{\pgfqpoint{4.040052in}{1.853308in}}%
\pgfpathlineto{\pgfqpoint{4.048859in}{1.764116in}}%
\pgfpathlineto{\pgfqpoint{4.057666in}{1.853308in}}%
\pgfpathlineto{\pgfqpoint{4.075279in}{1.619997in}}%
\pgfpathlineto{\pgfqpoint{4.084086in}{1.544494in}}%
\pgfpathlineto{\pgfqpoint{4.092893in}{1.825873in}}%
\pgfpathlineto{\pgfqpoint{4.101700in}{1.736653in}}%
\pgfpathlineto{\pgfqpoint{4.110507in}{1.867054in}}%
\pgfpathlineto{\pgfqpoint{4.119313in}{1.915094in}}%
\pgfpathlineto{\pgfqpoint{4.128120in}{2.278806in}}%
\pgfpathlineto{\pgfqpoint{4.136927in}{1.908207in}}%
\pgfpathlineto{\pgfqpoint{4.145734in}{1.729794in}}%
\pgfpathlineto{\pgfqpoint{4.154541in}{1.386658in}}%
\pgfpathlineto{\pgfqpoint{4.163347in}{1.462133in}}%
\pgfpathlineto{\pgfqpoint{4.172154in}{1.558212in}}%
\pgfpathlineto{\pgfqpoint{4.180961in}{1.819015in}}%
\pgfpathlineto{\pgfqpoint{4.189768in}{1.860167in}}%
\pgfpathlineto{\pgfqpoint{4.198575in}{1.846450in}}%
\pgfpathlineto{\pgfqpoint{4.207382in}{2.107252in}}%
\pgfpathlineto{\pgfqpoint{4.216188in}{2.278806in}}%
\pgfpathlineto{\pgfqpoint{4.224995in}{2.175868in}}%
\pgfpathlineto{\pgfqpoint{4.233802in}{1.853308in}}%
\pgfpathlineto{\pgfqpoint{4.242609in}{1.791551in}}%
\pgfpathlineto{\pgfqpoint{4.251416in}{1.976851in}}%
\pgfpathlineto{\pgfqpoint{4.260222in}{2.086648in}}%
\pgfpathlineto{\pgfqpoint{4.277836in}{1.743512in}}%
\pgfpathlineto{\pgfqpoint{4.286643in}{1.846450in}}%
\pgfpathlineto{\pgfqpoint{4.295450in}{1.468992in}}%
\pgfpathlineto{\pgfqpoint{4.304257in}{1.873913in}}%
\pgfpathlineto{\pgfqpoint{4.313063in}{1.969992in}}%
\pgfpathlineto{\pgfqpoint{4.321870in}{1.983710in}}%
\pgfpathlineto{\pgfqpoint{4.330677in}{2.052326in}}%
\pgfpathlineto{\pgfqpoint{4.339484in}{1.969992in}}%
\pgfpathlineto{\pgfqpoint{4.348291in}{1.908207in}}%
\pgfpathlineto{\pgfqpoint{4.357097in}{2.127829in}}%
\pgfpathlineto{\pgfqpoint{4.365904in}{2.182727in}}%
\pgfpathlineto{\pgfqpoint{4.374711in}{2.072930in}}%
\pgfpathlineto{\pgfqpoint{4.383518in}{2.217049in}}%
\pgfpathlineto{\pgfqpoint{4.392325in}{2.141546in}}%
\pgfpathlineto{\pgfqpoint{4.401132in}{1.661150in}}%
\pgfpathlineto{\pgfqpoint{4.409938in}{1.764116in}}%
\pgfpathlineto{\pgfqpoint{4.418745in}{1.537636in}}%
\pgfpathlineto{\pgfqpoint{4.427552in}{1.441556in}}%
\pgfpathlineto{\pgfqpoint{4.436359in}{1.647432in}}%
\pgfpathlineto{\pgfqpoint{4.445166in}{2.182727in}}%
\pgfpathlineto{\pgfqpoint{4.453972in}{2.278806in}}%
\pgfpathlineto{\pgfqpoint{4.462779in}{2.072930in}}%
\pgfpathlineto{\pgfqpoint{4.471586in}{1.736653in}}%
\pgfpathlineto{\pgfqpoint{4.480393in}{1.784692in}}%
\pgfpathlineto{\pgfqpoint{4.489200in}{1.606252in}}%
\pgfpathlineto{\pgfqpoint{4.498007in}{1.475878in}}%
\pgfpathlineto{\pgfqpoint{4.506813in}{1.688613in}}%
\pgfpathlineto{\pgfqpoint{4.515620in}{1.743512in}}%
\pgfpathlineto{\pgfqpoint{4.524427in}{1.517031in}}%
\pgfpathlineto{\pgfqpoint{4.533234in}{1.722935in}}%
\pgfpathlineto{\pgfqpoint{4.542041in}{2.114111in}}%
\pgfpathlineto{\pgfqpoint{4.550847in}{2.072930in}}%
\pgfpathlineto{\pgfqpoint{4.559654in}{2.127829in}}%
\pgfpathlineto{\pgfqpoint{4.568461in}{2.066071in}}%
\pgfpathlineto{\pgfqpoint{4.577268in}{2.107252in}}%
\pgfpathlineto{\pgfqpoint{4.586075in}{1.921953in}}%
\pgfpathlineto{\pgfqpoint{4.594882in}{1.908207in}}%
\pgfpathlineto{\pgfqpoint{4.603688in}{2.381744in}}%
\pgfpathlineto{\pgfqpoint{4.612495in}{2.512146in}}%
\pgfpathlineto{\pgfqpoint{4.621302in}{2.436643in}}%
\pgfpathlineto{\pgfqpoint{4.630109in}{2.457247in}}%
\pgfpathlineto{\pgfqpoint{4.638916in}{2.422925in}}%
\pgfpathlineto{\pgfqpoint{4.647722in}{2.422925in}}%
\pgfpathlineto{\pgfqpoint{4.656529in}{2.271947in}}%
\pgfpathlineto{\pgfqpoint{4.665336in}{1.770975in}}%
\pgfpathlineto{\pgfqpoint{4.674143in}{1.928811in}}%
\pgfpathlineto{\pgfqpoint{4.682950in}{1.915094in}}%
\pgfpathlineto{\pgfqpoint{4.691757in}{2.217049in}}%
\pgfpathlineto{\pgfqpoint{4.700563in}{2.038608in}}%
\pgfpathlineto{\pgfqpoint{4.709370in}{1.894489in}}%
\pgfpathlineto{\pgfqpoint{4.718177in}{1.482737in}}%
\pgfpathlineto{\pgfqpoint{4.726984in}{1.194499in}}%
\pgfpathlineto{\pgfqpoint{4.735791in}{1.496455in}}%
\pgfpathlineto{\pgfqpoint{4.744597in}{1.441556in}}%
\pgfpathlineto{\pgfqpoint{4.753404in}{2.066071in}}%
\pgfpathlineto{\pgfqpoint{4.762211in}{2.031749in}}%
\pgfpathlineto{\pgfqpoint{4.771018in}{1.366053in}}%
\pgfpathlineto{\pgfqpoint{4.779825in}{1.537636in}}%
\pgfpathlineto{\pgfqpoint{4.788632in}{1.633715in}}%
\pgfpathlineto{\pgfqpoint{4.797438in}{1.386658in}}%
\pgfpathlineto{\pgfqpoint{4.815052in}{1.963133in}}%
\pgfpathlineto{\pgfqpoint{4.823859in}{1.812128in}}%
\pgfpathlineto{\pgfqpoint{4.832666in}{1.619997in}}%
\pgfpathlineto{\pgfqpoint{4.841472in}{1.777834in}}%
\pgfpathlineto{\pgfqpoint{4.850279in}{1.901348in}}%
\pgfpathlineto{\pgfqpoint{4.859086in}{2.409207in}}%
\pgfpathlineto{\pgfqpoint{4.867893in}{2.313128in}}%
\pgfpathlineto{\pgfqpoint{4.876700in}{2.079789in}}%
\pgfpathlineto{\pgfqpoint{4.885507in}{2.395462in}}%
\pgfpathlineto{\pgfqpoint{4.894313in}{2.196445in}}%
\pgfpathlineto{\pgfqpoint{4.911927in}{1.908207in}}%
\pgfpathlineto{\pgfqpoint{4.920734in}{2.100365in}}%
\pgfpathlineto{\pgfqpoint{4.929541in}{1.819015in}}%
\pgfpathlineto{\pgfqpoint{4.938347in}{2.100365in}}%
\pgfpathlineto{\pgfqpoint{4.947154in}{2.066071in}}%
\pgfpathlineto{\pgfqpoint{4.955961in}{1.784692in}}%
\pgfpathlineto{\pgfqpoint{4.964768in}{1.887630in}}%
\pgfpathlineto{\pgfqpoint{4.973575in}{1.784692in}}%
\pgfpathlineto{\pgfqpoint{4.982382in}{1.853308in}}%
\pgfpathlineto{\pgfqpoint{4.991188in}{1.654291in}}%
\pgfpathlineto{\pgfqpoint{4.999995in}{1.750370in}}%
\pgfpathlineto{\pgfqpoint{5.008802in}{2.107252in}}%
\pgfpathlineto{\pgfqpoint{5.017609in}{1.825873in}}%
\pgfpathlineto{\pgfqpoint{5.026416in}{1.626856in}}%
\pgfpathlineto{\pgfqpoint{5.035222in}{1.901348in}}%
\pgfpathlineto{\pgfqpoint{5.044029in}{1.921953in}}%
\pgfpathlineto{\pgfqpoint{5.052836in}{1.956246in}}%
\pgfpathlineto{\pgfqpoint{5.061643in}{1.784692in}}%
\pgfpathlineto{\pgfqpoint{5.070450in}{1.819015in}}%
\pgfpathlineto{\pgfqpoint{5.088063in}{1.688613in}}%
\pgfpathlineto{\pgfqpoint{5.096870in}{1.585675in}}%
\pgfpathlineto{\pgfqpoint{5.105677in}{1.894489in}}%
\pgfpathlineto{\pgfqpoint{5.114484in}{1.894489in}}%
\pgfpathlineto{\pgfqpoint{5.123291in}{1.791551in}}%
\pgfpathlineto{\pgfqpoint{5.132097in}{1.517031in}}%
\pgfpathlineto{\pgfqpoint{5.140904in}{1.338618in}}%
\pgfpathlineto{\pgfqpoint{5.149711in}{1.626856in}}%
\pgfpathlineto{\pgfqpoint{5.158518in}{1.592534in}}%
\pgfpathlineto{\pgfqpoint{5.167325in}{1.791551in}}%
\pgfpathlineto{\pgfqpoint{5.176132in}{1.805269in}}%
\pgfpathlineto{\pgfqpoint{5.184938in}{1.565071in}}%
\pgfpathlineto{\pgfqpoint{5.193745in}{1.832732in}}%
\pgfpathlineto{\pgfqpoint{5.202552in}{1.853308in}}%
\pgfpathlineto{\pgfqpoint{5.211359in}{1.949388in}}%
\pgfpathlineto{\pgfqpoint{5.220166in}{2.059213in}}%
\pgfpathlineto{\pgfqpoint{5.228972in}{2.052326in}}%
\pgfpathlineto{\pgfqpoint{5.237779in}{1.805269in}}%
\pgfpathlineto{\pgfqpoint{5.246586in}{1.523918in}}%
\pgfpathlineto{\pgfqpoint{5.255393in}{1.613110in}}%
\pgfpathlineto{\pgfqpoint{5.264200in}{1.873913in}}%
\pgfpathlineto{\pgfqpoint{5.281813in}{2.525863in}}%
\pgfpathlineto{\pgfqpoint{5.290620in}{2.079789in}}%
\pgfpathlineto{\pgfqpoint{5.299427in}{2.004286in}}%
\pgfpathlineto{\pgfqpoint{5.308234in}{1.873913in}}%
\pgfpathlineto{\pgfqpoint{5.317041in}{2.024891in}}%
\pgfpathlineto{\pgfqpoint{5.325847in}{1.825873in}}%
\pgfpathlineto{\pgfqpoint{5.334654in}{1.688613in}}%
\pgfpathlineto{\pgfqpoint{5.343461in}{1.791551in}}%
\pgfpathlineto{\pgfqpoint{5.352268in}{1.867054in}}%
\pgfpathlineto{\pgfqpoint{5.361075in}{1.969992in}}%
\pgfpathlineto{\pgfqpoint{5.369882in}{2.086648in}}%
\pgfpathlineto{\pgfqpoint{5.387495in}{1.112138in}}%
\pgfpathlineto{\pgfqpoint{5.396302in}{1.571958in}}%
\pgfpathlineto{\pgfqpoint{5.405109in}{1.578816in}}%
\pgfpathlineto{\pgfqpoint{5.413916in}{1.770975in}}%
\pgfpathlineto{\pgfqpoint{5.422722in}{2.059213in}}%
\pgfpathlineto{\pgfqpoint{5.431529in}{2.169009in}}%
\pgfpathlineto{\pgfqpoint{5.440336in}{2.313128in}}%
\pgfpathlineto{\pgfqpoint{5.449143in}{2.169009in}}%
\pgfpathlineto{\pgfqpoint{5.457950in}{2.244484in}}%
\pgfpathlineto{\pgfqpoint{5.466757in}{2.072930in}}%
\pgfpathlineto{\pgfqpoint{5.475563in}{2.203303in}}%
\pgfpathlineto{\pgfqpoint{5.484370in}{1.935670in}}%
\pgfpathlineto{\pgfqpoint{5.493177in}{2.100365in}}%
\pgfpathlineto{\pgfqpoint{5.501984in}{2.313128in}}%
\pgfpathlineto{\pgfqpoint{5.510791in}{2.100365in}}%
\pgfpathlineto{\pgfqpoint{5.519597in}{2.278806in}}%
\pgfpathlineto{\pgfqpoint{5.528404in}{2.031749in}}%
\pgfpathlineto{\pgfqpoint{5.537211in}{2.059213in}}%
\pgfpathlineto{\pgfqpoint{5.546018in}{1.997427in}}%
\pgfpathlineto{\pgfqpoint{5.554825in}{1.517031in}}%
\pgfpathlineto{\pgfqpoint{5.563632in}{1.331760in}}%
\pgfpathlineto{\pgfqpoint{5.572438in}{1.331760in}}%
\pgfpathlineto{\pgfqpoint{5.581245in}{1.153319in}}%
\pgfpathlineto{\pgfqpoint{5.590052in}{1.386658in}}%
\pgfpathlineto{\pgfqpoint{5.598859in}{1.894489in}}%
\pgfpathlineto{\pgfqpoint{5.607666in}{1.894489in}}%
\pgfpathlineto{\pgfqpoint{5.616472in}{1.379799in}}%
\pgfpathlineto{\pgfqpoint{5.625279in}{1.510172in}}%
\pgfpathlineto{\pgfqpoint{5.634086in}{1.853308in}}%
\pgfpathlineto{\pgfqpoint{5.642893in}{1.819015in}}%
\pgfpathlineto{\pgfqpoint{5.651700in}{1.894489in}}%
\pgfpathlineto{\pgfqpoint{5.660507in}{1.873913in}}%
\pgfpathlineto{\pgfqpoint{5.669313in}{1.819015in}}%
\pgfpathlineto{\pgfqpoint{5.678120in}{1.983710in}}%
\pgfpathlineto{\pgfqpoint{5.686927in}{1.894489in}}%
\pgfpathlineto{\pgfqpoint{5.695734in}{1.791551in}}%
\pgfpathlineto{\pgfqpoint{5.704541in}{2.141546in}}%
\pgfpathlineto{\pgfqpoint{5.713347in}{1.894489in}}%
\pgfpathlineto{\pgfqpoint{5.722154in}{1.702331in}}%
\pgfpathlineto{\pgfqpoint{5.730961in}{1.455274in}}%
\pgfpathlineto{\pgfqpoint{5.739768in}{1.681754in}}%
\pgfpathlineto{\pgfqpoint{5.748575in}{1.832732in}}%
\pgfpathlineto{\pgfqpoint{5.757382in}{1.613110in}}%
\pgfpathlineto{\pgfqpoint{5.766188in}{1.352336in}}%
\pgfpathlineto{\pgfqpoint{5.774995in}{1.928811in}}%
\pgfpathlineto{\pgfqpoint{5.783802in}{2.244484in}}%
\pgfpathlineto{\pgfqpoint{5.792609in}{2.086648in}}%
\pgfpathlineto{\pgfqpoint{5.801416in}{2.175868in}}%
\pgfpathlineto{\pgfqpoint{5.810222in}{2.086648in}}%
\pgfpathlineto{\pgfqpoint{5.827836in}{1.825873in}}%
\pgfpathlineto{\pgfqpoint{5.836643in}{1.764116in}}%
\pgfpathlineto{\pgfqpoint{5.845450in}{1.887630in}}%
\pgfpathlineto{\pgfqpoint{5.854257in}{1.990569in}}%
\pgfpathlineto{\pgfqpoint{5.863063in}{1.798410in}}%
\pgfpathlineto{\pgfqpoint{5.871870in}{2.251343in}}%
\pgfpathlineto{\pgfqpoint{5.880677in}{2.265089in}}%
\pgfpathlineto{\pgfqpoint{5.889484in}{1.894489in}}%
\pgfpathlineto{\pgfqpoint{5.898291in}{2.120970in}}%
\pgfpathlineto{\pgfqpoint{5.907097in}{2.141546in}}%
\pgfpathlineto{\pgfqpoint{5.915904in}{2.402349in}}%
\pgfpathlineto{\pgfqpoint{5.924711in}{2.148405in}}%
\pgfpathlineto{\pgfqpoint{5.933518in}{2.182727in}}%
\pgfpathlineto{\pgfqpoint{5.942325in}{2.079789in}}%
\pgfpathlineto{\pgfqpoint{5.951132in}{2.223908in}}%
\pgfpathlineto{\pgfqpoint{5.968745in}{2.395462in}}%
\pgfpathlineto{\pgfqpoint{5.977552in}{2.319987in}}%
\pgfpathlineto{\pgfqpoint{5.986359in}{2.223908in}}%
\pgfpathlineto{\pgfqpoint{5.995166in}{1.949388in}}%
\pgfpathlineto{\pgfqpoint{6.003972in}{2.210190in}}%
\pgfpathlineto{\pgfqpoint{6.012779in}{2.018032in}}%
\pgfpathlineto{\pgfqpoint{6.021586in}{1.729794in}}%
\pgfpathlineto{\pgfqpoint{6.030393in}{1.928811in}}%
\pgfpathlineto{\pgfqpoint{6.039200in}{1.942529in}}%
\pgfpathlineto{\pgfqpoint{6.048007in}{2.018032in}}%
\pgfpathlineto{\pgfqpoint{6.056813in}{1.846450in}}%
\pgfpathlineto{\pgfqpoint{6.065620in}{2.120970in}}%
\pgfpathlineto{\pgfqpoint{6.074427in}{2.306269in}}%
\pgfpathlineto{\pgfqpoint{6.083234in}{2.237625in}}%
\pgfpathlineto{\pgfqpoint{6.092041in}{2.120970in}}%
\pgfpathlineto{\pgfqpoint{6.100847in}{2.306269in}}%
\pgfpathlineto{\pgfqpoint{6.109654in}{2.066071in}}%
\pgfpathlineto{\pgfqpoint{6.118461in}{1.921953in}}%
\pgfpathlineto{\pgfqpoint{6.127268in}{1.668037in}}%
\pgfpathlineto{\pgfqpoint{6.136075in}{1.894489in}}%
\pgfpathlineto{\pgfqpoint{6.144882in}{2.196445in}}%
\pgfpathlineto{\pgfqpoint{6.153688in}{1.983710in}}%
\pgfpathlineto{\pgfqpoint{6.162495in}{2.045467in}}%
\pgfpathlineto{\pgfqpoint{6.171302in}{1.976851in}}%
\pgfpathlineto{\pgfqpoint{6.180109in}{2.203303in}}%
\pgfpathlineto{\pgfqpoint{6.188916in}{2.230767in}}%
\pgfpathlineto{\pgfqpoint{6.197722in}{2.244484in}}%
\pgfpathlineto{\pgfqpoint{6.206529in}{2.011173in}}%
\pgfpathlineto{\pgfqpoint{6.215336in}{2.031749in}}%
\pgfpathlineto{\pgfqpoint{6.224143in}{1.976851in}}%
\pgfpathlineto{\pgfqpoint{6.241757in}{1.729794in}}%
\pgfpathlineto{\pgfqpoint{6.250563in}{1.983710in}}%
\pgfpathlineto{\pgfqpoint{6.259370in}{1.901348in}}%
\pgfpathlineto{\pgfqpoint{6.268177in}{1.976851in}}%
\pgfpathlineto{\pgfqpoint{6.276984in}{2.114111in}}%
\pgfpathlineto{\pgfqpoint{6.285791in}{1.805269in}}%
\pgfpathlineto{\pgfqpoint{6.294597in}{2.031749in}}%
\pgfpathlineto{\pgfqpoint{6.303404in}{2.066071in}}%
\pgfpathlineto{\pgfqpoint{6.312211in}{2.223908in}}%
\pgfpathlineto{\pgfqpoint{6.321018in}{1.983710in}}%
\pgfpathlineto{\pgfqpoint{6.329825in}{1.921953in}}%
\pgfpathlineto{\pgfqpoint{6.338632in}{2.052326in}}%
\pgfpathlineto{\pgfqpoint{6.347438in}{2.024891in}}%
\pgfpathlineto{\pgfqpoint{6.356245in}{1.956246in}}%
\pgfpathlineto{\pgfqpoint{6.373859in}{1.523918in}}%
\pgfpathlineto{\pgfqpoint{6.382666in}{1.702331in}}%
\pgfpathlineto{\pgfqpoint{6.391472in}{1.530777in}}%
\pgfpathlineto{\pgfqpoint{6.400279in}{1.716076in}}%
\pgfpathlineto{\pgfqpoint{6.409086in}{1.640574in}}%
\pgfpathlineto{\pgfqpoint{6.417893in}{1.633715in}}%
\pgfpathlineto{\pgfqpoint{6.426700in}{1.702331in}}%
\pgfpathlineto{\pgfqpoint{6.435507in}{1.695472in}}%
\pgfpathlineto{\pgfqpoint{6.444313in}{1.743512in}}%
\pgfpathlineto{\pgfqpoint{6.453120in}{1.661150in}}%
\pgfpathlineto{\pgfqpoint{6.461927in}{1.709190in}}%
\pgfpathlineto{\pgfqpoint{6.470734in}{1.640574in}}%
\pgfpathlineto{\pgfqpoint{6.479541in}{2.169009in}}%
\pgfpathlineto{\pgfqpoint{6.488347in}{2.045467in}}%
\pgfpathlineto{\pgfqpoint{6.497154in}{2.169009in}}%
\pgfpathlineto{\pgfqpoint{6.505961in}{2.422925in}}%
\pgfpathlineto{\pgfqpoint{6.514768in}{2.148405in}}%
\pgfpathlineto{\pgfqpoint{6.523575in}{1.812128in}}%
\pgfpathlineto{\pgfqpoint{6.532382in}{1.812128in}}%
\pgfpathlineto{\pgfqpoint{6.541188in}{1.846450in}}%
\pgfpathlineto{\pgfqpoint{6.549995in}{1.736653in}}%
\pgfpathlineto{\pgfqpoint{6.558802in}{2.100365in}}%
\pgfpathlineto{\pgfqpoint{6.567609in}{1.688613in}}%
\pgfpathlineto{\pgfqpoint{6.594029in}{2.340563in}}%
\pgfpathlineto{\pgfqpoint{6.602836in}{2.011173in}}%
\pgfpathlineto{\pgfqpoint{6.611643in}{1.812128in}}%
\pgfpathlineto{\pgfqpoint{6.620450in}{1.716076in}}%
\pgfpathlineto{\pgfqpoint{6.629257in}{1.729794in}}%
\pgfpathlineto{\pgfqpoint{6.638063in}{2.258230in}}%
\pgfpathlineto{\pgfqpoint{6.646870in}{2.354309in}}%
\pgfpathlineto{\pgfqpoint{6.655677in}{2.107252in}}%
\pgfpathlineto{\pgfqpoint{6.664484in}{2.196445in}}%
\pgfpathlineto{\pgfqpoint{6.673291in}{1.853308in}}%
\pgfpathlineto{\pgfqpoint{6.682097in}{1.736653in}}%
\pgfpathlineto{\pgfqpoint{6.690904in}{1.963133in}}%
\pgfpathlineto{\pgfqpoint{6.699711in}{1.949388in}}%
\pgfpathlineto{\pgfqpoint{6.708518in}{2.011173in}}%
\pgfpathlineto{\pgfqpoint{6.726132in}{2.388603in}}%
\pgfpathlineto{\pgfqpoint{6.734938in}{2.223908in}}%
\pgfpathlineto{\pgfqpoint{6.743745in}{2.464106in}}%
\pgfpathlineto{\pgfqpoint{6.752552in}{2.326846in}}%
\pgfpathlineto{\pgfqpoint{6.761359in}{2.031749in}}%
\pgfpathlineto{\pgfqpoint{6.770166in}{2.024891in}}%
\pgfpathlineto{\pgfqpoint{6.778972in}{1.969992in}}%
\pgfpathlineto{\pgfqpoint{6.787779in}{2.278806in}}%
\pgfpathlineto{\pgfqpoint{6.796586in}{2.443501in}}%
\pgfpathlineto{\pgfqpoint{6.805393in}{2.271947in}}%
\pgfpathlineto{\pgfqpoint{6.814200in}{1.963133in}}%
\pgfpathlineto{\pgfqpoint{6.823007in}{1.750370in}}%
\pgfpathlineto{\pgfqpoint{6.831813in}{2.086648in}}%
\pgfpathlineto{\pgfqpoint{6.840620in}{2.306269in}}%
\pgfpathlineto{\pgfqpoint{6.849427in}{2.251343in}}%
\pgfpathlineto{\pgfqpoint{6.858234in}{2.464106in}}%
\pgfpathlineto{\pgfqpoint{6.867041in}{2.093507in}}%
\pgfpathlineto{\pgfqpoint{6.875847in}{2.196445in}}%
\pgfpathlineto{\pgfqpoint{6.884654in}{2.072930in}}%
\pgfpathlineto{\pgfqpoint{6.893461in}{2.127829in}}%
\pgfpathlineto{\pgfqpoint{6.902268in}{1.887630in}}%
\pgfpathlineto{\pgfqpoint{6.911075in}{1.530777in}}%
\pgfpathlineto{\pgfqpoint{6.919882in}{1.633715in}}%
\pgfpathlineto{\pgfqpoint{6.928688in}{1.503314in}}%
\pgfpathlineto{\pgfqpoint{6.937495in}{1.565071in}}%
\pgfpathlineto{\pgfqpoint{6.946302in}{1.517031in}}%
\pgfpathlineto{\pgfqpoint{6.955109in}{1.668037in}}%
\pgfpathlineto{\pgfqpoint{6.963916in}{1.791551in}}%
\pgfpathlineto{\pgfqpoint{6.972722in}{2.244484in}}%
\pgfpathlineto{\pgfqpoint{6.981529in}{2.244484in}}%
\pgfpathlineto{\pgfqpoint{6.990336in}{2.189586in}}%
\pgfpathlineto{\pgfqpoint{7.007950in}{1.571958in}}%
\pgfpathlineto{\pgfqpoint{7.016757in}{1.777834in}}%
\pgfpathlineto{\pgfqpoint{7.025563in}{1.709190in}}%
\pgfpathlineto{\pgfqpoint{7.034370in}{1.695472in}}%
\pgfpathlineto{\pgfqpoint{7.043177in}{1.784692in}}%
\pgfpathlineto{\pgfqpoint{7.060791in}{2.114111in}}%
\pgfpathlineto{\pgfqpoint{7.069597in}{2.134687in}}%
\pgfpathlineto{\pgfqpoint{7.078404in}{2.319987in}}%
\pgfpathlineto{\pgfqpoint{7.087211in}{2.416066in}}%
\pgfpathlineto{\pgfqpoint{7.096018in}{2.402349in}}%
\pgfpathlineto{\pgfqpoint{7.104825in}{2.443501in}}%
\pgfpathlineto{\pgfqpoint{7.113632in}{2.388603in}}%
\pgfpathlineto{\pgfqpoint{7.122438in}{2.347422in}}%
\pgfpathlineto{\pgfqpoint{7.131245in}{2.120970in}}%
\pgfpathlineto{\pgfqpoint{7.140052in}{2.018032in}}%
\pgfpathlineto{\pgfqpoint{7.148859in}{1.873913in}}%
\pgfpathlineto{\pgfqpoint{7.157666in}{2.285665in}}%
\pgfpathlineto{\pgfqpoint{7.166472in}{2.223908in}}%
\pgfpathlineto{\pgfqpoint{7.175279in}{1.949388in}}%
\pgfpathlineto{\pgfqpoint{7.184086in}{1.420952in}}%
\pgfpathlineto{\pgfqpoint{7.192893in}{1.599393in}}%
\pgfpathlineto{\pgfqpoint{7.201700in}{1.441556in}}%
\pgfpathlineto{\pgfqpoint{7.210507in}{1.674896in}}%
\pgfpathlineto{\pgfqpoint{7.219313in}{1.695472in}}%
\pgfpathlineto{\pgfqpoint{7.228120in}{1.468992in}}%
\pgfpathlineto{\pgfqpoint{7.236927in}{1.626856in}}%
\pgfpathlineto{\pgfqpoint{7.245734in}{1.468992in}}%
\pgfpathlineto{\pgfqpoint{7.254541in}{1.592534in}}%
\pgfpathlineto{\pgfqpoint{7.263347in}{1.661150in}}%
\pgfpathlineto{\pgfqpoint{7.272154in}{2.086648in}}%
\pgfpathlineto{\pgfqpoint{7.280961in}{2.134687in}}%
\pgfpathlineto{\pgfqpoint{7.289768in}{1.921953in}}%
\pgfpathlineto{\pgfqpoint{7.298575in}{1.764116in}}%
\pgfpathlineto{\pgfqpoint{7.307382in}{1.873913in}}%
\pgfpathlineto{\pgfqpoint{7.316188in}{1.805269in}}%
\pgfpathlineto{\pgfqpoint{7.324995in}{1.832732in}}%
\pgfpathlineto{\pgfqpoint{7.333802in}{1.853308in}}%
\pgfpathlineto{\pgfqpoint{7.342609in}{1.860167in}}%
\pgfpathlineto{\pgfqpoint{7.351416in}{1.839591in}}%
\pgfpathlineto{\pgfqpoint{7.360222in}{1.537636in}}%
\pgfpathlineto{\pgfqpoint{7.369029in}{1.956246in}}%
\pgfpathlineto{\pgfqpoint{7.377836in}{2.162151in}}%
\pgfpathlineto{\pgfqpoint{7.386643in}{2.045467in}}%
\pgfpathlineto{\pgfqpoint{7.395450in}{2.217049in}}%
\pgfpathlineto{\pgfqpoint{7.404257in}{2.004286in}}%
\pgfpathlineto{\pgfqpoint{7.413063in}{2.127829in}}%
\pgfpathlineto{\pgfqpoint{7.421870in}{1.901348in}}%
\pgfpathlineto{\pgfqpoint{7.430677in}{1.551353in}}%
\pgfpathlineto{\pgfqpoint{7.439484in}{1.757229in}}%
\pgfpathlineto{\pgfqpoint{7.448291in}{1.414093in}}%
\pgfpathlineto{\pgfqpoint{7.457097in}{1.585675in}}%
\pgfpathlineto{\pgfqpoint{7.465904in}{1.695472in}}%
\pgfpathlineto{\pgfqpoint{7.474711in}{1.674896in}}%
\pgfpathlineto{\pgfqpoint{7.483518in}{1.846450in}}%
\pgfpathlineto{\pgfqpoint{7.492325in}{1.963133in}}%
\pgfpathlineto{\pgfqpoint{7.501132in}{1.873913in}}%
\pgfpathlineto{\pgfqpoint{7.509938in}{1.661150in}}%
\pgfpathlineto{\pgfqpoint{7.518745in}{1.770975in}}%
\pgfpathlineto{\pgfqpoint{7.527552in}{1.764116in}}%
\pgfpathlineto{\pgfqpoint{7.536359in}{1.853308in}}%
\pgfpathlineto{\pgfqpoint{7.545166in}{2.004286in}}%
\pgfpathlineto{\pgfqpoint{7.553972in}{2.059213in}}%
\pgfpathlineto{\pgfqpoint{7.562779in}{1.949388in}}%
\pgfpathlineto{\pgfqpoint{7.571586in}{1.784692in}}%
\pgfpathlineto{\pgfqpoint{7.580393in}{1.997427in}}%
\pgfpathlineto{\pgfqpoint{7.589200in}{2.265089in}}%
\pgfpathlineto{\pgfqpoint{7.598007in}{2.162151in}}%
\pgfpathlineto{\pgfqpoint{7.606813in}{2.114111in}}%
\pgfpathlineto{\pgfqpoint{7.615620in}{2.169009in}}%
\pgfpathlineto{\pgfqpoint{7.624427in}{1.908207in}}%
\pgfpathlineto{\pgfqpoint{7.633234in}{1.695472in}}%
\pgfpathlineto{\pgfqpoint{7.642041in}{2.024891in}}%
\pgfpathlineto{\pgfqpoint{7.650847in}{2.107252in}}%
\pgfpathlineto{\pgfqpoint{7.659654in}{1.668037in}}%
\pgfpathlineto{\pgfqpoint{7.668461in}{1.784692in}}%
\pgfpathlineto{\pgfqpoint{7.677268in}{1.928811in}}%
\pgfpathlineto{\pgfqpoint{7.686075in}{2.107252in}}%
\pgfpathlineto{\pgfqpoint{7.694882in}{1.915094in}}%
\pgfpathlineto{\pgfqpoint{7.703688in}{1.777834in}}%
\pgfpathlineto{\pgfqpoint{7.712495in}{1.969992in}}%
\pgfpathlineto{\pgfqpoint{7.721302in}{1.942529in}}%
\pgfpathlineto{\pgfqpoint{7.730109in}{1.942529in}}%
\pgfpathlineto{\pgfqpoint{7.738916in}{1.887630in}}%
\pgfpathlineto{\pgfqpoint{7.747722in}{1.894489in}}%
\pgfpathlineto{\pgfqpoint{7.756529in}{1.750370in}}%
\pgfpathlineto{\pgfqpoint{7.765336in}{1.990569in}}%
\pgfpathlineto{\pgfqpoint{7.774143in}{1.846450in}}%
\pgfpathlineto{\pgfqpoint{7.782950in}{2.114111in}}%
\pgfpathlineto{\pgfqpoint{7.791757in}{2.258230in}}%
\pgfpathlineto{\pgfqpoint{7.800563in}{2.052326in}}%
\pgfpathlineto{\pgfqpoint{7.809370in}{2.038608in}}%
\pgfpathlineto{\pgfqpoint{7.818177in}{1.812128in}}%
\pgfpathlineto{\pgfqpoint{7.826984in}{1.736653in}}%
\pgfpathlineto{\pgfqpoint{7.835791in}{1.770975in}}%
\pgfpathlineto{\pgfqpoint{7.844597in}{1.304296in}}%
\pgfpathlineto{\pgfqpoint{7.853404in}{1.324873in}}%
\pgfpathlineto{\pgfqpoint{7.862211in}{1.201358in}}%
\pgfpathlineto{\pgfqpoint{7.871018in}{1.393517in}}%
\pgfpathlineto{\pgfqpoint{7.879825in}{1.613110in}}%
\pgfpathlineto{\pgfqpoint{7.888632in}{1.935670in}}%
\pgfpathlineto{\pgfqpoint{7.897438in}{1.702331in}}%
\pgfpathlineto{\pgfqpoint{7.906245in}{1.819015in}}%
\pgfpathlineto{\pgfqpoint{7.915052in}{2.011173in}}%
\pgfpathlineto{\pgfqpoint{7.923859in}{2.258230in}}%
\pgfpathlineto{\pgfqpoint{7.941472in}{1.585675in}}%
\pgfpathlineto{\pgfqpoint{7.950279in}{1.338618in}}%
\pgfpathlineto{\pgfqpoint{7.959086in}{1.757229in}}%
\pgfpathlineto{\pgfqpoint{7.967893in}{1.688613in}}%
\pgfpathlineto{\pgfqpoint{7.976700in}{1.770975in}}%
\pgfpathlineto{\pgfqpoint{7.985507in}{2.148405in}}%
\pgfpathlineto{\pgfqpoint{7.994313in}{2.004286in}}%
\pgfpathlineto{\pgfqpoint{8.003120in}{1.894489in}}%
\pgfpathlineto{\pgfqpoint{8.011927in}{1.908207in}}%
\pgfpathlineto{\pgfqpoint{8.020734in}{1.722935in}}%
\pgfpathlineto{\pgfqpoint{8.029541in}{1.764116in}}%
\pgfpathlineto{\pgfqpoint{8.038347in}{1.729794in}}%
\pgfpathlineto{\pgfqpoint{8.047154in}{1.777834in}}%
\pgfpathlineto{\pgfqpoint{8.055961in}{1.633715in}}%
\pgfpathlineto{\pgfqpoint{8.064768in}{1.860167in}}%
\pgfpathlineto{\pgfqpoint{8.082382in}{2.114111in}}%
\pgfpathlineto{\pgfqpoint{8.091188in}{2.107252in}}%
\pgfpathlineto{\pgfqpoint{8.099995in}{2.162151in}}%
\pgfpathlineto{\pgfqpoint{8.108802in}{2.127829in}}%
\pgfpathlineto{\pgfqpoint{8.117609in}{2.175868in}}%
\pgfpathlineto{\pgfqpoint{8.126416in}{2.059213in}}%
\pgfpathlineto{\pgfqpoint{8.135222in}{1.997427in}}%
\pgfpathlineto{\pgfqpoint{8.144029in}{1.853308in}}%
\pgfpathlineto{\pgfqpoint{8.152836in}{2.052326in}}%
\pgfpathlineto{\pgfqpoint{8.161643in}{2.100365in}}%
\pgfpathlineto{\pgfqpoint{8.170450in}{2.230767in}}%
\pgfpathlineto{\pgfqpoint{8.179257in}{2.052326in}}%
\pgfpathlineto{\pgfqpoint{8.188063in}{2.011173in}}%
\pgfpathlineto{\pgfqpoint{8.196870in}{2.148405in}}%
\pgfpathlineto{\pgfqpoint{8.205677in}{1.695472in}}%
\pgfpathlineto{\pgfqpoint{8.214484in}{1.743512in}}%
\pgfpathlineto{\pgfqpoint{8.223291in}{1.606252in}}%
\pgfpathlineto{\pgfqpoint{8.232097in}{1.722935in}}%
\pgfpathlineto{\pgfqpoint{8.240904in}{1.359195in}}%
\pgfpathlineto{\pgfqpoint{8.249711in}{1.420952in}}%
\pgfpathlineto{\pgfqpoint{8.258518in}{1.983710in}}%
\pgfpathlineto{\pgfqpoint{8.267325in}{1.915094in}}%
\pgfpathlineto{\pgfqpoint{8.276132in}{2.354309in}}%
\pgfpathlineto{\pgfqpoint{8.284938in}{2.546468in}}%
\pgfpathlineto{\pgfqpoint{8.293745in}{2.820960in}}%
\pgfpathlineto{\pgfqpoint{8.302552in}{2.827818in}}%
\pgfpathlineto{\pgfqpoint{8.320166in}{2.011173in}}%
\pgfpathlineto{\pgfqpoint{8.328972in}{1.990569in}}%
\pgfpathlineto{\pgfqpoint{8.337779in}{2.292524in}}%
\pgfpathlineto{\pgfqpoint{8.346586in}{2.196445in}}%
\pgfpathlineto{\pgfqpoint{8.355393in}{1.764116in}}%
\pgfpathlineto{\pgfqpoint{8.364200in}{1.702331in}}%
\pgfpathlineto{\pgfqpoint{8.373007in}{1.517031in}}%
\pgfpathlineto{\pgfqpoint{8.381813in}{1.558212in}}%
\pgfpathlineto{\pgfqpoint{8.390620in}{1.530777in}}%
\pgfpathlineto{\pgfqpoint{8.399427in}{1.606252in}}%
\pgfpathlineto{\pgfqpoint{8.408234in}{1.915094in}}%
\pgfpathlineto{\pgfqpoint{8.417041in}{2.066071in}}%
\pgfpathlineto{\pgfqpoint{8.425847in}{2.464106in}}%
\pgfpathlineto{\pgfqpoint{8.434654in}{2.004286in}}%
\pgfpathlineto{\pgfqpoint{8.443461in}{2.237625in}}%
\pgfpathlineto{\pgfqpoint{8.452268in}{2.210190in}}%
\pgfpathlineto{\pgfqpoint{8.461075in}{2.395462in}}%
\pgfpathlineto{\pgfqpoint{8.469882in}{2.313128in}}%
\pgfpathlineto{\pgfqpoint{8.478688in}{2.608225in}}%
\pgfpathlineto{\pgfqpoint{8.487495in}{2.484682in}}%
\pgfpathlineto{\pgfqpoint{8.496302in}{2.690558in}}%
\pgfpathlineto{\pgfqpoint{8.505109in}{2.381744in}}%
\pgfpathlineto{\pgfqpoint{8.513916in}{2.621942in}}%
\pgfpathlineto{\pgfqpoint{8.522722in}{2.512146in}}%
\pgfpathlineto{\pgfqpoint{8.531529in}{2.203303in}}%
\pgfpathlineto{\pgfqpoint{8.540336in}{2.388603in}}%
\pgfpathlineto{\pgfqpoint{8.549143in}{2.038608in}}%
\pgfpathlineto{\pgfqpoint{8.557950in}{2.155292in}}%
\pgfpathlineto{\pgfqpoint{8.575563in}{2.512146in}}%
\pgfpathlineto{\pgfqpoint{8.584370in}{2.141546in}}%
\pgfpathlineto{\pgfqpoint{8.593177in}{1.956246in}}%
\pgfpathlineto{\pgfqpoint{8.601984in}{2.045467in}}%
\pgfpathlineto{\pgfqpoint{8.610791in}{1.983710in}}%
\pgfpathlineto{\pgfqpoint{8.619597in}{2.079789in}}%
\pgfpathlineto{\pgfqpoint{8.628404in}{2.059213in}}%
\pgfpathlineto{\pgfqpoint{8.637211in}{2.072930in}}%
\pgfpathlineto{\pgfqpoint{8.646018in}{1.722935in}}%
\pgfpathlineto{\pgfqpoint{8.654825in}{1.434698in}}%
\pgfpathlineto{\pgfqpoint{8.663632in}{1.517031in}}%
\pgfpathlineto{\pgfqpoint{8.672438in}{1.709190in}}%
\pgfpathlineto{\pgfqpoint{8.681245in}{2.004286in}}%
\pgfpathlineto{\pgfqpoint{8.690052in}{1.585675in}}%
\pgfpathlineto{\pgfqpoint{8.698859in}{1.475878in}}%
\pgfpathlineto{\pgfqpoint{8.707666in}{1.729794in}}%
\pgfpathlineto{\pgfqpoint{8.716472in}{1.585675in}}%
\pgfpathlineto{\pgfqpoint{8.725279in}{1.613110in}}%
\pgfpathlineto{\pgfqpoint{8.734086in}{1.791551in}}%
\pgfpathlineto{\pgfqpoint{8.742893in}{1.846450in}}%
\pgfpathlineto{\pgfqpoint{8.751700in}{1.750370in}}%
\pgfpathlineto{\pgfqpoint{8.760507in}{2.024891in}}%
\pgfpathlineto{\pgfqpoint{8.769313in}{2.340563in}}%
\pgfpathlineto{\pgfqpoint{8.778120in}{2.292524in}}%
\pgfpathlineto{\pgfqpoint{8.786927in}{2.567044in}}%
\pgfpathlineto{\pgfqpoint{8.795734in}{2.409207in}}%
\pgfpathlineto{\pgfqpoint{8.804541in}{2.409207in}}%
\pgfpathlineto{\pgfqpoint{8.813347in}{2.539581in}}%
\pgfpathlineto{\pgfqpoint{8.822154in}{2.141546in}}%
\pgfpathlineto{\pgfqpoint{8.830961in}{1.894489in}}%
\pgfpathlineto{\pgfqpoint{8.839768in}{2.045467in}}%
\pgfpathlineto{\pgfqpoint{8.848575in}{1.963133in}}%
\pgfpathlineto{\pgfqpoint{8.857382in}{1.770975in}}%
\pgfpathlineto{\pgfqpoint{8.866188in}{1.921953in}}%
\pgfpathlineto{\pgfqpoint{8.874995in}{2.011173in}}%
\pgfpathlineto{\pgfqpoint{8.883802in}{1.901348in}}%
\pgfpathlineto{\pgfqpoint{8.892609in}{1.867054in}}%
\pgfpathlineto{\pgfqpoint{8.901416in}{1.736653in}}%
\pgfpathlineto{\pgfqpoint{8.910222in}{2.038608in}}%
\pgfpathlineto{\pgfqpoint{8.919029in}{2.093507in}}%
\pgfpathlineto{\pgfqpoint{8.927836in}{1.969992in}}%
\pgfpathlineto{\pgfqpoint{8.945450in}{2.079789in}}%
\pgfpathlineto{\pgfqpoint{8.954257in}{1.969992in}}%
\pgfpathlineto{\pgfqpoint{8.963063in}{1.777834in}}%
\pgfpathlineto{\pgfqpoint{8.971870in}{2.100365in}}%
\pgfpathlineto{\pgfqpoint{8.989484in}{1.551353in}}%
\pgfpathlineto{\pgfqpoint{8.998291in}{1.510172in}}%
\pgfpathlineto{\pgfqpoint{9.007097in}{1.935670in}}%
\pgfpathlineto{\pgfqpoint{9.015904in}{1.819015in}}%
\pgfpathlineto{\pgfqpoint{9.024711in}{1.770975in}}%
\pgfpathlineto{\pgfqpoint{9.033518in}{1.983710in}}%
\pgfpathlineto{\pgfqpoint{9.042325in}{1.743512in}}%
\pgfpathlineto{\pgfqpoint{9.051132in}{1.640574in}}%
\pgfpathlineto{\pgfqpoint{9.059938in}{1.558212in}}%
\pgfpathlineto{\pgfqpoint{9.068745in}{1.928811in}}%
\pgfpathlineto{\pgfqpoint{9.077552in}{1.880772in}}%
\pgfpathlineto{\pgfqpoint{9.086359in}{2.230767in}}%
\pgfpathlineto{\pgfqpoint{9.095166in}{2.319987in}}%
\pgfpathlineto{\pgfqpoint{9.103972in}{2.594507in}}%
\pgfpathlineto{\pgfqpoint{9.112779in}{2.271947in}}%
\pgfpathlineto{\pgfqpoint{9.121586in}{2.265089in}}%
\pgfpathlineto{\pgfqpoint{9.130393in}{1.928811in}}%
\pgfpathlineto{\pgfqpoint{9.139200in}{1.928811in}}%
\pgfpathlineto{\pgfqpoint{9.148007in}{1.867054in}}%
\pgfpathlineto{\pgfqpoint{9.156813in}{1.743512in}}%
\pgfpathlineto{\pgfqpoint{9.165620in}{1.942529in}}%
\pgfpathlineto{\pgfqpoint{9.174427in}{1.928811in}}%
\pgfpathlineto{\pgfqpoint{9.183234in}{2.024891in}}%
\pgfpathlineto{\pgfqpoint{9.192041in}{2.306269in}}%
\pgfpathlineto{\pgfqpoint{9.200847in}{1.860167in}}%
\pgfpathlineto{\pgfqpoint{9.209654in}{1.599393in}}%
\pgfpathlineto{\pgfqpoint{9.218461in}{1.420952in}}%
\pgfpathlineto{\pgfqpoint{9.227268in}{1.441556in}}%
\pgfpathlineto{\pgfqpoint{9.253688in}{2.189586in}}%
\pgfpathlineto{\pgfqpoint{9.262495in}{1.674896in}}%
\pgfpathlineto{\pgfqpoint{9.271302in}{1.928811in}}%
\pgfpathlineto{\pgfqpoint{9.280109in}{1.860167in}}%
\pgfpathlineto{\pgfqpoint{9.288916in}{1.757229in}}%
\pgfpathlineto{\pgfqpoint{9.297722in}{2.059213in}}%
\pgfpathlineto{\pgfqpoint{9.306529in}{1.942529in}}%
\pgfpathlineto{\pgfqpoint{9.315336in}{1.716076in}}%
\pgfpathlineto{\pgfqpoint{9.324143in}{1.668037in}}%
\pgfpathlineto{\pgfqpoint{9.341757in}{1.276861in}}%
\pgfpathlineto{\pgfqpoint{9.350563in}{1.414093in}}%
\pgfpathlineto{\pgfqpoint{9.368177in}{1.791551in}}%
\pgfpathlineto{\pgfqpoint{9.376984in}{1.784692in}}%
\pgfpathlineto{\pgfqpoint{9.385791in}{1.626856in}}%
\pgfpathlineto{\pgfqpoint{9.394597in}{1.668037in}}%
\pgfpathlineto{\pgfqpoint{9.403404in}{1.263115in}}%
\pgfpathlineto{\pgfqpoint{9.412211in}{1.215076in}}%
\pgfpathlineto{\pgfqpoint{9.421018in}{1.263115in}}%
\pgfpathlineto{\pgfqpoint{9.429825in}{1.167036in}}%
\pgfpathlineto{\pgfqpoint{9.438632in}{1.455274in}}%
\pgfpathlineto{\pgfqpoint{9.447438in}{1.276861in}}%
\pgfpathlineto{\pgfqpoint{9.456245in}{1.311155in}}%
\pgfpathlineto{\pgfqpoint{9.465052in}{1.674896in}}%
\pgfpathlineto{\pgfqpoint{9.473859in}{1.819015in}}%
\pgfpathlineto{\pgfqpoint{9.482666in}{1.901348in}}%
\pgfpathlineto{\pgfqpoint{9.491472in}{2.107252in}}%
\pgfpathlineto{\pgfqpoint{9.500279in}{1.860167in}}%
\pgfpathlineto{\pgfqpoint{9.509086in}{1.839591in}}%
\pgfpathlineto{\pgfqpoint{9.517893in}{1.894489in}}%
\pgfpathlineto{\pgfqpoint{9.526700in}{1.750370in}}%
\pgfpathlineto{\pgfqpoint{9.535507in}{1.832732in}}%
\pgfpathlineto{\pgfqpoint{9.544313in}{1.894489in}}%
\pgfpathlineto{\pgfqpoint{9.553120in}{2.038608in}}%
\pgfpathlineto{\pgfqpoint{9.561927in}{1.791551in}}%
\pgfpathlineto{\pgfqpoint{9.570734in}{1.448415in}}%
\pgfpathlineto{\pgfqpoint{9.579541in}{1.414093in}}%
\pgfpathlineto{\pgfqpoint{9.588347in}{1.750370in}}%
\pgfpathlineto{\pgfqpoint{9.597154in}{1.750370in}}%
\pgfpathlineto{\pgfqpoint{9.605961in}{1.935670in}}%
\pgfpathlineto{\pgfqpoint{9.614768in}{2.093507in}}%
\pgfpathlineto{\pgfqpoint{9.623575in}{2.175868in}}%
\pgfpathlineto{\pgfqpoint{9.632382in}{1.832732in}}%
\pgfpathlineto{\pgfqpoint{9.641188in}{1.674896in}}%
\pgfpathlineto{\pgfqpoint{9.649995in}{1.654291in}}%
\pgfpathlineto{\pgfqpoint{9.658802in}{1.736653in}}%
\pgfpathlineto{\pgfqpoint{9.667609in}{1.846450in}}%
\pgfpathlineto{\pgfqpoint{9.685222in}{1.990569in}}%
\pgfpathlineto{\pgfqpoint{9.694029in}{1.688613in}}%
\pgfpathlineto{\pgfqpoint{9.702836in}{1.819015in}}%
\pgfpathlineto{\pgfqpoint{9.711643in}{1.819015in}}%
\pgfpathlineto{\pgfqpoint{9.720450in}{1.633715in}}%
\pgfpathlineto{\pgfqpoint{9.729257in}{2.066071in}}%
\pgfpathlineto{\pgfqpoint{9.738063in}{1.825873in}}%
\pgfpathlineto{\pgfqpoint{9.746870in}{1.832732in}}%
\pgfpathlineto{\pgfqpoint{9.755677in}{1.812128in}}%
\pgfpathlineto{\pgfqpoint{9.764484in}{1.949388in}}%
\pgfpathlineto{\pgfqpoint{9.773291in}{1.599393in}}%
\pgfpathlineto{\pgfqpoint{9.782097in}{1.990569in}}%
\pgfpathlineto{\pgfqpoint{9.790904in}{1.928811in}}%
\pgfpathlineto{\pgfqpoint{9.799711in}{1.942529in}}%
\pgfpathlineto{\pgfqpoint{9.808518in}{1.873913in}}%
\pgfpathlineto{\pgfqpoint{9.817325in}{1.880772in}}%
\pgfpathlineto{\pgfqpoint{9.826132in}{1.832732in}}%
\pgfpathlineto{\pgfqpoint{9.834938in}{2.079789in}}%
\pgfpathlineto{\pgfqpoint{9.843745in}{2.107252in}}%
\pgfpathlineto{\pgfqpoint{9.852552in}{2.066071in}}%
\pgfpathlineto{\pgfqpoint{9.861359in}{1.949388in}}%
\pgfpathlineto{\pgfqpoint{9.870166in}{1.757229in}}%
\pgfpathlineto{\pgfqpoint{9.878972in}{1.736653in}}%
\pgfpathlineto{\pgfqpoint{9.887779in}{1.901348in}}%
\pgfpathlineto{\pgfqpoint{9.905393in}{1.702331in}}%
\pgfpathlineto{\pgfqpoint{9.914200in}{1.867054in}}%
\pgfpathlineto{\pgfqpoint{9.923007in}{1.887630in}}%
\pgfpathlineto{\pgfqpoint{9.931813in}{1.997427in}}%
\pgfpathlineto{\pgfqpoint{9.940620in}{1.969992in}}%
\pgfpathlineto{\pgfqpoint{9.949427in}{1.839591in}}%
\pgfpathlineto{\pgfqpoint{9.949427in}{1.839591in}}%
\pgfusepath{stroke}%
\end{pgfscope}%
\begin{pgfscope}%
\pgfpathrectangle{\pgfqpoint{0.702268in}{0.521603in}}{\pgfqpoint{9.687500in}{4.235000in}}%
\pgfusepath{clip}%
\pgfsetrectcap%
\pgfsetroundjoin%
\pgfsetlinewidth{0.501875pt}%
\definecolor{currentstroke}{rgb}{0.501961,0.501961,0.501961}%
\pgfsetstrokecolor{currentstroke}%
\pgfsetstrokeopacity{0.250000}%
\pgfsetdash{}{0pt}%
\pgfpathmoveto{\pgfqpoint{1.142609in}{4.310188in}}%
\pgfpathlineto{\pgfqpoint{1.151416in}{3.843509in}}%
\pgfpathlineto{\pgfqpoint{1.160222in}{3.006259in}}%
\pgfpathlineto{\pgfqpoint{1.169029in}{2.635660in}}%
\pgfpathlineto{\pgfqpoint{1.177836in}{2.580761in}}%
\pgfpathlineto{\pgfqpoint{1.186643in}{2.285665in}}%
\pgfpathlineto{\pgfqpoint{1.195450in}{2.395462in}}%
\pgfpathlineto{\pgfqpoint{1.204257in}{2.347422in}}%
\pgfpathlineto{\pgfqpoint{1.213063in}{2.100365in}}%
\pgfpathlineto{\pgfqpoint{1.221870in}{1.777834in}}%
\pgfpathlineto{\pgfqpoint{1.230677in}{1.592534in}}%
\pgfpathlineto{\pgfqpoint{1.239484in}{1.819015in}}%
\pgfpathlineto{\pgfqpoint{1.248291in}{1.647432in}}%
\pgfpathlineto{\pgfqpoint{1.257097in}{1.956246in}}%
\pgfpathlineto{\pgfqpoint{1.265904in}{2.155292in}}%
\pgfpathlineto{\pgfqpoint{1.274711in}{1.997427in}}%
\pgfpathlineto{\pgfqpoint{1.283518in}{1.997427in}}%
\pgfpathlineto{\pgfqpoint{1.292325in}{2.217049in}}%
\pgfpathlineto{\pgfqpoint{1.301132in}{1.935670in}}%
\pgfpathlineto{\pgfqpoint{1.309938in}{2.066071in}}%
\pgfpathlineto{\pgfqpoint{1.318745in}{2.011173in}}%
\pgfpathlineto{\pgfqpoint{1.327552in}{2.443501in}}%
\pgfpathlineto{\pgfqpoint{1.336359in}{2.258230in}}%
\pgfpathlineto{\pgfqpoint{1.345166in}{1.887630in}}%
\pgfpathlineto{\pgfqpoint{1.362779in}{1.523918in}}%
\pgfpathlineto{\pgfqpoint{1.371586in}{1.565071in}}%
\pgfpathlineto{\pgfqpoint{1.380393in}{1.832732in}}%
\pgfpathlineto{\pgfqpoint{1.389200in}{2.265089in}}%
\pgfpathlineto{\pgfqpoint{1.398007in}{2.011173in}}%
\pgfpathlineto{\pgfqpoint{1.406813in}{1.976851in}}%
\pgfpathlineto{\pgfqpoint{1.415620in}{2.086648in}}%
\pgfpathlineto{\pgfqpoint{1.424427in}{1.585675in}}%
\pgfpathlineto{\pgfqpoint{1.433234in}{1.702331in}}%
\pgfpathlineto{\pgfqpoint{1.442041in}{1.969992in}}%
\pgfpathlineto{\pgfqpoint{1.450847in}{2.011173in}}%
\pgfpathlineto{\pgfqpoint{1.459654in}{1.928811in}}%
\pgfpathlineto{\pgfqpoint{1.468461in}{1.791551in}}%
\pgfpathlineto{\pgfqpoint{1.477268in}{1.908207in}}%
\pgfpathlineto{\pgfqpoint{1.486075in}{1.695472in}}%
\pgfpathlineto{\pgfqpoint{1.494882in}{1.935670in}}%
\pgfpathlineto{\pgfqpoint{1.503688in}{1.997427in}}%
\pgfpathlineto{\pgfqpoint{1.512495in}{2.368027in}}%
\pgfpathlineto{\pgfqpoint{1.521302in}{2.155292in}}%
\pgfpathlineto{\pgfqpoint{1.530109in}{2.258230in}}%
\pgfpathlineto{\pgfqpoint{1.538916in}{2.148405in}}%
\pgfpathlineto{\pgfqpoint{1.547722in}{1.963133in}}%
\pgfpathlineto{\pgfqpoint{1.565336in}{1.702331in}}%
\pgfpathlineto{\pgfqpoint{1.574143in}{1.647432in}}%
\pgfpathlineto{\pgfqpoint{1.582950in}{1.647432in}}%
\pgfpathlineto{\pgfqpoint{1.591757in}{1.867054in}}%
\pgfpathlineto{\pgfqpoint{1.600563in}{1.901348in}}%
\pgfpathlineto{\pgfqpoint{1.609370in}{2.024891in}}%
\pgfpathlineto{\pgfqpoint{1.618177in}{2.018032in}}%
\pgfpathlineto{\pgfqpoint{1.626984in}{2.196445in}}%
\pgfpathlineto{\pgfqpoint{1.635791in}{2.100365in}}%
\pgfpathlineto{\pgfqpoint{1.644597in}{1.626856in}}%
\pgfpathlineto{\pgfqpoint{1.653404in}{1.537636in}}%
\pgfpathlineto{\pgfqpoint{1.662211in}{1.860167in}}%
\pgfpathlineto{\pgfqpoint{1.671018in}{2.114111in}}%
\pgfpathlineto{\pgfqpoint{1.679825in}{1.976851in}}%
\pgfpathlineto{\pgfqpoint{1.688632in}{2.052326in}}%
\pgfpathlineto{\pgfqpoint{1.697438in}{2.066071in}}%
\pgfpathlineto{\pgfqpoint{1.706245in}{1.963133in}}%
\pgfpathlineto{\pgfqpoint{1.732666in}{2.169009in}}%
\pgfpathlineto{\pgfqpoint{1.741472in}{2.381744in}}%
\pgfpathlineto{\pgfqpoint{1.750279in}{2.223908in}}%
\pgfpathlineto{\pgfqpoint{1.759086in}{2.086648in}}%
\pgfpathlineto{\pgfqpoint{1.767893in}{2.024891in}}%
\pgfpathlineto{\pgfqpoint{1.776700in}{2.299383in}}%
\pgfpathlineto{\pgfqpoint{1.785507in}{2.066071in}}%
\pgfpathlineto{\pgfqpoint{1.794313in}{1.942529in}}%
\pgfpathlineto{\pgfqpoint{1.803120in}{1.921953in}}%
\pgfpathlineto{\pgfqpoint{1.811927in}{1.983710in}}%
\pgfpathlineto{\pgfqpoint{1.829541in}{1.805269in}}%
\pgfpathlineto{\pgfqpoint{1.838347in}{1.517031in}}%
\pgfpathlineto{\pgfqpoint{1.847154in}{1.427839in}}%
\pgfpathlineto{\pgfqpoint{1.855961in}{1.448415in}}%
\pgfpathlineto{\pgfqpoint{1.864768in}{1.832732in}}%
\pgfpathlineto{\pgfqpoint{1.873575in}{1.825873in}}%
\pgfpathlineto{\pgfqpoint{1.882382in}{1.434698in}}%
\pgfpathlineto{\pgfqpoint{1.899995in}{1.935670in}}%
\pgfpathlineto{\pgfqpoint{1.908802in}{1.867054in}}%
\pgfpathlineto{\pgfqpoint{1.926416in}{1.462133in}}%
\pgfpathlineto{\pgfqpoint{1.935222in}{1.619997in}}%
\pgfpathlineto{\pgfqpoint{1.952836in}{1.455274in}}%
\pgfpathlineto{\pgfqpoint{1.961643in}{1.462133in}}%
\pgfpathlineto{\pgfqpoint{1.970450in}{1.599393in}}%
\pgfpathlineto{\pgfqpoint{1.979257in}{1.647432in}}%
\pgfpathlineto{\pgfqpoint{1.988063in}{2.374885in}}%
\pgfpathlineto{\pgfqpoint{1.996870in}{2.127829in}}%
\pgfpathlineto{\pgfqpoint{2.005677in}{2.148405in}}%
\pgfpathlineto{\pgfqpoint{2.023291in}{2.470965in}}%
\pgfpathlineto{\pgfqpoint{2.032097in}{2.491541in}}%
\pgfpathlineto{\pgfqpoint{2.040904in}{2.299383in}}%
\pgfpathlineto{\pgfqpoint{2.049711in}{2.031749in}}%
\pgfpathlineto{\pgfqpoint{2.058518in}{2.306269in}}%
\pgfpathlineto{\pgfqpoint{2.067325in}{2.196445in}}%
\pgfpathlineto{\pgfqpoint{2.076132in}{2.340563in}}%
\pgfpathlineto{\pgfqpoint{2.084938in}{2.120970in}}%
\pgfpathlineto{\pgfqpoint{2.093745in}{1.770975in}}%
\pgfpathlineto{\pgfqpoint{2.102552in}{1.915094in}}%
\pgfpathlineto{\pgfqpoint{2.111359in}{1.496455in}}%
\pgfpathlineto{\pgfqpoint{2.120166in}{1.619997in}}%
\pgfpathlineto{\pgfqpoint{2.128972in}{1.702331in}}%
\pgfpathlineto{\pgfqpoint{2.137779in}{1.709190in}}%
\pgfpathlineto{\pgfqpoint{2.146586in}{1.626856in}}%
\pgfpathlineto{\pgfqpoint{2.155393in}{1.956246in}}%
\pgfpathlineto{\pgfqpoint{2.164200in}{2.196445in}}%
\pgfpathlineto{\pgfqpoint{2.173007in}{2.299383in}}%
\pgfpathlineto{\pgfqpoint{2.181813in}{2.038608in}}%
\pgfpathlineto{\pgfqpoint{2.190620in}{2.052326in}}%
\pgfpathlineto{\pgfqpoint{2.199427in}{2.011173in}}%
\pgfpathlineto{\pgfqpoint{2.208234in}{1.764116in}}%
\pgfpathlineto{\pgfqpoint{2.217041in}{1.633715in}}%
\pgfpathlineto{\pgfqpoint{2.225847in}{1.743512in}}%
\pgfpathlineto{\pgfqpoint{2.234654in}{1.743512in}}%
\pgfpathlineto{\pgfqpoint{2.243461in}{1.613110in}}%
\pgfpathlineto{\pgfqpoint{2.252268in}{1.537636in}}%
\pgfpathlineto{\pgfqpoint{2.261075in}{1.482737in}}%
\pgfpathlineto{\pgfqpoint{2.269882in}{1.791551in}}%
\pgfpathlineto{\pgfqpoint{2.278688in}{1.901348in}}%
\pgfpathlineto{\pgfqpoint{2.287495in}{1.729794in}}%
\pgfpathlineto{\pgfqpoint{2.296302in}{2.244484in}}%
\pgfpathlineto{\pgfqpoint{2.313916in}{1.770975in}}%
\pgfpathlineto{\pgfqpoint{2.322722in}{2.004286in}}%
\pgfpathlineto{\pgfqpoint{2.331529in}{2.141546in}}%
\pgfpathlineto{\pgfqpoint{2.340336in}{1.887630in}}%
\pgfpathlineto{\pgfqpoint{2.349143in}{1.983710in}}%
\pgfpathlineto{\pgfqpoint{2.357950in}{2.018032in}}%
\pgfpathlineto{\pgfqpoint{2.366757in}{1.757229in}}%
\pgfpathlineto{\pgfqpoint{2.375563in}{1.619997in}}%
\pgfpathlineto{\pgfqpoint{2.384370in}{1.928811in}}%
\pgfpathlineto{\pgfqpoint{2.393177in}{1.812128in}}%
\pgfpathlineto{\pgfqpoint{2.401984in}{2.066071in}}%
\pgfpathlineto{\pgfqpoint{2.410791in}{1.784692in}}%
\pgfpathlineto{\pgfqpoint{2.419597in}{2.155292in}}%
\pgfpathlineto{\pgfqpoint{2.428404in}{2.038608in}}%
\pgfpathlineto{\pgfqpoint{2.437211in}{1.949388in}}%
\pgfpathlineto{\pgfqpoint{2.446018in}{2.237625in}}%
\pgfpathlineto{\pgfqpoint{2.463632in}{1.784692in}}%
\pgfpathlineto{\pgfqpoint{2.472438in}{1.819015in}}%
\pgfpathlineto{\pgfqpoint{2.481245in}{1.825873in}}%
\pgfpathlineto{\pgfqpoint{2.490052in}{1.873913in}}%
\pgfpathlineto{\pgfqpoint{2.498859in}{1.805269in}}%
\pgfpathlineto{\pgfqpoint{2.507666in}{1.770975in}}%
\pgfpathlineto{\pgfqpoint{2.516472in}{1.791551in}}%
\pgfpathlineto{\pgfqpoint{2.525279in}{2.093507in}}%
\pgfpathlineto{\pgfqpoint{2.534086in}{2.086648in}}%
\pgfpathlineto{\pgfqpoint{2.542893in}{1.709190in}}%
\pgfpathlineto{\pgfqpoint{2.551700in}{2.210190in}}%
\pgfpathlineto{\pgfqpoint{2.560507in}{2.306269in}}%
\pgfpathlineto{\pgfqpoint{2.569313in}{2.354309in}}%
\pgfpathlineto{\pgfqpoint{2.578120in}{2.368027in}}%
\pgfpathlineto{\pgfqpoint{2.586927in}{2.436643in}}%
\pgfpathlineto{\pgfqpoint{2.595734in}{2.271947in}}%
\pgfpathlineto{\pgfqpoint{2.604541in}{1.997427in}}%
\pgfpathlineto{\pgfqpoint{2.613347in}{1.921953in}}%
\pgfpathlineto{\pgfqpoint{2.622154in}{1.606252in}}%
\pgfpathlineto{\pgfqpoint{2.630961in}{1.805269in}}%
\pgfpathlineto{\pgfqpoint{2.639768in}{1.798410in}}%
\pgfpathlineto{\pgfqpoint{2.648575in}{1.976851in}}%
\pgfpathlineto{\pgfqpoint{2.657382in}{2.258230in}}%
\pgfpathlineto{\pgfqpoint{2.666188in}{2.024891in}}%
\pgfpathlineto{\pgfqpoint{2.683802in}{2.079789in}}%
\pgfpathlineto{\pgfqpoint{2.692609in}{1.750370in}}%
\pgfpathlineto{\pgfqpoint{2.701416in}{1.523918in}}%
\pgfpathlineto{\pgfqpoint{2.710222in}{1.393517in}}%
\pgfpathlineto{\pgfqpoint{2.719029in}{1.791551in}}%
\pgfpathlineto{\pgfqpoint{2.727836in}{1.894489in}}%
\pgfpathlineto{\pgfqpoint{2.736643in}{2.354309in}}%
\pgfpathlineto{\pgfqpoint{2.745450in}{2.594507in}}%
\pgfpathlineto{\pgfqpoint{2.754257in}{2.388603in}}%
\pgfpathlineto{\pgfqpoint{2.763063in}{1.935670in}}%
\pgfpathlineto{\pgfqpoint{2.771870in}{1.757229in}}%
\pgfpathlineto{\pgfqpoint{2.780677in}{1.729794in}}%
\pgfpathlineto{\pgfqpoint{2.789484in}{1.537636in}}%
\pgfpathlineto{\pgfqpoint{2.798291in}{1.578816in}}%
\pgfpathlineto{\pgfqpoint{2.807097in}{1.462133in}}%
\pgfpathlineto{\pgfqpoint{2.815904in}{1.661150in}}%
\pgfpathlineto{\pgfqpoint{2.833518in}{1.983710in}}%
\pgfpathlineto{\pgfqpoint{2.842325in}{2.189586in}}%
\pgfpathlineto{\pgfqpoint{2.851132in}{1.873913in}}%
\pgfpathlineto{\pgfqpoint{2.859938in}{2.265089in}}%
\pgfpathlineto{\pgfqpoint{2.868745in}{2.326846in}}%
\pgfpathlineto{\pgfqpoint{2.877552in}{2.258230in}}%
\pgfpathlineto{\pgfqpoint{2.886359in}{2.340563in}}%
\pgfpathlineto{\pgfqpoint{2.895166in}{2.079789in}}%
\pgfpathlineto{\pgfqpoint{2.903972in}{2.313128in}}%
\pgfpathlineto{\pgfqpoint{2.912779in}{2.079789in}}%
\pgfpathlineto{\pgfqpoint{2.921586in}{1.928811in}}%
\pgfpathlineto{\pgfqpoint{2.930393in}{1.551353in}}%
\pgfpathlineto{\pgfqpoint{2.939200in}{1.894489in}}%
\pgfpathlineto{\pgfqpoint{2.948007in}{1.729794in}}%
\pgfpathlineto{\pgfqpoint{2.956813in}{1.825873in}}%
\pgfpathlineto{\pgfqpoint{2.965620in}{1.764116in}}%
\pgfpathlineto{\pgfqpoint{2.974427in}{2.120970in}}%
\pgfpathlineto{\pgfqpoint{2.983234in}{2.045467in}}%
\pgfpathlineto{\pgfqpoint{2.992041in}{1.901348in}}%
\pgfpathlineto{\pgfqpoint{3.000847in}{1.729794in}}%
\pgfpathlineto{\pgfqpoint{3.009654in}{1.949388in}}%
\pgfpathlineto{\pgfqpoint{3.018461in}{2.100365in}}%
\pgfpathlineto{\pgfqpoint{3.027268in}{2.127829in}}%
\pgfpathlineto{\pgfqpoint{3.036075in}{1.839591in}}%
\pgfpathlineto{\pgfqpoint{3.044882in}{1.901348in}}%
\pgfpathlineto{\pgfqpoint{3.053688in}{1.915094in}}%
\pgfpathlineto{\pgfqpoint{3.062495in}{1.949388in}}%
\pgfpathlineto{\pgfqpoint{3.071302in}{1.839591in}}%
\pgfpathlineto{\pgfqpoint{3.080109in}{1.661150in}}%
\pgfpathlineto{\pgfqpoint{3.088916in}{1.578816in}}%
\pgfpathlineto{\pgfqpoint{3.097722in}{1.674896in}}%
\pgfpathlineto{\pgfqpoint{3.106529in}{1.722935in}}%
\pgfpathlineto{\pgfqpoint{3.115336in}{1.400376in}}%
\pgfpathlineto{\pgfqpoint{3.124143in}{1.963133in}}%
\pgfpathlineto{\pgfqpoint{3.132950in}{1.770975in}}%
\pgfpathlineto{\pgfqpoint{3.141757in}{1.668037in}}%
\pgfpathlineto{\pgfqpoint{3.150563in}{1.757229in}}%
\pgfpathlineto{\pgfqpoint{3.159370in}{1.654291in}}%
\pgfpathlineto{\pgfqpoint{3.168177in}{1.606252in}}%
\pgfpathlineto{\pgfqpoint{3.176984in}{1.873913in}}%
\pgfpathlineto{\pgfqpoint{3.185791in}{1.805269in}}%
\pgfpathlineto{\pgfqpoint{3.194597in}{1.915094in}}%
\pgfpathlineto{\pgfqpoint{3.203404in}{1.613110in}}%
\pgfpathlineto{\pgfqpoint{3.212211in}{1.956246in}}%
\pgfpathlineto{\pgfqpoint{3.221018in}{2.086648in}}%
\pgfpathlineto{\pgfqpoint{3.229825in}{2.196445in}}%
\pgfpathlineto{\pgfqpoint{3.238632in}{2.038608in}}%
\pgfpathlineto{\pgfqpoint{3.247438in}{2.052326in}}%
\pgfpathlineto{\pgfqpoint{3.256245in}{2.237625in}}%
\pgfpathlineto{\pgfqpoint{3.265052in}{2.374885in}}%
\pgfpathlineto{\pgfqpoint{3.282666in}{2.031749in}}%
\pgfpathlineto{\pgfqpoint{3.291472in}{1.983710in}}%
\pgfpathlineto{\pgfqpoint{3.300279in}{2.120970in}}%
\pgfpathlineto{\pgfqpoint{3.309086in}{1.770975in}}%
\pgfpathlineto{\pgfqpoint{3.317893in}{2.072930in}}%
\pgfpathlineto{\pgfqpoint{3.326700in}{2.072930in}}%
\pgfpathlineto{\pgfqpoint{3.335507in}{2.340563in}}%
\pgfpathlineto{\pgfqpoint{3.344313in}{2.388603in}}%
\pgfpathlineto{\pgfqpoint{3.353120in}{2.175868in}}%
\pgfpathlineto{\pgfqpoint{3.361927in}{2.059213in}}%
\pgfpathlineto{\pgfqpoint{3.370734in}{2.093507in}}%
\pgfpathlineto{\pgfqpoint{3.379541in}{2.697445in}}%
\pgfpathlineto{\pgfqpoint{3.388347in}{2.567044in}}%
\pgfpathlineto{\pgfqpoint{3.397154in}{2.381744in}}%
\pgfpathlineto{\pgfqpoint{3.405961in}{2.381744in}}%
\pgfpathlineto{\pgfqpoint{3.414768in}{2.018032in}}%
\pgfpathlineto{\pgfqpoint{3.423575in}{2.443501in}}%
\pgfpathlineto{\pgfqpoint{3.432382in}{1.798410in}}%
\pgfpathlineto{\pgfqpoint{3.441188in}{1.695472in}}%
\pgfpathlineto{\pgfqpoint{3.449995in}{1.805269in}}%
\pgfpathlineto{\pgfqpoint{3.458802in}{1.619997in}}%
\pgfpathlineto{\pgfqpoint{3.467609in}{1.681754in}}%
\pgfpathlineto{\pgfqpoint{3.476416in}{1.668037in}}%
\pgfpathlineto{\pgfqpoint{3.485222in}{1.688613in}}%
\pgfpathlineto{\pgfqpoint{3.502836in}{2.024891in}}%
\pgfpathlineto{\pgfqpoint{3.511643in}{1.860167in}}%
\pgfpathlineto{\pgfqpoint{3.520450in}{1.462133in}}%
\pgfpathlineto{\pgfqpoint{3.529257in}{1.757229in}}%
\pgfpathlineto{\pgfqpoint{3.538063in}{1.661150in}}%
\pgfpathlineto{\pgfqpoint{3.546870in}{1.523918in}}%
\pgfpathlineto{\pgfqpoint{3.555677in}{1.537636in}}%
\pgfpathlineto{\pgfqpoint{3.564484in}{1.764116in}}%
\pgfpathlineto{\pgfqpoint{3.573291in}{1.860167in}}%
\pgfpathlineto{\pgfqpoint{3.582097in}{1.976851in}}%
\pgfpathlineto{\pgfqpoint{3.590904in}{1.928811in}}%
\pgfpathlineto{\pgfqpoint{3.599711in}{1.592534in}}%
\pgfpathlineto{\pgfqpoint{3.608518in}{1.743512in}}%
\pgfpathlineto{\pgfqpoint{3.617325in}{2.285665in}}%
\pgfpathlineto{\pgfqpoint{3.626132in}{2.251343in}}%
\pgfpathlineto{\pgfqpoint{3.634938in}{2.011173in}}%
\pgfpathlineto{\pgfqpoint{3.643745in}{2.223908in}}%
\pgfpathlineto{\pgfqpoint{3.652552in}{1.990569in}}%
\pgfpathlineto{\pgfqpoint{3.661359in}{2.251343in}}%
\pgfpathlineto{\pgfqpoint{3.670166in}{2.217049in}}%
\pgfpathlineto{\pgfqpoint{3.678972in}{2.189586in}}%
\pgfpathlineto{\pgfqpoint{3.687779in}{2.354309in}}%
\pgfpathlineto{\pgfqpoint{3.696586in}{2.175868in}}%
\pgfpathlineto{\pgfqpoint{3.705393in}{2.175868in}}%
\pgfpathlineto{\pgfqpoint{3.714200in}{1.928811in}}%
\pgfpathlineto{\pgfqpoint{3.723007in}{1.949388in}}%
\pgfpathlineto{\pgfqpoint{3.731813in}{2.018032in}}%
\pgfpathlineto{\pgfqpoint{3.740620in}{1.736653in}}%
\pgfpathlineto{\pgfqpoint{3.749427in}{1.963133in}}%
\pgfpathlineto{\pgfqpoint{3.758234in}{2.244484in}}%
\pgfpathlineto{\pgfqpoint{3.767041in}{2.072930in}}%
\pgfpathlineto{\pgfqpoint{3.775847in}{2.120970in}}%
\pgfpathlineto{\pgfqpoint{3.784654in}{1.949388in}}%
\pgfpathlineto{\pgfqpoint{3.793461in}{2.004286in}}%
\pgfpathlineto{\pgfqpoint{3.802268in}{1.825873in}}%
\pgfpathlineto{\pgfqpoint{3.811075in}{1.770975in}}%
\pgfpathlineto{\pgfqpoint{3.819882in}{1.674896in}}%
\pgfpathlineto{\pgfqpoint{3.828688in}{1.626856in}}%
\pgfpathlineto{\pgfqpoint{3.837495in}{1.571958in}}%
\pgfpathlineto{\pgfqpoint{3.846302in}{1.743512in}}%
\pgfpathlineto{\pgfqpoint{3.855109in}{1.846450in}}%
\pgfpathlineto{\pgfqpoint{3.863916in}{1.695472in}}%
\pgfpathlineto{\pgfqpoint{3.890336in}{1.908207in}}%
\pgfpathlineto{\pgfqpoint{3.899143in}{1.942529in}}%
\pgfpathlineto{\pgfqpoint{3.907950in}{2.107252in}}%
\pgfpathlineto{\pgfqpoint{3.916757in}{2.333705in}}%
\pgfpathlineto{\pgfqpoint{3.925563in}{2.155292in}}%
\pgfpathlineto{\pgfqpoint{3.934370in}{2.237625in}}%
\pgfpathlineto{\pgfqpoint{3.943177in}{2.107252in}}%
\pgfpathlineto{\pgfqpoint{3.951984in}{1.928811in}}%
\pgfpathlineto{\pgfqpoint{3.960791in}{2.031749in}}%
\pgfpathlineto{\pgfqpoint{3.969597in}{2.066071in}}%
\pgfpathlineto{\pgfqpoint{3.978404in}{2.004286in}}%
\pgfpathlineto{\pgfqpoint{3.987211in}{2.120970in}}%
\pgfpathlineto{\pgfqpoint{3.996018in}{1.901348in}}%
\pgfpathlineto{\pgfqpoint{4.004825in}{2.196445in}}%
\pgfpathlineto{\pgfqpoint{4.013632in}{1.633715in}}%
\pgfpathlineto{\pgfqpoint{4.022438in}{1.736653in}}%
\pgfpathlineto{\pgfqpoint{4.031245in}{2.127829in}}%
\pgfpathlineto{\pgfqpoint{4.040052in}{1.846450in}}%
\pgfpathlineto{\pgfqpoint{4.048859in}{1.860167in}}%
\pgfpathlineto{\pgfqpoint{4.057666in}{2.038608in}}%
\pgfpathlineto{\pgfqpoint{4.066472in}{1.894489in}}%
\pgfpathlineto{\pgfqpoint{4.075279in}{1.661150in}}%
\pgfpathlineto{\pgfqpoint{4.084086in}{1.825873in}}%
\pgfpathlineto{\pgfqpoint{4.092893in}{1.867054in}}%
\pgfpathlineto{\pgfqpoint{4.101700in}{2.141546in}}%
\pgfpathlineto{\pgfqpoint{4.110507in}{2.217049in}}%
\pgfpathlineto{\pgfqpoint{4.119313in}{2.052326in}}%
\pgfpathlineto{\pgfqpoint{4.128120in}{1.770975in}}%
\pgfpathlineto{\pgfqpoint{4.136927in}{1.812128in}}%
\pgfpathlineto{\pgfqpoint{4.145734in}{2.066071in}}%
\pgfpathlineto{\pgfqpoint{4.154541in}{2.162151in}}%
\pgfpathlineto{\pgfqpoint{4.163347in}{2.319987in}}%
\pgfpathlineto{\pgfqpoint{4.172154in}{1.915094in}}%
\pgfpathlineto{\pgfqpoint{4.180961in}{1.633715in}}%
\pgfpathlineto{\pgfqpoint{4.189768in}{1.263115in}}%
\pgfpathlineto{\pgfqpoint{4.198575in}{1.118997in}}%
\pgfpathlineto{\pgfqpoint{4.207382in}{1.846450in}}%
\pgfpathlineto{\pgfqpoint{4.216188in}{1.894489in}}%
\pgfpathlineto{\pgfqpoint{4.224995in}{1.482737in}}%
\pgfpathlineto{\pgfqpoint{4.233802in}{1.764116in}}%
\pgfpathlineto{\pgfqpoint{4.242609in}{1.805269in}}%
\pgfpathlineto{\pgfqpoint{4.251416in}{2.011173in}}%
\pgfpathlineto{\pgfqpoint{4.260222in}{2.395462in}}%
\pgfpathlineto{\pgfqpoint{4.269029in}{2.155292in}}%
\pgfpathlineto{\pgfqpoint{4.277836in}{2.114111in}}%
\pgfpathlineto{\pgfqpoint{4.286643in}{1.901348in}}%
\pgfpathlineto{\pgfqpoint{4.295450in}{1.901348in}}%
\pgfpathlineto{\pgfqpoint{4.304257in}{1.668037in}}%
\pgfpathlineto{\pgfqpoint{4.313063in}{1.743512in}}%
\pgfpathlineto{\pgfqpoint{4.321870in}{1.805269in}}%
\pgfpathlineto{\pgfqpoint{4.339484in}{1.860167in}}%
\pgfpathlineto{\pgfqpoint{4.348291in}{1.997427in}}%
\pgfpathlineto{\pgfqpoint{4.357097in}{2.052326in}}%
\pgfpathlineto{\pgfqpoint{4.365904in}{1.894489in}}%
\pgfpathlineto{\pgfqpoint{4.374711in}{2.011173in}}%
\pgfpathlineto{\pgfqpoint{4.383518in}{1.887630in}}%
\pgfpathlineto{\pgfqpoint{4.392325in}{1.949388in}}%
\pgfpathlineto{\pgfqpoint{4.401132in}{2.148405in}}%
\pgfpathlineto{\pgfqpoint{4.409938in}{2.066071in}}%
\pgfpathlineto{\pgfqpoint{4.418745in}{2.210190in}}%
\pgfpathlineto{\pgfqpoint{4.427552in}{1.825873in}}%
\pgfpathlineto{\pgfqpoint{4.436359in}{2.018032in}}%
\pgfpathlineto{\pgfqpoint{4.445166in}{2.134687in}}%
\pgfpathlineto{\pgfqpoint{4.453972in}{2.045467in}}%
\pgfpathlineto{\pgfqpoint{4.462779in}{2.086648in}}%
\pgfpathlineto{\pgfqpoint{4.471586in}{1.791551in}}%
\pgfpathlineto{\pgfqpoint{4.480393in}{1.805269in}}%
\pgfpathlineto{\pgfqpoint{4.489200in}{1.935670in}}%
\pgfpathlineto{\pgfqpoint{4.498007in}{1.880772in}}%
\pgfpathlineto{\pgfqpoint{4.506813in}{1.729794in}}%
\pgfpathlineto{\pgfqpoint{4.515620in}{1.551353in}}%
\pgfpathlineto{\pgfqpoint{4.524427in}{1.599393in}}%
\pgfpathlineto{\pgfqpoint{4.533234in}{1.846450in}}%
\pgfpathlineto{\pgfqpoint{4.542041in}{1.777834in}}%
\pgfpathlineto{\pgfqpoint{4.550847in}{1.661150in}}%
\pgfpathlineto{\pgfqpoint{4.559654in}{2.024891in}}%
\pgfpathlineto{\pgfqpoint{4.568461in}{2.066071in}}%
\pgfpathlineto{\pgfqpoint{4.577268in}{2.100365in}}%
\pgfpathlineto{\pgfqpoint{4.586075in}{2.031749in}}%
\pgfpathlineto{\pgfqpoint{4.594882in}{2.093507in}}%
\pgfpathlineto{\pgfqpoint{4.603688in}{2.230767in}}%
\pgfpathlineto{\pgfqpoint{4.612495in}{2.127829in}}%
\pgfpathlineto{\pgfqpoint{4.621302in}{2.059213in}}%
\pgfpathlineto{\pgfqpoint{4.630109in}{2.052326in}}%
\pgfpathlineto{\pgfqpoint{4.638916in}{1.983710in}}%
\pgfpathlineto{\pgfqpoint{4.647722in}{2.326846in}}%
\pgfpathlineto{\pgfqpoint{4.656529in}{2.381744in}}%
\pgfpathlineto{\pgfqpoint{4.665336in}{2.512146in}}%
\pgfpathlineto{\pgfqpoint{4.674143in}{2.278806in}}%
\pgfpathlineto{\pgfqpoint{4.682950in}{2.306269in}}%
\pgfpathlineto{\pgfqpoint{4.691757in}{2.018032in}}%
\pgfpathlineto{\pgfqpoint{4.700563in}{1.921953in}}%
\pgfpathlineto{\pgfqpoint{4.709370in}{1.770975in}}%
\pgfpathlineto{\pgfqpoint{4.718177in}{1.338618in}}%
\pgfpathlineto{\pgfqpoint{4.726984in}{1.482737in}}%
\pgfpathlineto{\pgfqpoint{4.735791in}{1.551353in}}%
\pgfpathlineto{\pgfqpoint{4.744597in}{1.578816in}}%
\pgfpathlineto{\pgfqpoint{4.753404in}{1.956246in}}%
\pgfpathlineto{\pgfqpoint{4.762211in}{2.265089in}}%
\pgfpathlineto{\pgfqpoint{4.771018in}{2.093507in}}%
\pgfpathlineto{\pgfqpoint{4.779825in}{2.347422in}}%
\pgfpathlineto{\pgfqpoint{4.788632in}{2.086648in}}%
\pgfpathlineto{\pgfqpoint{4.797438in}{2.587620in}}%
\pgfpathlineto{\pgfqpoint{4.806245in}{2.429784in}}%
\pgfpathlineto{\pgfqpoint{4.815052in}{2.237625in}}%
\pgfpathlineto{\pgfqpoint{4.823859in}{2.594507in}}%
\pgfpathlineto{\pgfqpoint{4.832666in}{2.539581in}}%
\pgfpathlineto{\pgfqpoint{4.841472in}{2.422925in}}%
\pgfpathlineto{\pgfqpoint{4.850279in}{1.887630in}}%
\pgfpathlineto{\pgfqpoint{4.859086in}{1.819015in}}%
\pgfpathlineto{\pgfqpoint{4.867893in}{1.873913in}}%
\pgfpathlineto{\pgfqpoint{4.876700in}{2.004286in}}%
\pgfpathlineto{\pgfqpoint{4.885507in}{1.908207in}}%
\pgfpathlineto{\pgfqpoint{4.894313in}{1.997427in}}%
\pgfpathlineto{\pgfqpoint{4.903120in}{1.880772in}}%
\pgfpathlineto{\pgfqpoint{4.911927in}{1.819015in}}%
\pgfpathlineto{\pgfqpoint{4.920734in}{1.935670in}}%
\pgfpathlineto{\pgfqpoint{4.929541in}{1.757229in}}%
\pgfpathlineto{\pgfqpoint{4.938347in}{2.210190in}}%
\pgfpathlineto{\pgfqpoint{4.947154in}{2.237625in}}%
\pgfpathlineto{\pgfqpoint{4.955961in}{1.819015in}}%
\pgfpathlineto{\pgfqpoint{4.964768in}{1.805269in}}%
\pgfpathlineto{\pgfqpoint{4.973575in}{1.901348in}}%
\pgfpathlineto{\pgfqpoint{4.982382in}{1.819015in}}%
\pgfpathlineto{\pgfqpoint{4.991188in}{1.832732in}}%
\pgfpathlineto{\pgfqpoint{5.008802in}{2.251343in}}%
\pgfpathlineto{\pgfqpoint{5.017609in}{2.340563in}}%
\pgfpathlineto{\pgfqpoint{5.026416in}{2.292524in}}%
\pgfpathlineto{\pgfqpoint{5.035222in}{2.203303in}}%
\pgfpathlineto{\pgfqpoint{5.044029in}{2.203303in}}%
\pgfpathlineto{\pgfqpoint{5.061643in}{1.784692in}}%
\pgfpathlineto{\pgfqpoint{5.070450in}{1.633715in}}%
\pgfpathlineto{\pgfqpoint{5.079257in}{1.503314in}}%
\pgfpathlineto{\pgfqpoint{5.088063in}{1.777834in}}%
\pgfpathlineto{\pgfqpoint{5.096870in}{1.626856in}}%
\pgfpathlineto{\pgfqpoint{5.105677in}{1.613110in}}%
\pgfpathlineto{\pgfqpoint{5.114484in}{1.626856in}}%
\pgfpathlineto{\pgfqpoint{5.123291in}{1.709190in}}%
\pgfpathlineto{\pgfqpoint{5.132097in}{1.702331in}}%
\pgfpathlineto{\pgfqpoint{5.140904in}{1.764116in}}%
\pgfpathlineto{\pgfqpoint{5.149711in}{1.743512in}}%
\pgfpathlineto{\pgfqpoint{5.158518in}{1.791551in}}%
\pgfpathlineto{\pgfqpoint{5.167325in}{1.963133in}}%
\pgfpathlineto{\pgfqpoint{5.176132in}{1.990569in}}%
\pgfpathlineto{\pgfqpoint{5.184938in}{2.292524in}}%
\pgfpathlineto{\pgfqpoint{5.193745in}{1.681754in}}%
\pgfpathlineto{\pgfqpoint{5.202552in}{1.846450in}}%
\pgfpathlineto{\pgfqpoint{5.211359in}{1.887630in}}%
\pgfpathlineto{\pgfqpoint{5.220166in}{2.018032in}}%
\pgfpathlineto{\pgfqpoint{5.228972in}{2.230767in}}%
\pgfpathlineto{\pgfqpoint{5.237779in}{1.915094in}}%
\pgfpathlineto{\pgfqpoint{5.246586in}{1.688613in}}%
\pgfpathlineto{\pgfqpoint{5.255393in}{1.921953in}}%
\pgfpathlineto{\pgfqpoint{5.264200in}{1.908207in}}%
\pgfpathlineto{\pgfqpoint{5.273007in}{2.066071in}}%
\pgfpathlineto{\pgfqpoint{5.281813in}{2.120970in}}%
\pgfpathlineto{\pgfqpoint{5.290620in}{2.265089in}}%
\pgfpathlineto{\pgfqpoint{5.299427in}{1.825873in}}%
\pgfpathlineto{\pgfqpoint{5.308234in}{1.784692in}}%
\pgfpathlineto{\pgfqpoint{5.317041in}{2.018032in}}%
\pgfpathlineto{\pgfqpoint{5.325847in}{1.990569in}}%
\pgfpathlineto{\pgfqpoint{5.334654in}{2.313128in}}%
\pgfpathlineto{\pgfqpoint{5.343461in}{2.354309in}}%
\pgfpathlineto{\pgfqpoint{5.352268in}{2.196445in}}%
\pgfpathlineto{\pgfqpoint{5.361075in}{1.963133in}}%
\pgfpathlineto{\pgfqpoint{5.369882in}{1.606252in}}%
\pgfpathlineto{\pgfqpoint{5.378688in}{2.107252in}}%
\pgfpathlineto{\pgfqpoint{5.396302in}{2.381744in}}%
\pgfpathlineto{\pgfqpoint{5.405109in}{2.100365in}}%
\pgfpathlineto{\pgfqpoint{5.413916in}{1.942529in}}%
\pgfpathlineto{\pgfqpoint{5.422722in}{1.716076in}}%
\pgfpathlineto{\pgfqpoint{5.431529in}{1.674896in}}%
\pgfpathlineto{\pgfqpoint{5.440336in}{1.915094in}}%
\pgfpathlineto{\pgfqpoint{5.449143in}{1.983710in}}%
\pgfpathlineto{\pgfqpoint{5.457950in}{1.990569in}}%
\pgfpathlineto{\pgfqpoint{5.466757in}{1.661150in}}%
\pgfpathlineto{\pgfqpoint{5.475563in}{1.750370in}}%
\pgfpathlineto{\pgfqpoint{5.484370in}{2.024891in}}%
\pgfpathlineto{\pgfqpoint{5.493177in}{2.450388in}}%
\pgfpathlineto{\pgfqpoint{5.501984in}{2.388603in}}%
\pgfpathlineto{\pgfqpoint{5.510791in}{2.347422in}}%
\pgfpathlineto{\pgfqpoint{5.519597in}{2.066071in}}%
\pgfpathlineto{\pgfqpoint{5.528404in}{1.908207in}}%
\pgfpathlineto{\pgfqpoint{5.537211in}{2.416066in}}%
\pgfpathlineto{\pgfqpoint{5.546018in}{1.928811in}}%
\pgfpathlineto{\pgfqpoint{5.554825in}{1.729794in}}%
\pgfpathlineto{\pgfqpoint{5.563632in}{1.846450in}}%
\pgfpathlineto{\pgfqpoint{5.572438in}{1.743512in}}%
\pgfpathlineto{\pgfqpoint{5.590052in}{1.894489in}}%
\pgfpathlineto{\pgfqpoint{5.598859in}{1.983710in}}%
\pgfpathlineto{\pgfqpoint{5.616472in}{1.544494in}}%
\pgfpathlineto{\pgfqpoint{5.625279in}{1.558212in}}%
\pgfpathlineto{\pgfqpoint{5.634086in}{1.585675in}}%
\pgfpathlineto{\pgfqpoint{5.642893in}{1.462133in}}%
\pgfpathlineto{\pgfqpoint{5.651700in}{1.585675in}}%
\pgfpathlineto{\pgfqpoint{5.660507in}{1.434698in}}%
\pgfpathlineto{\pgfqpoint{5.669313in}{1.215076in}}%
\pgfpathlineto{\pgfqpoint{5.678120in}{1.276861in}}%
\pgfpathlineto{\pgfqpoint{5.686927in}{1.173895in}}%
\pgfpathlineto{\pgfqpoint{5.695734in}{1.434698in}}%
\pgfpathlineto{\pgfqpoint{5.704541in}{1.400376in}}%
\pgfpathlineto{\pgfqpoint{5.713347in}{1.757229in}}%
\pgfpathlineto{\pgfqpoint{5.722154in}{1.839591in}}%
\pgfpathlineto{\pgfqpoint{5.730961in}{1.716076in}}%
\pgfpathlineto{\pgfqpoint{5.739768in}{1.976851in}}%
\pgfpathlineto{\pgfqpoint{5.748575in}{1.949388in}}%
\pgfpathlineto{\pgfqpoint{5.757382in}{1.969992in}}%
\pgfpathlineto{\pgfqpoint{5.766188in}{1.935670in}}%
\pgfpathlineto{\pgfqpoint{5.774995in}{2.004286in}}%
\pgfpathlineto{\pgfqpoint{5.783802in}{2.086648in}}%
\pgfpathlineto{\pgfqpoint{5.801416in}{1.743512in}}%
\pgfpathlineto{\pgfqpoint{5.810222in}{1.777834in}}%
\pgfpathlineto{\pgfqpoint{5.819029in}{1.839591in}}%
\pgfpathlineto{\pgfqpoint{5.827836in}{1.578816in}}%
\pgfpathlineto{\pgfqpoint{5.836643in}{1.599393in}}%
\pgfpathlineto{\pgfqpoint{5.845450in}{1.805269in}}%
\pgfpathlineto{\pgfqpoint{5.854257in}{1.915094in}}%
\pgfpathlineto{\pgfqpoint{5.863063in}{2.114111in}}%
\pgfpathlineto{\pgfqpoint{5.871870in}{2.285665in}}%
\pgfpathlineto{\pgfqpoint{5.880677in}{2.505287in}}%
\pgfpathlineto{\pgfqpoint{5.889484in}{2.278806in}}%
\pgfpathlineto{\pgfqpoint{5.898291in}{2.450388in}}%
\pgfpathlineto{\pgfqpoint{5.907097in}{2.045467in}}%
\pgfpathlineto{\pgfqpoint{5.915904in}{1.853308in}}%
\pgfpathlineto{\pgfqpoint{5.924711in}{2.148405in}}%
\pgfpathlineto{\pgfqpoint{5.933518in}{1.949388in}}%
\pgfpathlineto{\pgfqpoint{5.942325in}{2.038608in}}%
\pgfpathlineto{\pgfqpoint{5.951132in}{2.169009in}}%
\pgfpathlineto{\pgfqpoint{5.959938in}{2.217049in}}%
\pgfpathlineto{\pgfqpoint{5.968745in}{2.018032in}}%
\pgfpathlineto{\pgfqpoint{5.977552in}{1.873913in}}%
\pgfpathlineto{\pgfqpoint{5.986359in}{1.798410in}}%
\pgfpathlineto{\pgfqpoint{5.995166in}{2.004286in}}%
\pgfpathlineto{\pgfqpoint{6.003972in}{2.237625in}}%
\pgfpathlineto{\pgfqpoint{6.012779in}{2.059213in}}%
\pgfpathlineto{\pgfqpoint{6.021586in}{2.265089in}}%
\pgfpathlineto{\pgfqpoint{6.030393in}{1.860167in}}%
\pgfpathlineto{\pgfqpoint{6.039200in}{2.072930in}}%
\pgfpathlineto{\pgfqpoint{6.048007in}{1.853308in}}%
\pgfpathlineto{\pgfqpoint{6.056813in}{1.867054in}}%
\pgfpathlineto{\pgfqpoint{6.065620in}{2.175868in}}%
\pgfpathlineto{\pgfqpoint{6.074427in}{2.134687in}}%
\pgfpathlineto{\pgfqpoint{6.083234in}{1.853308in}}%
\pgfpathlineto{\pgfqpoint{6.092041in}{2.155292in}}%
\pgfpathlineto{\pgfqpoint{6.100847in}{2.313128in}}%
\pgfpathlineto{\pgfqpoint{6.109654in}{2.285665in}}%
\pgfpathlineto{\pgfqpoint{6.118461in}{1.969992in}}%
\pgfpathlineto{\pgfqpoint{6.127268in}{2.031749in}}%
\pgfpathlineto{\pgfqpoint{6.136075in}{2.134687in}}%
\pgfpathlineto{\pgfqpoint{6.144882in}{2.258230in}}%
\pgfpathlineto{\pgfqpoint{6.153688in}{2.059213in}}%
\pgfpathlineto{\pgfqpoint{6.162495in}{1.956246in}}%
\pgfpathlineto{\pgfqpoint{6.171302in}{2.196445in}}%
\pgfpathlineto{\pgfqpoint{6.180109in}{1.880772in}}%
\pgfpathlineto{\pgfqpoint{6.188916in}{1.626856in}}%
\pgfpathlineto{\pgfqpoint{6.206529in}{1.853308in}}%
\pgfpathlineto{\pgfqpoint{6.215336in}{2.086648in}}%
\pgfpathlineto{\pgfqpoint{6.224143in}{1.894489in}}%
\pgfpathlineto{\pgfqpoint{6.232950in}{2.107252in}}%
\pgfpathlineto{\pgfqpoint{6.241757in}{2.031749in}}%
\pgfpathlineto{\pgfqpoint{6.250563in}{2.333705in}}%
\pgfpathlineto{\pgfqpoint{6.259370in}{2.278806in}}%
\pgfpathlineto{\pgfqpoint{6.268177in}{2.319987in}}%
\pgfpathlineto{\pgfqpoint{6.276984in}{2.340563in}}%
\pgfpathlineto{\pgfqpoint{6.285791in}{2.457247in}}%
\pgfpathlineto{\pgfqpoint{6.294597in}{2.519004in}}%
\pgfpathlineto{\pgfqpoint{6.303404in}{2.258230in}}%
\pgfpathlineto{\pgfqpoint{6.312211in}{2.368027in}}%
\pgfpathlineto{\pgfqpoint{6.321018in}{2.127829in}}%
\pgfpathlineto{\pgfqpoint{6.329825in}{1.729794in}}%
\pgfpathlineto{\pgfqpoint{6.338632in}{1.722935in}}%
\pgfpathlineto{\pgfqpoint{6.347438in}{2.100365in}}%
\pgfpathlineto{\pgfqpoint{6.356245in}{2.244484in}}%
\pgfpathlineto{\pgfqpoint{6.365052in}{2.169009in}}%
\pgfpathlineto{\pgfqpoint{6.373859in}{2.004286in}}%
\pgfpathlineto{\pgfqpoint{6.382666in}{1.770975in}}%
\pgfpathlineto{\pgfqpoint{6.391472in}{1.633715in}}%
\pgfpathlineto{\pgfqpoint{6.400279in}{1.269974in}}%
\pgfpathlineto{\pgfqpoint{6.409086in}{1.318014in}}%
\pgfpathlineto{\pgfqpoint{6.426700in}{1.619997in}}%
\pgfpathlineto{\pgfqpoint{6.435507in}{1.839591in}}%
\pgfpathlineto{\pgfqpoint{6.444313in}{1.846450in}}%
\pgfpathlineto{\pgfqpoint{6.453120in}{1.743512in}}%
\pgfpathlineto{\pgfqpoint{6.461927in}{1.393517in}}%
\pgfpathlineto{\pgfqpoint{6.470734in}{1.455274in}}%
\pgfpathlineto{\pgfqpoint{6.479541in}{1.160177in}}%
\pgfpathlineto{\pgfqpoint{6.488347in}{1.606252in}}%
\pgfpathlineto{\pgfqpoint{6.497154in}{1.798410in}}%
\pgfpathlineto{\pgfqpoint{6.505961in}{1.592534in}}%
\pgfpathlineto{\pgfqpoint{6.514768in}{1.695472in}}%
\pgfpathlineto{\pgfqpoint{6.523575in}{1.592534in}}%
\pgfpathlineto{\pgfqpoint{6.532382in}{1.619997in}}%
\pgfpathlineto{\pgfqpoint{6.541188in}{1.709190in}}%
\pgfpathlineto{\pgfqpoint{6.549995in}{1.606252in}}%
\pgfpathlineto{\pgfqpoint{6.558802in}{1.462133in}}%
\pgfpathlineto{\pgfqpoint{6.567609in}{1.839591in}}%
\pgfpathlineto{\pgfqpoint{6.576416in}{1.860167in}}%
\pgfpathlineto{\pgfqpoint{6.594029in}{2.347422in}}%
\pgfpathlineto{\pgfqpoint{6.611643in}{1.860167in}}%
\pgfpathlineto{\pgfqpoint{6.620450in}{1.880772in}}%
\pgfpathlineto{\pgfqpoint{6.629257in}{1.805269in}}%
\pgfpathlineto{\pgfqpoint{6.638063in}{1.969992in}}%
\pgfpathlineto{\pgfqpoint{6.646870in}{2.031749in}}%
\pgfpathlineto{\pgfqpoint{6.655677in}{2.004286in}}%
\pgfpathlineto{\pgfqpoint{6.664484in}{1.791551in}}%
\pgfpathlineto{\pgfqpoint{6.673291in}{1.468992in}}%
\pgfpathlineto{\pgfqpoint{6.682097in}{1.668037in}}%
\pgfpathlineto{\pgfqpoint{6.699711in}{1.537636in}}%
\pgfpathlineto{\pgfqpoint{6.708518in}{2.066071in}}%
\pgfpathlineto{\pgfqpoint{6.717325in}{1.915094in}}%
\pgfpathlineto{\pgfqpoint{6.726132in}{2.230767in}}%
\pgfpathlineto{\pgfqpoint{6.734938in}{2.155292in}}%
\pgfpathlineto{\pgfqpoint{6.743745in}{2.148405in}}%
\pgfpathlineto{\pgfqpoint{6.752552in}{2.306269in}}%
\pgfpathlineto{\pgfqpoint{6.761359in}{2.182727in}}%
\pgfpathlineto{\pgfqpoint{6.770166in}{2.416066in}}%
\pgfpathlineto{\pgfqpoint{6.778972in}{2.402349in}}%
\pgfpathlineto{\pgfqpoint{6.787779in}{2.148405in}}%
\pgfpathlineto{\pgfqpoint{6.796586in}{1.832732in}}%
\pgfpathlineto{\pgfqpoint{6.805393in}{2.059213in}}%
\pgfpathlineto{\pgfqpoint{6.814200in}{1.757229in}}%
\pgfpathlineto{\pgfqpoint{6.823007in}{1.777834in}}%
\pgfpathlineto{\pgfqpoint{6.831813in}{2.223908in}}%
\pgfpathlineto{\pgfqpoint{6.849427in}{1.928811in}}%
\pgfpathlineto{\pgfqpoint{6.858234in}{1.873913in}}%
\pgfpathlineto{\pgfqpoint{6.867041in}{1.688613in}}%
\pgfpathlineto{\pgfqpoint{6.875847in}{1.633715in}}%
\pgfpathlineto{\pgfqpoint{6.893461in}{1.475878in}}%
\pgfpathlineto{\pgfqpoint{6.902268in}{1.894489in}}%
\pgfpathlineto{\pgfqpoint{6.911075in}{1.935670in}}%
\pgfpathlineto{\pgfqpoint{6.919882in}{2.127829in}}%
\pgfpathlineto{\pgfqpoint{6.928688in}{2.223908in}}%
\pgfpathlineto{\pgfqpoint{6.937495in}{2.100365in}}%
\pgfpathlineto{\pgfqpoint{6.946302in}{2.169009in}}%
\pgfpathlineto{\pgfqpoint{6.955109in}{2.093507in}}%
\pgfpathlineto{\pgfqpoint{6.963916in}{2.100365in}}%
\pgfpathlineto{\pgfqpoint{6.972722in}{1.963133in}}%
\pgfpathlineto{\pgfqpoint{6.981529in}{2.031749in}}%
\pgfpathlineto{\pgfqpoint{6.990336in}{1.983710in}}%
\pgfpathlineto{\pgfqpoint{6.999143in}{2.066071in}}%
\pgfpathlineto{\pgfqpoint{7.007950in}{1.990569in}}%
\pgfpathlineto{\pgfqpoint{7.016757in}{1.647432in}}%
\pgfpathlineto{\pgfqpoint{7.025563in}{1.825873in}}%
\pgfpathlineto{\pgfqpoint{7.043177in}{1.702331in}}%
\pgfpathlineto{\pgfqpoint{7.051984in}{1.887630in}}%
\pgfpathlineto{\pgfqpoint{7.060791in}{2.038608in}}%
\pgfpathlineto{\pgfqpoint{7.069597in}{2.368027in}}%
\pgfpathlineto{\pgfqpoint{7.078404in}{2.621942in}}%
\pgfpathlineto{\pgfqpoint{7.087211in}{2.546468in}}%
\pgfpathlineto{\pgfqpoint{7.096018in}{2.354309in}}%
\pgfpathlineto{\pgfqpoint{7.104825in}{2.402349in}}%
\pgfpathlineto{\pgfqpoint{7.113632in}{2.175868in}}%
\pgfpathlineto{\pgfqpoint{7.122438in}{2.436643in}}%
\pgfpathlineto{\pgfqpoint{7.131245in}{2.539581in}}%
\pgfpathlineto{\pgfqpoint{7.140052in}{2.230767in}}%
\pgfpathlineto{\pgfqpoint{7.148859in}{2.223908in}}%
\pgfpathlineto{\pgfqpoint{7.157666in}{2.120970in}}%
\pgfpathlineto{\pgfqpoint{7.166472in}{1.729794in}}%
\pgfpathlineto{\pgfqpoint{7.175279in}{2.011173in}}%
\pgfpathlineto{\pgfqpoint{7.184086in}{1.571958in}}%
\pgfpathlineto{\pgfqpoint{7.192893in}{1.777834in}}%
\pgfpathlineto{\pgfqpoint{7.201700in}{1.894489in}}%
\pgfpathlineto{\pgfqpoint{7.210507in}{1.983710in}}%
\pgfpathlineto{\pgfqpoint{7.219313in}{2.162151in}}%
\pgfpathlineto{\pgfqpoint{7.228120in}{1.997427in}}%
\pgfpathlineto{\pgfqpoint{7.236927in}{1.976851in}}%
\pgfpathlineto{\pgfqpoint{7.245734in}{2.093507in}}%
\pgfpathlineto{\pgfqpoint{7.254541in}{2.107252in}}%
\pgfpathlineto{\pgfqpoint{7.263347in}{2.333705in}}%
\pgfpathlineto{\pgfqpoint{7.272154in}{2.429784in}}%
\pgfpathlineto{\pgfqpoint{7.280961in}{2.182727in}}%
\pgfpathlineto{\pgfqpoint{7.289768in}{1.969992in}}%
\pgfpathlineto{\pgfqpoint{7.298575in}{2.381744in}}%
\pgfpathlineto{\pgfqpoint{7.307382in}{2.120970in}}%
\pgfpathlineto{\pgfqpoint{7.316188in}{2.230767in}}%
\pgfpathlineto{\pgfqpoint{7.324995in}{2.217049in}}%
\pgfpathlineto{\pgfqpoint{7.333802in}{1.976851in}}%
\pgfpathlineto{\pgfqpoint{7.342609in}{1.949388in}}%
\pgfpathlineto{\pgfqpoint{7.351416in}{1.668037in}}%
\pgfpathlineto{\pgfqpoint{7.360222in}{1.729794in}}%
\pgfpathlineto{\pgfqpoint{7.369029in}{1.853308in}}%
\pgfpathlineto{\pgfqpoint{7.377836in}{1.695472in}}%
\pgfpathlineto{\pgfqpoint{7.386643in}{1.757229in}}%
\pgfpathlineto{\pgfqpoint{7.395450in}{1.805269in}}%
\pgfpathlineto{\pgfqpoint{7.404257in}{2.018032in}}%
\pgfpathlineto{\pgfqpoint{7.413063in}{2.120970in}}%
\pgfpathlineto{\pgfqpoint{7.421870in}{2.031749in}}%
\pgfpathlineto{\pgfqpoint{7.430677in}{1.736653in}}%
\pgfpathlineto{\pgfqpoint{7.439484in}{1.633715in}}%
\pgfpathlineto{\pgfqpoint{7.448291in}{1.427839in}}%
\pgfpathlineto{\pgfqpoint{7.457097in}{1.297438in}}%
\pgfpathlineto{\pgfqpoint{7.465904in}{1.537636in}}%
\pgfpathlineto{\pgfqpoint{7.474711in}{1.324873in}}%
\pgfpathlineto{\pgfqpoint{7.483518in}{1.571958in}}%
\pgfpathlineto{\pgfqpoint{7.492325in}{1.729794in}}%
\pgfpathlineto{\pgfqpoint{7.501132in}{1.750370in}}%
\pgfpathlineto{\pgfqpoint{7.509938in}{1.661150in}}%
\pgfpathlineto{\pgfqpoint{7.518745in}{1.709190in}}%
\pgfpathlineto{\pgfqpoint{7.527552in}{1.647432in}}%
\pgfpathlineto{\pgfqpoint{7.536359in}{1.839591in}}%
\pgfpathlineto{\pgfqpoint{7.545166in}{1.441556in}}%
\pgfpathlineto{\pgfqpoint{7.553972in}{1.908207in}}%
\pgfpathlineto{\pgfqpoint{7.562779in}{2.162151in}}%
\pgfpathlineto{\pgfqpoint{7.571586in}{1.853308in}}%
\pgfpathlineto{\pgfqpoint{7.580393in}{1.983710in}}%
\pgfpathlineto{\pgfqpoint{7.589200in}{1.867054in}}%
\pgfpathlineto{\pgfqpoint{7.598007in}{1.633715in}}%
\pgfpathlineto{\pgfqpoint{7.606813in}{1.510172in}}%
\pgfpathlineto{\pgfqpoint{7.624427in}{2.306269in}}%
\pgfpathlineto{\pgfqpoint{7.633234in}{2.093507in}}%
\pgfpathlineto{\pgfqpoint{7.642041in}{2.011173in}}%
\pgfpathlineto{\pgfqpoint{7.650847in}{2.326846in}}%
\pgfpathlineto{\pgfqpoint{7.659654in}{2.477823in}}%
\pgfpathlineto{\pgfqpoint{7.668461in}{2.498428in}}%
\pgfpathlineto{\pgfqpoint{7.677268in}{2.203303in}}%
\pgfpathlineto{\pgfqpoint{7.686075in}{2.271947in}}%
\pgfpathlineto{\pgfqpoint{7.694882in}{1.949388in}}%
\pgfpathlineto{\pgfqpoint{7.703688in}{1.791551in}}%
\pgfpathlineto{\pgfqpoint{7.712495in}{1.482737in}}%
\pgfpathlineto{\pgfqpoint{7.721302in}{1.468992in}}%
\pgfpathlineto{\pgfqpoint{7.730109in}{1.530777in}}%
\pgfpathlineto{\pgfqpoint{7.738916in}{1.654291in}}%
\pgfpathlineto{\pgfqpoint{7.747722in}{1.297438in}}%
\pgfpathlineto{\pgfqpoint{7.756529in}{1.455274in}}%
\pgfpathlineto{\pgfqpoint{7.765336in}{1.414093in}}%
\pgfpathlineto{\pgfqpoint{7.774143in}{1.956246in}}%
\pgfpathlineto{\pgfqpoint{7.782950in}{1.880772in}}%
\pgfpathlineto{\pgfqpoint{7.791757in}{1.873913in}}%
\pgfpathlineto{\pgfqpoint{7.800563in}{2.045467in}}%
\pgfpathlineto{\pgfqpoint{7.809370in}{2.100365in}}%
\pgfpathlineto{\pgfqpoint{7.818177in}{1.976851in}}%
\pgfpathlineto{\pgfqpoint{7.826984in}{1.819015in}}%
\pgfpathlineto{\pgfqpoint{7.835791in}{1.969992in}}%
\pgfpathlineto{\pgfqpoint{7.844597in}{2.038608in}}%
\pgfpathlineto{\pgfqpoint{7.853404in}{1.997427in}}%
\pgfpathlineto{\pgfqpoint{7.862211in}{2.299383in}}%
\pgfpathlineto{\pgfqpoint{7.871018in}{2.422925in}}%
\pgfpathlineto{\pgfqpoint{7.879825in}{2.381744in}}%
\pgfpathlineto{\pgfqpoint{7.888632in}{1.942529in}}%
\pgfpathlineto{\pgfqpoint{7.897438in}{1.311155in}}%
\pgfpathlineto{\pgfqpoint{7.906245in}{1.393517in}}%
\pgfpathlineto{\pgfqpoint{7.915052in}{1.125855in}}%
\pgfpathlineto{\pgfqpoint{7.923859in}{1.468992in}}%
\pgfpathlineto{\pgfqpoint{7.932666in}{1.379799in}}%
\pgfpathlineto{\pgfqpoint{7.941472in}{1.510172in}}%
\pgfpathlineto{\pgfqpoint{7.950279in}{1.489596in}}%
\pgfpathlineto{\pgfqpoint{7.959086in}{1.523918in}}%
\pgfpathlineto{\pgfqpoint{7.967893in}{1.619997in}}%
\pgfpathlineto{\pgfqpoint{7.976700in}{1.551353in}}%
\pgfpathlineto{\pgfqpoint{7.985507in}{1.311155in}}%
\pgfpathlineto{\pgfqpoint{7.994313in}{1.215076in}}%
\pgfpathlineto{\pgfqpoint{8.003120in}{1.180782in}}%
\pgfpathlineto{\pgfqpoint{8.011927in}{1.043522in}}%
\pgfpathlineto{\pgfqpoint{8.020734in}{0.714103in}}%
\pgfpathlineto{\pgfqpoint{8.029541in}{1.050381in}}%
\pgfpathlineto{\pgfqpoint{8.038347in}{1.167036in}}%
\pgfpathlineto{\pgfqpoint{8.047154in}{1.077816in}}%
\pgfpathlineto{\pgfqpoint{8.055961in}{1.112138in}}%
\pgfpathlineto{\pgfqpoint{8.064768in}{0.844505in}}%
\pgfpathlineto{\pgfqpoint{8.073575in}{1.098420in}}%
\pgfpathlineto{\pgfqpoint{8.082382in}{1.125855in}}%
\pgfpathlineto{\pgfqpoint{8.091188in}{1.263115in}}%
\pgfpathlineto{\pgfqpoint{8.099995in}{1.283720in}}%
\pgfpathlineto{\pgfqpoint{8.108802in}{1.407234in}}%
\pgfpathlineto{\pgfqpoint{8.117609in}{1.434698in}}%
\pgfpathlineto{\pgfqpoint{8.126416in}{1.770975in}}%
\pgfpathlineto{\pgfqpoint{8.135222in}{1.825873in}}%
\pgfpathlineto{\pgfqpoint{8.144029in}{2.114111in}}%
\pgfpathlineto{\pgfqpoint{8.152836in}{2.100365in}}%
\pgfpathlineto{\pgfqpoint{8.161643in}{2.107252in}}%
\pgfpathlineto{\pgfqpoint{8.170450in}{1.626856in}}%
\pgfpathlineto{\pgfqpoint{8.179257in}{1.503314in}}%
\pgfpathlineto{\pgfqpoint{8.188063in}{1.510172in}}%
\pgfpathlineto{\pgfqpoint{8.196870in}{1.791551in}}%
\pgfpathlineto{\pgfqpoint{8.205677in}{1.736653in}}%
\pgfpathlineto{\pgfqpoint{8.214484in}{1.880772in}}%
\pgfpathlineto{\pgfqpoint{8.223291in}{1.729794in}}%
\pgfpathlineto{\pgfqpoint{8.232097in}{2.024891in}}%
\pgfpathlineto{\pgfqpoint{8.240904in}{2.203303in}}%
\pgfpathlineto{\pgfqpoint{8.249711in}{1.915094in}}%
\pgfpathlineto{\pgfqpoint{8.258518in}{2.045467in}}%
\pgfpathlineto{\pgfqpoint{8.267325in}{2.024891in}}%
\pgfpathlineto{\pgfqpoint{8.276132in}{2.100365in}}%
\pgfpathlineto{\pgfqpoint{8.284938in}{2.409207in}}%
\pgfpathlineto{\pgfqpoint{8.293745in}{2.573903in}}%
\pgfpathlineto{\pgfqpoint{8.302552in}{2.546468in}}%
\pgfpathlineto{\pgfqpoint{8.311359in}{2.175868in}}%
\pgfpathlineto{\pgfqpoint{8.320166in}{1.983710in}}%
\pgfpathlineto{\pgfqpoint{8.328972in}{1.901348in}}%
\pgfpathlineto{\pgfqpoint{8.337779in}{1.434698in}}%
\pgfpathlineto{\pgfqpoint{8.346586in}{1.523918in}}%
\pgfpathlineto{\pgfqpoint{8.364200in}{2.086648in}}%
\pgfpathlineto{\pgfqpoint{8.373007in}{1.990569in}}%
\pgfpathlineto{\pgfqpoint{8.381813in}{1.880772in}}%
\pgfpathlineto{\pgfqpoint{8.390620in}{2.127829in}}%
\pgfpathlineto{\pgfqpoint{8.399427in}{1.956246in}}%
\pgfpathlineto{\pgfqpoint{8.408234in}{1.578816in}}%
\pgfpathlineto{\pgfqpoint{8.417041in}{1.359195in}}%
\pgfpathlineto{\pgfqpoint{8.434654in}{1.619997in}}%
\pgfpathlineto{\pgfqpoint{8.443461in}{1.784692in}}%
\pgfpathlineto{\pgfqpoint{8.452268in}{1.558212in}}%
\pgfpathlineto{\pgfqpoint{8.461075in}{1.750370in}}%
\pgfpathlineto{\pgfqpoint{8.469882in}{1.915094in}}%
\pgfpathlineto{\pgfqpoint{8.478688in}{1.764116in}}%
\pgfpathlineto{\pgfqpoint{8.487495in}{1.935670in}}%
\pgfpathlineto{\pgfqpoint{8.496302in}{2.024891in}}%
\pgfpathlineto{\pgfqpoint{8.505109in}{2.237625in}}%
\pgfpathlineto{\pgfqpoint{8.513916in}{1.654291in}}%
\pgfpathlineto{\pgfqpoint{8.522722in}{1.647432in}}%
\pgfpathlineto{\pgfqpoint{8.531529in}{1.606252in}}%
\pgfpathlineto{\pgfqpoint{8.540336in}{1.468992in}}%
\pgfpathlineto{\pgfqpoint{8.549143in}{1.565071in}}%
\pgfpathlineto{\pgfqpoint{8.557950in}{1.523918in}}%
\pgfpathlineto{\pgfqpoint{8.566757in}{1.489596in}}%
\pgfpathlineto{\pgfqpoint{8.575563in}{1.297438in}}%
\pgfpathlineto{\pgfqpoint{8.584370in}{1.510172in}}%
\pgfpathlineto{\pgfqpoint{8.593177in}{1.661150in}}%
\pgfpathlineto{\pgfqpoint{8.601984in}{2.100365in}}%
\pgfpathlineto{\pgfqpoint{8.610791in}{1.969992in}}%
\pgfpathlineto{\pgfqpoint{8.619597in}{1.537636in}}%
\pgfpathlineto{\pgfqpoint{8.628404in}{1.935670in}}%
\pgfpathlineto{\pgfqpoint{8.637211in}{1.695472in}}%
\pgfpathlineto{\pgfqpoint{8.646018in}{1.867054in}}%
\pgfpathlineto{\pgfqpoint{8.654825in}{1.688613in}}%
\pgfpathlineto{\pgfqpoint{8.663632in}{1.722935in}}%
\pgfpathlineto{\pgfqpoint{8.672438in}{1.647432in}}%
\pgfpathlineto{\pgfqpoint{8.681245in}{1.688613in}}%
\pgfpathlineto{\pgfqpoint{8.690052in}{1.400376in}}%
\pgfpathlineto{\pgfqpoint{8.698859in}{1.160177in}}%
\pgfpathlineto{\pgfqpoint{8.707666in}{1.434698in}}%
\pgfpathlineto{\pgfqpoint{8.716472in}{1.668037in}}%
\pgfpathlineto{\pgfqpoint{8.725279in}{1.167036in}}%
\pgfpathlineto{\pgfqpoint{8.734086in}{1.105279in}}%
\pgfpathlineto{\pgfqpoint{8.742893in}{1.331760in}}%
\pgfpathlineto{\pgfqpoint{8.751700in}{1.462133in}}%
\pgfpathlineto{\pgfqpoint{8.760507in}{1.613110in}}%
\pgfpathlineto{\pgfqpoint{8.769313in}{1.592534in}}%
\pgfpathlineto{\pgfqpoint{8.786927in}{2.175868in}}%
\pgfpathlineto{\pgfqpoint{8.795734in}{2.285665in}}%
\pgfpathlineto{\pgfqpoint{8.813347in}{1.935670in}}%
\pgfpathlineto{\pgfqpoint{8.822154in}{2.072930in}}%
\pgfpathlineto{\pgfqpoint{8.830961in}{1.963133in}}%
\pgfpathlineto{\pgfqpoint{8.839768in}{2.120970in}}%
\pgfpathlineto{\pgfqpoint{8.848575in}{1.901348in}}%
\pgfpathlineto{\pgfqpoint{8.857382in}{2.004286in}}%
\pgfpathlineto{\pgfqpoint{8.866188in}{2.024891in}}%
\pgfpathlineto{\pgfqpoint{8.874995in}{2.450388in}}%
\pgfpathlineto{\pgfqpoint{8.883802in}{2.368027in}}%
\pgfpathlineto{\pgfqpoint{8.892609in}{2.169009in}}%
\pgfpathlineto{\pgfqpoint{8.910222in}{2.862140in}}%
\pgfpathlineto{\pgfqpoint{8.927836in}{2.141546in}}%
\pgfpathlineto{\pgfqpoint{8.936643in}{2.045467in}}%
\pgfpathlineto{\pgfqpoint{8.945450in}{1.839591in}}%
\pgfpathlineto{\pgfqpoint{8.954257in}{1.949388in}}%
\pgfpathlineto{\pgfqpoint{8.963063in}{2.292524in}}%
\pgfpathlineto{\pgfqpoint{8.971870in}{2.326846in}}%
\pgfpathlineto{\pgfqpoint{8.980677in}{2.436643in}}%
\pgfpathlineto{\pgfqpoint{8.989484in}{2.141546in}}%
\pgfpathlineto{\pgfqpoint{8.998291in}{2.066071in}}%
\pgfpathlineto{\pgfqpoint{9.007097in}{2.100365in}}%
\pgfpathlineto{\pgfqpoint{9.015904in}{1.976851in}}%
\pgfpathlineto{\pgfqpoint{9.024711in}{1.963133in}}%
\pgfpathlineto{\pgfqpoint{9.033518in}{2.045467in}}%
\pgfpathlineto{\pgfqpoint{9.042325in}{1.969992in}}%
\pgfpathlineto{\pgfqpoint{9.051132in}{1.770975in}}%
\pgfpathlineto{\pgfqpoint{9.059938in}{2.114111in}}%
\pgfpathlineto{\pgfqpoint{9.068745in}{2.079789in}}%
\pgfpathlineto{\pgfqpoint{9.077552in}{1.963133in}}%
\pgfpathlineto{\pgfqpoint{9.086359in}{2.024891in}}%
\pgfpathlineto{\pgfqpoint{9.095166in}{1.846450in}}%
\pgfpathlineto{\pgfqpoint{9.103972in}{2.011173in}}%
\pgfpathlineto{\pgfqpoint{9.112779in}{1.963133in}}%
\pgfpathlineto{\pgfqpoint{9.121586in}{2.285665in}}%
\pgfpathlineto{\pgfqpoint{9.130393in}{2.217049in}}%
\pgfpathlineto{\pgfqpoint{9.139200in}{2.120970in}}%
\pgfpathlineto{\pgfqpoint{9.148007in}{1.894489in}}%
\pgfpathlineto{\pgfqpoint{9.156813in}{1.770975in}}%
\pgfpathlineto{\pgfqpoint{9.165620in}{1.565071in}}%
\pgfpathlineto{\pgfqpoint{9.174427in}{1.462133in}}%
\pgfpathlineto{\pgfqpoint{9.183234in}{1.626856in}}%
\pgfpathlineto{\pgfqpoint{9.192041in}{1.873913in}}%
\pgfpathlineto{\pgfqpoint{9.200847in}{1.894489in}}%
\pgfpathlineto{\pgfqpoint{9.209654in}{1.743512in}}%
\pgfpathlineto{\pgfqpoint{9.218461in}{1.942529in}}%
\pgfpathlineto{\pgfqpoint{9.227268in}{2.347422in}}%
\pgfpathlineto{\pgfqpoint{9.236075in}{2.470965in}}%
\pgfpathlineto{\pgfqpoint{9.244882in}{2.354309in}}%
\pgfpathlineto{\pgfqpoint{9.253688in}{1.990569in}}%
\pgfpathlineto{\pgfqpoint{9.262495in}{1.537636in}}%
\pgfpathlineto{\pgfqpoint{9.271302in}{1.695472in}}%
\pgfpathlineto{\pgfqpoint{9.280109in}{1.873913in}}%
\pgfpathlineto{\pgfqpoint{9.297722in}{2.134687in}}%
\pgfpathlineto{\pgfqpoint{9.306529in}{1.853308in}}%
\pgfpathlineto{\pgfqpoint{9.315336in}{1.839591in}}%
\pgfpathlineto{\pgfqpoint{9.324143in}{1.688613in}}%
\pgfpathlineto{\pgfqpoint{9.332950in}{1.578816in}}%
\pgfpathlineto{\pgfqpoint{9.341757in}{1.722935in}}%
\pgfpathlineto{\pgfqpoint{9.359370in}{2.114111in}}%
\pgfpathlineto{\pgfqpoint{9.368177in}{1.963133in}}%
\pgfpathlineto{\pgfqpoint{9.376984in}{1.606252in}}%
\pgfpathlineto{\pgfqpoint{9.385791in}{2.134687in}}%
\pgfpathlineto{\pgfqpoint{9.394597in}{1.983710in}}%
\pgfpathlineto{\pgfqpoint{9.403404in}{1.880772in}}%
\pgfpathlineto{\pgfqpoint{9.412211in}{2.100365in}}%
\pgfpathlineto{\pgfqpoint{9.421018in}{1.688613in}}%
\pgfpathlineto{\pgfqpoint{9.429825in}{1.592534in}}%
\pgfpathlineto{\pgfqpoint{9.438632in}{1.599393in}}%
\pgfpathlineto{\pgfqpoint{9.447438in}{1.832732in}}%
\pgfpathlineto{\pgfqpoint{9.456245in}{2.011173in}}%
\pgfpathlineto{\pgfqpoint{9.465052in}{1.949388in}}%
\pgfpathlineto{\pgfqpoint{9.482666in}{2.031749in}}%
\pgfpathlineto{\pgfqpoint{9.491472in}{1.921953in}}%
\pgfpathlineto{\pgfqpoint{9.500279in}{2.237625in}}%
\pgfpathlineto{\pgfqpoint{9.517893in}{1.709190in}}%
\pgfpathlineto{\pgfqpoint{9.526700in}{1.722935in}}%
\pgfpathlineto{\pgfqpoint{9.535507in}{1.626856in}}%
\pgfpathlineto{\pgfqpoint{9.544313in}{1.585675in}}%
\pgfpathlineto{\pgfqpoint{9.553120in}{1.228822in}}%
\pgfpathlineto{\pgfqpoint{9.561927in}{1.640574in}}%
\pgfpathlineto{\pgfqpoint{9.570734in}{1.606252in}}%
\pgfpathlineto{\pgfqpoint{9.579541in}{2.018032in}}%
\pgfpathlineto{\pgfqpoint{9.588347in}{2.059213in}}%
\pgfpathlineto{\pgfqpoint{9.597154in}{2.038608in}}%
\pgfpathlineto{\pgfqpoint{9.605961in}{1.839591in}}%
\pgfpathlineto{\pgfqpoint{9.614768in}{2.169009in}}%
\pgfpathlineto{\pgfqpoint{9.623575in}{1.956246in}}%
\pgfpathlineto{\pgfqpoint{9.632382in}{2.271947in}}%
\pgfpathlineto{\pgfqpoint{9.641188in}{2.018032in}}%
\pgfpathlineto{\pgfqpoint{9.649995in}{1.956246in}}%
\pgfpathlineto{\pgfqpoint{9.658802in}{1.668037in}}%
\pgfpathlineto{\pgfqpoint{9.667609in}{1.839591in}}%
\pgfpathlineto{\pgfqpoint{9.676416in}{1.695472in}}%
\pgfpathlineto{\pgfqpoint{9.685222in}{1.709190in}}%
\pgfpathlineto{\pgfqpoint{9.694029in}{1.674896in}}%
\pgfpathlineto{\pgfqpoint{9.702836in}{1.777834in}}%
\pgfpathlineto{\pgfqpoint{9.711643in}{1.908207in}}%
\pgfpathlineto{\pgfqpoint{9.720450in}{1.860167in}}%
\pgfpathlineto{\pgfqpoint{9.729257in}{1.976851in}}%
\pgfpathlineto{\pgfqpoint{9.738063in}{2.169009in}}%
\pgfpathlineto{\pgfqpoint{9.746870in}{2.306269in}}%
\pgfpathlineto{\pgfqpoint{9.755677in}{2.052326in}}%
\pgfpathlineto{\pgfqpoint{9.764484in}{1.722935in}}%
\pgfpathlineto{\pgfqpoint{9.782097in}{2.175868in}}%
\pgfpathlineto{\pgfqpoint{9.799711in}{1.709190in}}%
\pgfpathlineto{\pgfqpoint{9.808518in}{1.880772in}}%
\pgfpathlineto{\pgfqpoint{9.817325in}{1.894489in}}%
\pgfpathlineto{\pgfqpoint{9.826132in}{1.681754in}}%
\pgfpathlineto{\pgfqpoint{9.834938in}{1.770975in}}%
\pgfpathlineto{\pgfqpoint{9.843745in}{1.736653in}}%
\pgfpathlineto{\pgfqpoint{9.852552in}{1.599393in}}%
\pgfpathlineto{\pgfqpoint{9.861359in}{1.722935in}}%
\pgfpathlineto{\pgfqpoint{9.870166in}{1.681754in}}%
\pgfpathlineto{\pgfqpoint{9.878972in}{1.345477in}}%
\pgfpathlineto{\pgfqpoint{9.887779in}{1.311155in}}%
\pgfpathlineto{\pgfqpoint{9.896586in}{1.441556in}}%
\pgfpathlineto{\pgfqpoint{9.905393in}{1.297438in}}%
\pgfpathlineto{\pgfqpoint{9.914200in}{1.352336in}}%
\pgfpathlineto{\pgfqpoint{9.923007in}{1.496455in}}%
\pgfpathlineto{\pgfqpoint{9.931813in}{1.777834in}}%
\pgfpathlineto{\pgfqpoint{9.940620in}{2.182727in}}%
\pgfpathlineto{\pgfqpoint{9.949427in}{2.697445in}}%
\pgfpathlineto{\pgfqpoint{9.949427in}{2.697445in}}%
\pgfusepath{stroke}%
\end{pgfscope}%
\begin{pgfscope}%
\pgfpathrectangle{\pgfqpoint{0.702268in}{0.521603in}}{\pgfqpoint{9.687500in}{4.235000in}}%
\pgfusepath{clip}%
\pgfsetrectcap%
\pgfsetroundjoin%
\pgfsetlinewidth{0.501875pt}%
\definecolor{currentstroke}{rgb}{0.501961,0.501961,0.501961}%
\pgfsetstrokecolor{currentstroke}%
\pgfsetstrokeopacity{0.250000}%
\pgfsetdash{}{0pt}%
\pgfpathmoveto{\pgfqpoint{1.142609in}{4.063131in}}%
\pgfpathlineto{\pgfqpoint{1.151416in}{3.514091in}}%
\pgfpathlineto{\pgfqpoint{1.160222in}{2.779779in}}%
\pgfpathlineto{\pgfqpoint{1.169029in}{2.416066in}}%
\pgfpathlineto{\pgfqpoint{1.177836in}{2.361168in}}%
\pgfpathlineto{\pgfqpoint{1.186643in}{2.450388in}}%
\pgfpathlineto{\pgfqpoint{1.195450in}{2.395462in}}%
\pgfpathlineto{\pgfqpoint{1.204257in}{2.120970in}}%
\pgfpathlineto{\pgfqpoint{1.213063in}{2.573903in}}%
\pgfpathlineto{\pgfqpoint{1.221870in}{2.175868in}}%
\pgfpathlineto{\pgfqpoint{1.230677in}{2.018032in}}%
\pgfpathlineto{\pgfqpoint{1.239484in}{2.011173in}}%
\pgfpathlineto{\pgfqpoint{1.248291in}{1.921953in}}%
\pgfpathlineto{\pgfqpoint{1.257097in}{1.716076in}}%
\pgfpathlineto{\pgfqpoint{1.265904in}{2.175868in}}%
\pgfpathlineto{\pgfqpoint{1.274711in}{1.709190in}}%
\pgfpathlineto{\pgfqpoint{1.283518in}{1.503314in}}%
\pgfpathlineto{\pgfqpoint{1.292325in}{1.770975in}}%
\pgfpathlineto{\pgfqpoint{1.301132in}{1.935670in}}%
\pgfpathlineto{\pgfqpoint{1.309938in}{2.203303in}}%
\pgfpathlineto{\pgfqpoint{1.318745in}{2.196445in}}%
\pgfpathlineto{\pgfqpoint{1.327552in}{1.990569in}}%
\pgfpathlineto{\pgfqpoint{1.336359in}{2.038608in}}%
\pgfpathlineto{\pgfqpoint{1.345166in}{1.812128in}}%
\pgfpathlineto{\pgfqpoint{1.353972in}{1.867054in}}%
\pgfpathlineto{\pgfqpoint{1.362779in}{2.313128in}}%
\pgfpathlineto{\pgfqpoint{1.371586in}{2.210190in}}%
\pgfpathlineto{\pgfqpoint{1.380393in}{2.278806in}}%
\pgfpathlineto{\pgfqpoint{1.389200in}{2.258230in}}%
\pgfpathlineto{\pgfqpoint{1.398007in}{2.258230in}}%
\pgfpathlineto{\pgfqpoint{1.406813in}{2.278806in}}%
\pgfpathlineto{\pgfqpoint{1.415620in}{2.258230in}}%
\pgfpathlineto{\pgfqpoint{1.424427in}{2.333705in}}%
\pgfpathlineto{\pgfqpoint{1.433234in}{1.997427in}}%
\pgfpathlineto{\pgfqpoint{1.442041in}{2.484682in}}%
\pgfpathlineto{\pgfqpoint{1.450847in}{1.935670in}}%
\pgfpathlineto{\pgfqpoint{1.459654in}{1.983710in}}%
\pgfpathlineto{\pgfqpoint{1.468461in}{2.251343in}}%
\pgfpathlineto{\pgfqpoint{1.477268in}{2.244484in}}%
\pgfpathlineto{\pgfqpoint{1.486075in}{1.688613in}}%
\pgfpathlineto{\pgfqpoint{1.494882in}{1.729794in}}%
\pgfpathlineto{\pgfqpoint{1.503688in}{1.805269in}}%
\pgfpathlineto{\pgfqpoint{1.512495in}{1.633715in}}%
\pgfpathlineto{\pgfqpoint{1.521302in}{1.757229in}}%
\pgfpathlineto{\pgfqpoint{1.530109in}{1.860167in}}%
\pgfpathlineto{\pgfqpoint{1.538916in}{2.127829in}}%
\pgfpathlineto{\pgfqpoint{1.547722in}{2.024891in}}%
\pgfpathlineto{\pgfqpoint{1.556529in}{2.217049in}}%
\pgfpathlineto{\pgfqpoint{1.565336in}{2.018032in}}%
\pgfpathlineto{\pgfqpoint{1.574143in}{2.024891in}}%
\pgfpathlineto{\pgfqpoint{1.582950in}{2.347422in}}%
\pgfpathlineto{\pgfqpoint{1.591757in}{2.271947in}}%
\pgfpathlineto{\pgfqpoint{1.600563in}{2.148405in}}%
\pgfpathlineto{\pgfqpoint{1.609370in}{2.319987in}}%
\pgfpathlineto{\pgfqpoint{1.618177in}{2.464106in}}%
\pgfpathlineto{\pgfqpoint{1.626984in}{2.278806in}}%
\pgfpathlineto{\pgfqpoint{1.635791in}{1.908207in}}%
\pgfpathlineto{\pgfqpoint{1.644597in}{2.141546in}}%
\pgfpathlineto{\pgfqpoint{1.653404in}{2.251343in}}%
\pgfpathlineto{\pgfqpoint{1.662211in}{1.963133in}}%
\pgfpathlineto{\pgfqpoint{1.671018in}{1.523918in}}%
\pgfpathlineto{\pgfqpoint{1.679825in}{1.640574in}}%
\pgfpathlineto{\pgfqpoint{1.688632in}{1.468992in}}%
\pgfpathlineto{\pgfqpoint{1.697438in}{1.441556in}}%
\pgfpathlineto{\pgfqpoint{1.706245in}{1.510172in}}%
\pgfpathlineto{\pgfqpoint{1.715052in}{1.565071in}}%
\pgfpathlineto{\pgfqpoint{1.723859in}{1.571958in}}%
\pgfpathlineto{\pgfqpoint{1.732666in}{1.853308in}}%
\pgfpathlineto{\pgfqpoint{1.741472in}{1.661150in}}%
\pgfpathlineto{\pgfqpoint{1.750279in}{1.764116in}}%
\pgfpathlineto{\pgfqpoint{1.759086in}{1.832732in}}%
\pgfpathlineto{\pgfqpoint{1.767893in}{2.265089in}}%
\pgfpathlineto{\pgfqpoint{1.785507in}{1.873913in}}%
\pgfpathlineto{\pgfqpoint{1.794313in}{2.024891in}}%
\pgfpathlineto{\pgfqpoint{1.803120in}{1.963133in}}%
\pgfpathlineto{\pgfqpoint{1.811927in}{1.969992in}}%
\pgfpathlineto{\pgfqpoint{1.820734in}{1.853308in}}%
\pgfpathlineto{\pgfqpoint{1.829541in}{1.860167in}}%
\pgfpathlineto{\pgfqpoint{1.838347in}{1.942529in}}%
\pgfpathlineto{\pgfqpoint{1.847154in}{1.812128in}}%
\pgfpathlineto{\pgfqpoint{1.855961in}{1.654291in}}%
\pgfpathlineto{\pgfqpoint{1.864768in}{1.420952in}}%
\pgfpathlineto{\pgfqpoint{1.882382in}{1.352336in}}%
\pgfpathlineto{\pgfqpoint{1.891188in}{1.928811in}}%
\pgfpathlineto{\pgfqpoint{1.899995in}{1.860167in}}%
\pgfpathlineto{\pgfqpoint{1.908802in}{1.990569in}}%
\pgfpathlineto{\pgfqpoint{1.917609in}{1.805269in}}%
\pgfpathlineto{\pgfqpoint{1.926416in}{2.265089in}}%
\pgfpathlineto{\pgfqpoint{1.935222in}{2.368027in}}%
\pgfpathlineto{\pgfqpoint{1.944029in}{2.011173in}}%
\pgfpathlineto{\pgfqpoint{1.952836in}{2.079789in}}%
\pgfpathlineto{\pgfqpoint{1.961643in}{2.457247in}}%
\pgfpathlineto{\pgfqpoint{1.970450in}{2.374885in}}%
\pgfpathlineto{\pgfqpoint{1.979257in}{2.628801in}}%
\pgfpathlineto{\pgfqpoint{1.988063in}{2.457247in}}%
\pgfpathlineto{\pgfqpoint{1.996870in}{2.141546in}}%
\pgfpathlineto{\pgfqpoint{2.005677in}{2.292524in}}%
\pgfpathlineto{\pgfqpoint{2.014484in}{2.210190in}}%
\pgfpathlineto{\pgfqpoint{2.023291in}{1.873913in}}%
\pgfpathlineto{\pgfqpoint{2.032097in}{1.983710in}}%
\pgfpathlineto{\pgfqpoint{2.040904in}{1.949388in}}%
\pgfpathlineto{\pgfqpoint{2.049711in}{1.805269in}}%
\pgfpathlineto{\pgfqpoint{2.058518in}{1.722935in}}%
\pgfpathlineto{\pgfqpoint{2.067325in}{1.997427in}}%
\pgfpathlineto{\pgfqpoint{2.076132in}{2.011173in}}%
\pgfpathlineto{\pgfqpoint{2.084938in}{1.942529in}}%
\pgfpathlineto{\pgfqpoint{2.093745in}{1.729794in}}%
\pgfpathlineto{\pgfqpoint{2.102552in}{1.873913in}}%
\pgfpathlineto{\pgfqpoint{2.111359in}{1.654291in}}%
\pgfpathlineto{\pgfqpoint{2.120166in}{1.921953in}}%
\pgfpathlineto{\pgfqpoint{2.128972in}{1.764116in}}%
\pgfpathlineto{\pgfqpoint{2.137779in}{2.038608in}}%
\pgfpathlineto{\pgfqpoint{2.146586in}{1.997427in}}%
\pgfpathlineto{\pgfqpoint{2.164200in}{1.812128in}}%
\pgfpathlineto{\pgfqpoint{2.173007in}{2.313128in}}%
\pgfpathlineto{\pgfqpoint{2.181813in}{2.265089in}}%
\pgfpathlineto{\pgfqpoint{2.190620in}{2.656264in}}%
\pgfpathlineto{\pgfqpoint{2.199427in}{2.573903in}}%
\pgfpathlineto{\pgfqpoint{2.208234in}{2.004286in}}%
\pgfpathlineto{\pgfqpoint{2.217041in}{1.805269in}}%
\pgfpathlineto{\pgfqpoint{2.225847in}{1.736653in}}%
\pgfpathlineto{\pgfqpoint{2.234654in}{1.441556in}}%
\pgfpathlineto{\pgfqpoint{2.243461in}{1.873913in}}%
\pgfpathlineto{\pgfqpoint{2.252268in}{2.011173in}}%
\pgfpathlineto{\pgfqpoint{2.261075in}{1.695472in}}%
\pgfpathlineto{\pgfqpoint{2.269882in}{1.915094in}}%
\pgfpathlineto{\pgfqpoint{2.278688in}{2.079789in}}%
\pgfpathlineto{\pgfqpoint{2.305109in}{1.503314in}}%
\pgfpathlineto{\pgfqpoint{2.313916in}{1.544494in}}%
\pgfpathlineto{\pgfqpoint{2.322722in}{1.928811in}}%
\pgfpathlineto{\pgfqpoint{2.331529in}{2.059213in}}%
\pgfpathlineto{\pgfqpoint{2.340336in}{2.512146in}}%
\pgfpathlineto{\pgfqpoint{2.349143in}{2.066071in}}%
\pgfpathlineto{\pgfqpoint{2.357950in}{1.921953in}}%
\pgfpathlineto{\pgfqpoint{2.366757in}{1.956246in}}%
\pgfpathlineto{\pgfqpoint{2.375563in}{1.976851in}}%
\pgfpathlineto{\pgfqpoint{2.384370in}{1.860167in}}%
\pgfpathlineto{\pgfqpoint{2.393177in}{1.853308in}}%
\pgfpathlineto{\pgfqpoint{2.401984in}{1.510172in}}%
\pgfpathlineto{\pgfqpoint{2.410791in}{1.517031in}}%
\pgfpathlineto{\pgfqpoint{2.428404in}{2.175868in}}%
\pgfpathlineto{\pgfqpoint{2.437211in}{2.265089in}}%
\pgfpathlineto{\pgfqpoint{2.446018in}{2.072930in}}%
\pgfpathlineto{\pgfqpoint{2.454825in}{2.251343in}}%
\pgfpathlineto{\pgfqpoint{2.463632in}{1.997427in}}%
\pgfpathlineto{\pgfqpoint{2.472438in}{2.313128in}}%
\pgfpathlineto{\pgfqpoint{2.481245in}{1.969992in}}%
\pgfpathlineto{\pgfqpoint{2.490052in}{1.935670in}}%
\pgfpathlineto{\pgfqpoint{2.498859in}{1.928811in}}%
\pgfpathlineto{\pgfqpoint{2.507666in}{1.832732in}}%
\pgfpathlineto{\pgfqpoint{2.525279in}{2.018032in}}%
\pgfpathlineto{\pgfqpoint{2.534086in}{2.120970in}}%
\pgfpathlineto{\pgfqpoint{2.560507in}{1.640574in}}%
\pgfpathlineto{\pgfqpoint{2.569313in}{1.716076in}}%
\pgfpathlineto{\pgfqpoint{2.578120in}{1.668037in}}%
\pgfpathlineto{\pgfqpoint{2.586927in}{1.661150in}}%
\pgfpathlineto{\pgfqpoint{2.595734in}{1.455274in}}%
\pgfpathlineto{\pgfqpoint{2.604541in}{1.709190in}}%
\pgfpathlineto{\pgfqpoint{2.613347in}{1.462133in}}%
\pgfpathlineto{\pgfqpoint{2.622154in}{1.537636in}}%
\pgfpathlineto{\pgfqpoint{2.630961in}{1.839591in}}%
\pgfpathlineto{\pgfqpoint{2.639768in}{1.825873in}}%
\pgfpathlineto{\pgfqpoint{2.648575in}{2.004286in}}%
\pgfpathlineto{\pgfqpoint{2.657382in}{1.894489in}}%
\pgfpathlineto{\pgfqpoint{2.666188in}{2.031749in}}%
\pgfpathlineto{\pgfqpoint{2.674995in}{2.210190in}}%
\pgfpathlineto{\pgfqpoint{2.683802in}{2.203303in}}%
\pgfpathlineto{\pgfqpoint{2.692609in}{2.141546in}}%
\pgfpathlineto{\pgfqpoint{2.701416in}{2.155292in}}%
\pgfpathlineto{\pgfqpoint{2.710222in}{1.880772in}}%
\pgfpathlineto{\pgfqpoint{2.719029in}{2.031749in}}%
\pgfpathlineto{\pgfqpoint{2.736643in}{2.004286in}}%
\pgfpathlineto{\pgfqpoint{2.754257in}{2.004286in}}%
\pgfpathlineto{\pgfqpoint{2.763063in}{1.743512in}}%
\pgfpathlineto{\pgfqpoint{2.771870in}{1.661150in}}%
\pgfpathlineto{\pgfqpoint{2.780677in}{1.928811in}}%
\pgfpathlineto{\pgfqpoint{2.789484in}{2.072930in}}%
\pgfpathlineto{\pgfqpoint{2.798291in}{2.347422in}}%
\pgfpathlineto{\pgfqpoint{2.807097in}{2.237625in}}%
\pgfpathlineto{\pgfqpoint{2.815904in}{2.024891in}}%
\pgfpathlineto{\pgfqpoint{2.824711in}{1.894489in}}%
\pgfpathlineto{\pgfqpoint{2.833518in}{2.004286in}}%
\pgfpathlineto{\pgfqpoint{2.842325in}{1.887630in}}%
\pgfpathlineto{\pgfqpoint{2.859938in}{2.148405in}}%
\pgfpathlineto{\pgfqpoint{2.868745in}{2.251343in}}%
\pgfpathlineto{\pgfqpoint{2.877552in}{2.512146in}}%
\pgfpathlineto{\pgfqpoint{2.886359in}{2.553326in}}%
\pgfpathlineto{\pgfqpoint{2.895166in}{2.464106in}}%
\pgfpathlineto{\pgfqpoint{2.903972in}{2.436643in}}%
\pgfpathlineto{\pgfqpoint{2.912779in}{2.690558in}}%
\pgfpathlineto{\pgfqpoint{2.921586in}{2.313128in}}%
\pgfpathlineto{\pgfqpoint{2.930393in}{2.477823in}}%
\pgfpathlineto{\pgfqpoint{2.939200in}{2.210190in}}%
\pgfpathlineto{\pgfqpoint{2.948007in}{2.491541in}}%
\pgfpathlineto{\pgfqpoint{2.956813in}{2.230767in}}%
\pgfpathlineto{\pgfqpoint{2.965620in}{2.244484in}}%
\pgfpathlineto{\pgfqpoint{2.974427in}{2.169009in}}%
\pgfpathlineto{\pgfqpoint{2.983234in}{2.251343in}}%
\pgfpathlineto{\pgfqpoint{2.992041in}{2.210190in}}%
\pgfpathlineto{\pgfqpoint{3.000847in}{2.285665in}}%
\pgfpathlineto{\pgfqpoint{3.009654in}{1.867054in}}%
\pgfpathlineto{\pgfqpoint{3.018461in}{1.585675in}}%
\pgfpathlineto{\pgfqpoint{3.027268in}{1.565071in}}%
\pgfpathlineto{\pgfqpoint{3.036075in}{1.791551in}}%
\pgfpathlineto{\pgfqpoint{3.044882in}{1.716076in}}%
\pgfpathlineto{\pgfqpoint{3.053688in}{1.489596in}}%
\pgfpathlineto{\pgfqpoint{3.062495in}{1.544494in}}%
\pgfpathlineto{\pgfqpoint{3.071302in}{1.832732in}}%
\pgfpathlineto{\pgfqpoint{3.080109in}{1.812128in}}%
\pgfpathlineto{\pgfqpoint{3.088916in}{1.571958in}}%
\pgfpathlineto{\pgfqpoint{3.097722in}{1.654291in}}%
\pgfpathlineto{\pgfqpoint{3.106529in}{1.599393in}}%
\pgfpathlineto{\pgfqpoint{3.115336in}{1.318014in}}%
\pgfpathlineto{\pgfqpoint{3.124143in}{1.208217in}}%
\pgfpathlineto{\pgfqpoint{3.132950in}{1.839591in}}%
\pgfpathlineto{\pgfqpoint{3.141757in}{1.976851in}}%
\pgfpathlineto{\pgfqpoint{3.150563in}{2.141546in}}%
\pgfpathlineto{\pgfqpoint{3.159370in}{2.169009in}}%
\pgfpathlineto{\pgfqpoint{3.168177in}{2.031749in}}%
\pgfpathlineto{\pgfqpoint{3.176984in}{2.237625in}}%
\pgfpathlineto{\pgfqpoint{3.185791in}{2.223908in}}%
\pgfpathlineto{\pgfqpoint{3.194597in}{2.381744in}}%
\pgfpathlineto{\pgfqpoint{3.203404in}{2.251343in}}%
\pgfpathlineto{\pgfqpoint{3.212211in}{2.285665in}}%
\pgfpathlineto{\pgfqpoint{3.221018in}{2.368027in}}%
\pgfpathlineto{\pgfqpoint{3.229825in}{2.230767in}}%
\pgfpathlineto{\pgfqpoint{3.238632in}{2.326846in}}%
\pgfpathlineto{\pgfqpoint{3.247438in}{2.196445in}}%
\pgfpathlineto{\pgfqpoint{3.256245in}{1.887630in}}%
\pgfpathlineto{\pgfqpoint{3.265052in}{1.777834in}}%
\pgfpathlineto{\pgfqpoint{3.273859in}{1.475878in}}%
\pgfpathlineto{\pgfqpoint{3.282666in}{1.654291in}}%
\pgfpathlineto{\pgfqpoint{3.291472in}{1.324873in}}%
\pgfpathlineto{\pgfqpoint{3.300279in}{1.702331in}}%
\pgfpathlineto{\pgfqpoint{3.309086in}{1.633715in}}%
\pgfpathlineto{\pgfqpoint{3.317893in}{1.976851in}}%
\pgfpathlineto{\pgfqpoint{3.326700in}{2.182727in}}%
\pgfpathlineto{\pgfqpoint{3.335507in}{2.285665in}}%
\pgfpathlineto{\pgfqpoint{3.344313in}{2.141546in}}%
\pgfpathlineto{\pgfqpoint{3.353120in}{2.292524in}}%
\pgfpathlineto{\pgfqpoint{3.361927in}{2.416066in}}%
\pgfpathlineto{\pgfqpoint{3.370734in}{2.429784in}}%
\pgfpathlineto{\pgfqpoint{3.379541in}{2.470965in}}%
\pgfpathlineto{\pgfqpoint{3.388347in}{2.827818in}}%
\pgfpathlineto{\pgfqpoint{3.397154in}{2.855282in}}%
\pgfpathlineto{\pgfqpoint{3.405961in}{2.841564in}}%
\pgfpathlineto{\pgfqpoint{3.414768in}{2.381744in}}%
\pgfpathlineto{\pgfqpoint{3.423575in}{2.354309in}}%
\pgfpathlineto{\pgfqpoint{3.432382in}{2.443501in}}%
\pgfpathlineto{\pgfqpoint{3.441188in}{2.189586in}}%
\pgfpathlineto{\pgfqpoint{3.449995in}{1.894489in}}%
\pgfpathlineto{\pgfqpoint{3.458802in}{1.983710in}}%
\pgfpathlineto{\pgfqpoint{3.467609in}{1.867054in}}%
\pgfpathlineto{\pgfqpoint{3.476416in}{1.846450in}}%
\pgfpathlineto{\pgfqpoint{3.485222in}{1.908207in}}%
\pgfpathlineto{\pgfqpoint{3.494029in}{2.079789in}}%
\pgfpathlineto{\pgfqpoint{3.502836in}{1.921953in}}%
\pgfpathlineto{\pgfqpoint{3.520450in}{2.470965in}}%
\pgfpathlineto{\pgfqpoint{3.529257in}{2.306269in}}%
\pgfpathlineto{\pgfqpoint{3.538063in}{2.271947in}}%
\pgfpathlineto{\pgfqpoint{3.546870in}{2.409207in}}%
\pgfpathlineto{\pgfqpoint{3.555677in}{2.374885in}}%
\pgfpathlineto{\pgfqpoint{3.564484in}{2.354309in}}%
\pgfpathlineto{\pgfqpoint{3.573291in}{2.148405in}}%
\pgfpathlineto{\pgfqpoint{3.582097in}{1.860167in}}%
\pgfpathlineto{\pgfqpoint{3.590904in}{1.414093in}}%
\pgfpathlineto{\pgfqpoint{3.599711in}{1.324873in}}%
\pgfpathlineto{\pgfqpoint{3.608518in}{1.544494in}}%
\pgfpathlineto{\pgfqpoint{3.617325in}{1.455274in}}%
\pgfpathlineto{\pgfqpoint{3.626132in}{1.825873in}}%
\pgfpathlineto{\pgfqpoint{3.634938in}{1.894489in}}%
\pgfpathlineto{\pgfqpoint{3.643745in}{1.867054in}}%
\pgfpathlineto{\pgfqpoint{3.652552in}{2.024891in}}%
\pgfpathlineto{\pgfqpoint{3.661359in}{1.798410in}}%
\pgfpathlineto{\pgfqpoint{3.670166in}{1.791551in}}%
\pgfpathlineto{\pgfqpoint{3.678972in}{1.434698in}}%
\pgfpathlineto{\pgfqpoint{3.687779in}{1.468992in}}%
\pgfpathlineto{\pgfqpoint{3.705393in}{1.832732in}}%
\pgfpathlineto{\pgfqpoint{3.714200in}{1.729794in}}%
\pgfpathlineto{\pgfqpoint{3.723007in}{1.269974in}}%
\pgfpathlineto{\pgfqpoint{3.731813in}{1.276861in}}%
\pgfpathlineto{\pgfqpoint{3.740620in}{1.592534in}}%
\pgfpathlineto{\pgfqpoint{3.749427in}{1.764116in}}%
\pgfpathlineto{\pgfqpoint{3.758234in}{1.963133in}}%
\pgfpathlineto{\pgfqpoint{3.767041in}{1.846450in}}%
\pgfpathlineto{\pgfqpoint{3.784654in}{1.510172in}}%
\pgfpathlineto{\pgfqpoint{3.793461in}{1.963133in}}%
\pgfpathlineto{\pgfqpoint{3.802268in}{1.661150in}}%
\pgfpathlineto{\pgfqpoint{3.811075in}{1.249398in}}%
\pgfpathlineto{\pgfqpoint{3.819882in}{1.619997in}}%
\pgfpathlineto{\pgfqpoint{3.828688in}{1.784692in}}%
\pgfpathlineto{\pgfqpoint{3.837495in}{2.162151in}}%
\pgfpathlineto{\pgfqpoint{3.846302in}{2.416066in}}%
\pgfpathlineto{\pgfqpoint{3.855109in}{2.169009in}}%
\pgfpathlineto{\pgfqpoint{3.863916in}{2.059213in}}%
\pgfpathlineto{\pgfqpoint{3.872722in}{2.093507in}}%
\pgfpathlineto{\pgfqpoint{3.881529in}{2.024891in}}%
\pgfpathlineto{\pgfqpoint{3.890336in}{2.306269in}}%
\pgfpathlineto{\pgfqpoint{3.907950in}{1.757229in}}%
\pgfpathlineto{\pgfqpoint{3.916757in}{1.990569in}}%
\pgfpathlineto{\pgfqpoint{3.925563in}{1.997427in}}%
\pgfpathlineto{\pgfqpoint{3.934370in}{1.997427in}}%
\pgfpathlineto{\pgfqpoint{3.943177in}{1.942529in}}%
\pgfpathlineto{\pgfqpoint{3.951984in}{1.935670in}}%
\pgfpathlineto{\pgfqpoint{3.960791in}{1.976851in}}%
\pgfpathlineto{\pgfqpoint{3.969597in}{1.908207in}}%
\pgfpathlineto{\pgfqpoint{3.978404in}{1.956246in}}%
\pgfpathlineto{\pgfqpoint{3.987211in}{1.722935in}}%
\pgfpathlineto{\pgfqpoint{4.004825in}{1.551353in}}%
\pgfpathlineto{\pgfqpoint{4.013632in}{1.825873in}}%
\pgfpathlineto{\pgfqpoint{4.022438in}{1.777834in}}%
\pgfpathlineto{\pgfqpoint{4.031245in}{1.812128in}}%
\pgfpathlineto{\pgfqpoint{4.040052in}{1.661150in}}%
\pgfpathlineto{\pgfqpoint{4.048859in}{1.750370in}}%
\pgfpathlineto{\pgfqpoint{4.057666in}{1.812128in}}%
\pgfpathlineto{\pgfqpoint{4.066472in}{2.045467in}}%
\pgfpathlineto{\pgfqpoint{4.075279in}{2.388603in}}%
\pgfpathlineto{\pgfqpoint{4.084086in}{2.196445in}}%
\pgfpathlineto{\pgfqpoint{4.092893in}{2.223908in}}%
\pgfpathlineto{\pgfqpoint{4.101700in}{2.223908in}}%
\pgfpathlineto{\pgfqpoint{4.110507in}{2.086648in}}%
\pgfpathlineto{\pgfqpoint{4.119313in}{1.777834in}}%
\pgfpathlineto{\pgfqpoint{4.128120in}{2.031749in}}%
\pgfpathlineto{\pgfqpoint{4.136927in}{1.318014in}}%
\pgfpathlineto{\pgfqpoint{4.145734in}{1.372912in}}%
\pgfpathlineto{\pgfqpoint{4.154541in}{1.551353in}}%
\pgfpathlineto{\pgfqpoint{4.163347in}{1.496455in}}%
\pgfpathlineto{\pgfqpoint{4.172154in}{1.468992in}}%
\pgfpathlineto{\pgfqpoint{4.180961in}{1.626856in}}%
\pgfpathlineto{\pgfqpoint{4.189768in}{1.414093in}}%
\pgfpathlineto{\pgfqpoint{4.198575in}{1.393517in}}%
\pgfpathlineto{\pgfqpoint{4.207382in}{1.729794in}}%
\pgfpathlineto{\pgfqpoint{4.216188in}{1.791551in}}%
\pgfpathlineto{\pgfqpoint{4.224995in}{2.072930in}}%
\pgfpathlineto{\pgfqpoint{4.233802in}{2.237625in}}%
\pgfpathlineto{\pgfqpoint{4.242609in}{2.525863in}}%
\pgfpathlineto{\pgfqpoint{4.251416in}{2.635660in}}%
\pgfpathlineto{\pgfqpoint{4.260222in}{2.306269in}}%
\pgfpathlineto{\pgfqpoint{4.269029in}{2.299383in}}%
\pgfpathlineto{\pgfqpoint{4.277836in}{1.695472in}}%
\pgfpathlineto{\pgfqpoint{4.286643in}{1.441556in}}%
\pgfpathlineto{\pgfqpoint{4.295450in}{1.283720in}}%
\pgfpathlineto{\pgfqpoint{4.304257in}{1.551353in}}%
\pgfpathlineto{\pgfqpoint{4.313063in}{1.674896in}}%
\pgfpathlineto{\pgfqpoint{4.321870in}{1.764116in}}%
\pgfpathlineto{\pgfqpoint{4.330677in}{1.777834in}}%
\pgfpathlineto{\pgfqpoint{4.339484in}{1.976851in}}%
\pgfpathlineto{\pgfqpoint{4.348291in}{2.114111in}}%
\pgfpathlineto{\pgfqpoint{4.357097in}{1.928811in}}%
\pgfpathlineto{\pgfqpoint{4.365904in}{1.928811in}}%
\pgfpathlineto{\pgfqpoint{4.374711in}{1.688613in}}%
\pgfpathlineto{\pgfqpoint{4.383518in}{1.386658in}}%
\pgfpathlineto{\pgfqpoint{4.392325in}{1.448415in}}%
\pgfpathlineto{\pgfqpoint{4.401132in}{1.441556in}}%
\pgfpathlineto{\pgfqpoint{4.409938in}{1.242539in}}%
\pgfpathlineto{\pgfqpoint{4.418745in}{1.379799in}}%
\pgfpathlineto{\pgfqpoint{4.427552in}{1.571958in}}%
\pgfpathlineto{\pgfqpoint{4.436359in}{1.537636in}}%
\pgfpathlineto{\pgfqpoint{4.445166in}{1.722935in}}%
\pgfpathlineto{\pgfqpoint{4.453972in}{1.619997in}}%
\pgfpathlineto{\pgfqpoint{4.462779in}{2.223908in}}%
\pgfpathlineto{\pgfqpoint{4.471586in}{2.724880in}}%
\pgfpathlineto{\pgfqpoint{4.480393in}{2.587620in}}%
\pgfpathlineto{\pgfqpoint{4.489200in}{2.265089in}}%
\pgfpathlineto{\pgfqpoint{4.498007in}{2.258230in}}%
\pgfpathlineto{\pgfqpoint{4.506813in}{2.319987in}}%
\pgfpathlineto{\pgfqpoint{4.515620in}{2.120970in}}%
\pgfpathlineto{\pgfqpoint{4.524427in}{2.127829in}}%
\pgfpathlineto{\pgfqpoint{4.533234in}{2.018032in}}%
\pgfpathlineto{\pgfqpoint{4.542041in}{2.340563in}}%
\pgfpathlineto{\pgfqpoint{4.550847in}{2.018032in}}%
\pgfpathlineto{\pgfqpoint{4.559654in}{1.976851in}}%
\pgfpathlineto{\pgfqpoint{4.568461in}{2.072930in}}%
\pgfpathlineto{\pgfqpoint{4.577268in}{1.825873in}}%
\pgfpathlineto{\pgfqpoint{4.586075in}{1.976851in}}%
\pgfpathlineto{\pgfqpoint{4.594882in}{2.072930in}}%
\pgfpathlineto{\pgfqpoint{4.603688in}{1.798410in}}%
\pgfpathlineto{\pgfqpoint{4.612495in}{1.654291in}}%
\pgfpathlineto{\pgfqpoint{4.621302in}{1.873913in}}%
\pgfpathlineto{\pgfqpoint{4.630109in}{1.867054in}}%
\pgfpathlineto{\pgfqpoint{4.638916in}{1.867054in}}%
\pgfpathlineto{\pgfqpoint{4.647722in}{1.537636in}}%
\pgfpathlineto{\pgfqpoint{4.656529in}{1.571958in}}%
\pgfpathlineto{\pgfqpoint{4.665336in}{1.427839in}}%
\pgfpathlineto{\pgfqpoint{4.674143in}{1.729794in}}%
\pgfpathlineto{\pgfqpoint{4.682950in}{1.819015in}}%
\pgfpathlineto{\pgfqpoint{4.691757in}{1.819015in}}%
\pgfpathlineto{\pgfqpoint{4.700563in}{2.024891in}}%
\pgfpathlineto{\pgfqpoint{4.718177in}{2.292524in}}%
\pgfpathlineto{\pgfqpoint{4.726984in}{1.812128in}}%
\pgfpathlineto{\pgfqpoint{4.735791in}{2.059213in}}%
\pgfpathlineto{\pgfqpoint{4.744597in}{2.429784in}}%
\pgfpathlineto{\pgfqpoint{4.753404in}{2.306269in}}%
\pgfpathlineto{\pgfqpoint{4.771018in}{1.908207in}}%
\pgfpathlineto{\pgfqpoint{4.779825in}{2.052326in}}%
\pgfpathlineto{\pgfqpoint{4.788632in}{1.716076in}}%
\pgfpathlineto{\pgfqpoint{4.797438in}{1.613110in}}%
\pgfpathlineto{\pgfqpoint{4.806245in}{1.283720in}}%
\pgfpathlineto{\pgfqpoint{4.815052in}{1.372912in}}%
\pgfpathlineto{\pgfqpoint{4.823859in}{1.633715in}}%
\pgfpathlineto{\pgfqpoint{4.832666in}{1.571958in}}%
\pgfpathlineto{\pgfqpoint{4.841472in}{1.592534in}}%
\pgfpathlineto{\pgfqpoint{4.850279in}{1.894489in}}%
\pgfpathlineto{\pgfqpoint{4.859086in}{1.812128in}}%
\pgfpathlineto{\pgfqpoint{4.867893in}{1.661150in}}%
\pgfpathlineto{\pgfqpoint{4.876700in}{1.613110in}}%
\pgfpathlineto{\pgfqpoint{4.885507in}{1.915094in}}%
\pgfpathlineto{\pgfqpoint{4.894313in}{2.079789in}}%
\pgfpathlineto{\pgfqpoint{4.903120in}{2.169009in}}%
\pgfpathlineto{\pgfqpoint{4.911927in}{1.963133in}}%
\pgfpathlineto{\pgfqpoint{4.920734in}{2.107252in}}%
\pgfpathlineto{\pgfqpoint{4.929541in}{1.969992in}}%
\pgfpathlineto{\pgfqpoint{4.938347in}{2.038608in}}%
\pgfpathlineto{\pgfqpoint{4.947154in}{1.901348in}}%
\pgfpathlineto{\pgfqpoint{4.955961in}{1.743512in}}%
\pgfpathlineto{\pgfqpoint{4.964768in}{1.805269in}}%
\pgfpathlineto{\pgfqpoint{4.973575in}{1.805269in}}%
\pgfpathlineto{\pgfqpoint{4.982382in}{1.757229in}}%
\pgfpathlineto{\pgfqpoint{4.991188in}{1.626856in}}%
\pgfpathlineto{\pgfqpoint{4.999995in}{1.819015in}}%
\pgfpathlineto{\pgfqpoint{5.008802in}{1.770975in}}%
\pgfpathlineto{\pgfqpoint{5.017609in}{1.832732in}}%
\pgfpathlineto{\pgfqpoint{5.026416in}{1.537636in}}%
\pgfpathlineto{\pgfqpoint{5.035222in}{1.475878in}}%
\pgfpathlineto{\pgfqpoint{5.044029in}{1.544494in}}%
\pgfpathlineto{\pgfqpoint{5.052836in}{1.551353in}}%
\pgfpathlineto{\pgfqpoint{5.061643in}{1.619997in}}%
\pgfpathlineto{\pgfqpoint{5.070450in}{1.976851in}}%
\pgfpathlineto{\pgfqpoint{5.079257in}{1.915094in}}%
\pgfpathlineto{\pgfqpoint{5.088063in}{1.805269in}}%
\pgfpathlineto{\pgfqpoint{5.096870in}{2.072930in}}%
\pgfpathlineto{\pgfqpoint{5.114484in}{2.319987in}}%
\pgfpathlineto{\pgfqpoint{5.123291in}{2.100365in}}%
\pgfpathlineto{\pgfqpoint{5.132097in}{1.592534in}}%
\pgfpathlineto{\pgfqpoint{5.149711in}{1.633715in}}%
\pgfpathlineto{\pgfqpoint{5.158518in}{1.503314in}}%
\pgfpathlineto{\pgfqpoint{5.176132in}{1.619997in}}%
\pgfpathlineto{\pgfqpoint{5.184938in}{2.265089in}}%
\pgfpathlineto{\pgfqpoint{5.193745in}{2.196445in}}%
\pgfpathlineto{\pgfqpoint{5.202552in}{2.306269in}}%
\pgfpathlineto{\pgfqpoint{5.211359in}{2.292524in}}%
\pgfpathlineto{\pgfqpoint{5.220166in}{2.203303in}}%
\pgfpathlineto{\pgfqpoint{5.228972in}{2.059213in}}%
\pgfpathlineto{\pgfqpoint{5.246586in}{2.223908in}}%
\pgfpathlineto{\pgfqpoint{5.255393in}{2.333705in}}%
\pgfpathlineto{\pgfqpoint{5.264200in}{2.326846in}}%
\pgfpathlineto{\pgfqpoint{5.273007in}{2.409207in}}%
\pgfpathlineto{\pgfqpoint{5.281813in}{2.464106in}}%
\pgfpathlineto{\pgfqpoint{5.290620in}{2.450388in}}%
\pgfpathlineto{\pgfqpoint{5.299427in}{2.175868in}}%
\pgfpathlineto{\pgfqpoint{5.308234in}{2.072930in}}%
\pgfpathlineto{\pgfqpoint{5.317041in}{2.470965in}}%
\pgfpathlineto{\pgfqpoint{5.325847in}{2.169009in}}%
\pgfpathlineto{\pgfqpoint{5.334654in}{1.963133in}}%
\pgfpathlineto{\pgfqpoint{5.343461in}{1.812128in}}%
\pgfpathlineto{\pgfqpoint{5.352268in}{1.750370in}}%
\pgfpathlineto{\pgfqpoint{5.361075in}{1.414093in}}%
\pgfpathlineto{\pgfqpoint{5.369882in}{1.736653in}}%
\pgfpathlineto{\pgfqpoint{5.378688in}{2.018032in}}%
\pgfpathlineto{\pgfqpoint{5.387495in}{2.100365in}}%
\pgfpathlineto{\pgfqpoint{5.396302in}{1.921953in}}%
\pgfpathlineto{\pgfqpoint{5.405109in}{2.059213in}}%
\pgfpathlineto{\pgfqpoint{5.413916in}{2.018032in}}%
\pgfpathlineto{\pgfqpoint{5.422722in}{1.949388in}}%
\pgfpathlineto{\pgfqpoint{5.431529in}{2.169009in}}%
\pgfpathlineto{\pgfqpoint{5.440336in}{1.942529in}}%
\pgfpathlineto{\pgfqpoint{5.449143in}{1.798410in}}%
\pgfpathlineto{\pgfqpoint{5.457950in}{1.372912in}}%
\pgfpathlineto{\pgfqpoint{5.466757in}{1.523918in}}%
\pgfpathlineto{\pgfqpoint{5.475563in}{1.716076in}}%
\pgfpathlineto{\pgfqpoint{5.484370in}{1.613110in}}%
\pgfpathlineto{\pgfqpoint{5.493177in}{1.757229in}}%
\pgfpathlineto{\pgfqpoint{5.501984in}{2.127829in}}%
\pgfpathlineto{\pgfqpoint{5.510791in}{2.265089in}}%
\pgfpathlineto{\pgfqpoint{5.519597in}{2.018032in}}%
\pgfpathlineto{\pgfqpoint{5.537211in}{1.873913in}}%
\pgfpathlineto{\pgfqpoint{5.546018in}{1.709190in}}%
\pgfpathlineto{\pgfqpoint{5.554825in}{2.018032in}}%
\pgfpathlineto{\pgfqpoint{5.563632in}{1.956246in}}%
\pgfpathlineto{\pgfqpoint{5.572438in}{1.777834in}}%
\pgfpathlineto{\pgfqpoint{5.581245in}{1.846450in}}%
\pgfpathlineto{\pgfqpoint{5.590052in}{1.949388in}}%
\pgfpathlineto{\pgfqpoint{5.598859in}{2.011173in}}%
\pgfpathlineto{\pgfqpoint{5.607666in}{2.169009in}}%
\pgfpathlineto{\pgfqpoint{5.616472in}{2.155292in}}%
\pgfpathlineto{\pgfqpoint{5.625279in}{2.203303in}}%
\pgfpathlineto{\pgfqpoint{5.634086in}{1.880772in}}%
\pgfpathlineto{\pgfqpoint{5.642893in}{1.681754in}}%
\pgfpathlineto{\pgfqpoint{5.651700in}{1.832732in}}%
\pgfpathlineto{\pgfqpoint{5.660507in}{1.448415in}}%
\pgfpathlineto{\pgfqpoint{5.669313in}{1.462133in}}%
\pgfpathlineto{\pgfqpoint{5.678120in}{1.393517in}}%
\pgfpathlineto{\pgfqpoint{5.686927in}{1.750370in}}%
\pgfpathlineto{\pgfqpoint{5.695734in}{1.716076in}}%
\pgfpathlineto{\pgfqpoint{5.704541in}{1.846450in}}%
\pgfpathlineto{\pgfqpoint{5.713347in}{2.203303in}}%
\pgfpathlineto{\pgfqpoint{5.722154in}{2.210190in}}%
\pgfpathlineto{\pgfqpoint{5.739768in}{1.606252in}}%
\pgfpathlineto{\pgfqpoint{5.748575in}{1.558212in}}%
\pgfpathlineto{\pgfqpoint{5.757382in}{1.517031in}}%
\pgfpathlineto{\pgfqpoint{5.766188in}{1.770975in}}%
\pgfpathlineto{\pgfqpoint{5.774995in}{1.695472in}}%
\pgfpathlineto{\pgfqpoint{5.783802in}{1.551353in}}%
\pgfpathlineto{\pgfqpoint{5.792609in}{1.565071in}}%
\pgfpathlineto{\pgfqpoint{5.801416in}{1.764116in}}%
\pgfpathlineto{\pgfqpoint{5.810222in}{1.819015in}}%
\pgfpathlineto{\pgfqpoint{5.819029in}{1.798410in}}%
\pgfpathlineto{\pgfqpoint{5.827836in}{1.846450in}}%
\pgfpathlineto{\pgfqpoint{5.836643in}{1.832732in}}%
\pgfpathlineto{\pgfqpoint{5.845450in}{1.928811in}}%
\pgfpathlineto{\pgfqpoint{5.854257in}{2.086648in}}%
\pgfpathlineto{\pgfqpoint{5.863063in}{2.093507in}}%
\pgfpathlineto{\pgfqpoint{5.871870in}{2.038608in}}%
\pgfpathlineto{\pgfqpoint{5.880677in}{2.031749in}}%
\pgfpathlineto{\pgfqpoint{5.889484in}{1.722935in}}%
\pgfpathlineto{\pgfqpoint{5.898291in}{1.764116in}}%
\pgfpathlineto{\pgfqpoint{5.907097in}{1.729794in}}%
\pgfpathlineto{\pgfqpoint{5.915904in}{2.004286in}}%
\pgfpathlineto{\pgfqpoint{5.924711in}{1.873913in}}%
\pgfpathlineto{\pgfqpoint{5.933518in}{2.031749in}}%
\pgfpathlineto{\pgfqpoint{5.942325in}{2.141546in}}%
\pgfpathlineto{\pgfqpoint{5.951132in}{2.306269in}}%
\pgfpathlineto{\pgfqpoint{5.959938in}{2.052326in}}%
\pgfpathlineto{\pgfqpoint{5.968745in}{1.963133in}}%
\pgfpathlineto{\pgfqpoint{5.977552in}{1.860167in}}%
\pgfpathlineto{\pgfqpoint{5.986359in}{1.784692in}}%
\pgfpathlineto{\pgfqpoint{5.995166in}{1.764116in}}%
\pgfpathlineto{\pgfqpoint{6.003972in}{1.757229in}}%
\pgfpathlineto{\pgfqpoint{6.012779in}{1.571958in}}%
\pgfpathlineto{\pgfqpoint{6.021586in}{1.825873in}}%
\pgfpathlineto{\pgfqpoint{6.030393in}{2.175868in}}%
\pgfpathlineto{\pgfqpoint{6.039200in}{2.388603in}}%
\pgfpathlineto{\pgfqpoint{6.048007in}{2.395462in}}%
\pgfpathlineto{\pgfqpoint{6.056813in}{2.416066in}}%
\pgfpathlineto{\pgfqpoint{6.065620in}{2.210190in}}%
\pgfpathlineto{\pgfqpoint{6.083234in}{1.942529in}}%
\pgfpathlineto{\pgfqpoint{6.092041in}{1.990569in}}%
\pgfpathlineto{\pgfqpoint{6.100847in}{2.072930in}}%
\pgfpathlineto{\pgfqpoint{6.109654in}{2.059213in}}%
\pgfpathlineto{\pgfqpoint{6.127268in}{2.045467in}}%
\pgfpathlineto{\pgfqpoint{6.136075in}{2.011173in}}%
\pgfpathlineto{\pgfqpoint{6.144882in}{2.244484in}}%
\pgfpathlineto{\pgfqpoint{6.153688in}{2.052326in}}%
\pgfpathlineto{\pgfqpoint{6.162495in}{2.251343in}}%
\pgfpathlineto{\pgfqpoint{6.171302in}{1.990569in}}%
\pgfpathlineto{\pgfqpoint{6.180109in}{2.148405in}}%
\pgfpathlineto{\pgfqpoint{6.188916in}{2.086648in}}%
\pgfpathlineto{\pgfqpoint{6.197722in}{2.079789in}}%
\pgfpathlineto{\pgfqpoint{6.206529in}{1.887630in}}%
\pgfpathlineto{\pgfqpoint{6.215336in}{1.626856in}}%
\pgfpathlineto{\pgfqpoint{6.224143in}{2.114111in}}%
\pgfpathlineto{\pgfqpoint{6.232950in}{1.674896in}}%
\pgfpathlineto{\pgfqpoint{6.241757in}{1.544494in}}%
\pgfpathlineto{\pgfqpoint{6.250563in}{1.318014in}}%
\pgfpathlineto{\pgfqpoint{6.268177in}{2.470965in}}%
\pgfpathlineto{\pgfqpoint{6.276984in}{2.519004in}}%
\pgfpathlineto{\pgfqpoint{6.285791in}{2.306269in}}%
\pgfpathlineto{\pgfqpoint{6.294597in}{2.532722in}}%
\pgfpathlineto{\pgfqpoint{6.303404in}{1.928811in}}%
\pgfpathlineto{\pgfqpoint{6.312211in}{2.038608in}}%
\pgfpathlineto{\pgfqpoint{6.321018in}{2.072930in}}%
\pgfpathlineto{\pgfqpoint{6.329825in}{2.244484in}}%
\pgfpathlineto{\pgfqpoint{6.338632in}{2.155292in}}%
\pgfpathlineto{\pgfqpoint{6.347438in}{2.114111in}}%
\pgfpathlineto{\pgfqpoint{6.356245in}{2.114111in}}%
\pgfpathlineto{\pgfqpoint{6.365052in}{2.072930in}}%
\pgfpathlineto{\pgfqpoint{6.373859in}{2.004286in}}%
\pgfpathlineto{\pgfqpoint{6.382666in}{2.539581in}}%
\pgfpathlineto{\pgfqpoint{6.391472in}{2.512146in}}%
\pgfpathlineto{\pgfqpoint{6.400279in}{2.299383in}}%
\pgfpathlineto{\pgfqpoint{6.409086in}{2.134687in}}%
\pgfpathlineto{\pgfqpoint{6.417893in}{2.079789in}}%
\pgfpathlineto{\pgfqpoint{6.426700in}{1.812128in}}%
\pgfpathlineto{\pgfqpoint{6.435507in}{1.873913in}}%
\pgfpathlineto{\pgfqpoint{6.444313in}{1.688613in}}%
\pgfpathlineto{\pgfqpoint{6.453120in}{2.079789in}}%
\pgfpathlineto{\pgfqpoint{6.461927in}{2.388603in}}%
\pgfpathlineto{\pgfqpoint{6.479541in}{1.764116in}}%
\pgfpathlineto{\pgfqpoint{6.488347in}{1.942529in}}%
\pgfpathlineto{\pgfqpoint{6.497154in}{1.805269in}}%
\pgfpathlineto{\pgfqpoint{6.505961in}{1.777834in}}%
\pgfpathlineto{\pgfqpoint{6.514768in}{1.770975in}}%
\pgfpathlineto{\pgfqpoint{6.523575in}{1.606252in}}%
\pgfpathlineto{\pgfqpoint{6.532382in}{1.318014in}}%
\pgfpathlineto{\pgfqpoint{6.541188in}{1.619997in}}%
\pgfpathlineto{\pgfqpoint{6.549995in}{1.565071in}}%
\pgfpathlineto{\pgfqpoint{6.558802in}{1.880772in}}%
\pgfpathlineto{\pgfqpoint{6.567609in}{1.702331in}}%
\pgfpathlineto{\pgfqpoint{6.576416in}{1.716076in}}%
\pgfpathlineto{\pgfqpoint{6.585222in}{1.688613in}}%
\pgfpathlineto{\pgfqpoint{6.594029in}{2.052326in}}%
\pgfpathlineto{\pgfqpoint{6.602836in}{2.107252in}}%
\pgfpathlineto{\pgfqpoint{6.611643in}{2.347422in}}%
\pgfpathlineto{\pgfqpoint{6.620450in}{2.052326in}}%
\pgfpathlineto{\pgfqpoint{6.629257in}{1.956246in}}%
\pgfpathlineto{\pgfqpoint{6.638063in}{1.736653in}}%
\pgfpathlineto{\pgfqpoint{6.646870in}{2.464106in}}%
\pgfpathlineto{\pgfqpoint{6.655677in}{2.697445in}}%
\pgfpathlineto{\pgfqpoint{6.664484in}{2.470965in}}%
\pgfpathlineto{\pgfqpoint{6.673291in}{2.285665in}}%
\pgfpathlineto{\pgfqpoint{6.682097in}{2.621942in}}%
\pgfpathlineto{\pgfqpoint{6.690904in}{2.560185in}}%
\pgfpathlineto{\pgfqpoint{6.699711in}{2.292524in}}%
\pgfpathlineto{\pgfqpoint{6.708518in}{2.361168in}}%
\pgfpathlineto{\pgfqpoint{6.717325in}{2.525863in}}%
\pgfpathlineto{\pgfqpoint{6.726132in}{2.169009in}}%
\pgfpathlineto{\pgfqpoint{6.734938in}{1.983710in}}%
\pgfpathlineto{\pgfqpoint{6.743745in}{2.464106in}}%
\pgfpathlineto{\pgfqpoint{6.752552in}{2.265089in}}%
\pgfpathlineto{\pgfqpoint{6.761359in}{2.134687in}}%
\pgfpathlineto{\pgfqpoint{6.770166in}{2.175868in}}%
\pgfpathlineto{\pgfqpoint{6.778972in}{2.182727in}}%
\pgfpathlineto{\pgfqpoint{6.787779in}{2.265089in}}%
\pgfpathlineto{\pgfqpoint{6.796586in}{1.894489in}}%
\pgfpathlineto{\pgfqpoint{6.805393in}{2.429784in}}%
\pgfpathlineto{\pgfqpoint{6.814200in}{1.702331in}}%
\pgfpathlineto{\pgfqpoint{6.823007in}{1.853308in}}%
\pgfpathlineto{\pgfqpoint{6.831813in}{1.832732in}}%
\pgfpathlineto{\pgfqpoint{6.840620in}{1.798410in}}%
\pgfpathlineto{\pgfqpoint{6.858234in}{2.038608in}}%
\pgfpathlineto{\pgfqpoint{6.867041in}{1.983710in}}%
\pgfpathlineto{\pgfqpoint{6.875847in}{1.812128in}}%
\pgfpathlineto{\pgfqpoint{6.884654in}{1.928811in}}%
\pgfpathlineto{\pgfqpoint{6.893461in}{1.901348in}}%
\pgfpathlineto{\pgfqpoint{6.902268in}{1.716076in}}%
\pgfpathlineto{\pgfqpoint{6.911075in}{1.887630in}}%
\pgfpathlineto{\pgfqpoint{6.919882in}{2.011173in}}%
\pgfpathlineto{\pgfqpoint{6.928688in}{1.956246in}}%
\pgfpathlineto{\pgfqpoint{6.937495in}{1.695472in}}%
\pgfpathlineto{\pgfqpoint{6.946302in}{1.997427in}}%
\pgfpathlineto{\pgfqpoint{6.955109in}{1.743512in}}%
\pgfpathlineto{\pgfqpoint{6.963916in}{1.921953in}}%
\pgfpathlineto{\pgfqpoint{6.972722in}{1.894489in}}%
\pgfpathlineto{\pgfqpoint{6.981529in}{1.894489in}}%
\pgfpathlineto{\pgfqpoint{6.990336in}{1.887630in}}%
\pgfpathlineto{\pgfqpoint{7.007950in}{1.414093in}}%
\pgfpathlineto{\pgfqpoint{7.016757in}{1.496455in}}%
\pgfpathlineto{\pgfqpoint{7.025563in}{1.860167in}}%
\pgfpathlineto{\pgfqpoint{7.034370in}{1.832732in}}%
\pgfpathlineto{\pgfqpoint{7.043177in}{2.045467in}}%
\pgfpathlineto{\pgfqpoint{7.051984in}{1.743512in}}%
\pgfpathlineto{\pgfqpoint{7.060791in}{1.764116in}}%
\pgfpathlineto{\pgfqpoint{7.069597in}{1.647432in}}%
\pgfpathlineto{\pgfqpoint{7.078404in}{2.024891in}}%
\pgfpathlineto{\pgfqpoint{7.096018in}{2.141546in}}%
\pgfpathlineto{\pgfqpoint{7.104825in}{2.265089in}}%
\pgfpathlineto{\pgfqpoint{7.113632in}{2.429784in}}%
\pgfpathlineto{\pgfqpoint{7.122438in}{2.457247in}}%
\pgfpathlineto{\pgfqpoint{7.131245in}{2.374885in}}%
\pgfpathlineto{\pgfqpoint{7.140052in}{2.374885in}}%
\pgfpathlineto{\pgfqpoint{7.148859in}{2.271947in}}%
\pgfpathlineto{\pgfqpoint{7.157666in}{1.805269in}}%
\pgfpathlineto{\pgfqpoint{7.166472in}{1.915094in}}%
\pgfpathlineto{\pgfqpoint{7.175279in}{1.832732in}}%
\pgfpathlineto{\pgfqpoint{7.184086in}{1.709190in}}%
\pgfpathlineto{\pgfqpoint{7.192893in}{1.716076in}}%
\pgfpathlineto{\pgfqpoint{7.201700in}{1.613110in}}%
\pgfpathlineto{\pgfqpoint{7.210507in}{2.100365in}}%
\pgfpathlineto{\pgfqpoint{7.219313in}{2.285665in}}%
\pgfpathlineto{\pgfqpoint{7.228120in}{2.079789in}}%
\pgfpathlineto{\pgfqpoint{7.236927in}{1.990569in}}%
\pgfpathlineto{\pgfqpoint{7.245734in}{1.942529in}}%
\pgfpathlineto{\pgfqpoint{7.263347in}{1.215076in}}%
\pgfpathlineto{\pgfqpoint{7.272154in}{2.134687in}}%
\pgfpathlineto{\pgfqpoint{7.280961in}{2.162151in}}%
\pgfpathlineto{\pgfqpoint{7.289768in}{2.182727in}}%
\pgfpathlineto{\pgfqpoint{7.298575in}{2.162151in}}%
\pgfpathlineto{\pgfqpoint{7.307382in}{2.114111in}}%
\pgfpathlineto{\pgfqpoint{7.316188in}{2.175868in}}%
\pgfpathlineto{\pgfqpoint{7.324995in}{2.155292in}}%
\pgfpathlineto{\pgfqpoint{7.333802in}{2.114111in}}%
\pgfpathlineto{\pgfqpoint{7.342609in}{2.244484in}}%
\pgfpathlineto{\pgfqpoint{7.351416in}{2.464106in}}%
\pgfpathlineto{\pgfqpoint{7.360222in}{1.976851in}}%
\pgfpathlineto{\pgfqpoint{7.369029in}{1.729794in}}%
\pgfpathlineto{\pgfqpoint{7.377836in}{1.640574in}}%
\pgfpathlineto{\pgfqpoint{7.386643in}{1.867054in}}%
\pgfpathlineto{\pgfqpoint{7.395450in}{1.963133in}}%
\pgfpathlineto{\pgfqpoint{7.404257in}{1.750370in}}%
\pgfpathlineto{\pgfqpoint{7.413063in}{1.283720in}}%
\pgfpathlineto{\pgfqpoint{7.421870in}{1.345477in}}%
\pgfpathlineto{\pgfqpoint{7.430677in}{1.338618in}}%
\pgfpathlineto{\pgfqpoint{7.439484in}{1.215076in}}%
\pgfpathlineto{\pgfqpoint{7.448291in}{1.482737in}}%
\pgfpathlineto{\pgfqpoint{7.457097in}{1.414093in}}%
\pgfpathlineto{\pgfqpoint{7.465904in}{1.194499in}}%
\pgfpathlineto{\pgfqpoint{7.474711in}{1.352336in}}%
\pgfpathlineto{\pgfqpoint{7.483518in}{1.729794in}}%
\pgfpathlineto{\pgfqpoint{7.492325in}{1.647432in}}%
\pgfpathlineto{\pgfqpoint{7.509938in}{2.072930in}}%
\pgfpathlineto{\pgfqpoint{7.518745in}{1.949388in}}%
\pgfpathlineto{\pgfqpoint{7.527552in}{1.510172in}}%
\pgfpathlineto{\pgfqpoint{7.536359in}{1.770975in}}%
\pgfpathlineto{\pgfqpoint{7.545166in}{1.674896in}}%
\pgfpathlineto{\pgfqpoint{7.553972in}{1.462133in}}%
\pgfpathlineto{\pgfqpoint{7.562779in}{1.400376in}}%
\pgfpathlineto{\pgfqpoint{7.580393in}{1.613110in}}%
\pgfpathlineto{\pgfqpoint{7.589200in}{1.619997in}}%
\pgfpathlineto{\pgfqpoint{7.598007in}{1.448415in}}%
\pgfpathlineto{\pgfqpoint{7.606813in}{1.345477in}}%
\pgfpathlineto{\pgfqpoint{7.615620in}{1.558212in}}%
\pgfpathlineto{\pgfqpoint{7.624427in}{1.983710in}}%
\pgfpathlineto{\pgfqpoint{7.633234in}{2.162151in}}%
\pgfpathlineto{\pgfqpoint{7.642041in}{1.997427in}}%
\pgfpathlineto{\pgfqpoint{7.650847in}{1.400376in}}%
\pgfpathlineto{\pgfqpoint{7.659654in}{1.825873in}}%
\pgfpathlineto{\pgfqpoint{7.668461in}{1.969992in}}%
\pgfpathlineto{\pgfqpoint{7.677268in}{1.709190in}}%
\pgfpathlineto{\pgfqpoint{7.686075in}{1.825873in}}%
\pgfpathlineto{\pgfqpoint{7.694882in}{2.004286in}}%
\pgfpathlineto{\pgfqpoint{7.703688in}{1.791551in}}%
\pgfpathlineto{\pgfqpoint{7.712495in}{1.887630in}}%
\pgfpathlineto{\pgfqpoint{7.721302in}{1.963133in}}%
\pgfpathlineto{\pgfqpoint{7.730109in}{1.956246in}}%
\pgfpathlineto{\pgfqpoint{7.738916in}{1.523918in}}%
\pgfpathlineto{\pgfqpoint{7.747722in}{2.024891in}}%
\pgfpathlineto{\pgfqpoint{7.756529in}{2.031749in}}%
\pgfpathlineto{\pgfqpoint{7.765336in}{1.997427in}}%
\pgfpathlineto{\pgfqpoint{7.782950in}{1.846450in}}%
\pgfpathlineto{\pgfqpoint{7.791757in}{1.832732in}}%
\pgfpathlineto{\pgfqpoint{7.800563in}{1.846450in}}%
\pgfpathlineto{\pgfqpoint{7.809370in}{2.004286in}}%
\pgfpathlineto{\pgfqpoint{7.818177in}{2.059213in}}%
\pgfpathlineto{\pgfqpoint{7.826984in}{1.969992in}}%
\pgfpathlineto{\pgfqpoint{7.835791in}{2.038608in}}%
\pgfpathlineto{\pgfqpoint{7.844597in}{1.784692in}}%
\pgfpathlineto{\pgfqpoint{7.853404in}{1.921953in}}%
\pgfpathlineto{\pgfqpoint{7.862211in}{1.921953in}}%
\pgfpathlineto{\pgfqpoint{7.871018in}{2.100365in}}%
\pgfpathlineto{\pgfqpoint{7.879825in}{1.935670in}}%
\pgfpathlineto{\pgfqpoint{7.888632in}{1.908207in}}%
\pgfpathlineto{\pgfqpoint{7.897438in}{1.489596in}}%
\pgfpathlineto{\pgfqpoint{7.906245in}{1.352336in}}%
\pgfpathlineto{\pgfqpoint{7.915052in}{1.709190in}}%
\pgfpathlineto{\pgfqpoint{7.923859in}{1.606252in}}%
\pgfpathlineto{\pgfqpoint{7.932666in}{1.743512in}}%
\pgfpathlineto{\pgfqpoint{7.941472in}{1.517031in}}%
\pgfpathlineto{\pgfqpoint{7.950279in}{1.805269in}}%
\pgfpathlineto{\pgfqpoint{7.959086in}{1.722935in}}%
\pgfpathlineto{\pgfqpoint{7.967893in}{1.743512in}}%
\pgfpathlineto{\pgfqpoint{7.976700in}{1.784692in}}%
\pgfpathlineto{\pgfqpoint{7.985507in}{2.120970in}}%
\pgfpathlineto{\pgfqpoint{7.994313in}{1.908207in}}%
\pgfpathlineto{\pgfqpoint{8.003120in}{2.120970in}}%
\pgfpathlineto{\pgfqpoint{8.011927in}{1.736653in}}%
\pgfpathlineto{\pgfqpoint{8.020734in}{1.716076in}}%
\pgfpathlineto{\pgfqpoint{8.038347in}{2.072930in}}%
\pgfpathlineto{\pgfqpoint{8.047154in}{1.716076in}}%
\pgfpathlineto{\pgfqpoint{8.055961in}{1.716076in}}%
\pgfpathlineto{\pgfqpoint{8.064768in}{1.345477in}}%
\pgfpathlineto{\pgfqpoint{8.073575in}{1.462133in}}%
\pgfpathlineto{\pgfqpoint{8.082382in}{1.400376in}}%
\pgfpathlineto{\pgfqpoint{8.091188in}{1.242539in}}%
\pgfpathlineto{\pgfqpoint{8.099995in}{1.263115in}}%
\pgfpathlineto{\pgfqpoint{8.108802in}{1.393517in}}%
\pgfpathlineto{\pgfqpoint{8.117609in}{1.757229in}}%
\pgfpathlineto{\pgfqpoint{8.126416in}{1.949388in}}%
\pgfpathlineto{\pgfqpoint{8.135222in}{1.853308in}}%
\pgfpathlineto{\pgfqpoint{8.144029in}{2.011173in}}%
\pgfpathlineto{\pgfqpoint{8.152836in}{2.210190in}}%
\pgfpathlineto{\pgfqpoint{8.161643in}{2.539581in}}%
\pgfpathlineto{\pgfqpoint{8.170450in}{2.532722in}}%
\pgfpathlineto{\pgfqpoint{8.179257in}{2.086648in}}%
\pgfpathlineto{\pgfqpoint{8.196870in}{1.647432in}}%
\pgfpathlineto{\pgfqpoint{8.214484in}{2.045467in}}%
\pgfpathlineto{\pgfqpoint{8.223291in}{1.853308in}}%
\pgfpathlineto{\pgfqpoint{8.240904in}{2.333705in}}%
\pgfpathlineto{\pgfqpoint{8.249711in}{2.079789in}}%
\pgfpathlineto{\pgfqpoint{8.258518in}{1.935670in}}%
\pgfpathlineto{\pgfqpoint{8.267325in}{2.086648in}}%
\pgfpathlineto{\pgfqpoint{8.276132in}{1.935670in}}%
\pgfpathlineto{\pgfqpoint{8.284938in}{2.120970in}}%
\pgfpathlineto{\pgfqpoint{8.293745in}{1.942529in}}%
\pgfpathlineto{\pgfqpoint{8.302552in}{1.825873in}}%
\pgfpathlineto{\pgfqpoint{8.311359in}{1.757229in}}%
\pgfpathlineto{\pgfqpoint{8.320166in}{1.812128in}}%
\pgfpathlineto{\pgfqpoint{8.328972in}{1.825873in}}%
\pgfpathlineto{\pgfqpoint{8.337779in}{1.647432in}}%
\pgfpathlineto{\pgfqpoint{8.346586in}{2.052326in}}%
\pgfpathlineto{\pgfqpoint{8.355393in}{1.908207in}}%
\pgfpathlineto{\pgfqpoint{8.364200in}{1.983710in}}%
\pgfpathlineto{\pgfqpoint{8.381813in}{1.867054in}}%
\pgfpathlineto{\pgfqpoint{8.390620in}{1.969992in}}%
\pgfpathlineto{\pgfqpoint{8.399427in}{2.134687in}}%
\pgfpathlineto{\pgfqpoint{8.408234in}{2.045467in}}%
\pgfpathlineto{\pgfqpoint{8.425847in}{1.819015in}}%
\pgfpathlineto{\pgfqpoint{8.434654in}{1.819015in}}%
\pgfpathlineto{\pgfqpoint{8.443461in}{2.141546in}}%
\pgfpathlineto{\pgfqpoint{8.452268in}{2.066071in}}%
\pgfpathlineto{\pgfqpoint{8.461075in}{2.169009in}}%
\pgfpathlineto{\pgfqpoint{8.469882in}{1.846450in}}%
\pgfpathlineto{\pgfqpoint{8.478688in}{1.880772in}}%
\pgfpathlineto{\pgfqpoint{8.487495in}{2.079789in}}%
\pgfpathlineto{\pgfqpoint{8.496302in}{1.894489in}}%
\pgfpathlineto{\pgfqpoint{8.505109in}{1.832732in}}%
\pgfpathlineto{\pgfqpoint{8.513916in}{1.976851in}}%
\pgfpathlineto{\pgfqpoint{8.522722in}{1.626856in}}%
\pgfpathlineto{\pgfqpoint{8.531529in}{1.366053in}}%
\pgfpathlineto{\pgfqpoint{8.540336in}{1.551353in}}%
\pgfpathlineto{\pgfqpoint{8.557950in}{1.867054in}}%
\pgfpathlineto{\pgfqpoint{8.566757in}{1.908207in}}%
\pgfpathlineto{\pgfqpoint{8.575563in}{1.688613in}}%
\pgfpathlineto{\pgfqpoint{8.584370in}{1.674896in}}%
\pgfpathlineto{\pgfqpoint{8.593177in}{1.578816in}}%
\pgfpathlineto{\pgfqpoint{8.601984in}{1.626856in}}%
\pgfpathlineto{\pgfqpoint{8.610791in}{1.969992in}}%
\pgfpathlineto{\pgfqpoint{8.619597in}{1.887630in}}%
\pgfpathlineto{\pgfqpoint{8.628404in}{1.901348in}}%
\pgfpathlineto{\pgfqpoint{8.637211in}{1.757229in}}%
\pgfpathlineto{\pgfqpoint{8.646018in}{1.819015in}}%
\pgfpathlineto{\pgfqpoint{8.654825in}{1.729794in}}%
\pgfpathlineto{\pgfqpoint{8.663632in}{1.537636in}}%
\pgfpathlineto{\pgfqpoint{8.672438in}{1.613110in}}%
\pgfpathlineto{\pgfqpoint{8.681245in}{1.743512in}}%
\pgfpathlineto{\pgfqpoint{8.698859in}{2.038608in}}%
\pgfpathlineto{\pgfqpoint{8.707666in}{1.729794in}}%
\pgfpathlineto{\pgfqpoint{8.716472in}{1.585675in}}%
\pgfpathlineto{\pgfqpoint{8.725279in}{1.668037in}}%
\pgfpathlineto{\pgfqpoint{8.734086in}{1.956246in}}%
\pgfpathlineto{\pgfqpoint{8.742893in}{1.853308in}}%
\pgfpathlineto{\pgfqpoint{8.751700in}{1.949388in}}%
\pgfpathlineto{\pgfqpoint{8.760507in}{1.853308in}}%
\pgfpathlineto{\pgfqpoint{8.769313in}{1.702331in}}%
\pgfpathlineto{\pgfqpoint{8.778120in}{1.722935in}}%
\pgfpathlineto{\pgfqpoint{8.786927in}{1.599393in}}%
\pgfpathlineto{\pgfqpoint{8.795734in}{1.702331in}}%
\pgfpathlineto{\pgfqpoint{8.804541in}{1.764116in}}%
\pgfpathlineto{\pgfqpoint{8.813347in}{1.805269in}}%
\pgfpathlineto{\pgfqpoint{8.822154in}{2.024891in}}%
\pgfpathlineto{\pgfqpoint{8.830961in}{2.169009in}}%
\pgfpathlineto{\pgfqpoint{8.839768in}{2.361168in}}%
\pgfpathlineto{\pgfqpoint{8.848575in}{2.649406in}}%
\pgfpathlineto{\pgfqpoint{8.857382in}{2.340563in}}%
\pgfpathlineto{\pgfqpoint{8.866188in}{2.251343in}}%
\pgfpathlineto{\pgfqpoint{8.874995in}{2.258230in}}%
\pgfpathlineto{\pgfqpoint{8.883802in}{2.258230in}}%
\pgfpathlineto{\pgfqpoint{8.892609in}{2.319987in}}%
\pgfpathlineto{\pgfqpoint{8.901416in}{2.045467in}}%
\pgfpathlineto{\pgfqpoint{8.910222in}{2.045467in}}%
\pgfpathlineto{\pgfqpoint{8.927836in}{2.416066in}}%
\pgfpathlineto{\pgfqpoint{8.936643in}{2.333705in}}%
\pgfpathlineto{\pgfqpoint{8.945450in}{2.189586in}}%
\pgfpathlineto{\pgfqpoint{8.954257in}{2.354309in}}%
\pgfpathlineto{\pgfqpoint{8.963063in}{1.969992in}}%
\pgfpathlineto{\pgfqpoint{8.971870in}{2.004286in}}%
\pgfpathlineto{\pgfqpoint{8.980677in}{2.258230in}}%
\pgfpathlineto{\pgfqpoint{8.989484in}{2.114111in}}%
\pgfpathlineto{\pgfqpoint{8.998291in}{2.086648in}}%
\pgfpathlineto{\pgfqpoint{9.007097in}{1.915094in}}%
\pgfpathlineto{\pgfqpoint{9.015904in}{1.798410in}}%
\pgfpathlineto{\pgfqpoint{9.024711in}{2.004286in}}%
\pgfpathlineto{\pgfqpoint{9.033518in}{1.832732in}}%
\pgfpathlineto{\pgfqpoint{9.042325in}{1.969992in}}%
\pgfpathlineto{\pgfqpoint{9.051132in}{1.894489in}}%
\pgfpathlineto{\pgfqpoint{9.059938in}{1.969992in}}%
\pgfpathlineto{\pgfqpoint{9.068745in}{1.736653in}}%
\pgfpathlineto{\pgfqpoint{9.077552in}{1.805269in}}%
\pgfpathlineto{\pgfqpoint{9.086359in}{1.777834in}}%
\pgfpathlineto{\pgfqpoint{9.095166in}{1.887630in}}%
\pgfpathlineto{\pgfqpoint{9.103972in}{1.949388in}}%
\pgfpathlineto{\pgfqpoint{9.112779in}{1.915094in}}%
\pgfpathlineto{\pgfqpoint{9.121586in}{2.402349in}}%
\pgfpathlineto{\pgfqpoint{9.130393in}{2.532722in}}%
\pgfpathlineto{\pgfqpoint{9.139200in}{2.498428in}}%
\pgfpathlineto{\pgfqpoint{9.148007in}{2.786638in}}%
\pgfpathlineto{\pgfqpoint{9.156813in}{2.539581in}}%
\pgfpathlineto{\pgfqpoint{9.165620in}{2.395462in}}%
\pgfpathlineto{\pgfqpoint{9.174427in}{2.354309in}}%
\pgfpathlineto{\pgfqpoint{9.183234in}{2.210190in}}%
\pgfpathlineto{\pgfqpoint{9.192041in}{2.237625in}}%
\pgfpathlineto{\pgfqpoint{9.200847in}{2.519004in}}%
\pgfpathlineto{\pgfqpoint{9.209654in}{2.271947in}}%
\pgfpathlineto{\pgfqpoint{9.218461in}{1.990569in}}%
\pgfpathlineto{\pgfqpoint{9.227268in}{1.894489in}}%
\pgfpathlineto{\pgfqpoint{9.236075in}{1.606252in}}%
\pgfpathlineto{\pgfqpoint{9.244882in}{1.523918in}}%
\pgfpathlineto{\pgfqpoint{9.253688in}{1.393517in}}%
\pgfpathlineto{\pgfqpoint{9.262495in}{1.867054in}}%
\pgfpathlineto{\pgfqpoint{9.271302in}{2.100365in}}%
\pgfpathlineto{\pgfqpoint{9.280109in}{2.175868in}}%
\pgfpathlineto{\pgfqpoint{9.288916in}{2.292524in}}%
\pgfpathlineto{\pgfqpoint{9.297722in}{2.066071in}}%
\pgfpathlineto{\pgfqpoint{9.306529in}{2.031749in}}%
\pgfpathlineto{\pgfqpoint{9.315336in}{2.031749in}}%
\pgfpathlineto{\pgfqpoint{9.324143in}{2.230767in}}%
\pgfpathlineto{\pgfqpoint{9.332950in}{2.223908in}}%
\pgfpathlineto{\pgfqpoint{9.341757in}{2.189586in}}%
\pgfpathlineto{\pgfqpoint{9.350563in}{2.429784in}}%
\pgfpathlineto{\pgfqpoint{9.359370in}{2.196445in}}%
\pgfpathlineto{\pgfqpoint{9.368177in}{2.210190in}}%
\pgfpathlineto{\pgfqpoint{9.376984in}{1.832732in}}%
\pgfpathlineto{\pgfqpoint{9.385791in}{1.825873in}}%
\pgfpathlineto{\pgfqpoint{9.394597in}{1.674896in}}%
\pgfpathlineto{\pgfqpoint{9.403404in}{1.290579in}}%
\pgfpathlineto{\pgfqpoint{9.412211in}{1.386658in}}%
\pgfpathlineto{\pgfqpoint{9.421018in}{1.352336in}}%
\pgfpathlineto{\pgfqpoint{9.429825in}{1.352336in}}%
\pgfpathlineto{\pgfqpoint{9.438632in}{1.263115in}}%
\pgfpathlineto{\pgfqpoint{9.447438in}{1.352336in}}%
\pgfpathlineto{\pgfqpoint{9.456245in}{1.379799in}}%
\pgfpathlineto{\pgfqpoint{9.473859in}{1.853308in}}%
\pgfpathlineto{\pgfqpoint{9.482666in}{1.640574in}}%
\pgfpathlineto{\pgfqpoint{9.491472in}{1.860167in}}%
\pgfpathlineto{\pgfqpoint{9.500279in}{1.908207in}}%
\pgfpathlineto{\pgfqpoint{9.509086in}{1.976851in}}%
\pgfpathlineto{\pgfqpoint{9.517893in}{1.743512in}}%
\pgfpathlineto{\pgfqpoint{9.526700in}{2.011173in}}%
\pgfpathlineto{\pgfqpoint{9.535507in}{2.175868in}}%
\pgfpathlineto{\pgfqpoint{9.544313in}{2.155292in}}%
\pgfpathlineto{\pgfqpoint{9.553120in}{2.127829in}}%
\pgfpathlineto{\pgfqpoint{9.561927in}{1.976851in}}%
\pgfpathlineto{\pgfqpoint{9.570734in}{2.141546in}}%
\pgfpathlineto{\pgfqpoint{9.579541in}{1.976851in}}%
\pgfpathlineto{\pgfqpoint{9.597154in}{2.114111in}}%
\pgfpathlineto{\pgfqpoint{9.605961in}{1.956246in}}%
\pgfpathlineto{\pgfqpoint{9.614768in}{1.777834in}}%
\pgfpathlineto{\pgfqpoint{9.623575in}{1.736653in}}%
\pgfpathlineto{\pgfqpoint{9.632382in}{1.969992in}}%
\pgfpathlineto{\pgfqpoint{9.641188in}{1.908207in}}%
\pgfpathlineto{\pgfqpoint{9.649995in}{2.100365in}}%
\pgfpathlineto{\pgfqpoint{9.658802in}{1.867054in}}%
\pgfpathlineto{\pgfqpoint{9.667609in}{2.189586in}}%
\pgfpathlineto{\pgfqpoint{9.676416in}{2.347422in}}%
\pgfpathlineto{\pgfqpoint{9.685222in}{2.374885in}}%
\pgfpathlineto{\pgfqpoint{9.694029in}{2.086648in}}%
\pgfpathlineto{\pgfqpoint{9.702836in}{2.066071in}}%
\pgfpathlineto{\pgfqpoint{9.711643in}{2.258230in}}%
\pgfpathlineto{\pgfqpoint{9.720450in}{2.182727in}}%
\pgfpathlineto{\pgfqpoint{9.729257in}{2.169009in}}%
\pgfpathlineto{\pgfqpoint{9.738063in}{2.031749in}}%
\pgfpathlineto{\pgfqpoint{9.746870in}{1.558212in}}%
\pgfpathlineto{\pgfqpoint{9.755677in}{1.661150in}}%
\pgfpathlineto{\pgfqpoint{9.764484in}{1.661150in}}%
\pgfpathlineto{\pgfqpoint{9.773291in}{1.846450in}}%
\pgfpathlineto{\pgfqpoint{9.782097in}{1.928811in}}%
\pgfpathlineto{\pgfqpoint{9.790904in}{1.976851in}}%
\pgfpathlineto{\pgfqpoint{9.799711in}{2.079789in}}%
\pgfpathlineto{\pgfqpoint{9.808518in}{1.853308in}}%
\pgfpathlineto{\pgfqpoint{9.817325in}{1.750370in}}%
\pgfpathlineto{\pgfqpoint{9.826132in}{1.496455in}}%
\pgfpathlineto{\pgfqpoint{9.834938in}{1.283720in}}%
\pgfpathlineto{\pgfqpoint{9.852552in}{1.819015in}}%
\pgfpathlineto{\pgfqpoint{9.861359in}{1.530777in}}%
\pgfpathlineto{\pgfqpoint{9.870166in}{1.784692in}}%
\pgfpathlineto{\pgfqpoint{9.878972in}{1.880772in}}%
\pgfpathlineto{\pgfqpoint{9.887779in}{1.784692in}}%
\pgfpathlineto{\pgfqpoint{9.896586in}{1.949388in}}%
\pgfpathlineto{\pgfqpoint{9.914200in}{1.963133in}}%
\pgfpathlineto{\pgfqpoint{9.923007in}{1.894489in}}%
\pgfpathlineto{\pgfqpoint{9.931813in}{1.668037in}}%
\pgfpathlineto{\pgfqpoint{9.940620in}{1.908207in}}%
\pgfpathlineto{\pgfqpoint{9.949427in}{1.777834in}}%
\pgfpathlineto{\pgfqpoint{9.949427in}{1.777834in}}%
\pgfusepath{stroke}%
\end{pgfscope}%
\begin{pgfscope}%
\pgfpathrectangle{\pgfqpoint{0.702268in}{0.521603in}}{\pgfqpoint{9.687500in}{4.235000in}}%
\pgfusepath{clip}%
\pgfsetrectcap%
\pgfsetroundjoin%
\pgfsetlinewidth{0.501875pt}%
\definecolor{currentstroke}{rgb}{0.501961,0.501961,0.501961}%
\pgfsetstrokecolor{currentstroke}%
\pgfsetstrokeopacity{0.250000}%
\pgfsetdash{}{0pt}%
\pgfpathmoveto{\pgfqpoint{1.142609in}{4.248402in}}%
\pgfpathlineto{\pgfqpoint{1.151416in}{3.102338in}}%
\pgfpathlineto{\pgfqpoint{1.160222in}{2.553326in}}%
\pgfpathlineto{\pgfqpoint{1.169029in}{2.608225in}}%
\pgfpathlineto{\pgfqpoint{1.177836in}{2.580761in}}%
\pgfpathlineto{\pgfqpoint{1.186643in}{2.292524in}}%
\pgfpathlineto{\pgfqpoint{1.195450in}{2.148405in}}%
\pgfpathlineto{\pgfqpoint{1.204257in}{1.915094in}}%
\pgfpathlineto{\pgfqpoint{1.213063in}{1.633715in}}%
\pgfpathlineto{\pgfqpoint{1.221870in}{1.565071in}}%
\pgfpathlineto{\pgfqpoint{1.230677in}{1.619997in}}%
\pgfpathlineto{\pgfqpoint{1.239484in}{1.400376in}}%
\pgfpathlineto{\pgfqpoint{1.248291in}{1.407234in}}%
\pgfpathlineto{\pgfqpoint{1.257097in}{1.496455in}}%
\pgfpathlineto{\pgfqpoint{1.265904in}{1.599393in}}%
\pgfpathlineto{\pgfqpoint{1.274711in}{1.812128in}}%
\pgfpathlineto{\pgfqpoint{1.283518in}{1.551353in}}%
\pgfpathlineto{\pgfqpoint{1.292325in}{1.798410in}}%
\pgfpathlineto{\pgfqpoint{1.301132in}{1.757229in}}%
\pgfpathlineto{\pgfqpoint{1.309938in}{2.114111in}}%
\pgfpathlineto{\pgfqpoint{1.318745in}{1.880772in}}%
\pgfpathlineto{\pgfqpoint{1.327552in}{1.558212in}}%
\pgfpathlineto{\pgfqpoint{1.336359in}{1.448415in}}%
\pgfpathlineto{\pgfqpoint{1.345166in}{1.736653in}}%
\pgfpathlineto{\pgfqpoint{1.353972in}{1.619997in}}%
\pgfpathlineto{\pgfqpoint{1.362779in}{1.606252in}}%
\pgfpathlineto{\pgfqpoint{1.371586in}{1.647432in}}%
\pgfpathlineto{\pgfqpoint{1.380393in}{1.764116in}}%
\pgfpathlineto{\pgfqpoint{1.389200in}{2.093507in}}%
\pgfpathlineto{\pgfqpoint{1.398007in}{2.066071in}}%
\pgfpathlineto{\pgfqpoint{1.406813in}{2.114111in}}%
\pgfpathlineto{\pgfqpoint{1.415620in}{2.141546in}}%
\pgfpathlineto{\pgfqpoint{1.433234in}{1.338618in}}%
\pgfpathlineto{\pgfqpoint{1.442041in}{1.510172in}}%
\pgfpathlineto{\pgfqpoint{1.450847in}{1.420952in}}%
\pgfpathlineto{\pgfqpoint{1.459654in}{1.709190in}}%
\pgfpathlineto{\pgfqpoint{1.468461in}{1.249398in}}%
\pgfpathlineto{\pgfqpoint{1.477268in}{1.523918in}}%
\pgfpathlineto{\pgfqpoint{1.486075in}{2.024891in}}%
\pgfpathlineto{\pgfqpoint{1.494882in}{2.038608in}}%
\pgfpathlineto{\pgfqpoint{1.503688in}{2.299383in}}%
\pgfpathlineto{\pgfqpoint{1.512495in}{2.079789in}}%
\pgfpathlineto{\pgfqpoint{1.521302in}{2.601366in}}%
\pgfpathlineto{\pgfqpoint{1.530109in}{2.621942in}}%
\pgfpathlineto{\pgfqpoint{1.538916in}{2.402349in}}%
\pgfpathlineto{\pgfqpoint{1.547722in}{2.024891in}}%
\pgfpathlineto{\pgfqpoint{1.556529in}{2.100365in}}%
\pgfpathlineto{\pgfqpoint{1.565336in}{1.681754in}}%
\pgfpathlineto{\pgfqpoint{1.574143in}{1.921953in}}%
\pgfpathlineto{\pgfqpoint{1.582950in}{2.086648in}}%
\pgfpathlineto{\pgfqpoint{1.591757in}{2.120970in}}%
\pgfpathlineto{\pgfqpoint{1.600563in}{2.182727in}}%
\pgfpathlineto{\pgfqpoint{1.609370in}{1.791551in}}%
\pgfpathlineto{\pgfqpoint{1.618177in}{1.832732in}}%
\pgfpathlineto{\pgfqpoint{1.626984in}{2.004286in}}%
\pgfpathlineto{\pgfqpoint{1.635791in}{2.045467in}}%
\pgfpathlineto{\pgfqpoint{1.644597in}{1.969992in}}%
\pgfpathlineto{\pgfqpoint{1.653404in}{2.052326in}}%
\pgfpathlineto{\pgfqpoint{1.662211in}{2.381744in}}%
\pgfpathlineto{\pgfqpoint{1.671018in}{2.107252in}}%
\pgfpathlineto{\pgfqpoint{1.679825in}{2.066071in}}%
\pgfpathlineto{\pgfqpoint{1.688632in}{1.935670in}}%
\pgfpathlineto{\pgfqpoint{1.697438in}{2.018032in}}%
\pgfpathlineto{\pgfqpoint{1.706245in}{1.901348in}}%
\pgfpathlineto{\pgfqpoint{1.715052in}{2.230767in}}%
\pgfpathlineto{\pgfqpoint{1.723859in}{2.422925in}}%
\pgfpathlineto{\pgfqpoint{1.732666in}{2.011173in}}%
\pgfpathlineto{\pgfqpoint{1.741472in}{1.908207in}}%
\pgfpathlineto{\pgfqpoint{1.750279in}{1.928811in}}%
\pgfpathlineto{\pgfqpoint{1.759086in}{1.853308in}}%
\pgfpathlineto{\pgfqpoint{1.767893in}{1.812128in}}%
\pgfpathlineto{\pgfqpoint{1.776700in}{1.379799in}}%
\pgfpathlineto{\pgfqpoint{1.785507in}{1.530777in}}%
\pgfpathlineto{\pgfqpoint{1.794313in}{1.599393in}}%
\pgfpathlineto{\pgfqpoint{1.803120in}{1.626856in}}%
\pgfpathlineto{\pgfqpoint{1.811927in}{1.578816in}}%
\pgfpathlineto{\pgfqpoint{1.820734in}{1.846450in}}%
\pgfpathlineto{\pgfqpoint{1.829541in}{1.695472in}}%
\pgfpathlineto{\pgfqpoint{1.838347in}{1.819015in}}%
\pgfpathlineto{\pgfqpoint{1.847154in}{1.963133in}}%
\pgfpathlineto{\pgfqpoint{1.855961in}{2.196445in}}%
\pgfpathlineto{\pgfqpoint{1.864768in}{2.093507in}}%
\pgfpathlineto{\pgfqpoint{1.873575in}{2.237625in}}%
\pgfpathlineto{\pgfqpoint{1.882382in}{2.134687in}}%
\pgfpathlineto{\pgfqpoint{1.891188in}{2.230767in}}%
\pgfpathlineto{\pgfqpoint{1.899995in}{2.539581in}}%
\pgfpathlineto{\pgfqpoint{1.908802in}{2.182727in}}%
\pgfpathlineto{\pgfqpoint{1.917609in}{2.285665in}}%
\pgfpathlineto{\pgfqpoint{1.926416in}{2.333705in}}%
\pgfpathlineto{\pgfqpoint{1.935222in}{2.306269in}}%
\pgfpathlineto{\pgfqpoint{1.944029in}{2.004286in}}%
\pgfpathlineto{\pgfqpoint{1.952836in}{2.464106in}}%
\pgfpathlineto{\pgfqpoint{1.961643in}{2.539581in}}%
\pgfpathlineto{\pgfqpoint{1.970450in}{2.484682in}}%
\pgfpathlineto{\pgfqpoint{1.979257in}{2.127829in}}%
\pgfpathlineto{\pgfqpoint{1.988063in}{1.976851in}}%
\pgfpathlineto{\pgfqpoint{2.005677in}{1.750370in}}%
\pgfpathlineto{\pgfqpoint{2.014484in}{1.963133in}}%
\pgfpathlineto{\pgfqpoint{2.023291in}{1.873913in}}%
\pgfpathlineto{\pgfqpoint{2.032097in}{2.031749in}}%
\pgfpathlineto{\pgfqpoint{2.040904in}{2.079789in}}%
\pgfpathlineto{\pgfqpoint{2.049711in}{1.942529in}}%
\pgfpathlineto{\pgfqpoint{2.058518in}{2.127829in}}%
\pgfpathlineto{\pgfqpoint{2.067325in}{2.011173in}}%
\pgfpathlineto{\pgfqpoint{2.076132in}{2.319987in}}%
\pgfpathlineto{\pgfqpoint{2.084938in}{2.230767in}}%
\pgfpathlineto{\pgfqpoint{2.093745in}{2.052326in}}%
\pgfpathlineto{\pgfqpoint{2.102552in}{2.004286in}}%
\pgfpathlineto{\pgfqpoint{2.111359in}{1.928811in}}%
\pgfpathlineto{\pgfqpoint{2.120166in}{1.894489in}}%
\pgfpathlineto{\pgfqpoint{2.128972in}{2.045467in}}%
\pgfpathlineto{\pgfqpoint{2.137779in}{2.162151in}}%
\pgfpathlineto{\pgfqpoint{2.146586in}{1.867054in}}%
\pgfpathlineto{\pgfqpoint{2.155393in}{2.086648in}}%
\pgfpathlineto{\pgfqpoint{2.173007in}{1.887630in}}%
\pgfpathlineto{\pgfqpoint{2.181813in}{1.764116in}}%
\pgfpathlineto{\pgfqpoint{2.190620in}{1.873913in}}%
\pgfpathlineto{\pgfqpoint{2.199427in}{1.619997in}}%
\pgfpathlineto{\pgfqpoint{2.208234in}{1.530777in}}%
\pgfpathlineto{\pgfqpoint{2.217041in}{1.853308in}}%
\pgfpathlineto{\pgfqpoint{2.225847in}{1.894489in}}%
\pgfpathlineto{\pgfqpoint{2.234654in}{2.182727in}}%
\pgfpathlineto{\pgfqpoint{2.243461in}{1.976851in}}%
\pgfpathlineto{\pgfqpoint{2.252268in}{2.011173in}}%
\pgfpathlineto{\pgfqpoint{2.261075in}{2.594507in}}%
\pgfpathlineto{\pgfqpoint{2.269882in}{2.148405in}}%
\pgfpathlineto{\pgfqpoint{2.278688in}{1.928811in}}%
\pgfpathlineto{\pgfqpoint{2.287495in}{1.633715in}}%
\pgfpathlineto{\pgfqpoint{2.296302in}{1.647432in}}%
\pgfpathlineto{\pgfqpoint{2.305109in}{1.544494in}}%
\pgfpathlineto{\pgfqpoint{2.313916in}{1.482737in}}%
\pgfpathlineto{\pgfqpoint{2.331529in}{1.867054in}}%
\pgfpathlineto{\pgfqpoint{2.340336in}{2.120970in}}%
\pgfpathlineto{\pgfqpoint{2.357950in}{1.304296in}}%
\pgfpathlineto{\pgfqpoint{2.366757in}{1.482737in}}%
\pgfpathlineto{\pgfqpoint{2.375563in}{1.805269in}}%
\pgfpathlineto{\pgfqpoint{2.384370in}{1.949388in}}%
\pgfpathlineto{\pgfqpoint{2.393177in}{2.052326in}}%
\pgfpathlineto{\pgfqpoint{2.401984in}{2.100365in}}%
\pgfpathlineto{\pgfqpoint{2.410791in}{1.969992in}}%
\pgfpathlineto{\pgfqpoint{2.419597in}{1.997427in}}%
\pgfpathlineto{\pgfqpoint{2.437211in}{2.491541in}}%
\pgfpathlineto{\pgfqpoint{2.446018in}{2.368027in}}%
\pgfpathlineto{\pgfqpoint{2.454825in}{2.539581in}}%
\pgfpathlineto{\pgfqpoint{2.463632in}{2.217049in}}%
\pgfpathlineto{\pgfqpoint{2.472438in}{2.340563in}}%
\pgfpathlineto{\pgfqpoint{2.481245in}{2.107252in}}%
\pgfpathlineto{\pgfqpoint{2.490052in}{1.784692in}}%
\pgfpathlineto{\pgfqpoint{2.498859in}{2.004286in}}%
\pgfpathlineto{\pgfqpoint{2.507666in}{2.416066in}}%
\pgfpathlineto{\pgfqpoint{2.516472in}{2.333705in}}%
\pgfpathlineto{\pgfqpoint{2.525279in}{2.553326in}}%
\pgfpathlineto{\pgfqpoint{2.534086in}{2.477823in}}%
\pgfpathlineto{\pgfqpoint{2.542893in}{1.969992in}}%
\pgfpathlineto{\pgfqpoint{2.551700in}{2.182727in}}%
\pgfpathlineto{\pgfqpoint{2.569313in}{1.805269in}}%
\pgfpathlineto{\pgfqpoint{2.578120in}{1.894489in}}%
\pgfpathlineto{\pgfqpoint{2.586927in}{1.750370in}}%
\pgfpathlineto{\pgfqpoint{2.595734in}{1.915094in}}%
\pgfpathlineto{\pgfqpoint{2.604541in}{1.873913in}}%
\pgfpathlineto{\pgfqpoint{2.613347in}{1.777834in}}%
\pgfpathlineto{\pgfqpoint{2.622154in}{1.963133in}}%
\pgfpathlineto{\pgfqpoint{2.630961in}{1.901348in}}%
\pgfpathlineto{\pgfqpoint{2.639768in}{1.880772in}}%
\pgfpathlineto{\pgfqpoint{2.648575in}{1.523918in}}%
\pgfpathlineto{\pgfqpoint{2.657382in}{1.544494in}}%
\pgfpathlineto{\pgfqpoint{2.666188in}{1.517031in}}%
\pgfpathlineto{\pgfqpoint{2.674995in}{1.633715in}}%
\pgfpathlineto{\pgfqpoint{2.683802in}{1.722935in}}%
\pgfpathlineto{\pgfqpoint{2.692609in}{1.695472in}}%
\pgfpathlineto{\pgfqpoint{2.701416in}{1.873913in}}%
\pgfpathlineto{\pgfqpoint{2.710222in}{1.798410in}}%
\pgfpathlineto{\pgfqpoint{2.719029in}{1.901348in}}%
\pgfpathlineto{\pgfqpoint{2.727836in}{2.127829in}}%
\pgfpathlineto{\pgfqpoint{2.736643in}{2.072930in}}%
\pgfpathlineto{\pgfqpoint{2.745450in}{2.059213in}}%
\pgfpathlineto{\pgfqpoint{2.754257in}{2.100365in}}%
\pgfpathlineto{\pgfqpoint{2.763063in}{2.038608in}}%
\pgfpathlineto{\pgfqpoint{2.771870in}{2.052326in}}%
\pgfpathlineto{\pgfqpoint{2.780677in}{1.873913in}}%
\pgfpathlineto{\pgfqpoint{2.789484in}{1.757229in}}%
\pgfpathlineto{\pgfqpoint{2.798291in}{1.928811in}}%
\pgfpathlineto{\pgfqpoint{2.807097in}{1.592534in}}%
\pgfpathlineto{\pgfqpoint{2.815904in}{1.757229in}}%
\pgfpathlineto{\pgfqpoint{2.824711in}{1.819015in}}%
\pgfpathlineto{\pgfqpoint{2.833518in}{1.510172in}}%
\pgfpathlineto{\pgfqpoint{2.842325in}{1.558212in}}%
\pgfpathlineto{\pgfqpoint{2.851132in}{1.812128in}}%
\pgfpathlineto{\pgfqpoint{2.859938in}{1.908207in}}%
\pgfpathlineto{\pgfqpoint{2.868745in}{1.709190in}}%
\pgfpathlineto{\pgfqpoint{2.877552in}{1.757229in}}%
\pgfpathlineto{\pgfqpoint{2.895166in}{2.024891in}}%
\pgfpathlineto{\pgfqpoint{2.903972in}{2.024891in}}%
\pgfpathlineto{\pgfqpoint{2.912779in}{1.990569in}}%
\pgfpathlineto{\pgfqpoint{2.921586in}{1.873913in}}%
\pgfpathlineto{\pgfqpoint{2.930393in}{1.674896in}}%
\pgfpathlineto{\pgfqpoint{2.939200in}{1.736653in}}%
\pgfpathlineto{\pgfqpoint{2.948007in}{1.764116in}}%
\pgfpathlineto{\pgfqpoint{2.956813in}{1.908207in}}%
\pgfpathlineto{\pgfqpoint{2.965620in}{2.148405in}}%
\pgfpathlineto{\pgfqpoint{2.974427in}{1.901348in}}%
\pgfpathlineto{\pgfqpoint{2.983234in}{1.825873in}}%
\pgfpathlineto{\pgfqpoint{2.992041in}{1.716076in}}%
\pgfpathlineto{\pgfqpoint{3.000847in}{1.304296in}}%
\pgfpathlineto{\pgfqpoint{3.009654in}{1.489596in}}%
\pgfpathlineto{\pgfqpoint{3.018461in}{1.784692in}}%
\pgfpathlineto{\pgfqpoint{3.027268in}{1.764116in}}%
\pgfpathlineto{\pgfqpoint{3.036075in}{1.647432in}}%
\pgfpathlineto{\pgfqpoint{3.044882in}{1.798410in}}%
\pgfpathlineto{\pgfqpoint{3.053688in}{1.571958in}}%
\pgfpathlineto{\pgfqpoint{3.062495in}{1.736653in}}%
\pgfpathlineto{\pgfqpoint{3.071302in}{1.324873in}}%
\pgfpathlineto{\pgfqpoint{3.080109in}{1.510172in}}%
\pgfpathlineto{\pgfqpoint{3.088916in}{1.853308in}}%
\pgfpathlineto{\pgfqpoint{3.097722in}{2.114111in}}%
\pgfpathlineto{\pgfqpoint{3.106529in}{1.832732in}}%
\pgfpathlineto{\pgfqpoint{3.115336in}{2.162151in}}%
\pgfpathlineto{\pgfqpoint{3.124143in}{2.567044in}}%
\pgfpathlineto{\pgfqpoint{3.132950in}{2.436643in}}%
\pgfpathlineto{\pgfqpoint{3.141757in}{2.189586in}}%
\pgfpathlineto{\pgfqpoint{3.150563in}{1.839591in}}%
\pgfpathlineto{\pgfqpoint{3.159370in}{1.722935in}}%
\pgfpathlineto{\pgfqpoint{3.168177in}{1.736653in}}%
\pgfpathlineto{\pgfqpoint{3.176984in}{1.819015in}}%
\pgfpathlineto{\pgfqpoint{3.185791in}{1.921953in}}%
\pgfpathlineto{\pgfqpoint{3.194597in}{1.750370in}}%
\pgfpathlineto{\pgfqpoint{3.203404in}{1.867054in}}%
\pgfpathlineto{\pgfqpoint{3.212211in}{1.825873in}}%
\pgfpathlineto{\pgfqpoint{3.221018in}{2.258230in}}%
\pgfpathlineto{\pgfqpoint{3.229825in}{2.134687in}}%
\pgfpathlineto{\pgfqpoint{3.238632in}{2.162151in}}%
\pgfpathlineto{\pgfqpoint{3.247438in}{2.258230in}}%
\pgfpathlineto{\pgfqpoint{3.256245in}{2.155292in}}%
\pgfpathlineto{\pgfqpoint{3.265052in}{2.127829in}}%
\pgfpathlineto{\pgfqpoint{3.273859in}{2.155292in}}%
\pgfpathlineto{\pgfqpoint{3.282666in}{1.674896in}}%
\pgfpathlineto{\pgfqpoint{3.291472in}{2.278806in}}%
\pgfpathlineto{\pgfqpoint{3.300279in}{2.326846in}}%
\pgfpathlineto{\pgfqpoint{3.309086in}{2.422925in}}%
\pgfpathlineto{\pgfqpoint{3.317893in}{2.621942in}}%
\pgfpathlineto{\pgfqpoint{3.326700in}{2.182727in}}%
\pgfpathlineto{\pgfqpoint{3.335507in}{1.997427in}}%
\pgfpathlineto{\pgfqpoint{3.344313in}{2.045467in}}%
\pgfpathlineto{\pgfqpoint{3.353120in}{1.997427in}}%
\pgfpathlineto{\pgfqpoint{3.361927in}{2.107252in}}%
\pgfpathlineto{\pgfqpoint{3.370734in}{2.100365in}}%
\pgfpathlineto{\pgfqpoint{3.379541in}{2.608225in}}%
\pgfpathlineto{\pgfqpoint{3.388347in}{2.690558in}}%
\pgfpathlineto{\pgfqpoint{3.397154in}{2.299383in}}%
\pgfpathlineto{\pgfqpoint{3.405961in}{2.505287in}}%
\pgfpathlineto{\pgfqpoint{3.414768in}{2.189586in}}%
\pgfpathlineto{\pgfqpoint{3.423575in}{2.155292in}}%
\pgfpathlineto{\pgfqpoint{3.432382in}{1.873913in}}%
\pgfpathlineto{\pgfqpoint{3.441188in}{2.079789in}}%
\pgfpathlineto{\pgfqpoint{3.449995in}{2.018032in}}%
\pgfpathlineto{\pgfqpoint{3.458802in}{1.901348in}}%
\pgfpathlineto{\pgfqpoint{3.467609in}{1.942529in}}%
\pgfpathlineto{\pgfqpoint{3.476416in}{1.805269in}}%
\pgfpathlineto{\pgfqpoint{3.485222in}{1.565071in}}%
\pgfpathlineto{\pgfqpoint{3.494029in}{1.736653in}}%
\pgfpathlineto{\pgfqpoint{3.502836in}{1.846450in}}%
\pgfpathlineto{\pgfqpoint{3.511643in}{1.736653in}}%
\pgfpathlineto{\pgfqpoint{3.520450in}{1.894489in}}%
\pgfpathlineto{\pgfqpoint{3.529257in}{2.299383in}}%
\pgfpathlineto{\pgfqpoint{3.538063in}{2.052326in}}%
\pgfpathlineto{\pgfqpoint{3.546870in}{2.189586in}}%
\pgfpathlineto{\pgfqpoint{3.555677in}{1.963133in}}%
\pgfpathlineto{\pgfqpoint{3.564484in}{2.319987in}}%
\pgfpathlineto{\pgfqpoint{3.573291in}{2.134687in}}%
\pgfpathlineto{\pgfqpoint{3.582097in}{2.086648in}}%
\pgfpathlineto{\pgfqpoint{3.590904in}{1.880772in}}%
\pgfpathlineto{\pgfqpoint{3.599711in}{1.784692in}}%
\pgfpathlineto{\pgfqpoint{3.608518in}{1.592534in}}%
\pgfpathlineto{\pgfqpoint{3.617325in}{1.798410in}}%
\pgfpathlineto{\pgfqpoint{3.626132in}{1.716076in}}%
\pgfpathlineto{\pgfqpoint{3.634938in}{1.331760in}}%
\pgfpathlineto{\pgfqpoint{3.643745in}{1.414093in}}%
\pgfpathlineto{\pgfqpoint{3.652552in}{1.681754in}}%
\pgfpathlineto{\pgfqpoint{3.661359in}{1.695472in}}%
\pgfpathlineto{\pgfqpoint{3.670166in}{2.024891in}}%
\pgfpathlineto{\pgfqpoint{3.678972in}{1.901348in}}%
\pgfpathlineto{\pgfqpoint{3.687779in}{1.832732in}}%
\pgfpathlineto{\pgfqpoint{3.696586in}{1.976851in}}%
\pgfpathlineto{\pgfqpoint{3.705393in}{1.997427in}}%
\pgfpathlineto{\pgfqpoint{3.714200in}{1.997427in}}%
\pgfpathlineto{\pgfqpoint{3.723007in}{1.873913in}}%
\pgfpathlineto{\pgfqpoint{3.731813in}{1.976851in}}%
\pgfpathlineto{\pgfqpoint{3.740620in}{2.127829in}}%
\pgfpathlineto{\pgfqpoint{3.749427in}{2.443501in}}%
\pgfpathlineto{\pgfqpoint{3.758234in}{2.031749in}}%
\pgfpathlineto{\pgfqpoint{3.767041in}{2.148405in}}%
\pgfpathlineto{\pgfqpoint{3.775847in}{1.489596in}}%
\pgfpathlineto{\pgfqpoint{3.784654in}{1.372912in}}%
\pgfpathlineto{\pgfqpoint{3.793461in}{0.906262in}}%
\pgfpathlineto{\pgfqpoint{3.811075in}{1.475878in}}%
\pgfpathlineto{\pgfqpoint{3.819882in}{1.517031in}}%
\pgfpathlineto{\pgfqpoint{3.828688in}{1.743512in}}%
\pgfpathlineto{\pgfqpoint{3.837495in}{1.695472in}}%
\pgfpathlineto{\pgfqpoint{3.846302in}{1.819015in}}%
\pgfpathlineto{\pgfqpoint{3.863916in}{2.210190in}}%
\pgfpathlineto{\pgfqpoint{3.872722in}{1.983710in}}%
\pgfpathlineto{\pgfqpoint{3.881529in}{2.011173in}}%
\pgfpathlineto{\pgfqpoint{3.890336in}{2.031749in}}%
\pgfpathlineto{\pgfqpoint{3.899143in}{1.969992in}}%
\pgfpathlineto{\pgfqpoint{3.907950in}{1.819015in}}%
\pgfpathlineto{\pgfqpoint{3.916757in}{1.805269in}}%
\pgfpathlineto{\pgfqpoint{3.925563in}{1.846450in}}%
\pgfpathlineto{\pgfqpoint{3.934370in}{1.585675in}}%
\pgfpathlineto{\pgfqpoint{3.943177in}{1.722935in}}%
\pgfpathlineto{\pgfqpoint{3.951984in}{1.908207in}}%
\pgfpathlineto{\pgfqpoint{3.960791in}{2.223908in}}%
\pgfpathlineto{\pgfqpoint{3.969597in}{1.798410in}}%
\pgfpathlineto{\pgfqpoint{3.978404in}{1.805269in}}%
\pgfpathlineto{\pgfqpoint{3.987211in}{2.011173in}}%
\pgfpathlineto{\pgfqpoint{3.996018in}{2.100365in}}%
\pgfpathlineto{\pgfqpoint{4.004825in}{1.585675in}}%
\pgfpathlineto{\pgfqpoint{4.013632in}{1.647432in}}%
\pgfpathlineto{\pgfqpoint{4.022438in}{2.134687in}}%
\pgfpathlineto{\pgfqpoint{4.031245in}{1.791551in}}%
\pgfpathlineto{\pgfqpoint{4.040052in}{1.764116in}}%
\pgfpathlineto{\pgfqpoint{4.048859in}{1.839591in}}%
\pgfpathlineto{\pgfqpoint{4.057666in}{1.681754in}}%
\pgfpathlineto{\pgfqpoint{4.066472in}{1.468992in}}%
\pgfpathlineto{\pgfqpoint{4.075279in}{1.297438in}}%
\pgfpathlineto{\pgfqpoint{4.084086in}{1.269974in}}%
\pgfpathlineto{\pgfqpoint{4.092893in}{1.530777in}}%
\pgfpathlineto{\pgfqpoint{4.101700in}{1.654291in}}%
\pgfpathlineto{\pgfqpoint{4.110507in}{1.318014in}}%
\pgfpathlineto{\pgfqpoint{4.119313in}{1.619997in}}%
\pgfpathlineto{\pgfqpoint{4.128120in}{1.558212in}}%
\pgfpathlineto{\pgfqpoint{4.136927in}{1.702331in}}%
\pgfpathlineto{\pgfqpoint{4.145734in}{1.908207in}}%
\pgfpathlineto{\pgfqpoint{4.154541in}{1.887630in}}%
\pgfpathlineto{\pgfqpoint{4.163347in}{1.832732in}}%
\pgfpathlineto{\pgfqpoint{4.172154in}{1.983710in}}%
\pgfpathlineto{\pgfqpoint{4.180961in}{1.997427in}}%
\pgfpathlineto{\pgfqpoint{4.189768in}{1.894489in}}%
\pgfpathlineto{\pgfqpoint{4.198575in}{1.585675in}}%
\pgfpathlineto{\pgfqpoint{4.207382in}{1.668037in}}%
\pgfpathlineto{\pgfqpoint{4.216188in}{1.722935in}}%
\pgfpathlineto{\pgfqpoint{4.224995in}{1.654291in}}%
\pgfpathlineto{\pgfqpoint{4.233802in}{1.757229in}}%
\pgfpathlineto{\pgfqpoint{4.242609in}{1.969992in}}%
\pgfpathlineto{\pgfqpoint{4.251416in}{1.805269in}}%
\pgfpathlineto{\pgfqpoint{4.260222in}{2.148405in}}%
\pgfpathlineto{\pgfqpoint{4.269029in}{1.722935in}}%
\pgfpathlineto{\pgfqpoint{4.277836in}{1.777834in}}%
\pgfpathlineto{\pgfqpoint{4.286643in}{1.661150in}}%
\pgfpathlineto{\pgfqpoint{4.295450in}{1.613110in}}%
\pgfpathlineto{\pgfqpoint{4.304257in}{1.681754in}}%
\pgfpathlineto{\pgfqpoint{4.313063in}{1.915094in}}%
\pgfpathlineto{\pgfqpoint{4.321870in}{1.832732in}}%
\pgfpathlineto{\pgfqpoint{4.330677in}{1.921953in}}%
\pgfpathlineto{\pgfqpoint{4.339484in}{1.915094in}}%
\pgfpathlineto{\pgfqpoint{4.348291in}{2.532722in}}%
\pgfpathlineto{\pgfqpoint{4.357097in}{2.210190in}}%
\pgfpathlineto{\pgfqpoint{4.365904in}{2.134687in}}%
\pgfpathlineto{\pgfqpoint{4.374711in}{2.223908in}}%
\pgfpathlineto{\pgfqpoint{4.383518in}{2.162151in}}%
\pgfpathlineto{\pgfqpoint{4.392325in}{2.024891in}}%
\pgfpathlineto{\pgfqpoint{4.401132in}{1.812128in}}%
\pgfpathlineto{\pgfqpoint{4.409938in}{2.100365in}}%
\pgfpathlineto{\pgfqpoint{4.418745in}{2.285665in}}%
\pgfpathlineto{\pgfqpoint{4.427552in}{2.066071in}}%
\pgfpathlineto{\pgfqpoint{4.436359in}{2.148405in}}%
\pgfpathlineto{\pgfqpoint{4.445166in}{1.956246in}}%
\pgfpathlineto{\pgfqpoint{4.453972in}{2.134687in}}%
\pgfpathlineto{\pgfqpoint{4.462779in}{1.832732in}}%
\pgfpathlineto{\pgfqpoint{4.471586in}{1.832732in}}%
\pgfpathlineto{\pgfqpoint{4.480393in}{2.045467in}}%
\pgfpathlineto{\pgfqpoint{4.489200in}{1.860167in}}%
\pgfpathlineto{\pgfqpoint{4.498007in}{1.640574in}}%
\pgfpathlineto{\pgfqpoint{4.506813in}{1.750370in}}%
\pgfpathlineto{\pgfqpoint{4.515620in}{2.237625in}}%
\pgfpathlineto{\pgfqpoint{4.524427in}{2.416066in}}%
\pgfpathlineto{\pgfqpoint{4.533234in}{2.464106in}}%
\pgfpathlineto{\pgfqpoint{4.542041in}{2.169009in}}%
\pgfpathlineto{\pgfqpoint{4.550847in}{1.976851in}}%
\pgfpathlineto{\pgfqpoint{4.559654in}{2.038608in}}%
\pgfpathlineto{\pgfqpoint{4.568461in}{2.141546in}}%
\pgfpathlineto{\pgfqpoint{4.577268in}{2.306269in}}%
\pgfpathlineto{\pgfqpoint{4.586075in}{2.326846in}}%
\pgfpathlineto{\pgfqpoint{4.594882in}{1.791551in}}%
\pgfpathlineto{\pgfqpoint{4.603688in}{1.750370in}}%
\pgfpathlineto{\pgfqpoint{4.612495in}{1.688613in}}%
\pgfpathlineto{\pgfqpoint{4.621302in}{1.853308in}}%
\pgfpathlineto{\pgfqpoint{4.638916in}{1.709190in}}%
\pgfpathlineto{\pgfqpoint{4.647722in}{1.983710in}}%
\pgfpathlineto{\pgfqpoint{4.656529in}{2.326846in}}%
\pgfpathlineto{\pgfqpoint{4.665336in}{2.443501in}}%
\pgfpathlineto{\pgfqpoint{4.674143in}{2.628801in}}%
\pgfpathlineto{\pgfqpoint{4.682950in}{2.354309in}}%
\pgfpathlineto{\pgfqpoint{4.691757in}{2.505287in}}%
\pgfpathlineto{\pgfqpoint{4.700563in}{2.464106in}}%
\pgfpathlineto{\pgfqpoint{4.709370in}{2.409207in}}%
\pgfpathlineto{\pgfqpoint{4.718177in}{2.333705in}}%
\pgfpathlineto{\pgfqpoint{4.726984in}{2.059213in}}%
\pgfpathlineto{\pgfqpoint{4.735791in}{2.114111in}}%
\pgfpathlineto{\pgfqpoint{4.744597in}{1.956246in}}%
\pgfpathlineto{\pgfqpoint{4.753404in}{2.416066in}}%
\pgfpathlineto{\pgfqpoint{4.762211in}{2.374885in}}%
\pgfpathlineto{\pgfqpoint{4.771018in}{2.498428in}}%
\pgfpathlineto{\pgfqpoint{4.779825in}{2.326846in}}%
\pgfpathlineto{\pgfqpoint{4.788632in}{2.066071in}}%
\pgfpathlineto{\pgfqpoint{4.797438in}{2.052326in}}%
\pgfpathlineto{\pgfqpoint{4.806245in}{1.654291in}}%
\pgfpathlineto{\pgfqpoint{4.815052in}{1.674896in}}%
\pgfpathlineto{\pgfqpoint{4.823859in}{1.832732in}}%
\pgfpathlineto{\pgfqpoint{4.832666in}{1.770975in}}%
\pgfpathlineto{\pgfqpoint{4.841472in}{1.558212in}}%
\pgfpathlineto{\pgfqpoint{4.850279in}{1.709190in}}%
\pgfpathlineto{\pgfqpoint{4.859086in}{2.120970in}}%
\pgfpathlineto{\pgfqpoint{4.867893in}{2.223908in}}%
\pgfpathlineto{\pgfqpoint{4.876700in}{2.169009in}}%
\pgfpathlineto{\pgfqpoint{4.885507in}{2.313128in}}%
\pgfpathlineto{\pgfqpoint{4.894313in}{1.908207in}}%
\pgfpathlineto{\pgfqpoint{4.903120in}{2.120970in}}%
\pgfpathlineto{\pgfqpoint{4.911927in}{2.189586in}}%
\pgfpathlineto{\pgfqpoint{4.920734in}{2.107252in}}%
\pgfpathlineto{\pgfqpoint{4.938347in}{1.716076in}}%
\pgfpathlineto{\pgfqpoint{4.947154in}{1.928811in}}%
\pgfpathlineto{\pgfqpoint{4.955961in}{1.901348in}}%
\pgfpathlineto{\pgfqpoint{4.964768in}{1.880772in}}%
\pgfpathlineto{\pgfqpoint{4.973575in}{1.606252in}}%
\pgfpathlineto{\pgfqpoint{4.982382in}{1.949388in}}%
\pgfpathlineto{\pgfqpoint{4.991188in}{2.107252in}}%
\pgfpathlineto{\pgfqpoint{4.999995in}{2.285665in}}%
\pgfpathlineto{\pgfqpoint{5.008802in}{2.251343in}}%
\pgfpathlineto{\pgfqpoint{5.017609in}{2.512146in}}%
\pgfpathlineto{\pgfqpoint{5.026416in}{2.422925in}}%
\pgfpathlineto{\pgfqpoint{5.035222in}{2.422925in}}%
\pgfpathlineto{\pgfqpoint{5.044029in}{2.443501in}}%
\pgfpathlineto{\pgfqpoint{5.052836in}{2.244484in}}%
\pgfpathlineto{\pgfqpoint{5.070450in}{1.990569in}}%
\pgfpathlineto{\pgfqpoint{5.079257in}{2.223908in}}%
\pgfpathlineto{\pgfqpoint{5.088063in}{2.134687in}}%
\pgfpathlineto{\pgfqpoint{5.096870in}{1.722935in}}%
\pgfpathlineto{\pgfqpoint{5.105677in}{1.702331in}}%
\pgfpathlineto{\pgfqpoint{5.114484in}{1.640574in}}%
\pgfpathlineto{\pgfqpoint{5.123291in}{1.750370in}}%
\pgfpathlineto{\pgfqpoint{5.132097in}{1.928811in}}%
\pgfpathlineto{\pgfqpoint{5.140904in}{1.839591in}}%
\pgfpathlineto{\pgfqpoint{5.149711in}{1.935670in}}%
\pgfpathlineto{\pgfqpoint{5.158518in}{1.770975in}}%
\pgfpathlineto{\pgfqpoint{5.167325in}{1.887630in}}%
\pgfpathlineto{\pgfqpoint{5.176132in}{2.278806in}}%
\pgfpathlineto{\pgfqpoint{5.184938in}{1.640574in}}%
\pgfpathlineto{\pgfqpoint{5.193745in}{1.997427in}}%
\pgfpathlineto{\pgfqpoint{5.202552in}{1.853308in}}%
\pgfpathlineto{\pgfqpoint{5.211359in}{1.853308in}}%
\pgfpathlineto{\pgfqpoint{5.220166in}{1.777834in}}%
\pgfpathlineto{\pgfqpoint{5.228972in}{1.873913in}}%
\pgfpathlineto{\pgfqpoint{5.237779in}{1.887630in}}%
\pgfpathlineto{\pgfqpoint{5.246586in}{1.757229in}}%
\pgfpathlineto{\pgfqpoint{5.255393in}{1.770975in}}%
\pgfpathlineto{\pgfqpoint{5.264200in}{1.969992in}}%
\pgfpathlineto{\pgfqpoint{5.273007in}{2.265089in}}%
\pgfpathlineto{\pgfqpoint{5.281813in}{1.784692in}}%
\pgfpathlineto{\pgfqpoint{5.290620in}{2.031749in}}%
\pgfpathlineto{\pgfqpoint{5.299427in}{1.894489in}}%
\pgfpathlineto{\pgfqpoint{5.308234in}{2.100365in}}%
\pgfpathlineto{\pgfqpoint{5.317041in}{2.038608in}}%
\pgfpathlineto{\pgfqpoint{5.325847in}{1.709190in}}%
\pgfpathlineto{\pgfqpoint{5.334654in}{1.647432in}}%
\pgfpathlineto{\pgfqpoint{5.343461in}{1.935670in}}%
\pgfpathlineto{\pgfqpoint{5.352268in}{2.361168in}}%
\pgfpathlineto{\pgfqpoint{5.361075in}{2.299383in}}%
\pgfpathlineto{\pgfqpoint{5.369882in}{2.079789in}}%
\pgfpathlineto{\pgfqpoint{5.378688in}{1.633715in}}%
\pgfpathlineto{\pgfqpoint{5.387495in}{1.263115in}}%
\pgfpathlineto{\pgfqpoint{5.396302in}{1.434698in}}%
\pgfpathlineto{\pgfqpoint{5.405109in}{1.331760in}}%
\pgfpathlineto{\pgfqpoint{5.413916in}{1.757229in}}%
\pgfpathlineto{\pgfqpoint{5.422722in}{1.510172in}}%
\pgfpathlineto{\pgfqpoint{5.431529in}{1.420952in}}%
\pgfpathlineto{\pgfqpoint{5.440336in}{1.510172in}}%
\pgfpathlineto{\pgfqpoint{5.449143in}{1.448415in}}%
\pgfpathlineto{\pgfqpoint{5.457950in}{1.544494in}}%
\pgfpathlineto{\pgfqpoint{5.466757in}{1.523918in}}%
\pgfpathlineto{\pgfqpoint{5.475563in}{1.585675in}}%
\pgfpathlineto{\pgfqpoint{5.484370in}{1.517031in}}%
\pgfpathlineto{\pgfqpoint{5.493177in}{1.565071in}}%
\pgfpathlineto{\pgfqpoint{5.501984in}{1.345477in}}%
\pgfpathlineto{\pgfqpoint{5.510791in}{1.386658in}}%
\pgfpathlineto{\pgfqpoint{5.519597in}{1.695472in}}%
\pgfpathlineto{\pgfqpoint{5.528404in}{2.093507in}}%
\pgfpathlineto{\pgfqpoint{5.537211in}{2.052326in}}%
\pgfpathlineto{\pgfqpoint{5.546018in}{1.990569in}}%
\pgfpathlineto{\pgfqpoint{5.554825in}{2.285665in}}%
\pgfpathlineto{\pgfqpoint{5.563632in}{1.949388in}}%
\pgfpathlineto{\pgfqpoint{5.572438in}{1.956246in}}%
\pgfpathlineto{\pgfqpoint{5.581245in}{2.086648in}}%
\pgfpathlineto{\pgfqpoint{5.590052in}{1.736653in}}%
\pgfpathlineto{\pgfqpoint{5.598859in}{1.729794in}}%
\pgfpathlineto{\pgfqpoint{5.607666in}{1.880772in}}%
\pgfpathlineto{\pgfqpoint{5.616472in}{2.175868in}}%
\pgfpathlineto{\pgfqpoint{5.625279in}{2.532722in}}%
\pgfpathlineto{\pgfqpoint{5.634086in}{2.189586in}}%
\pgfpathlineto{\pgfqpoint{5.642893in}{2.141546in}}%
\pgfpathlineto{\pgfqpoint{5.651700in}{1.976851in}}%
\pgfpathlineto{\pgfqpoint{5.669313in}{1.990569in}}%
\pgfpathlineto{\pgfqpoint{5.678120in}{2.011173in}}%
\pgfpathlineto{\pgfqpoint{5.686927in}{2.127829in}}%
\pgfpathlineto{\pgfqpoint{5.695734in}{1.969992in}}%
\pgfpathlineto{\pgfqpoint{5.704541in}{2.100365in}}%
\pgfpathlineto{\pgfqpoint{5.713347in}{1.757229in}}%
\pgfpathlineto{\pgfqpoint{5.722154in}{1.681754in}}%
\pgfpathlineto{\pgfqpoint{5.730961in}{2.265089in}}%
\pgfpathlineto{\pgfqpoint{5.739768in}{2.251343in}}%
\pgfpathlineto{\pgfqpoint{5.748575in}{2.271947in}}%
\pgfpathlineto{\pgfqpoint{5.757382in}{2.189586in}}%
\pgfpathlineto{\pgfqpoint{5.766188in}{1.729794in}}%
\pgfpathlineto{\pgfqpoint{5.774995in}{1.688613in}}%
\pgfpathlineto{\pgfqpoint{5.783802in}{1.894489in}}%
\pgfpathlineto{\pgfqpoint{5.792609in}{1.537636in}}%
\pgfpathlineto{\pgfqpoint{5.801416in}{1.743512in}}%
\pgfpathlineto{\pgfqpoint{5.810222in}{2.059213in}}%
\pgfpathlineto{\pgfqpoint{5.819029in}{1.805269in}}%
\pgfpathlineto{\pgfqpoint{5.827836in}{2.072930in}}%
\pgfpathlineto{\pgfqpoint{5.836643in}{2.045467in}}%
\pgfpathlineto{\pgfqpoint{5.845450in}{2.004286in}}%
\pgfpathlineto{\pgfqpoint{5.854257in}{2.072930in}}%
\pgfpathlineto{\pgfqpoint{5.863063in}{1.956246in}}%
\pgfpathlineto{\pgfqpoint{5.871870in}{1.997427in}}%
\pgfpathlineto{\pgfqpoint{5.880677in}{2.210190in}}%
\pgfpathlineto{\pgfqpoint{5.889484in}{1.983710in}}%
\pgfpathlineto{\pgfqpoint{5.898291in}{2.100365in}}%
\pgfpathlineto{\pgfqpoint{5.907097in}{2.313128in}}%
\pgfpathlineto{\pgfqpoint{5.915904in}{2.120970in}}%
\pgfpathlineto{\pgfqpoint{5.924711in}{2.223908in}}%
\pgfpathlineto{\pgfqpoint{5.933518in}{2.265089in}}%
\pgfpathlineto{\pgfqpoint{5.942325in}{2.038608in}}%
\pgfpathlineto{\pgfqpoint{5.951132in}{1.873913in}}%
\pgfpathlineto{\pgfqpoint{5.959938in}{1.846450in}}%
\pgfpathlineto{\pgfqpoint{5.977552in}{1.613110in}}%
\pgfpathlineto{\pgfqpoint{5.986359in}{1.462133in}}%
\pgfpathlineto{\pgfqpoint{5.995166in}{1.517031in}}%
\pgfpathlineto{\pgfqpoint{6.003972in}{1.489596in}}%
\pgfpathlineto{\pgfqpoint{6.012779in}{1.359195in}}%
\pgfpathlineto{\pgfqpoint{6.021586in}{1.736653in}}%
\pgfpathlineto{\pgfqpoint{6.030393in}{1.640574in}}%
\pgfpathlineto{\pgfqpoint{6.039200in}{1.379799in}}%
\pgfpathlineto{\pgfqpoint{6.048007in}{1.619997in}}%
\pgfpathlineto{\pgfqpoint{6.056813in}{1.537636in}}%
\pgfpathlineto{\pgfqpoint{6.065620in}{1.770975in}}%
\pgfpathlineto{\pgfqpoint{6.074427in}{1.468992in}}%
\pgfpathlineto{\pgfqpoint{6.083234in}{1.873913in}}%
\pgfpathlineto{\pgfqpoint{6.092041in}{1.688613in}}%
\pgfpathlineto{\pgfqpoint{6.100847in}{1.942529in}}%
\pgfpathlineto{\pgfqpoint{6.109654in}{1.764116in}}%
\pgfpathlineto{\pgfqpoint{6.118461in}{1.702331in}}%
\pgfpathlineto{\pgfqpoint{6.127268in}{1.448415in}}%
\pgfpathlineto{\pgfqpoint{6.136075in}{1.654291in}}%
\pgfpathlineto{\pgfqpoint{6.144882in}{1.633715in}}%
\pgfpathlineto{\pgfqpoint{6.153688in}{1.770975in}}%
\pgfpathlineto{\pgfqpoint{6.162495in}{1.722935in}}%
\pgfpathlineto{\pgfqpoint{6.171302in}{1.825873in}}%
\pgfpathlineto{\pgfqpoint{6.180109in}{1.654291in}}%
\pgfpathlineto{\pgfqpoint{6.188916in}{1.681754in}}%
\pgfpathlineto{\pgfqpoint{6.197722in}{1.318014in}}%
\pgfpathlineto{\pgfqpoint{6.206529in}{1.427839in}}%
\pgfpathlineto{\pgfqpoint{6.215336in}{1.832732in}}%
\pgfpathlineto{\pgfqpoint{6.224143in}{1.976851in}}%
\pgfpathlineto{\pgfqpoint{6.232950in}{2.230767in}}%
\pgfpathlineto{\pgfqpoint{6.241757in}{1.949388in}}%
\pgfpathlineto{\pgfqpoint{6.250563in}{2.251343in}}%
\pgfpathlineto{\pgfqpoint{6.259370in}{2.621942in}}%
\pgfpathlineto{\pgfqpoint{6.268177in}{2.752344in}}%
\pgfpathlineto{\pgfqpoint{6.276984in}{2.642547in}}%
\pgfpathlineto{\pgfqpoint{6.285791in}{2.409207in}}%
\pgfpathlineto{\pgfqpoint{6.294597in}{2.258230in}}%
\pgfpathlineto{\pgfqpoint{6.303404in}{1.983710in}}%
\pgfpathlineto{\pgfqpoint{6.312211in}{2.072930in}}%
\pgfpathlineto{\pgfqpoint{6.321018in}{2.223908in}}%
\pgfpathlineto{\pgfqpoint{6.329825in}{2.059213in}}%
\pgfpathlineto{\pgfqpoint{6.338632in}{1.928811in}}%
\pgfpathlineto{\pgfqpoint{6.347438in}{1.764116in}}%
\pgfpathlineto{\pgfqpoint{6.365052in}{1.969992in}}%
\pgfpathlineto{\pgfqpoint{6.373859in}{1.846450in}}%
\pgfpathlineto{\pgfqpoint{6.382666in}{1.887630in}}%
\pgfpathlineto{\pgfqpoint{6.391472in}{1.791551in}}%
\pgfpathlineto{\pgfqpoint{6.400279in}{2.210190in}}%
\pgfpathlineto{\pgfqpoint{6.409086in}{2.189586in}}%
\pgfpathlineto{\pgfqpoint{6.417893in}{2.271947in}}%
\pgfpathlineto{\pgfqpoint{6.426700in}{2.045467in}}%
\pgfpathlineto{\pgfqpoint{6.435507in}{2.169009in}}%
\pgfpathlineto{\pgfqpoint{6.444313in}{1.805269in}}%
\pgfpathlineto{\pgfqpoint{6.453120in}{1.743512in}}%
\pgfpathlineto{\pgfqpoint{6.461927in}{1.764116in}}%
\pgfpathlineto{\pgfqpoint{6.470734in}{1.695472in}}%
\pgfpathlineto{\pgfqpoint{6.479541in}{1.915094in}}%
\pgfpathlineto{\pgfqpoint{6.488347in}{2.553326in}}%
\pgfpathlineto{\pgfqpoint{6.497154in}{2.278806in}}%
\pgfpathlineto{\pgfqpoint{6.505961in}{2.326846in}}%
\pgfpathlineto{\pgfqpoint{6.514768in}{2.271947in}}%
\pgfpathlineto{\pgfqpoint{6.523575in}{2.436643in}}%
\pgfpathlineto{\pgfqpoint{6.532382in}{2.203303in}}%
\pgfpathlineto{\pgfqpoint{6.541188in}{2.237625in}}%
\pgfpathlineto{\pgfqpoint{6.549995in}{2.306269in}}%
\pgfpathlineto{\pgfqpoint{6.558802in}{2.265089in}}%
\pgfpathlineto{\pgfqpoint{6.567609in}{1.846450in}}%
\pgfpathlineto{\pgfqpoint{6.585222in}{1.963133in}}%
\pgfpathlineto{\pgfqpoint{6.594029in}{2.244484in}}%
\pgfpathlineto{\pgfqpoint{6.602836in}{2.175868in}}%
\pgfpathlineto{\pgfqpoint{6.611643in}{2.031749in}}%
\pgfpathlineto{\pgfqpoint{6.620450in}{2.114111in}}%
\pgfpathlineto{\pgfqpoint{6.629257in}{2.079789in}}%
\pgfpathlineto{\pgfqpoint{6.638063in}{1.853308in}}%
\pgfpathlineto{\pgfqpoint{6.646870in}{1.750370in}}%
\pgfpathlineto{\pgfqpoint{6.655677in}{1.565071in}}%
\pgfpathlineto{\pgfqpoint{6.664484in}{1.523918in}}%
\pgfpathlineto{\pgfqpoint{6.673291in}{2.107252in}}%
\pgfpathlineto{\pgfqpoint{6.682097in}{2.052326in}}%
\pgfpathlineto{\pgfqpoint{6.690904in}{2.374885in}}%
\pgfpathlineto{\pgfqpoint{6.699711in}{1.969992in}}%
\pgfpathlineto{\pgfqpoint{6.708518in}{2.319987in}}%
\pgfpathlineto{\pgfqpoint{6.717325in}{1.997427in}}%
\pgfpathlineto{\pgfqpoint{6.726132in}{2.251343in}}%
\pgfpathlineto{\pgfqpoint{6.734938in}{2.134687in}}%
\pgfpathlineto{\pgfqpoint{6.743745in}{2.271947in}}%
\pgfpathlineto{\pgfqpoint{6.752552in}{2.326846in}}%
\pgfpathlineto{\pgfqpoint{6.761359in}{2.237625in}}%
\pgfpathlineto{\pgfqpoint{6.770166in}{2.175868in}}%
\pgfpathlineto{\pgfqpoint{6.778972in}{2.196445in}}%
\pgfpathlineto{\pgfqpoint{6.787779in}{2.223908in}}%
\pgfpathlineto{\pgfqpoint{6.796586in}{1.908207in}}%
\pgfpathlineto{\pgfqpoint{6.805393in}{1.935670in}}%
\pgfpathlineto{\pgfqpoint{6.814200in}{2.045467in}}%
\pgfpathlineto{\pgfqpoint{6.823007in}{2.107252in}}%
\pgfpathlineto{\pgfqpoint{6.831813in}{2.326846in}}%
\pgfpathlineto{\pgfqpoint{6.840620in}{2.395462in}}%
\pgfpathlineto{\pgfqpoint{6.849427in}{2.567044in}}%
\pgfpathlineto{\pgfqpoint{6.858234in}{2.546468in}}%
\pgfpathlineto{\pgfqpoint{6.867041in}{2.779779in}}%
\pgfpathlineto{\pgfqpoint{6.875847in}{2.587620in}}%
\pgfpathlineto{\pgfqpoint{6.884654in}{2.141546in}}%
\pgfpathlineto{\pgfqpoint{6.893461in}{1.969992in}}%
\pgfpathlineto{\pgfqpoint{6.902268in}{2.237625in}}%
\pgfpathlineto{\pgfqpoint{6.911075in}{2.045467in}}%
\pgfpathlineto{\pgfqpoint{6.919882in}{1.743512in}}%
\pgfpathlineto{\pgfqpoint{6.928688in}{1.873913in}}%
\pgfpathlineto{\pgfqpoint{6.937495in}{1.709190in}}%
\pgfpathlineto{\pgfqpoint{6.946302in}{1.654291in}}%
\pgfpathlineto{\pgfqpoint{6.955109in}{1.613110in}}%
\pgfpathlineto{\pgfqpoint{6.963916in}{1.880772in}}%
\pgfpathlineto{\pgfqpoint{6.972722in}{1.935670in}}%
\pgfpathlineto{\pgfqpoint{6.981529in}{1.619997in}}%
\pgfpathlineto{\pgfqpoint{6.990336in}{1.736653in}}%
\pgfpathlineto{\pgfqpoint{6.999143in}{2.024891in}}%
\pgfpathlineto{\pgfqpoint{7.007950in}{1.976851in}}%
\pgfpathlineto{\pgfqpoint{7.016757in}{1.805269in}}%
\pgfpathlineto{\pgfqpoint{7.025563in}{1.853308in}}%
\pgfpathlineto{\pgfqpoint{7.034370in}{2.079789in}}%
\pgfpathlineto{\pgfqpoint{7.043177in}{1.976851in}}%
\pgfpathlineto{\pgfqpoint{7.051984in}{2.038608in}}%
\pgfpathlineto{\pgfqpoint{7.060791in}{2.072930in}}%
\pgfpathlineto{\pgfqpoint{7.069597in}{2.004286in}}%
\pgfpathlineto{\pgfqpoint{7.078404in}{2.134687in}}%
\pgfpathlineto{\pgfqpoint{7.087211in}{2.120970in}}%
\pgfpathlineto{\pgfqpoint{7.096018in}{2.024891in}}%
\pgfpathlineto{\pgfqpoint{7.104825in}{2.066071in}}%
\pgfpathlineto{\pgfqpoint{7.113632in}{2.072930in}}%
\pgfpathlineto{\pgfqpoint{7.122438in}{1.722935in}}%
\pgfpathlineto{\pgfqpoint{7.131245in}{1.880772in}}%
\pgfpathlineto{\pgfqpoint{7.140052in}{1.915094in}}%
\pgfpathlineto{\pgfqpoint{7.148859in}{1.578816in}}%
\pgfpathlineto{\pgfqpoint{7.157666in}{1.626856in}}%
\pgfpathlineto{\pgfqpoint{7.166472in}{1.908207in}}%
\pgfpathlineto{\pgfqpoint{7.175279in}{1.709190in}}%
\pgfpathlineto{\pgfqpoint{7.184086in}{1.894489in}}%
\pgfpathlineto{\pgfqpoint{7.192893in}{1.702331in}}%
\pgfpathlineto{\pgfqpoint{7.201700in}{1.647432in}}%
\pgfpathlineto{\pgfqpoint{7.219313in}{1.853308in}}%
\pgfpathlineto{\pgfqpoint{7.228120in}{1.901348in}}%
\pgfpathlineto{\pgfqpoint{7.236927in}{1.983710in}}%
\pgfpathlineto{\pgfqpoint{7.245734in}{2.196445in}}%
\pgfpathlineto{\pgfqpoint{7.254541in}{2.120970in}}%
\pgfpathlineto{\pgfqpoint{7.263347in}{1.839591in}}%
\pgfpathlineto{\pgfqpoint{7.272154in}{2.141546in}}%
\pgfpathlineto{\pgfqpoint{7.280961in}{1.976851in}}%
\pgfpathlineto{\pgfqpoint{7.289768in}{2.093507in}}%
\pgfpathlineto{\pgfqpoint{7.298575in}{2.374885in}}%
\pgfpathlineto{\pgfqpoint{7.307382in}{2.299383in}}%
\pgfpathlineto{\pgfqpoint{7.316188in}{2.093507in}}%
\pgfpathlineto{\pgfqpoint{7.324995in}{2.285665in}}%
\pgfpathlineto{\pgfqpoint{7.333802in}{2.306269in}}%
\pgfpathlineto{\pgfqpoint{7.342609in}{2.532722in}}%
\pgfpathlineto{\pgfqpoint{7.351416in}{2.697445in}}%
\pgfpathlineto{\pgfqpoint{7.360222in}{2.182727in}}%
\pgfpathlineto{\pgfqpoint{7.369029in}{1.894489in}}%
\pgfpathlineto{\pgfqpoint{7.377836in}{1.784692in}}%
\pgfpathlineto{\pgfqpoint{7.386643in}{1.990569in}}%
\pgfpathlineto{\pgfqpoint{7.395450in}{1.606252in}}%
\pgfpathlineto{\pgfqpoint{7.404257in}{1.510172in}}%
\pgfpathlineto{\pgfqpoint{7.413063in}{1.551353in}}%
\pgfpathlineto{\pgfqpoint{7.421870in}{1.812128in}}%
\pgfpathlineto{\pgfqpoint{7.430677in}{1.558212in}}%
\pgfpathlineto{\pgfqpoint{7.439484in}{1.661150in}}%
\pgfpathlineto{\pgfqpoint{7.448291in}{1.585675in}}%
\pgfpathlineto{\pgfqpoint{7.457097in}{1.606252in}}%
\pgfpathlineto{\pgfqpoint{7.465904in}{1.709190in}}%
\pgfpathlineto{\pgfqpoint{7.474711in}{1.379799in}}%
\pgfpathlineto{\pgfqpoint{7.483518in}{1.565071in}}%
\pgfpathlineto{\pgfqpoint{7.492325in}{1.674896in}}%
\pgfpathlineto{\pgfqpoint{7.501132in}{2.237625in}}%
\pgfpathlineto{\pgfqpoint{7.509938in}{2.230767in}}%
\pgfpathlineto{\pgfqpoint{7.518745in}{2.189586in}}%
\pgfpathlineto{\pgfqpoint{7.527552in}{2.546468in}}%
\pgfpathlineto{\pgfqpoint{7.536359in}{2.361168in}}%
\pgfpathlineto{\pgfqpoint{7.545166in}{2.313128in}}%
\pgfpathlineto{\pgfqpoint{7.553972in}{2.251343in}}%
\pgfpathlineto{\pgfqpoint{7.562779in}{2.285665in}}%
\pgfpathlineto{\pgfqpoint{7.571586in}{1.839591in}}%
\pgfpathlineto{\pgfqpoint{7.580393in}{1.722935in}}%
\pgfpathlineto{\pgfqpoint{7.589200in}{1.860167in}}%
\pgfpathlineto{\pgfqpoint{7.598007in}{1.784692in}}%
\pgfpathlineto{\pgfqpoint{7.606813in}{2.148405in}}%
\pgfpathlineto{\pgfqpoint{7.615620in}{2.237625in}}%
\pgfpathlineto{\pgfqpoint{7.624427in}{2.059213in}}%
\pgfpathlineto{\pgfqpoint{7.633234in}{2.278806in}}%
\pgfpathlineto{\pgfqpoint{7.642041in}{2.278806in}}%
\pgfpathlineto{\pgfqpoint{7.650847in}{1.805269in}}%
\pgfpathlineto{\pgfqpoint{7.659654in}{1.976851in}}%
\pgfpathlineto{\pgfqpoint{7.668461in}{2.237625in}}%
\pgfpathlineto{\pgfqpoint{7.677268in}{2.169009in}}%
\pgfpathlineto{\pgfqpoint{7.686075in}{1.798410in}}%
\pgfpathlineto{\pgfqpoint{7.694882in}{1.729794in}}%
\pgfpathlineto{\pgfqpoint{7.703688in}{1.647432in}}%
\pgfpathlineto{\pgfqpoint{7.712495in}{1.736653in}}%
\pgfpathlineto{\pgfqpoint{7.721302in}{1.619997in}}%
\pgfpathlineto{\pgfqpoint{7.730109in}{1.613110in}}%
\pgfpathlineto{\pgfqpoint{7.738916in}{1.812128in}}%
\pgfpathlineto{\pgfqpoint{7.747722in}{1.764116in}}%
\pgfpathlineto{\pgfqpoint{7.756529in}{1.517031in}}%
\pgfpathlineto{\pgfqpoint{7.765336in}{1.805269in}}%
\pgfpathlineto{\pgfqpoint{7.774143in}{1.750370in}}%
\pgfpathlineto{\pgfqpoint{7.782950in}{1.647432in}}%
\pgfpathlineto{\pgfqpoint{7.791757in}{1.640574in}}%
\pgfpathlineto{\pgfqpoint{7.800563in}{2.107252in}}%
\pgfpathlineto{\pgfqpoint{7.809370in}{2.169009in}}%
\pgfpathlineto{\pgfqpoint{7.818177in}{2.079789in}}%
\pgfpathlineto{\pgfqpoint{7.826984in}{1.949388in}}%
\pgfpathlineto{\pgfqpoint{7.835791in}{2.120970in}}%
\pgfpathlineto{\pgfqpoint{7.844597in}{1.832732in}}%
\pgfpathlineto{\pgfqpoint{7.853404in}{1.407234in}}%
\pgfpathlineto{\pgfqpoint{7.862211in}{1.160177in}}%
\pgfpathlineto{\pgfqpoint{7.871018in}{1.242539in}}%
\pgfpathlineto{\pgfqpoint{7.879825in}{1.599393in}}%
\pgfpathlineto{\pgfqpoint{7.888632in}{1.750370in}}%
\pgfpathlineto{\pgfqpoint{7.897438in}{2.189586in}}%
\pgfpathlineto{\pgfqpoint{7.906245in}{1.867054in}}%
\pgfpathlineto{\pgfqpoint{7.915052in}{2.052326in}}%
\pgfpathlineto{\pgfqpoint{7.923859in}{2.093507in}}%
\pgfpathlineto{\pgfqpoint{7.932666in}{1.619997in}}%
\pgfpathlineto{\pgfqpoint{7.941472in}{1.668037in}}%
\pgfpathlineto{\pgfqpoint{7.950279in}{1.674896in}}%
\pgfpathlineto{\pgfqpoint{7.959086in}{1.613110in}}%
\pgfpathlineto{\pgfqpoint{7.967893in}{1.839591in}}%
\pgfpathlineto{\pgfqpoint{7.976700in}{1.668037in}}%
\pgfpathlineto{\pgfqpoint{7.985507in}{1.722935in}}%
\pgfpathlineto{\pgfqpoint{7.994313in}{1.757229in}}%
\pgfpathlineto{\pgfqpoint{8.003120in}{1.928811in}}%
\pgfpathlineto{\pgfqpoint{8.011927in}{1.873913in}}%
\pgfpathlineto{\pgfqpoint{8.020734in}{1.908207in}}%
\pgfpathlineto{\pgfqpoint{8.029541in}{2.031749in}}%
\pgfpathlineto{\pgfqpoint{8.038347in}{1.633715in}}%
\pgfpathlineto{\pgfqpoint{8.047154in}{1.873913in}}%
\pgfpathlineto{\pgfqpoint{8.055961in}{1.674896in}}%
\pgfpathlineto{\pgfqpoint{8.064768in}{1.887630in}}%
\pgfpathlineto{\pgfqpoint{8.073575in}{2.004286in}}%
\pgfpathlineto{\pgfqpoint{8.082382in}{2.333705in}}%
\pgfpathlineto{\pgfqpoint{8.091188in}{1.915094in}}%
\pgfpathlineto{\pgfqpoint{8.099995in}{1.722935in}}%
\pgfpathlineto{\pgfqpoint{8.108802in}{1.853308in}}%
\pgfpathlineto{\pgfqpoint{8.117609in}{1.599393in}}%
\pgfpathlineto{\pgfqpoint{8.126416in}{1.386658in}}%
\pgfpathlineto{\pgfqpoint{8.135222in}{1.283720in}}%
\pgfpathlineto{\pgfqpoint{8.144029in}{1.716076in}}%
\pgfpathlineto{\pgfqpoint{8.152836in}{1.915094in}}%
\pgfpathlineto{\pgfqpoint{8.161643in}{1.468992in}}%
\pgfpathlineto{\pgfqpoint{8.170450in}{2.052326in}}%
\pgfpathlineto{\pgfqpoint{8.179257in}{2.134687in}}%
\pgfpathlineto{\pgfqpoint{8.188063in}{1.462133in}}%
\pgfpathlineto{\pgfqpoint{8.196870in}{1.386658in}}%
\pgfpathlineto{\pgfqpoint{8.205677in}{1.468992in}}%
\pgfpathlineto{\pgfqpoint{8.214484in}{1.626856in}}%
\pgfpathlineto{\pgfqpoint{8.223291in}{1.729794in}}%
\pgfpathlineto{\pgfqpoint{8.232097in}{1.880772in}}%
\pgfpathlineto{\pgfqpoint{8.240904in}{1.825873in}}%
\pgfpathlineto{\pgfqpoint{8.249711in}{2.155292in}}%
\pgfpathlineto{\pgfqpoint{8.258518in}{1.921953in}}%
\pgfpathlineto{\pgfqpoint{8.267325in}{1.626856in}}%
\pgfpathlineto{\pgfqpoint{8.276132in}{1.482737in}}%
\pgfpathlineto{\pgfqpoint{8.284938in}{1.668037in}}%
\pgfpathlineto{\pgfqpoint{8.293745in}{1.249398in}}%
\pgfpathlineto{\pgfqpoint{8.302552in}{1.167036in}}%
\pgfpathlineto{\pgfqpoint{8.311359in}{1.249398in}}%
\pgfpathlineto{\pgfqpoint{8.320166in}{1.647432in}}%
\pgfpathlineto{\pgfqpoint{8.328972in}{1.654291in}}%
\pgfpathlineto{\pgfqpoint{8.337779in}{1.812128in}}%
\pgfpathlineto{\pgfqpoint{8.346586in}{1.681754in}}%
\pgfpathlineto{\pgfqpoint{8.355393in}{2.292524in}}%
\pgfpathlineto{\pgfqpoint{8.364200in}{2.498428in}}%
\pgfpathlineto{\pgfqpoint{8.373007in}{2.553326in}}%
\pgfpathlineto{\pgfqpoint{8.381813in}{2.381744in}}%
\pgfpathlineto{\pgfqpoint{8.390620in}{2.429784in}}%
\pgfpathlineto{\pgfqpoint{8.399427in}{1.832732in}}%
\pgfpathlineto{\pgfqpoint{8.408234in}{1.908207in}}%
\pgfpathlineto{\pgfqpoint{8.425847in}{2.011173in}}%
\pgfpathlineto{\pgfqpoint{8.434654in}{1.860167in}}%
\pgfpathlineto{\pgfqpoint{8.443461in}{2.038608in}}%
\pgfpathlineto{\pgfqpoint{8.452268in}{1.935670in}}%
\pgfpathlineto{\pgfqpoint{8.461075in}{1.668037in}}%
\pgfpathlineto{\pgfqpoint{8.469882in}{1.606252in}}%
\pgfpathlineto{\pgfqpoint{8.478688in}{1.338618in}}%
\pgfpathlineto{\pgfqpoint{8.487495in}{1.496455in}}%
\pgfpathlineto{\pgfqpoint{8.496302in}{1.736653in}}%
\pgfpathlineto{\pgfqpoint{8.505109in}{1.407234in}}%
\pgfpathlineto{\pgfqpoint{8.513916in}{1.359195in}}%
\pgfpathlineto{\pgfqpoint{8.531529in}{2.162151in}}%
\pgfpathlineto{\pgfqpoint{8.540336in}{1.873913in}}%
\pgfpathlineto{\pgfqpoint{8.549143in}{1.942529in}}%
\pgfpathlineto{\pgfqpoint{8.557950in}{2.230767in}}%
\pgfpathlineto{\pgfqpoint{8.566757in}{1.867054in}}%
\pgfpathlineto{\pgfqpoint{8.575563in}{1.853308in}}%
\pgfpathlineto{\pgfqpoint{8.584370in}{1.345477in}}%
\pgfpathlineto{\pgfqpoint{8.593177in}{1.311155in}}%
\pgfpathlineto{\pgfqpoint{8.601984in}{1.393517in}}%
\pgfpathlineto{\pgfqpoint{8.610791in}{1.640574in}}%
\pgfpathlineto{\pgfqpoint{8.619597in}{1.352336in}}%
\pgfpathlineto{\pgfqpoint{8.628404in}{1.764116in}}%
\pgfpathlineto{\pgfqpoint{8.637211in}{1.750370in}}%
\pgfpathlineto{\pgfqpoint{8.646018in}{1.654291in}}%
\pgfpathlineto{\pgfqpoint{8.654825in}{1.379799in}}%
\pgfpathlineto{\pgfqpoint{8.663632in}{1.688613in}}%
\pgfpathlineto{\pgfqpoint{8.672438in}{1.764116in}}%
\pgfpathlineto{\pgfqpoint{8.681245in}{1.853308in}}%
\pgfpathlineto{\pgfqpoint{8.690052in}{2.086648in}}%
\pgfpathlineto{\pgfqpoint{8.698859in}{1.928811in}}%
\pgfpathlineto{\pgfqpoint{8.707666in}{2.004286in}}%
\pgfpathlineto{\pgfqpoint{8.716472in}{1.722935in}}%
\pgfpathlineto{\pgfqpoint{8.725279in}{1.832732in}}%
\pgfpathlineto{\pgfqpoint{8.734086in}{1.475878in}}%
\pgfpathlineto{\pgfqpoint{8.742893in}{1.599393in}}%
\pgfpathlineto{\pgfqpoint{8.751700in}{1.640574in}}%
\pgfpathlineto{\pgfqpoint{8.760507in}{1.654291in}}%
\pgfpathlineto{\pgfqpoint{8.769313in}{1.812128in}}%
\pgfpathlineto{\pgfqpoint{8.778120in}{1.647432in}}%
\pgfpathlineto{\pgfqpoint{8.786927in}{1.558212in}}%
\pgfpathlineto{\pgfqpoint{8.795734in}{1.523918in}}%
\pgfpathlineto{\pgfqpoint{8.804541in}{1.448415in}}%
\pgfpathlineto{\pgfqpoint{8.813347in}{1.523918in}}%
\pgfpathlineto{\pgfqpoint{8.822154in}{1.558212in}}%
\pgfpathlineto{\pgfqpoint{8.830961in}{1.613110in}}%
\pgfpathlineto{\pgfqpoint{8.839768in}{1.681754in}}%
\pgfpathlineto{\pgfqpoint{8.848575in}{1.853308in}}%
\pgfpathlineto{\pgfqpoint{8.857382in}{1.585675in}}%
\pgfpathlineto{\pgfqpoint{8.866188in}{2.120970in}}%
\pgfpathlineto{\pgfqpoint{8.874995in}{1.921953in}}%
\pgfpathlineto{\pgfqpoint{8.883802in}{1.798410in}}%
\pgfpathlineto{\pgfqpoint{8.892609in}{1.942529in}}%
\pgfpathlineto{\pgfqpoint{8.901416in}{1.736653in}}%
\pgfpathlineto{\pgfqpoint{8.910222in}{1.935670in}}%
\pgfpathlineto{\pgfqpoint{8.919029in}{1.702331in}}%
\pgfpathlineto{\pgfqpoint{8.927836in}{1.585675in}}%
\pgfpathlineto{\pgfqpoint{8.936643in}{1.571958in}}%
\pgfpathlineto{\pgfqpoint{8.945450in}{1.674896in}}%
\pgfpathlineto{\pgfqpoint{8.954257in}{1.935670in}}%
\pgfpathlineto{\pgfqpoint{8.963063in}{1.468992in}}%
\pgfpathlineto{\pgfqpoint{8.971870in}{1.530777in}}%
\pgfpathlineto{\pgfqpoint{8.980677in}{2.024891in}}%
\pgfpathlineto{\pgfqpoint{8.989484in}{1.969992in}}%
\pgfpathlineto{\pgfqpoint{8.998291in}{2.374885in}}%
\pgfpathlineto{\pgfqpoint{9.007097in}{2.059213in}}%
\pgfpathlineto{\pgfqpoint{9.015904in}{2.169009in}}%
\pgfpathlineto{\pgfqpoint{9.024711in}{1.901348in}}%
\pgfpathlineto{\pgfqpoint{9.033518in}{1.530777in}}%
\pgfpathlineto{\pgfqpoint{9.042325in}{1.434698in}}%
\pgfpathlineto{\pgfqpoint{9.051132in}{1.263115in}}%
\pgfpathlineto{\pgfqpoint{9.059938in}{1.057239in}}%
\pgfpathlineto{\pgfqpoint{9.068745in}{1.427839in}}%
\pgfpathlineto{\pgfqpoint{9.077552in}{1.263115in}}%
\pgfpathlineto{\pgfqpoint{9.086359in}{1.420952in}}%
\pgfpathlineto{\pgfqpoint{9.095166in}{1.489596in}}%
\pgfpathlineto{\pgfqpoint{9.103972in}{1.427839in}}%
\pgfpathlineto{\pgfqpoint{9.112779in}{1.640574in}}%
\pgfpathlineto{\pgfqpoint{9.121586in}{1.976851in}}%
\pgfpathlineto{\pgfqpoint{9.130393in}{2.189586in}}%
\pgfpathlineto{\pgfqpoint{9.139200in}{2.127829in}}%
\pgfpathlineto{\pgfqpoint{9.148007in}{2.045467in}}%
\pgfpathlineto{\pgfqpoint{9.156813in}{1.791551in}}%
\pgfpathlineto{\pgfqpoint{9.165620in}{1.969992in}}%
\pgfpathlineto{\pgfqpoint{9.174427in}{1.873913in}}%
\pgfpathlineto{\pgfqpoint{9.183234in}{1.963133in}}%
\pgfpathlineto{\pgfqpoint{9.200847in}{1.366053in}}%
\pgfpathlineto{\pgfqpoint{9.209654in}{1.599393in}}%
\pgfpathlineto{\pgfqpoint{9.218461in}{1.716076in}}%
\pgfpathlineto{\pgfqpoint{9.227268in}{1.722935in}}%
\pgfpathlineto{\pgfqpoint{9.236075in}{1.812128in}}%
\pgfpathlineto{\pgfqpoint{9.253688in}{2.443501in}}%
\pgfpathlineto{\pgfqpoint{9.262495in}{1.853308in}}%
\pgfpathlineto{\pgfqpoint{9.271302in}{1.468992in}}%
\pgfpathlineto{\pgfqpoint{9.280109in}{1.283720in}}%
\pgfpathlineto{\pgfqpoint{9.288916in}{1.503314in}}%
\pgfpathlineto{\pgfqpoint{9.297722in}{1.613110in}}%
\pgfpathlineto{\pgfqpoint{9.306529in}{1.750370in}}%
\pgfpathlineto{\pgfqpoint{9.315336in}{1.654291in}}%
\pgfpathlineto{\pgfqpoint{9.324143in}{1.757229in}}%
\pgfpathlineto{\pgfqpoint{9.332950in}{1.517031in}}%
\pgfpathlineto{\pgfqpoint{9.341757in}{1.743512in}}%
\pgfpathlineto{\pgfqpoint{9.350563in}{1.468992in}}%
\pgfpathlineto{\pgfqpoint{9.359370in}{1.544494in}}%
\pgfpathlineto{\pgfqpoint{9.368177in}{1.770975in}}%
\pgfpathlineto{\pgfqpoint{9.376984in}{1.709190in}}%
\pgfpathlineto{\pgfqpoint{9.385791in}{1.798410in}}%
\pgfpathlineto{\pgfqpoint{9.394597in}{1.599393in}}%
\pgfpathlineto{\pgfqpoint{9.403404in}{1.681754in}}%
\pgfpathlineto{\pgfqpoint{9.412211in}{1.537636in}}%
\pgfpathlineto{\pgfqpoint{9.421018in}{1.496455in}}%
\pgfpathlineto{\pgfqpoint{9.429825in}{1.825873in}}%
\pgfpathlineto{\pgfqpoint{9.438632in}{1.839591in}}%
\pgfpathlineto{\pgfqpoint{9.447438in}{1.887630in}}%
\pgfpathlineto{\pgfqpoint{9.456245in}{1.709190in}}%
\pgfpathlineto{\pgfqpoint{9.465052in}{1.784692in}}%
\pgfpathlineto{\pgfqpoint{9.473859in}{1.290579in}}%
\pgfpathlineto{\pgfqpoint{9.482666in}{1.503314in}}%
\pgfpathlineto{\pgfqpoint{9.491472in}{1.846450in}}%
\pgfpathlineto{\pgfqpoint{9.500279in}{1.812128in}}%
\pgfpathlineto{\pgfqpoint{9.509086in}{1.716076in}}%
\pgfpathlineto{\pgfqpoint{9.517893in}{1.722935in}}%
\pgfpathlineto{\pgfqpoint{9.526700in}{1.819015in}}%
\pgfpathlineto{\pgfqpoint{9.535507in}{1.599393in}}%
\pgfpathlineto{\pgfqpoint{9.544313in}{1.606252in}}%
\pgfpathlineto{\pgfqpoint{9.553120in}{1.839591in}}%
\pgfpathlineto{\pgfqpoint{9.561927in}{1.764116in}}%
\pgfpathlineto{\pgfqpoint{9.570734in}{1.956246in}}%
\pgfpathlineto{\pgfqpoint{9.579541in}{1.860167in}}%
\pgfpathlineto{\pgfqpoint{9.588347in}{1.743512in}}%
\pgfpathlineto{\pgfqpoint{9.597154in}{1.722935in}}%
\pgfpathlineto{\pgfqpoint{9.605961in}{1.873913in}}%
\pgfpathlineto{\pgfqpoint{9.614768in}{1.832732in}}%
\pgfpathlineto{\pgfqpoint{9.623575in}{1.860167in}}%
\pgfpathlineto{\pgfqpoint{9.632382in}{1.963133in}}%
\pgfpathlineto{\pgfqpoint{9.641188in}{1.894489in}}%
\pgfpathlineto{\pgfqpoint{9.649995in}{1.976851in}}%
\pgfpathlineto{\pgfqpoint{9.658802in}{2.024891in}}%
\pgfpathlineto{\pgfqpoint{9.667609in}{1.983710in}}%
\pgfpathlineto{\pgfqpoint{9.676416in}{2.141546in}}%
\pgfpathlineto{\pgfqpoint{9.685222in}{2.217049in}}%
\pgfpathlineto{\pgfqpoint{9.694029in}{2.169009in}}%
\pgfpathlineto{\pgfqpoint{9.702836in}{1.935670in}}%
\pgfpathlineto{\pgfqpoint{9.711643in}{1.894489in}}%
\pgfpathlineto{\pgfqpoint{9.720450in}{1.674896in}}%
\pgfpathlineto{\pgfqpoint{9.729257in}{1.571958in}}%
\pgfpathlineto{\pgfqpoint{9.738063in}{2.120970in}}%
\pgfpathlineto{\pgfqpoint{9.746870in}{1.633715in}}%
\pgfpathlineto{\pgfqpoint{9.755677in}{1.695472in}}%
\pgfpathlineto{\pgfqpoint{9.764484in}{1.400376in}}%
\pgfpathlineto{\pgfqpoint{9.773291in}{1.242539in}}%
\pgfpathlineto{\pgfqpoint{9.782097in}{1.592534in}}%
\pgfpathlineto{\pgfqpoint{9.790904in}{1.736653in}}%
\pgfpathlineto{\pgfqpoint{9.799711in}{1.839591in}}%
\pgfpathlineto{\pgfqpoint{9.808518in}{1.503314in}}%
\pgfpathlineto{\pgfqpoint{9.817325in}{1.263115in}}%
\pgfpathlineto{\pgfqpoint{9.826132in}{1.695472in}}%
\pgfpathlineto{\pgfqpoint{9.834938in}{1.887630in}}%
\pgfpathlineto{\pgfqpoint{9.843745in}{1.537636in}}%
\pgfpathlineto{\pgfqpoint{9.852552in}{1.475878in}}%
\pgfpathlineto{\pgfqpoint{9.861359in}{1.599393in}}%
\pgfpathlineto{\pgfqpoint{9.870166in}{1.496455in}}%
\pgfpathlineto{\pgfqpoint{9.878972in}{1.709190in}}%
\pgfpathlineto{\pgfqpoint{9.887779in}{1.695472in}}%
\pgfpathlineto{\pgfqpoint{9.896586in}{1.839591in}}%
\pgfpathlineto{\pgfqpoint{9.905393in}{2.251343in}}%
\pgfpathlineto{\pgfqpoint{9.914200in}{2.223908in}}%
\pgfpathlineto{\pgfqpoint{9.923007in}{2.079789in}}%
\pgfpathlineto{\pgfqpoint{9.931813in}{1.873913in}}%
\pgfpathlineto{\pgfqpoint{9.940620in}{1.750370in}}%
\pgfpathlineto{\pgfqpoint{9.949427in}{1.867054in}}%
\pgfpathlineto{\pgfqpoint{9.949427in}{1.867054in}}%
\pgfusepath{stroke}%
\end{pgfscope}%
\begin{pgfscope}%
\pgfpathrectangle{\pgfqpoint{0.702268in}{0.521603in}}{\pgfqpoint{9.687500in}{4.235000in}}%
\pgfusepath{clip}%
\pgfsetrectcap%
\pgfsetroundjoin%
\pgfsetlinewidth{0.501875pt}%
\definecolor{currentstroke}{rgb}{0.501961,0.501961,0.501961}%
\pgfsetstrokecolor{currentstroke}%
\pgfsetstrokeopacity{0.250000}%
\pgfsetdash{}{0pt}%
\pgfpathmoveto{\pgfqpoint{1.142609in}{4.193504in}}%
\pgfpathlineto{\pgfqpoint{1.160222in}{2.971937in}}%
\pgfpathlineto{\pgfqpoint{1.169029in}{2.093507in}}%
\pgfpathlineto{\pgfqpoint{1.177836in}{2.072930in}}%
\pgfpathlineto{\pgfqpoint{1.186643in}{1.976851in}}%
\pgfpathlineto{\pgfqpoint{1.195450in}{2.093507in}}%
\pgfpathlineto{\pgfqpoint{1.204257in}{2.237625in}}%
\pgfpathlineto{\pgfqpoint{1.213063in}{2.155292in}}%
\pgfpathlineto{\pgfqpoint{1.221870in}{2.395462in}}%
\pgfpathlineto{\pgfqpoint{1.230677in}{2.244484in}}%
\pgfpathlineto{\pgfqpoint{1.239484in}{2.066071in}}%
\pgfpathlineto{\pgfqpoint{1.248291in}{2.299383in}}%
\pgfpathlineto{\pgfqpoint{1.257097in}{2.244484in}}%
\pgfpathlineto{\pgfqpoint{1.265904in}{2.374885in}}%
\pgfpathlineto{\pgfqpoint{1.274711in}{2.237625in}}%
\pgfpathlineto{\pgfqpoint{1.283518in}{1.949388in}}%
\pgfpathlineto{\pgfqpoint{1.292325in}{1.784692in}}%
\pgfpathlineto{\pgfqpoint{1.301132in}{1.812128in}}%
\pgfpathlineto{\pgfqpoint{1.309938in}{1.764116in}}%
\pgfpathlineto{\pgfqpoint{1.318745in}{2.031749in}}%
\pgfpathlineto{\pgfqpoint{1.327552in}{1.942529in}}%
\pgfpathlineto{\pgfqpoint{1.336359in}{1.887630in}}%
\pgfpathlineto{\pgfqpoint{1.345166in}{2.182727in}}%
\pgfpathlineto{\pgfqpoint{1.353972in}{2.230767in}}%
\pgfpathlineto{\pgfqpoint{1.362779in}{1.928811in}}%
\pgfpathlineto{\pgfqpoint{1.371586in}{2.059213in}}%
\pgfpathlineto{\pgfqpoint{1.380393in}{1.928811in}}%
\pgfpathlineto{\pgfqpoint{1.389200in}{1.846450in}}%
\pgfpathlineto{\pgfqpoint{1.398007in}{1.997427in}}%
\pgfpathlineto{\pgfqpoint{1.406813in}{1.887630in}}%
\pgfpathlineto{\pgfqpoint{1.415620in}{1.880772in}}%
\pgfpathlineto{\pgfqpoint{1.424427in}{1.956246in}}%
\pgfpathlineto{\pgfqpoint{1.433234in}{1.928811in}}%
\pgfpathlineto{\pgfqpoint{1.442041in}{1.736653in}}%
\pgfpathlineto{\pgfqpoint{1.450847in}{1.448415in}}%
\pgfpathlineto{\pgfqpoint{1.459654in}{1.544494in}}%
\pgfpathlineto{\pgfqpoint{1.468461in}{1.311155in}}%
\pgfpathlineto{\pgfqpoint{1.486075in}{1.571958in}}%
\pgfpathlineto{\pgfqpoint{1.494882in}{1.819015in}}%
\pgfpathlineto{\pgfqpoint{1.503688in}{1.997427in}}%
\pgfpathlineto{\pgfqpoint{1.512495in}{2.079789in}}%
\pgfpathlineto{\pgfqpoint{1.521302in}{1.928811in}}%
\pgfpathlineto{\pgfqpoint{1.530109in}{1.908207in}}%
\pgfpathlineto{\pgfqpoint{1.538916in}{2.045467in}}%
\pgfpathlineto{\pgfqpoint{1.547722in}{1.613110in}}%
\pgfpathlineto{\pgfqpoint{1.556529in}{1.743512in}}%
\pgfpathlineto{\pgfqpoint{1.565336in}{1.784692in}}%
\pgfpathlineto{\pgfqpoint{1.574143in}{2.155292in}}%
\pgfpathlineto{\pgfqpoint{1.582950in}{2.107252in}}%
\pgfpathlineto{\pgfqpoint{1.591757in}{1.729794in}}%
\pgfpathlineto{\pgfqpoint{1.600563in}{1.434698in}}%
\pgfpathlineto{\pgfqpoint{1.609370in}{1.331760in}}%
\pgfpathlineto{\pgfqpoint{1.618177in}{1.324873in}}%
\pgfpathlineto{\pgfqpoint{1.626984in}{1.118997in}}%
\pgfpathlineto{\pgfqpoint{1.635791in}{1.242539in}}%
\pgfpathlineto{\pgfqpoint{1.653404in}{1.942529in}}%
\pgfpathlineto{\pgfqpoint{1.662211in}{1.819015in}}%
\pgfpathlineto{\pgfqpoint{1.671018in}{1.997427in}}%
\pgfpathlineto{\pgfqpoint{1.679825in}{2.258230in}}%
\pgfpathlineto{\pgfqpoint{1.688632in}{2.189586in}}%
\pgfpathlineto{\pgfqpoint{1.697438in}{2.141546in}}%
\pgfpathlineto{\pgfqpoint{1.706245in}{2.354309in}}%
\pgfpathlineto{\pgfqpoint{1.715052in}{2.491541in}}%
\pgfpathlineto{\pgfqpoint{1.723859in}{2.388603in}}%
\pgfpathlineto{\pgfqpoint{1.732666in}{2.120970in}}%
\pgfpathlineto{\pgfqpoint{1.741472in}{2.450388in}}%
\pgfpathlineto{\pgfqpoint{1.750279in}{2.210190in}}%
\pgfpathlineto{\pgfqpoint{1.759086in}{1.928811in}}%
\pgfpathlineto{\pgfqpoint{1.767893in}{1.722935in}}%
\pgfpathlineto{\pgfqpoint{1.776700in}{1.757229in}}%
\pgfpathlineto{\pgfqpoint{1.785507in}{1.633715in}}%
\pgfpathlineto{\pgfqpoint{1.794313in}{1.777834in}}%
\pgfpathlineto{\pgfqpoint{1.803120in}{1.729794in}}%
\pgfpathlineto{\pgfqpoint{1.811927in}{2.011173in}}%
\pgfpathlineto{\pgfqpoint{1.820734in}{1.976851in}}%
\pgfpathlineto{\pgfqpoint{1.829541in}{1.901348in}}%
\pgfpathlineto{\pgfqpoint{1.838347in}{2.079789in}}%
\pgfpathlineto{\pgfqpoint{1.847154in}{2.066071in}}%
\pgfpathlineto{\pgfqpoint{1.855961in}{1.873913in}}%
\pgfpathlineto{\pgfqpoint{1.864768in}{1.716076in}}%
\pgfpathlineto{\pgfqpoint{1.873575in}{1.475878in}}%
\pgfpathlineto{\pgfqpoint{1.882382in}{1.976851in}}%
\pgfpathlineto{\pgfqpoint{1.891188in}{1.640574in}}%
\pgfpathlineto{\pgfqpoint{1.899995in}{1.578816in}}%
\pgfpathlineto{\pgfqpoint{1.908802in}{1.825873in}}%
\pgfpathlineto{\pgfqpoint{1.917609in}{1.976851in}}%
\pgfpathlineto{\pgfqpoint{1.926416in}{2.182727in}}%
\pgfpathlineto{\pgfqpoint{1.935222in}{1.626856in}}%
\pgfpathlineto{\pgfqpoint{1.952836in}{1.839591in}}%
\pgfpathlineto{\pgfqpoint{1.961643in}{1.777834in}}%
\pgfpathlineto{\pgfqpoint{1.970450in}{1.853308in}}%
\pgfpathlineto{\pgfqpoint{1.979257in}{2.519004in}}%
\pgfpathlineto{\pgfqpoint{1.988063in}{2.374885in}}%
\pgfpathlineto{\pgfqpoint{1.996870in}{1.956246in}}%
\pgfpathlineto{\pgfqpoint{2.005677in}{2.011173in}}%
\pgfpathlineto{\pgfqpoint{2.014484in}{1.777834in}}%
\pgfpathlineto{\pgfqpoint{2.023291in}{1.661150in}}%
\pgfpathlineto{\pgfqpoint{2.032097in}{1.619997in}}%
\pgfpathlineto{\pgfqpoint{2.040904in}{1.688613in}}%
\pgfpathlineto{\pgfqpoint{2.049711in}{1.812128in}}%
\pgfpathlineto{\pgfqpoint{2.058518in}{1.839591in}}%
\pgfpathlineto{\pgfqpoint{2.067325in}{1.722935in}}%
\pgfpathlineto{\pgfqpoint{2.076132in}{1.736653in}}%
\pgfpathlineto{\pgfqpoint{2.084938in}{1.880772in}}%
\pgfpathlineto{\pgfqpoint{2.093745in}{2.066071in}}%
\pgfpathlineto{\pgfqpoint{2.102552in}{1.722935in}}%
\pgfpathlineto{\pgfqpoint{2.111359in}{1.654291in}}%
\pgfpathlineto{\pgfqpoint{2.120166in}{1.393517in}}%
\pgfpathlineto{\pgfqpoint{2.128972in}{1.256257in}}%
\pgfpathlineto{\pgfqpoint{2.137779in}{1.153319in}}%
\pgfpathlineto{\pgfqpoint{2.146586in}{1.407234in}}%
\pgfpathlineto{\pgfqpoint{2.155393in}{1.441556in}}%
\pgfpathlineto{\pgfqpoint{2.164200in}{1.626856in}}%
\pgfpathlineto{\pgfqpoint{2.173007in}{2.011173in}}%
\pgfpathlineto{\pgfqpoint{2.181813in}{1.983710in}}%
\pgfpathlineto{\pgfqpoint{2.190620in}{1.681754in}}%
\pgfpathlineto{\pgfqpoint{2.199427in}{1.757229in}}%
\pgfpathlineto{\pgfqpoint{2.208234in}{2.038608in}}%
\pgfpathlineto{\pgfqpoint{2.217041in}{1.956246in}}%
\pgfpathlineto{\pgfqpoint{2.225847in}{1.633715in}}%
\pgfpathlineto{\pgfqpoint{2.234654in}{1.812128in}}%
\pgfpathlineto{\pgfqpoint{2.243461in}{2.148405in}}%
\pgfpathlineto{\pgfqpoint{2.252268in}{2.148405in}}%
\pgfpathlineto{\pgfqpoint{2.261075in}{2.326846in}}%
\pgfpathlineto{\pgfqpoint{2.269882in}{2.265089in}}%
\pgfpathlineto{\pgfqpoint{2.278688in}{2.395462in}}%
\pgfpathlineto{\pgfqpoint{2.287495in}{2.223908in}}%
\pgfpathlineto{\pgfqpoint{2.296302in}{1.976851in}}%
\pgfpathlineto{\pgfqpoint{2.305109in}{2.141546in}}%
\pgfpathlineto{\pgfqpoint{2.313916in}{1.990569in}}%
\pgfpathlineto{\pgfqpoint{2.322722in}{1.873913in}}%
\pgfpathlineto{\pgfqpoint{2.331529in}{1.832732in}}%
\pgfpathlineto{\pgfqpoint{2.340336in}{2.018032in}}%
\pgfpathlineto{\pgfqpoint{2.349143in}{1.969992in}}%
\pgfpathlineto{\pgfqpoint{2.357950in}{2.189586in}}%
\pgfpathlineto{\pgfqpoint{2.366757in}{1.606252in}}%
\pgfpathlineto{\pgfqpoint{2.375563in}{1.770975in}}%
\pgfpathlineto{\pgfqpoint{2.384370in}{1.764116in}}%
\pgfpathlineto{\pgfqpoint{2.393177in}{2.018032in}}%
\pgfpathlineto{\pgfqpoint{2.401984in}{2.107252in}}%
\pgfpathlineto{\pgfqpoint{2.410791in}{2.155292in}}%
\pgfpathlineto{\pgfqpoint{2.419597in}{2.127829in}}%
\pgfpathlineto{\pgfqpoint{2.428404in}{2.120970in}}%
\pgfpathlineto{\pgfqpoint{2.437211in}{2.004286in}}%
\pgfpathlineto{\pgfqpoint{2.446018in}{1.860167in}}%
\pgfpathlineto{\pgfqpoint{2.454825in}{1.887630in}}%
\pgfpathlineto{\pgfqpoint{2.463632in}{2.120970in}}%
\pgfpathlineto{\pgfqpoint{2.472438in}{1.894489in}}%
\pgfpathlineto{\pgfqpoint{2.481245in}{2.120970in}}%
\pgfpathlineto{\pgfqpoint{2.490052in}{2.299383in}}%
\pgfpathlineto{\pgfqpoint{2.498859in}{2.319987in}}%
\pgfpathlineto{\pgfqpoint{2.507666in}{2.182727in}}%
\pgfpathlineto{\pgfqpoint{2.516472in}{2.203303in}}%
\pgfpathlineto{\pgfqpoint{2.525279in}{2.100365in}}%
\pgfpathlineto{\pgfqpoint{2.534086in}{2.052326in}}%
\pgfpathlineto{\pgfqpoint{2.542893in}{1.908207in}}%
\pgfpathlineto{\pgfqpoint{2.551700in}{2.114111in}}%
\pgfpathlineto{\pgfqpoint{2.560507in}{2.148405in}}%
\pgfpathlineto{\pgfqpoint{2.569313in}{1.619997in}}%
\pgfpathlineto{\pgfqpoint{2.586927in}{2.038608in}}%
\pgfpathlineto{\pgfqpoint{2.595734in}{1.668037in}}%
\pgfpathlineto{\pgfqpoint{2.604541in}{1.770975in}}%
\pgfpathlineto{\pgfqpoint{2.613347in}{1.894489in}}%
\pgfpathlineto{\pgfqpoint{2.622154in}{1.661150in}}%
\pgfpathlineto{\pgfqpoint{2.630961in}{1.640574in}}%
\pgfpathlineto{\pgfqpoint{2.639768in}{1.887630in}}%
\pgfpathlineto{\pgfqpoint{2.648575in}{1.688613in}}%
\pgfpathlineto{\pgfqpoint{2.657382in}{1.976851in}}%
\pgfpathlineto{\pgfqpoint{2.666188in}{2.313128in}}%
\pgfpathlineto{\pgfqpoint{2.674995in}{2.244484in}}%
\pgfpathlineto{\pgfqpoint{2.683802in}{2.045467in}}%
\pgfpathlineto{\pgfqpoint{2.692609in}{1.674896in}}%
\pgfpathlineto{\pgfqpoint{2.701416in}{1.853308in}}%
\pgfpathlineto{\pgfqpoint{2.710222in}{2.120970in}}%
\pgfpathlineto{\pgfqpoint{2.719029in}{2.223908in}}%
\pgfpathlineto{\pgfqpoint{2.727836in}{2.374885in}}%
\pgfpathlineto{\pgfqpoint{2.736643in}{2.416066in}}%
\pgfpathlineto{\pgfqpoint{2.745450in}{2.134687in}}%
\pgfpathlineto{\pgfqpoint{2.754257in}{2.045467in}}%
\pgfpathlineto{\pgfqpoint{2.763063in}{1.729794in}}%
\pgfpathlineto{\pgfqpoint{2.771870in}{1.523918in}}%
\pgfpathlineto{\pgfqpoint{2.780677in}{1.551353in}}%
\pgfpathlineto{\pgfqpoint{2.789484in}{1.722935in}}%
\pgfpathlineto{\pgfqpoint{2.798291in}{2.120970in}}%
\pgfpathlineto{\pgfqpoint{2.807097in}{2.175868in}}%
\pgfpathlineto{\pgfqpoint{2.815904in}{2.319987in}}%
\pgfpathlineto{\pgfqpoint{2.824711in}{1.949388in}}%
\pgfpathlineto{\pgfqpoint{2.833518in}{1.764116in}}%
\pgfpathlineto{\pgfqpoint{2.842325in}{1.908207in}}%
\pgfpathlineto{\pgfqpoint{2.851132in}{2.031749in}}%
\pgfpathlineto{\pgfqpoint{2.859938in}{1.633715in}}%
\pgfpathlineto{\pgfqpoint{2.868745in}{1.839591in}}%
\pgfpathlineto{\pgfqpoint{2.877552in}{1.901348in}}%
\pgfpathlineto{\pgfqpoint{2.886359in}{1.716076in}}%
\pgfpathlineto{\pgfqpoint{2.895166in}{1.619997in}}%
\pgfpathlineto{\pgfqpoint{2.903972in}{1.867054in}}%
\pgfpathlineto{\pgfqpoint{2.912779in}{2.237625in}}%
\pgfpathlineto{\pgfqpoint{2.921586in}{2.059213in}}%
\pgfpathlineto{\pgfqpoint{2.930393in}{2.162151in}}%
\pgfpathlineto{\pgfqpoint{2.939200in}{2.278806in}}%
\pgfpathlineto{\pgfqpoint{2.948007in}{2.182727in}}%
\pgfpathlineto{\pgfqpoint{2.956813in}{2.210190in}}%
\pgfpathlineto{\pgfqpoint{2.965620in}{2.175868in}}%
\pgfpathlineto{\pgfqpoint{2.974427in}{2.093507in}}%
\pgfpathlineto{\pgfqpoint{2.992041in}{1.722935in}}%
\pgfpathlineto{\pgfqpoint{3.000847in}{2.120970in}}%
\pgfpathlineto{\pgfqpoint{3.009654in}{1.867054in}}%
\pgfpathlineto{\pgfqpoint{3.018461in}{2.045467in}}%
\pgfpathlineto{\pgfqpoint{3.027268in}{1.681754in}}%
\pgfpathlineto{\pgfqpoint{3.036075in}{1.791551in}}%
\pgfpathlineto{\pgfqpoint{3.044882in}{1.784692in}}%
\pgfpathlineto{\pgfqpoint{3.053688in}{1.764116in}}%
\pgfpathlineto{\pgfqpoint{3.062495in}{1.976851in}}%
\pgfpathlineto{\pgfqpoint{3.071302in}{1.963133in}}%
\pgfpathlineto{\pgfqpoint{3.080109in}{2.086648in}}%
\pgfpathlineto{\pgfqpoint{3.088916in}{2.052326in}}%
\pgfpathlineto{\pgfqpoint{3.097722in}{1.997427in}}%
\pgfpathlineto{\pgfqpoint{3.106529in}{1.839591in}}%
\pgfpathlineto{\pgfqpoint{3.124143in}{2.381744in}}%
\pgfpathlineto{\pgfqpoint{3.132950in}{1.853308in}}%
\pgfpathlineto{\pgfqpoint{3.141757in}{1.448415in}}%
\pgfpathlineto{\pgfqpoint{3.150563in}{1.269974in}}%
\pgfpathlineto{\pgfqpoint{3.159370in}{1.201358in}}%
\pgfpathlineto{\pgfqpoint{3.168177in}{1.372912in}}%
\pgfpathlineto{\pgfqpoint{3.176984in}{1.242539in}}%
\pgfpathlineto{\pgfqpoint{3.185791in}{1.221935in}}%
\pgfpathlineto{\pgfqpoint{3.194597in}{1.194499in}}%
\pgfpathlineto{\pgfqpoint{3.203404in}{1.613110in}}%
\pgfpathlineto{\pgfqpoint{3.212211in}{1.812128in}}%
\pgfpathlineto{\pgfqpoint{3.221018in}{1.873913in}}%
\pgfpathlineto{\pgfqpoint{3.229825in}{1.846450in}}%
\pgfpathlineto{\pgfqpoint{3.238632in}{1.908207in}}%
\pgfpathlineto{\pgfqpoint{3.247438in}{1.633715in}}%
\pgfpathlineto{\pgfqpoint{3.256245in}{1.654291in}}%
\pgfpathlineto{\pgfqpoint{3.265052in}{1.654291in}}%
\pgfpathlineto{\pgfqpoint{3.273859in}{2.004286in}}%
\pgfpathlineto{\pgfqpoint{3.282666in}{1.921953in}}%
\pgfpathlineto{\pgfqpoint{3.291472in}{1.880772in}}%
\pgfpathlineto{\pgfqpoint{3.300279in}{2.374885in}}%
\pgfpathlineto{\pgfqpoint{3.309086in}{2.422925in}}%
\pgfpathlineto{\pgfqpoint{3.317893in}{2.319987in}}%
\pgfpathlineto{\pgfqpoint{3.326700in}{2.402349in}}%
\pgfpathlineto{\pgfqpoint{3.335507in}{2.148405in}}%
\pgfpathlineto{\pgfqpoint{3.344313in}{2.299383in}}%
\pgfpathlineto{\pgfqpoint{3.353120in}{2.120970in}}%
\pgfpathlineto{\pgfqpoint{3.361927in}{2.045467in}}%
\pgfpathlineto{\pgfqpoint{3.370734in}{2.196445in}}%
\pgfpathlineto{\pgfqpoint{3.379541in}{2.237625in}}%
\pgfpathlineto{\pgfqpoint{3.388347in}{2.072930in}}%
\pgfpathlineto{\pgfqpoint{3.397154in}{1.784692in}}%
\pgfpathlineto{\pgfqpoint{3.405961in}{2.237625in}}%
\pgfpathlineto{\pgfqpoint{3.414768in}{2.120970in}}%
\pgfpathlineto{\pgfqpoint{3.423575in}{2.100365in}}%
\pgfpathlineto{\pgfqpoint{3.432382in}{1.956246in}}%
\pgfpathlineto{\pgfqpoint{3.441188in}{1.908207in}}%
\pgfpathlineto{\pgfqpoint{3.449995in}{1.798410in}}%
\pgfpathlineto{\pgfqpoint{3.458802in}{1.853308in}}%
\pgfpathlineto{\pgfqpoint{3.476416in}{1.393517in}}%
\pgfpathlineto{\pgfqpoint{3.485222in}{1.393517in}}%
\pgfpathlineto{\pgfqpoint{3.502836in}{1.990569in}}%
\pgfpathlineto{\pgfqpoint{3.511643in}{1.846450in}}%
\pgfpathlineto{\pgfqpoint{3.520450in}{1.592534in}}%
\pgfpathlineto{\pgfqpoint{3.529257in}{1.784692in}}%
\pgfpathlineto{\pgfqpoint{3.538063in}{2.024891in}}%
\pgfpathlineto{\pgfqpoint{3.546870in}{1.729794in}}%
\pgfpathlineto{\pgfqpoint{3.555677in}{1.915094in}}%
\pgfpathlineto{\pgfqpoint{3.564484in}{1.571958in}}%
\pgfpathlineto{\pgfqpoint{3.573291in}{1.434698in}}%
\pgfpathlineto{\pgfqpoint{3.582097in}{1.592534in}}%
\pgfpathlineto{\pgfqpoint{3.590904in}{1.558212in}}%
\pgfpathlineto{\pgfqpoint{3.599711in}{1.791551in}}%
\pgfpathlineto{\pgfqpoint{3.608518in}{1.709190in}}%
\pgfpathlineto{\pgfqpoint{3.617325in}{1.956246in}}%
\pgfpathlineto{\pgfqpoint{3.626132in}{2.079789in}}%
\pgfpathlineto{\pgfqpoint{3.634938in}{2.059213in}}%
\pgfpathlineto{\pgfqpoint{3.643745in}{2.059213in}}%
\pgfpathlineto{\pgfqpoint{3.652552in}{1.839591in}}%
\pgfpathlineto{\pgfqpoint{3.661359in}{1.812128in}}%
\pgfpathlineto{\pgfqpoint{3.670166in}{1.613110in}}%
\pgfpathlineto{\pgfqpoint{3.678972in}{1.331760in}}%
\pgfpathlineto{\pgfqpoint{3.687779in}{1.674896in}}%
\pgfpathlineto{\pgfqpoint{3.696586in}{1.812128in}}%
\pgfpathlineto{\pgfqpoint{3.705393in}{1.770975in}}%
\pgfpathlineto{\pgfqpoint{3.714200in}{1.853308in}}%
\pgfpathlineto{\pgfqpoint{3.723007in}{1.743512in}}%
\pgfpathlineto{\pgfqpoint{3.731813in}{2.189586in}}%
\pgfpathlineto{\pgfqpoint{3.740620in}{2.175868in}}%
\pgfpathlineto{\pgfqpoint{3.749427in}{2.059213in}}%
\pgfpathlineto{\pgfqpoint{3.758234in}{1.908207in}}%
\pgfpathlineto{\pgfqpoint{3.767041in}{1.592534in}}%
\pgfpathlineto{\pgfqpoint{3.775847in}{1.798410in}}%
\pgfpathlineto{\pgfqpoint{3.784654in}{1.750370in}}%
\pgfpathlineto{\pgfqpoint{3.793461in}{1.736653in}}%
\pgfpathlineto{\pgfqpoint{3.802268in}{1.976851in}}%
\pgfpathlineto{\pgfqpoint{3.811075in}{1.668037in}}%
\pgfpathlineto{\pgfqpoint{3.819882in}{2.114111in}}%
\pgfpathlineto{\pgfqpoint{3.828688in}{2.155292in}}%
\pgfpathlineto{\pgfqpoint{3.837495in}{2.120970in}}%
\pgfpathlineto{\pgfqpoint{3.846302in}{2.210190in}}%
\pgfpathlineto{\pgfqpoint{3.855109in}{2.265089in}}%
\pgfpathlineto{\pgfqpoint{3.872722in}{1.770975in}}%
\pgfpathlineto{\pgfqpoint{3.890336in}{2.306269in}}%
\pgfpathlineto{\pgfqpoint{3.899143in}{1.997427in}}%
\pgfpathlineto{\pgfqpoint{3.916757in}{1.558212in}}%
\pgfpathlineto{\pgfqpoint{3.925563in}{1.784692in}}%
\pgfpathlineto{\pgfqpoint{3.934370in}{2.169009in}}%
\pgfpathlineto{\pgfqpoint{3.943177in}{1.969992in}}%
\pgfpathlineto{\pgfqpoint{3.951984in}{2.223908in}}%
\pgfpathlineto{\pgfqpoint{3.960791in}{2.587620in}}%
\pgfpathlineto{\pgfqpoint{3.969597in}{2.210190in}}%
\pgfpathlineto{\pgfqpoint{3.987211in}{1.832732in}}%
\pgfpathlineto{\pgfqpoint{3.996018in}{2.066071in}}%
\pgfpathlineto{\pgfqpoint{4.004825in}{2.175868in}}%
\pgfpathlineto{\pgfqpoint{4.013632in}{2.066071in}}%
\pgfpathlineto{\pgfqpoint{4.022438in}{2.052326in}}%
\pgfpathlineto{\pgfqpoint{4.031245in}{2.072930in}}%
\pgfpathlineto{\pgfqpoint{4.040052in}{2.354309in}}%
\pgfpathlineto{\pgfqpoint{4.048859in}{2.196445in}}%
\pgfpathlineto{\pgfqpoint{4.057666in}{2.148405in}}%
\pgfpathlineto{\pgfqpoint{4.066472in}{1.928811in}}%
\pgfpathlineto{\pgfqpoint{4.075279in}{1.606252in}}%
\pgfpathlineto{\pgfqpoint{4.084086in}{1.661150in}}%
\pgfpathlineto{\pgfqpoint{4.092893in}{1.798410in}}%
\pgfpathlineto{\pgfqpoint{4.101700in}{1.674896in}}%
\pgfpathlineto{\pgfqpoint{4.110507in}{1.860167in}}%
\pgfpathlineto{\pgfqpoint{4.128120in}{2.374885in}}%
\pgfpathlineto{\pgfqpoint{4.136927in}{2.560185in}}%
\pgfpathlineto{\pgfqpoint{4.145734in}{2.230767in}}%
\pgfpathlineto{\pgfqpoint{4.154541in}{2.422925in}}%
\pgfpathlineto{\pgfqpoint{4.163347in}{2.505287in}}%
\pgfpathlineto{\pgfqpoint{4.172154in}{2.354309in}}%
\pgfpathlineto{\pgfqpoint{4.180961in}{2.258230in}}%
\pgfpathlineto{\pgfqpoint{4.189768in}{2.285665in}}%
\pgfpathlineto{\pgfqpoint{4.198575in}{1.942529in}}%
\pgfpathlineto{\pgfqpoint{4.207382in}{1.997427in}}%
\pgfpathlineto{\pgfqpoint{4.216188in}{2.079789in}}%
\pgfpathlineto{\pgfqpoint{4.224995in}{2.141546in}}%
\pgfpathlineto{\pgfqpoint{4.233802in}{2.059213in}}%
\pgfpathlineto{\pgfqpoint{4.242609in}{2.230767in}}%
\pgfpathlineto{\pgfqpoint{4.251416in}{2.217049in}}%
\pgfpathlineto{\pgfqpoint{4.260222in}{2.223908in}}%
\pgfpathlineto{\pgfqpoint{4.269029in}{2.422925in}}%
\pgfpathlineto{\pgfqpoint{4.277836in}{2.313128in}}%
\pgfpathlineto{\pgfqpoint{4.286643in}{2.148405in}}%
\pgfpathlineto{\pgfqpoint{4.295450in}{2.120970in}}%
\pgfpathlineto{\pgfqpoint{4.313063in}{2.203303in}}%
\pgfpathlineto{\pgfqpoint{4.321870in}{2.004286in}}%
\pgfpathlineto{\pgfqpoint{4.330677in}{2.326846in}}%
\pgfpathlineto{\pgfqpoint{4.339484in}{1.990569in}}%
\pgfpathlineto{\pgfqpoint{4.348291in}{1.935670in}}%
\pgfpathlineto{\pgfqpoint{4.357097in}{2.031749in}}%
\pgfpathlineto{\pgfqpoint{4.365904in}{2.162151in}}%
\pgfpathlineto{\pgfqpoint{4.374711in}{2.457247in}}%
\pgfpathlineto{\pgfqpoint{4.383518in}{2.498428in}}%
\pgfpathlineto{\pgfqpoint{4.392325in}{2.560185in}}%
\pgfpathlineto{\pgfqpoint{4.401132in}{2.244484in}}%
\pgfpathlineto{\pgfqpoint{4.409938in}{2.237625in}}%
\pgfpathlineto{\pgfqpoint{4.418745in}{2.676841in}}%
\pgfpathlineto{\pgfqpoint{4.427552in}{2.601366in}}%
\pgfpathlineto{\pgfqpoint{4.436359in}{2.361168in}}%
\pgfpathlineto{\pgfqpoint{4.445166in}{2.278806in}}%
\pgfpathlineto{\pgfqpoint{4.453972in}{1.935670in}}%
\pgfpathlineto{\pgfqpoint{4.462779in}{1.764116in}}%
\pgfpathlineto{\pgfqpoint{4.480393in}{1.894489in}}%
\pgfpathlineto{\pgfqpoint{4.489200in}{1.935670in}}%
\pgfpathlineto{\pgfqpoint{4.498007in}{2.052326in}}%
\pgfpathlineto{\pgfqpoint{4.506813in}{1.963133in}}%
\pgfpathlineto{\pgfqpoint{4.515620in}{2.148405in}}%
\pgfpathlineto{\pgfqpoint{4.524427in}{1.942529in}}%
\pgfpathlineto{\pgfqpoint{4.533234in}{2.182727in}}%
\pgfpathlineto{\pgfqpoint{4.542041in}{2.265089in}}%
\pgfpathlineto{\pgfqpoint{4.559654in}{2.484682in}}%
\pgfpathlineto{\pgfqpoint{4.568461in}{2.546468in}}%
\pgfpathlineto{\pgfqpoint{4.577268in}{2.258230in}}%
\pgfpathlineto{\pgfqpoint{4.586075in}{2.244484in}}%
\pgfpathlineto{\pgfqpoint{4.594882in}{2.340563in}}%
\pgfpathlineto{\pgfqpoint{4.603688in}{2.052326in}}%
\pgfpathlineto{\pgfqpoint{4.612495in}{2.114111in}}%
\pgfpathlineto{\pgfqpoint{4.621302in}{2.031749in}}%
\pgfpathlineto{\pgfqpoint{4.630109in}{1.640574in}}%
\pgfpathlineto{\pgfqpoint{4.638916in}{1.674896in}}%
\pgfpathlineto{\pgfqpoint{4.647722in}{2.086648in}}%
\pgfpathlineto{\pgfqpoint{4.656529in}{1.976851in}}%
\pgfpathlineto{\pgfqpoint{4.665336in}{2.285665in}}%
\pgfpathlineto{\pgfqpoint{4.674143in}{1.928811in}}%
\pgfpathlineto{\pgfqpoint{4.682950in}{1.626856in}}%
\pgfpathlineto{\pgfqpoint{4.691757in}{1.379799in}}%
\pgfpathlineto{\pgfqpoint{4.700563in}{1.372912in}}%
\pgfpathlineto{\pgfqpoint{4.709370in}{1.290579in}}%
\pgfpathlineto{\pgfqpoint{4.718177in}{1.400376in}}%
\pgfpathlineto{\pgfqpoint{4.735791in}{1.894489in}}%
\pgfpathlineto{\pgfqpoint{4.744597in}{2.244484in}}%
\pgfpathlineto{\pgfqpoint{4.753404in}{2.258230in}}%
\pgfpathlineto{\pgfqpoint{4.762211in}{2.114111in}}%
\pgfpathlineto{\pgfqpoint{4.771018in}{1.990569in}}%
\pgfpathlineto{\pgfqpoint{4.779825in}{2.059213in}}%
\pgfpathlineto{\pgfqpoint{4.788632in}{2.018032in}}%
\pgfpathlineto{\pgfqpoint{4.797438in}{2.024891in}}%
\pgfpathlineto{\pgfqpoint{4.806245in}{1.908207in}}%
\pgfpathlineto{\pgfqpoint{4.815052in}{1.633715in}}%
\pgfpathlineto{\pgfqpoint{4.823859in}{1.695472in}}%
\pgfpathlineto{\pgfqpoint{4.841472in}{1.750370in}}%
\pgfpathlineto{\pgfqpoint{4.859086in}{2.086648in}}%
\pgfpathlineto{\pgfqpoint{4.867893in}{2.134687in}}%
\pgfpathlineto{\pgfqpoint{4.876700in}{2.018032in}}%
\pgfpathlineto{\pgfqpoint{4.885507in}{1.990569in}}%
\pgfpathlineto{\pgfqpoint{4.894313in}{2.169009in}}%
\pgfpathlineto{\pgfqpoint{4.903120in}{2.107252in}}%
\pgfpathlineto{\pgfqpoint{4.911927in}{2.018032in}}%
\pgfpathlineto{\pgfqpoint{4.920734in}{2.422925in}}%
\pgfpathlineto{\pgfqpoint{4.929541in}{2.196445in}}%
\pgfpathlineto{\pgfqpoint{4.938347in}{2.114111in}}%
\pgfpathlineto{\pgfqpoint{4.947154in}{1.873913in}}%
\pgfpathlineto{\pgfqpoint{4.955961in}{1.544494in}}%
\pgfpathlineto{\pgfqpoint{4.964768in}{1.517031in}}%
\pgfpathlineto{\pgfqpoint{4.973575in}{1.544494in}}%
\pgfpathlineto{\pgfqpoint{4.982382in}{1.633715in}}%
\pgfpathlineto{\pgfqpoint{4.991188in}{1.949388in}}%
\pgfpathlineto{\pgfqpoint{4.999995in}{1.757229in}}%
\pgfpathlineto{\pgfqpoint{5.008802in}{1.825873in}}%
\pgfpathlineto{\pgfqpoint{5.017609in}{2.182727in}}%
\pgfpathlineto{\pgfqpoint{5.026416in}{2.299383in}}%
\pgfpathlineto{\pgfqpoint{5.035222in}{2.230767in}}%
\pgfpathlineto{\pgfqpoint{5.044029in}{2.402349in}}%
\pgfpathlineto{\pgfqpoint{5.052836in}{2.903321in}}%
\pgfpathlineto{\pgfqpoint{5.070450in}{2.889604in}}%
\pgfpathlineto{\pgfqpoint{5.079257in}{2.711163in}}%
\pgfpathlineto{\pgfqpoint{5.088063in}{2.436643in}}%
\pgfpathlineto{\pgfqpoint{5.096870in}{2.436643in}}%
\pgfpathlineto{\pgfqpoint{5.105677in}{2.114111in}}%
\pgfpathlineto{\pgfqpoint{5.114484in}{2.498428in}}%
\pgfpathlineto{\pgfqpoint{5.123291in}{2.258230in}}%
\pgfpathlineto{\pgfqpoint{5.132097in}{1.969992in}}%
\pgfpathlineto{\pgfqpoint{5.140904in}{1.921953in}}%
\pgfpathlineto{\pgfqpoint{5.149711in}{2.210190in}}%
\pgfpathlineto{\pgfqpoint{5.158518in}{2.354309in}}%
\pgfpathlineto{\pgfqpoint{5.167325in}{2.697445in}}%
\pgfpathlineto{\pgfqpoint{5.176132in}{2.217049in}}%
\pgfpathlineto{\pgfqpoint{5.184938in}{1.983710in}}%
\pgfpathlineto{\pgfqpoint{5.193745in}{1.901348in}}%
\pgfpathlineto{\pgfqpoint{5.202552in}{2.100365in}}%
\pgfpathlineto{\pgfqpoint{5.211359in}{2.182727in}}%
\pgfpathlineto{\pgfqpoint{5.220166in}{2.045467in}}%
\pgfpathlineto{\pgfqpoint{5.228972in}{1.997427in}}%
\pgfpathlineto{\pgfqpoint{5.237779in}{1.716076in}}%
\pgfpathlineto{\pgfqpoint{5.246586in}{1.798410in}}%
\pgfpathlineto{\pgfqpoint{5.255393in}{1.915094in}}%
\pgfpathlineto{\pgfqpoint{5.264200in}{2.079789in}}%
\pgfpathlineto{\pgfqpoint{5.273007in}{2.114111in}}%
\pgfpathlineto{\pgfqpoint{5.281813in}{1.956246in}}%
\pgfpathlineto{\pgfqpoint{5.290620in}{1.846450in}}%
\pgfpathlineto{\pgfqpoint{5.299427in}{1.462133in}}%
\pgfpathlineto{\pgfqpoint{5.308234in}{1.482737in}}%
\pgfpathlineto{\pgfqpoint{5.317041in}{1.393517in}}%
\pgfpathlineto{\pgfqpoint{5.325847in}{1.702331in}}%
\pgfpathlineto{\pgfqpoint{5.334654in}{1.530777in}}%
\pgfpathlineto{\pgfqpoint{5.343461in}{1.558212in}}%
\pgfpathlineto{\pgfqpoint{5.352268in}{1.894489in}}%
\pgfpathlineto{\pgfqpoint{5.361075in}{2.120970in}}%
\pgfpathlineto{\pgfqpoint{5.369882in}{2.381744in}}%
\pgfpathlineto{\pgfqpoint{5.378688in}{1.935670in}}%
\pgfpathlineto{\pgfqpoint{5.387495in}{1.819015in}}%
\pgfpathlineto{\pgfqpoint{5.396302in}{1.949388in}}%
\pgfpathlineto{\pgfqpoint{5.405109in}{2.189586in}}%
\pgfpathlineto{\pgfqpoint{5.413916in}{2.333705in}}%
\pgfpathlineto{\pgfqpoint{5.422722in}{2.134687in}}%
\pgfpathlineto{\pgfqpoint{5.431529in}{2.100365in}}%
\pgfpathlineto{\pgfqpoint{5.440336in}{2.031749in}}%
\pgfpathlineto{\pgfqpoint{5.449143in}{1.674896in}}%
\pgfpathlineto{\pgfqpoint{5.457950in}{1.661150in}}%
\pgfpathlineto{\pgfqpoint{5.466757in}{1.517031in}}%
\pgfpathlineto{\pgfqpoint{5.484370in}{1.633715in}}%
\pgfpathlineto{\pgfqpoint{5.493177in}{2.093507in}}%
\pgfpathlineto{\pgfqpoint{5.501984in}{1.997427in}}%
\pgfpathlineto{\pgfqpoint{5.510791in}{2.024891in}}%
\pgfpathlineto{\pgfqpoint{5.519597in}{1.784692in}}%
\pgfpathlineto{\pgfqpoint{5.528404in}{2.230767in}}%
\pgfpathlineto{\pgfqpoint{5.537211in}{2.368027in}}%
\pgfpathlineto{\pgfqpoint{5.546018in}{2.409207in}}%
\pgfpathlineto{\pgfqpoint{5.554825in}{2.203303in}}%
\pgfpathlineto{\pgfqpoint{5.563632in}{1.935670in}}%
\pgfpathlineto{\pgfqpoint{5.572438in}{2.038608in}}%
\pgfpathlineto{\pgfqpoint{5.581245in}{2.539581in}}%
\pgfpathlineto{\pgfqpoint{5.590052in}{2.217049in}}%
\pgfpathlineto{\pgfqpoint{5.598859in}{2.292524in}}%
\pgfpathlineto{\pgfqpoint{5.607666in}{1.901348in}}%
\pgfpathlineto{\pgfqpoint{5.616472in}{2.052326in}}%
\pgfpathlineto{\pgfqpoint{5.625279in}{2.402349in}}%
\pgfpathlineto{\pgfqpoint{5.634086in}{2.251343in}}%
\pgfpathlineto{\pgfqpoint{5.642893in}{2.155292in}}%
\pgfpathlineto{\pgfqpoint{5.651700in}{1.880772in}}%
\pgfpathlineto{\pgfqpoint{5.660507in}{1.695472in}}%
\pgfpathlineto{\pgfqpoint{5.669313in}{1.990569in}}%
\pgfpathlineto{\pgfqpoint{5.678120in}{2.059213in}}%
\pgfpathlineto{\pgfqpoint{5.686927in}{2.361168in}}%
\pgfpathlineto{\pgfqpoint{5.695734in}{1.983710in}}%
\pgfpathlineto{\pgfqpoint{5.704541in}{1.860167in}}%
\pgfpathlineto{\pgfqpoint{5.713347in}{1.908207in}}%
\pgfpathlineto{\pgfqpoint{5.722154in}{1.867054in}}%
\pgfpathlineto{\pgfqpoint{5.730961in}{2.004286in}}%
\pgfpathlineto{\pgfqpoint{5.739768in}{2.175868in}}%
\pgfpathlineto{\pgfqpoint{5.748575in}{2.093507in}}%
\pgfpathlineto{\pgfqpoint{5.757382in}{2.072930in}}%
\pgfpathlineto{\pgfqpoint{5.766188in}{2.395462in}}%
\pgfpathlineto{\pgfqpoint{5.774995in}{2.203303in}}%
\pgfpathlineto{\pgfqpoint{5.783802in}{2.306269in}}%
\pgfpathlineto{\pgfqpoint{5.792609in}{2.107252in}}%
\pgfpathlineto{\pgfqpoint{5.801416in}{2.196445in}}%
\pgfpathlineto{\pgfqpoint{5.810222in}{1.969992in}}%
\pgfpathlineto{\pgfqpoint{5.819029in}{1.599393in}}%
\pgfpathlineto{\pgfqpoint{5.827836in}{1.311155in}}%
\pgfpathlineto{\pgfqpoint{5.836643in}{1.290579in}}%
\pgfpathlineto{\pgfqpoint{5.845450in}{1.249398in}}%
\pgfpathlineto{\pgfqpoint{5.854257in}{1.475878in}}%
\pgfpathlineto{\pgfqpoint{5.863063in}{1.825873in}}%
\pgfpathlineto{\pgfqpoint{5.871870in}{1.832732in}}%
\pgfpathlineto{\pgfqpoint{5.880677in}{2.004286in}}%
\pgfpathlineto{\pgfqpoint{5.889484in}{1.921953in}}%
\pgfpathlineto{\pgfqpoint{5.898291in}{1.791551in}}%
\pgfpathlineto{\pgfqpoint{5.915904in}{2.155292in}}%
\pgfpathlineto{\pgfqpoint{5.924711in}{2.443501in}}%
\pgfpathlineto{\pgfqpoint{5.933518in}{2.024891in}}%
\pgfpathlineto{\pgfqpoint{5.942325in}{2.079789in}}%
\pgfpathlineto{\pgfqpoint{5.951132in}{1.928811in}}%
\pgfpathlineto{\pgfqpoint{5.959938in}{1.901348in}}%
\pgfpathlineto{\pgfqpoint{5.968745in}{1.716076in}}%
\pgfpathlineto{\pgfqpoint{5.977552in}{1.626856in}}%
\pgfpathlineto{\pgfqpoint{5.986359in}{1.880772in}}%
\pgfpathlineto{\pgfqpoint{5.995166in}{1.983710in}}%
\pgfpathlineto{\pgfqpoint{6.003972in}{2.031749in}}%
\pgfpathlineto{\pgfqpoint{6.012779in}{2.059213in}}%
\pgfpathlineto{\pgfqpoint{6.021586in}{1.921953in}}%
\pgfpathlineto{\pgfqpoint{6.030393in}{2.340563in}}%
\pgfpathlineto{\pgfqpoint{6.039200in}{2.278806in}}%
\pgfpathlineto{\pgfqpoint{6.048007in}{2.031749in}}%
\pgfpathlineto{\pgfqpoint{6.056813in}{1.853308in}}%
\pgfpathlineto{\pgfqpoint{6.065620in}{1.887630in}}%
\pgfpathlineto{\pgfqpoint{6.074427in}{1.791551in}}%
\pgfpathlineto{\pgfqpoint{6.083234in}{2.285665in}}%
\pgfpathlineto{\pgfqpoint{6.092041in}{2.217049in}}%
\pgfpathlineto{\pgfqpoint{6.100847in}{2.114111in}}%
\pgfpathlineto{\pgfqpoint{6.109654in}{2.114111in}}%
\pgfpathlineto{\pgfqpoint{6.118461in}{2.189586in}}%
\pgfpathlineto{\pgfqpoint{6.127268in}{1.908207in}}%
\pgfpathlineto{\pgfqpoint{6.136075in}{1.530777in}}%
\pgfpathlineto{\pgfqpoint{6.144882in}{1.585675in}}%
\pgfpathlineto{\pgfqpoint{6.153688in}{1.736653in}}%
\pgfpathlineto{\pgfqpoint{6.162495in}{1.668037in}}%
\pgfpathlineto{\pgfqpoint{6.171302in}{1.729794in}}%
\pgfpathlineto{\pgfqpoint{6.180109in}{1.853308in}}%
\pgfpathlineto{\pgfqpoint{6.188916in}{1.661150in}}%
\pgfpathlineto{\pgfqpoint{6.197722in}{1.640574in}}%
\pgfpathlineto{\pgfqpoint{6.206529in}{1.791551in}}%
\pgfpathlineto{\pgfqpoint{6.215336in}{1.619997in}}%
\pgfpathlineto{\pgfqpoint{6.224143in}{1.750370in}}%
\pgfpathlineto{\pgfqpoint{6.232950in}{2.155292in}}%
\pgfpathlineto{\pgfqpoint{6.241757in}{1.949388in}}%
\pgfpathlineto{\pgfqpoint{6.250563in}{1.887630in}}%
\pgfpathlineto{\pgfqpoint{6.259370in}{1.867054in}}%
\pgfpathlineto{\pgfqpoint{6.268177in}{2.024891in}}%
\pgfpathlineto{\pgfqpoint{6.276984in}{2.066071in}}%
\pgfpathlineto{\pgfqpoint{6.285791in}{1.921953in}}%
\pgfpathlineto{\pgfqpoint{6.294597in}{1.688613in}}%
\pgfpathlineto{\pgfqpoint{6.303404in}{1.798410in}}%
\pgfpathlineto{\pgfqpoint{6.312211in}{1.201358in}}%
\pgfpathlineto{\pgfqpoint{6.321018in}{1.427839in}}%
\pgfpathlineto{\pgfqpoint{6.329825in}{1.551353in}}%
\pgfpathlineto{\pgfqpoint{6.338632in}{1.551353in}}%
\pgfpathlineto{\pgfqpoint{6.347438in}{1.592534in}}%
\pgfpathlineto{\pgfqpoint{6.356245in}{1.091561in}}%
\pgfpathlineto{\pgfqpoint{6.365052in}{1.345477in}}%
\pgfpathlineto{\pgfqpoint{6.373859in}{1.729794in}}%
\pgfpathlineto{\pgfqpoint{6.382666in}{1.647432in}}%
\pgfpathlineto{\pgfqpoint{6.391472in}{1.695472in}}%
\pgfpathlineto{\pgfqpoint{6.400279in}{1.832732in}}%
\pgfpathlineto{\pgfqpoint{6.409086in}{1.571958in}}%
\pgfpathlineto{\pgfqpoint{6.417893in}{1.613110in}}%
\pgfpathlineto{\pgfqpoint{6.426700in}{1.750370in}}%
\pgfpathlineto{\pgfqpoint{6.435507in}{1.407234in}}%
\pgfpathlineto{\pgfqpoint{6.444313in}{1.283720in}}%
\pgfpathlineto{\pgfqpoint{6.453120in}{1.407234in}}%
\pgfpathlineto{\pgfqpoint{6.461927in}{1.372912in}}%
\pgfpathlineto{\pgfqpoint{6.470734in}{1.517031in}}%
\pgfpathlineto{\pgfqpoint{6.479541in}{1.578816in}}%
\pgfpathlineto{\pgfqpoint{6.488347in}{1.915094in}}%
\pgfpathlineto{\pgfqpoint{6.497154in}{1.832732in}}%
\pgfpathlineto{\pgfqpoint{6.505961in}{1.805269in}}%
\pgfpathlineto{\pgfqpoint{6.514768in}{1.949388in}}%
\pgfpathlineto{\pgfqpoint{6.523575in}{1.956246in}}%
\pgfpathlineto{\pgfqpoint{6.532382in}{1.736653in}}%
\pgfpathlineto{\pgfqpoint{6.541188in}{1.805269in}}%
\pgfpathlineto{\pgfqpoint{6.549995in}{1.928811in}}%
\pgfpathlineto{\pgfqpoint{6.558802in}{1.853308in}}%
\pgfpathlineto{\pgfqpoint{6.567609in}{1.880772in}}%
\pgfpathlineto{\pgfqpoint{6.576416in}{2.018032in}}%
\pgfpathlineto{\pgfqpoint{6.585222in}{2.031749in}}%
\pgfpathlineto{\pgfqpoint{6.594029in}{2.018032in}}%
\pgfpathlineto{\pgfqpoint{6.602836in}{2.237625in}}%
\pgfpathlineto{\pgfqpoint{6.611643in}{1.702331in}}%
\pgfpathlineto{\pgfqpoint{6.629257in}{2.038608in}}%
\pgfpathlineto{\pgfqpoint{6.638063in}{1.997427in}}%
\pgfpathlineto{\pgfqpoint{6.646870in}{1.928811in}}%
\pgfpathlineto{\pgfqpoint{6.655677in}{2.162151in}}%
\pgfpathlineto{\pgfqpoint{6.664484in}{2.251343in}}%
\pgfpathlineto{\pgfqpoint{6.673291in}{2.169009in}}%
\pgfpathlineto{\pgfqpoint{6.682097in}{1.969992in}}%
\pgfpathlineto{\pgfqpoint{6.690904in}{2.141546in}}%
\pgfpathlineto{\pgfqpoint{6.699711in}{2.024891in}}%
\pgfpathlineto{\pgfqpoint{6.717325in}{2.141546in}}%
\pgfpathlineto{\pgfqpoint{6.734938in}{2.326846in}}%
\pgfpathlineto{\pgfqpoint{6.743745in}{2.464106in}}%
\pgfpathlineto{\pgfqpoint{6.752552in}{2.745485in}}%
\pgfpathlineto{\pgfqpoint{6.761359in}{2.759202in}}%
\pgfpathlineto{\pgfqpoint{6.770166in}{2.374885in}}%
\pgfpathlineto{\pgfqpoint{6.778972in}{2.155292in}}%
\pgfpathlineto{\pgfqpoint{6.787779in}{1.839591in}}%
\pgfpathlineto{\pgfqpoint{6.796586in}{2.004286in}}%
\pgfpathlineto{\pgfqpoint{6.805393in}{1.592534in}}%
\pgfpathlineto{\pgfqpoint{6.814200in}{1.434698in}}%
\pgfpathlineto{\pgfqpoint{6.823007in}{2.141546in}}%
\pgfpathlineto{\pgfqpoint{6.831813in}{1.997427in}}%
\pgfpathlineto{\pgfqpoint{6.840620in}{1.915094in}}%
\pgfpathlineto{\pgfqpoint{6.849427in}{1.668037in}}%
\pgfpathlineto{\pgfqpoint{6.858234in}{1.523918in}}%
\pgfpathlineto{\pgfqpoint{6.867041in}{1.777834in}}%
\pgfpathlineto{\pgfqpoint{6.875847in}{1.798410in}}%
\pgfpathlineto{\pgfqpoint{6.884654in}{1.592534in}}%
\pgfpathlineto{\pgfqpoint{6.893461in}{1.215076in}}%
\pgfpathlineto{\pgfqpoint{6.902268in}{1.434698in}}%
\pgfpathlineto{\pgfqpoint{6.911075in}{1.908207in}}%
\pgfpathlineto{\pgfqpoint{6.919882in}{1.716076in}}%
\pgfpathlineto{\pgfqpoint{6.928688in}{1.901348in}}%
\pgfpathlineto{\pgfqpoint{6.937495in}{1.908207in}}%
\pgfpathlineto{\pgfqpoint{6.955109in}{2.567044in}}%
\pgfpathlineto{\pgfqpoint{6.963916in}{2.450388in}}%
\pgfpathlineto{\pgfqpoint{6.972722in}{2.038608in}}%
\pgfpathlineto{\pgfqpoint{6.981529in}{2.011173in}}%
\pgfpathlineto{\pgfqpoint{6.990336in}{1.990569in}}%
\pgfpathlineto{\pgfqpoint{6.999143in}{1.873913in}}%
\pgfpathlineto{\pgfqpoint{7.007950in}{1.915094in}}%
\pgfpathlineto{\pgfqpoint{7.016757in}{2.127829in}}%
\pgfpathlineto{\pgfqpoint{7.025563in}{2.059213in}}%
\pgfpathlineto{\pgfqpoint{7.034370in}{1.825873in}}%
\pgfpathlineto{\pgfqpoint{7.043177in}{1.702331in}}%
\pgfpathlineto{\pgfqpoint{7.051984in}{1.441556in}}%
\pgfpathlineto{\pgfqpoint{7.060791in}{1.503314in}}%
\pgfpathlineto{\pgfqpoint{7.069597in}{1.318014in}}%
\pgfpathlineto{\pgfqpoint{7.078404in}{1.523918in}}%
\pgfpathlineto{\pgfqpoint{7.087211in}{1.503314in}}%
\pgfpathlineto{\pgfqpoint{7.096018in}{1.427839in}}%
\pgfpathlineto{\pgfqpoint{7.104825in}{1.558212in}}%
\pgfpathlineto{\pgfqpoint{7.113632in}{2.038608in}}%
\pgfpathlineto{\pgfqpoint{7.122438in}{1.647432in}}%
\pgfpathlineto{\pgfqpoint{7.131245in}{1.585675in}}%
\pgfpathlineto{\pgfqpoint{7.140052in}{1.688613in}}%
\pgfpathlineto{\pgfqpoint{7.148859in}{1.743512in}}%
\pgfpathlineto{\pgfqpoint{7.157666in}{1.688613in}}%
\pgfpathlineto{\pgfqpoint{7.166472in}{1.901348in}}%
\pgfpathlineto{\pgfqpoint{7.175279in}{1.640574in}}%
\pgfpathlineto{\pgfqpoint{7.184086in}{2.251343in}}%
\pgfpathlineto{\pgfqpoint{7.192893in}{2.368027in}}%
\pgfpathlineto{\pgfqpoint{7.201700in}{2.436643in}}%
\pgfpathlineto{\pgfqpoint{7.210507in}{2.340563in}}%
\pgfpathlineto{\pgfqpoint{7.219313in}{2.086648in}}%
\pgfpathlineto{\pgfqpoint{7.228120in}{2.162151in}}%
\pgfpathlineto{\pgfqpoint{7.236927in}{2.326846in}}%
\pgfpathlineto{\pgfqpoint{7.245734in}{2.319987in}}%
\pgfpathlineto{\pgfqpoint{7.254541in}{2.587620in}}%
\pgfpathlineto{\pgfqpoint{7.263347in}{2.621942in}}%
\pgfpathlineto{\pgfqpoint{7.272154in}{2.265089in}}%
\pgfpathlineto{\pgfqpoint{7.280961in}{1.853308in}}%
\pgfpathlineto{\pgfqpoint{7.289768in}{1.791551in}}%
\pgfpathlineto{\pgfqpoint{7.298575in}{1.475878in}}%
\pgfpathlineto{\pgfqpoint{7.307382in}{1.352336in}}%
\pgfpathlineto{\pgfqpoint{7.316188in}{1.798410in}}%
\pgfpathlineto{\pgfqpoint{7.324995in}{1.702331in}}%
\pgfpathlineto{\pgfqpoint{7.333802in}{1.647432in}}%
\pgfpathlineto{\pgfqpoint{7.342609in}{1.963133in}}%
\pgfpathlineto{\pgfqpoint{7.351416in}{2.134687in}}%
\pgfpathlineto{\pgfqpoint{7.360222in}{2.223908in}}%
\pgfpathlineto{\pgfqpoint{7.369029in}{2.299383in}}%
\pgfpathlineto{\pgfqpoint{7.377836in}{2.093507in}}%
\pgfpathlineto{\pgfqpoint{7.386643in}{1.750370in}}%
\pgfpathlineto{\pgfqpoint{7.395450in}{1.626856in}}%
\pgfpathlineto{\pgfqpoint{7.404257in}{2.189586in}}%
\pgfpathlineto{\pgfqpoint{7.413063in}{2.134687in}}%
\pgfpathlineto{\pgfqpoint{7.421870in}{2.045467in}}%
\pgfpathlineto{\pgfqpoint{7.430677in}{2.100365in}}%
\pgfpathlineto{\pgfqpoint{7.439484in}{2.059213in}}%
\pgfpathlineto{\pgfqpoint{7.448291in}{2.024891in}}%
\pgfpathlineto{\pgfqpoint{7.457097in}{2.354309in}}%
\pgfpathlineto{\pgfqpoint{7.465904in}{1.873913in}}%
\pgfpathlineto{\pgfqpoint{7.474711in}{1.887630in}}%
\pgfpathlineto{\pgfqpoint{7.483518in}{1.825873in}}%
\pgfpathlineto{\pgfqpoint{7.492325in}{1.448415in}}%
\pgfpathlineto{\pgfqpoint{7.501132in}{1.716076in}}%
\pgfpathlineto{\pgfqpoint{7.509938in}{1.921953in}}%
\pgfpathlineto{\pgfqpoint{7.518745in}{2.182727in}}%
\pgfpathlineto{\pgfqpoint{7.527552in}{2.066071in}}%
\pgfpathlineto{\pgfqpoint{7.545166in}{1.935670in}}%
\pgfpathlineto{\pgfqpoint{7.553972in}{1.812128in}}%
\pgfpathlineto{\pgfqpoint{7.562779in}{1.565071in}}%
\pgfpathlineto{\pgfqpoint{7.571586in}{1.482737in}}%
\pgfpathlineto{\pgfqpoint{7.580393in}{1.379799in}}%
\pgfpathlineto{\pgfqpoint{7.589200in}{1.352336in}}%
\pgfpathlineto{\pgfqpoint{7.598007in}{1.517031in}}%
\pgfpathlineto{\pgfqpoint{7.606813in}{1.626856in}}%
\pgfpathlineto{\pgfqpoint{7.615620in}{1.565071in}}%
\pgfpathlineto{\pgfqpoint{7.624427in}{1.578816in}}%
\pgfpathlineto{\pgfqpoint{7.633234in}{1.722935in}}%
\pgfpathlineto{\pgfqpoint{7.642041in}{1.908207in}}%
\pgfpathlineto{\pgfqpoint{7.659654in}{1.393517in}}%
\pgfpathlineto{\pgfqpoint{7.668461in}{1.633715in}}%
\pgfpathlineto{\pgfqpoint{7.677268in}{1.969992in}}%
\pgfpathlineto{\pgfqpoint{7.686075in}{1.908207in}}%
\pgfpathlineto{\pgfqpoint{7.694882in}{1.585675in}}%
\pgfpathlineto{\pgfqpoint{7.703688in}{1.441556in}}%
\pgfpathlineto{\pgfqpoint{7.712495in}{1.468992in}}%
\pgfpathlineto{\pgfqpoint{7.721302in}{1.345477in}}%
\pgfpathlineto{\pgfqpoint{7.738916in}{2.107252in}}%
\pgfpathlineto{\pgfqpoint{7.756529in}{1.942529in}}%
\pgfpathlineto{\pgfqpoint{7.765336in}{2.223908in}}%
\pgfpathlineto{\pgfqpoint{7.774143in}{2.169009in}}%
\pgfpathlineto{\pgfqpoint{7.782950in}{1.908207in}}%
\pgfpathlineto{\pgfqpoint{7.791757in}{2.024891in}}%
\pgfpathlineto{\pgfqpoint{7.800563in}{1.764116in}}%
\pgfpathlineto{\pgfqpoint{7.809370in}{1.668037in}}%
\pgfpathlineto{\pgfqpoint{7.818177in}{1.832732in}}%
\pgfpathlineto{\pgfqpoint{7.826984in}{2.114111in}}%
\pgfpathlineto{\pgfqpoint{7.835791in}{1.901348in}}%
\pgfpathlineto{\pgfqpoint{7.844597in}{1.860167in}}%
\pgfpathlineto{\pgfqpoint{7.853404in}{1.702331in}}%
\pgfpathlineto{\pgfqpoint{7.862211in}{1.427839in}}%
\pgfpathlineto{\pgfqpoint{7.871018in}{1.496455in}}%
\pgfpathlineto{\pgfqpoint{7.879825in}{1.510172in}}%
\pgfpathlineto{\pgfqpoint{7.888632in}{1.722935in}}%
\pgfpathlineto{\pgfqpoint{7.897438in}{1.722935in}}%
\pgfpathlineto{\pgfqpoint{7.906245in}{1.578816in}}%
\pgfpathlineto{\pgfqpoint{7.915052in}{1.736653in}}%
\pgfpathlineto{\pgfqpoint{7.923859in}{2.045467in}}%
\pgfpathlineto{\pgfqpoint{7.932666in}{2.175868in}}%
\pgfpathlineto{\pgfqpoint{7.941472in}{2.120970in}}%
\pgfpathlineto{\pgfqpoint{7.950279in}{1.887630in}}%
\pgfpathlineto{\pgfqpoint{7.959086in}{1.791551in}}%
\pgfpathlineto{\pgfqpoint{7.967893in}{1.668037in}}%
\pgfpathlineto{\pgfqpoint{7.976700in}{1.935670in}}%
\pgfpathlineto{\pgfqpoint{7.994313in}{2.169009in}}%
\pgfpathlineto{\pgfqpoint{8.003120in}{2.196445in}}%
\pgfpathlineto{\pgfqpoint{8.011927in}{2.004286in}}%
\pgfpathlineto{\pgfqpoint{8.020734in}{2.251343in}}%
\pgfpathlineto{\pgfqpoint{8.029541in}{2.011173in}}%
\pgfpathlineto{\pgfqpoint{8.038347in}{1.702331in}}%
\pgfpathlineto{\pgfqpoint{8.055961in}{1.674896in}}%
\pgfpathlineto{\pgfqpoint{8.073575in}{1.427839in}}%
\pgfpathlineto{\pgfqpoint{8.082382in}{1.565071in}}%
\pgfpathlineto{\pgfqpoint{8.091188in}{1.722935in}}%
\pgfpathlineto{\pgfqpoint{8.099995in}{2.066071in}}%
\pgfpathlineto{\pgfqpoint{8.108802in}{1.969992in}}%
\pgfpathlineto{\pgfqpoint{8.117609in}{2.086648in}}%
\pgfpathlineto{\pgfqpoint{8.126416in}{2.114111in}}%
\pgfpathlineto{\pgfqpoint{8.135222in}{1.963133in}}%
\pgfpathlineto{\pgfqpoint{8.144029in}{1.777834in}}%
\pgfpathlineto{\pgfqpoint{8.152836in}{1.805269in}}%
\pgfpathlineto{\pgfqpoint{8.161643in}{1.654291in}}%
\pgfpathlineto{\pgfqpoint{8.170450in}{1.867054in}}%
\pgfpathlineto{\pgfqpoint{8.179257in}{1.661150in}}%
\pgfpathlineto{\pgfqpoint{8.188063in}{1.867054in}}%
\pgfpathlineto{\pgfqpoint{8.196870in}{2.038608in}}%
\pgfpathlineto{\pgfqpoint{8.205677in}{1.969992in}}%
\pgfpathlineto{\pgfqpoint{8.214484in}{1.860167in}}%
\pgfpathlineto{\pgfqpoint{8.223291in}{2.107252in}}%
\pgfpathlineto{\pgfqpoint{8.232097in}{2.004286in}}%
\pgfpathlineto{\pgfqpoint{8.240904in}{1.565071in}}%
\pgfpathlineto{\pgfqpoint{8.249711in}{1.853308in}}%
\pgfpathlineto{\pgfqpoint{8.258518in}{2.457247in}}%
\pgfpathlineto{\pgfqpoint{8.267325in}{2.368027in}}%
\pgfpathlineto{\pgfqpoint{8.284938in}{2.114111in}}%
\pgfpathlineto{\pgfqpoint{8.293745in}{1.956246in}}%
\pgfpathlineto{\pgfqpoint{8.302552in}{1.503314in}}%
\pgfpathlineto{\pgfqpoint{8.311359in}{1.825873in}}%
\pgfpathlineto{\pgfqpoint{8.320166in}{1.633715in}}%
\pgfpathlineto{\pgfqpoint{8.328972in}{1.880772in}}%
\pgfpathlineto{\pgfqpoint{8.337779in}{1.880772in}}%
\pgfpathlineto{\pgfqpoint{8.346586in}{1.764116in}}%
\pgfpathlineto{\pgfqpoint{8.355393in}{1.743512in}}%
\pgfpathlineto{\pgfqpoint{8.364200in}{1.846450in}}%
\pgfpathlineto{\pgfqpoint{8.373007in}{2.162151in}}%
\pgfpathlineto{\pgfqpoint{8.381813in}{2.024891in}}%
\pgfpathlineto{\pgfqpoint{8.390620in}{2.114111in}}%
\pgfpathlineto{\pgfqpoint{8.399427in}{1.894489in}}%
\pgfpathlineto{\pgfqpoint{8.408234in}{1.743512in}}%
\pgfpathlineto{\pgfqpoint{8.417041in}{1.846450in}}%
\pgfpathlineto{\pgfqpoint{8.425847in}{2.004286in}}%
\pgfpathlineto{\pgfqpoint{8.434654in}{2.052326in}}%
\pgfpathlineto{\pgfqpoint{8.443461in}{2.038608in}}%
\pgfpathlineto{\pgfqpoint{8.452268in}{2.409207in}}%
\pgfpathlineto{\pgfqpoint{8.461075in}{2.217049in}}%
\pgfpathlineto{\pgfqpoint{8.469882in}{1.935670in}}%
\pgfpathlineto{\pgfqpoint{8.478688in}{2.155292in}}%
\pgfpathlineto{\pgfqpoint{8.487495in}{1.880772in}}%
\pgfpathlineto{\pgfqpoint{8.496302in}{2.004286in}}%
\pgfpathlineto{\pgfqpoint{8.505109in}{1.976851in}}%
\pgfpathlineto{\pgfqpoint{8.513916in}{1.880772in}}%
\pgfpathlineto{\pgfqpoint{8.522722in}{1.736653in}}%
\pgfpathlineto{\pgfqpoint{8.531529in}{1.990569in}}%
\pgfpathlineto{\pgfqpoint{8.540336in}{2.004286in}}%
\pgfpathlineto{\pgfqpoint{8.549143in}{2.079789in}}%
\pgfpathlineto{\pgfqpoint{8.557950in}{1.750370in}}%
\pgfpathlineto{\pgfqpoint{8.566757in}{1.839591in}}%
\pgfpathlineto{\pgfqpoint{8.575563in}{1.777834in}}%
\pgfpathlineto{\pgfqpoint{8.584370in}{1.921953in}}%
\pgfpathlineto{\pgfqpoint{8.593177in}{1.606252in}}%
\pgfpathlineto{\pgfqpoint{8.601984in}{1.716076in}}%
\pgfpathlineto{\pgfqpoint{8.619597in}{2.018032in}}%
\pgfpathlineto{\pgfqpoint{8.628404in}{1.805269in}}%
\pgfpathlineto{\pgfqpoint{8.637211in}{1.956246in}}%
\pgfpathlineto{\pgfqpoint{8.646018in}{1.949388in}}%
\pgfpathlineto{\pgfqpoint{8.654825in}{2.004286in}}%
\pgfpathlineto{\pgfqpoint{8.663632in}{1.880772in}}%
\pgfpathlineto{\pgfqpoint{8.672438in}{1.839591in}}%
\pgfpathlineto{\pgfqpoint{8.681245in}{1.496455in}}%
\pgfpathlineto{\pgfqpoint{8.690052in}{1.530777in}}%
\pgfpathlineto{\pgfqpoint{8.698859in}{1.537636in}}%
\pgfpathlineto{\pgfqpoint{8.707666in}{1.359195in}}%
\pgfpathlineto{\pgfqpoint{8.716472in}{1.420952in}}%
\pgfpathlineto{\pgfqpoint{8.725279in}{1.407234in}}%
\pgfpathlineto{\pgfqpoint{8.734086in}{1.887630in}}%
\pgfpathlineto{\pgfqpoint{8.742893in}{2.093507in}}%
\pgfpathlineto{\pgfqpoint{8.751700in}{2.114111in}}%
\pgfpathlineto{\pgfqpoint{8.760507in}{1.839591in}}%
\pgfpathlineto{\pgfqpoint{8.769313in}{1.983710in}}%
\pgfpathlineto{\pgfqpoint{8.778120in}{1.839591in}}%
\pgfpathlineto{\pgfqpoint{8.786927in}{1.832732in}}%
\pgfpathlineto{\pgfqpoint{8.795734in}{1.873913in}}%
\pgfpathlineto{\pgfqpoint{8.813347in}{2.416066in}}%
\pgfpathlineto{\pgfqpoint{8.822154in}{2.292524in}}%
\pgfpathlineto{\pgfqpoint{8.830961in}{2.498428in}}%
\pgfpathlineto{\pgfqpoint{8.839768in}{2.326846in}}%
\pgfpathlineto{\pgfqpoint{8.848575in}{2.628801in}}%
\pgfpathlineto{\pgfqpoint{8.857382in}{2.450388in}}%
\pgfpathlineto{\pgfqpoint{8.866188in}{2.601366in}}%
\pgfpathlineto{\pgfqpoint{8.874995in}{2.587620in}}%
\pgfpathlineto{\pgfqpoint{8.883802in}{2.066071in}}%
\pgfpathlineto{\pgfqpoint{8.892609in}{2.134687in}}%
\pgfpathlineto{\pgfqpoint{8.901416in}{1.976851in}}%
\pgfpathlineto{\pgfqpoint{8.910222in}{1.688613in}}%
\pgfpathlineto{\pgfqpoint{8.919029in}{1.661150in}}%
\pgfpathlineto{\pgfqpoint{8.927836in}{1.880772in}}%
\pgfpathlineto{\pgfqpoint{8.936643in}{2.031749in}}%
\pgfpathlineto{\pgfqpoint{8.945450in}{2.079789in}}%
\pgfpathlineto{\pgfqpoint{8.954257in}{1.928811in}}%
\pgfpathlineto{\pgfqpoint{8.963063in}{1.942529in}}%
\pgfpathlineto{\pgfqpoint{8.971870in}{2.004286in}}%
\pgfpathlineto{\pgfqpoint{8.980677in}{2.313128in}}%
\pgfpathlineto{\pgfqpoint{8.989484in}{1.969992in}}%
\pgfpathlineto{\pgfqpoint{8.998291in}{1.695472in}}%
\pgfpathlineto{\pgfqpoint{9.007097in}{1.825873in}}%
\pgfpathlineto{\pgfqpoint{9.015904in}{1.468992in}}%
\pgfpathlineto{\pgfqpoint{9.024711in}{1.626856in}}%
\pgfpathlineto{\pgfqpoint{9.033518in}{1.832732in}}%
\pgfpathlineto{\pgfqpoint{9.042325in}{1.695472in}}%
\pgfpathlineto{\pgfqpoint{9.051132in}{1.928811in}}%
\pgfpathlineto{\pgfqpoint{9.059938in}{1.695472in}}%
\pgfpathlineto{\pgfqpoint{9.068745in}{1.386658in}}%
\pgfpathlineto{\pgfqpoint{9.077552in}{1.311155in}}%
\pgfpathlineto{\pgfqpoint{9.086359in}{1.022917in}}%
\pgfpathlineto{\pgfqpoint{9.095166in}{1.249398in}}%
\pgfpathlineto{\pgfqpoint{9.103972in}{1.441556in}}%
\pgfpathlineto{\pgfqpoint{9.112779in}{1.764116in}}%
\pgfpathlineto{\pgfqpoint{9.121586in}{1.770975in}}%
\pgfpathlineto{\pgfqpoint{9.130393in}{1.695472in}}%
\pgfpathlineto{\pgfqpoint{9.139200in}{1.990569in}}%
\pgfpathlineto{\pgfqpoint{9.148007in}{1.873913in}}%
\pgfpathlineto{\pgfqpoint{9.156813in}{1.853308in}}%
\pgfpathlineto{\pgfqpoint{9.165620in}{1.688613in}}%
\pgfpathlineto{\pgfqpoint{9.174427in}{1.661150in}}%
\pgfpathlineto{\pgfqpoint{9.183234in}{1.668037in}}%
\pgfpathlineto{\pgfqpoint{9.192041in}{2.292524in}}%
\pgfpathlineto{\pgfqpoint{9.200847in}{2.107252in}}%
\pgfpathlineto{\pgfqpoint{9.209654in}{2.038608in}}%
\pgfpathlineto{\pgfqpoint{9.218461in}{2.120970in}}%
\pgfpathlineto{\pgfqpoint{9.227268in}{2.059213in}}%
\pgfpathlineto{\pgfqpoint{9.236075in}{2.443501in}}%
\pgfpathlineto{\pgfqpoint{9.244882in}{2.512146in}}%
\pgfpathlineto{\pgfqpoint{9.253688in}{2.265089in}}%
\pgfpathlineto{\pgfqpoint{9.262495in}{2.210190in}}%
\pgfpathlineto{\pgfqpoint{9.271302in}{2.038608in}}%
\pgfpathlineto{\pgfqpoint{9.280109in}{2.093507in}}%
\pgfpathlineto{\pgfqpoint{9.297722in}{1.777834in}}%
\pgfpathlineto{\pgfqpoint{9.306529in}{1.468992in}}%
\pgfpathlineto{\pgfqpoint{9.315336in}{1.942529in}}%
\pgfpathlineto{\pgfqpoint{9.324143in}{2.018032in}}%
\pgfpathlineto{\pgfqpoint{9.332950in}{2.107252in}}%
\pgfpathlineto{\pgfqpoint{9.341757in}{1.736653in}}%
\pgfpathlineto{\pgfqpoint{9.350563in}{1.846450in}}%
\pgfpathlineto{\pgfqpoint{9.368177in}{2.203303in}}%
\pgfpathlineto{\pgfqpoint{9.376984in}{2.230767in}}%
\pgfpathlineto{\pgfqpoint{9.385791in}{1.942529in}}%
\pgfpathlineto{\pgfqpoint{9.394597in}{1.976851in}}%
\pgfpathlineto{\pgfqpoint{9.403404in}{2.018032in}}%
\pgfpathlineto{\pgfqpoint{9.412211in}{1.873913in}}%
\pgfpathlineto{\pgfqpoint{9.421018in}{2.052326in}}%
\pgfpathlineto{\pgfqpoint{9.429825in}{1.901348in}}%
\pgfpathlineto{\pgfqpoint{9.438632in}{2.203303in}}%
\pgfpathlineto{\pgfqpoint{9.447438in}{2.045467in}}%
\pgfpathlineto{\pgfqpoint{9.456245in}{1.997427in}}%
\pgfpathlineto{\pgfqpoint{9.465052in}{1.894489in}}%
\pgfpathlineto{\pgfqpoint{9.473859in}{1.860167in}}%
\pgfpathlineto{\pgfqpoint{9.482666in}{1.983710in}}%
\pgfpathlineto{\pgfqpoint{9.491472in}{2.189586in}}%
\pgfpathlineto{\pgfqpoint{9.500279in}{2.285665in}}%
\pgfpathlineto{\pgfqpoint{9.509086in}{2.011173in}}%
\pgfpathlineto{\pgfqpoint{9.517893in}{2.107252in}}%
\pgfpathlineto{\pgfqpoint{9.526700in}{2.388603in}}%
\pgfpathlineto{\pgfqpoint{9.535507in}{2.313128in}}%
\pgfpathlineto{\pgfqpoint{9.553120in}{2.580761in}}%
\pgfpathlineto{\pgfqpoint{9.561927in}{2.450388in}}%
\pgfpathlineto{\pgfqpoint{9.570734in}{1.983710in}}%
\pgfpathlineto{\pgfqpoint{9.579541in}{1.702331in}}%
\pgfpathlineto{\pgfqpoint{9.588347in}{2.217049in}}%
\pgfpathlineto{\pgfqpoint{9.597154in}{1.777834in}}%
\pgfpathlineto{\pgfqpoint{9.605961in}{1.695472in}}%
\pgfpathlineto{\pgfqpoint{9.614768in}{1.468992in}}%
\pgfpathlineto{\pgfqpoint{9.623575in}{1.427839in}}%
\pgfpathlineto{\pgfqpoint{9.632382in}{1.619997in}}%
\pgfpathlineto{\pgfqpoint{9.641188in}{1.647432in}}%
\pgfpathlineto{\pgfqpoint{9.649995in}{1.743512in}}%
\pgfpathlineto{\pgfqpoint{9.658802in}{2.114111in}}%
\pgfpathlineto{\pgfqpoint{9.667609in}{2.059213in}}%
\pgfpathlineto{\pgfqpoint{9.676416in}{2.512146in}}%
\pgfpathlineto{\pgfqpoint{9.685222in}{2.368027in}}%
\pgfpathlineto{\pgfqpoint{9.694029in}{2.436643in}}%
\pgfpathlineto{\pgfqpoint{9.702836in}{2.319987in}}%
\pgfpathlineto{\pgfqpoint{9.711643in}{2.182727in}}%
\pgfpathlineto{\pgfqpoint{9.720450in}{2.429784in}}%
\pgfpathlineto{\pgfqpoint{9.729257in}{2.093507in}}%
\pgfpathlineto{\pgfqpoint{9.738063in}{2.024891in}}%
\pgfpathlineto{\pgfqpoint{9.746870in}{1.784692in}}%
\pgfpathlineto{\pgfqpoint{9.755677in}{1.908207in}}%
\pgfpathlineto{\pgfqpoint{9.764484in}{1.681754in}}%
\pgfpathlineto{\pgfqpoint{9.773291in}{1.880772in}}%
\pgfpathlineto{\pgfqpoint{9.782097in}{2.182727in}}%
\pgfpathlineto{\pgfqpoint{9.790904in}{1.853308in}}%
\pgfpathlineto{\pgfqpoint{9.799711in}{1.585675in}}%
\pgfpathlineto{\pgfqpoint{9.808518in}{1.805269in}}%
\pgfpathlineto{\pgfqpoint{9.817325in}{1.873913in}}%
\pgfpathlineto{\pgfqpoint{9.826132in}{1.674896in}}%
\pgfpathlineto{\pgfqpoint{9.834938in}{1.510172in}}%
\pgfpathlineto{\pgfqpoint{9.843745in}{1.839591in}}%
\pgfpathlineto{\pgfqpoint{9.852552in}{2.052326in}}%
\pgfpathlineto{\pgfqpoint{9.861359in}{1.894489in}}%
\pgfpathlineto{\pgfqpoint{9.870166in}{1.661150in}}%
\pgfpathlineto{\pgfqpoint{9.878972in}{1.819015in}}%
\pgfpathlineto{\pgfqpoint{9.887779in}{1.873913in}}%
\pgfpathlineto{\pgfqpoint{9.896586in}{1.798410in}}%
\pgfpathlineto{\pgfqpoint{9.905393in}{1.921953in}}%
\pgfpathlineto{\pgfqpoint{9.914200in}{1.880772in}}%
\pgfpathlineto{\pgfqpoint{9.923007in}{2.011173in}}%
\pgfpathlineto{\pgfqpoint{9.931813in}{1.784692in}}%
\pgfpathlineto{\pgfqpoint{9.940620in}{1.784692in}}%
\pgfpathlineto{\pgfqpoint{9.949427in}{1.894489in}}%
\pgfpathlineto{\pgfqpoint{9.949427in}{1.894489in}}%
\pgfusepath{stroke}%
\end{pgfscope}%
\begin{pgfscope}%
\pgfpathrectangle{\pgfqpoint{0.702268in}{0.521603in}}{\pgfqpoint{9.687500in}{4.235000in}}%
\pgfusepath{clip}%
\pgfsetrectcap%
\pgfsetroundjoin%
\pgfsetlinewidth{0.501875pt}%
\definecolor{currentstroke}{rgb}{0.501961,0.501961,0.501961}%
\pgfsetstrokecolor{currentstroke}%
\pgfsetstrokeopacity{0.250000}%
\pgfsetdash{}{0pt}%
\pgfpathmoveto{\pgfqpoint{1.142609in}{4.269007in}}%
\pgfpathlineto{\pgfqpoint{1.151416in}{3.459192in}}%
\pgfpathlineto{\pgfqpoint{1.160222in}{2.868999in}}%
\pgfpathlineto{\pgfqpoint{1.169029in}{2.532722in}}%
\pgfpathlineto{\pgfqpoint{1.177836in}{2.299383in}}%
\pgfpathlineto{\pgfqpoint{1.186643in}{2.182727in}}%
\pgfpathlineto{\pgfqpoint{1.204257in}{1.626856in}}%
\pgfpathlineto{\pgfqpoint{1.213063in}{1.839591in}}%
\pgfpathlineto{\pgfqpoint{1.221870in}{1.846450in}}%
\pgfpathlineto{\pgfqpoint{1.230677in}{1.887630in}}%
\pgfpathlineto{\pgfqpoint{1.239484in}{2.114111in}}%
\pgfpathlineto{\pgfqpoint{1.248291in}{1.530777in}}%
\pgfpathlineto{\pgfqpoint{1.257097in}{1.695472in}}%
\pgfpathlineto{\pgfqpoint{1.265904in}{2.086648in}}%
\pgfpathlineto{\pgfqpoint{1.274711in}{2.024891in}}%
\pgfpathlineto{\pgfqpoint{1.283518in}{2.162151in}}%
\pgfpathlineto{\pgfqpoint{1.292325in}{2.169009in}}%
\pgfpathlineto{\pgfqpoint{1.301132in}{1.928811in}}%
\pgfpathlineto{\pgfqpoint{1.309938in}{2.134687in}}%
\pgfpathlineto{\pgfqpoint{1.318745in}{1.860167in}}%
\pgfpathlineto{\pgfqpoint{1.327552in}{1.695472in}}%
\pgfpathlineto{\pgfqpoint{1.336359in}{1.791551in}}%
\pgfpathlineto{\pgfqpoint{1.345166in}{1.695472in}}%
\pgfpathlineto{\pgfqpoint{1.353972in}{1.949388in}}%
\pgfpathlineto{\pgfqpoint{1.362779in}{2.278806in}}%
\pgfpathlineto{\pgfqpoint{1.371586in}{1.949388in}}%
\pgfpathlineto{\pgfqpoint{1.380393in}{1.832732in}}%
\pgfpathlineto{\pgfqpoint{1.389200in}{2.354309in}}%
\pgfpathlineto{\pgfqpoint{1.398007in}{2.319987in}}%
\pgfpathlineto{\pgfqpoint{1.406813in}{2.155292in}}%
\pgfpathlineto{\pgfqpoint{1.415620in}{1.681754in}}%
\pgfpathlineto{\pgfqpoint{1.424427in}{1.853308in}}%
\pgfpathlineto{\pgfqpoint{1.433234in}{1.613110in}}%
\pgfpathlineto{\pgfqpoint{1.442041in}{1.688613in}}%
\pgfpathlineto{\pgfqpoint{1.450847in}{1.647432in}}%
\pgfpathlineto{\pgfqpoint{1.459654in}{1.544494in}}%
\pgfpathlineto{\pgfqpoint{1.468461in}{1.592534in}}%
\pgfpathlineto{\pgfqpoint{1.477268in}{1.475878in}}%
\pgfpathlineto{\pgfqpoint{1.486075in}{1.400376in}}%
\pgfpathlineto{\pgfqpoint{1.494882in}{1.695472in}}%
\pgfpathlineto{\pgfqpoint{1.503688in}{1.901348in}}%
\pgfpathlineto{\pgfqpoint{1.512495in}{2.018032in}}%
\pgfpathlineto{\pgfqpoint{1.521302in}{2.072930in}}%
\pgfpathlineto{\pgfqpoint{1.530109in}{2.278806in}}%
\pgfpathlineto{\pgfqpoint{1.538916in}{2.223908in}}%
\pgfpathlineto{\pgfqpoint{1.547722in}{2.443501in}}%
\pgfpathlineto{\pgfqpoint{1.556529in}{2.189586in}}%
\pgfpathlineto{\pgfqpoint{1.565336in}{1.764116in}}%
\pgfpathlineto{\pgfqpoint{1.574143in}{1.908207in}}%
\pgfpathlineto{\pgfqpoint{1.582950in}{1.990569in}}%
\pgfpathlineto{\pgfqpoint{1.591757in}{1.873913in}}%
\pgfpathlineto{\pgfqpoint{1.600563in}{1.729794in}}%
\pgfpathlineto{\pgfqpoint{1.609370in}{1.668037in}}%
\pgfpathlineto{\pgfqpoint{1.626984in}{2.690558in}}%
\pgfpathlineto{\pgfqpoint{1.635791in}{2.937643in}}%
\pgfpathlineto{\pgfqpoint{1.644597in}{2.814101in}}%
\pgfpathlineto{\pgfqpoint{1.653404in}{2.642547in}}%
\pgfpathlineto{\pgfqpoint{1.662211in}{2.512146in}}%
\pgfpathlineto{\pgfqpoint{1.671018in}{2.573903in}}%
\pgfpathlineto{\pgfqpoint{1.679825in}{2.120970in}}%
\pgfpathlineto{\pgfqpoint{1.688632in}{1.976851in}}%
\pgfpathlineto{\pgfqpoint{1.697438in}{1.860167in}}%
\pgfpathlineto{\pgfqpoint{1.706245in}{2.031749in}}%
\pgfpathlineto{\pgfqpoint{1.715052in}{1.743512in}}%
\pgfpathlineto{\pgfqpoint{1.723859in}{1.606252in}}%
\pgfpathlineto{\pgfqpoint{1.732666in}{1.812128in}}%
\pgfpathlineto{\pgfqpoint{1.741472in}{1.619997in}}%
\pgfpathlineto{\pgfqpoint{1.750279in}{1.812128in}}%
\pgfpathlineto{\pgfqpoint{1.759086in}{2.148405in}}%
\pgfpathlineto{\pgfqpoint{1.767893in}{2.237625in}}%
\pgfpathlineto{\pgfqpoint{1.776700in}{1.990569in}}%
\pgfpathlineto{\pgfqpoint{1.785507in}{2.031749in}}%
\pgfpathlineto{\pgfqpoint{1.794313in}{2.052326in}}%
\pgfpathlineto{\pgfqpoint{1.803120in}{2.429784in}}%
\pgfpathlineto{\pgfqpoint{1.811927in}{2.319987in}}%
\pgfpathlineto{\pgfqpoint{1.820734in}{2.429784in}}%
\pgfpathlineto{\pgfqpoint{1.829541in}{2.718022in}}%
\pgfpathlineto{\pgfqpoint{1.838347in}{2.251343in}}%
\pgfpathlineto{\pgfqpoint{1.847154in}{2.470965in}}%
\pgfpathlineto{\pgfqpoint{1.855961in}{2.539581in}}%
\pgfpathlineto{\pgfqpoint{1.864768in}{2.313128in}}%
\pgfpathlineto{\pgfqpoint{1.873575in}{1.846450in}}%
\pgfpathlineto{\pgfqpoint{1.882382in}{2.004286in}}%
\pgfpathlineto{\pgfqpoint{1.891188in}{1.921953in}}%
\pgfpathlineto{\pgfqpoint{1.899995in}{1.921953in}}%
\pgfpathlineto{\pgfqpoint{1.908802in}{1.976851in}}%
\pgfpathlineto{\pgfqpoint{1.917609in}{2.127829in}}%
\pgfpathlineto{\pgfqpoint{1.926416in}{2.120970in}}%
\pgfpathlineto{\pgfqpoint{1.935222in}{1.894489in}}%
\pgfpathlineto{\pgfqpoint{1.944029in}{1.716076in}}%
\pgfpathlineto{\pgfqpoint{1.952836in}{1.736653in}}%
\pgfpathlineto{\pgfqpoint{1.961643in}{2.059213in}}%
\pgfpathlineto{\pgfqpoint{1.970450in}{1.764116in}}%
\pgfpathlineto{\pgfqpoint{1.979257in}{1.661150in}}%
\pgfpathlineto{\pgfqpoint{1.988063in}{1.805269in}}%
\pgfpathlineto{\pgfqpoint{1.996870in}{1.304296in}}%
\pgfpathlineto{\pgfqpoint{2.005677in}{1.654291in}}%
\pgfpathlineto{\pgfqpoint{2.014484in}{1.819015in}}%
\pgfpathlineto{\pgfqpoint{2.023291in}{2.093507in}}%
\pgfpathlineto{\pgfqpoint{2.032097in}{1.990569in}}%
\pgfpathlineto{\pgfqpoint{2.040904in}{1.860167in}}%
\pgfpathlineto{\pgfqpoint{2.049711in}{1.791551in}}%
\pgfpathlineto{\pgfqpoint{2.058518in}{1.873913in}}%
\pgfpathlineto{\pgfqpoint{2.067325in}{1.764116in}}%
\pgfpathlineto{\pgfqpoint{2.076132in}{1.805269in}}%
\pgfpathlineto{\pgfqpoint{2.084938in}{1.839591in}}%
\pgfpathlineto{\pgfqpoint{2.093745in}{1.619997in}}%
\pgfpathlineto{\pgfqpoint{2.102552in}{1.510172in}}%
\pgfpathlineto{\pgfqpoint{2.111359in}{1.729794in}}%
\pgfpathlineto{\pgfqpoint{2.120166in}{1.750370in}}%
\pgfpathlineto{\pgfqpoint{2.128972in}{1.921953in}}%
\pgfpathlineto{\pgfqpoint{2.137779in}{1.942529in}}%
\pgfpathlineto{\pgfqpoint{2.146586in}{2.052326in}}%
\pgfpathlineto{\pgfqpoint{2.155393in}{1.908207in}}%
\pgfpathlineto{\pgfqpoint{2.164200in}{1.880772in}}%
\pgfpathlineto{\pgfqpoint{2.173007in}{1.867054in}}%
\pgfpathlineto{\pgfqpoint{2.181813in}{1.956246in}}%
\pgfpathlineto{\pgfqpoint{2.190620in}{2.031749in}}%
\pgfpathlineto{\pgfqpoint{2.199427in}{2.018032in}}%
\pgfpathlineto{\pgfqpoint{2.208234in}{2.182727in}}%
\pgfpathlineto{\pgfqpoint{2.217041in}{2.072930in}}%
\pgfpathlineto{\pgfqpoint{2.225847in}{1.819015in}}%
\pgfpathlineto{\pgfqpoint{2.234654in}{2.182727in}}%
\pgfpathlineto{\pgfqpoint{2.243461in}{2.354309in}}%
\pgfpathlineto{\pgfqpoint{2.261075in}{1.956246in}}%
\pgfpathlineto{\pgfqpoint{2.269882in}{1.963133in}}%
\pgfpathlineto{\pgfqpoint{2.278688in}{1.455274in}}%
\pgfpathlineto{\pgfqpoint{2.287495in}{1.427839in}}%
\pgfpathlineto{\pgfqpoint{2.296302in}{1.606252in}}%
\pgfpathlineto{\pgfqpoint{2.305109in}{1.716076in}}%
\pgfpathlineto{\pgfqpoint{2.313916in}{1.867054in}}%
\pgfpathlineto{\pgfqpoint{2.322722in}{1.894489in}}%
\pgfpathlineto{\pgfqpoint{2.331529in}{1.743512in}}%
\pgfpathlineto{\pgfqpoint{2.340336in}{1.633715in}}%
\pgfpathlineto{\pgfqpoint{2.349143in}{1.668037in}}%
\pgfpathlineto{\pgfqpoint{2.357950in}{1.969992in}}%
\pgfpathlineto{\pgfqpoint{2.366757in}{1.915094in}}%
\pgfpathlineto{\pgfqpoint{2.375563in}{2.361168in}}%
\pgfpathlineto{\pgfqpoint{2.384370in}{2.436643in}}%
\pgfpathlineto{\pgfqpoint{2.393177in}{1.969992in}}%
\pgfpathlineto{\pgfqpoint{2.401984in}{1.853308in}}%
\pgfpathlineto{\pgfqpoint{2.410791in}{1.894489in}}%
\pgfpathlineto{\pgfqpoint{2.419597in}{1.716076in}}%
\pgfpathlineto{\pgfqpoint{2.428404in}{1.619997in}}%
\pgfpathlineto{\pgfqpoint{2.437211in}{1.709190in}}%
\pgfpathlineto{\pgfqpoint{2.446018in}{1.462133in}}%
\pgfpathlineto{\pgfqpoint{2.454825in}{1.372912in}}%
\pgfpathlineto{\pgfqpoint{2.463632in}{1.558212in}}%
\pgfpathlineto{\pgfqpoint{2.472438in}{1.990569in}}%
\pgfpathlineto{\pgfqpoint{2.481245in}{1.606252in}}%
\pgfpathlineto{\pgfqpoint{2.490052in}{1.338618in}}%
\pgfpathlineto{\pgfqpoint{2.498859in}{1.393517in}}%
\pgfpathlineto{\pgfqpoint{2.507666in}{1.592534in}}%
\pgfpathlineto{\pgfqpoint{2.516472in}{1.743512in}}%
\pgfpathlineto{\pgfqpoint{2.525279in}{1.640574in}}%
\pgfpathlineto{\pgfqpoint{2.534086in}{1.942529in}}%
\pgfpathlineto{\pgfqpoint{2.542893in}{1.860167in}}%
\pgfpathlineto{\pgfqpoint{2.551700in}{1.997427in}}%
\pgfpathlineto{\pgfqpoint{2.560507in}{2.086648in}}%
\pgfpathlineto{\pgfqpoint{2.569313in}{2.141546in}}%
\pgfpathlineto{\pgfqpoint{2.578120in}{2.045467in}}%
\pgfpathlineto{\pgfqpoint{2.586927in}{2.237625in}}%
\pgfpathlineto{\pgfqpoint{2.604541in}{1.770975in}}%
\pgfpathlineto{\pgfqpoint{2.613347in}{1.729794in}}%
\pgfpathlineto{\pgfqpoint{2.622154in}{2.299383in}}%
\pgfpathlineto{\pgfqpoint{2.630961in}{2.230767in}}%
\pgfpathlineto{\pgfqpoint{2.639768in}{2.313128in}}%
\pgfpathlineto{\pgfqpoint{2.648575in}{2.210190in}}%
\pgfpathlineto{\pgfqpoint{2.657382in}{2.093507in}}%
\pgfpathlineto{\pgfqpoint{2.666188in}{2.196445in}}%
\pgfpathlineto{\pgfqpoint{2.674995in}{2.045467in}}%
\pgfpathlineto{\pgfqpoint{2.683802in}{1.921953in}}%
\pgfpathlineto{\pgfqpoint{2.692609in}{1.523918in}}%
\pgfpathlineto{\pgfqpoint{2.701416in}{1.468992in}}%
\pgfpathlineto{\pgfqpoint{2.710222in}{1.523918in}}%
\pgfpathlineto{\pgfqpoint{2.719029in}{1.613110in}}%
\pgfpathlineto{\pgfqpoint{2.727836in}{2.093507in}}%
\pgfpathlineto{\pgfqpoint{2.736643in}{2.285665in}}%
\pgfpathlineto{\pgfqpoint{2.745450in}{2.196445in}}%
\pgfpathlineto{\pgfqpoint{2.754257in}{2.354309in}}%
\pgfpathlineto{\pgfqpoint{2.763063in}{2.608225in}}%
\pgfpathlineto{\pgfqpoint{2.771870in}{2.450388in}}%
\pgfpathlineto{\pgfqpoint{2.780677in}{2.649406in}}%
\pgfpathlineto{\pgfqpoint{2.789484in}{2.175868in}}%
\pgfpathlineto{\pgfqpoint{2.798291in}{1.983710in}}%
\pgfpathlineto{\pgfqpoint{2.815904in}{1.901348in}}%
\pgfpathlineto{\pgfqpoint{2.824711in}{1.825873in}}%
\pgfpathlineto{\pgfqpoint{2.833518in}{1.736653in}}%
\pgfpathlineto{\pgfqpoint{2.842325in}{1.880772in}}%
\pgfpathlineto{\pgfqpoint{2.851132in}{2.079789in}}%
\pgfpathlineto{\pgfqpoint{2.859938in}{2.237625in}}%
\pgfpathlineto{\pgfqpoint{2.868745in}{2.443501in}}%
\pgfpathlineto{\pgfqpoint{2.877552in}{1.997427in}}%
\pgfpathlineto{\pgfqpoint{2.886359in}{1.626856in}}%
\pgfpathlineto{\pgfqpoint{2.895166in}{1.709190in}}%
\pgfpathlineto{\pgfqpoint{2.903972in}{2.018032in}}%
\pgfpathlineto{\pgfqpoint{2.912779in}{1.983710in}}%
\pgfpathlineto{\pgfqpoint{2.921586in}{2.230767in}}%
\pgfpathlineto{\pgfqpoint{2.930393in}{2.024891in}}%
\pgfpathlineto{\pgfqpoint{2.939200in}{1.956246in}}%
\pgfpathlineto{\pgfqpoint{2.948007in}{1.935670in}}%
\pgfpathlineto{\pgfqpoint{2.956813in}{1.935670in}}%
\pgfpathlineto{\pgfqpoint{2.965620in}{1.867054in}}%
\pgfpathlineto{\pgfqpoint{2.974427in}{1.846450in}}%
\pgfpathlineto{\pgfqpoint{2.983234in}{1.928811in}}%
\pgfpathlineto{\pgfqpoint{2.992041in}{1.853308in}}%
\pgfpathlineto{\pgfqpoint{3.000847in}{1.743512in}}%
\pgfpathlineto{\pgfqpoint{3.009654in}{1.949388in}}%
\pgfpathlineto{\pgfqpoint{3.018461in}{1.688613in}}%
\pgfpathlineto{\pgfqpoint{3.027268in}{1.736653in}}%
\pgfpathlineto{\pgfqpoint{3.036075in}{1.640574in}}%
\pgfpathlineto{\pgfqpoint{3.044882in}{1.530777in}}%
\pgfpathlineto{\pgfqpoint{3.053688in}{1.489596in}}%
\pgfpathlineto{\pgfqpoint{3.062495in}{1.558212in}}%
\pgfpathlineto{\pgfqpoint{3.071302in}{2.018032in}}%
\pgfpathlineto{\pgfqpoint{3.080109in}{2.066071in}}%
\pgfpathlineto{\pgfqpoint{3.088916in}{1.853308in}}%
\pgfpathlineto{\pgfqpoint{3.097722in}{2.093507in}}%
\pgfpathlineto{\pgfqpoint{3.106529in}{1.688613in}}%
\pgfpathlineto{\pgfqpoint{3.115336in}{1.928811in}}%
\pgfpathlineto{\pgfqpoint{3.124143in}{1.867054in}}%
\pgfpathlineto{\pgfqpoint{3.132950in}{1.489596in}}%
\pgfpathlineto{\pgfqpoint{3.141757in}{1.434698in}}%
\pgfpathlineto{\pgfqpoint{3.150563in}{1.448415in}}%
\pgfpathlineto{\pgfqpoint{3.159370in}{1.235680in}}%
\pgfpathlineto{\pgfqpoint{3.168177in}{1.304296in}}%
\pgfpathlineto{\pgfqpoint{3.176984in}{1.496455in}}%
\pgfpathlineto{\pgfqpoint{3.185791in}{1.571958in}}%
\pgfpathlineto{\pgfqpoint{3.194597in}{1.990569in}}%
\pgfpathlineto{\pgfqpoint{3.203404in}{1.839591in}}%
\pgfpathlineto{\pgfqpoint{3.212211in}{1.633715in}}%
\pgfpathlineto{\pgfqpoint{3.221018in}{1.757229in}}%
\pgfpathlineto{\pgfqpoint{3.229825in}{1.853308in}}%
\pgfpathlineto{\pgfqpoint{3.238632in}{1.709190in}}%
\pgfpathlineto{\pgfqpoint{3.247438in}{1.921953in}}%
\pgfpathlineto{\pgfqpoint{3.256245in}{1.654291in}}%
\pgfpathlineto{\pgfqpoint{3.265052in}{1.565071in}}%
\pgfpathlineto{\pgfqpoint{3.273859in}{1.757229in}}%
\pgfpathlineto{\pgfqpoint{3.282666in}{2.004286in}}%
\pgfpathlineto{\pgfqpoint{3.291472in}{1.832732in}}%
\pgfpathlineto{\pgfqpoint{3.300279in}{2.011173in}}%
\pgfpathlineto{\pgfqpoint{3.309086in}{2.100365in}}%
\pgfpathlineto{\pgfqpoint{3.317893in}{1.853308in}}%
\pgfpathlineto{\pgfqpoint{3.326700in}{1.530777in}}%
\pgfpathlineto{\pgfqpoint{3.335507in}{1.585675in}}%
\pgfpathlineto{\pgfqpoint{3.344313in}{1.901348in}}%
\pgfpathlineto{\pgfqpoint{3.353120in}{1.969992in}}%
\pgfpathlineto{\pgfqpoint{3.361927in}{1.695472in}}%
\pgfpathlineto{\pgfqpoint{3.370734in}{1.791551in}}%
\pgfpathlineto{\pgfqpoint{3.379541in}{1.633715in}}%
\pgfpathlineto{\pgfqpoint{3.388347in}{1.606252in}}%
\pgfpathlineto{\pgfqpoint{3.397154in}{1.462133in}}%
\pgfpathlineto{\pgfqpoint{3.405961in}{1.592534in}}%
\pgfpathlineto{\pgfqpoint{3.414768in}{1.551353in}}%
\pgfpathlineto{\pgfqpoint{3.423575in}{1.880772in}}%
\pgfpathlineto{\pgfqpoint{3.432382in}{2.120970in}}%
\pgfpathlineto{\pgfqpoint{3.441188in}{1.695472in}}%
\pgfpathlineto{\pgfqpoint{3.449995in}{1.757229in}}%
\pgfpathlineto{\pgfqpoint{3.458802in}{1.887630in}}%
\pgfpathlineto{\pgfqpoint{3.467609in}{2.086648in}}%
\pgfpathlineto{\pgfqpoint{3.476416in}{1.894489in}}%
\pgfpathlineto{\pgfqpoint{3.485222in}{1.921953in}}%
\pgfpathlineto{\pgfqpoint{3.502836in}{1.455274in}}%
\pgfpathlineto{\pgfqpoint{3.511643in}{1.571958in}}%
\pgfpathlineto{\pgfqpoint{3.520450in}{1.661150in}}%
\pgfpathlineto{\pgfqpoint{3.529257in}{1.736653in}}%
\pgfpathlineto{\pgfqpoint{3.538063in}{1.750370in}}%
\pgfpathlineto{\pgfqpoint{3.546870in}{1.736653in}}%
\pgfpathlineto{\pgfqpoint{3.555677in}{1.983710in}}%
\pgfpathlineto{\pgfqpoint{3.564484in}{1.716076in}}%
\pgfpathlineto{\pgfqpoint{3.573291in}{1.729794in}}%
\pgfpathlineto{\pgfqpoint{3.582097in}{1.901348in}}%
\pgfpathlineto{\pgfqpoint{3.590904in}{1.949388in}}%
\pgfpathlineto{\pgfqpoint{3.599711in}{1.860167in}}%
\pgfpathlineto{\pgfqpoint{3.608518in}{1.812128in}}%
\pgfpathlineto{\pgfqpoint{3.617325in}{1.592534in}}%
\pgfpathlineto{\pgfqpoint{3.626132in}{1.640574in}}%
\pgfpathlineto{\pgfqpoint{3.634938in}{1.880772in}}%
\pgfpathlineto{\pgfqpoint{3.643745in}{2.004286in}}%
\pgfpathlineto{\pgfqpoint{3.652552in}{1.839591in}}%
\pgfpathlineto{\pgfqpoint{3.661359in}{1.956246in}}%
\pgfpathlineto{\pgfqpoint{3.678972in}{1.750370in}}%
\pgfpathlineto{\pgfqpoint{3.687779in}{1.702331in}}%
\pgfpathlineto{\pgfqpoint{3.696586in}{1.901348in}}%
\pgfpathlineto{\pgfqpoint{3.705393in}{1.736653in}}%
\pgfpathlineto{\pgfqpoint{3.714200in}{1.839591in}}%
\pgfpathlineto{\pgfqpoint{3.723007in}{2.066071in}}%
\pgfpathlineto{\pgfqpoint{3.731813in}{2.120970in}}%
\pgfpathlineto{\pgfqpoint{3.740620in}{2.347422in}}%
\pgfpathlineto{\pgfqpoint{3.749427in}{1.983710in}}%
\pgfpathlineto{\pgfqpoint{3.758234in}{2.237625in}}%
\pgfpathlineto{\pgfqpoint{3.775847in}{1.716076in}}%
\pgfpathlineto{\pgfqpoint{3.784654in}{1.949388in}}%
\pgfpathlineto{\pgfqpoint{3.793461in}{1.832732in}}%
\pgfpathlineto{\pgfqpoint{3.802268in}{1.805269in}}%
\pgfpathlineto{\pgfqpoint{3.811075in}{1.839591in}}%
\pgfpathlineto{\pgfqpoint{3.819882in}{1.921953in}}%
\pgfpathlineto{\pgfqpoint{3.828688in}{1.928811in}}%
\pgfpathlineto{\pgfqpoint{3.837495in}{1.956246in}}%
\pgfpathlineto{\pgfqpoint{3.846302in}{2.100365in}}%
\pgfpathlineto{\pgfqpoint{3.855109in}{2.031749in}}%
\pgfpathlineto{\pgfqpoint{3.863916in}{1.832732in}}%
\pgfpathlineto{\pgfqpoint{3.872722in}{1.812128in}}%
\pgfpathlineto{\pgfqpoint{3.881529in}{1.702331in}}%
\pgfpathlineto{\pgfqpoint{3.890336in}{2.072930in}}%
\pgfpathlineto{\pgfqpoint{3.899143in}{1.901348in}}%
\pgfpathlineto{\pgfqpoint{3.907950in}{2.038608in}}%
\pgfpathlineto{\pgfqpoint{3.916757in}{1.791551in}}%
\pgfpathlineto{\pgfqpoint{3.925563in}{1.585675in}}%
\pgfpathlineto{\pgfqpoint{3.934370in}{1.578816in}}%
\pgfpathlineto{\pgfqpoint{3.943177in}{1.695472in}}%
\pgfpathlineto{\pgfqpoint{3.951984in}{1.668037in}}%
\pgfpathlineto{\pgfqpoint{3.960791in}{1.832732in}}%
\pgfpathlineto{\pgfqpoint{3.969597in}{1.908207in}}%
\pgfpathlineto{\pgfqpoint{3.978404in}{2.086648in}}%
\pgfpathlineto{\pgfqpoint{3.987211in}{2.175868in}}%
\pgfpathlineto{\pgfqpoint{3.996018in}{2.155292in}}%
\pgfpathlineto{\pgfqpoint{4.004825in}{2.587620in}}%
\pgfpathlineto{\pgfqpoint{4.013632in}{2.656264in}}%
\pgfpathlineto{\pgfqpoint{4.022438in}{2.539581in}}%
\pgfpathlineto{\pgfqpoint{4.031245in}{2.498428in}}%
\pgfpathlineto{\pgfqpoint{4.040052in}{2.319987in}}%
\pgfpathlineto{\pgfqpoint{4.048859in}{2.429784in}}%
\pgfpathlineto{\pgfqpoint{4.057666in}{2.141546in}}%
\pgfpathlineto{\pgfqpoint{4.066472in}{2.120970in}}%
\pgfpathlineto{\pgfqpoint{4.075279in}{1.743512in}}%
\pgfpathlineto{\pgfqpoint{4.084086in}{1.585675in}}%
\pgfpathlineto{\pgfqpoint{4.092893in}{1.908207in}}%
\pgfpathlineto{\pgfqpoint{4.101700in}{1.983710in}}%
\pgfpathlineto{\pgfqpoint{4.110507in}{1.976851in}}%
\pgfpathlineto{\pgfqpoint{4.119313in}{2.217049in}}%
\pgfpathlineto{\pgfqpoint{4.128120in}{2.059213in}}%
\pgfpathlineto{\pgfqpoint{4.136927in}{2.100365in}}%
\pgfpathlineto{\pgfqpoint{4.145734in}{2.230767in}}%
\pgfpathlineto{\pgfqpoint{4.154541in}{2.148405in}}%
\pgfpathlineto{\pgfqpoint{4.163347in}{1.722935in}}%
\pgfpathlineto{\pgfqpoint{4.172154in}{1.764116in}}%
\pgfpathlineto{\pgfqpoint{4.180961in}{1.750370in}}%
\pgfpathlineto{\pgfqpoint{4.189768in}{2.210190in}}%
\pgfpathlineto{\pgfqpoint{4.207382in}{1.921953in}}%
\pgfpathlineto{\pgfqpoint{4.216188in}{2.018032in}}%
\pgfpathlineto{\pgfqpoint{4.224995in}{2.155292in}}%
\pgfpathlineto{\pgfqpoint{4.233802in}{1.942529in}}%
\pgfpathlineto{\pgfqpoint{4.242609in}{1.825873in}}%
\pgfpathlineto{\pgfqpoint{4.251416in}{1.880772in}}%
\pgfpathlineto{\pgfqpoint{4.260222in}{1.798410in}}%
\pgfpathlineto{\pgfqpoint{4.269029in}{1.736653in}}%
\pgfpathlineto{\pgfqpoint{4.277836in}{1.819015in}}%
\pgfpathlineto{\pgfqpoint{4.286643in}{1.688613in}}%
\pgfpathlineto{\pgfqpoint{4.295450in}{2.148405in}}%
\pgfpathlineto{\pgfqpoint{4.304257in}{1.839591in}}%
\pgfpathlineto{\pgfqpoint{4.313063in}{1.585675in}}%
\pgfpathlineto{\pgfqpoint{4.321870in}{1.777834in}}%
\pgfpathlineto{\pgfqpoint{4.330677in}{1.359195in}}%
\pgfpathlineto{\pgfqpoint{4.339484in}{1.551353in}}%
\pgfpathlineto{\pgfqpoint{4.348291in}{1.455274in}}%
\pgfpathlineto{\pgfqpoint{4.365904in}{2.059213in}}%
\pgfpathlineto{\pgfqpoint{4.374711in}{2.038608in}}%
\pgfpathlineto{\pgfqpoint{4.383518in}{2.072930in}}%
\pgfpathlineto{\pgfqpoint{4.392325in}{2.381744in}}%
\pgfpathlineto{\pgfqpoint{4.401132in}{2.079789in}}%
\pgfpathlineto{\pgfqpoint{4.409938in}{2.265089in}}%
\pgfpathlineto{\pgfqpoint{4.418745in}{2.361168in}}%
\pgfpathlineto{\pgfqpoint{4.427552in}{2.265089in}}%
\pgfpathlineto{\pgfqpoint{4.436359in}{2.203303in}}%
\pgfpathlineto{\pgfqpoint{4.445166in}{1.743512in}}%
\pgfpathlineto{\pgfqpoint{4.453972in}{2.189586in}}%
\pgfpathlineto{\pgfqpoint{4.462779in}{2.134687in}}%
\pgfpathlineto{\pgfqpoint{4.471586in}{2.107252in}}%
\pgfpathlineto{\pgfqpoint{4.480393in}{2.182727in}}%
\pgfpathlineto{\pgfqpoint{4.489200in}{2.196445in}}%
\pgfpathlineto{\pgfqpoint{4.498007in}{2.292524in}}%
\pgfpathlineto{\pgfqpoint{4.506813in}{2.038608in}}%
\pgfpathlineto{\pgfqpoint{4.515620in}{2.018032in}}%
\pgfpathlineto{\pgfqpoint{4.524427in}{1.935670in}}%
\pgfpathlineto{\pgfqpoint{4.533234in}{1.462133in}}%
\pgfpathlineto{\pgfqpoint{4.550847in}{2.203303in}}%
\pgfpathlineto{\pgfqpoint{4.559654in}{2.340563in}}%
\pgfpathlineto{\pgfqpoint{4.568461in}{2.230767in}}%
\pgfpathlineto{\pgfqpoint{4.577268in}{1.867054in}}%
\pgfpathlineto{\pgfqpoint{4.586075in}{1.764116in}}%
\pgfpathlineto{\pgfqpoint{4.594882in}{1.626856in}}%
\pgfpathlineto{\pgfqpoint{4.603688in}{1.928811in}}%
\pgfpathlineto{\pgfqpoint{4.612495in}{1.853308in}}%
\pgfpathlineto{\pgfqpoint{4.621302in}{1.729794in}}%
\pgfpathlineto{\pgfqpoint{4.630109in}{1.880772in}}%
\pgfpathlineto{\pgfqpoint{4.638916in}{1.819015in}}%
\pgfpathlineto{\pgfqpoint{4.647722in}{2.141546in}}%
\pgfpathlineto{\pgfqpoint{4.656529in}{2.169009in}}%
\pgfpathlineto{\pgfqpoint{4.665336in}{2.333705in}}%
\pgfpathlineto{\pgfqpoint{4.674143in}{1.956246in}}%
\pgfpathlineto{\pgfqpoint{4.682950in}{1.791551in}}%
\pgfpathlineto{\pgfqpoint{4.691757in}{2.251343in}}%
\pgfpathlineto{\pgfqpoint{4.700563in}{2.162151in}}%
\pgfpathlineto{\pgfqpoint{4.709370in}{2.127829in}}%
\pgfpathlineto{\pgfqpoint{4.726984in}{1.736653in}}%
\pgfpathlineto{\pgfqpoint{4.735791in}{1.716076in}}%
\pgfpathlineto{\pgfqpoint{4.744597in}{1.640574in}}%
\pgfpathlineto{\pgfqpoint{4.753404in}{1.544494in}}%
\pgfpathlineto{\pgfqpoint{4.762211in}{1.674896in}}%
\pgfpathlineto{\pgfqpoint{4.771018in}{1.517031in}}%
\pgfpathlineto{\pgfqpoint{4.779825in}{1.276861in}}%
\pgfpathlineto{\pgfqpoint{4.788632in}{1.441556in}}%
\pgfpathlineto{\pgfqpoint{4.797438in}{1.530777in}}%
\pgfpathlineto{\pgfqpoint{4.806245in}{1.558212in}}%
\pgfpathlineto{\pgfqpoint{4.815052in}{1.736653in}}%
\pgfpathlineto{\pgfqpoint{4.823859in}{1.613110in}}%
\pgfpathlineto{\pgfqpoint{4.832666in}{1.592534in}}%
\pgfpathlineto{\pgfqpoint{4.841472in}{1.311155in}}%
\pgfpathlineto{\pgfqpoint{4.850279in}{1.619997in}}%
\pgfpathlineto{\pgfqpoint{4.859086in}{2.141546in}}%
\pgfpathlineto{\pgfqpoint{4.867893in}{2.162151in}}%
\pgfpathlineto{\pgfqpoint{4.876700in}{1.942529in}}%
\pgfpathlineto{\pgfqpoint{4.885507in}{1.921953in}}%
\pgfpathlineto{\pgfqpoint{4.894313in}{1.908207in}}%
\pgfpathlineto{\pgfqpoint{4.903120in}{1.537636in}}%
\pgfpathlineto{\pgfqpoint{4.911927in}{1.249398in}}%
\pgfpathlineto{\pgfqpoint{4.920734in}{1.640574in}}%
\pgfpathlineto{\pgfqpoint{4.929541in}{1.160177in}}%
\pgfpathlineto{\pgfqpoint{4.938347in}{1.022917in}}%
\pgfpathlineto{\pgfqpoint{4.947154in}{1.331760in}}%
\pgfpathlineto{\pgfqpoint{4.955961in}{1.592534in}}%
\pgfpathlineto{\pgfqpoint{4.964768in}{1.770975in}}%
\pgfpathlineto{\pgfqpoint{4.973575in}{1.729794in}}%
\pgfpathlineto{\pgfqpoint{4.982382in}{1.414093in}}%
\pgfpathlineto{\pgfqpoint{4.991188in}{1.393517in}}%
\pgfpathlineto{\pgfqpoint{4.999995in}{1.633715in}}%
\pgfpathlineto{\pgfqpoint{5.008802in}{1.750370in}}%
\pgfpathlineto{\pgfqpoint{5.017609in}{1.475878in}}%
\pgfpathlineto{\pgfqpoint{5.035222in}{1.969992in}}%
\pgfpathlineto{\pgfqpoint{5.044029in}{1.935670in}}%
\pgfpathlineto{\pgfqpoint{5.052836in}{1.860167in}}%
\pgfpathlineto{\pgfqpoint{5.061643in}{1.681754in}}%
\pgfpathlineto{\pgfqpoint{5.070450in}{2.258230in}}%
\pgfpathlineto{\pgfqpoint{5.079257in}{2.162151in}}%
\pgfpathlineto{\pgfqpoint{5.088063in}{1.722935in}}%
\pgfpathlineto{\pgfqpoint{5.096870in}{1.812128in}}%
\pgfpathlineto{\pgfqpoint{5.105677in}{1.949388in}}%
\pgfpathlineto{\pgfqpoint{5.114484in}{1.640574in}}%
\pgfpathlineto{\pgfqpoint{5.123291in}{1.908207in}}%
\pgfpathlineto{\pgfqpoint{5.132097in}{1.825873in}}%
\pgfpathlineto{\pgfqpoint{5.140904in}{1.969992in}}%
\pgfpathlineto{\pgfqpoint{5.149711in}{1.935670in}}%
\pgfpathlineto{\pgfqpoint{5.158518in}{1.839591in}}%
\pgfpathlineto{\pgfqpoint{5.176132in}{1.695472in}}%
\pgfpathlineto{\pgfqpoint{5.184938in}{1.462133in}}%
\pgfpathlineto{\pgfqpoint{5.193745in}{1.695472in}}%
\pgfpathlineto{\pgfqpoint{5.202552in}{1.331760in}}%
\pgfpathlineto{\pgfqpoint{5.211359in}{1.462133in}}%
\pgfpathlineto{\pgfqpoint{5.220166in}{1.503314in}}%
\pgfpathlineto{\pgfqpoint{5.237779in}{1.784692in}}%
\pgfpathlineto{\pgfqpoint{5.246586in}{1.462133in}}%
\pgfpathlineto{\pgfqpoint{5.255393in}{1.448415in}}%
\pgfpathlineto{\pgfqpoint{5.264200in}{1.118997in}}%
\pgfpathlineto{\pgfqpoint{5.273007in}{1.249398in}}%
\pgfpathlineto{\pgfqpoint{5.281813in}{1.324873in}}%
\pgfpathlineto{\pgfqpoint{5.290620in}{1.592534in}}%
\pgfpathlineto{\pgfqpoint{5.299427in}{1.517031in}}%
\pgfpathlineto{\pgfqpoint{5.308234in}{1.400376in}}%
\pgfpathlineto{\pgfqpoint{5.325847in}{1.969992in}}%
\pgfpathlineto{\pgfqpoint{5.334654in}{2.107252in}}%
\pgfpathlineto{\pgfqpoint{5.343461in}{2.086648in}}%
\pgfpathlineto{\pgfqpoint{5.352268in}{1.928811in}}%
\pgfpathlineto{\pgfqpoint{5.361075in}{1.736653in}}%
\pgfpathlineto{\pgfqpoint{5.369882in}{1.812128in}}%
\pgfpathlineto{\pgfqpoint{5.378688in}{1.420952in}}%
\pgfpathlineto{\pgfqpoint{5.387495in}{1.311155in}}%
\pgfpathlineto{\pgfqpoint{5.396302in}{1.722935in}}%
\pgfpathlineto{\pgfqpoint{5.405109in}{1.867054in}}%
\pgfpathlineto{\pgfqpoint{5.413916in}{2.031749in}}%
\pgfpathlineto{\pgfqpoint{5.422722in}{2.004286in}}%
\pgfpathlineto{\pgfqpoint{5.431529in}{1.942529in}}%
\pgfpathlineto{\pgfqpoint{5.440336in}{2.107252in}}%
\pgfpathlineto{\pgfqpoint{5.449143in}{1.736653in}}%
\pgfpathlineto{\pgfqpoint{5.457950in}{2.237625in}}%
\pgfpathlineto{\pgfqpoint{5.466757in}{2.539581in}}%
\pgfpathlineto{\pgfqpoint{5.475563in}{2.196445in}}%
\pgfpathlineto{\pgfqpoint{5.484370in}{2.120970in}}%
\pgfpathlineto{\pgfqpoint{5.493177in}{2.072930in}}%
\pgfpathlineto{\pgfqpoint{5.501984in}{2.210190in}}%
\pgfpathlineto{\pgfqpoint{5.510791in}{2.258230in}}%
\pgfpathlineto{\pgfqpoint{5.519597in}{1.915094in}}%
\pgfpathlineto{\pgfqpoint{5.528404in}{1.619997in}}%
\pgfpathlineto{\pgfqpoint{5.537211in}{1.283720in}}%
\pgfpathlineto{\pgfqpoint{5.546018in}{1.146460in}}%
\pgfpathlineto{\pgfqpoint{5.554825in}{1.208217in}}%
\pgfpathlineto{\pgfqpoint{5.563632in}{1.551353in}}%
\pgfpathlineto{\pgfqpoint{5.572438in}{1.668037in}}%
\pgfpathlineto{\pgfqpoint{5.581245in}{1.750370in}}%
\pgfpathlineto{\pgfqpoint{5.590052in}{2.059213in}}%
\pgfpathlineto{\pgfqpoint{5.598859in}{1.935670in}}%
\pgfpathlineto{\pgfqpoint{5.607666in}{1.722935in}}%
\pgfpathlineto{\pgfqpoint{5.616472in}{1.825873in}}%
\pgfpathlineto{\pgfqpoint{5.625279in}{1.969992in}}%
\pgfpathlineto{\pgfqpoint{5.634086in}{1.722935in}}%
\pgfpathlineto{\pgfqpoint{5.642893in}{1.695472in}}%
\pgfpathlineto{\pgfqpoint{5.651700in}{1.311155in}}%
\pgfpathlineto{\pgfqpoint{5.660507in}{1.640574in}}%
\pgfpathlineto{\pgfqpoint{5.669313in}{1.791551in}}%
\pgfpathlineto{\pgfqpoint{5.678120in}{1.805269in}}%
\pgfpathlineto{\pgfqpoint{5.686927in}{1.976851in}}%
\pgfpathlineto{\pgfqpoint{5.695734in}{1.894489in}}%
\pgfpathlineto{\pgfqpoint{5.704541in}{1.825873in}}%
\pgfpathlineto{\pgfqpoint{5.713347in}{1.887630in}}%
\pgfpathlineto{\pgfqpoint{5.722154in}{1.887630in}}%
\pgfpathlineto{\pgfqpoint{5.730961in}{2.230767in}}%
\pgfpathlineto{\pgfqpoint{5.739768in}{2.361168in}}%
\pgfpathlineto{\pgfqpoint{5.748575in}{2.258230in}}%
\pgfpathlineto{\pgfqpoint{5.757382in}{2.285665in}}%
\pgfpathlineto{\pgfqpoint{5.766188in}{1.997427in}}%
\pgfpathlineto{\pgfqpoint{5.774995in}{1.592534in}}%
\pgfpathlineto{\pgfqpoint{5.783802in}{1.695472in}}%
\pgfpathlineto{\pgfqpoint{5.801416in}{2.134687in}}%
\pgfpathlineto{\pgfqpoint{5.810222in}{2.141546in}}%
\pgfpathlineto{\pgfqpoint{5.819029in}{1.921953in}}%
\pgfpathlineto{\pgfqpoint{5.827836in}{2.429784in}}%
\pgfpathlineto{\pgfqpoint{5.836643in}{2.278806in}}%
\pgfpathlineto{\pgfqpoint{5.845450in}{1.867054in}}%
\pgfpathlineto{\pgfqpoint{5.854257in}{1.750370in}}%
\pgfpathlineto{\pgfqpoint{5.863063in}{1.798410in}}%
\pgfpathlineto{\pgfqpoint{5.871870in}{1.894489in}}%
\pgfpathlineto{\pgfqpoint{5.880677in}{1.956246in}}%
\pgfpathlineto{\pgfqpoint{5.889484in}{2.059213in}}%
\pgfpathlineto{\pgfqpoint{5.898291in}{1.935670in}}%
\pgfpathlineto{\pgfqpoint{5.907097in}{1.441556in}}%
\pgfpathlineto{\pgfqpoint{5.915904in}{1.441556in}}%
\pgfpathlineto{\pgfqpoint{5.924711in}{1.729794in}}%
\pgfpathlineto{\pgfqpoint{5.933518in}{1.736653in}}%
\pgfpathlineto{\pgfqpoint{5.942325in}{2.052326in}}%
\pgfpathlineto{\pgfqpoint{5.951132in}{1.942529in}}%
\pgfpathlineto{\pgfqpoint{5.959938in}{1.997427in}}%
\pgfpathlineto{\pgfqpoint{5.968745in}{1.674896in}}%
\pgfpathlineto{\pgfqpoint{5.977552in}{1.880772in}}%
\pgfpathlineto{\pgfqpoint{5.986359in}{1.853308in}}%
\pgfpathlineto{\pgfqpoint{5.995166in}{1.812128in}}%
\pgfpathlineto{\pgfqpoint{6.003972in}{1.846450in}}%
\pgfpathlineto{\pgfqpoint{6.012779in}{2.093507in}}%
\pgfpathlineto{\pgfqpoint{6.021586in}{1.805269in}}%
\pgfpathlineto{\pgfqpoint{6.030393in}{1.956246in}}%
\pgfpathlineto{\pgfqpoint{6.039200in}{1.722935in}}%
\pgfpathlineto{\pgfqpoint{6.048007in}{1.832732in}}%
\pgfpathlineto{\pgfqpoint{6.056813in}{1.633715in}}%
\pgfpathlineto{\pgfqpoint{6.065620in}{1.695472in}}%
\pgfpathlineto{\pgfqpoint{6.074427in}{1.894489in}}%
\pgfpathlineto{\pgfqpoint{6.083234in}{1.901348in}}%
\pgfpathlineto{\pgfqpoint{6.100847in}{2.189586in}}%
\pgfpathlineto{\pgfqpoint{6.109654in}{2.210190in}}%
\pgfpathlineto{\pgfqpoint{6.118461in}{2.340563in}}%
\pgfpathlineto{\pgfqpoint{6.127268in}{2.436643in}}%
\pgfpathlineto{\pgfqpoint{6.136075in}{2.052326in}}%
\pgfpathlineto{\pgfqpoint{6.144882in}{2.120970in}}%
\pgfpathlineto{\pgfqpoint{6.153688in}{2.038608in}}%
\pgfpathlineto{\pgfqpoint{6.162495in}{1.764116in}}%
\pgfpathlineto{\pgfqpoint{6.171302in}{2.114111in}}%
\pgfpathlineto{\pgfqpoint{6.180109in}{2.114111in}}%
\pgfpathlineto{\pgfqpoint{6.188916in}{2.024891in}}%
\pgfpathlineto{\pgfqpoint{6.197722in}{2.134687in}}%
\pgfpathlineto{\pgfqpoint{6.206529in}{2.628801in}}%
\pgfpathlineto{\pgfqpoint{6.215336in}{2.251343in}}%
\pgfpathlineto{\pgfqpoint{6.224143in}{2.196445in}}%
\pgfpathlineto{\pgfqpoint{6.232950in}{1.764116in}}%
\pgfpathlineto{\pgfqpoint{6.241757in}{1.949388in}}%
\pgfpathlineto{\pgfqpoint{6.250563in}{1.867054in}}%
\pgfpathlineto{\pgfqpoint{6.259370in}{1.798410in}}%
\pgfpathlineto{\pgfqpoint{6.268177in}{1.626856in}}%
\pgfpathlineto{\pgfqpoint{6.276984in}{1.283720in}}%
\pgfpathlineto{\pgfqpoint{6.285791in}{1.352336in}}%
\pgfpathlineto{\pgfqpoint{6.294597in}{1.482737in}}%
\pgfpathlineto{\pgfqpoint{6.303404in}{1.969992in}}%
\pgfpathlineto{\pgfqpoint{6.312211in}{2.285665in}}%
\pgfpathlineto{\pgfqpoint{6.321018in}{2.223908in}}%
\pgfpathlineto{\pgfqpoint{6.329825in}{2.100365in}}%
\pgfpathlineto{\pgfqpoint{6.338632in}{2.217049in}}%
\pgfpathlineto{\pgfqpoint{6.347438in}{1.983710in}}%
\pgfpathlineto{\pgfqpoint{6.356245in}{1.908207in}}%
\pgfpathlineto{\pgfqpoint{6.365052in}{2.100365in}}%
\pgfpathlineto{\pgfqpoint{6.373859in}{2.196445in}}%
\pgfpathlineto{\pgfqpoint{6.382666in}{2.148405in}}%
\pgfpathlineto{\pgfqpoint{6.391472in}{2.251343in}}%
\pgfpathlineto{\pgfqpoint{6.400279in}{2.072930in}}%
\pgfpathlineto{\pgfqpoint{6.409086in}{2.018032in}}%
\pgfpathlineto{\pgfqpoint{6.426700in}{2.580761in}}%
\pgfpathlineto{\pgfqpoint{6.435507in}{2.059213in}}%
\pgfpathlineto{\pgfqpoint{6.444313in}{2.066071in}}%
\pgfpathlineto{\pgfqpoint{6.453120in}{1.921953in}}%
\pgfpathlineto{\pgfqpoint{6.461927in}{1.942529in}}%
\pgfpathlineto{\pgfqpoint{6.470734in}{1.750370in}}%
\pgfpathlineto{\pgfqpoint{6.479541in}{1.626856in}}%
\pgfpathlineto{\pgfqpoint{6.488347in}{1.805269in}}%
\pgfpathlineto{\pgfqpoint{6.497154in}{1.949388in}}%
\pgfpathlineto{\pgfqpoint{6.505961in}{1.935670in}}%
\pgfpathlineto{\pgfqpoint{6.514768in}{2.052326in}}%
\pgfpathlineto{\pgfqpoint{6.523575in}{2.086648in}}%
\pgfpathlineto{\pgfqpoint{6.532382in}{1.812128in}}%
\pgfpathlineto{\pgfqpoint{6.541188in}{1.716076in}}%
\pgfpathlineto{\pgfqpoint{6.549995in}{2.189586in}}%
\pgfpathlineto{\pgfqpoint{6.558802in}{2.114111in}}%
\pgfpathlineto{\pgfqpoint{6.567609in}{2.422925in}}%
\pgfpathlineto{\pgfqpoint{6.576416in}{2.052326in}}%
\pgfpathlineto{\pgfqpoint{6.585222in}{2.258230in}}%
\pgfpathlineto{\pgfqpoint{6.594029in}{2.086648in}}%
\pgfpathlineto{\pgfqpoint{6.602836in}{1.736653in}}%
\pgfpathlineto{\pgfqpoint{6.611643in}{1.592534in}}%
\pgfpathlineto{\pgfqpoint{6.620450in}{1.935670in}}%
\pgfpathlineto{\pgfqpoint{6.638063in}{2.066071in}}%
\pgfpathlineto{\pgfqpoint{6.646870in}{1.873913in}}%
\pgfpathlineto{\pgfqpoint{6.655677in}{1.455274in}}%
\pgfpathlineto{\pgfqpoint{6.664484in}{1.688613in}}%
\pgfpathlineto{\pgfqpoint{6.673291in}{1.269974in}}%
\pgfpathlineto{\pgfqpoint{6.682097in}{1.441556in}}%
\pgfpathlineto{\pgfqpoint{6.699711in}{1.304296in}}%
\pgfpathlineto{\pgfqpoint{6.708518in}{1.414093in}}%
\pgfpathlineto{\pgfqpoint{6.717325in}{1.592534in}}%
\pgfpathlineto{\pgfqpoint{6.726132in}{1.661150in}}%
\pgfpathlineto{\pgfqpoint{6.734938in}{1.606252in}}%
\pgfpathlineto{\pgfqpoint{6.743745in}{1.729794in}}%
\pgfpathlineto{\pgfqpoint{6.752552in}{1.619997in}}%
\pgfpathlineto{\pgfqpoint{6.761359in}{1.805269in}}%
\pgfpathlineto{\pgfqpoint{6.770166in}{2.093507in}}%
\pgfpathlineto{\pgfqpoint{6.778972in}{2.196445in}}%
\pgfpathlineto{\pgfqpoint{6.787779in}{2.217049in}}%
\pgfpathlineto{\pgfqpoint{6.796586in}{1.709190in}}%
\pgfpathlineto{\pgfqpoint{6.805393in}{1.434698in}}%
\pgfpathlineto{\pgfqpoint{6.814200in}{1.523918in}}%
\pgfpathlineto{\pgfqpoint{6.823007in}{1.695472in}}%
\pgfpathlineto{\pgfqpoint{6.831813in}{1.777834in}}%
\pgfpathlineto{\pgfqpoint{6.840620in}{1.846450in}}%
\pgfpathlineto{\pgfqpoint{6.849427in}{1.928811in}}%
\pgfpathlineto{\pgfqpoint{6.858234in}{1.887630in}}%
\pgfpathlineto{\pgfqpoint{6.867041in}{1.654291in}}%
\pgfpathlineto{\pgfqpoint{6.875847in}{1.887630in}}%
\pgfpathlineto{\pgfqpoint{6.884654in}{1.606252in}}%
\pgfpathlineto{\pgfqpoint{6.893461in}{1.530777in}}%
\pgfpathlineto{\pgfqpoint{6.902268in}{1.963133in}}%
\pgfpathlineto{\pgfqpoint{6.911075in}{1.798410in}}%
\pgfpathlineto{\pgfqpoint{6.919882in}{1.798410in}}%
\pgfpathlineto{\pgfqpoint{6.928688in}{2.340563in}}%
\pgfpathlineto{\pgfqpoint{6.937495in}{2.155292in}}%
\pgfpathlineto{\pgfqpoint{6.946302in}{2.217049in}}%
\pgfpathlineto{\pgfqpoint{6.955109in}{2.416066in}}%
\pgfpathlineto{\pgfqpoint{6.963916in}{2.114111in}}%
\pgfpathlineto{\pgfqpoint{6.972722in}{2.093507in}}%
\pgfpathlineto{\pgfqpoint{6.981529in}{2.038608in}}%
\pgfpathlineto{\pgfqpoint{6.999143in}{1.338618in}}%
\pgfpathlineto{\pgfqpoint{7.007950in}{1.757229in}}%
\pgfpathlineto{\pgfqpoint{7.016757in}{1.517031in}}%
\pgfpathlineto{\pgfqpoint{7.025563in}{1.455274in}}%
\pgfpathlineto{\pgfqpoint{7.034370in}{1.853308in}}%
\pgfpathlineto{\pgfqpoint{7.043177in}{1.791551in}}%
\pgfpathlineto{\pgfqpoint{7.051984in}{1.633715in}}%
\pgfpathlineto{\pgfqpoint{7.060791in}{2.024891in}}%
\pgfpathlineto{\pgfqpoint{7.069597in}{2.134687in}}%
\pgfpathlineto{\pgfqpoint{7.078404in}{1.750370in}}%
\pgfpathlineto{\pgfqpoint{7.087211in}{1.805269in}}%
\pgfpathlineto{\pgfqpoint{7.096018in}{2.560185in}}%
\pgfpathlineto{\pgfqpoint{7.104825in}{2.340563in}}%
\pgfpathlineto{\pgfqpoint{7.113632in}{2.244484in}}%
\pgfpathlineto{\pgfqpoint{7.122438in}{2.120970in}}%
\pgfpathlineto{\pgfqpoint{7.131245in}{2.052326in}}%
\pgfpathlineto{\pgfqpoint{7.140052in}{2.045467in}}%
\pgfpathlineto{\pgfqpoint{7.148859in}{1.729794in}}%
\pgfpathlineto{\pgfqpoint{7.157666in}{1.688613in}}%
\pgfpathlineto{\pgfqpoint{7.166472in}{1.434698in}}%
\pgfpathlineto{\pgfqpoint{7.175279in}{1.427839in}}%
\pgfpathlineto{\pgfqpoint{7.184086in}{1.729794in}}%
\pgfpathlineto{\pgfqpoint{7.192893in}{1.599393in}}%
\pgfpathlineto{\pgfqpoint{7.201700in}{1.860167in}}%
\pgfpathlineto{\pgfqpoint{7.210507in}{1.832732in}}%
\pgfpathlineto{\pgfqpoint{7.219313in}{1.716076in}}%
\pgfpathlineto{\pgfqpoint{7.228120in}{1.565071in}}%
\pgfpathlineto{\pgfqpoint{7.236927in}{1.448415in}}%
\pgfpathlineto{\pgfqpoint{7.245734in}{1.983710in}}%
\pgfpathlineto{\pgfqpoint{7.254541in}{1.867054in}}%
\pgfpathlineto{\pgfqpoint{7.263347in}{2.004286in}}%
\pgfpathlineto{\pgfqpoint{7.272154in}{2.189586in}}%
\pgfpathlineto{\pgfqpoint{7.280961in}{1.729794in}}%
\pgfpathlineto{\pgfqpoint{7.289768in}{2.134687in}}%
\pgfpathlineto{\pgfqpoint{7.298575in}{1.798410in}}%
\pgfpathlineto{\pgfqpoint{7.307382in}{1.915094in}}%
\pgfpathlineto{\pgfqpoint{7.316188in}{1.784692in}}%
\pgfpathlineto{\pgfqpoint{7.324995in}{2.169009in}}%
\pgfpathlineto{\pgfqpoint{7.342609in}{1.750370in}}%
\pgfpathlineto{\pgfqpoint{7.351416in}{2.107252in}}%
\pgfpathlineto{\pgfqpoint{7.360222in}{2.100365in}}%
\pgfpathlineto{\pgfqpoint{7.369029in}{2.059213in}}%
\pgfpathlineto{\pgfqpoint{7.377836in}{2.148405in}}%
\pgfpathlineto{\pgfqpoint{7.386643in}{2.114111in}}%
\pgfpathlineto{\pgfqpoint{7.395450in}{1.990569in}}%
\pgfpathlineto{\pgfqpoint{7.413063in}{2.175868in}}%
\pgfpathlineto{\pgfqpoint{7.421870in}{2.175868in}}%
\pgfpathlineto{\pgfqpoint{7.430677in}{1.908207in}}%
\pgfpathlineto{\pgfqpoint{7.439484in}{1.935670in}}%
\pgfpathlineto{\pgfqpoint{7.448291in}{2.292524in}}%
\pgfpathlineto{\pgfqpoint{7.457097in}{2.484682in}}%
\pgfpathlineto{\pgfqpoint{7.465904in}{2.024891in}}%
\pgfpathlineto{\pgfqpoint{7.474711in}{1.908207in}}%
\pgfpathlineto{\pgfqpoint{7.483518in}{2.265089in}}%
\pgfpathlineto{\pgfqpoint{7.492325in}{2.182727in}}%
\pgfpathlineto{\pgfqpoint{7.501132in}{2.361168in}}%
\pgfpathlineto{\pgfqpoint{7.509938in}{2.086648in}}%
\pgfpathlineto{\pgfqpoint{7.518745in}{1.750370in}}%
\pgfpathlineto{\pgfqpoint{7.527552in}{1.640574in}}%
\pgfpathlineto{\pgfqpoint{7.536359in}{1.626856in}}%
\pgfpathlineto{\pgfqpoint{7.553972in}{1.125855in}}%
\pgfpathlineto{\pgfqpoint{7.571586in}{1.921953in}}%
\pgfpathlineto{\pgfqpoint{7.580393in}{1.949388in}}%
\pgfpathlineto{\pgfqpoint{7.589200in}{1.743512in}}%
\pgfpathlineto{\pgfqpoint{7.598007in}{2.004286in}}%
\pgfpathlineto{\pgfqpoint{7.606813in}{1.894489in}}%
\pgfpathlineto{\pgfqpoint{7.615620in}{1.880772in}}%
\pgfpathlineto{\pgfqpoint{7.624427in}{1.764116in}}%
\pgfpathlineto{\pgfqpoint{7.633234in}{2.120970in}}%
\pgfpathlineto{\pgfqpoint{7.642041in}{1.777834in}}%
\pgfpathlineto{\pgfqpoint{7.650847in}{1.640574in}}%
\pgfpathlineto{\pgfqpoint{7.659654in}{1.736653in}}%
\pgfpathlineto{\pgfqpoint{7.668461in}{1.729794in}}%
\pgfpathlineto{\pgfqpoint{7.677268in}{1.407234in}}%
\pgfpathlineto{\pgfqpoint{7.686075in}{1.407234in}}%
\pgfpathlineto{\pgfqpoint{7.694882in}{1.764116in}}%
\pgfpathlineto{\pgfqpoint{7.703688in}{1.482737in}}%
\pgfpathlineto{\pgfqpoint{7.712495in}{1.448415in}}%
\pgfpathlineto{\pgfqpoint{7.721302in}{1.585675in}}%
\pgfpathlineto{\pgfqpoint{7.730109in}{1.448415in}}%
\pgfpathlineto{\pgfqpoint{7.738916in}{1.695472in}}%
\pgfpathlineto{\pgfqpoint{7.747722in}{1.777834in}}%
\pgfpathlineto{\pgfqpoint{7.756529in}{2.038608in}}%
\pgfpathlineto{\pgfqpoint{7.765336in}{2.505287in}}%
\pgfpathlineto{\pgfqpoint{7.774143in}{2.539581in}}%
\pgfpathlineto{\pgfqpoint{7.782950in}{2.210190in}}%
\pgfpathlineto{\pgfqpoint{7.791757in}{1.997427in}}%
\pgfpathlineto{\pgfqpoint{7.809370in}{1.873913in}}%
\pgfpathlineto{\pgfqpoint{7.818177in}{1.942529in}}%
\pgfpathlineto{\pgfqpoint{7.826984in}{1.908207in}}%
\pgfpathlineto{\pgfqpoint{7.835791in}{1.969992in}}%
\pgfpathlineto{\pgfqpoint{7.844597in}{1.963133in}}%
\pgfpathlineto{\pgfqpoint{7.853404in}{1.825873in}}%
\pgfpathlineto{\pgfqpoint{7.862211in}{1.812128in}}%
\pgfpathlineto{\pgfqpoint{7.871018in}{1.853308in}}%
\pgfpathlineto{\pgfqpoint{7.879825in}{1.640574in}}%
\pgfpathlineto{\pgfqpoint{7.888632in}{1.585675in}}%
\pgfpathlineto{\pgfqpoint{7.897438in}{2.059213in}}%
\pgfpathlineto{\pgfqpoint{7.906245in}{2.120970in}}%
\pgfpathlineto{\pgfqpoint{7.915052in}{1.969992in}}%
\pgfpathlineto{\pgfqpoint{7.923859in}{2.196445in}}%
\pgfpathlineto{\pgfqpoint{7.932666in}{1.832732in}}%
\pgfpathlineto{\pgfqpoint{7.941472in}{1.722935in}}%
\pgfpathlineto{\pgfqpoint{7.950279in}{1.853308in}}%
\pgfpathlineto{\pgfqpoint{7.959086in}{1.661150in}}%
\pgfpathlineto{\pgfqpoint{7.976700in}{1.517031in}}%
\pgfpathlineto{\pgfqpoint{7.985507in}{1.393517in}}%
\pgfpathlineto{\pgfqpoint{7.994313in}{1.798410in}}%
\pgfpathlineto{\pgfqpoint{8.003120in}{1.976851in}}%
\pgfpathlineto{\pgfqpoint{8.011927in}{1.764116in}}%
\pgfpathlineto{\pgfqpoint{8.020734in}{1.366053in}}%
\pgfpathlineto{\pgfqpoint{8.029541in}{1.551353in}}%
\pgfpathlineto{\pgfqpoint{8.038347in}{1.324873in}}%
\pgfpathlineto{\pgfqpoint{8.047154in}{1.324873in}}%
\pgfpathlineto{\pgfqpoint{8.073575in}{2.155292in}}%
\pgfpathlineto{\pgfqpoint{8.082382in}{1.990569in}}%
\pgfpathlineto{\pgfqpoint{8.091188in}{2.512146in}}%
\pgfpathlineto{\pgfqpoint{8.099995in}{2.690558in}}%
\pgfpathlineto{\pgfqpoint{8.108802in}{2.683700in}}%
\pgfpathlineto{\pgfqpoint{8.117609in}{2.711163in}}%
\pgfpathlineto{\pgfqpoint{8.126416in}{2.800383in}}%
\pgfpathlineto{\pgfqpoint{8.135222in}{2.676841in}}%
\pgfpathlineto{\pgfqpoint{8.144029in}{2.271947in}}%
\pgfpathlineto{\pgfqpoint{8.152836in}{2.464106in}}%
\pgfpathlineto{\pgfqpoint{8.161643in}{2.210190in}}%
\pgfpathlineto{\pgfqpoint{8.170450in}{2.024891in}}%
\pgfpathlineto{\pgfqpoint{8.179257in}{2.059213in}}%
\pgfpathlineto{\pgfqpoint{8.188063in}{1.805269in}}%
\pgfpathlineto{\pgfqpoint{8.196870in}{1.791551in}}%
\pgfpathlineto{\pgfqpoint{8.205677in}{1.702331in}}%
\pgfpathlineto{\pgfqpoint{8.214484in}{1.482737in}}%
\pgfpathlineto{\pgfqpoint{8.223291in}{1.420952in}}%
\pgfpathlineto{\pgfqpoint{8.240904in}{1.668037in}}%
\pgfpathlineto{\pgfqpoint{8.249711in}{1.496455in}}%
\pgfpathlineto{\pgfqpoint{8.267325in}{1.743512in}}%
\pgfpathlineto{\pgfqpoint{8.276132in}{1.832732in}}%
\pgfpathlineto{\pgfqpoint{8.284938in}{1.867054in}}%
\pgfpathlineto{\pgfqpoint{8.293745in}{2.038608in}}%
\pgfpathlineto{\pgfqpoint{8.302552in}{1.839591in}}%
\pgfpathlineto{\pgfqpoint{8.311359in}{1.750370in}}%
\pgfpathlineto{\pgfqpoint{8.320166in}{1.407234in}}%
\pgfpathlineto{\pgfqpoint{8.328972in}{1.544494in}}%
\pgfpathlineto{\pgfqpoint{8.337779in}{1.462133in}}%
\pgfpathlineto{\pgfqpoint{8.346586in}{1.599393in}}%
\pgfpathlineto{\pgfqpoint{8.355393in}{1.372912in}}%
\pgfpathlineto{\pgfqpoint{8.364200in}{1.420952in}}%
\pgfpathlineto{\pgfqpoint{8.381813in}{1.709190in}}%
\pgfpathlineto{\pgfqpoint{8.390620in}{1.578816in}}%
\pgfpathlineto{\pgfqpoint{8.399427in}{1.558212in}}%
\pgfpathlineto{\pgfqpoint{8.408234in}{1.702331in}}%
\pgfpathlineto{\pgfqpoint{8.417041in}{1.887630in}}%
\pgfpathlineto{\pgfqpoint{8.425847in}{1.599393in}}%
\pgfpathlineto{\pgfqpoint{8.434654in}{1.592534in}}%
\pgfpathlineto{\pgfqpoint{8.443461in}{1.729794in}}%
\pgfpathlineto{\pgfqpoint{8.452268in}{2.079789in}}%
\pgfpathlineto{\pgfqpoint{8.461075in}{1.894489in}}%
\pgfpathlineto{\pgfqpoint{8.469882in}{1.928811in}}%
\pgfpathlineto{\pgfqpoint{8.478688in}{1.956246in}}%
\pgfpathlineto{\pgfqpoint{8.487495in}{1.812128in}}%
\pgfpathlineto{\pgfqpoint{8.496302in}{1.956246in}}%
\pgfpathlineto{\pgfqpoint{8.505109in}{1.839591in}}%
\pgfpathlineto{\pgfqpoint{8.513916in}{2.148405in}}%
\pgfpathlineto{\pgfqpoint{8.522722in}{2.155292in}}%
\pgfpathlineto{\pgfqpoint{8.531529in}{1.839591in}}%
\pgfpathlineto{\pgfqpoint{8.540336in}{1.969992in}}%
\pgfpathlineto{\pgfqpoint{8.549143in}{2.011173in}}%
\pgfpathlineto{\pgfqpoint{8.557950in}{1.990569in}}%
\pgfpathlineto{\pgfqpoint{8.566757in}{2.024891in}}%
\pgfpathlineto{\pgfqpoint{8.575563in}{1.956246in}}%
\pgfpathlineto{\pgfqpoint{8.584370in}{1.976851in}}%
\pgfpathlineto{\pgfqpoint{8.593177in}{1.860167in}}%
\pgfpathlineto{\pgfqpoint{8.601984in}{2.251343in}}%
\pgfpathlineto{\pgfqpoint{8.610791in}{2.230767in}}%
\pgfpathlineto{\pgfqpoint{8.619597in}{1.963133in}}%
\pgfpathlineto{\pgfqpoint{8.628404in}{2.169009in}}%
\pgfpathlineto{\pgfqpoint{8.637211in}{1.949388in}}%
\pgfpathlineto{\pgfqpoint{8.646018in}{2.100365in}}%
\pgfpathlineto{\pgfqpoint{8.654825in}{2.018032in}}%
\pgfpathlineto{\pgfqpoint{8.663632in}{2.567044in}}%
\pgfpathlineto{\pgfqpoint{8.672438in}{2.354309in}}%
\pgfpathlineto{\pgfqpoint{8.681245in}{2.244484in}}%
\pgfpathlineto{\pgfqpoint{8.690052in}{2.594507in}}%
\pgfpathlineto{\pgfqpoint{8.698859in}{1.949388in}}%
\pgfpathlineto{\pgfqpoint{8.707666in}{1.901348in}}%
\pgfpathlineto{\pgfqpoint{8.716472in}{2.086648in}}%
\pgfpathlineto{\pgfqpoint{8.725279in}{2.120970in}}%
\pgfpathlineto{\pgfqpoint{8.734086in}{2.230767in}}%
\pgfpathlineto{\pgfqpoint{8.742893in}{1.963133in}}%
\pgfpathlineto{\pgfqpoint{8.751700in}{2.464106in}}%
\pgfpathlineto{\pgfqpoint{8.760507in}{2.258230in}}%
\pgfpathlineto{\pgfqpoint{8.769313in}{2.258230in}}%
\pgfpathlineto{\pgfqpoint{8.778120in}{2.162151in}}%
\pgfpathlineto{\pgfqpoint{8.786927in}{2.189586in}}%
\pgfpathlineto{\pgfqpoint{8.795734in}{1.928811in}}%
\pgfpathlineto{\pgfqpoint{8.804541in}{1.716076in}}%
\pgfpathlineto{\pgfqpoint{8.813347in}{1.674896in}}%
\pgfpathlineto{\pgfqpoint{8.822154in}{2.018032in}}%
\pgfpathlineto{\pgfqpoint{8.830961in}{2.107252in}}%
\pgfpathlineto{\pgfqpoint{8.839768in}{2.319987in}}%
\pgfpathlineto{\pgfqpoint{8.848575in}{2.395462in}}%
\pgfpathlineto{\pgfqpoint{8.857382in}{2.258230in}}%
\pgfpathlineto{\pgfqpoint{8.866188in}{2.299383in}}%
\pgfpathlineto{\pgfqpoint{8.874995in}{2.230767in}}%
\pgfpathlineto{\pgfqpoint{8.883802in}{2.265089in}}%
\pgfpathlineto{\pgfqpoint{8.892609in}{2.319987in}}%
\pgfpathlineto{\pgfqpoint{8.901416in}{2.621942in}}%
\pgfpathlineto{\pgfqpoint{8.910222in}{2.470965in}}%
\pgfpathlineto{\pgfqpoint{8.919029in}{2.134687in}}%
\pgfpathlineto{\pgfqpoint{8.927836in}{2.031749in}}%
\pgfpathlineto{\pgfqpoint{8.936643in}{1.990569in}}%
\pgfpathlineto{\pgfqpoint{8.945450in}{2.052326in}}%
\pgfpathlineto{\pgfqpoint{8.954257in}{2.361168in}}%
\pgfpathlineto{\pgfqpoint{8.963063in}{2.450388in}}%
\pgfpathlineto{\pgfqpoint{8.971870in}{1.928811in}}%
\pgfpathlineto{\pgfqpoint{8.980677in}{1.695472in}}%
\pgfpathlineto{\pgfqpoint{8.989484in}{1.743512in}}%
\pgfpathlineto{\pgfqpoint{8.998291in}{1.606252in}}%
\pgfpathlineto{\pgfqpoint{9.007097in}{1.571958in}}%
\pgfpathlineto{\pgfqpoint{9.015904in}{1.448415in}}%
\pgfpathlineto{\pgfqpoint{9.024711in}{1.798410in}}%
\pgfpathlineto{\pgfqpoint{9.033518in}{1.832732in}}%
\pgfpathlineto{\pgfqpoint{9.042325in}{1.770975in}}%
\pgfpathlineto{\pgfqpoint{9.051132in}{1.860167in}}%
\pgfpathlineto{\pgfqpoint{9.059938in}{1.530777in}}%
\pgfpathlineto{\pgfqpoint{9.068745in}{1.489596in}}%
\pgfpathlineto{\pgfqpoint{9.077552in}{1.681754in}}%
\pgfpathlineto{\pgfqpoint{9.086359in}{1.565071in}}%
\pgfpathlineto{\pgfqpoint{9.095166in}{1.716076in}}%
\pgfpathlineto{\pgfqpoint{9.103972in}{1.832732in}}%
\pgfpathlineto{\pgfqpoint{9.112779in}{1.825873in}}%
\pgfpathlineto{\pgfqpoint{9.121586in}{1.709190in}}%
\pgfpathlineto{\pgfqpoint{9.130393in}{1.537636in}}%
\pgfpathlineto{\pgfqpoint{9.148007in}{1.695472in}}%
\pgfpathlineto{\pgfqpoint{9.156813in}{1.894489in}}%
\pgfpathlineto{\pgfqpoint{9.165620in}{2.210190in}}%
\pgfpathlineto{\pgfqpoint{9.174427in}{2.230767in}}%
\pgfpathlineto{\pgfqpoint{9.183234in}{2.052326in}}%
\pgfpathlineto{\pgfqpoint{9.192041in}{2.148405in}}%
\pgfpathlineto{\pgfqpoint{9.200847in}{1.798410in}}%
\pgfpathlineto{\pgfqpoint{9.209654in}{1.990569in}}%
\pgfpathlineto{\pgfqpoint{9.218461in}{2.230767in}}%
\pgfpathlineto{\pgfqpoint{9.227268in}{2.265089in}}%
\pgfpathlineto{\pgfqpoint{9.236075in}{2.278806in}}%
\pgfpathlineto{\pgfqpoint{9.244882in}{2.004286in}}%
\pgfpathlineto{\pgfqpoint{9.253688in}{1.942529in}}%
\pgfpathlineto{\pgfqpoint{9.262495in}{1.798410in}}%
\pgfpathlineto{\pgfqpoint{9.271302in}{1.990569in}}%
\pgfpathlineto{\pgfqpoint{9.280109in}{1.873913in}}%
\pgfpathlineto{\pgfqpoint{9.297722in}{1.846450in}}%
\pgfpathlineto{\pgfqpoint{9.306529in}{2.278806in}}%
\pgfpathlineto{\pgfqpoint{9.315336in}{2.141546in}}%
\pgfpathlineto{\pgfqpoint{9.332950in}{2.333705in}}%
\pgfpathlineto{\pgfqpoint{9.341757in}{1.853308in}}%
\pgfpathlineto{\pgfqpoint{9.350563in}{1.798410in}}%
\pgfpathlineto{\pgfqpoint{9.359370in}{1.791551in}}%
\pgfpathlineto{\pgfqpoint{9.368177in}{2.210190in}}%
\pgfpathlineto{\pgfqpoint{9.376984in}{2.278806in}}%
\pgfpathlineto{\pgfqpoint{9.385791in}{2.175868in}}%
\pgfpathlineto{\pgfqpoint{9.394597in}{2.422925in}}%
\pgfpathlineto{\pgfqpoint{9.403404in}{2.731739in}}%
\pgfpathlineto{\pgfqpoint{9.412211in}{2.594507in}}%
\pgfpathlineto{\pgfqpoint{9.421018in}{2.477823in}}%
\pgfpathlineto{\pgfqpoint{9.429825in}{2.429784in}}%
\pgfpathlineto{\pgfqpoint{9.438632in}{2.169009in}}%
\pgfpathlineto{\pgfqpoint{9.447438in}{1.791551in}}%
\pgfpathlineto{\pgfqpoint{9.465052in}{2.045467in}}%
\pgfpathlineto{\pgfqpoint{9.473859in}{2.011173in}}%
\pgfpathlineto{\pgfqpoint{9.482666in}{2.244484in}}%
\pgfpathlineto{\pgfqpoint{9.491472in}{2.271947in}}%
\pgfpathlineto{\pgfqpoint{9.500279in}{2.155292in}}%
\pgfpathlineto{\pgfqpoint{9.509086in}{1.757229in}}%
\pgfpathlineto{\pgfqpoint{9.517893in}{1.908207in}}%
\pgfpathlineto{\pgfqpoint{9.526700in}{1.963133in}}%
\pgfpathlineto{\pgfqpoint{9.535507in}{2.086648in}}%
\pgfpathlineto{\pgfqpoint{9.544313in}{2.038608in}}%
\pgfpathlineto{\pgfqpoint{9.553120in}{2.120970in}}%
\pgfpathlineto{\pgfqpoint{9.561927in}{1.997427in}}%
\pgfpathlineto{\pgfqpoint{9.570734in}{1.949388in}}%
\pgfpathlineto{\pgfqpoint{9.579541in}{1.867054in}}%
\pgfpathlineto{\pgfqpoint{9.588347in}{2.361168in}}%
\pgfpathlineto{\pgfqpoint{9.597154in}{2.505287in}}%
\pgfpathlineto{\pgfqpoint{9.605961in}{2.361168in}}%
\pgfpathlineto{\pgfqpoint{9.614768in}{2.052326in}}%
\pgfpathlineto{\pgfqpoint{9.623575in}{1.949388in}}%
\pgfpathlineto{\pgfqpoint{9.632382in}{2.155292in}}%
\pgfpathlineto{\pgfqpoint{9.641188in}{2.127829in}}%
\pgfpathlineto{\pgfqpoint{9.649995in}{2.059213in}}%
\pgfpathlineto{\pgfqpoint{9.658802in}{2.079789in}}%
\pgfpathlineto{\pgfqpoint{9.667609in}{2.175868in}}%
\pgfpathlineto{\pgfqpoint{9.676416in}{1.750370in}}%
\pgfpathlineto{\pgfqpoint{9.685222in}{1.455274in}}%
\pgfpathlineto{\pgfqpoint{9.694029in}{1.722935in}}%
\pgfpathlineto{\pgfqpoint{9.702836in}{2.169009in}}%
\pgfpathlineto{\pgfqpoint{9.711643in}{2.306269in}}%
\pgfpathlineto{\pgfqpoint{9.720450in}{2.326846in}}%
\pgfpathlineto{\pgfqpoint{9.729257in}{2.319987in}}%
\pgfpathlineto{\pgfqpoint{9.738063in}{2.498428in}}%
\pgfpathlineto{\pgfqpoint{9.746870in}{2.553326in}}%
\pgfpathlineto{\pgfqpoint{9.755677in}{2.436643in}}%
\pgfpathlineto{\pgfqpoint{9.764484in}{2.278806in}}%
\pgfpathlineto{\pgfqpoint{9.773291in}{2.189586in}}%
\pgfpathlineto{\pgfqpoint{9.782097in}{2.265089in}}%
\pgfpathlineto{\pgfqpoint{9.790904in}{2.114111in}}%
\pgfpathlineto{\pgfqpoint{9.799711in}{1.798410in}}%
\pgfpathlineto{\pgfqpoint{9.808518in}{1.942529in}}%
\pgfpathlineto{\pgfqpoint{9.817325in}{2.018032in}}%
\pgfpathlineto{\pgfqpoint{9.826132in}{1.990569in}}%
\pgfpathlineto{\pgfqpoint{9.834938in}{2.011173in}}%
\pgfpathlineto{\pgfqpoint{9.843745in}{1.825873in}}%
\pgfpathlineto{\pgfqpoint{9.852552in}{1.777834in}}%
\pgfpathlineto{\pgfqpoint{9.861359in}{1.757229in}}%
\pgfpathlineto{\pgfqpoint{9.878972in}{1.729794in}}%
\pgfpathlineto{\pgfqpoint{9.887779in}{1.764116in}}%
\pgfpathlineto{\pgfqpoint{9.896586in}{1.764116in}}%
\pgfpathlineto{\pgfqpoint{9.905393in}{1.681754in}}%
\pgfpathlineto{\pgfqpoint{9.914200in}{1.434698in}}%
\pgfpathlineto{\pgfqpoint{9.923007in}{1.372912in}}%
\pgfpathlineto{\pgfqpoint{9.931813in}{1.249398in}}%
\pgfpathlineto{\pgfqpoint{9.940620in}{1.475878in}}%
\pgfpathlineto{\pgfqpoint{9.949427in}{1.997427in}}%
\pgfpathlineto{\pgfqpoint{9.949427in}{1.997427in}}%
\pgfusepath{stroke}%
\end{pgfscope}%
\begin{pgfscope}%
\pgfpathrectangle{\pgfqpoint{0.702268in}{0.521603in}}{\pgfqpoint{9.687500in}{4.235000in}}%
\pgfusepath{clip}%
\pgfsetrectcap%
\pgfsetroundjoin%
\pgfsetlinewidth{0.501875pt}%
\definecolor{currentstroke}{rgb}{0.501961,0.501961,0.501961}%
\pgfsetstrokecolor{currentstroke}%
\pgfsetstrokeopacity{0.250000}%
\pgfsetdash{}{0pt}%
\pgfpathmoveto{\pgfqpoint{1.142609in}{3.946447in}}%
\pgfpathlineto{\pgfqpoint{1.160222in}{2.820960in}}%
\pgfpathlineto{\pgfqpoint{1.169029in}{2.814101in}}%
\pgfpathlineto{\pgfqpoint{1.177836in}{2.251343in}}%
\pgfpathlineto{\pgfqpoint{1.186643in}{2.196445in}}%
\pgfpathlineto{\pgfqpoint{1.195450in}{2.642547in}}%
\pgfpathlineto{\pgfqpoint{1.204257in}{2.697445in}}%
\pgfpathlineto{\pgfqpoint{1.213063in}{2.498428in}}%
\pgfpathlineto{\pgfqpoint{1.221870in}{2.223908in}}%
\pgfpathlineto{\pgfqpoint{1.230677in}{2.148405in}}%
\pgfpathlineto{\pgfqpoint{1.239484in}{2.237625in}}%
\pgfpathlineto{\pgfqpoint{1.248291in}{2.313128in}}%
\pgfpathlineto{\pgfqpoint{1.257097in}{2.182727in}}%
\pgfpathlineto{\pgfqpoint{1.265904in}{2.148405in}}%
\pgfpathlineto{\pgfqpoint{1.274711in}{2.175868in}}%
\pgfpathlineto{\pgfqpoint{1.283518in}{2.251343in}}%
\pgfpathlineto{\pgfqpoint{1.292325in}{2.388603in}}%
\pgfpathlineto{\pgfqpoint{1.301132in}{2.203303in}}%
\pgfpathlineto{\pgfqpoint{1.309938in}{1.976851in}}%
\pgfpathlineto{\pgfqpoint{1.318745in}{2.340563in}}%
\pgfpathlineto{\pgfqpoint{1.327552in}{2.642547in}}%
\pgfpathlineto{\pgfqpoint{1.336359in}{2.436643in}}%
\pgfpathlineto{\pgfqpoint{1.345166in}{2.134687in}}%
\pgfpathlineto{\pgfqpoint{1.353972in}{2.299383in}}%
\pgfpathlineto{\pgfqpoint{1.362779in}{2.230767in}}%
\pgfpathlineto{\pgfqpoint{1.371586in}{1.990569in}}%
\pgfpathlineto{\pgfqpoint{1.380393in}{2.244484in}}%
\pgfpathlineto{\pgfqpoint{1.389200in}{1.777834in}}%
\pgfpathlineto{\pgfqpoint{1.398007in}{1.482737in}}%
\pgfpathlineto{\pgfqpoint{1.406813in}{1.702331in}}%
\pgfpathlineto{\pgfqpoint{1.415620in}{2.189586in}}%
\pgfpathlineto{\pgfqpoint{1.424427in}{2.388603in}}%
\pgfpathlineto{\pgfqpoint{1.433234in}{2.374885in}}%
\pgfpathlineto{\pgfqpoint{1.442041in}{2.306269in}}%
\pgfpathlineto{\pgfqpoint{1.450847in}{2.189586in}}%
\pgfpathlineto{\pgfqpoint{1.459654in}{2.278806in}}%
\pgfpathlineto{\pgfqpoint{1.468461in}{2.271947in}}%
\pgfpathlineto{\pgfqpoint{1.477268in}{2.018032in}}%
\pgfpathlineto{\pgfqpoint{1.486075in}{1.832732in}}%
\pgfpathlineto{\pgfqpoint{1.494882in}{1.578816in}}%
\pgfpathlineto{\pgfqpoint{1.503688in}{1.626856in}}%
\pgfpathlineto{\pgfqpoint{1.512495in}{1.825873in}}%
\pgfpathlineto{\pgfqpoint{1.521302in}{2.059213in}}%
\pgfpathlineto{\pgfqpoint{1.530109in}{1.901348in}}%
\pgfpathlineto{\pgfqpoint{1.538916in}{1.592534in}}%
\pgfpathlineto{\pgfqpoint{1.547722in}{1.963133in}}%
\pgfpathlineto{\pgfqpoint{1.556529in}{2.134687in}}%
\pgfpathlineto{\pgfqpoint{1.565336in}{2.217049in}}%
\pgfpathlineto{\pgfqpoint{1.574143in}{2.141546in}}%
\pgfpathlineto{\pgfqpoint{1.582950in}{2.162151in}}%
\pgfpathlineto{\pgfqpoint{1.591757in}{2.100365in}}%
\pgfpathlineto{\pgfqpoint{1.600563in}{1.825873in}}%
\pgfpathlineto{\pgfqpoint{1.609370in}{2.093507in}}%
\pgfpathlineto{\pgfqpoint{1.618177in}{2.072930in}}%
\pgfpathlineto{\pgfqpoint{1.626984in}{2.237625in}}%
\pgfpathlineto{\pgfqpoint{1.635791in}{1.880772in}}%
\pgfpathlineto{\pgfqpoint{1.644597in}{1.798410in}}%
\pgfpathlineto{\pgfqpoint{1.653404in}{2.004286in}}%
\pgfpathlineto{\pgfqpoint{1.662211in}{1.942529in}}%
\pgfpathlineto{\pgfqpoint{1.671018in}{2.024891in}}%
\pgfpathlineto{\pgfqpoint{1.679825in}{1.976851in}}%
\pgfpathlineto{\pgfqpoint{1.688632in}{1.661150in}}%
\pgfpathlineto{\pgfqpoint{1.697438in}{1.764116in}}%
\pgfpathlineto{\pgfqpoint{1.706245in}{2.120970in}}%
\pgfpathlineto{\pgfqpoint{1.715052in}{1.750370in}}%
\pgfpathlineto{\pgfqpoint{1.723859in}{1.468992in}}%
\pgfpathlineto{\pgfqpoint{1.732666in}{1.427839in}}%
\pgfpathlineto{\pgfqpoint{1.741472in}{1.448415in}}%
\pgfpathlineto{\pgfqpoint{1.750279in}{1.997427in}}%
\pgfpathlineto{\pgfqpoint{1.759086in}{1.921953in}}%
\pgfpathlineto{\pgfqpoint{1.767893in}{2.018032in}}%
\pgfpathlineto{\pgfqpoint{1.776700in}{2.031749in}}%
\pgfpathlineto{\pgfqpoint{1.785507in}{1.860167in}}%
\pgfpathlineto{\pgfqpoint{1.794313in}{1.990569in}}%
\pgfpathlineto{\pgfqpoint{1.803120in}{1.750370in}}%
\pgfpathlineto{\pgfqpoint{1.811927in}{1.770975in}}%
\pgfpathlineto{\pgfqpoint{1.820734in}{1.887630in}}%
\pgfpathlineto{\pgfqpoint{1.829541in}{1.921953in}}%
\pgfpathlineto{\pgfqpoint{1.838347in}{2.182727in}}%
\pgfpathlineto{\pgfqpoint{1.847154in}{2.217049in}}%
\pgfpathlineto{\pgfqpoint{1.864768in}{2.361168in}}%
\pgfpathlineto{\pgfqpoint{1.873575in}{2.059213in}}%
\pgfpathlineto{\pgfqpoint{1.882382in}{2.100365in}}%
\pgfpathlineto{\pgfqpoint{1.891188in}{1.839591in}}%
\pgfpathlineto{\pgfqpoint{1.899995in}{1.908207in}}%
\pgfpathlineto{\pgfqpoint{1.908802in}{1.928811in}}%
\pgfpathlineto{\pgfqpoint{1.917609in}{1.757229in}}%
\pgfpathlineto{\pgfqpoint{1.926416in}{1.688613in}}%
\pgfpathlineto{\pgfqpoint{1.935222in}{1.729794in}}%
\pgfpathlineto{\pgfqpoint{1.944029in}{2.024891in}}%
\pgfpathlineto{\pgfqpoint{1.952836in}{1.812128in}}%
\pgfpathlineto{\pgfqpoint{1.970450in}{1.654291in}}%
\pgfpathlineto{\pgfqpoint{1.979257in}{1.503314in}}%
\pgfpathlineto{\pgfqpoint{1.988063in}{1.668037in}}%
\pgfpathlineto{\pgfqpoint{1.996870in}{1.894489in}}%
\pgfpathlineto{\pgfqpoint{2.005677in}{1.963133in}}%
\pgfpathlineto{\pgfqpoint{2.014484in}{2.141546in}}%
\pgfpathlineto{\pgfqpoint{2.023291in}{1.688613in}}%
\pgfpathlineto{\pgfqpoint{2.032097in}{1.750370in}}%
\pgfpathlineto{\pgfqpoint{2.040904in}{1.709190in}}%
\pgfpathlineto{\pgfqpoint{2.049711in}{1.599393in}}%
\pgfpathlineto{\pgfqpoint{2.058518in}{1.544494in}}%
\pgfpathlineto{\pgfqpoint{2.067325in}{1.757229in}}%
\pgfpathlineto{\pgfqpoint{2.076132in}{2.004286in}}%
\pgfpathlineto{\pgfqpoint{2.084938in}{1.805269in}}%
\pgfpathlineto{\pgfqpoint{2.093745in}{1.571958in}}%
\pgfpathlineto{\pgfqpoint{2.102552in}{1.915094in}}%
\pgfpathlineto{\pgfqpoint{2.111359in}{1.969992in}}%
\pgfpathlineto{\pgfqpoint{2.120166in}{2.011173in}}%
\pgfpathlineto{\pgfqpoint{2.128972in}{1.997427in}}%
\pgfpathlineto{\pgfqpoint{2.137779in}{1.736653in}}%
\pgfpathlineto{\pgfqpoint{2.146586in}{2.031749in}}%
\pgfpathlineto{\pgfqpoint{2.155393in}{1.729794in}}%
\pgfpathlineto{\pgfqpoint{2.164200in}{1.503314in}}%
\pgfpathlineto{\pgfqpoint{2.173007in}{1.606252in}}%
\pgfpathlineto{\pgfqpoint{2.181813in}{1.523918in}}%
\pgfpathlineto{\pgfqpoint{2.190620in}{1.468992in}}%
\pgfpathlineto{\pgfqpoint{2.208234in}{1.647432in}}%
\pgfpathlineto{\pgfqpoint{2.217041in}{1.791551in}}%
\pgfpathlineto{\pgfqpoint{2.225847in}{1.770975in}}%
\pgfpathlineto{\pgfqpoint{2.234654in}{2.038608in}}%
\pgfpathlineto{\pgfqpoint{2.243461in}{1.976851in}}%
\pgfpathlineto{\pgfqpoint{2.252268in}{2.354309in}}%
\pgfpathlineto{\pgfqpoint{2.261075in}{1.674896in}}%
\pgfpathlineto{\pgfqpoint{2.269882in}{1.647432in}}%
\pgfpathlineto{\pgfqpoint{2.278688in}{1.585675in}}%
\pgfpathlineto{\pgfqpoint{2.287495in}{1.770975in}}%
\pgfpathlineto{\pgfqpoint{2.296302in}{1.867054in}}%
\pgfpathlineto{\pgfqpoint{2.305109in}{1.626856in}}%
\pgfpathlineto{\pgfqpoint{2.313916in}{1.537636in}}%
\pgfpathlineto{\pgfqpoint{2.322722in}{1.372912in}}%
\pgfpathlineto{\pgfqpoint{2.331529in}{1.599393in}}%
\pgfpathlineto{\pgfqpoint{2.340336in}{1.942529in}}%
\pgfpathlineto{\pgfqpoint{2.349143in}{2.230767in}}%
\pgfpathlineto{\pgfqpoint{2.357950in}{2.265089in}}%
\pgfpathlineto{\pgfqpoint{2.366757in}{2.244484in}}%
\pgfpathlineto{\pgfqpoint{2.375563in}{2.470965in}}%
\pgfpathlineto{\pgfqpoint{2.384370in}{2.120970in}}%
\pgfpathlineto{\pgfqpoint{2.393177in}{1.935670in}}%
\pgfpathlineto{\pgfqpoint{2.401984in}{2.024891in}}%
\pgfpathlineto{\pgfqpoint{2.410791in}{1.565071in}}%
\pgfpathlineto{\pgfqpoint{2.419597in}{1.407234in}}%
\pgfpathlineto{\pgfqpoint{2.428404in}{1.359195in}}%
\pgfpathlineto{\pgfqpoint{2.437211in}{1.242539in}}%
\pgfpathlineto{\pgfqpoint{2.446018in}{1.379799in}}%
\pgfpathlineto{\pgfqpoint{2.454825in}{1.441556in}}%
\pgfpathlineto{\pgfqpoint{2.463632in}{1.674896in}}%
\pgfpathlineto{\pgfqpoint{2.472438in}{1.695472in}}%
\pgfpathlineto{\pgfqpoint{2.481245in}{2.018032in}}%
\pgfpathlineto{\pgfqpoint{2.490052in}{1.661150in}}%
\pgfpathlineto{\pgfqpoint{2.498859in}{1.475878in}}%
\pgfpathlineto{\pgfqpoint{2.507666in}{1.633715in}}%
\pgfpathlineto{\pgfqpoint{2.516472in}{2.038608in}}%
\pgfpathlineto{\pgfqpoint{2.525279in}{1.901348in}}%
\pgfpathlineto{\pgfqpoint{2.542893in}{2.072930in}}%
\pgfpathlineto{\pgfqpoint{2.551700in}{1.853308in}}%
\pgfpathlineto{\pgfqpoint{2.560507in}{2.182727in}}%
\pgfpathlineto{\pgfqpoint{2.569313in}{2.120970in}}%
\pgfpathlineto{\pgfqpoint{2.578120in}{1.695472in}}%
\pgfpathlineto{\pgfqpoint{2.586927in}{2.093507in}}%
\pgfpathlineto{\pgfqpoint{2.595734in}{1.832732in}}%
\pgfpathlineto{\pgfqpoint{2.604541in}{1.997427in}}%
\pgfpathlineto{\pgfqpoint{2.613347in}{2.004286in}}%
\pgfpathlineto{\pgfqpoint{2.622154in}{2.210190in}}%
\pgfpathlineto{\pgfqpoint{2.630961in}{2.107252in}}%
\pgfpathlineto{\pgfqpoint{2.639768in}{1.812128in}}%
\pgfpathlineto{\pgfqpoint{2.648575in}{1.935670in}}%
\pgfpathlineto{\pgfqpoint{2.657382in}{1.729794in}}%
\pgfpathlineto{\pgfqpoint{2.666188in}{1.894489in}}%
\pgfpathlineto{\pgfqpoint{2.674995in}{1.688613in}}%
\pgfpathlineto{\pgfqpoint{2.683802in}{1.805269in}}%
\pgfpathlineto{\pgfqpoint{2.692609in}{1.640574in}}%
\pgfpathlineto{\pgfqpoint{2.701416in}{1.702331in}}%
\pgfpathlineto{\pgfqpoint{2.710222in}{1.571958in}}%
\pgfpathlineto{\pgfqpoint{2.719029in}{1.606252in}}%
\pgfpathlineto{\pgfqpoint{2.727836in}{1.496455in}}%
\pgfpathlineto{\pgfqpoint{2.736643in}{1.613110in}}%
\pgfpathlineto{\pgfqpoint{2.745450in}{1.949388in}}%
\pgfpathlineto{\pgfqpoint{2.754257in}{1.921953in}}%
\pgfpathlineto{\pgfqpoint{2.763063in}{2.018032in}}%
\pgfpathlineto{\pgfqpoint{2.771870in}{2.052326in}}%
\pgfpathlineto{\pgfqpoint{2.780677in}{2.066071in}}%
\pgfpathlineto{\pgfqpoint{2.789484in}{2.066071in}}%
\pgfpathlineto{\pgfqpoint{2.798291in}{2.395462in}}%
\pgfpathlineto{\pgfqpoint{2.807097in}{2.402349in}}%
\pgfpathlineto{\pgfqpoint{2.815904in}{2.601366in}}%
\pgfpathlineto{\pgfqpoint{2.824711in}{2.354309in}}%
\pgfpathlineto{\pgfqpoint{2.833518in}{2.210190in}}%
\pgfpathlineto{\pgfqpoint{2.842325in}{2.217049in}}%
\pgfpathlineto{\pgfqpoint{2.851132in}{2.402349in}}%
\pgfpathlineto{\pgfqpoint{2.859938in}{2.230767in}}%
\pgfpathlineto{\pgfqpoint{2.868745in}{1.592534in}}%
\pgfpathlineto{\pgfqpoint{2.877552in}{1.311155in}}%
\pgfpathlineto{\pgfqpoint{2.886359in}{1.455274in}}%
\pgfpathlineto{\pgfqpoint{2.895166in}{1.338618in}}%
\pgfpathlineto{\pgfqpoint{2.903972in}{1.613110in}}%
\pgfpathlineto{\pgfqpoint{2.912779in}{1.839591in}}%
\pgfpathlineto{\pgfqpoint{2.930393in}{1.441556in}}%
\pgfpathlineto{\pgfqpoint{2.939200in}{1.517031in}}%
\pgfpathlineto{\pgfqpoint{2.948007in}{2.018032in}}%
\pgfpathlineto{\pgfqpoint{2.956813in}{2.361168in}}%
\pgfpathlineto{\pgfqpoint{2.965620in}{2.251343in}}%
\pgfpathlineto{\pgfqpoint{2.974427in}{2.271947in}}%
\pgfpathlineto{\pgfqpoint{2.983234in}{1.935670in}}%
\pgfpathlineto{\pgfqpoint{2.992041in}{2.100365in}}%
\pgfpathlineto{\pgfqpoint{3.000847in}{1.990569in}}%
\pgfpathlineto{\pgfqpoint{3.009654in}{1.901348in}}%
\pgfpathlineto{\pgfqpoint{3.018461in}{1.757229in}}%
\pgfpathlineto{\pgfqpoint{3.027268in}{1.537636in}}%
\pgfpathlineto{\pgfqpoint{3.036075in}{1.606252in}}%
\pgfpathlineto{\pgfqpoint{3.044882in}{1.565071in}}%
\pgfpathlineto{\pgfqpoint{3.053688in}{1.674896in}}%
\pgfpathlineto{\pgfqpoint{3.062495in}{2.072930in}}%
\pgfpathlineto{\pgfqpoint{3.071302in}{2.175868in}}%
\pgfpathlineto{\pgfqpoint{3.080109in}{1.894489in}}%
\pgfpathlineto{\pgfqpoint{3.088916in}{1.853308in}}%
\pgfpathlineto{\pgfqpoint{3.097722in}{1.626856in}}%
\pgfpathlineto{\pgfqpoint{3.106529in}{1.503314in}}%
\pgfpathlineto{\pgfqpoint{3.115336in}{1.510172in}}%
\pgfpathlineto{\pgfqpoint{3.124143in}{1.366053in}}%
\pgfpathlineto{\pgfqpoint{3.132950in}{1.633715in}}%
\pgfpathlineto{\pgfqpoint{3.141757in}{1.784692in}}%
\pgfpathlineto{\pgfqpoint{3.150563in}{1.599393in}}%
\pgfpathlineto{\pgfqpoint{3.159370in}{1.956246in}}%
\pgfpathlineto{\pgfqpoint{3.168177in}{1.750370in}}%
\pgfpathlineto{\pgfqpoint{3.176984in}{1.585675in}}%
\pgfpathlineto{\pgfqpoint{3.185791in}{1.894489in}}%
\pgfpathlineto{\pgfqpoint{3.194597in}{2.038608in}}%
\pgfpathlineto{\pgfqpoint{3.203404in}{2.278806in}}%
\pgfpathlineto{\pgfqpoint{3.212211in}{1.928811in}}%
\pgfpathlineto{\pgfqpoint{3.229825in}{1.599393in}}%
\pgfpathlineto{\pgfqpoint{3.238632in}{1.791551in}}%
\pgfpathlineto{\pgfqpoint{3.247438in}{1.894489in}}%
\pgfpathlineto{\pgfqpoint{3.256245in}{1.805269in}}%
\pgfpathlineto{\pgfqpoint{3.265052in}{1.894489in}}%
\pgfpathlineto{\pgfqpoint{3.273859in}{1.619997in}}%
\pgfpathlineto{\pgfqpoint{3.282666in}{1.661150in}}%
\pgfpathlineto{\pgfqpoint{3.291472in}{1.812128in}}%
\pgfpathlineto{\pgfqpoint{3.300279in}{2.182727in}}%
\pgfpathlineto{\pgfqpoint{3.309086in}{2.230767in}}%
\pgfpathlineto{\pgfqpoint{3.317893in}{2.244484in}}%
\pgfpathlineto{\pgfqpoint{3.326700in}{2.395462in}}%
\pgfpathlineto{\pgfqpoint{3.335507in}{2.230767in}}%
\pgfpathlineto{\pgfqpoint{3.344313in}{2.244484in}}%
\pgfpathlineto{\pgfqpoint{3.353120in}{2.285665in}}%
\pgfpathlineto{\pgfqpoint{3.361927in}{2.333705in}}%
\pgfpathlineto{\pgfqpoint{3.370734in}{2.162151in}}%
\pgfpathlineto{\pgfqpoint{3.379541in}{2.100365in}}%
\pgfpathlineto{\pgfqpoint{3.388347in}{2.265089in}}%
\pgfpathlineto{\pgfqpoint{3.397154in}{2.120970in}}%
\pgfpathlineto{\pgfqpoint{3.405961in}{2.066071in}}%
\pgfpathlineto{\pgfqpoint{3.414768in}{2.251343in}}%
\pgfpathlineto{\pgfqpoint{3.423575in}{2.169009in}}%
\pgfpathlineto{\pgfqpoint{3.432382in}{2.210190in}}%
\pgfpathlineto{\pgfqpoint{3.441188in}{2.223908in}}%
\pgfpathlineto{\pgfqpoint{3.449995in}{1.633715in}}%
\pgfpathlineto{\pgfqpoint{3.458802in}{1.716076in}}%
\pgfpathlineto{\pgfqpoint{3.467609in}{1.825873in}}%
\pgfpathlineto{\pgfqpoint{3.476416in}{2.107252in}}%
\pgfpathlineto{\pgfqpoint{3.485222in}{1.976851in}}%
\pgfpathlineto{\pgfqpoint{3.494029in}{2.114111in}}%
\pgfpathlineto{\pgfqpoint{3.502836in}{1.901348in}}%
\pgfpathlineto{\pgfqpoint{3.511643in}{2.251343in}}%
\pgfpathlineto{\pgfqpoint{3.520450in}{2.045467in}}%
\pgfpathlineto{\pgfqpoint{3.529257in}{2.313128in}}%
\pgfpathlineto{\pgfqpoint{3.538063in}{2.326846in}}%
\pgfpathlineto{\pgfqpoint{3.546870in}{1.784692in}}%
\pgfpathlineto{\pgfqpoint{3.555677in}{1.709190in}}%
\pgfpathlineto{\pgfqpoint{3.564484in}{1.757229in}}%
\pgfpathlineto{\pgfqpoint{3.573291in}{1.702331in}}%
\pgfpathlineto{\pgfqpoint{3.582097in}{1.626856in}}%
\pgfpathlineto{\pgfqpoint{3.590904in}{2.024891in}}%
\pgfpathlineto{\pgfqpoint{3.599711in}{1.681754in}}%
\pgfpathlineto{\pgfqpoint{3.608518in}{1.695472in}}%
\pgfpathlineto{\pgfqpoint{3.617325in}{1.784692in}}%
\pgfpathlineto{\pgfqpoint{3.626132in}{1.784692in}}%
\pgfpathlineto{\pgfqpoint{3.634938in}{1.963133in}}%
\pgfpathlineto{\pgfqpoint{3.643745in}{1.757229in}}%
\pgfpathlineto{\pgfqpoint{3.652552in}{1.688613in}}%
\pgfpathlineto{\pgfqpoint{3.661359in}{1.757229in}}%
\pgfpathlineto{\pgfqpoint{3.670166in}{1.585675in}}%
\pgfpathlineto{\pgfqpoint{3.678972in}{1.860167in}}%
\pgfpathlineto{\pgfqpoint{3.687779in}{1.990569in}}%
\pgfpathlineto{\pgfqpoint{3.696586in}{1.619997in}}%
\pgfpathlineto{\pgfqpoint{3.705393in}{1.674896in}}%
\pgfpathlineto{\pgfqpoint{3.714200in}{1.805269in}}%
\pgfpathlineto{\pgfqpoint{3.723007in}{1.798410in}}%
\pgfpathlineto{\pgfqpoint{3.731813in}{1.949388in}}%
\pgfpathlineto{\pgfqpoint{3.740620in}{1.990569in}}%
\pgfpathlineto{\pgfqpoint{3.749427in}{2.100365in}}%
\pgfpathlineto{\pgfqpoint{3.758234in}{2.628801in}}%
\pgfpathlineto{\pgfqpoint{3.767041in}{2.484682in}}%
\pgfpathlineto{\pgfqpoint{3.775847in}{2.072930in}}%
\pgfpathlineto{\pgfqpoint{3.784654in}{2.182727in}}%
\pgfpathlineto{\pgfqpoint{3.793461in}{2.436643in}}%
\pgfpathlineto{\pgfqpoint{3.802268in}{2.141546in}}%
\pgfpathlineto{\pgfqpoint{3.811075in}{2.148405in}}%
\pgfpathlineto{\pgfqpoint{3.819882in}{1.908207in}}%
\pgfpathlineto{\pgfqpoint{3.828688in}{1.846450in}}%
\pgfpathlineto{\pgfqpoint{3.837495in}{1.681754in}}%
\pgfpathlineto{\pgfqpoint{3.846302in}{1.565071in}}%
\pgfpathlineto{\pgfqpoint{3.855109in}{1.661150in}}%
\pgfpathlineto{\pgfqpoint{3.863916in}{1.791551in}}%
\pgfpathlineto{\pgfqpoint{3.872722in}{1.825873in}}%
\pgfpathlineto{\pgfqpoint{3.881529in}{2.285665in}}%
\pgfpathlineto{\pgfqpoint{3.890336in}{2.429784in}}%
\pgfpathlineto{\pgfqpoint{3.899143in}{2.326846in}}%
\pgfpathlineto{\pgfqpoint{3.907950in}{1.990569in}}%
\pgfpathlineto{\pgfqpoint{3.916757in}{2.066071in}}%
\pgfpathlineto{\pgfqpoint{3.925563in}{2.175868in}}%
\pgfpathlineto{\pgfqpoint{3.934370in}{1.997427in}}%
\pgfpathlineto{\pgfqpoint{3.943177in}{2.018032in}}%
\pgfpathlineto{\pgfqpoint{3.951984in}{1.613110in}}%
\pgfpathlineto{\pgfqpoint{3.960791in}{1.599393in}}%
\pgfpathlineto{\pgfqpoint{3.969597in}{1.729794in}}%
\pgfpathlineto{\pgfqpoint{3.978404in}{1.668037in}}%
\pgfpathlineto{\pgfqpoint{3.987211in}{1.928811in}}%
\pgfpathlineto{\pgfqpoint{3.996018in}{2.079789in}}%
\pgfpathlineto{\pgfqpoint{4.004825in}{1.860167in}}%
\pgfpathlineto{\pgfqpoint{4.022438in}{1.832732in}}%
\pgfpathlineto{\pgfqpoint{4.031245in}{1.640574in}}%
\pgfpathlineto{\pgfqpoint{4.040052in}{1.523918in}}%
\pgfpathlineto{\pgfqpoint{4.048859in}{1.688613in}}%
\pgfpathlineto{\pgfqpoint{4.057666in}{1.510172in}}%
\pgfpathlineto{\pgfqpoint{4.066472in}{1.750370in}}%
\pgfpathlineto{\pgfqpoint{4.075279in}{2.182727in}}%
\pgfpathlineto{\pgfqpoint{4.084086in}{2.127829in}}%
\pgfpathlineto{\pgfqpoint{4.092893in}{1.956246in}}%
\pgfpathlineto{\pgfqpoint{4.101700in}{1.640574in}}%
\pgfpathlineto{\pgfqpoint{4.110507in}{1.551353in}}%
\pgfpathlineto{\pgfqpoint{4.119313in}{1.187641in}}%
\pgfpathlineto{\pgfqpoint{4.128120in}{1.386658in}}%
\pgfpathlineto{\pgfqpoint{4.136927in}{1.077816in}}%
\pgfpathlineto{\pgfqpoint{4.145734in}{1.160177in}}%
\pgfpathlineto{\pgfqpoint{4.154541in}{1.489596in}}%
\pgfpathlineto{\pgfqpoint{4.163347in}{1.640574in}}%
\pgfpathlineto{\pgfqpoint{4.172154in}{1.661150in}}%
\pgfpathlineto{\pgfqpoint{4.180961in}{1.949388in}}%
\pgfpathlineto{\pgfqpoint{4.189768in}{1.956246in}}%
\pgfpathlineto{\pgfqpoint{4.198575in}{1.921953in}}%
\pgfpathlineto{\pgfqpoint{4.207382in}{1.976851in}}%
\pgfpathlineto{\pgfqpoint{4.216188in}{1.585675in}}%
\pgfpathlineto{\pgfqpoint{4.224995in}{1.517031in}}%
\pgfpathlineto{\pgfqpoint{4.233802in}{1.764116in}}%
\pgfpathlineto{\pgfqpoint{4.242609in}{1.496455in}}%
\pgfpathlineto{\pgfqpoint{4.251416in}{1.304296in}}%
\pgfpathlineto{\pgfqpoint{4.260222in}{1.414093in}}%
\pgfpathlineto{\pgfqpoint{4.269029in}{1.613110in}}%
\pgfpathlineto{\pgfqpoint{4.277836in}{1.846450in}}%
\pgfpathlineto{\pgfqpoint{4.286643in}{1.592534in}}%
\pgfpathlineto{\pgfqpoint{4.295450in}{1.640574in}}%
\pgfpathlineto{\pgfqpoint{4.304257in}{1.716076in}}%
\pgfpathlineto{\pgfqpoint{4.313063in}{1.537636in}}%
\pgfpathlineto{\pgfqpoint{4.321870in}{1.462133in}}%
\pgfpathlineto{\pgfqpoint{4.330677in}{1.503314in}}%
\pgfpathlineto{\pgfqpoint{4.339484in}{1.414093in}}%
\pgfpathlineto{\pgfqpoint{4.348291in}{1.530777in}}%
\pgfpathlineto{\pgfqpoint{4.357097in}{2.148405in}}%
\pgfpathlineto{\pgfqpoint{4.365904in}{1.976851in}}%
\pgfpathlineto{\pgfqpoint{4.374711in}{1.894489in}}%
\pgfpathlineto{\pgfqpoint{4.383518in}{1.791551in}}%
\pgfpathlineto{\pgfqpoint{4.392325in}{1.894489in}}%
\pgfpathlineto{\pgfqpoint{4.401132in}{1.839591in}}%
\pgfpathlineto{\pgfqpoint{4.409938in}{1.764116in}}%
\pgfpathlineto{\pgfqpoint{4.418745in}{1.935670in}}%
\pgfpathlineto{\pgfqpoint{4.427552in}{2.416066in}}%
\pgfpathlineto{\pgfqpoint{4.436359in}{2.107252in}}%
\pgfpathlineto{\pgfqpoint{4.445166in}{1.983710in}}%
\pgfpathlineto{\pgfqpoint{4.453972in}{1.894489in}}%
\pgfpathlineto{\pgfqpoint{4.462779in}{1.969992in}}%
\pgfpathlineto{\pgfqpoint{4.471586in}{2.024891in}}%
\pgfpathlineto{\pgfqpoint{4.480393in}{1.709190in}}%
\pgfpathlineto{\pgfqpoint{4.489200in}{1.510172in}}%
\pgfpathlineto{\pgfqpoint{4.498007in}{1.798410in}}%
\pgfpathlineto{\pgfqpoint{4.506813in}{1.764116in}}%
\pgfpathlineto{\pgfqpoint{4.515620in}{1.420952in}}%
\pgfpathlineto{\pgfqpoint{4.524427in}{1.283720in}}%
\pgfpathlineto{\pgfqpoint{4.533234in}{1.441556in}}%
\pgfpathlineto{\pgfqpoint{4.542041in}{1.633715in}}%
\pgfpathlineto{\pgfqpoint{4.550847in}{1.331760in}}%
\pgfpathlineto{\pgfqpoint{4.559654in}{1.345477in}}%
\pgfpathlineto{\pgfqpoint{4.568461in}{1.633715in}}%
\pgfpathlineto{\pgfqpoint{4.577268in}{1.729794in}}%
\pgfpathlineto{\pgfqpoint{4.586075in}{1.722935in}}%
\pgfpathlineto{\pgfqpoint{4.594882in}{2.052326in}}%
\pgfpathlineto{\pgfqpoint{4.603688in}{1.784692in}}%
\pgfpathlineto{\pgfqpoint{4.612495in}{1.791551in}}%
\pgfpathlineto{\pgfqpoint{4.621302in}{1.537636in}}%
\pgfpathlineto{\pgfqpoint{4.630109in}{1.921953in}}%
\pgfpathlineto{\pgfqpoint{4.638916in}{1.606252in}}%
\pgfpathlineto{\pgfqpoint{4.647722in}{1.819015in}}%
\pgfpathlineto{\pgfqpoint{4.656529in}{1.853308in}}%
\pgfpathlineto{\pgfqpoint{4.665336in}{1.935670in}}%
\pgfpathlineto{\pgfqpoint{4.674143in}{2.079789in}}%
\pgfpathlineto{\pgfqpoint{4.682950in}{1.976851in}}%
\pgfpathlineto{\pgfqpoint{4.691757in}{1.997427in}}%
\pgfpathlineto{\pgfqpoint{4.700563in}{2.155292in}}%
\pgfpathlineto{\pgfqpoint{4.709370in}{1.949388in}}%
\pgfpathlineto{\pgfqpoint{4.718177in}{1.839591in}}%
\pgfpathlineto{\pgfqpoint{4.726984in}{1.764116in}}%
\pgfpathlineto{\pgfqpoint{4.735791in}{1.853308in}}%
\pgfpathlineto{\pgfqpoint{4.744597in}{2.072930in}}%
\pgfpathlineto{\pgfqpoint{4.753404in}{2.141546in}}%
\pgfpathlineto{\pgfqpoint{4.762211in}{2.361168in}}%
\pgfpathlineto{\pgfqpoint{4.771018in}{2.443501in}}%
\pgfpathlineto{\pgfqpoint{4.779825in}{2.326846in}}%
\pgfpathlineto{\pgfqpoint{4.788632in}{2.567044in}}%
\pgfpathlineto{\pgfqpoint{4.797438in}{2.265089in}}%
\pgfpathlineto{\pgfqpoint{4.815052in}{2.615084in}}%
\pgfpathlineto{\pgfqpoint{4.823859in}{2.546468in}}%
\pgfpathlineto{\pgfqpoint{4.832666in}{2.546468in}}%
\pgfpathlineto{\pgfqpoint{4.841472in}{2.512146in}}%
\pgfpathlineto{\pgfqpoint{4.859086in}{1.921953in}}%
\pgfpathlineto{\pgfqpoint{4.867893in}{1.770975in}}%
\pgfpathlineto{\pgfqpoint{4.876700in}{1.949388in}}%
\pgfpathlineto{\pgfqpoint{4.885507in}{2.045467in}}%
\pgfpathlineto{\pgfqpoint{4.894313in}{1.894489in}}%
\pgfpathlineto{\pgfqpoint{4.903120in}{2.004286in}}%
\pgfpathlineto{\pgfqpoint{4.911927in}{1.908207in}}%
\pgfpathlineto{\pgfqpoint{4.920734in}{1.928811in}}%
\pgfpathlineto{\pgfqpoint{4.929541in}{1.901348in}}%
\pgfpathlineto{\pgfqpoint{4.938347in}{1.997427in}}%
\pgfpathlineto{\pgfqpoint{4.947154in}{1.928811in}}%
\pgfpathlineto{\pgfqpoint{4.955961in}{2.127829in}}%
\pgfpathlineto{\pgfqpoint{4.964768in}{2.155292in}}%
\pgfpathlineto{\pgfqpoint{4.973575in}{2.079789in}}%
\pgfpathlineto{\pgfqpoint{4.982382in}{1.990569in}}%
\pgfpathlineto{\pgfqpoint{4.991188in}{1.956246in}}%
\pgfpathlineto{\pgfqpoint{4.999995in}{1.798410in}}%
\pgfpathlineto{\pgfqpoint{5.008802in}{2.100365in}}%
\pgfpathlineto{\pgfqpoint{5.017609in}{1.860167in}}%
\pgfpathlineto{\pgfqpoint{5.026416in}{2.059213in}}%
\pgfpathlineto{\pgfqpoint{5.035222in}{1.825873in}}%
\pgfpathlineto{\pgfqpoint{5.044029in}{1.832732in}}%
\pgfpathlineto{\pgfqpoint{5.052836in}{1.633715in}}%
\pgfpathlineto{\pgfqpoint{5.061643in}{1.736653in}}%
\pgfpathlineto{\pgfqpoint{5.079257in}{1.709190in}}%
\pgfpathlineto{\pgfqpoint{5.088063in}{1.455274in}}%
\pgfpathlineto{\pgfqpoint{5.105677in}{2.018032in}}%
\pgfpathlineto{\pgfqpoint{5.114484in}{1.921953in}}%
\pgfpathlineto{\pgfqpoint{5.123291in}{2.340563in}}%
\pgfpathlineto{\pgfqpoint{5.132097in}{1.901348in}}%
\pgfpathlineto{\pgfqpoint{5.140904in}{2.285665in}}%
\pgfpathlineto{\pgfqpoint{5.149711in}{2.230767in}}%
\pgfpathlineto{\pgfqpoint{5.158518in}{2.244484in}}%
\pgfpathlineto{\pgfqpoint{5.167325in}{2.388603in}}%
\pgfpathlineto{\pgfqpoint{5.176132in}{2.512146in}}%
\pgfpathlineto{\pgfqpoint{5.184938in}{2.560185in}}%
\pgfpathlineto{\pgfqpoint{5.193745in}{2.223908in}}%
\pgfpathlineto{\pgfqpoint{5.202552in}{2.217049in}}%
\pgfpathlineto{\pgfqpoint{5.211359in}{2.175868in}}%
\pgfpathlineto{\pgfqpoint{5.220166in}{2.031749in}}%
\pgfpathlineto{\pgfqpoint{5.237779in}{1.153319in}}%
\pgfpathlineto{\pgfqpoint{5.246586in}{1.606252in}}%
\pgfpathlineto{\pgfqpoint{5.264200in}{1.578816in}}%
\pgfpathlineto{\pgfqpoint{5.273007in}{1.688613in}}%
\pgfpathlineto{\pgfqpoint{5.281813in}{1.640574in}}%
\pgfpathlineto{\pgfqpoint{5.290620in}{1.846450in}}%
\pgfpathlineto{\pgfqpoint{5.299427in}{1.915094in}}%
\pgfpathlineto{\pgfqpoint{5.308234in}{2.333705in}}%
\pgfpathlineto{\pgfqpoint{5.317041in}{2.038608in}}%
\pgfpathlineto{\pgfqpoint{5.325847in}{2.237625in}}%
\pgfpathlineto{\pgfqpoint{5.334654in}{2.251343in}}%
\pgfpathlineto{\pgfqpoint{5.343461in}{2.361168in}}%
\pgfpathlineto{\pgfqpoint{5.352268in}{2.663123in}}%
\pgfpathlineto{\pgfqpoint{5.361075in}{2.203303in}}%
\pgfpathlineto{\pgfqpoint{5.369882in}{2.011173in}}%
\pgfpathlineto{\pgfqpoint{5.378688in}{1.880772in}}%
\pgfpathlineto{\pgfqpoint{5.387495in}{1.805269in}}%
\pgfpathlineto{\pgfqpoint{5.396302in}{1.681754in}}%
\pgfpathlineto{\pgfqpoint{5.405109in}{1.791551in}}%
\pgfpathlineto{\pgfqpoint{5.422722in}{2.127829in}}%
\pgfpathlineto{\pgfqpoint{5.431529in}{2.189586in}}%
\pgfpathlineto{\pgfqpoint{5.440336in}{2.079789in}}%
\pgfpathlineto{\pgfqpoint{5.449143in}{1.825873in}}%
\pgfpathlineto{\pgfqpoint{5.457950in}{1.894489in}}%
\pgfpathlineto{\pgfqpoint{5.466757in}{1.853308in}}%
\pgfpathlineto{\pgfqpoint{5.475563in}{1.825873in}}%
\pgfpathlineto{\pgfqpoint{5.484370in}{2.141546in}}%
\pgfpathlineto{\pgfqpoint{5.493177in}{2.258230in}}%
\pgfpathlineto{\pgfqpoint{5.501984in}{1.832732in}}%
\pgfpathlineto{\pgfqpoint{5.510791in}{1.606252in}}%
\pgfpathlineto{\pgfqpoint{5.519597in}{1.633715in}}%
\pgfpathlineto{\pgfqpoint{5.528404in}{1.674896in}}%
\pgfpathlineto{\pgfqpoint{5.537211in}{1.565071in}}%
\pgfpathlineto{\pgfqpoint{5.546018in}{1.510172in}}%
\pgfpathlineto{\pgfqpoint{5.554825in}{1.482737in}}%
\pgfpathlineto{\pgfqpoint{5.563632in}{1.702331in}}%
\pgfpathlineto{\pgfqpoint{5.572438in}{1.736653in}}%
\pgfpathlineto{\pgfqpoint{5.581245in}{1.702331in}}%
\pgfpathlineto{\pgfqpoint{5.590052in}{1.969992in}}%
\pgfpathlineto{\pgfqpoint{5.598859in}{1.757229in}}%
\pgfpathlineto{\pgfqpoint{5.607666in}{1.510172in}}%
\pgfpathlineto{\pgfqpoint{5.616472in}{1.702331in}}%
\pgfpathlineto{\pgfqpoint{5.625279in}{1.997427in}}%
\pgfpathlineto{\pgfqpoint{5.634086in}{2.100365in}}%
\pgfpathlineto{\pgfqpoint{5.642893in}{1.873913in}}%
\pgfpathlineto{\pgfqpoint{5.651700in}{1.969992in}}%
\pgfpathlineto{\pgfqpoint{5.669313in}{2.237625in}}%
\pgfpathlineto{\pgfqpoint{5.678120in}{2.230767in}}%
\pgfpathlineto{\pgfqpoint{5.686927in}{2.258230in}}%
\pgfpathlineto{\pgfqpoint{5.695734in}{2.141546in}}%
\pgfpathlineto{\pgfqpoint{5.704541in}{2.374885in}}%
\pgfpathlineto{\pgfqpoint{5.713347in}{2.265089in}}%
\pgfpathlineto{\pgfqpoint{5.722154in}{2.402349in}}%
\pgfpathlineto{\pgfqpoint{5.730961in}{1.791551in}}%
\pgfpathlineto{\pgfqpoint{5.739768in}{1.894489in}}%
\pgfpathlineto{\pgfqpoint{5.748575in}{1.846450in}}%
\pgfpathlineto{\pgfqpoint{5.757382in}{1.544494in}}%
\pgfpathlineto{\pgfqpoint{5.766188in}{1.578816in}}%
\pgfpathlineto{\pgfqpoint{5.774995in}{1.585675in}}%
\pgfpathlineto{\pgfqpoint{5.783802in}{1.112138in}}%
\pgfpathlineto{\pgfqpoint{5.801416in}{1.661150in}}%
\pgfpathlineto{\pgfqpoint{5.810222in}{1.819015in}}%
\pgfpathlineto{\pgfqpoint{5.819029in}{2.244484in}}%
\pgfpathlineto{\pgfqpoint{5.827836in}{2.107252in}}%
\pgfpathlineto{\pgfqpoint{5.836643in}{2.031749in}}%
\pgfpathlineto{\pgfqpoint{5.845450in}{2.230767in}}%
\pgfpathlineto{\pgfqpoint{5.854257in}{2.326846in}}%
\pgfpathlineto{\pgfqpoint{5.863063in}{2.347422in}}%
\pgfpathlineto{\pgfqpoint{5.871870in}{2.052326in}}%
\pgfpathlineto{\pgfqpoint{5.889484in}{1.661150in}}%
\pgfpathlineto{\pgfqpoint{5.898291in}{1.496455in}}%
\pgfpathlineto{\pgfqpoint{5.907097in}{1.359195in}}%
\pgfpathlineto{\pgfqpoint{5.915904in}{1.736653in}}%
\pgfpathlineto{\pgfqpoint{5.924711in}{1.784692in}}%
\pgfpathlineto{\pgfqpoint{5.933518in}{1.969992in}}%
\pgfpathlineto{\pgfqpoint{5.942325in}{1.921953in}}%
\pgfpathlineto{\pgfqpoint{5.951132in}{2.093507in}}%
\pgfpathlineto{\pgfqpoint{5.959938in}{2.189586in}}%
\pgfpathlineto{\pgfqpoint{5.968745in}{1.901348in}}%
\pgfpathlineto{\pgfqpoint{5.977552in}{2.011173in}}%
\pgfpathlineto{\pgfqpoint{5.986359in}{1.928811in}}%
\pgfpathlineto{\pgfqpoint{5.995166in}{1.873913in}}%
\pgfpathlineto{\pgfqpoint{6.003972in}{1.983710in}}%
\pgfpathlineto{\pgfqpoint{6.012779in}{1.846450in}}%
\pgfpathlineto{\pgfqpoint{6.021586in}{2.292524in}}%
\pgfpathlineto{\pgfqpoint{6.030393in}{2.381744in}}%
\pgfpathlineto{\pgfqpoint{6.039200in}{2.484682in}}%
\pgfpathlineto{\pgfqpoint{6.048007in}{2.210190in}}%
\pgfpathlineto{\pgfqpoint{6.056813in}{2.217049in}}%
\pgfpathlineto{\pgfqpoint{6.065620in}{1.983710in}}%
\pgfpathlineto{\pgfqpoint{6.074427in}{1.805269in}}%
\pgfpathlineto{\pgfqpoint{6.083234in}{1.743512in}}%
\pgfpathlineto{\pgfqpoint{6.092041in}{2.011173in}}%
\pgfpathlineto{\pgfqpoint{6.100847in}{1.764116in}}%
\pgfpathlineto{\pgfqpoint{6.109654in}{1.674896in}}%
\pgfpathlineto{\pgfqpoint{6.118461in}{1.825873in}}%
\pgfpathlineto{\pgfqpoint{6.136075in}{0.981737in}}%
\pgfpathlineto{\pgfqpoint{6.144882in}{1.064098in}}%
\pgfpathlineto{\pgfqpoint{6.153688in}{1.324873in}}%
\pgfpathlineto{\pgfqpoint{6.162495in}{1.318014in}}%
\pgfpathlineto{\pgfqpoint{6.171302in}{1.798410in}}%
\pgfpathlineto{\pgfqpoint{6.180109in}{1.915094in}}%
\pgfpathlineto{\pgfqpoint{6.188916in}{1.674896in}}%
\pgfpathlineto{\pgfqpoint{6.197722in}{1.736653in}}%
\pgfpathlineto{\pgfqpoint{6.215336in}{2.292524in}}%
\pgfpathlineto{\pgfqpoint{6.224143in}{2.340563in}}%
\pgfpathlineto{\pgfqpoint{6.232950in}{2.086648in}}%
\pgfpathlineto{\pgfqpoint{6.241757in}{2.107252in}}%
\pgfpathlineto{\pgfqpoint{6.250563in}{2.326846in}}%
\pgfpathlineto{\pgfqpoint{6.259370in}{1.901348in}}%
\pgfpathlineto{\pgfqpoint{6.268177in}{1.928811in}}%
\pgfpathlineto{\pgfqpoint{6.276984in}{1.709190in}}%
\pgfpathlineto{\pgfqpoint{6.285791in}{1.819015in}}%
\pgfpathlineto{\pgfqpoint{6.294597in}{1.819015in}}%
\pgfpathlineto{\pgfqpoint{6.303404in}{1.626856in}}%
\pgfpathlineto{\pgfqpoint{6.312211in}{1.688613in}}%
\pgfpathlineto{\pgfqpoint{6.321018in}{2.182727in}}%
\pgfpathlineto{\pgfqpoint{6.329825in}{2.086648in}}%
\pgfpathlineto{\pgfqpoint{6.338632in}{2.107252in}}%
\pgfpathlineto{\pgfqpoint{6.347438in}{2.381744in}}%
\pgfpathlineto{\pgfqpoint{6.356245in}{2.271947in}}%
\pgfpathlineto{\pgfqpoint{6.365052in}{2.271947in}}%
\pgfpathlineto{\pgfqpoint{6.373859in}{1.969992in}}%
\pgfpathlineto{\pgfqpoint{6.382666in}{2.114111in}}%
\pgfpathlineto{\pgfqpoint{6.391472in}{2.052326in}}%
\pgfpathlineto{\pgfqpoint{6.400279in}{2.333705in}}%
\pgfpathlineto{\pgfqpoint{6.409086in}{2.031749in}}%
\pgfpathlineto{\pgfqpoint{6.417893in}{2.148405in}}%
\pgfpathlineto{\pgfqpoint{6.426700in}{2.347422in}}%
\pgfpathlineto{\pgfqpoint{6.435507in}{1.832732in}}%
\pgfpathlineto{\pgfqpoint{6.444313in}{1.825873in}}%
\pgfpathlineto{\pgfqpoint{6.453120in}{1.942529in}}%
\pgfpathlineto{\pgfqpoint{6.461927in}{1.770975in}}%
\pgfpathlineto{\pgfqpoint{6.470734in}{1.908207in}}%
\pgfpathlineto{\pgfqpoint{6.479541in}{1.956246in}}%
\pgfpathlineto{\pgfqpoint{6.488347in}{2.203303in}}%
\pgfpathlineto{\pgfqpoint{6.497154in}{2.237625in}}%
\pgfpathlineto{\pgfqpoint{6.505961in}{2.313128in}}%
\pgfpathlineto{\pgfqpoint{6.514768in}{1.805269in}}%
\pgfpathlineto{\pgfqpoint{6.523575in}{1.949388in}}%
\pgfpathlineto{\pgfqpoint{6.532382in}{1.798410in}}%
\pgfpathlineto{\pgfqpoint{6.541188in}{1.867054in}}%
\pgfpathlineto{\pgfqpoint{6.549995in}{2.319987in}}%
\pgfpathlineto{\pgfqpoint{6.558802in}{2.340563in}}%
\pgfpathlineto{\pgfqpoint{6.567609in}{2.374885in}}%
\pgfpathlineto{\pgfqpoint{6.576416in}{2.381744in}}%
\pgfpathlineto{\pgfqpoint{6.585222in}{2.354309in}}%
\pgfpathlineto{\pgfqpoint{6.594029in}{2.347422in}}%
\pgfpathlineto{\pgfqpoint{6.602836in}{1.853308in}}%
\pgfpathlineto{\pgfqpoint{6.611643in}{1.894489in}}%
\pgfpathlineto{\pgfqpoint{6.620450in}{1.743512in}}%
\pgfpathlineto{\pgfqpoint{6.629257in}{1.441556in}}%
\pgfpathlineto{\pgfqpoint{6.638063in}{1.393517in}}%
\pgfpathlineto{\pgfqpoint{6.646870in}{1.626856in}}%
\pgfpathlineto{\pgfqpoint{6.655677in}{2.093507in}}%
\pgfpathlineto{\pgfqpoint{6.664484in}{2.258230in}}%
\pgfpathlineto{\pgfqpoint{6.673291in}{1.997427in}}%
\pgfpathlineto{\pgfqpoint{6.682097in}{1.681754in}}%
\pgfpathlineto{\pgfqpoint{6.690904in}{1.619997in}}%
\pgfpathlineto{\pgfqpoint{6.699711in}{1.805269in}}%
\pgfpathlineto{\pgfqpoint{6.708518in}{1.819015in}}%
\pgfpathlineto{\pgfqpoint{6.717325in}{1.805269in}}%
\pgfpathlineto{\pgfqpoint{6.726132in}{1.736653in}}%
\pgfpathlineto{\pgfqpoint{6.734938in}{1.880772in}}%
\pgfpathlineto{\pgfqpoint{6.743745in}{1.963133in}}%
\pgfpathlineto{\pgfqpoint{6.752552in}{2.162151in}}%
\pgfpathlineto{\pgfqpoint{6.761359in}{1.867054in}}%
\pgfpathlineto{\pgfqpoint{6.770166in}{1.791551in}}%
\pgfpathlineto{\pgfqpoint{6.778972in}{1.976851in}}%
\pgfpathlineto{\pgfqpoint{6.787779in}{1.969992in}}%
\pgfpathlineto{\pgfqpoint{6.796586in}{2.004286in}}%
\pgfpathlineto{\pgfqpoint{6.805393in}{2.018032in}}%
\pgfpathlineto{\pgfqpoint{6.814200in}{2.182727in}}%
\pgfpathlineto{\pgfqpoint{6.823007in}{2.127829in}}%
\pgfpathlineto{\pgfqpoint{6.831813in}{2.011173in}}%
\pgfpathlineto{\pgfqpoint{6.840620in}{1.983710in}}%
\pgfpathlineto{\pgfqpoint{6.849427in}{2.409207in}}%
\pgfpathlineto{\pgfqpoint{6.858234in}{2.340563in}}%
\pgfpathlineto{\pgfqpoint{6.884654in}{1.523918in}}%
\pgfpathlineto{\pgfqpoint{6.893461in}{1.777834in}}%
\pgfpathlineto{\pgfqpoint{6.902268in}{1.928811in}}%
\pgfpathlineto{\pgfqpoint{6.911075in}{1.729794in}}%
\pgfpathlineto{\pgfqpoint{6.919882in}{1.503314in}}%
\pgfpathlineto{\pgfqpoint{6.928688in}{1.510172in}}%
\pgfpathlineto{\pgfqpoint{6.937495in}{1.530777in}}%
\pgfpathlineto{\pgfqpoint{6.946302in}{1.695472in}}%
\pgfpathlineto{\pgfqpoint{6.955109in}{1.887630in}}%
\pgfpathlineto{\pgfqpoint{6.963916in}{1.784692in}}%
\pgfpathlineto{\pgfqpoint{6.972722in}{1.956246in}}%
\pgfpathlineto{\pgfqpoint{6.981529in}{1.331760in}}%
\pgfpathlineto{\pgfqpoint{6.990336in}{1.242539in}}%
\pgfpathlineto{\pgfqpoint{6.999143in}{1.427839in}}%
\pgfpathlineto{\pgfqpoint{7.007950in}{1.709190in}}%
\pgfpathlineto{\pgfqpoint{7.016757in}{1.468992in}}%
\pgfpathlineto{\pgfqpoint{7.025563in}{1.571958in}}%
\pgfpathlineto{\pgfqpoint{7.034370in}{1.585675in}}%
\pgfpathlineto{\pgfqpoint{7.043177in}{1.585675in}}%
\pgfpathlineto{\pgfqpoint{7.051984in}{1.716076in}}%
\pgfpathlineto{\pgfqpoint{7.060791in}{1.565071in}}%
\pgfpathlineto{\pgfqpoint{7.069597in}{1.839591in}}%
\pgfpathlineto{\pgfqpoint{7.078404in}{1.942529in}}%
\pgfpathlineto{\pgfqpoint{7.087211in}{1.983710in}}%
\pgfpathlineto{\pgfqpoint{7.096018in}{1.990569in}}%
\pgfpathlineto{\pgfqpoint{7.104825in}{2.100365in}}%
\pgfpathlineto{\pgfqpoint{7.113632in}{2.162151in}}%
\pgfpathlineto{\pgfqpoint{7.131245in}{2.011173in}}%
\pgfpathlineto{\pgfqpoint{7.140052in}{2.066071in}}%
\pgfpathlineto{\pgfqpoint{7.148859in}{1.976851in}}%
\pgfpathlineto{\pgfqpoint{7.157666in}{2.072930in}}%
\pgfpathlineto{\pgfqpoint{7.166472in}{1.963133in}}%
\pgfpathlineto{\pgfqpoint{7.184086in}{1.606252in}}%
\pgfpathlineto{\pgfqpoint{7.192893in}{1.407234in}}%
\pgfpathlineto{\pgfqpoint{7.201700in}{1.544494in}}%
\pgfpathlineto{\pgfqpoint{7.210507in}{2.120970in}}%
\pgfpathlineto{\pgfqpoint{7.219313in}{2.223908in}}%
\pgfpathlineto{\pgfqpoint{7.228120in}{1.969992in}}%
\pgfpathlineto{\pgfqpoint{7.236927in}{1.812128in}}%
\pgfpathlineto{\pgfqpoint{7.245734in}{1.825873in}}%
\pgfpathlineto{\pgfqpoint{7.254541in}{1.544494in}}%
\pgfpathlineto{\pgfqpoint{7.263347in}{1.647432in}}%
\pgfpathlineto{\pgfqpoint{7.272154in}{1.867054in}}%
\pgfpathlineto{\pgfqpoint{7.280961in}{1.949388in}}%
\pgfpathlineto{\pgfqpoint{7.289768in}{2.018032in}}%
\pgfpathlineto{\pgfqpoint{7.298575in}{2.326846in}}%
\pgfpathlineto{\pgfqpoint{7.307382in}{2.203303in}}%
\pgfpathlineto{\pgfqpoint{7.316188in}{2.210190in}}%
\pgfpathlineto{\pgfqpoint{7.324995in}{1.908207in}}%
\pgfpathlineto{\pgfqpoint{7.333802in}{1.832732in}}%
\pgfpathlineto{\pgfqpoint{7.342609in}{2.127829in}}%
\pgfpathlineto{\pgfqpoint{7.351416in}{2.018032in}}%
\pgfpathlineto{\pgfqpoint{7.360222in}{2.066071in}}%
\pgfpathlineto{\pgfqpoint{7.369029in}{1.928811in}}%
\pgfpathlineto{\pgfqpoint{7.377836in}{1.839591in}}%
\pgfpathlineto{\pgfqpoint{7.386643in}{1.668037in}}%
\pgfpathlineto{\pgfqpoint{7.395450in}{1.640574in}}%
\pgfpathlineto{\pgfqpoint{7.404257in}{1.619997in}}%
\pgfpathlineto{\pgfqpoint{7.413063in}{1.537636in}}%
\pgfpathlineto{\pgfqpoint{7.421870in}{1.750370in}}%
\pgfpathlineto{\pgfqpoint{7.430677in}{1.640574in}}%
\pgfpathlineto{\pgfqpoint{7.439484in}{1.894489in}}%
\pgfpathlineto{\pgfqpoint{7.448291in}{2.018032in}}%
\pgfpathlineto{\pgfqpoint{7.457097in}{1.729794in}}%
\pgfpathlineto{\pgfqpoint{7.465904in}{1.798410in}}%
\pgfpathlineto{\pgfqpoint{7.474711in}{1.750370in}}%
\pgfpathlineto{\pgfqpoint{7.492325in}{1.867054in}}%
\pgfpathlineto{\pgfqpoint{7.501132in}{1.846450in}}%
\pgfpathlineto{\pgfqpoint{7.509938in}{1.585675in}}%
\pgfpathlineto{\pgfqpoint{7.518745in}{1.640574in}}%
\pgfpathlineto{\pgfqpoint{7.527552in}{1.633715in}}%
\pgfpathlineto{\pgfqpoint{7.536359in}{1.661150in}}%
\pgfpathlineto{\pgfqpoint{7.545166in}{1.784692in}}%
\pgfpathlineto{\pgfqpoint{7.553972in}{1.942529in}}%
\pgfpathlineto{\pgfqpoint{7.562779in}{2.134687in}}%
\pgfpathlineto{\pgfqpoint{7.571586in}{1.983710in}}%
\pgfpathlineto{\pgfqpoint{7.580393in}{2.107252in}}%
\pgfpathlineto{\pgfqpoint{7.589200in}{1.640574in}}%
\pgfpathlineto{\pgfqpoint{7.598007in}{1.722935in}}%
\pgfpathlineto{\pgfqpoint{7.606813in}{1.613110in}}%
\pgfpathlineto{\pgfqpoint{7.615620in}{1.784692in}}%
\pgfpathlineto{\pgfqpoint{7.624427in}{2.086648in}}%
\pgfpathlineto{\pgfqpoint{7.642041in}{1.709190in}}%
\pgfpathlineto{\pgfqpoint{7.650847in}{1.894489in}}%
\pgfpathlineto{\pgfqpoint{7.659654in}{1.674896in}}%
\pgfpathlineto{\pgfqpoint{7.668461in}{1.743512in}}%
\pgfpathlineto{\pgfqpoint{7.677268in}{1.969992in}}%
\pgfpathlineto{\pgfqpoint{7.686075in}{1.736653in}}%
\pgfpathlineto{\pgfqpoint{7.694882in}{1.764116in}}%
\pgfpathlineto{\pgfqpoint{7.703688in}{1.853308in}}%
\pgfpathlineto{\pgfqpoint{7.712495in}{1.709190in}}%
\pgfpathlineto{\pgfqpoint{7.721302in}{2.127829in}}%
\pgfpathlineto{\pgfqpoint{7.730109in}{2.210190in}}%
\pgfpathlineto{\pgfqpoint{7.738916in}{2.196445in}}%
\pgfpathlineto{\pgfqpoint{7.747722in}{2.560185in}}%
\pgfpathlineto{\pgfqpoint{7.756529in}{2.745485in}}%
\pgfpathlineto{\pgfqpoint{7.765336in}{2.560185in}}%
\pgfpathlineto{\pgfqpoint{7.774143in}{2.223908in}}%
\pgfpathlineto{\pgfqpoint{7.782950in}{2.120970in}}%
\pgfpathlineto{\pgfqpoint{7.791757in}{2.066071in}}%
\pgfpathlineto{\pgfqpoint{7.800563in}{2.436643in}}%
\pgfpathlineto{\pgfqpoint{7.809370in}{2.649406in}}%
\pgfpathlineto{\pgfqpoint{7.818177in}{2.395462in}}%
\pgfpathlineto{\pgfqpoint{7.826984in}{2.512146in}}%
\pgfpathlineto{\pgfqpoint{7.835791in}{2.374885in}}%
\pgfpathlineto{\pgfqpoint{7.844597in}{2.120970in}}%
\pgfpathlineto{\pgfqpoint{7.862211in}{1.750370in}}%
\pgfpathlineto{\pgfqpoint{7.871018in}{2.004286in}}%
\pgfpathlineto{\pgfqpoint{7.879825in}{1.750370in}}%
\pgfpathlineto{\pgfqpoint{7.888632in}{1.764116in}}%
\pgfpathlineto{\pgfqpoint{7.897438in}{1.880772in}}%
\pgfpathlineto{\pgfqpoint{7.906245in}{1.592534in}}%
\pgfpathlineto{\pgfqpoint{7.923859in}{1.290579in}}%
\pgfpathlineto{\pgfqpoint{7.932666in}{1.668037in}}%
\pgfpathlineto{\pgfqpoint{7.941472in}{1.880772in}}%
\pgfpathlineto{\pgfqpoint{7.950279in}{2.031749in}}%
\pgfpathlineto{\pgfqpoint{7.959086in}{1.839591in}}%
\pgfpathlineto{\pgfqpoint{7.967893in}{1.736653in}}%
\pgfpathlineto{\pgfqpoint{7.976700in}{2.079789in}}%
\pgfpathlineto{\pgfqpoint{7.985507in}{2.368027in}}%
\pgfpathlineto{\pgfqpoint{7.994313in}{2.086648in}}%
\pgfpathlineto{\pgfqpoint{8.003120in}{2.416066in}}%
\pgfpathlineto{\pgfqpoint{8.011927in}{2.615084in}}%
\pgfpathlineto{\pgfqpoint{8.038347in}{2.079789in}}%
\pgfpathlineto{\pgfqpoint{8.047154in}{2.114111in}}%
\pgfpathlineto{\pgfqpoint{8.055961in}{2.024891in}}%
\pgfpathlineto{\pgfqpoint{8.064768in}{2.093507in}}%
\pgfpathlineto{\pgfqpoint{8.082382in}{1.599393in}}%
\pgfpathlineto{\pgfqpoint{8.091188in}{1.482737in}}%
\pgfpathlineto{\pgfqpoint{8.099995in}{1.640574in}}%
\pgfpathlineto{\pgfqpoint{8.108802in}{1.736653in}}%
\pgfpathlineto{\pgfqpoint{8.117609in}{1.674896in}}%
\pgfpathlineto{\pgfqpoint{8.126416in}{1.709190in}}%
\pgfpathlineto{\pgfqpoint{8.135222in}{1.372912in}}%
\pgfpathlineto{\pgfqpoint{8.144029in}{1.468992in}}%
\pgfpathlineto{\pgfqpoint{8.152836in}{1.386658in}}%
\pgfpathlineto{\pgfqpoint{8.161643in}{1.839591in}}%
\pgfpathlineto{\pgfqpoint{8.170450in}{1.832732in}}%
\pgfpathlineto{\pgfqpoint{8.179257in}{1.949388in}}%
\pgfpathlineto{\pgfqpoint{8.188063in}{2.182727in}}%
\pgfpathlineto{\pgfqpoint{8.196870in}{2.107252in}}%
\pgfpathlineto{\pgfqpoint{8.205677in}{2.066071in}}%
\pgfpathlineto{\pgfqpoint{8.214484in}{2.361168in}}%
\pgfpathlineto{\pgfqpoint{8.223291in}{2.031749in}}%
\pgfpathlineto{\pgfqpoint{8.232097in}{1.839591in}}%
\pgfpathlineto{\pgfqpoint{8.249711in}{1.736653in}}%
\pgfpathlineto{\pgfqpoint{8.258518in}{1.825873in}}%
\pgfpathlineto{\pgfqpoint{8.267325in}{1.873913in}}%
\pgfpathlineto{\pgfqpoint{8.276132in}{1.819015in}}%
\pgfpathlineto{\pgfqpoint{8.284938in}{1.805269in}}%
\pgfpathlineto{\pgfqpoint{8.293745in}{1.743512in}}%
\pgfpathlineto{\pgfqpoint{8.302552in}{1.462133in}}%
\pgfpathlineto{\pgfqpoint{8.311359in}{1.578816in}}%
\pgfpathlineto{\pgfqpoint{8.320166in}{1.729794in}}%
\pgfpathlineto{\pgfqpoint{8.337779in}{2.285665in}}%
\pgfpathlineto{\pgfqpoint{8.346586in}{2.210190in}}%
\pgfpathlineto{\pgfqpoint{8.355393in}{1.887630in}}%
\pgfpathlineto{\pgfqpoint{8.364200in}{1.928811in}}%
\pgfpathlineto{\pgfqpoint{8.373007in}{1.942529in}}%
\pgfpathlineto{\pgfqpoint{8.381813in}{1.729794in}}%
\pgfpathlineto{\pgfqpoint{8.390620in}{2.210190in}}%
\pgfpathlineto{\pgfqpoint{8.399427in}{2.141546in}}%
\pgfpathlineto{\pgfqpoint{8.408234in}{1.887630in}}%
\pgfpathlineto{\pgfqpoint{8.417041in}{1.592534in}}%
\pgfpathlineto{\pgfqpoint{8.425847in}{1.366053in}}%
\pgfpathlineto{\pgfqpoint{8.434654in}{1.386658in}}%
\pgfpathlineto{\pgfqpoint{8.443461in}{1.105279in}}%
\pgfpathlineto{\pgfqpoint{8.452268in}{1.070957in}}%
\pgfpathlineto{\pgfqpoint{8.461075in}{1.208217in}}%
\pgfpathlineto{\pgfqpoint{8.469882in}{1.366053in}}%
\pgfpathlineto{\pgfqpoint{8.478688in}{1.626856in}}%
\pgfpathlineto{\pgfqpoint{8.487495in}{1.722935in}}%
\pgfpathlineto{\pgfqpoint{8.496302in}{2.004286in}}%
\pgfpathlineto{\pgfqpoint{8.505109in}{2.141546in}}%
\pgfpathlineto{\pgfqpoint{8.513916in}{1.935670in}}%
\pgfpathlineto{\pgfqpoint{8.522722in}{1.969992in}}%
\pgfpathlineto{\pgfqpoint{8.531529in}{1.688613in}}%
\pgfpathlineto{\pgfqpoint{8.540336in}{1.976851in}}%
\pgfpathlineto{\pgfqpoint{8.549143in}{1.695472in}}%
\pgfpathlineto{\pgfqpoint{8.557950in}{1.647432in}}%
\pgfpathlineto{\pgfqpoint{8.566757in}{2.024891in}}%
\pgfpathlineto{\pgfqpoint{8.575563in}{1.997427in}}%
\pgfpathlineto{\pgfqpoint{8.584370in}{2.004286in}}%
\pgfpathlineto{\pgfqpoint{8.593177in}{2.162151in}}%
\pgfpathlineto{\pgfqpoint{8.601984in}{2.182727in}}%
\pgfpathlineto{\pgfqpoint{8.610791in}{2.292524in}}%
\pgfpathlineto{\pgfqpoint{8.619597in}{2.285665in}}%
\pgfpathlineto{\pgfqpoint{8.628404in}{2.251343in}}%
\pgfpathlineto{\pgfqpoint{8.637211in}{2.601366in}}%
\pgfpathlineto{\pgfqpoint{8.646018in}{2.395462in}}%
\pgfpathlineto{\pgfqpoint{8.654825in}{2.141546in}}%
\pgfpathlineto{\pgfqpoint{8.663632in}{2.059213in}}%
\pgfpathlineto{\pgfqpoint{8.672438in}{2.086648in}}%
\pgfpathlineto{\pgfqpoint{8.681245in}{2.031749in}}%
\pgfpathlineto{\pgfqpoint{8.690052in}{2.319987in}}%
\pgfpathlineto{\pgfqpoint{8.698859in}{2.086648in}}%
\pgfpathlineto{\pgfqpoint{8.707666in}{1.963133in}}%
\pgfpathlineto{\pgfqpoint{8.716472in}{2.210190in}}%
\pgfpathlineto{\pgfqpoint{8.725279in}{2.532722in}}%
\pgfpathlineto{\pgfqpoint{8.734086in}{2.594507in}}%
\pgfpathlineto{\pgfqpoint{8.742893in}{2.354309in}}%
\pgfpathlineto{\pgfqpoint{8.751700in}{2.319987in}}%
\pgfpathlineto{\pgfqpoint{8.760507in}{2.162151in}}%
\pgfpathlineto{\pgfqpoint{8.769313in}{2.457247in}}%
\pgfpathlineto{\pgfqpoint{8.778120in}{2.429784in}}%
\pgfpathlineto{\pgfqpoint{8.786927in}{2.175868in}}%
\pgfpathlineto{\pgfqpoint{8.795734in}{2.107252in}}%
\pgfpathlineto{\pgfqpoint{8.804541in}{2.292524in}}%
\pgfpathlineto{\pgfqpoint{8.813347in}{2.429784in}}%
\pgfpathlineto{\pgfqpoint{8.822154in}{2.265089in}}%
\pgfpathlineto{\pgfqpoint{8.830961in}{2.313128in}}%
\pgfpathlineto{\pgfqpoint{8.839768in}{2.306269in}}%
\pgfpathlineto{\pgfqpoint{8.848575in}{2.292524in}}%
\pgfpathlineto{\pgfqpoint{8.866188in}{1.969992in}}%
\pgfpathlineto{\pgfqpoint{8.874995in}{2.127829in}}%
\pgfpathlineto{\pgfqpoint{8.883802in}{2.182727in}}%
\pgfpathlineto{\pgfqpoint{8.892609in}{2.285665in}}%
\pgfpathlineto{\pgfqpoint{8.901416in}{2.148405in}}%
\pgfpathlineto{\pgfqpoint{8.910222in}{1.825873in}}%
\pgfpathlineto{\pgfqpoint{8.919029in}{1.819015in}}%
\pgfpathlineto{\pgfqpoint{8.927836in}{2.162151in}}%
\pgfpathlineto{\pgfqpoint{8.936643in}{1.894489in}}%
\pgfpathlineto{\pgfqpoint{8.945450in}{1.901348in}}%
\pgfpathlineto{\pgfqpoint{8.954257in}{1.846450in}}%
\pgfpathlineto{\pgfqpoint{8.963063in}{1.990569in}}%
\pgfpathlineto{\pgfqpoint{8.971870in}{2.292524in}}%
\pgfpathlineto{\pgfqpoint{8.980677in}{2.306269in}}%
\pgfpathlineto{\pgfqpoint{8.989484in}{2.443501in}}%
\pgfpathlineto{\pgfqpoint{8.998291in}{2.381744in}}%
\pgfpathlineto{\pgfqpoint{9.007097in}{2.114111in}}%
\pgfpathlineto{\pgfqpoint{9.015904in}{2.450388in}}%
\pgfpathlineto{\pgfqpoint{9.024711in}{2.340563in}}%
\pgfpathlineto{\pgfqpoint{9.033518in}{2.265089in}}%
\pgfpathlineto{\pgfqpoint{9.042325in}{2.203303in}}%
\pgfpathlineto{\pgfqpoint{9.051132in}{2.457247in}}%
\pgfpathlineto{\pgfqpoint{9.059938in}{2.086648in}}%
\pgfpathlineto{\pgfqpoint{9.068745in}{1.791551in}}%
\pgfpathlineto{\pgfqpoint{9.077552in}{1.867054in}}%
\pgfpathlineto{\pgfqpoint{9.086359in}{1.647432in}}%
\pgfpathlineto{\pgfqpoint{9.095166in}{2.244484in}}%
\pgfpathlineto{\pgfqpoint{9.103972in}{2.278806in}}%
\pgfpathlineto{\pgfqpoint{9.112779in}{2.223908in}}%
\pgfpathlineto{\pgfqpoint{9.121586in}{2.292524in}}%
\pgfpathlineto{\pgfqpoint{9.130393in}{2.223908in}}%
\pgfpathlineto{\pgfqpoint{9.139200in}{2.374885in}}%
\pgfpathlineto{\pgfqpoint{9.148007in}{2.457247in}}%
\pgfpathlineto{\pgfqpoint{9.156813in}{2.203303in}}%
\pgfpathlineto{\pgfqpoint{9.165620in}{1.558212in}}%
\pgfpathlineto{\pgfqpoint{9.174427in}{1.407234in}}%
\pgfpathlineto{\pgfqpoint{9.183234in}{1.558212in}}%
\pgfpathlineto{\pgfqpoint{9.192041in}{1.414093in}}%
\pgfpathlineto{\pgfqpoint{9.200847in}{1.819015in}}%
\pgfpathlineto{\pgfqpoint{9.209654in}{2.018032in}}%
\pgfpathlineto{\pgfqpoint{9.218461in}{1.963133in}}%
\pgfpathlineto{\pgfqpoint{9.227268in}{1.983710in}}%
\pgfpathlineto{\pgfqpoint{9.236075in}{1.716076in}}%
\pgfpathlineto{\pgfqpoint{9.244882in}{1.798410in}}%
\pgfpathlineto{\pgfqpoint{9.253688in}{1.743512in}}%
\pgfpathlineto{\pgfqpoint{9.262495in}{1.613110in}}%
\pgfpathlineto{\pgfqpoint{9.271302in}{1.654291in}}%
\pgfpathlineto{\pgfqpoint{9.280109in}{2.072930in}}%
\pgfpathlineto{\pgfqpoint{9.288916in}{2.011173in}}%
\pgfpathlineto{\pgfqpoint{9.297722in}{2.422925in}}%
\pgfpathlineto{\pgfqpoint{9.306529in}{2.333705in}}%
\pgfpathlineto{\pgfqpoint{9.315336in}{2.381744in}}%
\pgfpathlineto{\pgfqpoint{9.324143in}{2.477823in}}%
\pgfpathlineto{\pgfqpoint{9.332950in}{2.381744in}}%
\pgfpathlineto{\pgfqpoint{9.341757in}{2.347422in}}%
\pgfpathlineto{\pgfqpoint{9.350563in}{2.422925in}}%
\pgfpathlineto{\pgfqpoint{9.359370in}{2.340563in}}%
\pgfpathlineto{\pgfqpoint{9.368177in}{2.134687in}}%
\pgfpathlineto{\pgfqpoint{9.376984in}{2.169009in}}%
\pgfpathlineto{\pgfqpoint{9.385791in}{2.306269in}}%
\pgfpathlineto{\pgfqpoint{9.394597in}{1.709190in}}%
\pgfpathlineto{\pgfqpoint{9.403404in}{1.894489in}}%
\pgfpathlineto{\pgfqpoint{9.412211in}{1.366053in}}%
\pgfpathlineto{\pgfqpoint{9.429825in}{1.791551in}}%
\pgfpathlineto{\pgfqpoint{9.438632in}{1.949388in}}%
\pgfpathlineto{\pgfqpoint{9.447438in}{1.873913in}}%
\pgfpathlineto{\pgfqpoint{9.456245in}{1.949388in}}%
\pgfpathlineto{\pgfqpoint{9.465052in}{2.196445in}}%
\pgfpathlineto{\pgfqpoint{9.473859in}{2.313128in}}%
\pgfpathlineto{\pgfqpoint{9.482666in}{2.313128in}}%
\pgfpathlineto{\pgfqpoint{9.491472in}{2.340563in}}%
\pgfpathlineto{\pgfqpoint{9.500279in}{2.148405in}}%
\pgfpathlineto{\pgfqpoint{9.509086in}{2.141546in}}%
\pgfpathlineto{\pgfqpoint{9.517893in}{1.963133in}}%
\pgfpathlineto{\pgfqpoint{9.526700in}{1.722935in}}%
\pgfpathlineto{\pgfqpoint{9.535507in}{1.530777in}}%
\pgfpathlineto{\pgfqpoint{9.544313in}{1.668037in}}%
\pgfpathlineto{\pgfqpoint{9.553120in}{1.599393in}}%
\pgfpathlineto{\pgfqpoint{9.561927in}{1.722935in}}%
\pgfpathlineto{\pgfqpoint{9.570734in}{1.935670in}}%
\pgfpathlineto{\pgfqpoint{9.579541in}{1.867054in}}%
\pgfpathlineto{\pgfqpoint{9.588347in}{1.681754in}}%
\pgfpathlineto{\pgfqpoint{9.597154in}{1.777834in}}%
\pgfpathlineto{\pgfqpoint{9.605961in}{1.482737in}}%
\pgfpathlineto{\pgfqpoint{9.614768in}{1.640574in}}%
\pgfpathlineto{\pgfqpoint{9.623575in}{1.640574in}}%
\pgfpathlineto{\pgfqpoint{9.632382in}{1.578816in}}%
\pgfpathlineto{\pgfqpoint{9.641188in}{1.578816in}}%
\pgfpathlineto{\pgfqpoint{9.649995in}{2.018032in}}%
\pgfpathlineto{\pgfqpoint{9.658802in}{2.100365in}}%
\pgfpathlineto{\pgfqpoint{9.667609in}{1.969992in}}%
\pgfpathlineto{\pgfqpoint{9.676416in}{2.148405in}}%
\pgfpathlineto{\pgfqpoint{9.685222in}{2.422925in}}%
\pgfpathlineto{\pgfqpoint{9.694029in}{2.285665in}}%
\pgfpathlineto{\pgfqpoint{9.702836in}{2.271947in}}%
\pgfpathlineto{\pgfqpoint{9.711643in}{1.887630in}}%
\pgfpathlineto{\pgfqpoint{9.720450in}{1.626856in}}%
\pgfpathlineto{\pgfqpoint{9.729257in}{1.805269in}}%
\pgfpathlineto{\pgfqpoint{9.738063in}{1.915094in}}%
\pgfpathlineto{\pgfqpoint{9.746870in}{2.004286in}}%
\pgfpathlineto{\pgfqpoint{9.755677in}{2.120970in}}%
\pgfpathlineto{\pgfqpoint{9.764484in}{2.141546in}}%
\pgfpathlineto{\pgfqpoint{9.773291in}{2.107252in}}%
\pgfpathlineto{\pgfqpoint{9.782097in}{2.395462in}}%
\pgfpathlineto{\pgfqpoint{9.790904in}{2.162151in}}%
\pgfpathlineto{\pgfqpoint{9.799711in}{2.333705in}}%
\pgfpathlineto{\pgfqpoint{9.808518in}{2.402349in}}%
\pgfpathlineto{\pgfqpoint{9.817325in}{2.388603in}}%
\pgfpathlineto{\pgfqpoint{9.826132in}{2.313128in}}%
\pgfpathlineto{\pgfqpoint{9.834938in}{2.203303in}}%
\pgfpathlineto{\pgfqpoint{9.843745in}{1.887630in}}%
\pgfpathlineto{\pgfqpoint{9.852552in}{1.825873in}}%
\pgfpathlineto{\pgfqpoint{9.861359in}{2.018032in}}%
\pgfpathlineto{\pgfqpoint{9.870166in}{2.120970in}}%
\pgfpathlineto{\pgfqpoint{9.878972in}{2.203303in}}%
\pgfpathlineto{\pgfqpoint{9.887779in}{1.935670in}}%
\pgfpathlineto{\pgfqpoint{9.896586in}{1.784692in}}%
\pgfpathlineto{\pgfqpoint{9.905393in}{1.915094in}}%
\pgfpathlineto{\pgfqpoint{9.914200in}{1.764116in}}%
\pgfpathlineto{\pgfqpoint{9.923007in}{1.846450in}}%
\pgfpathlineto{\pgfqpoint{9.931813in}{1.949388in}}%
\pgfpathlineto{\pgfqpoint{9.940620in}{2.024891in}}%
\pgfpathlineto{\pgfqpoint{9.949427in}{2.313128in}}%
\pgfpathlineto{\pgfqpoint{9.949427in}{2.313128in}}%
\pgfusepath{stroke}%
\end{pgfscope}%
\begin{pgfscope}%
\pgfpathrectangle{\pgfqpoint{0.702268in}{0.521603in}}{\pgfqpoint{9.687500in}{4.235000in}}%
\pgfusepath{clip}%
\pgfsetrectcap%
\pgfsetroundjoin%
\pgfsetlinewidth{0.501875pt}%
\definecolor{currentstroke}{rgb}{0.501961,0.501961,0.501961}%
\pgfsetstrokecolor{currentstroke}%
\pgfsetstrokeopacity{0.250000}%
\pgfsetdash{}{0pt}%
\pgfpathmoveto{\pgfqpoint{1.142609in}{4.317046in}}%
\pgfpathlineto{\pgfqpoint{1.151416in}{3.884690in}}%
\pgfpathlineto{\pgfqpoint{1.160222in}{3.129774in}}%
\pgfpathlineto{\pgfqpoint{1.169029in}{2.800383in}}%
\pgfpathlineto{\pgfqpoint{1.177836in}{2.697445in}}%
\pgfpathlineto{\pgfqpoint{1.186643in}{2.333705in}}%
\pgfpathlineto{\pgfqpoint{1.195450in}{1.784692in}}%
\pgfpathlineto{\pgfqpoint{1.204257in}{1.702331in}}%
\pgfpathlineto{\pgfqpoint{1.213063in}{1.846450in}}%
\pgfpathlineto{\pgfqpoint{1.221870in}{1.867054in}}%
\pgfpathlineto{\pgfqpoint{1.230677in}{1.860167in}}%
\pgfpathlineto{\pgfqpoint{1.239484in}{1.812128in}}%
\pgfpathlineto{\pgfqpoint{1.257097in}{1.571958in}}%
\pgfpathlineto{\pgfqpoint{1.265904in}{1.777834in}}%
\pgfpathlineto{\pgfqpoint{1.274711in}{1.681754in}}%
\pgfpathlineto{\pgfqpoint{1.283518in}{1.565071in}}%
\pgfpathlineto{\pgfqpoint{1.292325in}{1.475878in}}%
\pgfpathlineto{\pgfqpoint{1.301132in}{1.228822in}}%
\pgfpathlineto{\pgfqpoint{1.309938in}{1.311155in}}%
\pgfpathlineto{\pgfqpoint{1.318745in}{1.517031in}}%
\pgfpathlineto{\pgfqpoint{1.327552in}{1.283720in}}%
\pgfpathlineto{\pgfqpoint{1.336359in}{1.462133in}}%
\pgfpathlineto{\pgfqpoint{1.345166in}{1.784692in}}%
\pgfpathlineto{\pgfqpoint{1.353972in}{1.867054in}}%
\pgfpathlineto{\pgfqpoint{1.362779in}{1.695472in}}%
\pgfpathlineto{\pgfqpoint{1.371586in}{1.661150in}}%
\pgfpathlineto{\pgfqpoint{1.380393in}{1.880772in}}%
\pgfpathlineto{\pgfqpoint{1.389200in}{1.757229in}}%
\pgfpathlineto{\pgfqpoint{1.398007in}{1.990569in}}%
\pgfpathlineto{\pgfqpoint{1.406813in}{2.059213in}}%
\pgfpathlineto{\pgfqpoint{1.415620in}{1.976851in}}%
\pgfpathlineto{\pgfqpoint{1.424427in}{1.880772in}}%
\pgfpathlineto{\pgfqpoint{1.433234in}{1.716076in}}%
\pgfpathlineto{\pgfqpoint{1.442041in}{1.949388in}}%
\pgfpathlineto{\pgfqpoint{1.450847in}{2.148405in}}%
\pgfpathlineto{\pgfqpoint{1.459654in}{2.038608in}}%
\pgfpathlineto{\pgfqpoint{1.468461in}{1.956246in}}%
\pgfpathlineto{\pgfqpoint{1.486075in}{2.182727in}}%
\pgfpathlineto{\pgfqpoint{1.494882in}{1.997427in}}%
\pgfpathlineto{\pgfqpoint{1.503688in}{2.004286in}}%
\pgfpathlineto{\pgfqpoint{1.512495in}{1.915094in}}%
\pgfpathlineto{\pgfqpoint{1.521302in}{2.093507in}}%
\pgfpathlineto{\pgfqpoint{1.530109in}{2.162151in}}%
\pgfpathlineto{\pgfqpoint{1.538916in}{2.429784in}}%
\pgfpathlineto{\pgfqpoint{1.547722in}{2.127829in}}%
\pgfpathlineto{\pgfqpoint{1.556529in}{2.086648in}}%
\pgfpathlineto{\pgfqpoint{1.565336in}{2.368027in}}%
\pgfpathlineto{\pgfqpoint{1.574143in}{2.436643in}}%
\pgfpathlineto{\pgfqpoint{1.582950in}{2.978796in}}%
\pgfpathlineto{\pgfqpoint{1.591757in}{2.971937in}}%
\pgfpathlineto{\pgfqpoint{1.600563in}{2.587620in}}%
\pgfpathlineto{\pgfqpoint{1.609370in}{2.292524in}}%
\pgfpathlineto{\pgfqpoint{1.618177in}{2.114111in}}%
\pgfpathlineto{\pgfqpoint{1.626984in}{1.963133in}}%
\pgfpathlineto{\pgfqpoint{1.635791in}{1.640574in}}%
\pgfpathlineto{\pgfqpoint{1.644597in}{1.743512in}}%
\pgfpathlineto{\pgfqpoint{1.653404in}{2.340563in}}%
\pgfpathlineto{\pgfqpoint{1.662211in}{2.031749in}}%
\pgfpathlineto{\pgfqpoint{1.671018in}{1.770975in}}%
\pgfpathlineto{\pgfqpoint{1.679825in}{1.695472in}}%
\pgfpathlineto{\pgfqpoint{1.688632in}{1.688613in}}%
\pgfpathlineto{\pgfqpoint{1.697438in}{1.860167in}}%
\pgfpathlineto{\pgfqpoint{1.706245in}{1.558212in}}%
\pgfpathlineto{\pgfqpoint{1.715052in}{1.551353in}}%
\pgfpathlineto{\pgfqpoint{1.723859in}{1.475878in}}%
\pgfpathlineto{\pgfqpoint{1.732666in}{1.654291in}}%
\pgfpathlineto{\pgfqpoint{1.741472in}{1.805269in}}%
\pgfpathlineto{\pgfqpoint{1.750279in}{1.880772in}}%
\pgfpathlineto{\pgfqpoint{1.759086in}{1.654291in}}%
\pgfpathlineto{\pgfqpoint{1.767893in}{1.743512in}}%
\pgfpathlineto{\pgfqpoint{1.776700in}{2.031749in}}%
\pgfpathlineto{\pgfqpoint{1.785507in}{2.189586in}}%
\pgfpathlineto{\pgfqpoint{1.794313in}{2.464106in}}%
\pgfpathlineto{\pgfqpoint{1.803120in}{2.381744in}}%
\pgfpathlineto{\pgfqpoint{1.811927in}{2.422925in}}%
\pgfpathlineto{\pgfqpoint{1.820734in}{2.182727in}}%
\pgfpathlineto{\pgfqpoint{1.829541in}{2.189586in}}%
\pgfpathlineto{\pgfqpoint{1.838347in}{1.812128in}}%
\pgfpathlineto{\pgfqpoint{1.847154in}{1.496455in}}%
\pgfpathlineto{\pgfqpoint{1.855961in}{1.578816in}}%
\pgfpathlineto{\pgfqpoint{1.864768in}{1.770975in}}%
\pgfpathlineto{\pgfqpoint{1.873575in}{1.805269in}}%
\pgfpathlineto{\pgfqpoint{1.882382in}{1.860167in}}%
\pgfpathlineto{\pgfqpoint{1.891188in}{1.839591in}}%
\pgfpathlineto{\pgfqpoint{1.899995in}{1.578816in}}%
\pgfpathlineto{\pgfqpoint{1.908802in}{1.215076in}}%
\pgfpathlineto{\pgfqpoint{1.917609in}{1.276861in}}%
\pgfpathlineto{\pgfqpoint{1.935222in}{1.976851in}}%
\pgfpathlineto{\pgfqpoint{1.944029in}{1.942529in}}%
\pgfpathlineto{\pgfqpoint{1.952836in}{1.963133in}}%
\pgfpathlineto{\pgfqpoint{1.961643in}{1.921953in}}%
\pgfpathlineto{\pgfqpoint{1.970450in}{2.079789in}}%
\pgfpathlineto{\pgfqpoint{1.979257in}{1.846450in}}%
\pgfpathlineto{\pgfqpoint{1.988063in}{1.812128in}}%
\pgfpathlineto{\pgfqpoint{1.996870in}{1.709190in}}%
\pgfpathlineto{\pgfqpoint{2.005677in}{1.846450in}}%
\pgfpathlineto{\pgfqpoint{2.014484in}{1.757229in}}%
\pgfpathlineto{\pgfqpoint{2.023291in}{1.345477in}}%
\pgfpathlineto{\pgfqpoint{2.032097in}{1.386658in}}%
\pgfpathlineto{\pgfqpoint{2.040904in}{1.482737in}}%
\pgfpathlineto{\pgfqpoint{2.049711in}{1.812128in}}%
\pgfpathlineto{\pgfqpoint{2.058518in}{1.544494in}}%
\pgfpathlineto{\pgfqpoint{2.067325in}{1.585675in}}%
\pgfpathlineto{\pgfqpoint{2.076132in}{1.743512in}}%
\pgfpathlineto{\pgfqpoint{2.084938in}{1.757229in}}%
\pgfpathlineto{\pgfqpoint{2.093745in}{1.619997in}}%
\pgfpathlineto{\pgfqpoint{2.102552in}{1.674896in}}%
\pgfpathlineto{\pgfqpoint{2.128972in}{2.251343in}}%
\pgfpathlineto{\pgfqpoint{2.137779in}{2.196445in}}%
\pgfpathlineto{\pgfqpoint{2.146586in}{2.292524in}}%
\pgfpathlineto{\pgfqpoint{2.155393in}{2.237625in}}%
\pgfpathlineto{\pgfqpoint{2.164200in}{2.443501in}}%
\pgfpathlineto{\pgfqpoint{2.173007in}{2.477823in}}%
\pgfpathlineto{\pgfqpoint{2.181813in}{2.587620in}}%
\pgfpathlineto{\pgfqpoint{2.190620in}{2.196445in}}%
\pgfpathlineto{\pgfqpoint{2.199427in}{2.018032in}}%
\pgfpathlineto{\pgfqpoint{2.208234in}{2.100365in}}%
\pgfpathlineto{\pgfqpoint{2.217041in}{2.265089in}}%
\pgfpathlineto{\pgfqpoint{2.225847in}{2.354309in}}%
\pgfpathlineto{\pgfqpoint{2.234654in}{2.203303in}}%
\pgfpathlineto{\pgfqpoint{2.243461in}{2.100365in}}%
\pgfpathlineto{\pgfqpoint{2.252268in}{2.182727in}}%
\pgfpathlineto{\pgfqpoint{2.261075in}{2.306269in}}%
\pgfpathlineto{\pgfqpoint{2.269882in}{2.361168in}}%
\pgfpathlineto{\pgfqpoint{2.278688in}{2.203303in}}%
\pgfpathlineto{\pgfqpoint{2.287495in}{2.210190in}}%
\pgfpathlineto{\pgfqpoint{2.296302in}{2.388603in}}%
\pgfpathlineto{\pgfqpoint{2.305109in}{2.237625in}}%
\pgfpathlineto{\pgfqpoint{2.313916in}{2.018032in}}%
\pgfpathlineto{\pgfqpoint{2.322722in}{2.004286in}}%
\pgfpathlineto{\pgfqpoint{2.331529in}{2.004286in}}%
\pgfpathlineto{\pgfqpoint{2.340336in}{1.647432in}}%
\pgfpathlineto{\pgfqpoint{2.357950in}{1.921953in}}%
\pgfpathlineto{\pgfqpoint{2.366757in}{1.729794in}}%
\pgfpathlineto{\pgfqpoint{2.375563in}{1.661150in}}%
\pgfpathlineto{\pgfqpoint{2.384370in}{1.915094in}}%
\pgfpathlineto{\pgfqpoint{2.393177in}{1.976851in}}%
\pgfpathlineto{\pgfqpoint{2.401984in}{2.024891in}}%
\pgfpathlineto{\pgfqpoint{2.410791in}{2.107252in}}%
\pgfpathlineto{\pgfqpoint{2.419597in}{1.887630in}}%
\pgfpathlineto{\pgfqpoint{2.428404in}{1.873913in}}%
\pgfpathlineto{\pgfqpoint{2.437211in}{1.626856in}}%
\pgfpathlineto{\pgfqpoint{2.446018in}{1.661150in}}%
\pgfpathlineto{\pgfqpoint{2.454825in}{1.674896in}}%
\pgfpathlineto{\pgfqpoint{2.463632in}{1.510172in}}%
\pgfpathlineto{\pgfqpoint{2.472438in}{1.716076in}}%
\pgfpathlineto{\pgfqpoint{2.481245in}{1.743512in}}%
\pgfpathlineto{\pgfqpoint{2.490052in}{1.489596in}}%
\pgfpathlineto{\pgfqpoint{2.498859in}{1.571958in}}%
\pgfpathlineto{\pgfqpoint{2.507666in}{1.537636in}}%
\pgfpathlineto{\pgfqpoint{2.516472in}{1.544494in}}%
\pgfpathlineto{\pgfqpoint{2.525279in}{1.709190in}}%
\pgfpathlineto{\pgfqpoint{2.534086in}{1.626856in}}%
\pgfpathlineto{\pgfqpoint{2.542893in}{1.681754in}}%
\pgfpathlineto{\pgfqpoint{2.551700in}{1.661150in}}%
\pgfpathlineto{\pgfqpoint{2.560507in}{1.990569in}}%
\pgfpathlineto{\pgfqpoint{2.569313in}{1.757229in}}%
\pgfpathlineto{\pgfqpoint{2.578120in}{2.134687in}}%
\pgfpathlineto{\pgfqpoint{2.586927in}{2.141546in}}%
\pgfpathlineto{\pgfqpoint{2.595734in}{2.189586in}}%
\pgfpathlineto{\pgfqpoint{2.604541in}{1.702331in}}%
\pgfpathlineto{\pgfqpoint{2.613347in}{2.223908in}}%
\pgfpathlineto{\pgfqpoint{2.622154in}{2.443501in}}%
\pgfpathlineto{\pgfqpoint{2.630961in}{2.175868in}}%
\pgfpathlineto{\pgfqpoint{2.639768in}{2.498428in}}%
\pgfpathlineto{\pgfqpoint{2.657382in}{2.093507in}}%
\pgfpathlineto{\pgfqpoint{2.666188in}{2.100365in}}%
\pgfpathlineto{\pgfqpoint{2.674995in}{1.887630in}}%
\pgfpathlineto{\pgfqpoint{2.683802in}{1.530777in}}%
\pgfpathlineto{\pgfqpoint{2.692609in}{1.695472in}}%
\pgfpathlineto{\pgfqpoint{2.701416in}{1.997427in}}%
\pgfpathlineto{\pgfqpoint{2.710222in}{2.148405in}}%
\pgfpathlineto{\pgfqpoint{2.719029in}{1.839591in}}%
\pgfpathlineto{\pgfqpoint{2.727836in}{1.887630in}}%
\pgfpathlineto{\pgfqpoint{2.736643in}{1.921953in}}%
\pgfpathlineto{\pgfqpoint{2.745450in}{1.702331in}}%
\pgfpathlineto{\pgfqpoint{2.754257in}{2.004286in}}%
\pgfpathlineto{\pgfqpoint{2.763063in}{1.867054in}}%
\pgfpathlineto{\pgfqpoint{2.771870in}{1.956246in}}%
\pgfpathlineto{\pgfqpoint{2.780677in}{1.812128in}}%
\pgfpathlineto{\pgfqpoint{2.789484in}{2.031749in}}%
\pgfpathlineto{\pgfqpoint{2.798291in}{1.764116in}}%
\pgfpathlineto{\pgfqpoint{2.807097in}{1.880772in}}%
\pgfpathlineto{\pgfqpoint{2.815904in}{2.127829in}}%
\pgfpathlineto{\pgfqpoint{2.824711in}{1.942529in}}%
\pgfpathlineto{\pgfqpoint{2.833518in}{2.045467in}}%
\pgfpathlineto{\pgfqpoint{2.842325in}{2.059213in}}%
\pgfpathlineto{\pgfqpoint{2.851132in}{2.045467in}}%
\pgfpathlineto{\pgfqpoint{2.859938in}{1.969992in}}%
\pgfpathlineto{\pgfqpoint{2.868745in}{1.983710in}}%
\pgfpathlineto{\pgfqpoint{2.877552in}{2.031749in}}%
\pgfpathlineto{\pgfqpoint{2.886359in}{1.674896in}}%
\pgfpathlineto{\pgfqpoint{2.895166in}{2.031749in}}%
\pgfpathlineto{\pgfqpoint{2.903972in}{1.880772in}}%
\pgfpathlineto{\pgfqpoint{2.912779in}{1.784692in}}%
\pgfpathlineto{\pgfqpoint{2.921586in}{1.997427in}}%
\pgfpathlineto{\pgfqpoint{2.930393in}{1.887630in}}%
\pgfpathlineto{\pgfqpoint{2.939200in}{1.551353in}}%
\pgfpathlineto{\pgfqpoint{2.948007in}{1.475878in}}%
\pgfpathlineto{\pgfqpoint{2.956813in}{1.942529in}}%
\pgfpathlineto{\pgfqpoint{2.965620in}{1.963133in}}%
\pgfpathlineto{\pgfqpoint{2.974427in}{2.292524in}}%
\pgfpathlineto{\pgfqpoint{2.983234in}{1.894489in}}%
\pgfpathlineto{\pgfqpoint{2.992041in}{2.079789in}}%
\pgfpathlineto{\pgfqpoint{3.000847in}{2.045467in}}%
\pgfpathlineto{\pgfqpoint{3.009654in}{2.210190in}}%
\pgfpathlineto{\pgfqpoint{3.018461in}{1.976851in}}%
\pgfpathlineto{\pgfqpoint{3.027268in}{2.045467in}}%
\pgfpathlineto{\pgfqpoint{3.036075in}{2.498428in}}%
\pgfpathlineto{\pgfqpoint{3.044882in}{2.354309in}}%
\pgfpathlineto{\pgfqpoint{3.053688in}{2.278806in}}%
\pgfpathlineto{\pgfqpoint{3.062495in}{2.244484in}}%
\pgfpathlineto{\pgfqpoint{3.071302in}{1.935670in}}%
\pgfpathlineto{\pgfqpoint{3.080109in}{1.846450in}}%
\pgfpathlineto{\pgfqpoint{3.088916in}{1.468992in}}%
\pgfpathlineto{\pgfqpoint{3.097722in}{1.468992in}}%
\pgfpathlineto{\pgfqpoint{3.106529in}{1.681754in}}%
\pgfpathlineto{\pgfqpoint{3.115336in}{1.757229in}}%
\pgfpathlineto{\pgfqpoint{3.124143in}{1.571958in}}%
\pgfpathlineto{\pgfqpoint{3.132950in}{1.441556in}}%
\pgfpathlineto{\pgfqpoint{3.141757in}{1.571958in}}%
\pgfpathlineto{\pgfqpoint{3.150563in}{1.743512in}}%
\pgfpathlineto{\pgfqpoint{3.159370in}{1.688613in}}%
\pgfpathlineto{\pgfqpoint{3.168177in}{1.668037in}}%
\pgfpathlineto{\pgfqpoint{3.176984in}{1.743512in}}%
\pgfpathlineto{\pgfqpoint{3.185791in}{1.839591in}}%
\pgfpathlineto{\pgfqpoint{3.194597in}{1.537636in}}%
\pgfpathlineto{\pgfqpoint{3.203404in}{1.640574in}}%
\pgfpathlineto{\pgfqpoint{3.212211in}{1.668037in}}%
\pgfpathlineto{\pgfqpoint{3.221018in}{1.633715in}}%
\pgfpathlineto{\pgfqpoint{3.238632in}{1.901348in}}%
\pgfpathlineto{\pgfqpoint{3.247438in}{1.688613in}}%
\pgfpathlineto{\pgfqpoint{3.256245in}{1.688613in}}%
\pgfpathlineto{\pgfqpoint{3.265052in}{1.544494in}}%
\pgfpathlineto{\pgfqpoint{3.273859in}{1.805269in}}%
\pgfpathlineto{\pgfqpoint{3.282666in}{1.990569in}}%
\pgfpathlineto{\pgfqpoint{3.291472in}{1.894489in}}%
\pgfpathlineto{\pgfqpoint{3.300279in}{1.997427in}}%
\pgfpathlineto{\pgfqpoint{3.309086in}{1.880772in}}%
\pgfpathlineto{\pgfqpoint{3.317893in}{1.867054in}}%
\pgfpathlineto{\pgfqpoint{3.326700in}{1.784692in}}%
\pgfpathlineto{\pgfqpoint{3.335507in}{1.928811in}}%
\pgfpathlineto{\pgfqpoint{3.344313in}{2.326846in}}%
\pgfpathlineto{\pgfqpoint{3.353120in}{2.093507in}}%
\pgfpathlineto{\pgfqpoint{3.361927in}{2.107252in}}%
\pgfpathlineto{\pgfqpoint{3.370734in}{1.963133in}}%
\pgfpathlineto{\pgfqpoint{3.379541in}{1.565071in}}%
\pgfpathlineto{\pgfqpoint{3.388347in}{1.269974in}}%
\pgfpathlineto{\pgfqpoint{3.397154in}{1.201358in}}%
\pgfpathlineto{\pgfqpoint{3.405961in}{1.201358in}}%
\pgfpathlineto{\pgfqpoint{3.414768in}{1.221935in}}%
\pgfpathlineto{\pgfqpoint{3.423575in}{1.427839in}}%
\pgfpathlineto{\pgfqpoint{3.432382in}{1.318014in}}%
\pgfpathlineto{\pgfqpoint{3.441188in}{1.523918in}}%
\pgfpathlineto{\pgfqpoint{3.449995in}{1.921953in}}%
\pgfpathlineto{\pgfqpoint{3.458802in}{1.942529in}}%
\pgfpathlineto{\pgfqpoint{3.467609in}{1.777834in}}%
\pgfpathlineto{\pgfqpoint{3.476416in}{1.880772in}}%
\pgfpathlineto{\pgfqpoint{3.485222in}{2.052326in}}%
\pgfpathlineto{\pgfqpoint{3.494029in}{2.148405in}}%
\pgfpathlineto{\pgfqpoint{3.502836in}{2.162151in}}%
\pgfpathlineto{\pgfqpoint{3.511643in}{2.148405in}}%
\pgfpathlineto{\pgfqpoint{3.520450in}{2.141546in}}%
\pgfpathlineto{\pgfqpoint{3.529257in}{2.326846in}}%
\pgfpathlineto{\pgfqpoint{3.538063in}{2.258230in}}%
\pgfpathlineto{\pgfqpoint{3.546870in}{2.038608in}}%
\pgfpathlineto{\pgfqpoint{3.555677in}{1.565071in}}%
\pgfpathlineto{\pgfqpoint{3.564484in}{2.127829in}}%
\pgfpathlineto{\pgfqpoint{3.573291in}{1.969992in}}%
\pgfpathlineto{\pgfqpoint{3.590904in}{2.519004in}}%
\pgfpathlineto{\pgfqpoint{3.599711in}{2.436643in}}%
\pgfpathlineto{\pgfqpoint{3.608518in}{2.443501in}}%
\pgfpathlineto{\pgfqpoint{3.617325in}{2.436643in}}%
\pgfpathlineto{\pgfqpoint{3.626132in}{2.162151in}}%
\pgfpathlineto{\pgfqpoint{3.634938in}{2.196445in}}%
\pgfpathlineto{\pgfqpoint{3.643745in}{2.018032in}}%
\pgfpathlineto{\pgfqpoint{3.652552in}{1.496455in}}%
\pgfpathlineto{\pgfqpoint{3.661359in}{1.969992in}}%
\pgfpathlineto{\pgfqpoint{3.670166in}{1.661150in}}%
\pgfpathlineto{\pgfqpoint{3.678972in}{1.695472in}}%
\pgfpathlineto{\pgfqpoint{3.687779in}{1.750370in}}%
\pgfpathlineto{\pgfqpoint{3.696586in}{1.873913in}}%
\pgfpathlineto{\pgfqpoint{3.705393in}{1.928811in}}%
\pgfpathlineto{\pgfqpoint{3.714200in}{1.825873in}}%
\pgfpathlineto{\pgfqpoint{3.723007in}{1.784692in}}%
\pgfpathlineto{\pgfqpoint{3.731813in}{1.887630in}}%
\pgfpathlineto{\pgfqpoint{3.740620in}{1.709190in}}%
\pgfpathlineto{\pgfqpoint{3.749427in}{1.887630in}}%
\pgfpathlineto{\pgfqpoint{3.758234in}{1.942529in}}%
\pgfpathlineto{\pgfqpoint{3.767041in}{1.825873in}}%
\pgfpathlineto{\pgfqpoint{3.775847in}{1.819015in}}%
\pgfpathlineto{\pgfqpoint{3.784654in}{1.942529in}}%
\pgfpathlineto{\pgfqpoint{3.793461in}{2.155292in}}%
\pgfpathlineto{\pgfqpoint{3.802268in}{1.695472in}}%
\pgfpathlineto{\pgfqpoint{3.811075in}{1.530777in}}%
\pgfpathlineto{\pgfqpoint{3.819882in}{1.860167in}}%
\pgfpathlineto{\pgfqpoint{3.828688in}{1.729794in}}%
\pgfpathlineto{\pgfqpoint{3.837495in}{1.949388in}}%
\pgfpathlineto{\pgfqpoint{3.846302in}{1.764116in}}%
\pgfpathlineto{\pgfqpoint{3.855109in}{1.674896in}}%
\pgfpathlineto{\pgfqpoint{3.863916in}{1.688613in}}%
\pgfpathlineto{\pgfqpoint{3.872722in}{1.777834in}}%
\pgfpathlineto{\pgfqpoint{3.881529in}{1.681754in}}%
\pgfpathlineto{\pgfqpoint{3.890336in}{1.729794in}}%
\pgfpathlineto{\pgfqpoint{3.899143in}{1.482737in}}%
\pgfpathlineto{\pgfqpoint{3.907950in}{1.832732in}}%
\pgfpathlineto{\pgfqpoint{3.916757in}{1.578816in}}%
\pgfpathlineto{\pgfqpoint{3.925563in}{1.860167in}}%
\pgfpathlineto{\pgfqpoint{3.934370in}{1.674896in}}%
\pgfpathlineto{\pgfqpoint{3.943177in}{1.901348in}}%
\pgfpathlineto{\pgfqpoint{3.951984in}{2.347422in}}%
\pgfpathlineto{\pgfqpoint{3.960791in}{2.251343in}}%
\pgfpathlineto{\pgfqpoint{3.969597in}{2.388603in}}%
\pgfpathlineto{\pgfqpoint{3.978404in}{2.155292in}}%
\pgfpathlineto{\pgfqpoint{3.996018in}{1.757229in}}%
\pgfpathlineto{\pgfqpoint{4.004825in}{1.777834in}}%
\pgfpathlineto{\pgfqpoint{4.013632in}{1.558212in}}%
\pgfpathlineto{\pgfqpoint{4.022438in}{1.736653in}}%
\pgfpathlineto{\pgfqpoint{4.031245in}{1.949388in}}%
\pgfpathlineto{\pgfqpoint{4.040052in}{1.921953in}}%
\pgfpathlineto{\pgfqpoint{4.048859in}{1.599393in}}%
\pgfpathlineto{\pgfqpoint{4.057666in}{1.688613in}}%
\pgfpathlineto{\pgfqpoint{4.066472in}{1.729794in}}%
\pgfpathlineto{\pgfqpoint{4.075279in}{1.887630in}}%
\pgfpathlineto{\pgfqpoint{4.084086in}{2.141546in}}%
\pgfpathlineto{\pgfqpoint{4.092893in}{1.949388in}}%
\pgfpathlineto{\pgfqpoint{4.101700in}{2.210190in}}%
\pgfpathlineto{\pgfqpoint{4.110507in}{2.024891in}}%
\pgfpathlineto{\pgfqpoint{4.119313in}{1.880772in}}%
\pgfpathlineto{\pgfqpoint{4.128120in}{1.880772in}}%
\pgfpathlineto{\pgfqpoint{4.136927in}{1.867054in}}%
\pgfpathlineto{\pgfqpoint{4.145734in}{1.702331in}}%
\pgfpathlineto{\pgfqpoint{4.154541in}{1.688613in}}%
\pgfpathlineto{\pgfqpoint{4.172154in}{1.427839in}}%
\pgfpathlineto{\pgfqpoint{4.180961in}{1.235680in}}%
\pgfpathlineto{\pgfqpoint{4.189768in}{1.215076in}}%
\pgfpathlineto{\pgfqpoint{4.198575in}{1.757229in}}%
\pgfpathlineto{\pgfqpoint{4.207382in}{1.832732in}}%
\pgfpathlineto{\pgfqpoint{4.216188in}{1.983710in}}%
\pgfpathlineto{\pgfqpoint{4.224995in}{1.976851in}}%
\pgfpathlineto{\pgfqpoint{4.233802in}{2.258230in}}%
\pgfpathlineto{\pgfqpoint{4.242609in}{2.031749in}}%
\pgfpathlineto{\pgfqpoint{4.251416in}{1.935670in}}%
\pgfpathlineto{\pgfqpoint{4.260222in}{1.983710in}}%
\pgfpathlineto{\pgfqpoint{4.269029in}{2.120970in}}%
\pgfpathlineto{\pgfqpoint{4.277836in}{1.935670in}}%
\pgfpathlineto{\pgfqpoint{4.286643in}{1.798410in}}%
\pgfpathlineto{\pgfqpoint{4.295450in}{1.517031in}}%
\pgfpathlineto{\pgfqpoint{4.304257in}{1.496455in}}%
\pgfpathlineto{\pgfqpoint{4.313063in}{1.709190in}}%
\pgfpathlineto{\pgfqpoint{4.321870in}{1.839591in}}%
\pgfpathlineto{\pgfqpoint{4.330677in}{1.455274in}}%
\pgfpathlineto{\pgfqpoint{4.339484in}{1.565071in}}%
\pgfpathlineto{\pgfqpoint{4.348291in}{1.894489in}}%
\pgfpathlineto{\pgfqpoint{4.357097in}{1.853308in}}%
\pgfpathlineto{\pgfqpoint{4.365904in}{1.599393in}}%
\pgfpathlineto{\pgfqpoint{4.374711in}{1.537636in}}%
\pgfpathlineto{\pgfqpoint{4.383518in}{1.592534in}}%
\pgfpathlineto{\pgfqpoint{4.392325in}{1.215076in}}%
\pgfpathlineto{\pgfqpoint{4.401132in}{1.613110in}}%
\pgfpathlineto{\pgfqpoint{4.409938in}{1.585675in}}%
\pgfpathlineto{\pgfqpoint{4.418745in}{1.805269in}}%
\pgfpathlineto{\pgfqpoint{4.427552in}{2.107252in}}%
\pgfpathlineto{\pgfqpoint{4.436359in}{2.148405in}}%
\pgfpathlineto{\pgfqpoint{4.445166in}{2.175868in}}%
\pgfpathlineto{\pgfqpoint{4.453972in}{2.313128in}}%
\pgfpathlineto{\pgfqpoint{4.462779in}{2.265089in}}%
\pgfpathlineto{\pgfqpoint{4.471586in}{2.271947in}}%
\pgfpathlineto{\pgfqpoint{4.480393in}{2.182727in}}%
\pgfpathlineto{\pgfqpoint{4.489200in}{1.750370in}}%
\pgfpathlineto{\pgfqpoint{4.498007in}{1.839591in}}%
\pgfpathlineto{\pgfqpoint{4.506813in}{1.654291in}}%
\pgfpathlineto{\pgfqpoint{4.515620in}{1.976851in}}%
\pgfpathlineto{\pgfqpoint{4.524427in}{1.819015in}}%
\pgfpathlineto{\pgfqpoint{4.533234in}{1.764116in}}%
\pgfpathlineto{\pgfqpoint{4.542041in}{1.915094in}}%
\pgfpathlineto{\pgfqpoint{4.550847in}{2.114111in}}%
\pgfpathlineto{\pgfqpoint{4.559654in}{2.169009in}}%
\pgfpathlineto{\pgfqpoint{4.568461in}{1.764116in}}%
\pgfpathlineto{\pgfqpoint{4.577268in}{1.798410in}}%
\pgfpathlineto{\pgfqpoint{4.586075in}{2.299383in}}%
\pgfpathlineto{\pgfqpoint{4.594882in}{2.141546in}}%
\pgfpathlineto{\pgfqpoint{4.603688in}{2.285665in}}%
\pgfpathlineto{\pgfqpoint{4.612495in}{2.244484in}}%
\pgfpathlineto{\pgfqpoint{4.621302in}{2.093507in}}%
\pgfpathlineto{\pgfqpoint{4.630109in}{2.018032in}}%
\pgfpathlineto{\pgfqpoint{4.638916in}{2.018032in}}%
\pgfpathlineto{\pgfqpoint{4.647722in}{1.949388in}}%
\pgfpathlineto{\pgfqpoint{4.656529in}{1.819015in}}%
\pgfpathlineto{\pgfqpoint{4.665336in}{1.647432in}}%
\pgfpathlineto{\pgfqpoint{4.674143in}{2.038608in}}%
\pgfpathlineto{\pgfqpoint{4.682950in}{1.702331in}}%
\pgfpathlineto{\pgfqpoint{4.691757in}{1.791551in}}%
\pgfpathlineto{\pgfqpoint{4.700563in}{1.956246in}}%
\pgfpathlineto{\pgfqpoint{4.709370in}{2.175868in}}%
\pgfpathlineto{\pgfqpoint{4.718177in}{2.251343in}}%
\pgfpathlineto{\pgfqpoint{4.735791in}{2.114111in}}%
\pgfpathlineto{\pgfqpoint{4.744597in}{2.086648in}}%
\pgfpathlineto{\pgfqpoint{4.753404in}{2.251343in}}%
\pgfpathlineto{\pgfqpoint{4.762211in}{2.024891in}}%
\pgfpathlineto{\pgfqpoint{4.771018in}{2.285665in}}%
\pgfpathlineto{\pgfqpoint{4.779825in}{1.990569in}}%
\pgfpathlineto{\pgfqpoint{4.788632in}{2.333705in}}%
\pgfpathlineto{\pgfqpoint{4.797438in}{2.237625in}}%
\pgfpathlineto{\pgfqpoint{4.806245in}{1.908207in}}%
\pgfpathlineto{\pgfqpoint{4.815052in}{1.969992in}}%
\pgfpathlineto{\pgfqpoint{4.823859in}{1.722935in}}%
\pgfpathlineto{\pgfqpoint{4.832666in}{1.626856in}}%
\pgfpathlineto{\pgfqpoint{4.841472in}{1.462133in}}%
\pgfpathlineto{\pgfqpoint{4.850279in}{1.558212in}}%
\pgfpathlineto{\pgfqpoint{4.859086in}{1.626856in}}%
\pgfpathlineto{\pgfqpoint{4.867893in}{1.894489in}}%
\pgfpathlineto{\pgfqpoint{4.876700in}{2.079789in}}%
\pgfpathlineto{\pgfqpoint{4.885507in}{2.148405in}}%
\pgfpathlineto{\pgfqpoint{4.894313in}{2.340563in}}%
\pgfpathlineto{\pgfqpoint{4.903120in}{2.374885in}}%
\pgfpathlineto{\pgfqpoint{4.911927in}{2.162151in}}%
\pgfpathlineto{\pgfqpoint{4.920734in}{1.976851in}}%
\pgfpathlineto{\pgfqpoint{4.929541in}{2.093507in}}%
\pgfpathlineto{\pgfqpoint{4.938347in}{2.038608in}}%
\pgfpathlineto{\pgfqpoint{4.947154in}{2.066071in}}%
\pgfpathlineto{\pgfqpoint{4.955961in}{1.805269in}}%
\pgfpathlineto{\pgfqpoint{4.964768in}{2.024891in}}%
\pgfpathlineto{\pgfqpoint{4.973575in}{1.956246in}}%
\pgfpathlineto{\pgfqpoint{4.982382in}{1.915094in}}%
\pgfpathlineto{\pgfqpoint{4.991188in}{2.107252in}}%
\pgfpathlineto{\pgfqpoint{4.999995in}{2.093507in}}%
\pgfpathlineto{\pgfqpoint{5.008802in}{1.777834in}}%
\pgfpathlineto{\pgfqpoint{5.017609in}{1.777834in}}%
\pgfpathlineto{\pgfqpoint{5.026416in}{1.551353in}}%
\pgfpathlineto{\pgfqpoint{5.035222in}{1.757229in}}%
\pgfpathlineto{\pgfqpoint{5.044029in}{1.695472in}}%
\pgfpathlineto{\pgfqpoint{5.052836in}{1.654291in}}%
\pgfpathlineto{\pgfqpoint{5.061643in}{1.867054in}}%
\pgfpathlineto{\pgfqpoint{5.070450in}{1.921953in}}%
\pgfpathlineto{\pgfqpoint{5.079257in}{2.038608in}}%
\pgfpathlineto{\pgfqpoint{5.088063in}{1.873913in}}%
\pgfpathlineto{\pgfqpoint{5.096870in}{1.873913in}}%
\pgfpathlineto{\pgfqpoint{5.105677in}{2.361168in}}%
\pgfpathlineto{\pgfqpoint{5.114484in}{2.278806in}}%
\pgfpathlineto{\pgfqpoint{5.132097in}{1.887630in}}%
\pgfpathlineto{\pgfqpoint{5.140904in}{2.031749in}}%
\pgfpathlineto{\pgfqpoint{5.149711in}{1.860167in}}%
\pgfpathlineto{\pgfqpoint{5.158518in}{2.100365in}}%
\pgfpathlineto{\pgfqpoint{5.167325in}{1.976851in}}%
\pgfpathlineto{\pgfqpoint{5.176132in}{2.299383in}}%
\pgfpathlineto{\pgfqpoint{5.184938in}{2.470965in}}%
\pgfpathlineto{\pgfqpoint{5.193745in}{2.278806in}}%
\pgfpathlineto{\pgfqpoint{5.202552in}{1.976851in}}%
\pgfpathlineto{\pgfqpoint{5.211359in}{2.127829in}}%
\pgfpathlineto{\pgfqpoint{5.220166in}{2.155292in}}%
\pgfpathlineto{\pgfqpoint{5.228972in}{2.210190in}}%
\pgfpathlineto{\pgfqpoint{5.237779in}{1.969992in}}%
\pgfpathlineto{\pgfqpoint{5.246586in}{2.018032in}}%
\pgfpathlineto{\pgfqpoint{5.255393in}{1.853308in}}%
\pgfpathlineto{\pgfqpoint{5.264200in}{1.764116in}}%
\pgfpathlineto{\pgfqpoint{5.273007in}{2.210190in}}%
\pgfpathlineto{\pgfqpoint{5.281813in}{2.059213in}}%
\pgfpathlineto{\pgfqpoint{5.290620in}{2.368027in}}%
\pgfpathlineto{\pgfqpoint{5.299427in}{2.127829in}}%
\pgfpathlineto{\pgfqpoint{5.308234in}{2.011173in}}%
\pgfpathlineto{\pgfqpoint{5.317041in}{2.072930in}}%
\pgfpathlineto{\pgfqpoint{5.325847in}{1.812128in}}%
\pgfpathlineto{\pgfqpoint{5.334654in}{1.867054in}}%
\pgfpathlineto{\pgfqpoint{5.343461in}{1.997427in}}%
\pgfpathlineto{\pgfqpoint{5.352268in}{1.668037in}}%
\pgfpathlineto{\pgfqpoint{5.361075in}{1.805269in}}%
\pgfpathlineto{\pgfqpoint{5.369882in}{1.743512in}}%
\pgfpathlineto{\pgfqpoint{5.378688in}{1.949388in}}%
\pgfpathlineto{\pgfqpoint{5.387495in}{2.127829in}}%
\pgfpathlineto{\pgfqpoint{5.396302in}{2.169009in}}%
\pgfpathlineto{\pgfqpoint{5.405109in}{2.018032in}}%
\pgfpathlineto{\pgfqpoint{5.413916in}{2.189586in}}%
\pgfpathlineto{\pgfqpoint{5.422722in}{2.155292in}}%
\pgfpathlineto{\pgfqpoint{5.431529in}{2.155292in}}%
\pgfpathlineto{\pgfqpoint{5.440336in}{2.079789in}}%
\pgfpathlineto{\pgfqpoint{5.449143in}{1.853308in}}%
\pgfpathlineto{\pgfqpoint{5.457950in}{1.716076in}}%
\pgfpathlineto{\pgfqpoint{5.466757in}{1.688613in}}%
\pgfpathlineto{\pgfqpoint{5.475563in}{1.928811in}}%
\pgfpathlineto{\pgfqpoint{5.484370in}{1.901348in}}%
\pgfpathlineto{\pgfqpoint{5.493177in}{1.736653in}}%
\pgfpathlineto{\pgfqpoint{5.501984in}{1.832732in}}%
\pgfpathlineto{\pgfqpoint{5.510791in}{1.908207in}}%
\pgfpathlineto{\pgfqpoint{5.519597in}{2.107252in}}%
\pgfpathlineto{\pgfqpoint{5.528404in}{1.928811in}}%
\pgfpathlineto{\pgfqpoint{5.546018in}{1.736653in}}%
\pgfpathlineto{\pgfqpoint{5.554825in}{1.798410in}}%
\pgfpathlineto{\pgfqpoint{5.563632in}{1.661150in}}%
\pgfpathlineto{\pgfqpoint{5.572438in}{1.626856in}}%
\pgfpathlineto{\pgfqpoint{5.581245in}{1.668037in}}%
\pgfpathlineto{\pgfqpoint{5.590052in}{1.544494in}}%
\pgfpathlineto{\pgfqpoint{5.598859in}{1.215076in}}%
\pgfpathlineto{\pgfqpoint{5.607666in}{1.256257in}}%
\pgfpathlineto{\pgfqpoint{5.625279in}{1.482737in}}%
\pgfpathlineto{\pgfqpoint{5.634086in}{1.489596in}}%
\pgfpathlineto{\pgfqpoint{5.642893in}{1.434698in}}%
\pgfpathlineto{\pgfqpoint{5.651700in}{1.722935in}}%
\pgfpathlineto{\pgfqpoint{5.660507in}{2.141546in}}%
\pgfpathlineto{\pgfqpoint{5.669313in}{1.853308in}}%
\pgfpathlineto{\pgfqpoint{5.678120in}{1.839591in}}%
\pgfpathlineto{\pgfqpoint{5.686927in}{1.757229in}}%
\pgfpathlineto{\pgfqpoint{5.695734in}{1.420952in}}%
\pgfpathlineto{\pgfqpoint{5.704541in}{1.585675in}}%
\pgfpathlineto{\pgfqpoint{5.713347in}{1.565071in}}%
\pgfpathlineto{\pgfqpoint{5.722154in}{1.352336in}}%
\pgfpathlineto{\pgfqpoint{5.730961in}{1.640574in}}%
\pgfpathlineto{\pgfqpoint{5.739768in}{2.011173in}}%
\pgfpathlineto{\pgfqpoint{5.748575in}{1.908207in}}%
\pgfpathlineto{\pgfqpoint{5.757382in}{1.674896in}}%
\pgfpathlineto{\pgfqpoint{5.766188in}{1.626856in}}%
\pgfpathlineto{\pgfqpoint{5.774995in}{1.695472in}}%
\pgfpathlineto{\pgfqpoint{5.783802in}{2.155292in}}%
\pgfpathlineto{\pgfqpoint{5.792609in}{2.196445in}}%
\pgfpathlineto{\pgfqpoint{5.801416in}{1.825873in}}%
\pgfpathlineto{\pgfqpoint{5.810222in}{1.681754in}}%
\pgfpathlineto{\pgfqpoint{5.819029in}{1.805269in}}%
\pgfpathlineto{\pgfqpoint{5.827836in}{1.468992in}}%
\pgfpathlineto{\pgfqpoint{5.836643in}{1.571958in}}%
\pgfpathlineto{\pgfqpoint{5.845450in}{2.045467in}}%
\pgfpathlineto{\pgfqpoint{5.854257in}{1.942529in}}%
\pgfpathlineto{\pgfqpoint{5.863063in}{2.086648in}}%
\pgfpathlineto{\pgfqpoint{5.871870in}{2.265089in}}%
\pgfpathlineto{\pgfqpoint{5.880677in}{2.169009in}}%
\pgfpathlineto{\pgfqpoint{5.889484in}{2.086648in}}%
\pgfpathlineto{\pgfqpoint{5.898291in}{2.354309in}}%
\pgfpathlineto{\pgfqpoint{5.907097in}{2.326846in}}%
\pgfpathlineto{\pgfqpoint{5.915904in}{1.983710in}}%
\pgfpathlineto{\pgfqpoint{5.924711in}{2.127829in}}%
\pgfpathlineto{\pgfqpoint{5.933518in}{2.155292in}}%
\pgfpathlineto{\pgfqpoint{5.942325in}{2.196445in}}%
\pgfpathlineto{\pgfqpoint{5.951132in}{2.271947in}}%
\pgfpathlineto{\pgfqpoint{5.959938in}{2.148405in}}%
\pgfpathlineto{\pgfqpoint{5.968745in}{1.819015in}}%
\pgfpathlineto{\pgfqpoint{5.977552in}{1.942529in}}%
\pgfpathlineto{\pgfqpoint{5.986359in}{1.750370in}}%
\pgfpathlineto{\pgfqpoint{5.995166in}{1.990569in}}%
\pgfpathlineto{\pgfqpoint{6.003972in}{2.306269in}}%
\pgfpathlineto{\pgfqpoint{6.012779in}{2.422925in}}%
\pgfpathlineto{\pgfqpoint{6.021586in}{2.621942in}}%
\pgfpathlineto{\pgfqpoint{6.030393in}{2.738598in}}%
\pgfpathlineto{\pgfqpoint{6.048007in}{2.319987in}}%
\pgfpathlineto{\pgfqpoint{6.056813in}{2.251343in}}%
\pgfpathlineto{\pgfqpoint{6.065620in}{2.333705in}}%
\pgfpathlineto{\pgfqpoint{6.074427in}{2.443501in}}%
\pgfpathlineto{\pgfqpoint{6.083234in}{2.484682in}}%
\pgfpathlineto{\pgfqpoint{6.092041in}{2.326846in}}%
\pgfpathlineto{\pgfqpoint{6.100847in}{2.093507in}}%
\pgfpathlineto{\pgfqpoint{6.109654in}{2.134687in}}%
\pgfpathlineto{\pgfqpoint{6.118461in}{2.299383in}}%
\pgfpathlineto{\pgfqpoint{6.127268in}{2.169009in}}%
\pgfpathlineto{\pgfqpoint{6.136075in}{2.319987in}}%
\pgfpathlineto{\pgfqpoint{6.144882in}{2.265089in}}%
\pgfpathlineto{\pgfqpoint{6.153688in}{2.271947in}}%
\pgfpathlineto{\pgfqpoint{6.162495in}{2.745485in}}%
\pgfpathlineto{\pgfqpoint{6.171302in}{2.841564in}}%
\pgfpathlineto{\pgfqpoint{6.180109in}{2.189586in}}%
\pgfpathlineto{\pgfqpoint{6.188916in}{2.120970in}}%
\pgfpathlineto{\pgfqpoint{6.197722in}{1.921953in}}%
\pgfpathlineto{\pgfqpoint{6.206529in}{1.956246in}}%
\pgfpathlineto{\pgfqpoint{6.215336in}{1.695472in}}%
\pgfpathlineto{\pgfqpoint{6.224143in}{1.901348in}}%
\pgfpathlineto{\pgfqpoint{6.232950in}{1.956246in}}%
\pgfpathlineto{\pgfqpoint{6.241757in}{1.983710in}}%
\pgfpathlineto{\pgfqpoint{6.250563in}{1.654291in}}%
\pgfpathlineto{\pgfqpoint{6.259370in}{1.853308in}}%
\pgfpathlineto{\pgfqpoint{6.268177in}{1.654291in}}%
\pgfpathlineto{\pgfqpoint{6.276984in}{1.956246in}}%
\pgfpathlineto{\pgfqpoint{6.285791in}{1.757229in}}%
\pgfpathlineto{\pgfqpoint{6.294597in}{1.887630in}}%
\pgfpathlineto{\pgfqpoint{6.303404in}{2.155292in}}%
\pgfpathlineto{\pgfqpoint{6.312211in}{1.956246in}}%
\pgfpathlineto{\pgfqpoint{6.321018in}{2.018032in}}%
\pgfpathlineto{\pgfqpoint{6.338632in}{1.716076in}}%
\pgfpathlineto{\pgfqpoint{6.347438in}{1.997427in}}%
\pgfpathlineto{\pgfqpoint{6.356245in}{1.873913in}}%
\pgfpathlineto{\pgfqpoint{6.365052in}{1.633715in}}%
\pgfpathlineto{\pgfqpoint{6.373859in}{1.311155in}}%
\pgfpathlineto{\pgfqpoint{6.382666in}{1.311155in}}%
\pgfpathlineto{\pgfqpoint{6.391472in}{1.661150in}}%
\pgfpathlineto{\pgfqpoint{6.400279in}{1.702331in}}%
\pgfpathlineto{\pgfqpoint{6.409086in}{1.352336in}}%
\pgfpathlineto{\pgfqpoint{6.417893in}{1.366053in}}%
\pgfpathlineto{\pgfqpoint{6.426700in}{1.558212in}}%
\pgfpathlineto{\pgfqpoint{6.435507in}{1.915094in}}%
\pgfpathlineto{\pgfqpoint{6.444313in}{1.839591in}}%
\pgfpathlineto{\pgfqpoint{6.453120in}{1.681754in}}%
\pgfpathlineto{\pgfqpoint{6.461927in}{1.908207in}}%
\pgfpathlineto{\pgfqpoint{6.470734in}{1.908207in}}%
\pgfpathlineto{\pgfqpoint{6.479541in}{1.963133in}}%
\pgfpathlineto{\pgfqpoint{6.488347in}{2.388603in}}%
\pgfpathlineto{\pgfqpoint{6.497154in}{2.354309in}}%
\pgfpathlineto{\pgfqpoint{6.505961in}{2.162151in}}%
\pgfpathlineto{\pgfqpoint{6.514768in}{1.757229in}}%
\pgfpathlineto{\pgfqpoint{6.523575in}{2.134687in}}%
\pgfpathlineto{\pgfqpoint{6.532382in}{2.148405in}}%
\pgfpathlineto{\pgfqpoint{6.541188in}{2.223908in}}%
\pgfpathlineto{\pgfqpoint{6.549995in}{2.326846in}}%
\pgfpathlineto{\pgfqpoint{6.558802in}{2.381744in}}%
\pgfpathlineto{\pgfqpoint{6.567609in}{2.182727in}}%
\pgfpathlineto{\pgfqpoint{6.576416in}{1.777834in}}%
\pgfpathlineto{\pgfqpoint{6.585222in}{1.825873in}}%
\pgfpathlineto{\pgfqpoint{6.594029in}{2.155292in}}%
\pgfpathlineto{\pgfqpoint{6.602836in}{1.873913in}}%
\pgfpathlineto{\pgfqpoint{6.611643in}{1.880772in}}%
\pgfpathlineto{\pgfqpoint{6.620450in}{1.812128in}}%
\pgfpathlineto{\pgfqpoint{6.629257in}{1.661150in}}%
\pgfpathlineto{\pgfqpoint{6.638063in}{1.633715in}}%
\pgfpathlineto{\pgfqpoint{6.646870in}{1.894489in}}%
\pgfpathlineto{\pgfqpoint{6.655677in}{1.921953in}}%
\pgfpathlineto{\pgfqpoint{6.664484in}{2.182727in}}%
\pgfpathlineto{\pgfqpoint{6.673291in}{1.839591in}}%
\pgfpathlineto{\pgfqpoint{6.682097in}{1.722935in}}%
\pgfpathlineto{\pgfqpoint{6.690904in}{1.468992in}}%
\pgfpathlineto{\pgfqpoint{6.699711in}{1.269974in}}%
\pgfpathlineto{\pgfqpoint{6.708518in}{1.160177in}}%
\pgfpathlineto{\pgfqpoint{6.717325in}{1.263115in}}%
\pgfpathlineto{\pgfqpoint{6.726132in}{1.379799in}}%
\pgfpathlineto{\pgfqpoint{6.734938in}{1.352336in}}%
\pgfpathlineto{\pgfqpoint{6.743745in}{1.400376in}}%
\pgfpathlineto{\pgfqpoint{6.752552in}{1.654291in}}%
\pgfpathlineto{\pgfqpoint{6.761359in}{1.722935in}}%
\pgfpathlineto{\pgfqpoint{6.770166in}{1.976851in}}%
\pgfpathlineto{\pgfqpoint{6.778972in}{1.661150in}}%
\pgfpathlineto{\pgfqpoint{6.787779in}{1.695472in}}%
\pgfpathlineto{\pgfqpoint{6.796586in}{1.585675in}}%
\pgfpathlineto{\pgfqpoint{6.805393in}{1.379799in}}%
\pgfpathlineto{\pgfqpoint{6.814200in}{1.592534in}}%
\pgfpathlineto{\pgfqpoint{6.823007in}{1.832732in}}%
\pgfpathlineto{\pgfqpoint{6.831813in}{2.114111in}}%
\pgfpathlineto{\pgfqpoint{6.840620in}{1.949388in}}%
\pgfpathlineto{\pgfqpoint{6.849427in}{2.100365in}}%
\pgfpathlineto{\pgfqpoint{6.858234in}{2.141546in}}%
\pgfpathlineto{\pgfqpoint{6.867041in}{1.716076in}}%
\pgfpathlineto{\pgfqpoint{6.875847in}{1.716076in}}%
\pgfpathlineto{\pgfqpoint{6.884654in}{1.743512in}}%
\pgfpathlineto{\pgfqpoint{6.893461in}{1.983710in}}%
\pgfpathlineto{\pgfqpoint{6.902268in}{1.880772in}}%
\pgfpathlineto{\pgfqpoint{6.911075in}{1.894489in}}%
\pgfpathlineto{\pgfqpoint{6.919882in}{2.265089in}}%
\pgfpathlineto{\pgfqpoint{6.928688in}{1.825873in}}%
\pgfpathlineto{\pgfqpoint{6.937495in}{1.757229in}}%
\pgfpathlineto{\pgfqpoint{6.946302in}{2.072930in}}%
\pgfpathlineto{\pgfqpoint{6.955109in}{1.482737in}}%
\pgfpathlineto{\pgfqpoint{6.972722in}{1.825873in}}%
\pgfpathlineto{\pgfqpoint{6.981529in}{2.313128in}}%
\pgfpathlineto{\pgfqpoint{6.990336in}{2.429784in}}%
\pgfpathlineto{\pgfqpoint{6.999143in}{2.237625in}}%
\pgfpathlineto{\pgfqpoint{7.007950in}{2.155292in}}%
\pgfpathlineto{\pgfqpoint{7.016757in}{2.004286in}}%
\pgfpathlineto{\pgfqpoint{7.025563in}{2.059213in}}%
\pgfpathlineto{\pgfqpoint{7.034370in}{1.860167in}}%
\pgfpathlineto{\pgfqpoint{7.043177in}{1.366053in}}%
\pgfpathlineto{\pgfqpoint{7.051984in}{1.640574in}}%
\pgfpathlineto{\pgfqpoint{7.060791in}{1.599393in}}%
\pgfpathlineto{\pgfqpoint{7.069597in}{1.626856in}}%
\pgfpathlineto{\pgfqpoint{7.078404in}{1.819015in}}%
\pgfpathlineto{\pgfqpoint{7.087211in}{1.750370in}}%
\pgfpathlineto{\pgfqpoint{7.096018in}{2.031749in}}%
\pgfpathlineto{\pgfqpoint{7.104825in}{1.777834in}}%
\pgfpathlineto{\pgfqpoint{7.113632in}{1.571958in}}%
\pgfpathlineto{\pgfqpoint{7.122438in}{1.420952in}}%
\pgfpathlineto{\pgfqpoint{7.131245in}{1.757229in}}%
\pgfpathlineto{\pgfqpoint{7.140052in}{1.873913in}}%
\pgfpathlineto{\pgfqpoint{7.148859in}{1.798410in}}%
\pgfpathlineto{\pgfqpoint{7.157666in}{1.558212in}}%
\pgfpathlineto{\pgfqpoint{7.166472in}{1.517031in}}%
\pgfpathlineto{\pgfqpoint{7.175279in}{1.695472in}}%
\pgfpathlineto{\pgfqpoint{7.184086in}{1.839591in}}%
\pgfpathlineto{\pgfqpoint{7.192893in}{2.175868in}}%
\pgfpathlineto{\pgfqpoint{7.201700in}{2.258230in}}%
\pgfpathlineto{\pgfqpoint{7.210507in}{2.100365in}}%
\pgfpathlineto{\pgfqpoint{7.219313in}{1.963133in}}%
\pgfpathlineto{\pgfqpoint{7.228120in}{1.928811in}}%
\pgfpathlineto{\pgfqpoint{7.236927in}{2.093507in}}%
\pgfpathlineto{\pgfqpoint{7.245734in}{2.093507in}}%
\pgfpathlineto{\pgfqpoint{7.254541in}{2.519004in}}%
\pgfpathlineto{\pgfqpoint{7.263347in}{2.148405in}}%
\pgfpathlineto{\pgfqpoint{7.272154in}{2.532722in}}%
\pgfpathlineto{\pgfqpoint{7.280961in}{2.223908in}}%
\pgfpathlineto{\pgfqpoint{7.289768in}{2.217049in}}%
\pgfpathlineto{\pgfqpoint{7.298575in}{2.052326in}}%
\pgfpathlineto{\pgfqpoint{7.307382in}{1.935670in}}%
\pgfpathlineto{\pgfqpoint{7.316188in}{2.271947in}}%
\pgfpathlineto{\pgfqpoint{7.324995in}{2.223908in}}%
\pgfpathlineto{\pgfqpoint{7.333802in}{2.313128in}}%
\pgfpathlineto{\pgfqpoint{7.342609in}{2.381744in}}%
\pgfpathlineto{\pgfqpoint{7.351416in}{2.066071in}}%
\pgfpathlineto{\pgfqpoint{7.360222in}{1.969992in}}%
\pgfpathlineto{\pgfqpoint{7.369029in}{2.024891in}}%
\pgfpathlineto{\pgfqpoint{7.377836in}{1.860167in}}%
\pgfpathlineto{\pgfqpoint{7.386643in}{2.107252in}}%
\pgfpathlineto{\pgfqpoint{7.395450in}{2.258230in}}%
\pgfpathlineto{\pgfqpoint{7.404257in}{2.169009in}}%
\pgfpathlineto{\pgfqpoint{7.421870in}{2.052326in}}%
\pgfpathlineto{\pgfqpoint{7.430677in}{2.120970in}}%
\pgfpathlineto{\pgfqpoint{7.439484in}{2.361168in}}%
\pgfpathlineto{\pgfqpoint{7.448291in}{2.285665in}}%
\pgfpathlineto{\pgfqpoint{7.457097in}{2.120970in}}%
\pgfpathlineto{\pgfqpoint{7.465904in}{1.983710in}}%
\pgfpathlineto{\pgfqpoint{7.474711in}{2.182727in}}%
\pgfpathlineto{\pgfqpoint{7.483518in}{2.292524in}}%
\pgfpathlineto{\pgfqpoint{7.492325in}{2.045467in}}%
\pgfpathlineto{\pgfqpoint{7.501132in}{1.935670in}}%
\pgfpathlineto{\pgfqpoint{7.509938in}{2.072930in}}%
\pgfpathlineto{\pgfqpoint{7.518745in}{1.853308in}}%
\pgfpathlineto{\pgfqpoint{7.527552in}{1.825873in}}%
\pgfpathlineto{\pgfqpoint{7.536359in}{1.530777in}}%
\pgfpathlineto{\pgfqpoint{7.545166in}{1.812128in}}%
\pgfpathlineto{\pgfqpoint{7.553972in}{1.839591in}}%
\pgfpathlineto{\pgfqpoint{7.562779in}{1.585675in}}%
\pgfpathlineto{\pgfqpoint{7.571586in}{1.661150in}}%
\pgfpathlineto{\pgfqpoint{7.580393in}{2.011173in}}%
\pgfpathlineto{\pgfqpoint{7.589200in}{2.251343in}}%
\pgfpathlineto{\pgfqpoint{7.598007in}{2.059213in}}%
\pgfpathlineto{\pgfqpoint{7.606813in}{1.928811in}}%
\pgfpathlineto{\pgfqpoint{7.615620in}{1.983710in}}%
\pgfpathlineto{\pgfqpoint{7.624427in}{2.251343in}}%
\pgfpathlineto{\pgfqpoint{7.633234in}{2.052326in}}%
\pgfpathlineto{\pgfqpoint{7.642041in}{1.969992in}}%
\pgfpathlineto{\pgfqpoint{7.650847in}{1.908207in}}%
\pgfpathlineto{\pgfqpoint{7.659654in}{1.915094in}}%
\pgfpathlineto{\pgfqpoint{7.668461in}{1.832732in}}%
\pgfpathlineto{\pgfqpoint{7.677268in}{1.812128in}}%
\pgfpathlineto{\pgfqpoint{7.686075in}{1.908207in}}%
\pgfpathlineto{\pgfqpoint{7.694882in}{1.777834in}}%
\pgfpathlineto{\pgfqpoint{7.712495in}{1.935670in}}%
\pgfpathlineto{\pgfqpoint{7.721302in}{1.743512in}}%
\pgfpathlineto{\pgfqpoint{7.730109in}{1.688613in}}%
\pgfpathlineto{\pgfqpoint{7.738916in}{1.819015in}}%
\pgfpathlineto{\pgfqpoint{7.756529in}{1.846450in}}%
\pgfpathlineto{\pgfqpoint{7.765336in}{2.155292in}}%
\pgfpathlineto{\pgfqpoint{7.774143in}{1.963133in}}%
\pgfpathlineto{\pgfqpoint{7.782950in}{1.921953in}}%
\pgfpathlineto{\pgfqpoint{7.791757in}{1.619997in}}%
\pgfpathlineto{\pgfqpoint{7.800563in}{1.578816in}}%
\pgfpathlineto{\pgfqpoint{7.809370in}{1.290579in}}%
\pgfpathlineto{\pgfqpoint{7.818177in}{1.263115in}}%
\pgfpathlineto{\pgfqpoint{7.826984in}{1.551353in}}%
\pgfpathlineto{\pgfqpoint{7.835791in}{1.523918in}}%
\pgfpathlineto{\pgfqpoint{7.844597in}{1.647432in}}%
\pgfpathlineto{\pgfqpoint{7.853404in}{1.832732in}}%
\pgfpathlineto{\pgfqpoint{7.862211in}{2.107252in}}%
\pgfpathlineto{\pgfqpoint{7.871018in}{2.271947in}}%
\pgfpathlineto{\pgfqpoint{7.879825in}{2.292524in}}%
\pgfpathlineto{\pgfqpoint{7.888632in}{2.409207in}}%
\pgfpathlineto{\pgfqpoint{7.897438in}{2.079789in}}%
\pgfpathlineto{\pgfqpoint{7.915052in}{1.626856in}}%
\pgfpathlineto{\pgfqpoint{7.923859in}{1.619997in}}%
\pgfpathlineto{\pgfqpoint{7.932666in}{1.633715in}}%
\pgfpathlineto{\pgfqpoint{7.941472in}{1.544494in}}%
\pgfpathlineto{\pgfqpoint{7.950279in}{1.578816in}}%
\pgfpathlineto{\pgfqpoint{7.959086in}{1.812128in}}%
\pgfpathlineto{\pgfqpoint{7.967893in}{1.709190in}}%
\pgfpathlineto{\pgfqpoint{7.976700in}{1.901348in}}%
\pgfpathlineto{\pgfqpoint{7.985507in}{1.695472in}}%
\pgfpathlineto{\pgfqpoint{7.994313in}{1.887630in}}%
\pgfpathlineto{\pgfqpoint{8.003120in}{1.709190in}}%
\pgfpathlineto{\pgfqpoint{8.011927in}{1.846450in}}%
\pgfpathlineto{\pgfqpoint{8.020734in}{1.921953in}}%
\pgfpathlineto{\pgfqpoint{8.029541in}{1.791551in}}%
\pgfpathlineto{\pgfqpoint{8.038347in}{1.626856in}}%
\pgfpathlineto{\pgfqpoint{8.047154in}{2.093507in}}%
\pgfpathlineto{\pgfqpoint{8.055961in}{2.223908in}}%
\pgfpathlineto{\pgfqpoint{8.064768in}{2.258230in}}%
\pgfpathlineto{\pgfqpoint{8.073575in}{2.086648in}}%
\pgfpathlineto{\pgfqpoint{8.082382in}{2.093507in}}%
\pgfpathlineto{\pgfqpoint{8.091188in}{1.935670in}}%
\pgfpathlineto{\pgfqpoint{8.099995in}{2.024891in}}%
\pgfpathlineto{\pgfqpoint{8.108802in}{2.306269in}}%
\pgfpathlineto{\pgfqpoint{8.117609in}{2.450388in}}%
\pgfpathlineto{\pgfqpoint{8.126416in}{2.175868in}}%
\pgfpathlineto{\pgfqpoint{8.135222in}{2.093507in}}%
\pgfpathlineto{\pgfqpoint{8.144029in}{2.024891in}}%
\pgfpathlineto{\pgfqpoint{8.152836in}{1.578816in}}%
\pgfpathlineto{\pgfqpoint{8.161643in}{1.770975in}}%
\pgfpathlineto{\pgfqpoint{8.170450in}{1.674896in}}%
\pgfpathlineto{\pgfqpoint{8.179257in}{1.915094in}}%
\pgfpathlineto{\pgfqpoint{8.188063in}{1.777834in}}%
\pgfpathlineto{\pgfqpoint{8.196870in}{1.935670in}}%
\pgfpathlineto{\pgfqpoint{8.214484in}{1.990569in}}%
\pgfpathlineto{\pgfqpoint{8.223291in}{1.805269in}}%
\pgfpathlineto{\pgfqpoint{8.232097in}{1.832732in}}%
\pgfpathlineto{\pgfqpoint{8.240904in}{1.894489in}}%
\pgfpathlineto{\pgfqpoint{8.249711in}{1.880772in}}%
\pgfpathlineto{\pgfqpoint{8.258518in}{1.812128in}}%
\pgfpathlineto{\pgfqpoint{8.267325in}{1.983710in}}%
\pgfpathlineto{\pgfqpoint{8.284938in}{1.640574in}}%
\pgfpathlineto{\pgfqpoint{8.293745in}{1.668037in}}%
\pgfpathlineto{\pgfqpoint{8.302552in}{1.819015in}}%
\pgfpathlineto{\pgfqpoint{8.311359in}{1.661150in}}%
\pgfpathlineto{\pgfqpoint{8.320166in}{1.640574in}}%
\pgfpathlineto{\pgfqpoint{8.328972in}{1.784692in}}%
\pgfpathlineto{\pgfqpoint{8.337779in}{1.578816in}}%
\pgfpathlineto{\pgfqpoint{8.355393in}{1.736653in}}%
\pgfpathlineto{\pgfqpoint{8.364200in}{1.695472in}}%
\pgfpathlineto{\pgfqpoint{8.373007in}{1.606252in}}%
\pgfpathlineto{\pgfqpoint{8.381813in}{1.366053in}}%
\pgfpathlineto{\pgfqpoint{8.390620in}{1.640574in}}%
\pgfpathlineto{\pgfqpoint{8.399427in}{2.004286in}}%
\pgfpathlineto{\pgfqpoint{8.408234in}{1.976851in}}%
\pgfpathlineto{\pgfqpoint{8.417041in}{1.716076in}}%
\pgfpathlineto{\pgfqpoint{8.425847in}{1.551353in}}%
\pgfpathlineto{\pgfqpoint{8.434654in}{1.455274in}}%
\pgfpathlineto{\pgfqpoint{8.443461in}{1.702331in}}%
\pgfpathlineto{\pgfqpoint{8.452268in}{1.839591in}}%
\pgfpathlineto{\pgfqpoint{8.461075in}{1.592534in}}%
\pgfpathlineto{\pgfqpoint{8.469882in}{2.079789in}}%
\pgfpathlineto{\pgfqpoint{8.478688in}{1.873913in}}%
\pgfpathlineto{\pgfqpoint{8.487495in}{2.079789in}}%
\pgfpathlineto{\pgfqpoint{8.496302in}{1.729794in}}%
\pgfpathlineto{\pgfqpoint{8.505109in}{1.750370in}}%
\pgfpathlineto{\pgfqpoint{8.522722in}{2.114111in}}%
\pgfpathlineto{\pgfqpoint{8.531529in}{2.251343in}}%
\pgfpathlineto{\pgfqpoint{8.540336in}{2.127829in}}%
\pgfpathlineto{\pgfqpoint{8.549143in}{2.285665in}}%
\pgfpathlineto{\pgfqpoint{8.557950in}{2.093507in}}%
\pgfpathlineto{\pgfqpoint{8.566757in}{2.498428in}}%
\pgfpathlineto{\pgfqpoint{8.575563in}{2.443501in}}%
\pgfpathlineto{\pgfqpoint{8.584370in}{2.285665in}}%
\pgfpathlineto{\pgfqpoint{8.593177in}{2.100365in}}%
\pgfpathlineto{\pgfqpoint{8.601984in}{2.011173in}}%
\pgfpathlineto{\pgfqpoint{8.610791in}{1.990569in}}%
\pgfpathlineto{\pgfqpoint{8.619597in}{2.120970in}}%
\pgfpathlineto{\pgfqpoint{8.628404in}{1.619997in}}%
\pgfpathlineto{\pgfqpoint{8.637211in}{1.668037in}}%
\pgfpathlineto{\pgfqpoint{8.646018in}{1.791551in}}%
\pgfpathlineto{\pgfqpoint{8.654825in}{1.798410in}}%
\pgfpathlineto{\pgfqpoint{8.663632in}{2.093507in}}%
\pgfpathlineto{\pgfqpoint{8.672438in}{2.072930in}}%
\pgfpathlineto{\pgfqpoint{8.681245in}{2.072930in}}%
\pgfpathlineto{\pgfqpoint{8.690052in}{2.388603in}}%
\pgfpathlineto{\pgfqpoint{8.698859in}{1.819015in}}%
\pgfpathlineto{\pgfqpoint{8.707666in}{1.393517in}}%
\pgfpathlineto{\pgfqpoint{8.716472in}{1.468992in}}%
\pgfpathlineto{\pgfqpoint{8.725279in}{1.523918in}}%
\pgfpathlineto{\pgfqpoint{8.734086in}{1.544494in}}%
\pgfpathlineto{\pgfqpoint{8.742893in}{1.805269in}}%
\pgfpathlineto{\pgfqpoint{8.751700in}{1.743512in}}%
\pgfpathlineto{\pgfqpoint{8.760507in}{1.825873in}}%
\pgfpathlineto{\pgfqpoint{8.769313in}{1.640574in}}%
\pgfpathlineto{\pgfqpoint{8.778120in}{1.517031in}}%
\pgfpathlineto{\pgfqpoint{8.786927in}{1.784692in}}%
\pgfpathlineto{\pgfqpoint{8.795734in}{1.901348in}}%
\pgfpathlineto{\pgfqpoint{8.804541in}{1.853308in}}%
\pgfpathlineto{\pgfqpoint{8.813347in}{1.791551in}}%
\pgfpathlineto{\pgfqpoint{8.822154in}{1.709190in}}%
\pgfpathlineto{\pgfqpoint{8.830961in}{1.860167in}}%
\pgfpathlineto{\pgfqpoint{8.839768in}{1.551353in}}%
\pgfpathlineto{\pgfqpoint{8.848575in}{1.969992in}}%
\pgfpathlineto{\pgfqpoint{8.857382in}{2.306269in}}%
\pgfpathlineto{\pgfqpoint{8.866188in}{2.402349in}}%
\pgfpathlineto{\pgfqpoint{8.874995in}{2.532722in}}%
\pgfpathlineto{\pgfqpoint{8.883802in}{2.498428in}}%
\pgfpathlineto{\pgfqpoint{8.892609in}{2.498428in}}%
\pgfpathlineto{\pgfqpoint{8.901416in}{2.278806in}}%
\pgfpathlineto{\pgfqpoint{8.910222in}{2.134687in}}%
\pgfpathlineto{\pgfqpoint{8.919029in}{2.196445in}}%
\pgfpathlineto{\pgfqpoint{8.927836in}{1.908207in}}%
\pgfpathlineto{\pgfqpoint{8.936643in}{1.819015in}}%
\pgfpathlineto{\pgfqpoint{8.945450in}{1.825873in}}%
\pgfpathlineto{\pgfqpoint{8.954257in}{1.997427in}}%
\pgfpathlineto{\pgfqpoint{8.963063in}{2.066071in}}%
\pgfpathlineto{\pgfqpoint{8.971870in}{1.976851in}}%
\pgfpathlineto{\pgfqpoint{8.980677in}{2.093507in}}%
\pgfpathlineto{\pgfqpoint{8.989484in}{1.798410in}}%
\pgfpathlineto{\pgfqpoint{8.998291in}{2.169009in}}%
\pgfpathlineto{\pgfqpoint{9.007097in}{1.784692in}}%
\pgfpathlineto{\pgfqpoint{9.015904in}{1.791551in}}%
\pgfpathlineto{\pgfqpoint{9.024711in}{2.011173in}}%
\pgfpathlineto{\pgfqpoint{9.033518in}{1.880772in}}%
\pgfpathlineto{\pgfqpoint{9.042325in}{2.127829in}}%
\pgfpathlineto{\pgfqpoint{9.051132in}{2.024891in}}%
\pgfpathlineto{\pgfqpoint{9.068745in}{1.908207in}}%
\pgfpathlineto{\pgfqpoint{9.077552in}{1.901348in}}%
\pgfpathlineto{\pgfqpoint{9.086359in}{1.935670in}}%
\pgfpathlineto{\pgfqpoint{9.095166in}{1.544494in}}%
\pgfpathlineto{\pgfqpoint{9.103972in}{1.269974in}}%
\pgfpathlineto{\pgfqpoint{9.112779in}{1.695472in}}%
\pgfpathlineto{\pgfqpoint{9.121586in}{1.503314in}}%
\pgfpathlineto{\pgfqpoint{9.139200in}{1.750370in}}%
\pgfpathlineto{\pgfqpoint{9.148007in}{1.565071in}}%
\pgfpathlineto{\pgfqpoint{9.156813in}{2.004286in}}%
\pgfpathlineto{\pgfqpoint{9.165620in}{2.093507in}}%
\pgfpathlineto{\pgfqpoint{9.174427in}{1.915094in}}%
\pgfpathlineto{\pgfqpoint{9.183234in}{2.004286in}}%
\pgfpathlineto{\pgfqpoint{9.192041in}{2.182727in}}%
\pgfpathlineto{\pgfqpoint{9.200847in}{2.333705in}}%
\pgfpathlineto{\pgfqpoint{9.218461in}{2.031749in}}%
\pgfpathlineto{\pgfqpoint{9.227268in}{2.361168in}}%
\pgfpathlineto{\pgfqpoint{9.236075in}{2.278806in}}%
\pgfpathlineto{\pgfqpoint{9.244882in}{2.127829in}}%
\pgfpathlineto{\pgfqpoint{9.253688in}{2.086648in}}%
\pgfpathlineto{\pgfqpoint{9.262495in}{2.072930in}}%
\pgfpathlineto{\pgfqpoint{9.271302in}{2.031749in}}%
\pgfpathlineto{\pgfqpoint{9.280109in}{1.770975in}}%
\pgfpathlineto{\pgfqpoint{9.297722in}{2.278806in}}%
\pgfpathlineto{\pgfqpoint{9.306529in}{2.457247in}}%
\pgfpathlineto{\pgfqpoint{9.315336in}{2.436643in}}%
\pgfpathlineto{\pgfqpoint{9.332950in}{2.148405in}}%
\pgfpathlineto{\pgfqpoint{9.341757in}{2.148405in}}%
\pgfpathlineto{\pgfqpoint{9.350563in}{1.949388in}}%
\pgfpathlineto{\pgfqpoint{9.359370in}{1.956246in}}%
\pgfpathlineto{\pgfqpoint{9.368177in}{1.640574in}}%
\pgfpathlineto{\pgfqpoint{9.376984in}{1.695472in}}%
\pgfpathlineto{\pgfqpoint{9.385791in}{1.860167in}}%
\pgfpathlineto{\pgfqpoint{9.394597in}{1.990569in}}%
\pgfpathlineto{\pgfqpoint{9.403404in}{1.956246in}}%
\pgfpathlineto{\pgfqpoint{9.412211in}{1.613110in}}%
\pgfpathlineto{\pgfqpoint{9.421018in}{1.681754in}}%
\pgfpathlineto{\pgfqpoint{9.429825in}{1.599393in}}%
\pgfpathlineto{\pgfqpoint{9.438632in}{1.702331in}}%
\pgfpathlineto{\pgfqpoint{9.447438in}{1.770975in}}%
\pgfpathlineto{\pgfqpoint{9.456245in}{1.551353in}}%
\pgfpathlineto{\pgfqpoint{9.465052in}{1.777834in}}%
\pgfpathlineto{\pgfqpoint{9.473859in}{1.832732in}}%
\pgfpathlineto{\pgfqpoint{9.482666in}{2.155292in}}%
\pgfpathlineto{\pgfqpoint{9.491472in}{2.024891in}}%
\pgfpathlineto{\pgfqpoint{9.500279in}{1.798410in}}%
\pgfpathlineto{\pgfqpoint{9.509086in}{1.757229in}}%
\pgfpathlineto{\pgfqpoint{9.517893in}{1.654291in}}%
\pgfpathlineto{\pgfqpoint{9.526700in}{1.990569in}}%
\pgfpathlineto{\pgfqpoint{9.535507in}{1.757229in}}%
\pgfpathlineto{\pgfqpoint{9.544313in}{1.688613in}}%
\pgfpathlineto{\pgfqpoint{9.553120in}{1.757229in}}%
\pgfpathlineto{\pgfqpoint{9.561927in}{1.777834in}}%
\pgfpathlineto{\pgfqpoint{9.570734in}{1.873913in}}%
\pgfpathlineto{\pgfqpoint{9.579541in}{1.695472in}}%
\pgfpathlineto{\pgfqpoint{9.588347in}{1.633715in}}%
\pgfpathlineto{\pgfqpoint{9.597154in}{1.613110in}}%
\pgfpathlineto{\pgfqpoint{9.605961in}{1.915094in}}%
\pgfpathlineto{\pgfqpoint{9.614768in}{1.949388in}}%
\pgfpathlineto{\pgfqpoint{9.623575in}{1.928811in}}%
\pgfpathlineto{\pgfqpoint{9.632382in}{2.347422in}}%
\pgfpathlineto{\pgfqpoint{9.658802in}{2.175868in}}%
\pgfpathlineto{\pgfqpoint{9.667609in}{1.983710in}}%
\pgfpathlineto{\pgfqpoint{9.676416in}{2.066071in}}%
\pgfpathlineto{\pgfqpoint{9.685222in}{2.031749in}}%
\pgfpathlineto{\pgfqpoint{9.694029in}{1.908207in}}%
\pgfpathlineto{\pgfqpoint{9.702836in}{2.271947in}}%
\pgfpathlineto{\pgfqpoint{9.711643in}{2.477823in}}%
\pgfpathlineto{\pgfqpoint{9.720450in}{2.333705in}}%
\pgfpathlineto{\pgfqpoint{9.729257in}{2.409207in}}%
\pgfpathlineto{\pgfqpoint{9.738063in}{2.210190in}}%
\pgfpathlineto{\pgfqpoint{9.746870in}{2.052326in}}%
\pgfpathlineto{\pgfqpoint{9.755677in}{2.018032in}}%
\pgfpathlineto{\pgfqpoint{9.764484in}{2.319987in}}%
\pgfpathlineto{\pgfqpoint{9.773291in}{2.066071in}}%
\pgfpathlineto{\pgfqpoint{9.782097in}{1.983710in}}%
\pgfpathlineto{\pgfqpoint{9.790904in}{2.127829in}}%
\pgfpathlineto{\pgfqpoint{9.799711in}{1.860167in}}%
\pgfpathlineto{\pgfqpoint{9.808518in}{1.935670in}}%
\pgfpathlineto{\pgfqpoint{9.817325in}{1.908207in}}%
\pgfpathlineto{\pgfqpoint{9.826132in}{1.976851in}}%
\pgfpathlineto{\pgfqpoint{9.834938in}{2.086648in}}%
\pgfpathlineto{\pgfqpoint{9.843745in}{2.018032in}}%
\pgfpathlineto{\pgfqpoint{9.852552in}{2.169009in}}%
\pgfpathlineto{\pgfqpoint{9.861359in}{2.361168in}}%
\pgfpathlineto{\pgfqpoint{9.870166in}{2.326846in}}%
\pgfpathlineto{\pgfqpoint{9.878972in}{2.388603in}}%
\pgfpathlineto{\pgfqpoint{9.887779in}{1.784692in}}%
\pgfpathlineto{\pgfqpoint{9.896586in}{2.059213in}}%
\pgfpathlineto{\pgfqpoint{9.905393in}{2.066071in}}%
\pgfpathlineto{\pgfqpoint{9.914200in}{1.942529in}}%
\pgfpathlineto{\pgfqpoint{9.923007in}{2.107252in}}%
\pgfpathlineto{\pgfqpoint{9.931813in}{1.969992in}}%
\pgfpathlineto{\pgfqpoint{9.940620in}{2.120970in}}%
\pgfpathlineto{\pgfqpoint{9.949427in}{1.894489in}}%
\pgfpathlineto{\pgfqpoint{9.949427in}{1.894489in}}%
\pgfusepath{stroke}%
\end{pgfscope}%
\begin{pgfscope}%
\pgfpathrectangle{\pgfqpoint{0.702268in}{0.521603in}}{\pgfqpoint{9.687500in}{4.235000in}}%
\pgfusepath{clip}%
\pgfsetrectcap%
\pgfsetroundjoin%
\pgfsetlinewidth{0.501875pt}%
\definecolor{currentstroke}{rgb}{0.501961,0.501961,0.501961}%
\pgfsetstrokecolor{currentstroke}%
\pgfsetstrokeopacity{0.250000}%
\pgfsetdash{}{0pt}%
\pgfpathmoveto{\pgfqpoint{1.142609in}{4.083707in}}%
\pgfpathlineto{\pgfqpoint{1.151416in}{3.116056in}}%
\pgfpathlineto{\pgfqpoint{1.160222in}{2.724880in}}%
\pgfpathlineto{\pgfqpoint{1.169029in}{2.615084in}}%
\pgfpathlineto{\pgfqpoint{1.177836in}{2.141546in}}%
\pgfpathlineto{\pgfqpoint{1.186643in}{2.045467in}}%
\pgfpathlineto{\pgfqpoint{1.195450in}{1.805269in}}%
\pgfpathlineto{\pgfqpoint{1.204257in}{2.011173in}}%
\pgfpathlineto{\pgfqpoint{1.213063in}{2.086648in}}%
\pgfpathlineto{\pgfqpoint{1.221870in}{1.969992in}}%
\pgfpathlineto{\pgfqpoint{1.230677in}{1.942529in}}%
\pgfpathlineto{\pgfqpoint{1.239484in}{1.770975in}}%
\pgfpathlineto{\pgfqpoint{1.248291in}{2.038608in}}%
\pgfpathlineto{\pgfqpoint{1.257097in}{2.237625in}}%
\pgfpathlineto{\pgfqpoint{1.265904in}{2.244484in}}%
\pgfpathlineto{\pgfqpoint{1.274711in}{2.388603in}}%
\pgfpathlineto{\pgfqpoint{1.283518in}{2.258230in}}%
\pgfpathlineto{\pgfqpoint{1.292325in}{2.313128in}}%
\pgfpathlineto{\pgfqpoint{1.301132in}{2.052326in}}%
\pgfpathlineto{\pgfqpoint{1.309938in}{2.285665in}}%
\pgfpathlineto{\pgfqpoint{1.318745in}{2.141546in}}%
\pgfpathlineto{\pgfqpoint{1.327552in}{2.519004in}}%
\pgfpathlineto{\pgfqpoint{1.336359in}{2.223908in}}%
\pgfpathlineto{\pgfqpoint{1.345166in}{2.223908in}}%
\pgfpathlineto{\pgfqpoint{1.353972in}{2.354309in}}%
\pgfpathlineto{\pgfqpoint{1.362779in}{2.560185in}}%
\pgfpathlineto{\pgfqpoint{1.371586in}{2.457247in}}%
\pgfpathlineto{\pgfqpoint{1.389200in}{2.086648in}}%
\pgfpathlineto{\pgfqpoint{1.398007in}{2.223908in}}%
\pgfpathlineto{\pgfqpoint{1.415620in}{2.031749in}}%
\pgfpathlineto{\pgfqpoint{1.424427in}{1.915094in}}%
\pgfpathlineto{\pgfqpoint{1.433234in}{1.750370in}}%
\pgfpathlineto{\pgfqpoint{1.442041in}{1.949388in}}%
\pgfpathlineto{\pgfqpoint{1.450847in}{1.915094in}}%
\pgfpathlineto{\pgfqpoint{1.459654in}{1.969992in}}%
\pgfpathlineto{\pgfqpoint{1.468461in}{1.757229in}}%
\pgfpathlineto{\pgfqpoint{1.477268in}{1.983710in}}%
\pgfpathlineto{\pgfqpoint{1.486075in}{1.743512in}}%
\pgfpathlineto{\pgfqpoint{1.494882in}{1.770975in}}%
\pgfpathlineto{\pgfqpoint{1.503688in}{2.162151in}}%
\pgfpathlineto{\pgfqpoint{1.512495in}{2.141546in}}%
\pgfpathlineto{\pgfqpoint{1.521302in}{2.217049in}}%
\pgfpathlineto{\pgfqpoint{1.530109in}{2.024891in}}%
\pgfpathlineto{\pgfqpoint{1.547722in}{2.120970in}}%
\pgfpathlineto{\pgfqpoint{1.556529in}{2.223908in}}%
\pgfpathlineto{\pgfqpoint{1.565336in}{1.812128in}}%
\pgfpathlineto{\pgfqpoint{1.574143in}{1.764116in}}%
\pgfpathlineto{\pgfqpoint{1.582950in}{2.066071in}}%
\pgfpathlineto{\pgfqpoint{1.591757in}{2.313128in}}%
\pgfpathlineto{\pgfqpoint{1.600563in}{2.258230in}}%
\pgfpathlineto{\pgfqpoint{1.609370in}{2.217049in}}%
\pgfpathlineto{\pgfqpoint{1.618177in}{2.072930in}}%
\pgfpathlineto{\pgfqpoint{1.626984in}{1.997427in}}%
\pgfpathlineto{\pgfqpoint{1.635791in}{1.901348in}}%
\pgfpathlineto{\pgfqpoint{1.644597in}{1.894489in}}%
\pgfpathlineto{\pgfqpoint{1.653404in}{1.736653in}}%
\pgfpathlineto{\pgfqpoint{1.662211in}{1.908207in}}%
\pgfpathlineto{\pgfqpoint{1.671018in}{2.196445in}}%
\pgfpathlineto{\pgfqpoint{1.679825in}{2.244484in}}%
\pgfpathlineto{\pgfqpoint{1.688632in}{2.045467in}}%
\pgfpathlineto{\pgfqpoint{1.697438in}{2.052326in}}%
\pgfpathlineto{\pgfqpoint{1.706245in}{2.251343in}}%
\pgfpathlineto{\pgfqpoint{1.715052in}{2.326846in}}%
\pgfpathlineto{\pgfqpoint{1.723859in}{2.237625in}}%
\pgfpathlineto{\pgfqpoint{1.732666in}{2.299383in}}%
\pgfpathlineto{\pgfqpoint{1.741472in}{2.395462in}}%
\pgfpathlineto{\pgfqpoint{1.750279in}{2.306269in}}%
\pgfpathlineto{\pgfqpoint{1.759086in}{2.361168in}}%
\pgfpathlineto{\pgfqpoint{1.767893in}{2.169009in}}%
\pgfpathlineto{\pgfqpoint{1.785507in}{1.654291in}}%
\pgfpathlineto{\pgfqpoint{1.803120in}{1.489596in}}%
\pgfpathlineto{\pgfqpoint{1.811927in}{1.668037in}}%
\pgfpathlineto{\pgfqpoint{1.820734in}{1.647432in}}%
\pgfpathlineto{\pgfqpoint{1.829541in}{1.963133in}}%
\pgfpathlineto{\pgfqpoint{1.838347in}{1.949388in}}%
\pgfpathlineto{\pgfqpoint{1.847154in}{1.928811in}}%
\pgfpathlineto{\pgfqpoint{1.855961in}{2.120970in}}%
\pgfpathlineto{\pgfqpoint{1.864768in}{2.011173in}}%
\pgfpathlineto{\pgfqpoint{1.873575in}{1.654291in}}%
\pgfpathlineto{\pgfqpoint{1.882382in}{1.496455in}}%
\pgfpathlineto{\pgfqpoint{1.891188in}{1.716076in}}%
\pgfpathlineto{\pgfqpoint{1.899995in}{1.819015in}}%
\pgfpathlineto{\pgfqpoint{1.908802in}{1.798410in}}%
\pgfpathlineto{\pgfqpoint{1.917609in}{1.722935in}}%
\pgfpathlineto{\pgfqpoint{1.926416in}{1.619997in}}%
\pgfpathlineto{\pgfqpoint{1.935222in}{1.599393in}}%
\pgfpathlineto{\pgfqpoint{1.944029in}{1.839591in}}%
\pgfpathlineto{\pgfqpoint{1.952836in}{1.764116in}}%
\pgfpathlineto{\pgfqpoint{1.961643in}{2.203303in}}%
\pgfpathlineto{\pgfqpoint{1.970450in}{2.354309in}}%
\pgfpathlineto{\pgfqpoint{1.979257in}{2.477823in}}%
\pgfpathlineto{\pgfqpoint{1.988063in}{2.443501in}}%
\pgfpathlineto{\pgfqpoint{1.996870in}{2.251343in}}%
\pgfpathlineto{\pgfqpoint{2.005677in}{2.182727in}}%
\pgfpathlineto{\pgfqpoint{2.014484in}{2.038608in}}%
\pgfpathlineto{\pgfqpoint{2.023291in}{2.278806in}}%
\pgfpathlineto{\pgfqpoint{2.032097in}{2.011173in}}%
\pgfpathlineto{\pgfqpoint{2.040904in}{1.983710in}}%
\pgfpathlineto{\pgfqpoint{2.049711in}{1.894489in}}%
\pgfpathlineto{\pgfqpoint{2.058518in}{1.654291in}}%
\pgfpathlineto{\pgfqpoint{2.067325in}{2.038608in}}%
\pgfpathlineto{\pgfqpoint{2.076132in}{2.127829in}}%
\pgfpathlineto{\pgfqpoint{2.084938in}{2.182727in}}%
\pgfpathlineto{\pgfqpoint{2.093745in}{2.210190in}}%
\pgfpathlineto{\pgfqpoint{2.102552in}{2.072930in}}%
\pgfpathlineto{\pgfqpoint{2.111359in}{1.770975in}}%
\pgfpathlineto{\pgfqpoint{2.120166in}{2.052326in}}%
\pgfpathlineto{\pgfqpoint{2.128972in}{2.470965in}}%
\pgfpathlineto{\pgfqpoint{2.137779in}{2.011173in}}%
\pgfpathlineto{\pgfqpoint{2.146586in}{1.777834in}}%
\pgfpathlineto{\pgfqpoint{2.155393in}{1.839591in}}%
\pgfpathlineto{\pgfqpoint{2.164200in}{1.983710in}}%
\pgfpathlineto{\pgfqpoint{2.173007in}{1.928811in}}%
\pgfpathlineto{\pgfqpoint{2.181813in}{2.148405in}}%
\pgfpathlineto{\pgfqpoint{2.190620in}{1.551353in}}%
\pgfpathlineto{\pgfqpoint{2.199427in}{1.626856in}}%
\pgfpathlineto{\pgfqpoint{2.208234in}{1.791551in}}%
\pgfpathlineto{\pgfqpoint{2.217041in}{1.619997in}}%
\pgfpathlineto{\pgfqpoint{2.225847in}{1.379799in}}%
\pgfpathlineto{\pgfqpoint{2.234654in}{1.173895in}}%
\pgfpathlineto{\pgfqpoint{2.252268in}{1.921953in}}%
\pgfpathlineto{\pgfqpoint{2.261075in}{2.237625in}}%
\pgfpathlineto{\pgfqpoint{2.269882in}{2.217049in}}%
\pgfpathlineto{\pgfqpoint{2.278688in}{1.963133in}}%
\pgfpathlineto{\pgfqpoint{2.287495in}{2.004286in}}%
\pgfpathlineto{\pgfqpoint{2.296302in}{2.120970in}}%
\pgfpathlineto{\pgfqpoint{2.305109in}{1.942529in}}%
\pgfpathlineto{\pgfqpoint{2.313916in}{2.141546in}}%
\pgfpathlineto{\pgfqpoint{2.322722in}{2.155292in}}%
\pgfpathlineto{\pgfqpoint{2.331529in}{2.148405in}}%
\pgfpathlineto{\pgfqpoint{2.340336in}{2.258230in}}%
\pgfpathlineto{\pgfqpoint{2.349143in}{2.038608in}}%
\pgfpathlineto{\pgfqpoint{2.357950in}{2.127829in}}%
\pgfpathlineto{\pgfqpoint{2.366757in}{2.395462in}}%
\pgfpathlineto{\pgfqpoint{2.375563in}{2.230767in}}%
\pgfpathlineto{\pgfqpoint{2.384370in}{2.175868in}}%
\pgfpathlineto{\pgfqpoint{2.393177in}{1.722935in}}%
\pgfpathlineto{\pgfqpoint{2.410791in}{1.853308in}}%
\pgfpathlineto{\pgfqpoint{2.419597in}{1.928811in}}%
\pgfpathlineto{\pgfqpoint{2.428404in}{2.052326in}}%
\pgfpathlineto{\pgfqpoint{2.437211in}{1.908207in}}%
\pgfpathlineto{\pgfqpoint{2.446018in}{1.846450in}}%
\pgfpathlineto{\pgfqpoint{2.454825in}{2.436643in}}%
\pgfpathlineto{\pgfqpoint{2.472438in}{1.846450in}}%
\pgfpathlineto{\pgfqpoint{2.481245in}{1.860167in}}%
\pgfpathlineto{\pgfqpoint{2.490052in}{1.791551in}}%
\pgfpathlineto{\pgfqpoint{2.498859in}{2.079789in}}%
\pgfpathlineto{\pgfqpoint{2.507666in}{2.120970in}}%
\pgfpathlineto{\pgfqpoint{2.516472in}{1.949388in}}%
\pgfpathlineto{\pgfqpoint{2.525279in}{1.921953in}}%
\pgfpathlineto{\pgfqpoint{2.534086in}{1.551353in}}%
\pgfpathlineto{\pgfqpoint{2.542893in}{1.736653in}}%
\pgfpathlineto{\pgfqpoint{2.551700in}{1.798410in}}%
\pgfpathlineto{\pgfqpoint{2.560507in}{1.798410in}}%
\pgfpathlineto{\pgfqpoint{2.569313in}{1.722935in}}%
\pgfpathlineto{\pgfqpoint{2.578120in}{1.990569in}}%
\pgfpathlineto{\pgfqpoint{2.586927in}{1.805269in}}%
\pgfpathlineto{\pgfqpoint{2.595734in}{1.873913in}}%
\pgfpathlineto{\pgfqpoint{2.604541in}{1.894489in}}%
\pgfpathlineto{\pgfqpoint{2.622154in}{1.654291in}}%
\pgfpathlineto{\pgfqpoint{2.630961in}{1.819015in}}%
\pgfpathlineto{\pgfqpoint{2.639768in}{1.702331in}}%
\pgfpathlineto{\pgfqpoint{2.648575in}{1.674896in}}%
\pgfpathlineto{\pgfqpoint{2.657382in}{1.661150in}}%
\pgfpathlineto{\pgfqpoint{2.666188in}{1.496455in}}%
\pgfpathlineto{\pgfqpoint{2.683802in}{1.400376in}}%
\pgfpathlineto{\pgfqpoint{2.692609in}{1.606252in}}%
\pgfpathlineto{\pgfqpoint{2.701416in}{1.709190in}}%
\pgfpathlineto{\pgfqpoint{2.710222in}{2.024891in}}%
\pgfpathlineto{\pgfqpoint{2.719029in}{1.647432in}}%
\pgfpathlineto{\pgfqpoint{2.727836in}{1.558212in}}%
\pgfpathlineto{\pgfqpoint{2.736643in}{1.558212in}}%
\pgfpathlineto{\pgfqpoint{2.745450in}{1.695472in}}%
\pgfpathlineto{\pgfqpoint{2.754257in}{1.716076in}}%
\pgfpathlineto{\pgfqpoint{2.763063in}{1.777834in}}%
\pgfpathlineto{\pgfqpoint{2.771870in}{1.894489in}}%
\pgfpathlineto{\pgfqpoint{2.780677in}{2.251343in}}%
\pgfpathlineto{\pgfqpoint{2.789484in}{2.244484in}}%
\pgfpathlineto{\pgfqpoint{2.798291in}{1.983710in}}%
\pgfpathlineto{\pgfqpoint{2.807097in}{2.024891in}}%
\pgfpathlineto{\pgfqpoint{2.815904in}{1.825873in}}%
\pgfpathlineto{\pgfqpoint{2.824711in}{1.791551in}}%
\pgfpathlineto{\pgfqpoint{2.833518in}{1.517031in}}%
\pgfpathlineto{\pgfqpoint{2.842325in}{1.819015in}}%
\pgfpathlineto{\pgfqpoint{2.851132in}{1.784692in}}%
\pgfpathlineto{\pgfqpoint{2.859938in}{1.578816in}}%
\pgfpathlineto{\pgfqpoint{2.868745in}{1.510172in}}%
\pgfpathlineto{\pgfqpoint{2.877552in}{1.475878in}}%
\pgfpathlineto{\pgfqpoint{2.886359in}{1.681754in}}%
\pgfpathlineto{\pgfqpoint{2.895166in}{1.674896in}}%
\pgfpathlineto{\pgfqpoint{2.903972in}{1.921953in}}%
\pgfpathlineto{\pgfqpoint{2.912779in}{1.770975in}}%
\pgfpathlineto{\pgfqpoint{2.921586in}{1.921953in}}%
\pgfpathlineto{\pgfqpoint{2.930393in}{1.935670in}}%
\pgfpathlineto{\pgfqpoint{2.939200in}{1.709190in}}%
\pgfpathlineto{\pgfqpoint{2.948007in}{1.901348in}}%
\pgfpathlineto{\pgfqpoint{2.956813in}{2.155292in}}%
\pgfpathlineto{\pgfqpoint{2.965620in}{1.764116in}}%
\pgfpathlineto{\pgfqpoint{2.974427in}{2.011173in}}%
\pgfpathlineto{\pgfqpoint{2.983234in}{2.107252in}}%
\pgfpathlineto{\pgfqpoint{2.992041in}{2.004286in}}%
\pgfpathlineto{\pgfqpoint{3.000847in}{1.592534in}}%
\pgfpathlineto{\pgfqpoint{3.009654in}{1.832732in}}%
\pgfpathlineto{\pgfqpoint{3.018461in}{1.571958in}}%
\pgfpathlineto{\pgfqpoint{3.027268in}{1.743512in}}%
\pgfpathlineto{\pgfqpoint{3.036075in}{1.393517in}}%
\pgfpathlineto{\pgfqpoint{3.044882in}{1.427839in}}%
\pgfpathlineto{\pgfqpoint{3.053688in}{1.551353in}}%
\pgfpathlineto{\pgfqpoint{3.062495in}{1.764116in}}%
\pgfpathlineto{\pgfqpoint{3.071302in}{1.743512in}}%
\pgfpathlineto{\pgfqpoint{3.080109in}{2.024891in}}%
\pgfpathlineto{\pgfqpoint{3.088916in}{1.791551in}}%
\pgfpathlineto{\pgfqpoint{3.097722in}{1.887630in}}%
\pgfpathlineto{\pgfqpoint{3.106529in}{2.457247in}}%
\pgfpathlineto{\pgfqpoint{3.115336in}{2.436643in}}%
\pgfpathlineto{\pgfqpoint{3.124143in}{2.450388in}}%
\pgfpathlineto{\pgfqpoint{3.132950in}{2.278806in}}%
\pgfpathlineto{\pgfqpoint{3.141757in}{2.340563in}}%
\pgfpathlineto{\pgfqpoint{3.150563in}{2.045467in}}%
\pgfpathlineto{\pgfqpoint{3.159370in}{2.299383in}}%
\pgfpathlineto{\pgfqpoint{3.168177in}{1.873913in}}%
\pgfpathlineto{\pgfqpoint{3.176984in}{1.867054in}}%
\pgfpathlineto{\pgfqpoint{3.185791in}{2.045467in}}%
\pgfpathlineto{\pgfqpoint{3.194597in}{1.976851in}}%
\pgfpathlineto{\pgfqpoint{3.203404in}{1.921953in}}%
\pgfpathlineto{\pgfqpoint{3.212211in}{1.654291in}}%
\pgfpathlineto{\pgfqpoint{3.221018in}{1.825873in}}%
\pgfpathlineto{\pgfqpoint{3.229825in}{2.244484in}}%
\pgfpathlineto{\pgfqpoint{3.238632in}{2.100365in}}%
\pgfpathlineto{\pgfqpoint{3.247438in}{2.169009in}}%
\pgfpathlineto{\pgfqpoint{3.256245in}{1.537636in}}%
\pgfpathlineto{\pgfqpoint{3.265052in}{1.798410in}}%
\pgfpathlineto{\pgfqpoint{3.273859in}{1.551353in}}%
\pgfpathlineto{\pgfqpoint{3.282666in}{1.681754in}}%
\pgfpathlineto{\pgfqpoint{3.291472in}{1.668037in}}%
\pgfpathlineto{\pgfqpoint{3.300279in}{1.592534in}}%
\pgfpathlineto{\pgfqpoint{3.309086in}{1.613110in}}%
\pgfpathlineto{\pgfqpoint{3.317893in}{1.441556in}}%
\pgfpathlineto{\pgfqpoint{3.326700in}{1.475878in}}%
\pgfpathlineto{\pgfqpoint{3.335507in}{1.757229in}}%
\pgfpathlineto{\pgfqpoint{3.353120in}{2.004286in}}%
\pgfpathlineto{\pgfqpoint{3.361927in}{2.189586in}}%
\pgfpathlineto{\pgfqpoint{3.370734in}{2.024891in}}%
\pgfpathlineto{\pgfqpoint{3.379541in}{2.169009in}}%
\pgfpathlineto{\pgfqpoint{3.388347in}{1.812128in}}%
\pgfpathlineto{\pgfqpoint{3.397154in}{1.894489in}}%
\pgfpathlineto{\pgfqpoint{3.405961in}{1.736653in}}%
\pgfpathlineto{\pgfqpoint{3.414768in}{1.743512in}}%
\pgfpathlineto{\pgfqpoint{3.423575in}{2.004286in}}%
\pgfpathlineto{\pgfqpoint{3.432382in}{2.004286in}}%
\pgfpathlineto{\pgfqpoint{3.441188in}{2.210190in}}%
\pgfpathlineto{\pgfqpoint{3.449995in}{2.182727in}}%
\pgfpathlineto{\pgfqpoint{3.458802in}{2.169009in}}%
\pgfpathlineto{\pgfqpoint{3.476416in}{2.340563in}}%
\pgfpathlineto{\pgfqpoint{3.485222in}{1.908207in}}%
\pgfpathlineto{\pgfqpoint{3.494029in}{2.031749in}}%
\pgfpathlineto{\pgfqpoint{3.502836in}{1.722935in}}%
\pgfpathlineto{\pgfqpoint{3.511643in}{1.592534in}}%
\pgfpathlineto{\pgfqpoint{3.520450in}{1.578816in}}%
\pgfpathlineto{\pgfqpoint{3.529257in}{1.784692in}}%
\pgfpathlineto{\pgfqpoint{3.538063in}{1.661150in}}%
\pgfpathlineto{\pgfqpoint{3.546870in}{1.860167in}}%
\pgfpathlineto{\pgfqpoint{3.555677in}{1.867054in}}%
\pgfpathlineto{\pgfqpoint{3.564484in}{1.990569in}}%
\pgfpathlineto{\pgfqpoint{3.573291in}{1.647432in}}%
\pgfpathlineto{\pgfqpoint{3.590904in}{1.619997in}}%
\pgfpathlineto{\pgfqpoint{3.599711in}{1.489596in}}%
\pgfpathlineto{\pgfqpoint{3.608518in}{1.448415in}}%
\pgfpathlineto{\pgfqpoint{3.617325in}{1.681754in}}%
\pgfpathlineto{\pgfqpoint{3.626132in}{2.072930in}}%
\pgfpathlineto{\pgfqpoint{3.634938in}{2.059213in}}%
\pgfpathlineto{\pgfqpoint{3.643745in}{2.059213in}}%
\pgfpathlineto{\pgfqpoint{3.652552in}{2.340563in}}%
\pgfpathlineto{\pgfqpoint{3.661359in}{2.045467in}}%
\pgfpathlineto{\pgfqpoint{3.670166in}{2.306269in}}%
\pgfpathlineto{\pgfqpoint{3.678972in}{2.285665in}}%
\pgfpathlineto{\pgfqpoint{3.687779in}{2.024891in}}%
\pgfpathlineto{\pgfqpoint{3.696586in}{2.395462in}}%
\pgfpathlineto{\pgfqpoint{3.714200in}{2.031749in}}%
\pgfpathlineto{\pgfqpoint{3.723007in}{2.162151in}}%
\pgfpathlineto{\pgfqpoint{3.731813in}{2.011173in}}%
\pgfpathlineto{\pgfqpoint{3.740620in}{2.326846in}}%
\pgfpathlineto{\pgfqpoint{3.749427in}{2.587620in}}%
\pgfpathlineto{\pgfqpoint{3.758234in}{2.148405in}}%
\pgfpathlineto{\pgfqpoint{3.775847in}{1.668037in}}%
\pgfpathlineto{\pgfqpoint{3.784654in}{2.100365in}}%
\pgfpathlineto{\pgfqpoint{3.793461in}{2.120970in}}%
\pgfpathlineto{\pgfqpoint{3.802268in}{2.052326in}}%
\pgfpathlineto{\pgfqpoint{3.811075in}{2.368027in}}%
\pgfpathlineto{\pgfqpoint{3.819882in}{2.134687in}}%
\pgfpathlineto{\pgfqpoint{3.828688in}{2.004286in}}%
\pgfpathlineto{\pgfqpoint{3.837495in}{1.915094in}}%
\pgfpathlineto{\pgfqpoint{3.846302in}{1.846450in}}%
\pgfpathlineto{\pgfqpoint{3.855109in}{2.018032in}}%
\pgfpathlineto{\pgfqpoint{3.863916in}{2.114111in}}%
\pgfpathlineto{\pgfqpoint{3.872722in}{1.777834in}}%
\pgfpathlineto{\pgfqpoint{3.881529in}{1.798410in}}%
\pgfpathlineto{\pgfqpoint{3.890336in}{1.757229in}}%
\pgfpathlineto{\pgfqpoint{3.899143in}{1.990569in}}%
\pgfpathlineto{\pgfqpoint{3.907950in}{1.942529in}}%
\pgfpathlineto{\pgfqpoint{3.916757in}{1.956246in}}%
\pgfpathlineto{\pgfqpoint{3.943177in}{1.750370in}}%
\pgfpathlineto{\pgfqpoint{3.951984in}{2.107252in}}%
\pgfpathlineto{\pgfqpoint{3.960791in}{1.661150in}}%
\pgfpathlineto{\pgfqpoint{3.969597in}{1.571958in}}%
\pgfpathlineto{\pgfqpoint{3.978404in}{1.530777in}}%
\pgfpathlineto{\pgfqpoint{3.987211in}{1.619997in}}%
\pgfpathlineto{\pgfqpoint{3.996018in}{1.441556in}}%
\pgfpathlineto{\pgfqpoint{4.004825in}{1.764116in}}%
\pgfpathlineto{\pgfqpoint{4.013632in}{1.674896in}}%
\pgfpathlineto{\pgfqpoint{4.022438in}{1.736653in}}%
\pgfpathlineto{\pgfqpoint{4.031245in}{1.860167in}}%
\pgfpathlineto{\pgfqpoint{4.040052in}{1.777834in}}%
\pgfpathlineto{\pgfqpoint{4.048859in}{1.949388in}}%
\pgfpathlineto{\pgfqpoint{4.057666in}{2.004286in}}%
\pgfpathlineto{\pgfqpoint{4.066472in}{1.832732in}}%
\pgfpathlineto{\pgfqpoint{4.075279in}{2.086648in}}%
\pgfpathlineto{\pgfqpoint{4.084086in}{2.251343in}}%
\pgfpathlineto{\pgfqpoint{4.092893in}{1.860167in}}%
\pgfpathlineto{\pgfqpoint{4.101700in}{1.743512in}}%
\pgfpathlineto{\pgfqpoint{4.110507in}{1.709190in}}%
\pgfpathlineto{\pgfqpoint{4.119313in}{1.606252in}}%
\pgfpathlineto{\pgfqpoint{4.128120in}{1.716076in}}%
\pgfpathlineto{\pgfqpoint{4.136927in}{1.846450in}}%
\pgfpathlineto{\pgfqpoint{4.145734in}{1.709190in}}%
\pgfpathlineto{\pgfqpoint{4.154541in}{1.736653in}}%
\pgfpathlineto{\pgfqpoint{4.163347in}{1.537636in}}%
\pgfpathlineto{\pgfqpoint{4.172154in}{1.427839in}}%
\pgfpathlineto{\pgfqpoint{4.180961in}{1.434698in}}%
\pgfpathlineto{\pgfqpoint{4.189768in}{1.633715in}}%
\pgfpathlineto{\pgfqpoint{4.198575in}{1.455274in}}%
\pgfpathlineto{\pgfqpoint{4.207382in}{1.503314in}}%
\pgfpathlineto{\pgfqpoint{4.216188in}{1.647432in}}%
\pgfpathlineto{\pgfqpoint{4.224995in}{1.517031in}}%
\pgfpathlineto{\pgfqpoint{4.242609in}{1.976851in}}%
\pgfpathlineto{\pgfqpoint{4.251416in}{2.100365in}}%
\pgfpathlineto{\pgfqpoint{4.260222in}{2.148405in}}%
\pgfpathlineto{\pgfqpoint{4.269029in}{1.619997in}}%
\pgfpathlineto{\pgfqpoint{4.277836in}{1.393517in}}%
\pgfpathlineto{\pgfqpoint{4.286643in}{1.228822in}}%
\pgfpathlineto{\pgfqpoint{4.295450in}{1.503314in}}%
\pgfpathlineto{\pgfqpoint{4.304257in}{1.619997in}}%
\pgfpathlineto{\pgfqpoint{4.313063in}{1.640574in}}%
\pgfpathlineto{\pgfqpoint{4.321870in}{1.352336in}}%
\pgfpathlineto{\pgfqpoint{4.330677in}{1.228822in}}%
\pgfpathlineto{\pgfqpoint{4.339484in}{1.626856in}}%
\pgfpathlineto{\pgfqpoint{4.348291in}{1.901348in}}%
\pgfpathlineto{\pgfqpoint{4.374711in}{2.340563in}}%
\pgfpathlineto{\pgfqpoint{4.383518in}{2.580761in}}%
\pgfpathlineto{\pgfqpoint{4.392325in}{2.011173in}}%
\pgfpathlineto{\pgfqpoint{4.401132in}{1.887630in}}%
\pgfpathlineto{\pgfqpoint{4.409938in}{2.011173in}}%
\pgfpathlineto{\pgfqpoint{4.418745in}{1.770975in}}%
\pgfpathlineto{\pgfqpoint{4.436359in}{2.045467in}}%
\pgfpathlineto{\pgfqpoint{4.445166in}{2.052326in}}%
\pgfpathlineto{\pgfqpoint{4.453972in}{1.805269in}}%
\pgfpathlineto{\pgfqpoint{4.462779in}{1.407234in}}%
\pgfpathlineto{\pgfqpoint{4.471586in}{1.448415in}}%
\pgfpathlineto{\pgfqpoint{4.480393in}{1.544494in}}%
\pgfpathlineto{\pgfqpoint{4.489200in}{1.695472in}}%
\pgfpathlineto{\pgfqpoint{4.498007in}{1.716076in}}%
\pgfpathlineto{\pgfqpoint{4.506813in}{1.338618in}}%
\pgfpathlineto{\pgfqpoint{4.515620in}{1.462133in}}%
\pgfpathlineto{\pgfqpoint{4.524427in}{1.393517in}}%
\pgfpathlineto{\pgfqpoint{4.533234in}{1.098420in}}%
\pgfpathlineto{\pgfqpoint{4.542041in}{1.125855in}}%
\pgfpathlineto{\pgfqpoint{4.550847in}{1.420952in}}%
\pgfpathlineto{\pgfqpoint{4.559654in}{1.345477in}}%
\pgfpathlineto{\pgfqpoint{4.568461in}{1.496455in}}%
\pgfpathlineto{\pgfqpoint{4.577268in}{1.784692in}}%
\pgfpathlineto{\pgfqpoint{4.586075in}{1.558212in}}%
\pgfpathlineto{\pgfqpoint{4.594882in}{1.489596in}}%
\pgfpathlineto{\pgfqpoint{4.603688in}{1.722935in}}%
\pgfpathlineto{\pgfqpoint{4.612495in}{1.825873in}}%
\pgfpathlineto{\pgfqpoint{4.621302in}{2.120970in}}%
\pgfpathlineto{\pgfqpoint{4.630109in}{2.024891in}}%
\pgfpathlineto{\pgfqpoint{4.638916in}{2.306269in}}%
\pgfpathlineto{\pgfqpoint{4.647722in}{2.066071in}}%
\pgfpathlineto{\pgfqpoint{4.656529in}{1.997427in}}%
\pgfpathlineto{\pgfqpoint{4.665336in}{2.018032in}}%
\pgfpathlineto{\pgfqpoint{4.674143in}{1.585675in}}%
\pgfpathlineto{\pgfqpoint{4.682950in}{1.496455in}}%
\pgfpathlineto{\pgfqpoint{4.691757in}{1.819015in}}%
\pgfpathlineto{\pgfqpoint{4.700563in}{1.839591in}}%
\pgfpathlineto{\pgfqpoint{4.709370in}{1.819015in}}%
\pgfpathlineto{\pgfqpoint{4.718177in}{2.107252in}}%
\pgfpathlineto{\pgfqpoint{4.726984in}{2.251343in}}%
\pgfpathlineto{\pgfqpoint{4.735791in}{1.880772in}}%
\pgfpathlineto{\pgfqpoint{4.744597in}{1.915094in}}%
\pgfpathlineto{\pgfqpoint{4.753404in}{2.024891in}}%
\pgfpathlineto{\pgfqpoint{4.762211in}{2.011173in}}%
\pgfpathlineto{\pgfqpoint{4.771018in}{2.011173in}}%
\pgfpathlineto{\pgfqpoint{4.779825in}{2.134687in}}%
\pgfpathlineto{\pgfqpoint{4.788632in}{1.976851in}}%
\pgfpathlineto{\pgfqpoint{4.797438in}{2.066071in}}%
\pgfpathlineto{\pgfqpoint{4.806245in}{2.127829in}}%
\pgfpathlineto{\pgfqpoint{4.815052in}{2.436643in}}%
\pgfpathlineto{\pgfqpoint{4.823859in}{2.292524in}}%
\pgfpathlineto{\pgfqpoint{4.832666in}{2.251343in}}%
\pgfpathlineto{\pgfqpoint{4.841472in}{2.230767in}}%
\pgfpathlineto{\pgfqpoint{4.850279in}{2.470965in}}%
\pgfpathlineto{\pgfqpoint{4.859086in}{2.072930in}}%
\pgfpathlineto{\pgfqpoint{4.867893in}{2.024891in}}%
\pgfpathlineto{\pgfqpoint{4.876700in}{1.915094in}}%
\pgfpathlineto{\pgfqpoint{4.885507in}{1.736653in}}%
\pgfpathlineto{\pgfqpoint{4.894313in}{1.695472in}}%
\pgfpathlineto{\pgfqpoint{4.903120in}{1.819015in}}%
\pgfpathlineto{\pgfqpoint{4.911927in}{1.647432in}}%
\pgfpathlineto{\pgfqpoint{4.920734in}{1.668037in}}%
\pgfpathlineto{\pgfqpoint{4.929541in}{1.770975in}}%
\pgfpathlineto{\pgfqpoint{4.938347in}{1.825873in}}%
\pgfpathlineto{\pgfqpoint{4.947154in}{1.894489in}}%
\pgfpathlineto{\pgfqpoint{4.964768in}{2.217049in}}%
\pgfpathlineto{\pgfqpoint{4.973575in}{1.880772in}}%
\pgfpathlineto{\pgfqpoint{4.982382in}{2.052326in}}%
\pgfpathlineto{\pgfqpoint{4.991188in}{1.819015in}}%
\pgfpathlineto{\pgfqpoint{4.999995in}{1.839591in}}%
\pgfpathlineto{\pgfqpoint{5.008802in}{1.729794in}}%
\pgfpathlineto{\pgfqpoint{5.017609in}{1.867054in}}%
\pgfpathlineto{\pgfqpoint{5.026416in}{1.825873in}}%
\pgfpathlineto{\pgfqpoint{5.035222in}{1.867054in}}%
\pgfpathlineto{\pgfqpoint{5.044029in}{1.764116in}}%
\pgfpathlineto{\pgfqpoint{5.052836in}{1.839591in}}%
\pgfpathlineto{\pgfqpoint{5.061643in}{1.976851in}}%
\pgfpathlineto{\pgfqpoint{5.070450in}{1.928811in}}%
\pgfpathlineto{\pgfqpoint{5.079257in}{2.285665in}}%
\pgfpathlineto{\pgfqpoint{5.088063in}{2.079789in}}%
\pgfpathlineto{\pgfqpoint{5.096870in}{2.354309in}}%
\pgfpathlineto{\pgfqpoint{5.105677in}{1.976851in}}%
\pgfpathlineto{\pgfqpoint{5.123291in}{1.743512in}}%
\pgfpathlineto{\pgfqpoint{5.132097in}{1.736653in}}%
\pgfpathlineto{\pgfqpoint{5.140904in}{1.853308in}}%
\pgfpathlineto{\pgfqpoint{5.149711in}{1.743512in}}%
\pgfpathlineto{\pgfqpoint{5.158518in}{2.306269in}}%
\pgfpathlineto{\pgfqpoint{5.167325in}{1.908207in}}%
\pgfpathlineto{\pgfqpoint{5.176132in}{1.983710in}}%
\pgfpathlineto{\pgfqpoint{5.184938in}{2.024891in}}%
\pgfpathlineto{\pgfqpoint{5.193745in}{2.059213in}}%
\pgfpathlineto{\pgfqpoint{5.202552in}{1.928811in}}%
\pgfpathlineto{\pgfqpoint{5.211359in}{1.839591in}}%
\pgfpathlineto{\pgfqpoint{5.220166in}{2.107252in}}%
\pgfpathlineto{\pgfqpoint{5.228972in}{2.052326in}}%
\pgfpathlineto{\pgfqpoint{5.237779in}{2.066071in}}%
\pgfpathlineto{\pgfqpoint{5.246586in}{1.969992in}}%
\pgfpathlineto{\pgfqpoint{5.255393in}{2.093507in}}%
\pgfpathlineto{\pgfqpoint{5.264200in}{2.107252in}}%
\pgfpathlineto{\pgfqpoint{5.273007in}{1.935670in}}%
\pgfpathlineto{\pgfqpoint{5.281813in}{1.695472in}}%
\pgfpathlineto{\pgfqpoint{5.290620in}{1.551353in}}%
\pgfpathlineto{\pgfqpoint{5.299427in}{1.434698in}}%
\pgfpathlineto{\pgfqpoint{5.308234in}{1.400376in}}%
\pgfpathlineto{\pgfqpoint{5.317041in}{1.626856in}}%
\pgfpathlineto{\pgfqpoint{5.325847in}{1.729794in}}%
\pgfpathlineto{\pgfqpoint{5.343461in}{1.880772in}}%
\pgfpathlineto{\pgfqpoint{5.352268in}{1.571958in}}%
\pgfpathlineto{\pgfqpoint{5.361075in}{1.729794in}}%
\pgfpathlineto{\pgfqpoint{5.369882in}{1.558212in}}%
\pgfpathlineto{\pgfqpoint{5.378688in}{1.681754in}}%
\pgfpathlineto{\pgfqpoint{5.387495in}{2.134687in}}%
\pgfpathlineto{\pgfqpoint{5.396302in}{2.100365in}}%
\pgfpathlineto{\pgfqpoint{5.405109in}{2.285665in}}%
\pgfpathlineto{\pgfqpoint{5.413916in}{2.319987in}}%
\pgfpathlineto{\pgfqpoint{5.422722in}{2.443501in}}%
\pgfpathlineto{\pgfqpoint{5.431529in}{2.196445in}}%
\pgfpathlineto{\pgfqpoint{5.440336in}{2.505287in}}%
\pgfpathlineto{\pgfqpoint{5.449143in}{2.175868in}}%
\pgfpathlineto{\pgfqpoint{5.457950in}{1.716076in}}%
\pgfpathlineto{\pgfqpoint{5.466757in}{1.613110in}}%
\pgfpathlineto{\pgfqpoint{5.475563in}{1.860167in}}%
\pgfpathlineto{\pgfqpoint{5.484370in}{1.819015in}}%
\pgfpathlineto{\pgfqpoint{5.493177in}{1.647432in}}%
\pgfpathlineto{\pgfqpoint{5.501984in}{1.647432in}}%
\pgfpathlineto{\pgfqpoint{5.510791in}{1.805269in}}%
\pgfpathlineto{\pgfqpoint{5.519597in}{1.805269in}}%
\pgfpathlineto{\pgfqpoint{5.528404in}{2.038608in}}%
\pgfpathlineto{\pgfqpoint{5.537211in}{1.750370in}}%
\pgfpathlineto{\pgfqpoint{5.546018in}{2.155292in}}%
\pgfpathlineto{\pgfqpoint{5.554825in}{2.361168in}}%
\pgfpathlineto{\pgfqpoint{5.563632in}{2.326846in}}%
\pgfpathlineto{\pgfqpoint{5.572438in}{2.120970in}}%
\pgfpathlineto{\pgfqpoint{5.581245in}{2.285665in}}%
\pgfpathlineto{\pgfqpoint{5.590052in}{1.928811in}}%
\pgfpathlineto{\pgfqpoint{5.598859in}{1.839591in}}%
\pgfpathlineto{\pgfqpoint{5.607666in}{1.825873in}}%
\pgfpathlineto{\pgfqpoint{5.616472in}{1.619997in}}%
\pgfpathlineto{\pgfqpoint{5.625279in}{1.867054in}}%
\pgfpathlineto{\pgfqpoint{5.634086in}{2.052326in}}%
\pgfpathlineto{\pgfqpoint{5.642893in}{1.983710in}}%
\pgfpathlineto{\pgfqpoint{5.651700in}{2.422925in}}%
\pgfpathlineto{\pgfqpoint{5.660507in}{2.196445in}}%
\pgfpathlineto{\pgfqpoint{5.669313in}{1.880772in}}%
\pgfpathlineto{\pgfqpoint{5.678120in}{1.956246in}}%
\pgfpathlineto{\pgfqpoint{5.686927in}{2.079789in}}%
\pgfpathlineto{\pgfqpoint{5.695734in}{1.867054in}}%
\pgfpathlineto{\pgfqpoint{5.713347in}{1.722935in}}%
\pgfpathlineto{\pgfqpoint{5.722154in}{1.716076in}}%
\pgfpathlineto{\pgfqpoint{5.730961in}{1.743512in}}%
\pgfpathlineto{\pgfqpoint{5.739768in}{1.839591in}}%
\pgfpathlineto{\pgfqpoint{5.748575in}{2.203303in}}%
\pgfpathlineto{\pgfqpoint{5.757382in}{2.120970in}}%
\pgfpathlineto{\pgfqpoint{5.766188in}{1.942529in}}%
\pgfpathlineto{\pgfqpoint{5.774995in}{1.832732in}}%
\pgfpathlineto{\pgfqpoint{5.783802in}{1.688613in}}%
\pgfpathlineto{\pgfqpoint{5.792609in}{1.963133in}}%
\pgfpathlineto{\pgfqpoint{5.801416in}{2.045467in}}%
\pgfpathlineto{\pgfqpoint{5.810222in}{2.045467in}}%
\pgfpathlineto{\pgfqpoint{5.819029in}{2.505287in}}%
\pgfpathlineto{\pgfqpoint{5.827836in}{2.278806in}}%
\pgfpathlineto{\pgfqpoint{5.836643in}{1.935670in}}%
\pgfpathlineto{\pgfqpoint{5.845450in}{1.860167in}}%
\pgfpathlineto{\pgfqpoint{5.854257in}{1.990569in}}%
\pgfpathlineto{\pgfqpoint{5.863063in}{2.093507in}}%
\pgfpathlineto{\pgfqpoint{5.871870in}{2.114111in}}%
\pgfpathlineto{\pgfqpoint{5.880677in}{1.770975in}}%
\pgfpathlineto{\pgfqpoint{5.889484in}{1.475878in}}%
\pgfpathlineto{\pgfqpoint{5.898291in}{1.510172in}}%
\pgfpathlineto{\pgfqpoint{5.907097in}{1.668037in}}%
\pgfpathlineto{\pgfqpoint{5.915904in}{1.709190in}}%
\pgfpathlineto{\pgfqpoint{5.924711in}{1.798410in}}%
\pgfpathlineto{\pgfqpoint{5.933518in}{1.764116in}}%
\pgfpathlineto{\pgfqpoint{5.942325in}{2.011173in}}%
\pgfpathlineto{\pgfqpoint{5.951132in}{1.997427in}}%
\pgfpathlineto{\pgfqpoint{5.959938in}{2.134687in}}%
\pgfpathlineto{\pgfqpoint{5.968745in}{2.230767in}}%
\pgfpathlineto{\pgfqpoint{5.977552in}{2.072930in}}%
\pgfpathlineto{\pgfqpoint{5.986359in}{2.265089in}}%
\pgfpathlineto{\pgfqpoint{5.995166in}{2.278806in}}%
\pgfpathlineto{\pgfqpoint{6.003972in}{2.093507in}}%
\pgfpathlineto{\pgfqpoint{6.012779in}{2.031749in}}%
\pgfpathlineto{\pgfqpoint{6.021586in}{1.935670in}}%
\pgfpathlineto{\pgfqpoint{6.030393in}{1.894489in}}%
\pgfpathlineto{\pgfqpoint{6.039200in}{1.935670in}}%
\pgfpathlineto{\pgfqpoint{6.048007in}{1.695472in}}%
\pgfpathlineto{\pgfqpoint{6.056813in}{1.832732in}}%
\pgfpathlineto{\pgfqpoint{6.065620in}{1.901348in}}%
\pgfpathlineto{\pgfqpoint{6.074427in}{1.997427in}}%
\pgfpathlineto{\pgfqpoint{6.083234in}{1.832732in}}%
\pgfpathlineto{\pgfqpoint{6.092041in}{1.750370in}}%
\pgfpathlineto{\pgfqpoint{6.118461in}{2.285665in}}%
\pgfpathlineto{\pgfqpoint{6.127268in}{2.580761in}}%
\pgfpathlineto{\pgfqpoint{6.136075in}{2.539581in}}%
\pgfpathlineto{\pgfqpoint{6.144882in}{2.642547in}}%
\pgfpathlineto{\pgfqpoint{6.153688in}{2.340563in}}%
\pgfpathlineto{\pgfqpoint{6.162495in}{1.928811in}}%
\pgfpathlineto{\pgfqpoint{6.171302in}{2.141546in}}%
\pgfpathlineto{\pgfqpoint{6.180109in}{2.285665in}}%
\pgfpathlineto{\pgfqpoint{6.188916in}{2.333705in}}%
\pgfpathlineto{\pgfqpoint{6.197722in}{2.210190in}}%
\pgfpathlineto{\pgfqpoint{6.206529in}{2.189586in}}%
\pgfpathlineto{\pgfqpoint{6.215336in}{1.867054in}}%
\pgfpathlineto{\pgfqpoint{6.224143in}{1.770975in}}%
\pgfpathlineto{\pgfqpoint{6.232950in}{1.969992in}}%
\pgfpathlineto{\pgfqpoint{6.241757in}{1.956246in}}%
\pgfpathlineto{\pgfqpoint{6.250563in}{2.072930in}}%
\pgfpathlineto{\pgfqpoint{6.259370in}{1.983710in}}%
\pgfpathlineto{\pgfqpoint{6.268177in}{1.674896in}}%
\pgfpathlineto{\pgfqpoint{6.276984in}{1.722935in}}%
\pgfpathlineto{\pgfqpoint{6.285791in}{1.448415in}}%
\pgfpathlineto{\pgfqpoint{6.294597in}{1.633715in}}%
\pgfpathlineto{\pgfqpoint{6.303404in}{1.448415in}}%
\pgfpathlineto{\pgfqpoint{6.312211in}{1.695472in}}%
\pgfpathlineto{\pgfqpoint{6.321018in}{1.489596in}}%
\pgfpathlineto{\pgfqpoint{6.329825in}{1.331760in}}%
\pgfpathlineto{\pgfqpoint{6.338632in}{1.825873in}}%
\pgfpathlineto{\pgfqpoint{6.347438in}{1.661150in}}%
\pgfpathlineto{\pgfqpoint{6.356245in}{1.571958in}}%
\pgfpathlineto{\pgfqpoint{6.365052in}{1.963133in}}%
\pgfpathlineto{\pgfqpoint{6.373859in}{2.141546in}}%
\pgfpathlineto{\pgfqpoint{6.382666in}{2.278806in}}%
\pgfpathlineto{\pgfqpoint{6.391472in}{1.990569in}}%
\pgfpathlineto{\pgfqpoint{6.400279in}{1.517031in}}%
\pgfpathlineto{\pgfqpoint{6.409086in}{1.887630in}}%
\pgfpathlineto{\pgfqpoint{6.417893in}{1.846450in}}%
\pgfpathlineto{\pgfqpoint{6.426700in}{1.832732in}}%
\pgfpathlineto{\pgfqpoint{6.435507in}{1.825873in}}%
\pgfpathlineto{\pgfqpoint{6.444313in}{1.928811in}}%
\pgfpathlineto{\pgfqpoint{6.453120in}{1.873913in}}%
\pgfpathlineto{\pgfqpoint{6.461927in}{2.148405in}}%
\pgfpathlineto{\pgfqpoint{6.470734in}{2.066071in}}%
\pgfpathlineto{\pgfqpoint{6.479541in}{2.086648in}}%
\pgfpathlineto{\pgfqpoint{6.488347in}{2.045467in}}%
\pgfpathlineto{\pgfqpoint{6.497154in}{2.148405in}}%
\pgfpathlineto{\pgfqpoint{6.505961in}{2.347422in}}%
\pgfpathlineto{\pgfqpoint{6.523575in}{2.539581in}}%
\pgfpathlineto{\pgfqpoint{6.541188in}{1.873913in}}%
\pgfpathlineto{\pgfqpoint{6.549995in}{1.832732in}}%
\pgfpathlineto{\pgfqpoint{6.558802in}{2.038608in}}%
\pgfpathlineto{\pgfqpoint{6.567609in}{2.306269in}}%
\pgfpathlineto{\pgfqpoint{6.585222in}{1.956246in}}%
\pgfpathlineto{\pgfqpoint{6.594029in}{1.496455in}}%
\pgfpathlineto{\pgfqpoint{6.611643in}{1.846450in}}%
\pgfpathlineto{\pgfqpoint{6.620450in}{1.901348in}}%
\pgfpathlineto{\pgfqpoint{6.629257in}{1.757229in}}%
\pgfpathlineto{\pgfqpoint{6.638063in}{1.770975in}}%
\pgfpathlineto{\pgfqpoint{6.646870in}{2.066071in}}%
\pgfpathlineto{\pgfqpoint{6.655677in}{1.757229in}}%
\pgfpathlineto{\pgfqpoint{6.664484in}{1.722935in}}%
\pgfpathlineto{\pgfqpoint{6.673291in}{1.633715in}}%
\pgfpathlineto{\pgfqpoint{6.682097in}{1.633715in}}%
\pgfpathlineto{\pgfqpoint{6.690904in}{1.784692in}}%
\pgfpathlineto{\pgfqpoint{6.699711in}{1.633715in}}%
\pgfpathlineto{\pgfqpoint{6.708518in}{1.503314in}}%
\pgfpathlineto{\pgfqpoint{6.717325in}{1.716076in}}%
\pgfpathlineto{\pgfqpoint{6.726132in}{1.722935in}}%
\pgfpathlineto{\pgfqpoint{6.734938in}{1.647432in}}%
\pgfpathlineto{\pgfqpoint{6.743745in}{1.969992in}}%
\pgfpathlineto{\pgfqpoint{6.752552in}{1.805269in}}%
\pgfpathlineto{\pgfqpoint{6.761359in}{1.997427in}}%
\pgfpathlineto{\pgfqpoint{6.770166in}{1.647432in}}%
\pgfpathlineto{\pgfqpoint{6.778972in}{1.688613in}}%
\pgfpathlineto{\pgfqpoint{6.787779in}{1.736653in}}%
\pgfpathlineto{\pgfqpoint{6.796586in}{1.901348in}}%
\pgfpathlineto{\pgfqpoint{6.805393in}{1.990569in}}%
\pgfpathlineto{\pgfqpoint{6.814200in}{1.530777in}}%
\pgfpathlineto{\pgfqpoint{6.823007in}{1.681754in}}%
\pgfpathlineto{\pgfqpoint{6.831813in}{1.633715in}}%
\pgfpathlineto{\pgfqpoint{6.840620in}{1.688613in}}%
\pgfpathlineto{\pgfqpoint{6.849427in}{2.107252in}}%
\pgfpathlineto{\pgfqpoint{6.858234in}{2.436643in}}%
\pgfpathlineto{\pgfqpoint{6.867041in}{2.127829in}}%
\pgfpathlineto{\pgfqpoint{6.875847in}{2.230767in}}%
\pgfpathlineto{\pgfqpoint{6.884654in}{2.086648in}}%
\pgfpathlineto{\pgfqpoint{6.893461in}{2.059213in}}%
\pgfpathlineto{\pgfqpoint{6.902268in}{2.175868in}}%
\pgfpathlineto{\pgfqpoint{6.911075in}{1.969992in}}%
\pgfpathlineto{\pgfqpoint{6.919882in}{1.942529in}}%
\pgfpathlineto{\pgfqpoint{6.928688in}{1.976851in}}%
\pgfpathlineto{\pgfqpoint{6.937495in}{1.935670in}}%
\pgfpathlineto{\pgfqpoint{6.946302in}{2.237625in}}%
\pgfpathlineto{\pgfqpoint{6.955109in}{2.127829in}}%
\pgfpathlineto{\pgfqpoint{6.963916in}{2.210190in}}%
\pgfpathlineto{\pgfqpoint{6.972722in}{2.162151in}}%
\pgfpathlineto{\pgfqpoint{6.981529in}{2.278806in}}%
\pgfpathlineto{\pgfqpoint{6.990336in}{2.477823in}}%
\pgfpathlineto{\pgfqpoint{6.999143in}{2.210190in}}%
\pgfpathlineto{\pgfqpoint{7.007950in}{2.155292in}}%
\pgfpathlineto{\pgfqpoint{7.016757in}{2.045467in}}%
\pgfpathlineto{\pgfqpoint{7.025563in}{2.059213in}}%
\pgfpathlineto{\pgfqpoint{7.034370in}{1.674896in}}%
\pgfpathlineto{\pgfqpoint{7.043177in}{1.764116in}}%
\pgfpathlineto{\pgfqpoint{7.051984in}{2.100365in}}%
\pgfpathlineto{\pgfqpoint{7.060791in}{1.990569in}}%
\pgfpathlineto{\pgfqpoint{7.069597in}{2.141546in}}%
\pgfpathlineto{\pgfqpoint{7.078404in}{1.949388in}}%
\pgfpathlineto{\pgfqpoint{7.087211in}{1.880772in}}%
\pgfpathlineto{\pgfqpoint{7.096018in}{2.052326in}}%
\pgfpathlineto{\pgfqpoint{7.104825in}{2.347422in}}%
\pgfpathlineto{\pgfqpoint{7.122438in}{2.018032in}}%
\pgfpathlineto{\pgfqpoint{7.131245in}{1.949388in}}%
\pgfpathlineto{\pgfqpoint{7.140052in}{1.915094in}}%
\pgfpathlineto{\pgfqpoint{7.148859in}{1.901348in}}%
\pgfpathlineto{\pgfqpoint{7.157666in}{1.935670in}}%
\pgfpathlineto{\pgfqpoint{7.166472in}{1.736653in}}%
\pgfpathlineto{\pgfqpoint{7.175279in}{1.805269in}}%
\pgfpathlineto{\pgfqpoint{7.184086in}{1.921953in}}%
\pgfpathlineto{\pgfqpoint{7.192893in}{1.674896in}}%
\pgfpathlineto{\pgfqpoint{7.201700in}{1.942529in}}%
\pgfpathlineto{\pgfqpoint{7.210507in}{1.777834in}}%
\pgfpathlineto{\pgfqpoint{7.219313in}{1.661150in}}%
\pgfpathlineto{\pgfqpoint{7.228120in}{1.736653in}}%
\pgfpathlineto{\pgfqpoint{7.236927in}{1.681754in}}%
\pgfpathlineto{\pgfqpoint{7.245734in}{1.825873in}}%
\pgfpathlineto{\pgfqpoint{7.254541in}{1.709190in}}%
\pgfpathlineto{\pgfqpoint{7.263347in}{1.750370in}}%
\pgfpathlineto{\pgfqpoint{7.272154in}{1.956246in}}%
\pgfpathlineto{\pgfqpoint{7.280961in}{1.606252in}}%
\pgfpathlineto{\pgfqpoint{7.289768in}{1.887630in}}%
\pgfpathlineto{\pgfqpoint{7.298575in}{1.702331in}}%
\pgfpathlineto{\pgfqpoint{7.307382in}{1.592534in}}%
\pgfpathlineto{\pgfqpoint{7.316188in}{1.873913in}}%
\pgfpathlineto{\pgfqpoint{7.324995in}{1.832732in}}%
\pgfpathlineto{\pgfqpoint{7.333802in}{1.983710in}}%
\pgfpathlineto{\pgfqpoint{7.342609in}{2.079789in}}%
\pgfpathlineto{\pgfqpoint{7.351416in}{2.031749in}}%
\pgfpathlineto{\pgfqpoint{7.360222in}{1.901348in}}%
\pgfpathlineto{\pgfqpoint{7.369029in}{1.908207in}}%
\pgfpathlineto{\pgfqpoint{7.377836in}{1.729794in}}%
\pgfpathlineto{\pgfqpoint{7.386643in}{2.059213in}}%
\pgfpathlineto{\pgfqpoint{7.395450in}{1.935670in}}%
\pgfpathlineto{\pgfqpoint{7.404257in}{1.928811in}}%
\pgfpathlineto{\pgfqpoint{7.413063in}{2.004286in}}%
\pgfpathlineto{\pgfqpoint{7.421870in}{1.908207in}}%
\pgfpathlineto{\pgfqpoint{7.430677in}{1.654291in}}%
\pgfpathlineto{\pgfqpoint{7.439484in}{1.489596in}}%
\pgfpathlineto{\pgfqpoint{7.448291in}{1.167036in}}%
\pgfpathlineto{\pgfqpoint{7.457097in}{1.386658in}}%
\pgfpathlineto{\pgfqpoint{7.465904in}{1.558212in}}%
\pgfpathlineto{\pgfqpoint{7.474711in}{2.031749in}}%
\pgfpathlineto{\pgfqpoint{7.483518in}{2.004286in}}%
\pgfpathlineto{\pgfqpoint{7.492325in}{1.935670in}}%
\pgfpathlineto{\pgfqpoint{7.501132in}{1.894489in}}%
\pgfpathlineto{\pgfqpoint{7.509938in}{1.736653in}}%
\pgfpathlineto{\pgfqpoint{7.518745in}{1.674896in}}%
\pgfpathlineto{\pgfqpoint{7.527552in}{1.688613in}}%
\pgfpathlineto{\pgfqpoint{7.536359in}{1.571958in}}%
\pgfpathlineto{\pgfqpoint{7.545166in}{1.544494in}}%
\pgfpathlineto{\pgfqpoint{7.553972in}{1.784692in}}%
\pgfpathlineto{\pgfqpoint{7.562779in}{1.448415in}}%
\pgfpathlineto{\pgfqpoint{7.571586in}{1.819015in}}%
\pgfpathlineto{\pgfqpoint{7.580393in}{2.018032in}}%
\pgfpathlineto{\pgfqpoint{7.589200in}{2.189586in}}%
\pgfpathlineto{\pgfqpoint{7.598007in}{1.949388in}}%
\pgfpathlineto{\pgfqpoint{7.606813in}{1.770975in}}%
\pgfpathlineto{\pgfqpoint{7.615620in}{1.969992in}}%
\pgfpathlineto{\pgfqpoint{7.624427in}{2.306269in}}%
\pgfpathlineto{\pgfqpoint{7.633234in}{2.237625in}}%
\pgfpathlineto{\pgfqpoint{7.642041in}{2.354309in}}%
\pgfpathlineto{\pgfqpoint{7.650847in}{2.532722in}}%
\pgfpathlineto{\pgfqpoint{7.668461in}{2.086648in}}%
\pgfpathlineto{\pgfqpoint{7.677268in}{1.935670in}}%
\pgfpathlineto{\pgfqpoint{7.686075in}{1.908207in}}%
\pgfpathlineto{\pgfqpoint{7.694882in}{2.374885in}}%
\pgfpathlineto{\pgfqpoint{7.703688in}{2.271947in}}%
\pgfpathlineto{\pgfqpoint{7.712495in}{2.251343in}}%
\pgfpathlineto{\pgfqpoint{7.721302in}{2.045467in}}%
\pgfpathlineto{\pgfqpoint{7.730109in}{2.127829in}}%
\pgfpathlineto{\pgfqpoint{7.738916in}{1.743512in}}%
\pgfpathlineto{\pgfqpoint{7.747722in}{1.812128in}}%
\pgfpathlineto{\pgfqpoint{7.756529in}{2.018032in}}%
\pgfpathlineto{\pgfqpoint{7.765336in}{2.278806in}}%
\pgfpathlineto{\pgfqpoint{7.774143in}{2.148405in}}%
\pgfpathlineto{\pgfqpoint{7.782950in}{2.278806in}}%
\pgfpathlineto{\pgfqpoint{7.791757in}{2.182727in}}%
\pgfpathlineto{\pgfqpoint{7.800563in}{2.189586in}}%
\pgfpathlineto{\pgfqpoint{7.809370in}{2.285665in}}%
\pgfpathlineto{\pgfqpoint{7.818177in}{2.313128in}}%
\pgfpathlineto{\pgfqpoint{7.826984in}{1.921953in}}%
\pgfpathlineto{\pgfqpoint{7.835791in}{1.963133in}}%
\pgfpathlineto{\pgfqpoint{7.844597in}{2.100365in}}%
\pgfpathlineto{\pgfqpoint{7.853404in}{2.155292in}}%
\pgfpathlineto{\pgfqpoint{7.862211in}{2.182727in}}%
\pgfpathlineto{\pgfqpoint{7.871018in}{1.880772in}}%
\pgfpathlineto{\pgfqpoint{7.879825in}{2.072930in}}%
\pgfpathlineto{\pgfqpoint{7.888632in}{1.832732in}}%
\pgfpathlineto{\pgfqpoint{7.897438in}{1.825873in}}%
\pgfpathlineto{\pgfqpoint{7.906245in}{1.894489in}}%
\pgfpathlineto{\pgfqpoint{7.915052in}{1.860167in}}%
\pgfpathlineto{\pgfqpoint{7.923859in}{1.873913in}}%
\pgfpathlineto{\pgfqpoint{7.932666in}{1.928811in}}%
\pgfpathlineto{\pgfqpoint{7.941472in}{1.880772in}}%
\pgfpathlineto{\pgfqpoint{7.950279in}{2.189586in}}%
\pgfpathlineto{\pgfqpoint{7.959086in}{2.189586in}}%
\pgfpathlineto{\pgfqpoint{7.967893in}{1.997427in}}%
\pgfpathlineto{\pgfqpoint{7.976700in}{1.606252in}}%
\pgfpathlineto{\pgfqpoint{7.985507in}{1.853308in}}%
\pgfpathlineto{\pgfqpoint{7.994313in}{1.901348in}}%
\pgfpathlineto{\pgfqpoint{8.003120in}{2.018032in}}%
\pgfpathlineto{\pgfqpoint{8.011927in}{2.265089in}}%
\pgfpathlineto{\pgfqpoint{8.020734in}{2.120970in}}%
\pgfpathlineto{\pgfqpoint{8.029541in}{2.127829in}}%
\pgfpathlineto{\pgfqpoint{8.038347in}{1.956246in}}%
\pgfpathlineto{\pgfqpoint{8.047154in}{2.244484in}}%
\pgfpathlineto{\pgfqpoint{8.055961in}{2.278806in}}%
\pgfpathlineto{\pgfqpoint{8.064768in}{2.086648in}}%
\pgfpathlineto{\pgfqpoint{8.073575in}{2.265089in}}%
\pgfpathlineto{\pgfqpoint{8.091188in}{1.853308in}}%
\pgfpathlineto{\pgfqpoint{8.099995in}{1.757229in}}%
\pgfpathlineto{\pgfqpoint{8.117609in}{1.304296in}}%
\pgfpathlineto{\pgfqpoint{8.126416in}{1.468992in}}%
\pgfpathlineto{\pgfqpoint{8.135222in}{1.551353in}}%
\pgfpathlineto{\pgfqpoint{8.144029in}{1.585675in}}%
\pgfpathlineto{\pgfqpoint{8.152836in}{1.709190in}}%
\pgfpathlineto{\pgfqpoint{8.161643in}{1.654291in}}%
\pgfpathlineto{\pgfqpoint{8.170450in}{1.668037in}}%
\pgfpathlineto{\pgfqpoint{8.179257in}{1.901348in}}%
\pgfpathlineto{\pgfqpoint{8.188063in}{1.750370in}}%
\pgfpathlineto{\pgfqpoint{8.196870in}{1.825873in}}%
\pgfpathlineto{\pgfqpoint{8.205677in}{1.997427in}}%
\pgfpathlineto{\pgfqpoint{8.214484in}{1.949388in}}%
\pgfpathlineto{\pgfqpoint{8.223291in}{1.860167in}}%
\pgfpathlineto{\pgfqpoint{8.232097in}{2.018032in}}%
\pgfpathlineto{\pgfqpoint{8.240904in}{1.908207in}}%
\pgfpathlineto{\pgfqpoint{8.249711in}{1.757229in}}%
\pgfpathlineto{\pgfqpoint{8.258518in}{1.661150in}}%
\pgfpathlineto{\pgfqpoint{8.267325in}{1.825873in}}%
\pgfpathlineto{\pgfqpoint{8.276132in}{2.079789in}}%
\pgfpathlineto{\pgfqpoint{8.284938in}{2.175868in}}%
\pgfpathlineto{\pgfqpoint{8.293745in}{1.956246in}}%
\pgfpathlineto{\pgfqpoint{8.302552in}{1.846450in}}%
\pgfpathlineto{\pgfqpoint{8.311359in}{1.558212in}}%
\pgfpathlineto{\pgfqpoint{8.320166in}{1.359195in}}%
\pgfpathlineto{\pgfqpoint{8.328972in}{1.304296in}}%
\pgfpathlineto{\pgfqpoint{8.337779in}{1.681754in}}%
\pgfpathlineto{\pgfqpoint{8.346586in}{1.990569in}}%
\pgfpathlineto{\pgfqpoint{8.355393in}{1.784692in}}%
\pgfpathlineto{\pgfqpoint{8.373007in}{1.167036in}}%
\pgfpathlineto{\pgfqpoint{8.381813in}{1.290579in}}%
\pgfpathlineto{\pgfqpoint{8.390620in}{1.489596in}}%
\pgfpathlineto{\pgfqpoint{8.399427in}{1.798410in}}%
\pgfpathlineto{\pgfqpoint{8.408234in}{1.770975in}}%
\pgfpathlineto{\pgfqpoint{8.417041in}{2.086648in}}%
\pgfpathlineto{\pgfqpoint{8.425847in}{1.674896in}}%
\pgfpathlineto{\pgfqpoint{8.434654in}{1.661150in}}%
\pgfpathlineto{\pgfqpoint{8.443461in}{1.736653in}}%
\pgfpathlineto{\pgfqpoint{8.452268in}{2.038608in}}%
\pgfpathlineto{\pgfqpoint{8.461075in}{2.114111in}}%
\pgfpathlineto{\pgfqpoint{8.469882in}{1.853308in}}%
\pgfpathlineto{\pgfqpoint{8.478688in}{1.983710in}}%
\pgfpathlineto{\pgfqpoint{8.487495in}{2.258230in}}%
\pgfpathlineto{\pgfqpoint{8.496302in}{2.388603in}}%
\pgfpathlineto{\pgfqpoint{8.505109in}{2.285665in}}%
\pgfpathlineto{\pgfqpoint{8.513916in}{1.983710in}}%
\pgfpathlineto{\pgfqpoint{8.522722in}{2.011173in}}%
\pgfpathlineto{\pgfqpoint{8.531529in}{2.059213in}}%
\pgfpathlineto{\pgfqpoint{8.540336in}{1.963133in}}%
\pgfpathlineto{\pgfqpoint{8.549143in}{2.217049in}}%
\pgfpathlineto{\pgfqpoint{8.557950in}{2.107252in}}%
\pgfpathlineto{\pgfqpoint{8.566757in}{2.347422in}}%
\pgfpathlineto{\pgfqpoint{8.575563in}{2.120970in}}%
\pgfpathlineto{\pgfqpoint{8.584370in}{1.764116in}}%
\pgfpathlineto{\pgfqpoint{8.593177in}{1.798410in}}%
\pgfpathlineto{\pgfqpoint{8.601984in}{2.038608in}}%
\pgfpathlineto{\pgfqpoint{8.619597in}{1.434698in}}%
\pgfpathlineto{\pgfqpoint{8.628404in}{1.530777in}}%
\pgfpathlineto{\pgfqpoint{8.637211in}{1.420952in}}%
\pgfpathlineto{\pgfqpoint{8.646018in}{1.448415in}}%
\pgfpathlineto{\pgfqpoint{8.654825in}{1.626856in}}%
\pgfpathlineto{\pgfqpoint{8.663632in}{1.366053in}}%
\pgfpathlineto{\pgfqpoint{8.672438in}{1.709190in}}%
\pgfpathlineto{\pgfqpoint{8.681245in}{1.722935in}}%
\pgfpathlineto{\pgfqpoint{8.690052in}{1.908207in}}%
\pgfpathlineto{\pgfqpoint{8.698859in}{2.244484in}}%
\pgfpathlineto{\pgfqpoint{8.707666in}{2.196445in}}%
\pgfpathlineto{\pgfqpoint{8.716472in}{2.086648in}}%
\pgfpathlineto{\pgfqpoint{8.725279in}{2.011173in}}%
\pgfpathlineto{\pgfqpoint{8.734086in}{2.162151in}}%
\pgfpathlineto{\pgfqpoint{8.742893in}{2.169009in}}%
\pgfpathlineto{\pgfqpoint{8.751700in}{1.963133in}}%
\pgfpathlineto{\pgfqpoint{8.760507in}{2.120970in}}%
\pgfpathlineto{\pgfqpoint{8.769313in}{2.079789in}}%
\pgfpathlineto{\pgfqpoint{8.778120in}{2.031749in}}%
\pgfpathlineto{\pgfqpoint{8.786927in}{1.722935in}}%
\pgfpathlineto{\pgfqpoint{8.795734in}{1.825873in}}%
\pgfpathlineto{\pgfqpoint{8.804541in}{2.031749in}}%
\pgfpathlineto{\pgfqpoint{8.813347in}{1.729794in}}%
\pgfpathlineto{\pgfqpoint{8.822154in}{1.736653in}}%
\pgfpathlineto{\pgfqpoint{8.830961in}{1.770975in}}%
\pgfpathlineto{\pgfqpoint{8.839768in}{2.066071in}}%
\pgfpathlineto{\pgfqpoint{8.848575in}{1.729794in}}%
\pgfpathlineto{\pgfqpoint{8.857382in}{1.976851in}}%
\pgfpathlineto{\pgfqpoint{8.866188in}{2.155292in}}%
\pgfpathlineto{\pgfqpoint{8.874995in}{2.052326in}}%
\pgfpathlineto{\pgfqpoint{8.883802in}{1.983710in}}%
\pgfpathlineto{\pgfqpoint{8.892609in}{2.512146in}}%
\pgfpathlineto{\pgfqpoint{8.901416in}{2.470965in}}%
\pgfpathlineto{\pgfqpoint{8.910222in}{2.093507in}}%
\pgfpathlineto{\pgfqpoint{8.919029in}{1.997427in}}%
\pgfpathlineto{\pgfqpoint{8.927836in}{1.928811in}}%
\pgfpathlineto{\pgfqpoint{8.936643in}{1.908207in}}%
\pgfpathlineto{\pgfqpoint{8.945450in}{2.148405in}}%
\pgfpathlineto{\pgfqpoint{8.954257in}{2.141546in}}%
\pgfpathlineto{\pgfqpoint{8.963063in}{2.079789in}}%
\pgfpathlineto{\pgfqpoint{8.971870in}{1.901348in}}%
\pgfpathlineto{\pgfqpoint{8.980677in}{1.908207in}}%
\pgfpathlineto{\pgfqpoint{8.989484in}{1.661150in}}%
\pgfpathlineto{\pgfqpoint{8.998291in}{2.018032in}}%
\pgfpathlineto{\pgfqpoint{9.007097in}{2.031749in}}%
\pgfpathlineto{\pgfqpoint{9.015904in}{2.066071in}}%
\pgfpathlineto{\pgfqpoint{9.024711in}{2.004286in}}%
\pgfpathlineto{\pgfqpoint{9.033518in}{2.120970in}}%
\pgfpathlineto{\pgfqpoint{9.042325in}{2.189586in}}%
\pgfpathlineto{\pgfqpoint{9.051132in}{1.880772in}}%
\pgfpathlineto{\pgfqpoint{9.059938in}{1.688613in}}%
\pgfpathlineto{\pgfqpoint{9.068745in}{1.729794in}}%
\pgfpathlineto{\pgfqpoint{9.077552in}{1.599393in}}%
\pgfpathlineto{\pgfqpoint{9.086359in}{1.578816in}}%
\pgfpathlineto{\pgfqpoint{9.095166in}{1.764116in}}%
\pgfpathlineto{\pgfqpoint{9.103972in}{1.764116in}}%
\pgfpathlineto{\pgfqpoint{9.112779in}{1.949388in}}%
\pgfpathlineto{\pgfqpoint{9.121586in}{1.819015in}}%
\pgfpathlineto{\pgfqpoint{9.130393in}{1.654291in}}%
\pgfpathlineto{\pgfqpoint{9.139200in}{1.839591in}}%
\pgfpathlineto{\pgfqpoint{9.148007in}{2.175868in}}%
\pgfpathlineto{\pgfqpoint{9.156813in}{1.963133in}}%
\pgfpathlineto{\pgfqpoint{9.165620in}{2.052326in}}%
\pgfpathlineto{\pgfqpoint{9.174427in}{2.079789in}}%
\pgfpathlineto{\pgfqpoint{9.183234in}{1.997427in}}%
\pgfpathlineto{\pgfqpoint{9.192041in}{2.114111in}}%
\pgfpathlineto{\pgfqpoint{9.200847in}{2.011173in}}%
\pgfpathlineto{\pgfqpoint{9.209654in}{2.292524in}}%
\pgfpathlineto{\pgfqpoint{9.218461in}{2.457247in}}%
\pgfpathlineto{\pgfqpoint{9.227268in}{2.546468in}}%
\pgfpathlineto{\pgfqpoint{9.236075in}{2.299383in}}%
\pgfpathlineto{\pgfqpoint{9.244882in}{2.368027in}}%
\pgfpathlineto{\pgfqpoint{9.262495in}{1.832732in}}%
\pgfpathlineto{\pgfqpoint{9.271302in}{1.812128in}}%
\pgfpathlineto{\pgfqpoint{9.280109in}{1.743512in}}%
\pgfpathlineto{\pgfqpoint{9.288916in}{1.503314in}}%
\pgfpathlineto{\pgfqpoint{9.297722in}{1.736653in}}%
\pgfpathlineto{\pgfqpoint{9.306529in}{1.777834in}}%
\pgfpathlineto{\pgfqpoint{9.315336in}{1.606252in}}%
\pgfpathlineto{\pgfqpoint{9.324143in}{1.784692in}}%
\pgfpathlineto{\pgfqpoint{9.332950in}{1.702331in}}%
\pgfpathlineto{\pgfqpoint{9.341757in}{2.093507in}}%
\pgfpathlineto{\pgfqpoint{9.350563in}{2.031749in}}%
\pgfpathlineto{\pgfqpoint{9.359370in}{2.018032in}}%
\pgfpathlineto{\pgfqpoint{9.368177in}{2.120970in}}%
\pgfpathlineto{\pgfqpoint{9.385791in}{2.347422in}}%
\pgfpathlineto{\pgfqpoint{9.394597in}{2.230767in}}%
\pgfpathlineto{\pgfqpoint{9.403404in}{2.237625in}}%
\pgfpathlineto{\pgfqpoint{9.412211in}{2.299383in}}%
\pgfpathlineto{\pgfqpoint{9.421018in}{2.120970in}}%
\pgfpathlineto{\pgfqpoint{9.429825in}{2.162151in}}%
\pgfpathlineto{\pgfqpoint{9.438632in}{2.299383in}}%
\pgfpathlineto{\pgfqpoint{9.447438in}{2.162151in}}%
\pgfpathlineto{\pgfqpoint{9.456245in}{2.230767in}}%
\pgfpathlineto{\pgfqpoint{9.465052in}{2.127829in}}%
\pgfpathlineto{\pgfqpoint{9.473859in}{2.278806in}}%
\pgfpathlineto{\pgfqpoint{9.482666in}{2.203303in}}%
\pgfpathlineto{\pgfqpoint{9.491472in}{1.901348in}}%
\pgfpathlineto{\pgfqpoint{9.500279in}{2.004286in}}%
\pgfpathlineto{\pgfqpoint{9.509086in}{1.832732in}}%
\pgfpathlineto{\pgfqpoint{9.517893in}{1.942529in}}%
\pgfpathlineto{\pgfqpoint{9.526700in}{1.702331in}}%
\pgfpathlineto{\pgfqpoint{9.535507in}{1.729794in}}%
\pgfpathlineto{\pgfqpoint{9.544313in}{1.633715in}}%
\pgfpathlineto{\pgfqpoint{9.553120in}{1.613110in}}%
\pgfpathlineto{\pgfqpoint{9.561927in}{1.235680in}}%
\pgfpathlineto{\pgfqpoint{9.570734in}{1.661150in}}%
\pgfpathlineto{\pgfqpoint{9.579541in}{2.354309in}}%
\pgfpathlineto{\pgfqpoint{9.597154in}{2.120970in}}%
\pgfpathlineto{\pgfqpoint{9.605961in}{2.059213in}}%
\pgfpathlineto{\pgfqpoint{9.614768in}{1.887630in}}%
\pgfpathlineto{\pgfqpoint{9.623575in}{1.867054in}}%
\pgfpathlineto{\pgfqpoint{9.641188in}{1.743512in}}%
\pgfpathlineto{\pgfqpoint{9.649995in}{1.722935in}}%
\pgfpathlineto{\pgfqpoint{9.658802in}{1.668037in}}%
\pgfpathlineto{\pgfqpoint{9.667609in}{1.942529in}}%
\pgfpathlineto{\pgfqpoint{9.676416in}{2.059213in}}%
\pgfpathlineto{\pgfqpoint{9.685222in}{2.278806in}}%
\pgfpathlineto{\pgfqpoint{9.694029in}{2.237625in}}%
\pgfpathlineto{\pgfqpoint{9.702836in}{2.285665in}}%
\pgfpathlineto{\pgfqpoint{9.711643in}{1.997427in}}%
\pgfpathlineto{\pgfqpoint{9.729257in}{1.873913in}}%
\pgfpathlineto{\pgfqpoint{9.738063in}{2.072930in}}%
\pgfpathlineto{\pgfqpoint{9.746870in}{1.908207in}}%
\pgfpathlineto{\pgfqpoint{9.755677in}{1.668037in}}%
\pgfpathlineto{\pgfqpoint{9.764484in}{1.372912in}}%
\pgfpathlineto{\pgfqpoint{9.773291in}{1.462133in}}%
\pgfpathlineto{\pgfqpoint{9.782097in}{1.221935in}}%
\pgfpathlineto{\pgfqpoint{9.790904in}{1.331760in}}%
\pgfpathlineto{\pgfqpoint{9.799711in}{1.228822in}}%
\pgfpathlineto{\pgfqpoint{9.808518in}{1.359195in}}%
\pgfpathlineto{\pgfqpoint{9.817325in}{1.517031in}}%
\pgfpathlineto{\pgfqpoint{9.826132in}{1.455274in}}%
\pgfpathlineto{\pgfqpoint{9.834938in}{1.462133in}}%
\pgfpathlineto{\pgfqpoint{9.843745in}{1.606252in}}%
\pgfpathlineto{\pgfqpoint{9.852552in}{1.523918in}}%
\pgfpathlineto{\pgfqpoint{9.861359in}{1.743512in}}%
\pgfpathlineto{\pgfqpoint{9.870166in}{1.674896in}}%
\pgfpathlineto{\pgfqpoint{9.878972in}{1.462133in}}%
\pgfpathlineto{\pgfqpoint{9.887779in}{1.338618in}}%
\pgfpathlineto{\pgfqpoint{9.896586in}{1.599393in}}%
\pgfpathlineto{\pgfqpoint{9.905393in}{1.798410in}}%
\pgfpathlineto{\pgfqpoint{9.914200in}{1.668037in}}%
\pgfpathlineto{\pgfqpoint{9.923007in}{1.407234in}}%
\pgfpathlineto{\pgfqpoint{9.931813in}{1.647432in}}%
\pgfpathlineto{\pgfqpoint{9.940620in}{1.661150in}}%
\pgfpathlineto{\pgfqpoint{9.949427in}{1.359195in}}%
\pgfpathlineto{\pgfqpoint{9.949427in}{1.359195in}}%
\pgfusepath{stroke}%
\end{pgfscope}%
\begin{pgfscope}%
\pgfpathrectangle{\pgfqpoint{0.702268in}{0.521603in}}{\pgfqpoint{9.687500in}{4.235000in}}%
\pgfusepath{clip}%
\pgfsetrectcap%
\pgfsetroundjoin%
\pgfsetlinewidth{0.501875pt}%
\definecolor{currentstroke}{rgb}{0.501961,0.501961,0.501961}%
\pgfsetstrokecolor{currentstroke}%
\pgfsetstrokeopacity{0.250000}%
\pgfsetdash{}{0pt}%
\pgfpathmoveto{\pgfqpoint{1.142609in}{4.234685in}}%
\pgfpathlineto{\pgfqpoint{1.151416in}{3.218994in}}%
\pgfpathlineto{\pgfqpoint{1.160222in}{2.615084in}}%
\pgfpathlineto{\pgfqpoint{1.169029in}{2.772920in}}%
\pgfpathlineto{\pgfqpoint{1.177836in}{2.251343in}}%
\pgfpathlineto{\pgfqpoint{1.186643in}{2.127829in}}%
\pgfpathlineto{\pgfqpoint{1.195450in}{1.949388in}}%
\pgfpathlineto{\pgfqpoint{1.204257in}{1.503314in}}%
\pgfpathlineto{\pgfqpoint{1.213063in}{1.269974in}}%
\pgfpathlineto{\pgfqpoint{1.221870in}{1.379799in}}%
\pgfpathlineto{\pgfqpoint{1.230677in}{1.352336in}}%
\pgfpathlineto{\pgfqpoint{1.239484in}{1.118997in}}%
\pgfpathlineto{\pgfqpoint{1.248291in}{1.276861in}}%
\pgfpathlineto{\pgfqpoint{1.257097in}{1.263115in}}%
\pgfpathlineto{\pgfqpoint{1.265904in}{1.427839in}}%
\pgfpathlineto{\pgfqpoint{1.274711in}{1.235680in}}%
\pgfpathlineto{\pgfqpoint{1.283518in}{1.407234in}}%
\pgfpathlineto{\pgfqpoint{1.301132in}{1.873913in}}%
\pgfpathlineto{\pgfqpoint{1.309938in}{2.244484in}}%
\pgfpathlineto{\pgfqpoint{1.318745in}{2.196445in}}%
\pgfpathlineto{\pgfqpoint{1.327552in}{2.079789in}}%
\pgfpathlineto{\pgfqpoint{1.336359in}{2.155292in}}%
\pgfpathlineto{\pgfqpoint{1.345166in}{2.100365in}}%
\pgfpathlineto{\pgfqpoint{1.353972in}{1.812128in}}%
\pgfpathlineto{\pgfqpoint{1.362779in}{1.743512in}}%
\pgfpathlineto{\pgfqpoint{1.371586in}{1.606252in}}%
\pgfpathlineto{\pgfqpoint{1.380393in}{1.400376in}}%
\pgfpathlineto{\pgfqpoint{1.389200in}{1.695472in}}%
\pgfpathlineto{\pgfqpoint{1.398007in}{2.182727in}}%
\pgfpathlineto{\pgfqpoint{1.406813in}{2.134687in}}%
\pgfpathlineto{\pgfqpoint{1.415620in}{2.306269in}}%
\pgfpathlineto{\pgfqpoint{1.424427in}{2.018032in}}%
\pgfpathlineto{\pgfqpoint{1.433234in}{1.949388in}}%
\pgfpathlineto{\pgfqpoint{1.442041in}{2.244484in}}%
\pgfpathlineto{\pgfqpoint{1.450847in}{2.031749in}}%
\pgfpathlineto{\pgfqpoint{1.459654in}{1.942529in}}%
\pgfpathlineto{\pgfqpoint{1.468461in}{1.736653in}}%
\pgfpathlineto{\pgfqpoint{1.477268in}{1.873913in}}%
\pgfpathlineto{\pgfqpoint{1.486075in}{1.832732in}}%
\pgfpathlineto{\pgfqpoint{1.494882in}{2.024891in}}%
\pgfpathlineto{\pgfqpoint{1.503688in}{2.052326in}}%
\pgfpathlineto{\pgfqpoint{1.512495in}{1.894489in}}%
\pgfpathlineto{\pgfqpoint{1.521302in}{1.606252in}}%
\pgfpathlineto{\pgfqpoint{1.538916in}{1.949388in}}%
\pgfpathlineto{\pgfqpoint{1.547722in}{1.901348in}}%
\pgfpathlineto{\pgfqpoint{1.556529in}{2.024891in}}%
\pgfpathlineto{\pgfqpoint{1.565336in}{2.251343in}}%
\pgfpathlineto{\pgfqpoint{1.582950in}{1.633715in}}%
\pgfpathlineto{\pgfqpoint{1.591757in}{1.791551in}}%
\pgfpathlineto{\pgfqpoint{1.600563in}{1.585675in}}%
\pgfpathlineto{\pgfqpoint{1.609370in}{1.592534in}}%
\pgfpathlineto{\pgfqpoint{1.618177in}{1.668037in}}%
\pgfpathlineto{\pgfqpoint{1.626984in}{1.619997in}}%
\pgfpathlineto{\pgfqpoint{1.635791in}{1.695472in}}%
\pgfpathlineto{\pgfqpoint{1.644597in}{1.729794in}}%
\pgfpathlineto{\pgfqpoint{1.653404in}{1.791551in}}%
\pgfpathlineto{\pgfqpoint{1.662211in}{1.894489in}}%
\pgfpathlineto{\pgfqpoint{1.671018in}{1.860167in}}%
\pgfpathlineto{\pgfqpoint{1.679825in}{1.674896in}}%
\pgfpathlineto{\pgfqpoint{1.688632in}{2.203303in}}%
\pgfpathlineto{\pgfqpoint{1.697438in}{2.368027in}}%
\pgfpathlineto{\pgfqpoint{1.706245in}{1.908207in}}%
\pgfpathlineto{\pgfqpoint{1.715052in}{2.100365in}}%
\pgfpathlineto{\pgfqpoint{1.723859in}{1.777834in}}%
\pgfpathlineto{\pgfqpoint{1.732666in}{1.894489in}}%
\pgfpathlineto{\pgfqpoint{1.741472in}{1.414093in}}%
\pgfpathlineto{\pgfqpoint{1.750279in}{1.688613in}}%
\pgfpathlineto{\pgfqpoint{1.759086in}{1.716076in}}%
\pgfpathlineto{\pgfqpoint{1.767893in}{2.018032in}}%
\pgfpathlineto{\pgfqpoint{1.776700in}{1.853308in}}%
\pgfpathlineto{\pgfqpoint{1.785507in}{2.059213in}}%
\pgfpathlineto{\pgfqpoint{1.803120in}{2.004286in}}%
\pgfpathlineto{\pgfqpoint{1.811927in}{2.031749in}}%
\pgfpathlineto{\pgfqpoint{1.820734in}{1.832732in}}%
\pgfpathlineto{\pgfqpoint{1.829541in}{1.805269in}}%
\pgfpathlineto{\pgfqpoint{1.838347in}{1.935670in}}%
\pgfpathlineto{\pgfqpoint{1.847154in}{2.175868in}}%
\pgfpathlineto{\pgfqpoint{1.855961in}{2.203303in}}%
\pgfpathlineto{\pgfqpoint{1.864768in}{2.162151in}}%
\pgfpathlineto{\pgfqpoint{1.873575in}{1.949388in}}%
\pgfpathlineto{\pgfqpoint{1.882382in}{2.148405in}}%
\pgfpathlineto{\pgfqpoint{1.891188in}{2.093507in}}%
\pgfpathlineto{\pgfqpoint{1.899995in}{1.839591in}}%
\pgfpathlineto{\pgfqpoint{1.908802in}{1.819015in}}%
\pgfpathlineto{\pgfqpoint{1.917609in}{2.169009in}}%
\pgfpathlineto{\pgfqpoint{1.926416in}{2.175868in}}%
\pgfpathlineto{\pgfqpoint{1.935222in}{2.066071in}}%
\pgfpathlineto{\pgfqpoint{1.944029in}{2.066071in}}%
\pgfpathlineto{\pgfqpoint{1.952836in}{2.285665in}}%
\pgfpathlineto{\pgfqpoint{1.961643in}{2.217049in}}%
\pgfpathlineto{\pgfqpoint{1.970450in}{2.100365in}}%
\pgfpathlineto{\pgfqpoint{1.979257in}{1.873913in}}%
\pgfpathlineto{\pgfqpoint{1.988063in}{1.832732in}}%
\pgfpathlineto{\pgfqpoint{1.996870in}{2.066071in}}%
\pgfpathlineto{\pgfqpoint{2.005677in}{1.915094in}}%
\pgfpathlineto{\pgfqpoint{2.014484in}{1.873913in}}%
\pgfpathlineto{\pgfqpoint{2.023291in}{1.867054in}}%
\pgfpathlineto{\pgfqpoint{2.032097in}{1.956246in}}%
\pgfpathlineto{\pgfqpoint{2.040904in}{1.928811in}}%
\pgfpathlineto{\pgfqpoint{2.049711in}{1.750370in}}%
\pgfpathlineto{\pgfqpoint{2.058518in}{1.455274in}}%
\pgfpathlineto{\pgfqpoint{2.067325in}{1.503314in}}%
\pgfpathlineto{\pgfqpoint{2.076132in}{1.537636in}}%
\pgfpathlineto{\pgfqpoint{2.084938in}{1.393517in}}%
\pgfpathlineto{\pgfqpoint{2.093745in}{1.770975in}}%
\pgfpathlineto{\pgfqpoint{2.102552in}{1.949388in}}%
\pgfpathlineto{\pgfqpoint{2.111359in}{1.997427in}}%
\pgfpathlineto{\pgfqpoint{2.120166in}{1.921953in}}%
\pgfpathlineto{\pgfqpoint{2.128972in}{2.024891in}}%
\pgfpathlineto{\pgfqpoint{2.137779in}{2.319987in}}%
\pgfpathlineto{\pgfqpoint{2.146586in}{2.313128in}}%
\pgfpathlineto{\pgfqpoint{2.155393in}{2.683700in}}%
\pgfpathlineto{\pgfqpoint{2.164200in}{2.422925in}}%
\pgfpathlineto{\pgfqpoint{2.173007in}{1.969992in}}%
\pgfpathlineto{\pgfqpoint{2.181813in}{1.935670in}}%
\pgfpathlineto{\pgfqpoint{2.190620in}{2.182727in}}%
\pgfpathlineto{\pgfqpoint{2.199427in}{1.853308in}}%
\pgfpathlineto{\pgfqpoint{2.208234in}{1.921953in}}%
\pgfpathlineto{\pgfqpoint{2.217041in}{1.825873in}}%
\pgfpathlineto{\pgfqpoint{2.225847in}{1.846450in}}%
\pgfpathlineto{\pgfqpoint{2.234654in}{2.072930in}}%
\pgfpathlineto{\pgfqpoint{2.243461in}{1.880772in}}%
\pgfpathlineto{\pgfqpoint{2.252268in}{1.880772in}}%
\pgfpathlineto{\pgfqpoint{2.261075in}{2.011173in}}%
\pgfpathlineto{\pgfqpoint{2.269882in}{2.120970in}}%
\pgfpathlineto{\pgfqpoint{2.278688in}{2.038608in}}%
\pgfpathlineto{\pgfqpoint{2.296302in}{1.681754in}}%
\pgfpathlineto{\pgfqpoint{2.305109in}{1.921953in}}%
\pgfpathlineto{\pgfqpoint{2.313916in}{1.963133in}}%
\pgfpathlineto{\pgfqpoint{2.322722in}{2.182727in}}%
\pgfpathlineto{\pgfqpoint{2.331529in}{2.100365in}}%
\pgfpathlineto{\pgfqpoint{2.340336in}{2.203303in}}%
\pgfpathlineto{\pgfqpoint{2.349143in}{2.086648in}}%
\pgfpathlineto{\pgfqpoint{2.357950in}{2.196445in}}%
\pgfpathlineto{\pgfqpoint{2.366757in}{2.608225in}}%
\pgfpathlineto{\pgfqpoint{2.375563in}{2.580761in}}%
\pgfpathlineto{\pgfqpoint{2.393177in}{2.388603in}}%
\pgfpathlineto{\pgfqpoint{2.401984in}{2.278806in}}%
\pgfpathlineto{\pgfqpoint{2.410791in}{2.100365in}}%
\pgfpathlineto{\pgfqpoint{2.419597in}{1.887630in}}%
\pgfpathlineto{\pgfqpoint{2.428404in}{1.702331in}}%
\pgfpathlineto{\pgfqpoint{2.437211in}{1.619997in}}%
\pgfpathlineto{\pgfqpoint{2.446018in}{1.729794in}}%
\pgfpathlineto{\pgfqpoint{2.454825in}{1.592534in}}%
\pgfpathlineto{\pgfqpoint{2.472438in}{1.819015in}}%
\pgfpathlineto{\pgfqpoint{2.481245in}{2.024891in}}%
\pgfpathlineto{\pgfqpoint{2.490052in}{2.004286in}}%
\pgfpathlineto{\pgfqpoint{2.498859in}{2.107252in}}%
\pgfpathlineto{\pgfqpoint{2.507666in}{2.155292in}}%
\pgfpathlineto{\pgfqpoint{2.516472in}{1.935670in}}%
\pgfpathlineto{\pgfqpoint{2.525279in}{2.072930in}}%
\pgfpathlineto{\pgfqpoint{2.534086in}{2.031749in}}%
\pgfpathlineto{\pgfqpoint{2.542893in}{2.093507in}}%
\pgfpathlineto{\pgfqpoint{2.551700in}{2.285665in}}%
\pgfpathlineto{\pgfqpoint{2.560507in}{2.278806in}}%
\pgfpathlineto{\pgfqpoint{2.569313in}{1.963133in}}%
\pgfpathlineto{\pgfqpoint{2.586927in}{2.120970in}}%
\pgfpathlineto{\pgfqpoint{2.595734in}{2.169009in}}%
\pgfpathlineto{\pgfqpoint{2.604541in}{2.292524in}}%
\pgfpathlineto{\pgfqpoint{2.613347in}{2.093507in}}%
\pgfpathlineto{\pgfqpoint{2.622154in}{2.011173in}}%
\pgfpathlineto{\pgfqpoint{2.630961in}{2.066071in}}%
\pgfpathlineto{\pgfqpoint{2.639768in}{1.956246in}}%
\pgfpathlineto{\pgfqpoint{2.648575in}{2.196445in}}%
\pgfpathlineto{\pgfqpoint{2.657382in}{1.805269in}}%
\pgfpathlineto{\pgfqpoint{2.666188in}{1.777834in}}%
\pgfpathlineto{\pgfqpoint{2.674995in}{1.352336in}}%
\pgfpathlineto{\pgfqpoint{2.683802in}{1.455274in}}%
\pgfpathlineto{\pgfqpoint{2.692609in}{1.434698in}}%
\pgfpathlineto{\pgfqpoint{2.701416in}{1.427839in}}%
\pgfpathlineto{\pgfqpoint{2.710222in}{1.716076in}}%
\pgfpathlineto{\pgfqpoint{2.719029in}{1.489596in}}%
\pgfpathlineto{\pgfqpoint{2.727836in}{1.716076in}}%
\pgfpathlineto{\pgfqpoint{2.736643in}{2.066071in}}%
\pgfpathlineto{\pgfqpoint{2.754257in}{1.661150in}}%
\pgfpathlineto{\pgfqpoint{2.763063in}{1.784692in}}%
\pgfpathlineto{\pgfqpoint{2.771870in}{1.880772in}}%
\pgfpathlineto{\pgfqpoint{2.780677in}{1.784692in}}%
\pgfpathlineto{\pgfqpoint{2.789484in}{1.565071in}}%
\pgfpathlineto{\pgfqpoint{2.798291in}{1.551353in}}%
\pgfpathlineto{\pgfqpoint{2.807097in}{1.668037in}}%
\pgfpathlineto{\pgfqpoint{2.815904in}{1.688613in}}%
\pgfpathlineto{\pgfqpoint{2.824711in}{1.873913in}}%
\pgfpathlineto{\pgfqpoint{2.833518in}{2.155292in}}%
\pgfpathlineto{\pgfqpoint{2.842325in}{2.395462in}}%
\pgfpathlineto{\pgfqpoint{2.851132in}{2.313128in}}%
\pgfpathlineto{\pgfqpoint{2.868745in}{1.729794in}}%
\pgfpathlineto{\pgfqpoint{2.877552in}{1.688613in}}%
\pgfpathlineto{\pgfqpoint{2.886359in}{1.750370in}}%
\pgfpathlineto{\pgfqpoint{2.895166in}{1.921953in}}%
\pgfpathlineto{\pgfqpoint{2.903972in}{2.237625in}}%
\pgfpathlineto{\pgfqpoint{2.912779in}{2.004286in}}%
\pgfpathlineto{\pgfqpoint{2.921586in}{1.681754in}}%
\pgfpathlineto{\pgfqpoint{2.930393in}{1.592534in}}%
\pgfpathlineto{\pgfqpoint{2.939200in}{1.661150in}}%
\pgfpathlineto{\pgfqpoint{2.948007in}{1.606252in}}%
\pgfpathlineto{\pgfqpoint{2.956813in}{1.681754in}}%
\pgfpathlineto{\pgfqpoint{2.965620in}{1.880772in}}%
\pgfpathlineto{\pgfqpoint{2.974427in}{1.990569in}}%
\pgfpathlineto{\pgfqpoint{2.983234in}{2.175868in}}%
\pgfpathlineto{\pgfqpoint{2.992041in}{1.867054in}}%
\pgfpathlineto{\pgfqpoint{3.000847in}{1.997427in}}%
\pgfpathlineto{\pgfqpoint{3.009654in}{1.674896in}}%
\pgfpathlineto{\pgfqpoint{3.018461in}{1.750370in}}%
\pgfpathlineto{\pgfqpoint{3.027268in}{2.169009in}}%
\pgfpathlineto{\pgfqpoint{3.036075in}{1.867054in}}%
\pgfpathlineto{\pgfqpoint{3.044882in}{1.963133in}}%
\pgfpathlineto{\pgfqpoint{3.053688in}{2.217049in}}%
\pgfpathlineto{\pgfqpoint{3.062495in}{2.416066in}}%
\pgfpathlineto{\pgfqpoint{3.071302in}{2.395462in}}%
\pgfpathlineto{\pgfqpoint{3.080109in}{1.873913in}}%
\pgfpathlineto{\pgfqpoint{3.088916in}{2.059213in}}%
\pgfpathlineto{\pgfqpoint{3.097722in}{1.812128in}}%
\pgfpathlineto{\pgfqpoint{3.106529in}{1.743512in}}%
\pgfpathlineto{\pgfqpoint{3.115336in}{1.825873in}}%
\pgfpathlineto{\pgfqpoint{3.124143in}{1.688613in}}%
\pgfpathlineto{\pgfqpoint{3.132950in}{1.928811in}}%
\pgfpathlineto{\pgfqpoint{3.141757in}{1.983710in}}%
\pgfpathlineto{\pgfqpoint{3.150563in}{2.011173in}}%
\pgfpathlineto{\pgfqpoint{3.159370in}{2.210190in}}%
\pgfpathlineto{\pgfqpoint{3.168177in}{2.169009in}}%
\pgfpathlineto{\pgfqpoint{3.176984in}{2.059213in}}%
\pgfpathlineto{\pgfqpoint{3.185791in}{1.983710in}}%
\pgfpathlineto{\pgfqpoint{3.194597in}{1.819015in}}%
\pgfpathlineto{\pgfqpoint{3.203404in}{2.004286in}}%
\pgfpathlineto{\pgfqpoint{3.212211in}{2.148405in}}%
\pgfpathlineto{\pgfqpoint{3.221018in}{2.107252in}}%
\pgfpathlineto{\pgfqpoint{3.229825in}{2.079789in}}%
\pgfpathlineto{\pgfqpoint{3.238632in}{2.182727in}}%
\pgfpathlineto{\pgfqpoint{3.247438in}{2.196445in}}%
\pgfpathlineto{\pgfqpoint{3.256245in}{1.990569in}}%
\pgfpathlineto{\pgfqpoint{3.265052in}{1.983710in}}%
\pgfpathlineto{\pgfqpoint{3.273859in}{1.969992in}}%
\pgfpathlineto{\pgfqpoint{3.282666in}{1.949388in}}%
\pgfpathlineto{\pgfqpoint{3.291472in}{2.038608in}}%
\pgfpathlineto{\pgfqpoint{3.300279in}{2.409207in}}%
\pgfpathlineto{\pgfqpoint{3.309086in}{2.395462in}}%
\pgfpathlineto{\pgfqpoint{3.317893in}{2.711163in}}%
\pgfpathlineto{\pgfqpoint{3.326700in}{2.766061in}}%
\pgfpathlineto{\pgfqpoint{3.335507in}{2.683700in}}%
\pgfpathlineto{\pgfqpoint{3.344313in}{2.724880in}}%
\pgfpathlineto{\pgfqpoint{3.353120in}{2.402349in}}%
\pgfpathlineto{\pgfqpoint{3.361927in}{2.292524in}}%
\pgfpathlineto{\pgfqpoint{3.370734in}{2.155292in}}%
\pgfpathlineto{\pgfqpoint{3.379541in}{2.059213in}}%
\pgfpathlineto{\pgfqpoint{3.388347in}{1.688613in}}%
\pgfpathlineto{\pgfqpoint{3.397154in}{1.846450in}}%
\pgfpathlineto{\pgfqpoint{3.405961in}{2.148405in}}%
\pgfpathlineto{\pgfqpoint{3.414768in}{2.018032in}}%
\pgfpathlineto{\pgfqpoint{3.423575in}{1.743512in}}%
\pgfpathlineto{\pgfqpoint{3.432382in}{1.832732in}}%
\pgfpathlineto{\pgfqpoint{3.441188in}{1.503314in}}%
\pgfpathlineto{\pgfqpoint{3.449995in}{1.674896in}}%
\pgfpathlineto{\pgfqpoint{3.458802in}{1.523918in}}%
\pgfpathlineto{\pgfqpoint{3.467609in}{1.475878in}}%
\pgfpathlineto{\pgfqpoint{3.476416in}{1.503314in}}%
\pgfpathlineto{\pgfqpoint{3.485222in}{1.517031in}}%
\pgfpathlineto{\pgfqpoint{3.494029in}{1.688613in}}%
\pgfpathlineto{\pgfqpoint{3.502836in}{1.352336in}}%
\pgfpathlineto{\pgfqpoint{3.511643in}{1.517031in}}%
\pgfpathlineto{\pgfqpoint{3.520450in}{1.784692in}}%
\pgfpathlineto{\pgfqpoint{3.529257in}{1.729794in}}%
\pgfpathlineto{\pgfqpoint{3.546870in}{1.599393in}}%
\pgfpathlineto{\pgfqpoint{3.555677in}{1.668037in}}%
\pgfpathlineto{\pgfqpoint{3.564484in}{1.805269in}}%
\pgfpathlineto{\pgfqpoint{3.573291in}{2.031749in}}%
\pgfpathlineto{\pgfqpoint{3.582097in}{1.757229in}}%
\pgfpathlineto{\pgfqpoint{3.590904in}{1.867054in}}%
\pgfpathlineto{\pgfqpoint{3.599711in}{1.613110in}}%
\pgfpathlineto{\pgfqpoint{3.608518in}{1.400376in}}%
\pgfpathlineto{\pgfqpoint{3.617325in}{1.311155in}}%
\pgfpathlineto{\pgfqpoint{3.626132in}{1.626856in}}%
\pgfpathlineto{\pgfqpoint{3.634938in}{2.072930in}}%
\pgfpathlineto{\pgfqpoint{3.643745in}{2.182727in}}%
\pgfpathlineto{\pgfqpoint{3.652552in}{2.045467in}}%
\pgfpathlineto{\pgfqpoint{3.661359in}{2.368027in}}%
\pgfpathlineto{\pgfqpoint{3.670166in}{2.203303in}}%
\pgfpathlineto{\pgfqpoint{3.678972in}{2.120970in}}%
\pgfpathlineto{\pgfqpoint{3.687779in}{2.265089in}}%
\pgfpathlineto{\pgfqpoint{3.696586in}{2.148405in}}%
\pgfpathlineto{\pgfqpoint{3.705393in}{1.942529in}}%
\pgfpathlineto{\pgfqpoint{3.714200in}{1.702331in}}%
\pgfpathlineto{\pgfqpoint{3.723007in}{1.681754in}}%
\pgfpathlineto{\pgfqpoint{3.731813in}{1.716076in}}%
\pgfpathlineto{\pgfqpoint{3.740620in}{1.901348in}}%
\pgfpathlineto{\pgfqpoint{3.758234in}{2.059213in}}%
\pgfpathlineto{\pgfqpoint{3.767041in}{2.217049in}}%
\pgfpathlineto{\pgfqpoint{3.775847in}{2.024891in}}%
\pgfpathlineto{\pgfqpoint{3.784654in}{2.333705in}}%
\pgfpathlineto{\pgfqpoint{3.793461in}{2.347422in}}%
\pgfpathlineto{\pgfqpoint{3.802268in}{1.908207in}}%
\pgfpathlineto{\pgfqpoint{3.811075in}{1.887630in}}%
\pgfpathlineto{\pgfqpoint{3.819882in}{1.825873in}}%
\pgfpathlineto{\pgfqpoint{3.828688in}{1.565071in}}%
\pgfpathlineto{\pgfqpoint{3.837495in}{1.798410in}}%
\pgfpathlineto{\pgfqpoint{3.846302in}{2.525863in}}%
\pgfpathlineto{\pgfqpoint{3.855109in}{2.217049in}}%
\pgfpathlineto{\pgfqpoint{3.872722in}{2.361168in}}%
\pgfpathlineto{\pgfqpoint{3.881529in}{1.764116in}}%
\pgfpathlineto{\pgfqpoint{3.890336in}{2.141546in}}%
\pgfpathlineto{\pgfqpoint{3.899143in}{1.908207in}}%
\pgfpathlineto{\pgfqpoint{3.907950in}{1.743512in}}%
\pgfpathlineto{\pgfqpoint{3.916757in}{2.237625in}}%
\pgfpathlineto{\pgfqpoint{3.925563in}{2.203303in}}%
\pgfpathlineto{\pgfqpoint{3.934370in}{1.969992in}}%
\pgfpathlineto{\pgfqpoint{3.943177in}{2.045467in}}%
\pgfpathlineto{\pgfqpoint{3.951984in}{1.544494in}}%
\pgfpathlineto{\pgfqpoint{3.960791in}{1.695472in}}%
\pgfpathlineto{\pgfqpoint{3.969597in}{1.455274in}}%
\pgfpathlineto{\pgfqpoint{3.978404in}{1.544494in}}%
\pgfpathlineto{\pgfqpoint{3.987211in}{1.681754in}}%
\pgfpathlineto{\pgfqpoint{3.996018in}{1.757229in}}%
\pgfpathlineto{\pgfqpoint{4.004825in}{1.956246in}}%
\pgfpathlineto{\pgfqpoint{4.013632in}{2.210190in}}%
\pgfpathlineto{\pgfqpoint{4.022438in}{2.169009in}}%
\pgfpathlineto{\pgfqpoint{4.031245in}{2.045467in}}%
\pgfpathlineto{\pgfqpoint{4.040052in}{2.450388in}}%
\pgfpathlineto{\pgfqpoint{4.048859in}{2.532722in}}%
\pgfpathlineto{\pgfqpoint{4.057666in}{2.450388in}}%
\pgfpathlineto{\pgfqpoint{4.066472in}{2.422925in}}%
\pgfpathlineto{\pgfqpoint{4.075279in}{2.134687in}}%
\pgfpathlineto{\pgfqpoint{4.084086in}{2.162151in}}%
\pgfpathlineto{\pgfqpoint{4.092893in}{1.928811in}}%
\pgfpathlineto{\pgfqpoint{4.101700in}{2.100365in}}%
\pgfpathlineto{\pgfqpoint{4.110507in}{2.031749in}}%
\pgfpathlineto{\pgfqpoint{4.119313in}{2.114111in}}%
\pgfpathlineto{\pgfqpoint{4.128120in}{2.422925in}}%
\pgfpathlineto{\pgfqpoint{4.136927in}{2.429784in}}%
\pgfpathlineto{\pgfqpoint{4.145734in}{1.846450in}}%
\pgfpathlineto{\pgfqpoint{4.154541in}{2.210190in}}%
\pgfpathlineto{\pgfqpoint{4.163347in}{2.127829in}}%
\pgfpathlineto{\pgfqpoint{4.172154in}{2.230767in}}%
\pgfpathlineto{\pgfqpoint{4.180961in}{2.141546in}}%
\pgfpathlineto{\pgfqpoint{4.189768in}{2.018032in}}%
\pgfpathlineto{\pgfqpoint{4.198575in}{2.072930in}}%
\pgfpathlineto{\pgfqpoint{4.207382in}{1.729794in}}%
\pgfpathlineto{\pgfqpoint{4.224995in}{2.066071in}}%
\pgfpathlineto{\pgfqpoint{4.233802in}{2.127829in}}%
\pgfpathlineto{\pgfqpoint{4.242609in}{1.887630in}}%
\pgfpathlineto{\pgfqpoint{4.251416in}{1.805269in}}%
\pgfpathlineto{\pgfqpoint{4.260222in}{1.839591in}}%
\pgfpathlineto{\pgfqpoint{4.269029in}{1.880772in}}%
\pgfpathlineto{\pgfqpoint{4.277836in}{1.674896in}}%
\pgfpathlineto{\pgfqpoint{4.286643in}{1.716076in}}%
\pgfpathlineto{\pgfqpoint{4.295450in}{2.072930in}}%
\pgfpathlineto{\pgfqpoint{4.313063in}{2.196445in}}%
\pgfpathlineto{\pgfqpoint{4.321870in}{1.791551in}}%
\pgfpathlineto{\pgfqpoint{4.330677in}{1.853308in}}%
\pgfpathlineto{\pgfqpoint{4.339484in}{1.750370in}}%
\pgfpathlineto{\pgfqpoint{4.348291in}{1.805269in}}%
\pgfpathlineto{\pgfqpoint{4.357097in}{1.668037in}}%
\pgfpathlineto{\pgfqpoint{4.365904in}{1.640574in}}%
\pgfpathlineto{\pgfqpoint{4.374711in}{1.853308in}}%
\pgfpathlineto{\pgfqpoint{4.383518in}{1.949388in}}%
\pgfpathlineto{\pgfqpoint{4.392325in}{1.963133in}}%
\pgfpathlineto{\pgfqpoint{4.401132in}{1.921953in}}%
\pgfpathlineto{\pgfqpoint{4.409938in}{2.120970in}}%
\pgfpathlineto{\pgfqpoint{4.418745in}{1.928811in}}%
\pgfpathlineto{\pgfqpoint{4.427552in}{2.127829in}}%
\pgfpathlineto{\pgfqpoint{4.436359in}{2.114111in}}%
\pgfpathlineto{\pgfqpoint{4.445166in}{2.251343in}}%
\pgfpathlineto{\pgfqpoint{4.453972in}{2.093507in}}%
\pgfpathlineto{\pgfqpoint{4.462779in}{1.963133in}}%
\pgfpathlineto{\pgfqpoint{4.471586in}{1.860167in}}%
\pgfpathlineto{\pgfqpoint{4.480393in}{1.928811in}}%
\pgfpathlineto{\pgfqpoint{4.489200in}{2.175868in}}%
\pgfpathlineto{\pgfqpoint{4.498007in}{2.059213in}}%
\pgfpathlineto{\pgfqpoint{4.506813in}{2.114111in}}%
\pgfpathlineto{\pgfqpoint{4.515620in}{2.052326in}}%
\pgfpathlineto{\pgfqpoint{4.524427in}{2.244484in}}%
\pgfpathlineto{\pgfqpoint{4.533234in}{2.299383in}}%
\pgfpathlineto{\pgfqpoint{4.550847in}{1.777834in}}%
\pgfpathlineto{\pgfqpoint{4.559654in}{1.647432in}}%
\pgfpathlineto{\pgfqpoint{4.568461in}{1.757229in}}%
\pgfpathlineto{\pgfqpoint{4.577268in}{1.578816in}}%
\pgfpathlineto{\pgfqpoint{4.586075in}{1.908207in}}%
\pgfpathlineto{\pgfqpoint{4.594882in}{1.997427in}}%
\pgfpathlineto{\pgfqpoint{4.603688in}{2.326846in}}%
\pgfpathlineto{\pgfqpoint{4.612495in}{2.347422in}}%
\pgfpathlineto{\pgfqpoint{4.621302in}{2.299383in}}%
\pgfpathlineto{\pgfqpoint{4.630109in}{2.072930in}}%
\pgfpathlineto{\pgfqpoint{4.638916in}{2.134687in}}%
\pgfpathlineto{\pgfqpoint{4.647722in}{2.162151in}}%
\pgfpathlineto{\pgfqpoint{4.656529in}{2.155292in}}%
\pgfpathlineto{\pgfqpoint{4.665336in}{1.894489in}}%
\pgfpathlineto{\pgfqpoint{4.674143in}{1.791551in}}%
\pgfpathlineto{\pgfqpoint{4.682950in}{1.599393in}}%
\pgfpathlineto{\pgfqpoint{4.691757in}{1.592534in}}%
\pgfpathlineto{\pgfqpoint{4.700563in}{1.407234in}}%
\pgfpathlineto{\pgfqpoint{4.709370in}{1.420952in}}%
\pgfpathlineto{\pgfqpoint{4.718177in}{1.276861in}}%
\pgfpathlineto{\pgfqpoint{4.726984in}{1.249398in}}%
\pgfpathlineto{\pgfqpoint{4.735791in}{1.606252in}}%
\pgfpathlineto{\pgfqpoint{4.744597in}{1.901348in}}%
\pgfpathlineto{\pgfqpoint{4.753404in}{2.024891in}}%
\pgfpathlineto{\pgfqpoint{4.762211in}{1.695472in}}%
\pgfpathlineto{\pgfqpoint{4.771018in}{1.764116in}}%
\pgfpathlineto{\pgfqpoint{4.779825in}{1.681754in}}%
\pgfpathlineto{\pgfqpoint{4.788632in}{1.743512in}}%
\pgfpathlineto{\pgfqpoint{4.797438in}{1.976851in}}%
\pgfpathlineto{\pgfqpoint{4.806245in}{2.052326in}}%
\pgfpathlineto{\pgfqpoint{4.815052in}{2.024891in}}%
\pgfpathlineto{\pgfqpoint{4.823859in}{1.908207in}}%
\pgfpathlineto{\pgfqpoint{4.832666in}{1.654291in}}%
\pgfpathlineto{\pgfqpoint{4.841472in}{1.709190in}}%
\pgfpathlineto{\pgfqpoint{4.850279in}{2.086648in}}%
\pgfpathlineto{\pgfqpoint{4.859086in}{2.340563in}}%
\pgfpathlineto{\pgfqpoint{4.867893in}{2.004286in}}%
\pgfpathlineto{\pgfqpoint{4.876700in}{2.175868in}}%
\pgfpathlineto{\pgfqpoint{4.885507in}{2.175868in}}%
\pgfpathlineto{\pgfqpoint{4.894313in}{2.361168in}}%
\pgfpathlineto{\pgfqpoint{4.903120in}{2.313128in}}%
\pgfpathlineto{\pgfqpoint{4.911927in}{2.505287in}}%
\pgfpathlineto{\pgfqpoint{4.920734in}{2.271947in}}%
\pgfpathlineto{\pgfqpoint{4.929541in}{2.134687in}}%
\pgfpathlineto{\pgfqpoint{4.938347in}{2.018032in}}%
\pgfpathlineto{\pgfqpoint{4.947154in}{1.873913in}}%
\pgfpathlineto{\pgfqpoint{4.955961in}{1.613110in}}%
\pgfpathlineto{\pgfqpoint{4.964768in}{1.551353in}}%
\pgfpathlineto{\pgfqpoint{4.973575in}{1.235680in}}%
\pgfpathlineto{\pgfqpoint{4.982382in}{1.139601in}}%
\pgfpathlineto{\pgfqpoint{4.991188in}{1.077816in}}%
\pgfpathlineto{\pgfqpoint{4.999995in}{1.160177in}}%
\pgfpathlineto{\pgfqpoint{5.008802in}{1.750370in}}%
\pgfpathlineto{\pgfqpoint{5.017609in}{1.846450in}}%
\pgfpathlineto{\pgfqpoint{5.026416in}{1.784692in}}%
\pgfpathlineto{\pgfqpoint{5.035222in}{1.949388in}}%
\pgfpathlineto{\pgfqpoint{5.044029in}{1.901348in}}%
\pgfpathlineto{\pgfqpoint{5.061643in}{1.420952in}}%
\pgfpathlineto{\pgfqpoint{5.070450in}{1.283720in}}%
\pgfpathlineto{\pgfqpoint{5.079257in}{1.359195in}}%
\pgfpathlineto{\pgfqpoint{5.088063in}{1.400376in}}%
\pgfpathlineto{\pgfqpoint{5.096870in}{1.654291in}}%
\pgfpathlineto{\pgfqpoint{5.105677in}{1.770975in}}%
\pgfpathlineto{\pgfqpoint{5.114484in}{1.716076in}}%
\pgfpathlineto{\pgfqpoint{5.123291in}{1.544494in}}%
\pgfpathlineto{\pgfqpoint{5.132097in}{1.846450in}}%
\pgfpathlineto{\pgfqpoint{5.140904in}{1.997427in}}%
\pgfpathlineto{\pgfqpoint{5.149711in}{1.997427in}}%
\pgfpathlineto{\pgfqpoint{5.158518in}{1.805269in}}%
\pgfpathlineto{\pgfqpoint{5.167325in}{1.647432in}}%
\pgfpathlineto{\pgfqpoint{5.176132in}{1.839591in}}%
\pgfpathlineto{\pgfqpoint{5.184938in}{1.832732in}}%
\pgfpathlineto{\pgfqpoint{5.193745in}{2.011173in}}%
\pgfpathlineto{\pgfqpoint{5.202552in}{2.388603in}}%
\pgfpathlineto{\pgfqpoint{5.211359in}{2.484682in}}%
\pgfpathlineto{\pgfqpoint{5.220166in}{2.134687in}}%
\pgfpathlineto{\pgfqpoint{5.228972in}{2.223908in}}%
\pgfpathlineto{\pgfqpoint{5.237779in}{1.983710in}}%
\pgfpathlineto{\pgfqpoint{5.246586in}{1.887630in}}%
\pgfpathlineto{\pgfqpoint{5.255393in}{1.976851in}}%
\pgfpathlineto{\pgfqpoint{5.264200in}{1.976851in}}%
\pgfpathlineto{\pgfqpoint{5.281813in}{1.517031in}}%
\pgfpathlineto{\pgfqpoint{5.290620in}{1.674896in}}%
\pgfpathlineto{\pgfqpoint{5.299427in}{1.942529in}}%
\pgfpathlineto{\pgfqpoint{5.308234in}{1.743512in}}%
\pgfpathlineto{\pgfqpoint{5.317041in}{2.100365in}}%
\pgfpathlineto{\pgfqpoint{5.325847in}{2.319987in}}%
\pgfpathlineto{\pgfqpoint{5.334654in}{2.072930in}}%
\pgfpathlineto{\pgfqpoint{5.343461in}{2.189586in}}%
\pgfpathlineto{\pgfqpoint{5.352268in}{2.237625in}}%
\pgfpathlineto{\pgfqpoint{5.361075in}{1.901348in}}%
\pgfpathlineto{\pgfqpoint{5.369882in}{2.120970in}}%
\pgfpathlineto{\pgfqpoint{5.378688in}{2.223908in}}%
\pgfpathlineto{\pgfqpoint{5.387495in}{2.182727in}}%
\pgfpathlineto{\pgfqpoint{5.396302in}{2.422925in}}%
\pgfpathlineto{\pgfqpoint{5.405109in}{2.525863in}}%
\pgfpathlineto{\pgfqpoint{5.413916in}{2.594507in}}%
\pgfpathlineto{\pgfqpoint{5.422722in}{1.997427in}}%
\pgfpathlineto{\pgfqpoint{5.431529in}{2.271947in}}%
\pgfpathlineto{\pgfqpoint{5.449143in}{1.901348in}}%
\pgfpathlineto{\pgfqpoint{5.457950in}{1.619997in}}%
\pgfpathlineto{\pgfqpoint{5.466757in}{2.141546in}}%
\pgfpathlineto{\pgfqpoint{5.475563in}{2.093507in}}%
\pgfpathlineto{\pgfqpoint{5.484370in}{2.251343in}}%
\pgfpathlineto{\pgfqpoint{5.501984in}{1.894489in}}%
\pgfpathlineto{\pgfqpoint{5.510791in}{2.093507in}}%
\pgfpathlineto{\pgfqpoint{5.519597in}{1.839591in}}%
\pgfpathlineto{\pgfqpoint{5.528404in}{1.887630in}}%
\pgfpathlineto{\pgfqpoint{5.537211in}{1.757229in}}%
\pgfpathlineto{\pgfqpoint{5.546018in}{1.647432in}}%
\pgfpathlineto{\pgfqpoint{5.554825in}{1.619997in}}%
\pgfpathlineto{\pgfqpoint{5.563632in}{1.853308in}}%
\pgfpathlineto{\pgfqpoint{5.572438in}{2.244484in}}%
\pgfpathlineto{\pgfqpoint{5.581245in}{2.066071in}}%
\pgfpathlineto{\pgfqpoint{5.590052in}{1.688613in}}%
\pgfpathlineto{\pgfqpoint{5.598859in}{1.867054in}}%
\pgfpathlineto{\pgfqpoint{5.607666in}{2.422925in}}%
\pgfpathlineto{\pgfqpoint{5.616472in}{2.086648in}}%
\pgfpathlineto{\pgfqpoint{5.625279in}{2.196445in}}%
\pgfpathlineto{\pgfqpoint{5.634086in}{1.963133in}}%
\pgfpathlineto{\pgfqpoint{5.642893in}{2.038608in}}%
\pgfpathlineto{\pgfqpoint{5.651700in}{2.203303in}}%
\pgfpathlineto{\pgfqpoint{5.660507in}{2.169009in}}%
\pgfpathlineto{\pgfqpoint{5.669313in}{2.086648in}}%
\pgfpathlineto{\pgfqpoint{5.678120in}{1.915094in}}%
\pgfpathlineto{\pgfqpoint{5.686927in}{1.873913in}}%
\pgfpathlineto{\pgfqpoint{5.695734in}{1.565071in}}%
\pgfpathlineto{\pgfqpoint{5.704541in}{2.120970in}}%
\pgfpathlineto{\pgfqpoint{5.713347in}{2.114111in}}%
\pgfpathlineto{\pgfqpoint{5.722154in}{2.018032in}}%
\pgfpathlineto{\pgfqpoint{5.730961in}{2.052326in}}%
\pgfpathlineto{\pgfqpoint{5.739768in}{1.757229in}}%
\pgfpathlineto{\pgfqpoint{5.748575in}{1.647432in}}%
\pgfpathlineto{\pgfqpoint{5.757382in}{1.565071in}}%
\pgfpathlineto{\pgfqpoint{5.766188in}{1.846450in}}%
\pgfpathlineto{\pgfqpoint{5.774995in}{1.729794in}}%
\pgfpathlineto{\pgfqpoint{5.783802in}{1.894489in}}%
\pgfpathlineto{\pgfqpoint{5.792609in}{1.770975in}}%
\pgfpathlineto{\pgfqpoint{5.801416in}{1.894489in}}%
\pgfpathlineto{\pgfqpoint{5.810222in}{2.313128in}}%
\pgfpathlineto{\pgfqpoint{5.819029in}{2.285665in}}%
\pgfpathlineto{\pgfqpoint{5.827836in}{2.340563in}}%
\pgfpathlineto{\pgfqpoint{5.836643in}{1.901348in}}%
\pgfpathlineto{\pgfqpoint{5.845450in}{2.114111in}}%
\pgfpathlineto{\pgfqpoint{5.854257in}{2.223908in}}%
\pgfpathlineto{\pgfqpoint{5.863063in}{1.832732in}}%
\pgfpathlineto{\pgfqpoint{5.871870in}{2.093507in}}%
\pgfpathlineto{\pgfqpoint{5.880677in}{2.024891in}}%
\pgfpathlineto{\pgfqpoint{5.889484in}{2.141546in}}%
\pgfpathlineto{\pgfqpoint{5.898291in}{1.887630in}}%
\pgfpathlineto{\pgfqpoint{5.907097in}{1.867054in}}%
\pgfpathlineto{\pgfqpoint{5.915904in}{1.853308in}}%
\pgfpathlineto{\pgfqpoint{5.924711in}{2.038608in}}%
\pgfpathlineto{\pgfqpoint{5.933518in}{2.045467in}}%
\pgfpathlineto{\pgfqpoint{5.942325in}{2.120970in}}%
\pgfpathlineto{\pgfqpoint{5.951132in}{2.347422in}}%
\pgfpathlineto{\pgfqpoint{5.959938in}{2.333705in}}%
\pgfpathlineto{\pgfqpoint{5.968745in}{2.422925in}}%
\pgfpathlineto{\pgfqpoint{5.977552in}{2.368027in}}%
\pgfpathlineto{\pgfqpoint{5.995166in}{2.381744in}}%
\pgfpathlineto{\pgfqpoint{6.003972in}{1.812128in}}%
\pgfpathlineto{\pgfqpoint{6.012779in}{1.764116in}}%
\pgfpathlineto{\pgfqpoint{6.021586in}{1.784692in}}%
\pgfpathlineto{\pgfqpoint{6.030393in}{2.045467in}}%
\pgfpathlineto{\pgfqpoint{6.039200in}{2.120970in}}%
\pgfpathlineto{\pgfqpoint{6.048007in}{2.374885in}}%
\pgfpathlineto{\pgfqpoint{6.056813in}{2.285665in}}%
\pgfpathlineto{\pgfqpoint{6.065620in}{2.244484in}}%
\pgfpathlineto{\pgfqpoint{6.074427in}{2.107252in}}%
\pgfpathlineto{\pgfqpoint{6.083234in}{1.592534in}}%
\pgfpathlineto{\pgfqpoint{6.092041in}{1.407234in}}%
\pgfpathlineto{\pgfqpoint{6.100847in}{1.585675in}}%
\pgfpathlineto{\pgfqpoint{6.109654in}{1.311155in}}%
\pgfpathlineto{\pgfqpoint{6.118461in}{1.729794in}}%
\pgfpathlineto{\pgfqpoint{6.127268in}{1.688613in}}%
\pgfpathlineto{\pgfqpoint{6.136075in}{1.784692in}}%
\pgfpathlineto{\pgfqpoint{6.144882in}{1.784692in}}%
\pgfpathlineto{\pgfqpoint{6.153688in}{2.114111in}}%
\pgfpathlineto{\pgfqpoint{6.162495in}{1.873913in}}%
\pgfpathlineto{\pgfqpoint{6.171302in}{1.935670in}}%
\pgfpathlineto{\pgfqpoint{6.180109in}{1.983710in}}%
\pgfpathlineto{\pgfqpoint{6.188916in}{1.805269in}}%
\pgfpathlineto{\pgfqpoint{6.197722in}{2.107252in}}%
\pgfpathlineto{\pgfqpoint{6.206529in}{1.661150in}}%
\pgfpathlineto{\pgfqpoint{6.215336in}{1.366053in}}%
\pgfpathlineto{\pgfqpoint{6.224143in}{1.613110in}}%
\pgfpathlineto{\pgfqpoint{6.241757in}{1.880772in}}%
\pgfpathlineto{\pgfqpoint{6.250563in}{2.079789in}}%
\pgfpathlineto{\pgfqpoint{6.259370in}{1.997427in}}%
\pgfpathlineto{\pgfqpoint{6.268177in}{1.654291in}}%
\pgfpathlineto{\pgfqpoint{6.276984in}{1.729794in}}%
\pgfpathlineto{\pgfqpoint{6.285791in}{1.571958in}}%
\pgfpathlineto{\pgfqpoint{6.294597in}{1.585675in}}%
\pgfpathlineto{\pgfqpoint{6.303404in}{1.805269in}}%
\pgfpathlineto{\pgfqpoint{6.312211in}{1.825873in}}%
\pgfpathlineto{\pgfqpoint{6.321018in}{1.661150in}}%
\pgfpathlineto{\pgfqpoint{6.329825in}{1.578816in}}%
\pgfpathlineto{\pgfqpoint{6.338632in}{1.976851in}}%
\pgfpathlineto{\pgfqpoint{6.347438in}{1.654291in}}%
\pgfpathlineto{\pgfqpoint{6.365052in}{1.256257in}}%
\pgfpathlineto{\pgfqpoint{6.373859in}{1.503314in}}%
\pgfpathlineto{\pgfqpoint{6.382666in}{1.949388in}}%
\pgfpathlineto{\pgfqpoint{6.391472in}{1.887630in}}%
\pgfpathlineto{\pgfqpoint{6.400279in}{1.668037in}}%
\pgfpathlineto{\pgfqpoint{6.409086in}{1.812128in}}%
\pgfpathlineto{\pgfqpoint{6.417893in}{1.853308in}}%
\pgfpathlineto{\pgfqpoint{6.426700in}{1.832732in}}%
\pgfpathlineto{\pgfqpoint{6.435507in}{1.839591in}}%
\pgfpathlineto{\pgfqpoint{6.444313in}{2.388603in}}%
\pgfpathlineto{\pgfqpoint{6.453120in}{2.196445in}}%
\pgfpathlineto{\pgfqpoint{6.461927in}{2.079789in}}%
\pgfpathlineto{\pgfqpoint{6.470734in}{2.429784in}}%
\pgfpathlineto{\pgfqpoint{6.479541in}{2.601366in}}%
\pgfpathlineto{\pgfqpoint{6.488347in}{2.512146in}}%
\pgfpathlineto{\pgfqpoint{6.497154in}{2.669982in}}%
\pgfpathlineto{\pgfqpoint{6.505961in}{2.567044in}}%
\pgfpathlineto{\pgfqpoint{6.514768in}{2.491541in}}%
\pgfpathlineto{\pgfqpoint{6.523575in}{2.402349in}}%
\pgfpathlineto{\pgfqpoint{6.532382in}{2.182727in}}%
\pgfpathlineto{\pgfqpoint{6.541188in}{2.402349in}}%
\pgfpathlineto{\pgfqpoint{6.549995in}{2.258230in}}%
\pgfpathlineto{\pgfqpoint{6.567609in}{1.681754in}}%
\pgfpathlineto{\pgfqpoint{6.576416in}{1.599393in}}%
\pgfpathlineto{\pgfqpoint{6.585222in}{1.722935in}}%
\pgfpathlineto{\pgfqpoint{6.594029in}{1.983710in}}%
\pgfpathlineto{\pgfqpoint{6.602836in}{1.777834in}}%
\pgfpathlineto{\pgfqpoint{6.611643in}{1.969992in}}%
\pgfpathlineto{\pgfqpoint{6.620450in}{1.908207in}}%
\pgfpathlineto{\pgfqpoint{6.629257in}{1.928811in}}%
\pgfpathlineto{\pgfqpoint{6.638063in}{1.846450in}}%
\pgfpathlineto{\pgfqpoint{6.646870in}{2.011173in}}%
\pgfpathlineto{\pgfqpoint{6.655677in}{2.278806in}}%
\pgfpathlineto{\pgfqpoint{6.664484in}{2.313128in}}%
\pgfpathlineto{\pgfqpoint{6.673291in}{2.237625in}}%
\pgfpathlineto{\pgfqpoint{6.682097in}{2.230767in}}%
\pgfpathlineto{\pgfqpoint{6.690904in}{2.038608in}}%
\pgfpathlineto{\pgfqpoint{6.699711in}{1.626856in}}%
\pgfpathlineto{\pgfqpoint{6.708518in}{1.887630in}}%
\pgfpathlineto{\pgfqpoint{6.717325in}{1.510172in}}%
\pgfpathlineto{\pgfqpoint{6.726132in}{1.578816in}}%
\pgfpathlineto{\pgfqpoint{6.734938in}{1.846450in}}%
\pgfpathlineto{\pgfqpoint{6.743745in}{1.887630in}}%
\pgfpathlineto{\pgfqpoint{6.752552in}{2.093507in}}%
\pgfpathlineto{\pgfqpoint{6.761359in}{2.169009in}}%
\pgfpathlineto{\pgfqpoint{6.770166in}{2.196445in}}%
\pgfpathlineto{\pgfqpoint{6.778972in}{2.388603in}}%
\pgfpathlineto{\pgfqpoint{6.787779in}{2.155292in}}%
\pgfpathlineto{\pgfqpoint{6.796586in}{2.402349in}}%
\pgfpathlineto{\pgfqpoint{6.805393in}{1.921953in}}%
\pgfpathlineto{\pgfqpoint{6.814200in}{1.770975in}}%
\pgfpathlineto{\pgfqpoint{6.823007in}{1.578816in}}%
\pgfpathlineto{\pgfqpoint{6.831813in}{1.668037in}}%
\pgfpathlineto{\pgfqpoint{6.840620in}{1.743512in}}%
\pgfpathlineto{\pgfqpoint{6.849427in}{1.963133in}}%
\pgfpathlineto{\pgfqpoint{6.858234in}{1.832732in}}%
\pgfpathlineto{\pgfqpoint{6.867041in}{2.230767in}}%
\pgfpathlineto{\pgfqpoint{6.875847in}{2.361168in}}%
\pgfpathlineto{\pgfqpoint{6.884654in}{2.011173in}}%
\pgfpathlineto{\pgfqpoint{6.893461in}{2.292524in}}%
\pgfpathlineto{\pgfqpoint{6.902268in}{1.997427in}}%
\pgfpathlineto{\pgfqpoint{6.911075in}{2.072930in}}%
\pgfpathlineto{\pgfqpoint{6.919882in}{2.223908in}}%
\pgfpathlineto{\pgfqpoint{6.928688in}{2.340563in}}%
\pgfpathlineto{\pgfqpoint{6.937495in}{2.182727in}}%
\pgfpathlineto{\pgfqpoint{6.946302in}{2.374885in}}%
\pgfpathlineto{\pgfqpoint{6.955109in}{2.292524in}}%
\pgfpathlineto{\pgfqpoint{6.963916in}{2.100365in}}%
\pgfpathlineto{\pgfqpoint{6.972722in}{2.203303in}}%
\pgfpathlineto{\pgfqpoint{6.981529in}{2.230767in}}%
\pgfpathlineto{\pgfqpoint{6.999143in}{1.880772in}}%
\pgfpathlineto{\pgfqpoint{7.007950in}{2.429784in}}%
\pgfpathlineto{\pgfqpoint{7.016757in}{2.230767in}}%
\pgfpathlineto{\pgfqpoint{7.025563in}{2.278806in}}%
\pgfpathlineto{\pgfqpoint{7.043177in}{2.072930in}}%
\pgfpathlineto{\pgfqpoint{7.051984in}{1.908207in}}%
\pgfpathlineto{\pgfqpoint{7.060791in}{1.812128in}}%
\pgfpathlineto{\pgfqpoint{7.069597in}{1.619997in}}%
\pgfpathlineto{\pgfqpoint{7.078404in}{1.757229in}}%
\pgfpathlineto{\pgfqpoint{7.087211in}{1.585675in}}%
\pgfpathlineto{\pgfqpoint{7.096018in}{1.544494in}}%
\pgfpathlineto{\pgfqpoint{7.104825in}{1.674896in}}%
\pgfpathlineto{\pgfqpoint{7.113632in}{1.825873in}}%
\pgfpathlineto{\pgfqpoint{7.122438in}{1.894489in}}%
\pgfpathlineto{\pgfqpoint{7.131245in}{1.777834in}}%
\pgfpathlineto{\pgfqpoint{7.140052in}{1.805269in}}%
\pgfpathlineto{\pgfqpoint{7.148859in}{2.018032in}}%
\pgfpathlineto{\pgfqpoint{7.157666in}{2.470965in}}%
\pgfpathlineto{\pgfqpoint{7.166472in}{2.368027in}}%
\pgfpathlineto{\pgfqpoint{7.175279in}{2.553326in}}%
\pgfpathlineto{\pgfqpoint{7.184086in}{2.436643in}}%
\pgfpathlineto{\pgfqpoint{7.192893in}{2.072930in}}%
\pgfpathlineto{\pgfqpoint{7.201700in}{1.949388in}}%
\pgfpathlineto{\pgfqpoint{7.210507in}{1.921953in}}%
\pgfpathlineto{\pgfqpoint{7.219313in}{1.681754in}}%
\pgfpathlineto{\pgfqpoint{7.228120in}{2.045467in}}%
\pgfpathlineto{\pgfqpoint{7.236927in}{1.983710in}}%
\pgfpathlineto{\pgfqpoint{7.245734in}{1.969992in}}%
\pgfpathlineto{\pgfqpoint{7.254541in}{1.935670in}}%
\pgfpathlineto{\pgfqpoint{7.263347in}{1.832732in}}%
\pgfpathlineto{\pgfqpoint{7.272154in}{1.983710in}}%
\pgfpathlineto{\pgfqpoint{7.280961in}{1.523918in}}%
\pgfpathlineto{\pgfqpoint{7.289768in}{1.592534in}}%
\pgfpathlineto{\pgfqpoint{7.298575in}{1.942529in}}%
\pgfpathlineto{\pgfqpoint{7.307382in}{1.695472in}}%
\pgfpathlineto{\pgfqpoint{7.316188in}{1.558212in}}%
\pgfpathlineto{\pgfqpoint{7.324995in}{1.695472in}}%
\pgfpathlineto{\pgfqpoint{7.333802in}{1.915094in}}%
\pgfpathlineto{\pgfqpoint{7.342609in}{1.839591in}}%
\pgfpathlineto{\pgfqpoint{7.351416in}{1.695472in}}%
\pgfpathlineto{\pgfqpoint{7.360222in}{1.613110in}}%
\pgfpathlineto{\pgfqpoint{7.369029in}{1.736653in}}%
\pgfpathlineto{\pgfqpoint{7.377836in}{1.661150in}}%
\pgfpathlineto{\pgfqpoint{7.386643in}{1.764116in}}%
\pgfpathlineto{\pgfqpoint{7.413063in}{1.983710in}}%
\pgfpathlineto{\pgfqpoint{7.421870in}{2.306269in}}%
\pgfpathlineto{\pgfqpoint{7.430677in}{2.436643in}}%
\pgfpathlineto{\pgfqpoint{7.439484in}{1.915094in}}%
\pgfpathlineto{\pgfqpoint{7.448291in}{2.052326in}}%
\pgfpathlineto{\pgfqpoint{7.457097in}{1.791551in}}%
\pgfpathlineto{\pgfqpoint{7.465904in}{1.880772in}}%
\pgfpathlineto{\pgfqpoint{7.483518in}{1.626856in}}%
\pgfpathlineto{\pgfqpoint{7.492325in}{1.599393in}}%
\pgfpathlineto{\pgfqpoint{7.501132in}{1.393517in}}%
\pgfpathlineto{\pgfqpoint{7.509938in}{1.146460in}}%
\pgfpathlineto{\pgfqpoint{7.518745in}{1.235680in}}%
\pgfpathlineto{\pgfqpoint{7.536359in}{1.832732in}}%
\pgfpathlineto{\pgfqpoint{7.545166in}{1.757229in}}%
\pgfpathlineto{\pgfqpoint{7.553972in}{1.427839in}}%
\pgfpathlineto{\pgfqpoint{7.562779in}{1.819015in}}%
\pgfpathlineto{\pgfqpoint{7.571586in}{1.455274in}}%
\pgfpathlineto{\pgfqpoint{7.580393in}{1.249398in}}%
\pgfpathlineto{\pgfqpoint{7.589200in}{1.372912in}}%
\pgfpathlineto{\pgfqpoint{7.598007in}{1.400376in}}%
\pgfpathlineto{\pgfqpoint{7.606813in}{1.523918in}}%
\pgfpathlineto{\pgfqpoint{7.615620in}{2.052326in}}%
\pgfpathlineto{\pgfqpoint{7.624427in}{1.688613in}}%
\pgfpathlineto{\pgfqpoint{7.633234in}{1.812128in}}%
\pgfpathlineto{\pgfqpoint{7.642041in}{1.997427in}}%
\pgfpathlineto{\pgfqpoint{7.650847in}{1.867054in}}%
\pgfpathlineto{\pgfqpoint{7.659654in}{2.189586in}}%
\pgfpathlineto{\pgfqpoint{7.668461in}{2.251343in}}%
\pgfpathlineto{\pgfqpoint{7.677268in}{2.340563in}}%
\pgfpathlineto{\pgfqpoint{7.686075in}{2.217049in}}%
\pgfpathlineto{\pgfqpoint{7.694882in}{2.258230in}}%
\pgfpathlineto{\pgfqpoint{7.703688in}{2.388603in}}%
\pgfpathlineto{\pgfqpoint{7.712495in}{2.182727in}}%
\pgfpathlineto{\pgfqpoint{7.721302in}{2.066071in}}%
\pgfpathlineto{\pgfqpoint{7.730109in}{2.361168in}}%
\pgfpathlineto{\pgfqpoint{7.738916in}{2.484682in}}%
\pgfpathlineto{\pgfqpoint{7.747722in}{2.669982in}}%
\pgfpathlineto{\pgfqpoint{7.756529in}{2.244484in}}%
\pgfpathlineto{\pgfqpoint{7.765336in}{2.237625in}}%
\pgfpathlineto{\pgfqpoint{7.774143in}{2.319987in}}%
\pgfpathlineto{\pgfqpoint{7.782950in}{2.189586in}}%
\pgfpathlineto{\pgfqpoint{7.791757in}{2.416066in}}%
\pgfpathlineto{\pgfqpoint{7.800563in}{2.114111in}}%
\pgfpathlineto{\pgfqpoint{7.809370in}{2.052326in}}%
\pgfpathlineto{\pgfqpoint{7.818177in}{2.148405in}}%
\pgfpathlineto{\pgfqpoint{7.844597in}{2.024891in}}%
\pgfpathlineto{\pgfqpoint{7.853404in}{2.079789in}}%
\pgfpathlineto{\pgfqpoint{7.862211in}{2.182727in}}%
\pgfpathlineto{\pgfqpoint{7.871018in}{1.819015in}}%
\pgfpathlineto{\pgfqpoint{7.879825in}{1.743512in}}%
\pgfpathlineto{\pgfqpoint{7.888632in}{1.434698in}}%
\pgfpathlineto{\pgfqpoint{7.897438in}{1.716076in}}%
\pgfpathlineto{\pgfqpoint{7.906245in}{1.942529in}}%
\pgfpathlineto{\pgfqpoint{7.915052in}{1.935670in}}%
\pgfpathlineto{\pgfqpoint{7.923859in}{2.162151in}}%
\pgfpathlineto{\pgfqpoint{7.932666in}{1.784692in}}%
\pgfpathlineto{\pgfqpoint{7.941472in}{1.654291in}}%
\pgfpathlineto{\pgfqpoint{7.959086in}{1.819015in}}%
\pgfpathlineto{\pgfqpoint{7.967893in}{1.784692in}}%
\pgfpathlineto{\pgfqpoint{7.976700in}{1.565071in}}%
\pgfpathlineto{\pgfqpoint{7.985507in}{1.757229in}}%
\pgfpathlineto{\pgfqpoint{7.994313in}{2.066071in}}%
\pgfpathlineto{\pgfqpoint{8.003120in}{2.217049in}}%
\pgfpathlineto{\pgfqpoint{8.011927in}{2.169009in}}%
\pgfpathlineto{\pgfqpoint{8.020734in}{2.285665in}}%
\pgfpathlineto{\pgfqpoint{8.029541in}{1.894489in}}%
\pgfpathlineto{\pgfqpoint{8.038347in}{2.292524in}}%
\pgfpathlineto{\pgfqpoint{8.047154in}{2.457247in}}%
\pgfpathlineto{\pgfqpoint{8.055961in}{2.340563in}}%
\pgfpathlineto{\pgfqpoint{8.064768in}{2.265089in}}%
\pgfpathlineto{\pgfqpoint{8.073575in}{2.086648in}}%
\pgfpathlineto{\pgfqpoint{8.082382in}{2.368027in}}%
\pgfpathlineto{\pgfqpoint{8.091188in}{2.141546in}}%
\pgfpathlineto{\pgfqpoint{8.099995in}{2.203303in}}%
\pgfpathlineto{\pgfqpoint{8.108802in}{2.141546in}}%
\pgfpathlineto{\pgfqpoint{8.117609in}{2.182727in}}%
\pgfpathlineto{\pgfqpoint{8.126416in}{2.182727in}}%
\pgfpathlineto{\pgfqpoint{8.135222in}{2.093507in}}%
\pgfpathlineto{\pgfqpoint{8.144029in}{2.182727in}}%
\pgfpathlineto{\pgfqpoint{8.152836in}{2.237625in}}%
\pgfpathlineto{\pgfqpoint{8.161643in}{2.079789in}}%
\pgfpathlineto{\pgfqpoint{8.170450in}{2.278806in}}%
\pgfpathlineto{\pgfqpoint{8.179257in}{2.223908in}}%
\pgfpathlineto{\pgfqpoint{8.188063in}{2.072930in}}%
\pgfpathlineto{\pgfqpoint{8.196870in}{1.949388in}}%
\pgfpathlineto{\pgfqpoint{8.205677in}{1.860167in}}%
\pgfpathlineto{\pgfqpoint{8.214484in}{1.969992in}}%
\pgfpathlineto{\pgfqpoint{8.223291in}{2.210190in}}%
\pgfpathlineto{\pgfqpoint{8.232097in}{2.347422in}}%
\pgfpathlineto{\pgfqpoint{8.240904in}{2.347422in}}%
\pgfpathlineto{\pgfqpoint{8.249711in}{2.230767in}}%
\pgfpathlineto{\pgfqpoint{8.267325in}{1.510172in}}%
\pgfpathlineto{\pgfqpoint{8.276132in}{1.777834in}}%
\pgfpathlineto{\pgfqpoint{8.284938in}{1.599393in}}%
\pgfpathlineto{\pgfqpoint{8.293745in}{1.544494in}}%
\pgfpathlineto{\pgfqpoint{8.302552in}{1.784692in}}%
\pgfpathlineto{\pgfqpoint{8.311359in}{1.702331in}}%
\pgfpathlineto{\pgfqpoint{8.320166in}{1.414093in}}%
\pgfpathlineto{\pgfqpoint{8.328972in}{1.544494in}}%
\pgfpathlineto{\pgfqpoint{8.337779in}{1.523918in}}%
\pgfpathlineto{\pgfqpoint{8.346586in}{1.468992in}}%
\pgfpathlineto{\pgfqpoint{8.355393in}{1.448415in}}%
\pgfpathlineto{\pgfqpoint{8.364200in}{1.702331in}}%
\pgfpathlineto{\pgfqpoint{8.373007in}{1.606252in}}%
\pgfpathlineto{\pgfqpoint{8.381813in}{1.839591in}}%
\pgfpathlineto{\pgfqpoint{8.390620in}{1.887630in}}%
\pgfpathlineto{\pgfqpoint{8.399427in}{1.661150in}}%
\pgfpathlineto{\pgfqpoint{8.408234in}{1.565071in}}%
\pgfpathlineto{\pgfqpoint{8.417041in}{1.510172in}}%
\pgfpathlineto{\pgfqpoint{8.425847in}{1.674896in}}%
\pgfpathlineto{\pgfqpoint{8.434654in}{2.169009in}}%
\pgfpathlineto{\pgfqpoint{8.443461in}{2.066071in}}%
\pgfpathlineto{\pgfqpoint{8.452268in}{2.169009in}}%
\pgfpathlineto{\pgfqpoint{8.461075in}{2.162151in}}%
\pgfpathlineto{\pgfqpoint{8.469882in}{2.059213in}}%
\pgfpathlineto{\pgfqpoint{8.478688in}{1.983710in}}%
\pgfpathlineto{\pgfqpoint{8.487495in}{2.011173in}}%
\pgfpathlineto{\pgfqpoint{8.496302in}{1.976851in}}%
\pgfpathlineto{\pgfqpoint{8.505109in}{1.640574in}}%
\pgfpathlineto{\pgfqpoint{8.513916in}{1.702331in}}%
\pgfpathlineto{\pgfqpoint{8.522722in}{1.674896in}}%
\pgfpathlineto{\pgfqpoint{8.531529in}{2.127829in}}%
\pgfpathlineto{\pgfqpoint{8.540336in}{2.470965in}}%
\pgfpathlineto{\pgfqpoint{8.549143in}{2.292524in}}%
\pgfpathlineto{\pgfqpoint{8.557950in}{2.457247in}}%
\pgfpathlineto{\pgfqpoint{8.566757in}{2.450388in}}%
\pgfpathlineto{\pgfqpoint{8.575563in}{2.361168in}}%
\pgfpathlineto{\pgfqpoint{8.593177in}{2.155292in}}%
\pgfpathlineto{\pgfqpoint{8.601984in}{2.265089in}}%
\pgfpathlineto{\pgfqpoint{8.610791in}{2.182727in}}%
\pgfpathlineto{\pgfqpoint{8.619597in}{1.736653in}}%
\pgfpathlineto{\pgfqpoint{8.628404in}{1.640574in}}%
\pgfpathlineto{\pgfqpoint{8.637211in}{1.626856in}}%
\pgfpathlineto{\pgfqpoint{8.646018in}{1.654291in}}%
\pgfpathlineto{\pgfqpoint{8.654825in}{1.983710in}}%
\pgfpathlineto{\pgfqpoint{8.663632in}{1.661150in}}%
\pgfpathlineto{\pgfqpoint{8.672438in}{1.475878in}}%
\pgfpathlineto{\pgfqpoint{8.681245in}{1.434698in}}%
\pgfpathlineto{\pgfqpoint{8.690052in}{1.489596in}}%
\pgfpathlineto{\pgfqpoint{8.698859in}{1.372912in}}%
\pgfpathlineto{\pgfqpoint{8.716472in}{1.716076in}}%
\pgfpathlineto{\pgfqpoint{8.725279in}{1.750370in}}%
\pgfpathlineto{\pgfqpoint{8.734086in}{1.887630in}}%
\pgfpathlineto{\pgfqpoint{8.742893in}{1.963133in}}%
\pgfpathlineto{\pgfqpoint{8.751700in}{1.976851in}}%
\pgfpathlineto{\pgfqpoint{8.760507in}{2.038608in}}%
\pgfpathlineto{\pgfqpoint{8.769313in}{1.777834in}}%
\pgfpathlineto{\pgfqpoint{8.778120in}{1.853308in}}%
\pgfpathlineto{\pgfqpoint{8.786927in}{2.210190in}}%
\pgfpathlineto{\pgfqpoint{8.795734in}{1.997427in}}%
\pgfpathlineto{\pgfqpoint{8.813347in}{1.496455in}}%
\pgfpathlineto{\pgfqpoint{8.822154in}{1.798410in}}%
\pgfpathlineto{\pgfqpoint{8.830961in}{1.832732in}}%
\pgfpathlineto{\pgfqpoint{8.839768in}{2.155292in}}%
\pgfpathlineto{\pgfqpoint{8.848575in}{2.237625in}}%
\pgfpathlineto{\pgfqpoint{8.857382in}{1.949388in}}%
\pgfpathlineto{\pgfqpoint{8.866188in}{2.114111in}}%
\pgfpathlineto{\pgfqpoint{8.874995in}{2.127829in}}%
\pgfpathlineto{\pgfqpoint{8.883802in}{2.189586in}}%
\pgfpathlineto{\pgfqpoint{8.892609in}{2.148405in}}%
\pgfpathlineto{\pgfqpoint{8.901416in}{2.155292in}}%
\pgfpathlineto{\pgfqpoint{8.910222in}{2.223908in}}%
\pgfpathlineto{\pgfqpoint{8.919029in}{1.770975in}}%
\pgfpathlineto{\pgfqpoint{8.927836in}{1.832732in}}%
\pgfpathlineto{\pgfqpoint{8.936643in}{1.578816in}}%
\pgfpathlineto{\pgfqpoint{8.945450in}{1.558212in}}%
\pgfpathlineto{\pgfqpoint{8.954257in}{1.887630in}}%
\pgfpathlineto{\pgfqpoint{8.963063in}{2.100365in}}%
\pgfpathlineto{\pgfqpoint{8.971870in}{2.100365in}}%
\pgfpathlineto{\pgfqpoint{8.980677in}{2.011173in}}%
\pgfpathlineto{\pgfqpoint{8.989484in}{1.510172in}}%
\pgfpathlineto{\pgfqpoint{8.998291in}{1.599393in}}%
\pgfpathlineto{\pgfqpoint{9.007097in}{1.366053in}}%
\pgfpathlineto{\pgfqpoint{9.015904in}{1.475878in}}%
\pgfpathlineto{\pgfqpoint{9.024711in}{1.640574in}}%
\pgfpathlineto{\pgfqpoint{9.033518in}{1.764116in}}%
\pgfpathlineto{\pgfqpoint{9.042325in}{1.674896in}}%
\pgfpathlineto{\pgfqpoint{9.051132in}{2.031749in}}%
\pgfpathlineto{\pgfqpoint{9.059938in}{1.729794in}}%
\pgfpathlineto{\pgfqpoint{9.068745in}{1.819015in}}%
\pgfpathlineto{\pgfqpoint{9.077552in}{1.812128in}}%
\pgfpathlineto{\pgfqpoint{9.086359in}{1.846450in}}%
\pgfpathlineto{\pgfqpoint{9.095166in}{1.983710in}}%
\pgfpathlineto{\pgfqpoint{9.103972in}{1.949388in}}%
\pgfpathlineto{\pgfqpoint{9.112779in}{1.420952in}}%
\pgfpathlineto{\pgfqpoint{9.121586in}{1.208217in}}%
\pgfpathlineto{\pgfqpoint{9.130393in}{1.338618in}}%
\pgfpathlineto{\pgfqpoint{9.139200in}{1.318014in}}%
\pgfpathlineto{\pgfqpoint{9.148007in}{1.475878in}}%
\pgfpathlineto{\pgfqpoint{9.156813in}{1.462133in}}%
\pgfpathlineto{\pgfqpoint{9.165620in}{1.571958in}}%
\pgfpathlineto{\pgfqpoint{9.174427in}{1.386658in}}%
\pgfpathlineto{\pgfqpoint{9.183234in}{1.386658in}}%
\pgfpathlineto{\pgfqpoint{9.192041in}{1.571958in}}%
\pgfpathlineto{\pgfqpoint{9.209654in}{2.093507in}}%
\pgfpathlineto{\pgfqpoint{9.218461in}{2.072930in}}%
\pgfpathlineto{\pgfqpoint{9.227268in}{2.326846in}}%
\pgfpathlineto{\pgfqpoint{9.236075in}{2.491541in}}%
\pgfpathlineto{\pgfqpoint{9.244882in}{2.601366in}}%
\pgfpathlineto{\pgfqpoint{9.253688in}{2.615084in}}%
\pgfpathlineto{\pgfqpoint{9.262495in}{2.402349in}}%
\pgfpathlineto{\pgfqpoint{9.271302in}{1.956246in}}%
\pgfpathlineto{\pgfqpoint{9.280109in}{1.633715in}}%
\pgfpathlineto{\pgfqpoint{9.288916in}{1.997427in}}%
\pgfpathlineto{\pgfqpoint{9.297722in}{1.825873in}}%
\pgfpathlineto{\pgfqpoint{9.306529in}{2.107252in}}%
\pgfpathlineto{\pgfqpoint{9.315336in}{1.661150in}}%
\pgfpathlineto{\pgfqpoint{9.324143in}{1.729794in}}%
\pgfpathlineto{\pgfqpoint{9.332950in}{1.908207in}}%
\pgfpathlineto{\pgfqpoint{9.341757in}{2.011173in}}%
\pgfpathlineto{\pgfqpoint{9.350563in}{2.313128in}}%
\pgfpathlineto{\pgfqpoint{9.359370in}{2.210190in}}%
\pgfpathlineto{\pgfqpoint{9.368177in}{1.626856in}}%
\pgfpathlineto{\pgfqpoint{9.376984in}{1.702331in}}%
\pgfpathlineto{\pgfqpoint{9.385791in}{1.503314in}}%
\pgfpathlineto{\pgfqpoint{9.394597in}{1.619997in}}%
\pgfpathlineto{\pgfqpoint{9.403404in}{1.613110in}}%
\pgfpathlineto{\pgfqpoint{9.412211in}{1.619997in}}%
\pgfpathlineto{\pgfqpoint{9.421018in}{1.722935in}}%
\pgfpathlineto{\pgfqpoint{9.429825in}{1.551353in}}%
\pgfpathlineto{\pgfqpoint{9.447438in}{1.928811in}}%
\pgfpathlineto{\pgfqpoint{9.456245in}{1.757229in}}%
\pgfpathlineto{\pgfqpoint{9.465052in}{1.867054in}}%
\pgfpathlineto{\pgfqpoint{9.482666in}{2.217049in}}%
\pgfpathlineto{\pgfqpoint{9.491472in}{2.018032in}}%
\pgfpathlineto{\pgfqpoint{9.500279in}{2.072930in}}%
\pgfpathlineto{\pgfqpoint{9.509086in}{2.107252in}}%
\pgfpathlineto{\pgfqpoint{9.517893in}{2.189586in}}%
\pgfpathlineto{\pgfqpoint{9.526700in}{1.928811in}}%
\pgfpathlineto{\pgfqpoint{9.535507in}{1.798410in}}%
\pgfpathlineto{\pgfqpoint{9.544313in}{2.148405in}}%
\pgfpathlineto{\pgfqpoint{9.553120in}{1.928811in}}%
\pgfpathlineto{\pgfqpoint{9.561927in}{1.963133in}}%
\pgfpathlineto{\pgfqpoint{9.570734in}{2.443501in}}%
\pgfpathlineto{\pgfqpoint{9.579541in}{2.525863in}}%
\pgfpathlineto{\pgfqpoint{9.588347in}{2.333705in}}%
\pgfpathlineto{\pgfqpoint{9.597154in}{1.990569in}}%
\pgfpathlineto{\pgfqpoint{9.605961in}{1.963133in}}%
\pgfpathlineto{\pgfqpoint{9.614768in}{1.606252in}}%
\pgfpathlineto{\pgfqpoint{9.623575in}{1.963133in}}%
\pgfpathlineto{\pgfqpoint{9.632382in}{1.908207in}}%
\pgfpathlineto{\pgfqpoint{9.641188in}{1.921953in}}%
\pgfpathlineto{\pgfqpoint{9.649995in}{1.832732in}}%
\pgfpathlineto{\pgfqpoint{9.658802in}{2.292524in}}%
\pgfpathlineto{\pgfqpoint{9.667609in}{2.299383in}}%
\pgfpathlineto{\pgfqpoint{9.676416in}{2.217049in}}%
\pgfpathlineto{\pgfqpoint{9.685222in}{2.148405in}}%
\pgfpathlineto{\pgfqpoint{9.694029in}{2.319987in}}%
\pgfpathlineto{\pgfqpoint{9.702836in}{2.127829in}}%
\pgfpathlineto{\pgfqpoint{9.711643in}{1.716076in}}%
\pgfpathlineto{\pgfqpoint{9.720450in}{1.613110in}}%
\pgfpathlineto{\pgfqpoint{9.729257in}{1.331760in}}%
\pgfpathlineto{\pgfqpoint{9.738063in}{1.324873in}}%
\pgfpathlineto{\pgfqpoint{9.746870in}{1.345477in}}%
\pgfpathlineto{\pgfqpoint{9.755677in}{1.647432in}}%
\pgfpathlineto{\pgfqpoint{9.773291in}{1.221935in}}%
\pgfpathlineto{\pgfqpoint{9.782097in}{1.619997in}}%
\pgfpathlineto{\pgfqpoint{9.790904in}{1.626856in}}%
\pgfpathlineto{\pgfqpoint{9.799711in}{1.825873in}}%
\pgfpathlineto{\pgfqpoint{9.808518in}{1.963133in}}%
\pgfpathlineto{\pgfqpoint{9.817325in}{1.722935in}}%
\pgfpathlineto{\pgfqpoint{9.826132in}{1.915094in}}%
\pgfpathlineto{\pgfqpoint{9.834938in}{1.599393in}}%
\pgfpathlineto{\pgfqpoint{9.843745in}{2.011173in}}%
\pgfpathlineto{\pgfqpoint{9.852552in}{2.333705in}}%
\pgfpathlineto{\pgfqpoint{9.861359in}{2.484682in}}%
\pgfpathlineto{\pgfqpoint{9.870166in}{2.120970in}}%
\pgfpathlineto{\pgfqpoint{9.878972in}{2.196445in}}%
\pgfpathlineto{\pgfqpoint{9.887779in}{1.880772in}}%
\pgfpathlineto{\pgfqpoint{9.896586in}{1.873913in}}%
\pgfpathlineto{\pgfqpoint{9.905393in}{1.976851in}}%
\pgfpathlineto{\pgfqpoint{9.914200in}{2.127829in}}%
\pgfpathlineto{\pgfqpoint{9.923007in}{1.743512in}}%
\pgfpathlineto{\pgfqpoint{9.931813in}{1.640574in}}%
\pgfpathlineto{\pgfqpoint{9.940620in}{1.867054in}}%
\pgfpathlineto{\pgfqpoint{9.949427in}{1.935670in}}%
\pgfpathlineto{\pgfqpoint{9.949427in}{1.935670in}}%
\pgfusepath{stroke}%
\end{pgfscope}%
\begin{pgfscope}%
\pgfpathrectangle{\pgfqpoint{0.702268in}{0.521603in}}{\pgfqpoint{9.687500in}{4.235000in}}%
\pgfusepath{clip}%
\pgfsetrectcap%
\pgfsetroundjoin%
\pgfsetlinewidth{0.501875pt}%
\definecolor{currentstroke}{rgb}{0.501961,0.501961,0.501961}%
\pgfsetstrokecolor{currentstroke}%
\pgfsetstrokeopacity{0.250000}%
\pgfsetdash{}{0pt}%
\pgfpathmoveto{\pgfqpoint{1.142609in}{4.351369in}}%
\pgfpathlineto{\pgfqpoint{1.151416in}{3.452333in}}%
\pgfpathlineto{\pgfqpoint{1.160222in}{2.786638in}}%
\pgfpathlineto{\pgfqpoint{1.169029in}{2.388603in}}%
\pgfpathlineto{\pgfqpoint{1.177836in}{2.093507in}}%
\pgfpathlineto{\pgfqpoint{1.186643in}{2.155292in}}%
\pgfpathlineto{\pgfqpoint{1.195450in}{2.477823in}}%
\pgfpathlineto{\pgfqpoint{1.204257in}{2.306269in}}%
\pgfpathlineto{\pgfqpoint{1.213063in}{1.873913in}}%
\pgfpathlineto{\pgfqpoint{1.221870in}{1.530777in}}%
\pgfpathlineto{\pgfqpoint{1.230677in}{1.702331in}}%
\pgfpathlineto{\pgfqpoint{1.248291in}{1.887630in}}%
\pgfpathlineto{\pgfqpoint{1.257097in}{1.668037in}}%
\pgfpathlineto{\pgfqpoint{1.265904in}{1.606252in}}%
\pgfpathlineto{\pgfqpoint{1.274711in}{1.681754in}}%
\pgfpathlineto{\pgfqpoint{1.283518in}{1.777834in}}%
\pgfpathlineto{\pgfqpoint{1.292325in}{1.908207in}}%
\pgfpathlineto{\pgfqpoint{1.301132in}{1.983710in}}%
\pgfpathlineto{\pgfqpoint{1.309938in}{2.100365in}}%
\pgfpathlineto{\pgfqpoint{1.318745in}{1.942529in}}%
\pgfpathlineto{\pgfqpoint{1.327552in}{1.887630in}}%
\pgfpathlineto{\pgfqpoint{1.336359in}{1.798410in}}%
\pgfpathlineto{\pgfqpoint{1.345166in}{2.086648in}}%
\pgfpathlineto{\pgfqpoint{1.353972in}{2.251343in}}%
\pgfpathlineto{\pgfqpoint{1.371586in}{1.599393in}}%
\pgfpathlineto{\pgfqpoint{1.380393in}{2.079789in}}%
\pgfpathlineto{\pgfqpoint{1.389200in}{1.846450in}}%
\pgfpathlineto{\pgfqpoint{1.398007in}{1.928811in}}%
\pgfpathlineto{\pgfqpoint{1.406813in}{1.942529in}}%
\pgfpathlineto{\pgfqpoint{1.415620in}{1.805269in}}%
\pgfpathlineto{\pgfqpoint{1.433234in}{2.079789in}}%
\pgfpathlineto{\pgfqpoint{1.442041in}{1.805269in}}%
\pgfpathlineto{\pgfqpoint{1.450847in}{1.990569in}}%
\pgfpathlineto{\pgfqpoint{1.459654in}{2.114111in}}%
\pgfpathlineto{\pgfqpoint{1.468461in}{2.196445in}}%
\pgfpathlineto{\pgfqpoint{1.477268in}{1.901348in}}%
\pgfpathlineto{\pgfqpoint{1.486075in}{2.127829in}}%
\pgfpathlineto{\pgfqpoint{1.494882in}{2.038608in}}%
\pgfpathlineto{\pgfqpoint{1.503688in}{1.770975in}}%
\pgfpathlineto{\pgfqpoint{1.512495in}{1.757229in}}%
\pgfpathlineto{\pgfqpoint{1.521302in}{2.072930in}}%
\pgfpathlineto{\pgfqpoint{1.538916in}{1.757229in}}%
\pgfpathlineto{\pgfqpoint{1.547722in}{1.716076in}}%
\pgfpathlineto{\pgfqpoint{1.556529in}{1.455274in}}%
\pgfpathlineto{\pgfqpoint{1.565336in}{1.352336in}}%
\pgfpathlineto{\pgfqpoint{1.582950in}{1.873913in}}%
\pgfpathlineto{\pgfqpoint{1.591757in}{1.825873in}}%
\pgfpathlineto{\pgfqpoint{1.600563in}{1.867054in}}%
\pgfpathlineto{\pgfqpoint{1.609370in}{1.770975in}}%
\pgfpathlineto{\pgfqpoint{1.618177in}{1.654291in}}%
\pgfpathlineto{\pgfqpoint{1.626984in}{1.654291in}}%
\pgfpathlineto{\pgfqpoint{1.635791in}{1.695472in}}%
\pgfpathlineto{\pgfqpoint{1.644597in}{1.695472in}}%
\pgfpathlineto{\pgfqpoint{1.653404in}{2.079789in}}%
\pgfpathlineto{\pgfqpoint{1.662211in}{1.942529in}}%
\pgfpathlineto{\pgfqpoint{1.671018in}{1.990569in}}%
\pgfpathlineto{\pgfqpoint{1.679825in}{2.045467in}}%
\pgfpathlineto{\pgfqpoint{1.688632in}{2.024891in}}%
\pgfpathlineto{\pgfqpoint{1.697438in}{1.976851in}}%
\pgfpathlineto{\pgfqpoint{1.706245in}{2.155292in}}%
\pgfpathlineto{\pgfqpoint{1.715052in}{2.038608in}}%
\pgfpathlineto{\pgfqpoint{1.723859in}{1.983710in}}%
\pgfpathlineto{\pgfqpoint{1.732666in}{1.915094in}}%
\pgfpathlineto{\pgfqpoint{1.741472in}{1.915094in}}%
\pgfpathlineto{\pgfqpoint{1.750279in}{2.258230in}}%
\pgfpathlineto{\pgfqpoint{1.759086in}{2.477823in}}%
\pgfpathlineto{\pgfqpoint{1.767893in}{2.306269in}}%
\pgfpathlineto{\pgfqpoint{1.776700in}{2.422925in}}%
\pgfpathlineto{\pgfqpoint{1.785507in}{2.038608in}}%
\pgfpathlineto{\pgfqpoint{1.794313in}{1.990569in}}%
\pgfpathlineto{\pgfqpoint{1.803120in}{1.839591in}}%
\pgfpathlineto{\pgfqpoint{1.811927in}{2.155292in}}%
\pgfpathlineto{\pgfqpoint{1.820734in}{1.798410in}}%
\pgfpathlineto{\pgfqpoint{1.829541in}{1.681754in}}%
\pgfpathlineto{\pgfqpoint{1.838347in}{1.832732in}}%
\pgfpathlineto{\pgfqpoint{1.847154in}{1.702331in}}%
\pgfpathlineto{\pgfqpoint{1.855961in}{1.764116in}}%
\pgfpathlineto{\pgfqpoint{1.864768in}{1.860167in}}%
\pgfpathlineto{\pgfqpoint{1.873575in}{2.052326in}}%
\pgfpathlineto{\pgfqpoint{1.891188in}{1.908207in}}%
\pgfpathlineto{\pgfqpoint{1.899995in}{2.175868in}}%
\pgfpathlineto{\pgfqpoint{1.908802in}{1.963133in}}%
\pgfpathlineto{\pgfqpoint{1.917609in}{2.038608in}}%
\pgfpathlineto{\pgfqpoint{1.926416in}{1.867054in}}%
\pgfpathlineto{\pgfqpoint{1.944029in}{2.244484in}}%
\pgfpathlineto{\pgfqpoint{1.952836in}{2.072930in}}%
\pgfpathlineto{\pgfqpoint{1.961643in}{2.244484in}}%
\pgfpathlineto{\pgfqpoint{1.970450in}{2.134687in}}%
\pgfpathlineto{\pgfqpoint{1.979257in}{2.141546in}}%
\pgfpathlineto{\pgfqpoint{1.988063in}{2.100365in}}%
\pgfpathlineto{\pgfqpoint{1.996870in}{1.942529in}}%
\pgfpathlineto{\pgfqpoint{2.005677in}{1.613110in}}%
\pgfpathlineto{\pgfqpoint{2.014484in}{1.839591in}}%
\pgfpathlineto{\pgfqpoint{2.023291in}{1.956246in}}%
\pgfpathlineto{\pgfqpoint{2.032097in}{1.722935in}}%
\pgfpathlineto{\pgfqpoint{2.040904in}{1.990569in}}%
\pgfpathlineto{\pgfqpoint{2.049711in}{2.011173in}}%
\pgfpathlineto{\pgfqpoint{2.058518in}{1.983710in}}%
\pgfpathlineto{\pgfqpoint{2.067325in}{2.072930in}}%
\pgfpathlineto{\pgfqpoint{2.076132in}{2.319987in}}%
\pgfpathlineto{\pgfqpoint{2.084938in}{2.409207in}}%
\pgfpathlineto{\pgfqpoint{2.093745in}{2.395462in}}%
\pgfpathlineto{\pgfqpoint{2.102552in}{2.100365in}}%
\pgfpathlineto{\pgfqpoint{2.111359in}{2.141546in}}%
\pgfpathlineto{\pgfqpoint{2.120166in}{1.860167in}}%
\pgfpathlineto{\pgfqpoint{2.128972in}{1.503314in}}%
\pgfpathlineto{\pgfqpoint{2.137779in}{1.832732in}}%
\pgfpathlineto{\pgfqpoint{2.146586in}{2.107252in}}%
\pgfpathlineto{\pgfqpoint{2.155393in}{1.812128in}}%
\pgfpathlineto{\pgfqpoint{2.164200in}{1.915094in}}%
\pgfpathlineto{\pgfqpoint{2.173007in}{1.976851in}}%
\pgfpathlineto{\pgfqpoint{2.181813in}{1.894489in}}%
\pgfpathlineto{\pgfqpoint{2.190620in}{1.867054in}}%
\pgfpathlineto{\pgfqpoint{2.199427in}{1.928811in}}%
\pgfpathlineto{\pgfqpoint{2.208234in}{1.674896in}}%
\pgfpathlineto{\pgfqpoint{2.217041in}{1.956246in}}%
\pgfpathlineto{\pgfqpoint{2.225847in}{1.661150in}}%
\pgfpathlineto{\pgfqpoint{2.234654in}{1.784692in}}%
\pgfpathlineto{\pgfqpoint{2.243461in}{1.963133in}}%
\pgfpathlineto{\pgfqpoint{2.252268in}{2.244484in}}%
\pgfpathlineto{\pgfqpoint{2.261075in}{2.265089in}}%
\pgfpathlineto{\pgfqpoint{2.269882in}{2.018032in}}%
\pgfpathlineto{\pgfqpoint{2.278688in}{1.805269in}}%
\pgfpathlineto{\pgfqpoint{2.287495in}{2.072930in}}%
\pgfpathlineto{\pgfqpoint{2.296302in}{2.223908in}}%
\pgfpathlineto{\pgfqpoint{2.305109in}{2.230767in}}%
\pgfpathlineto{\pgfqpoint{2.313916in}{2.024891in}}%
\pgfpathlineto{\pgfqpoint{2.322722in}{1.860167in}}%
\pgfpathlineto{\pgfqpoint{2.331529in}{2.120970in}}%
\pgfpathlineto{\pgfqpoint{2.340336in}{2.024891in}}%
\pgfpathlineto{\pgfqpoint{2.349143in}{2.031749in}}%
\pgfpathlineto{\pgfqpoint{2.357950in}{1.928811in}}%
\pgfpathlineto{\pgfqpoint{2.366757in}{1.784692in}}%
\pgfpathlineto{\pgfqpoint{2.384370in}{2.347422in}}%
\pgfpathlineto{\pgfqpoint{2.393177in}{2.100365in}}%
\pgfpathlineto{\pgfqpoint{2.401984in}{1.997427in}}%
\pgfpathlineto{\pgfqpoint{2.410791in}{2.134687in}}%
\pgfpathlineto{\pgfqpoint{2.419597in}{1.942529in}}%
\pgfpathlineto{\pgfqpoint{2.428404in}{1.846450in}}%
\pgfpathlineto{\pgfqpoint{2.437211in}{1.455274in}}%
\pgfpathlineto{\pgfqpoint{2.446018in}{1.716076in}}%
\pgfpathlineto{\pgfqpoint{2.454825in}{1.846450in}}%
\pgfpathlineto{\pgfqpoint{2.463632in}{2.127829in}}%
\pgfpathlineto{\pgfqpoint{2.472438in}{2.011173in}}%
\pgfpathlineto{\pgfqpoint{2.481245in}{2.093507in}}%
\pgfpathlineto{\pgfqpoint{2.490052in}{2.402349in}}%
\pgfpathlineto{\pgfqpoint{2.498859in}{2.402349in}}%
\pgfpathlineto{\pgfqpoint{2.507666in}{2.018032in}}%
\pgfpathlineto{\pgfqpoint{2.516472in}{2.031749in}}%
\pgfpathlineto{\pgfqpoint{2.525279in}{1.928811in}}%
\pgfpathlineto{\pgfqpoint{2.534086in}{2.114111in}}%
\pgfpathlineto{\pgfqpoint{2.542893in}{2.196445in}}%
\pgfpathlineto{\pgfqpoint{2.551700in}{2.525863in}}%
\pgfpathlineto{\pgfqpoint{2.560507in}{2.635660in}}%
\pgfpathlineto{\pgfqpoint{2.569313in}{2.354309in}}%
\pgfpathlineto{\pgfqpoint{2.578120in}{2.429784in}}%
\pgfpathlineto{\pgfqpoint{2.586927in}{2.230767in}}%
\pgfpathlineto{\pgfqpoint{2.595734in}{1.976851in}}%
\pgfpathlineto{\pgfqpoint{2.604541in}{2.031749in}}%
\pgfpathlineto{\pgfqpoint{2.613347in}{1.777834in}}%
\pgfpathlineto{\pgfqpoint{2.622154in}{1.880772in}}%
\pgfpathlineto{\pgfqpoint{2.630961in}{1.640574in}}%
\pgfpathlineto{\pgfqpoint{2.639768in}{1.537636in}}%
\pgfpathlineto{\pgfqpoint{2.648575in}{1.661150in}}%
\pgfpathlineto{\pgfqpoint{2.657382in}{1.805269in}}%
\pgfpathlineto{\pgfqpoint{2.666188in}{1.455274in}}%
\pgfpathlineto{\pgfqpoint{2.674995in}{1.523918in}}%
\pgfpathlineto{\pgfqpoint{2.683802in}{1.743512in}}%
\pgfpathlineto{\pgfqpoint{2.692609in}{2.052326in}}%
\pgfpathlineto{\pgfqpoint{2.701416in}{1.668037in}}%
\pgfpathlineto{\pgfqpoint{2.710222in}{1.867054in}}%
\pgfpathlineto{\pgfqpoint{2.719029in}{1.963133in}}%
\pgfpathlineto{\pgfqpoint{2.727836in}{1.681754in}}%
\pgfpathlineto{\pgfqpoint{2.736643in}{1.832732in}}%
\pgfpathlineto{\pgfqpoint{2.745450in}{1.921953in}}%
\pgfpathlineto{\pgfqpoint{2.754257in}{1.887630in}}%
\pgfpathlineto{\pgfqpoint{2.763063in}{1.997427in}}%
\pgfpathlineto{\pgfqpoint{2.771870in}{1.702331in}}%
\pgfpathlineto{\pgfqpoint{2.780677in}{1.832732in}}%
\pgfpathlineto{\pgfqpoint{2.789484in}{1.578816in}}%
\pgfpathlineto{\pgfqpoint{2.798291in}{1.976851in}}%
\pgfpathlineto{\pgfqpoint{2.807097in}{1.949388in}}%
\pgfpathlineto{\pgfqpoint{2.815904in}{1.942529in}}%
\pgfpathlineto{\pgfqpoint{2.824711in}{2.416066in}}%
\pgfpathlineto{\pgfqpoint{2.833518in}{2.196445in}}%
\pgfpathlineto{\pgfqpoint{2.842325in}{2.244484in}}%
\pgfpathlineto{\pgfqpoint{2.851132in}{2.388603in}}%
\pgfpathlineto{\pgfqpoint{2.859938in}{2.361168in}}%
\pgfpathlineto{\pgfqpoint{2.868745in}{1.873913in}}%
\pgfpathlineto{\pgfqpoint{2.877552in}{1.613110in}}%
\pgfpathlineto{\pgfqpoint{2.886359in}{1.537636in}}%
\pgfpathlineto{\pgfqpoint{2.895166in}{1.661150in}}%
\pgfpathlineto{\pgfqpoint{2.903972in}{1.441556in}}%
\pgfpathlineto{\pgfqpoint{2.912779in}{1.969992in}}%
\pgfpathlineto{\pgfqpoint{2.921586in}{1.983710in}}%
\pgfpathlineto{\pgfqpoint{2.930393in}{1.990569in}}%
\pgfpathlineto{\pgfqpoint{2.939200in}{1.873913in}}%
\pgfpathlineto{\pgfqpoint{2.948007in}{1.784692in}}%
\pgfpathlineto{\pgfqpoint{2.956813in}{1.764116in}}%
\pgfpathlineto{\pgfqpoint{2.965620in}{1.784692in}}%
\pgfpathlineto{\pgfqpoint{2.974427in}{1.873913in}}%
\pgfpathlineto{\pgfqpoint{2.983234in}{1.654291in}}%
\pgfpathlineto{\pgfqpoint{2.992041in}{1.915094in}}%
\pgfpathlineto{\pgfqpoint{3.000847in}{1.798410in}}%
\pgfpathlineto{\pgfqpoint{3.009654in}{2.086648in}}%
\pgfpathlineto{\pgfqpoint{3.018461in}{2.086648in}}%
\pgfpathlineto{\pgfqpoint{3.027268in}{1.942529in}}%
\pgfpathlineto{\pgfqpoint{3.036075in}{1.928811in}}%
\pgfpathlineto{\pgfqpoint{3.044882in}{1.757229in}}%
\pgfpathlineto{\pgfqpoint{3.053688in}{1.716076in}}%
\pgfpathlineto{\pgfqpoint{3.062495in}{1.599393in}}%
\pgfpathlineto{\pgfqpoint{3.071302in}{1.592534in}}%
\pgfpathlineto{\pgfqpoint{3.088916in}{1.482737in}}%
\pgfpathlineto{\pgfqpoint{3.097722in}{1.695472in}}%
\pgfpathlineto{\pgfqpoint{3.106529in}{2.100365in}}%
\pgfpathlineto{\pgfqpoint{3.115336in}{1.853308in}}%
\pgfpathlineto{\pgfqpoint{3.124143in}{2.024891in}}%
\pgfpathlineto{\pgfqpoint{3.132950in}{1.770975in}}%
\pgfpathlineto{\pgfqpoint{3.150563in}{1.420952in}}%
\pgfpathlineto{\pgfqpoint{3.159370in}{1.633715in}}%
\pgfpathlineto{\pgfqpoint{3.168177in}{1.777834in}}%
\pgfpathlineto{\pgfqpoint{3.176984in}{1.784692in}}%
\pgfpathlineto{\pgfqpoint{3.185791in}{1.496455in}}%
\pgfpathlineto{\pgfqpoint{3.194597in}{1.578816in}}%
\pgfpathlineto{\pgfqpoint{3.203404in}{1.640574in}}%
\pgfpathlineto{\pgfqpoint{3.212211in}{1.510172in}}%
\pgfpathlineto{\pgfqpoint{3.221018in}{1.489596in}}%
\pgfpathlineto{\pgfqpoint{3.238632in}{2.148405in}}%
\pgfpathlineto{\pgfqpoint{3.247438in}{2.045467in}}%
\pgfpathlineto{\pgfqpoint{3.265052in}{1.770975in}}%
\pgfpathlineto{\pgfqpoint{3.273859in}{1.606252in}}%
\pgfpathlineto{\pgfqpoint{3.282666in}{1.825873in}}%
\pgfpathlineto{\pgfqpoint{3.291472in}{1.722935in}}%
\pgfpathlineto{\pgfqpoint{3.300279in}{1.736653in}}%
\pgfpathlineto{\pgfqpoint{3.309086in}{1.915094in}}%
\pgfpathlineto{\pgfqpoint{3.317893in}{1.825873in}}%
\pgfpathlineto{\pgfqpoint{3.326700in}{1.928811in}}%
\pgfpathlineto{\pgfqpoint{3.335507in}{2.278806in}}%
\pgfpathlineto{\pgfqpoint{3.344313in}{2.066071in}}%
\pgfpathlineto{\pgfqpoint{3.353120in}{2.251343in}}%
\pgfpathlineto{\pgfqpoint{3.361927in}{2.519004in}}%
\pgfpathlineto{\pgfqpoint{3.370734in}{2.587620in}}%
\pgfpathlineto{\pgfqpoint{3.379541in}{2.340563in}}%
\pgfpathlineto{\pgfqpoint{3.388347in}{2.319987in}}%
\pgfpathlineto{\pgfqpoint{3.405961in}{1.942529in}}%
\pgfpathlineto{\pgfqpoint{3.414768in}{2.038608in}}%
\pgfpathlineto{\pgfqpoint{3.423575in}{2.079789in}}%
\pgfpathlineto{\pgfqpoint{3.432382in}{2.169009in}}%
\pgfpathlineto{\pgfqpoint{3.441188in}{2.045467in}}%
\pgfpathlineto{\pgfqpoint{3.449995in}{1.819015in}}%
\pgfpathlineto{\pgfqpoint{3.458802in}{1.489596in}}%
\pgfpathlineto{\pgfqpoint{3.467609in}{1.640574in}}%
\pgfpathlineto{\pgfqpoint{3.476416in}{1.592534in}}%
\pgfpathlineto{\pgfqpoint{3.485222in}{1.915094in}}%
\pgfpathlineto{\pgfqpoint{3.494029in}{1.640574in}}%
\pgfpathlineto{\pgfqpoint{3.502836in}{1.764116in}}%
\pgfpathlineto{\pgfqpoint{3.511643in}{1.613110in}}%
\pgfpathlineto{\pgfqpoint{3.529257in}{2.333705in}}%
\pgfpathlineto{\pgfqpoint{3.538063in}{1.805269in}}%
\pgfpathlineto{\pgfqpoint{3.546870in}{1.695472in}}%
\pgfpathlineto{\pgfqpoint{3.555677in}{2.251343in}}%
\pgfpathlineto{\pgfqpoint{3.564484in}{2.422925in}}%
\pgfpathlineto{\pgfqpoint{3.573291in}{2.251343in}}%
\pgfpathlineto{\pgfqpoint{3.582097in}{2.114111in}}%
\pgfpathlineto{\pgfqpoint{3.590904in}{2.395462in}}%
\pgfpathlineto{\pgfqpoint{3.599711in}{2.388603in}}%
\pgfpathlineto{\pgfqpoint{3.608518in}{2.587620in}}%
\pgfpathlineto{\pgfqpoint{3.617325in}{2.450388in}}%
\pgfpathlineto{\pgfqpoint{3.626132in}{2.271947in}}%
\pgfpathlineto{\pgfqpoint{3.634938in}{1.997427in}}%
\pgfpathlineto{\pgfqpoint{3.643745in}{2.306269in}}%
\pgfpathlineto{\pgfqpoint{3.652552in}{2.230767in}}%
\pgfpathlineto{\pgfqpoint{3.661359in}{2.114111in}}%
\pgfpathlineto{\pgfqpoint{3.670166in}{2.018032in}}%
\pgfpathlineto{\pgfqpoint{3.678972in}{2.203303in}}%
\pgfpathlineto{\pgfqpoint{3.687779in}{1.969992in}}%
\pgfpathlineto{\pgfqpoint{3.696586in}{1.942529in}}%
\pgfpathlineto{\pgfqpoint{3.705393in}{1.846450in}}%
\pgfpathlineto{\pgfqpoint{3.714200in}{1.853308in}}%
\pgfpathlineto{\pgfqpoint{3.723007in}{1.523918in}}%
\pgfpathlineto{\pgfqpoint{3.731813in}{1.832732in}}%
\pgfpathlineto{\pgfqpoint{3.740620in}{2.306269in}}%
\pgfpathlineto{\pgfqpoint{3.749427in}{1.867054in}}%
\pgfpathlineto{\pgfqpoint{3.758234in}{1.853308in}}%
\pgfpathlineto{\pgfqpoint{3.767041in}{1.867054in}}%
\pgfpathlineto{\pgfqpoint{3.775847in}{1.647432in}}%
\pgfpathlineto{\pgfqpoint{3.784654in}{1.798410in}}%
\pgfpathlineto{\pgfqpoint{3.793461in}{1.846450in}}%
\pgfpathlineto{\pgfqpoint{3.802268in}{1.839591in}}%
\pgfpathlineto{\pgfqpoint{3.811075in}{1.798410in}}%
\pgfpathlineto{\pgfqpoint{3.819882in}{2.079789in}}%
\pgfpathlineto{\pgfqpoint{3.828688in}{1.812128in}}%
\pgfpathlineto{\pgfqpoint{3.837495in}{1.729794in}}%
\pgfpathlineto{\pgfqpoint{3.846302in}{1.770975in}}%
\pgfpathlineto{\pgfqpoint{3.855109in}{2.045467in}}%
\pgfpathlineto{\pgfqpoint{3.863916in}{2.457247in}}%
\pgfpathlineto{\pgfqpoint{3.872722in}{2.354309in}}%
\pgfpathlineto{\pgfqpoint{3.881529in}{2.354309in}}%
\pgfpathlineto{\pgfqpoint{3.890336in}{2.230767in}}%
\pgfpathlineto{\pgfqpoint{3.899143in}{2.086648in}}%
\pgfpathlineto{\pgfqpoint{3.907950in}{2.086648in}}%
\pgfpathlineto{\pgfqpoint{3.916757in}{1.942529in}}%
\pgfpathlineto{\pgfqpoint{3.925563in}{1.969992in}}%
\pgfpathlineto{\pgfqpoint{3.934370in}{1.729794in}}%
\pgfpathlineto{\pgfqpoint{3.943177in}{1.571958in}}%
\pgfpathlineto{\pgfqpoint{3.951984in}{1.867054in}}%
\pgfpathlineto{\pgfqpoint{3.960791in}{2.072930in}}%
\pgfpathlineto{\pgfqpoint{3.969597in}{2.169009in}}%
\pgfpathlineto{\pgfqpoint{3.978404in}{1.976851in}}%
\pgfpathlineto{\pgfqpoint{3.987211in}{2.004286in}}%
\pgfpathlineto{\pgfqpoint{3.996018in}{2.100365in}}%
\pgfpathlineto{\pgfqpoint{4.004825in}{1.983710in}}%
\pgfpathlineto{\pgfqpoint{4.013632in}{1.722935in}}%
\pgfpathlineto{\pgfqpoint{4.022438in}{1.496455in}}%
\pgfpathlineto{\pgfqpoint{4.031245in}{1.228822in}}%
\pgfpathlineto{\pgfqpoint{4.040052in}{1.283720in}}%
\pgfpathlineto{\pgfqpoint{4.048859in}{1.290579in}}%
\pgfpathlineto{\pgfqpoint{4.057666in}{1.283720in}}%
\pgfpathlineto{\pgfqpoint{4.066472in}{1.290579in}}%
\pgfpathlineto{\pgfqpoint{4.075279in}{1.695472in}}%
\pgfpathlineto{\pgfqpoint{4.084086in}{1.729794in}}%
\pgfpathlineto{\pgfqpoint{4.092893in}{2.271947in}}%
\pgfpathlineto{\pgfqpoint{4.101700in}{2.182727in}}%
\pgfpathlineto{\pgfqpoint{4.110507in}{2.134687in}}%
\pgfpathlineto{\pgfqpoint{4.119313in}{2.018032in}}%
\pgfpathlineto{\pgfqpoint{4.128120in}{1.935670in}}%
\pgfpathlineto{\pgfqpoint{4.136927in}{1.386658in}}%
\pgfpathlineto{\pgfqpoint{4.154541in}{1.503314in}}%
\pgfpathlineto{\pgfqpoint{4.163347in}{1.764116in}}%
\pgfpathlineto{\pgfqpoint{4.172154in}{1.908207in}}%
\pgfpathlineto{\pgfqpoint{4.180961in}{1.942529in}}%
\pgfpathlineto{\pgfqpoint{4.189768in}{1.633715in}}%
\pgfpathlineto{\pgfqpoint{4.198575in}{1.462133in}}%
\pgfpathlineto{\pgfqpoint{4.207382in}{1.146460in}}%
\pgfpathlineto{\pgfqpoint{4.216188in}{1.173895in}}%
\pgfpathlineto{\pgfqpoint{4.224995in}{1.249398in}}%
\pgfpathlineto{\pgfqpoint{4.233802in}{1.578816in}}%
\pgfpathlineto{\pgfqpoint{4.242609in}{1.825873in}}%
\pgfpathlineto{\pgfqpoint{4.251416in}{2.038608in}}%
\pgfpathlineto{\pgfqpoint{4.260222in}{2.038608in}}%
\pgfpathlineto{\pgfqpoint{4.269029in}{2.649406in}}%
\pgfpathlineto{\pgfqpoint{4.277836in}{2.354309in}}%
\pgfpathlineto{\pgfqpoint{4.286643in}{2.519004in}}%
\pgfpathlineto{\pgfqpoint{4.295450in}{2.484682in}}%
\pgfpathlineto{\pgfqpoint{4.304257in}{2.573903in}}%
\pgfpathlineto{\pgfqpoint{4.313063in}{2.395462in}}%
\pgfpathlineto{\pgfqpoint{4.321870in}{1.784692in}}%
\pgfpathlineto{\pgfqpoint{4.330677in}{1.997427in}}%
\pgfpathlineto{\pgfqpoint{4.339484in}{1.928811in}}%
\pgfpathlineto{\pgfqpoint{4.348291in}{1.661150in}}%
\pgfpathlineto{\pgfqpoint{4.365904in}{1.558212in}}%
\pgfpathlineto{\pgfqpoint{4.374711in}{1.908207in}}%
\pgfpathlineto{\pgfqpoint{4.383518in}{1.688613in}}%
\pgfpathlineto{\pgfqpoint{4.392325in}{1.743512in}}%
\pgfpathlineto{\pgfqpoint{4.401132in}{2.217049in}}%
\pgfpathlineto{\pgfqpoint{4.409938in}{2.285665in}}%
\pgfpathlineto{\pgfqpoint{4.427552in}{2.100365in}}%
\pgfpathlineto{\pgfqpoint{4.436359in}{2.189586in}}%
\pgfpathlineto{\pgfqpoint{4.445166in}{2.292524in}}%
\pgfpathlineto{\pgfqpoint{4.453972in}{2.059213in}}%
\pgfpathlineto{\pgfqpoint{4.462779in}{2.326846in}}%
\pgfpathlineto{\pgfqpoint{4.471586in}{2.100365in}}%
\pgfpathlineto{\pgfqpoint{4.480393in}{1.908207in}}%
\pgfpathlineto{\pgfqpoint{4.489200in}{2.024891in}}%
\pgfpathlineto{\pgfqpoint{4.498007in}{1.894489in}}%
\pgfpathlineto{\pgfqpoint{4.506813in}{2.066071in}}%
\pgfpathlineto{\pgfqpoint{4.515620in}{2.203303in}}%
\pgfpathlineto{\pgfqpoint{4.524427in}{2.169009in}}%
\pgfpathlineto{\pgfqpoint{4.533234in}{1.983710in}}%
\pgfpathlineto{\pgfqpoint{4.542041in}{2.100365in}}%
\pgfpathlineto{\pgfqpoint{4.550847in}{2.169009in}}%
\pgfpathlineto{\pgfqpoint{4.559654in}{2.162151in}}%
\pgfpathlineto{\pgfqpoint{4.568461in}{1.949388in}}%
\pgfpathlineto{\pgfqpoint{4.586075in}{2.573903in}}%
\pgfpathlineto{\pgfqpoint{4.594882in}{2.326846in}}%
\pgfpathlineto{\pgfqpoint{4.603688in}{2.306269in}}%
\pgfpathlineto{\pgfqpoint{4.612495in}{2.361168in}}%
\pgfpathlineto{\pgfqpoint{4.621302in}{2.018032in}}%
\pgfpathlineto{\pgfqpoint{4.630109in}{1.894489in}}%
\pgfpathlineto{\pgfqpoint{4.638916in}{1.921953in}}%
\pgfpathlineto{\pgfqpoint{4.647722in}{1.901348in}}%
\pgfpathlineto{\pgfqpoint{4.656529in}{2.100365in}}%
\pgfpathlineto{\pgfqpoint{4.665336in}{2.402349in}}%
\pgfpathlineto{\pgfqpoint{4.674143in}{2.182727in}}%
\pgfpathlineto{\pgfqpoint{4.682950in}{2.189586in}}%
\pgfpathlineto{\pgfqpoint{4.691757in}{2.045467in}}%
\pgfpathlineto{\pgfqpoint{4.700563in}{1.873913in}}%
\pgfpathlineto{\pgfqpoint{4.709370in}{1.729794in}}%
\pgfpathlineto{\pgfqpoint{4.718177in}{1.743512in}}%
\pgfpathlineto{\pgfqpoint{4.726984in}{1.722935in}}%
\pgfpathlineto{\pgfqpoint{4.735791in}{1.867054in}}%
\pgfpathlineto{\pgfqpoint{4.744597in}{1.558212in}}%
\pgfpathlineto{\pgfqpoint{4.753404in}{1.764116in}}%
\pgfpathlineto{\pgfqpoint{4.762211in}{1.517031in}}%
\pgfpathlineto{\pgfqpoint{4.771018in}{1.819015in}}%
\pgfpathlineto{\pgfqpoint{4.779825in}{1.853308in}}%
\pgfpathlineto{\pgfqpoint{4.788632in}{1.832732in}}%
\pgfpathlineto{\pgfqpoint{4.797438in}{2.011173in}}%
\pgfpathlineto{\pgfqpoint{4.806245in}{1.819015in}}%
\pgfpathlineto{\pgfqpoint{4.815052in}{1.688613in}}%
\pgfpathlineto{\pgfqpoint{4.823859in}{1.688613in}}%
\pgfpathlineto{\pgfqpoint{4.832666in}{1.537636in}}%
\pgfpathlineto{\pgfqpoint{4.841472in}{1.565071in}}%
\pgfpathlineto{\pgfqpoint{4.850279in}{1.640574in}}%
\pgfpathlineto{\pgfqpoint{4.859086in}{1.668037in}}%
\pgfpathlineto{\pgfqpoint{4.867893in}{1.935670in}}%
\pgfpathlineto{\pgfqpoint{4.876700in}{1.867054in}}%
\pgfpathlineto{\pgfqpoint{4.885507in}{1.482737in}}%
\pgfpathlineto{\pgfqpoint{4.894313in}{1.448415in}}%
\pgfpathlineto{\pgfqpoint{4.911927in}{1.674896in}}%
\pgfpathlineto{\pgfqpoint{4.920734in}{1.873913in}}%
\pgfpathlineto{\pgfqpoint{4.929541in}{1.853308in}}%
\pgfpathlineto{\pgfqpoint{4.938347in}{2.196445in}}%
\pgfpathlineto{\pgfqpoint{4.947154in}{2.155292in}}%
\pgfpathlineto{\pgfqpoint{4.955961in}{2.011173in}}%
\pgfpathlineto{\pgfqpoint{4.964768in}{1.976851in}}%
\pgfpathlineto{\pgfqpoint{4.973575in}{2.477823in}}%
\pgfpathlineto{\pgfqpoint{4.982382in}{2.361168in}}%
\pgfpathlineto{\pgfqpoint{4.991188in}{2.416066in}}%
\pgfpathlineto{\pgfqpoint{4.999995in}{2.004286in}}%
\pgfpathlineto{\pgfqpoint{5.008802in}{1.743512in}}%
\pgfpathlineto{\pgfqpoint{5.017609in}{1.791551in}}%
\pgfpathlineto{\pgfqpoint{5.026416in}{1.956246in}}%
\pgfpathlineto{\pgfqpoint{5.035222in}{2.155292in}}%
\pgfpathlineto{\pgfqpoint{5.044029in}{1.901348in}}%
\pgfpathlineto{\pgfqpoint{5.061643in}{2.210190in}}%
\pgfpathlineto{\pgfqpoint{5.070450in}{2.155292in}}%
\pgfpathlineto{\pgfqpoint{5.079257in}{2.422925in}}%
\pgfpathlineto{\pgfqpoint{5.088063in}{2.258230in}}%
\pgfpathlineto{\pgfqpoint{5.096870in}{2.244484in}}%
\pgfpathlineto{\pgfqpoint{5.105677in}{1.853308in}}%
\pgfpathlineto{\pgfqpoint{5.114484in}{1.956246in}}%
\pgfpathlineto{\pgfqpoint{5.123291in}{1.969992in}}%
\pgfpathlineto{\pgfqpoint{5.132097in}{2.169009in}}%
\pgfpathlineto{\pgfqpoint{5.140904in}{2.319987in}}%
\pgfpathlineto{\pgfqpoint{5.149711in}{2.223908in}}%
\pgfpathlineto{\pgfqpoint{5.158518in}{2.155292in}}%
\pgfpathlineto{\pgfqpoint{5.167325in}{1.668037in}}%
\pgfpathlineto{\pgfqpoint{5.176132in}{1.626856in}}%
\pgfpathlineto{\pgfqpoint{5.184938in}{1.791551in}}%
\pgfpathlineto{\pgfqpoint{5.193745in}{1.640574in}}%
\pgfpathlineto{\pgfqpoint{5.202552in}{1.791551in}}%
\pgfpathlineto{\pgfqpoint{5.211359in}{1.846450in}}%
\pgfpathlineto{\pgfqpoint{5.220166in}{1.825873in}}%
\pgfpathlineto{\pgfqpoint{5.228972in}{2.107252in}}%
\pgfpathlineto{\pgfqpoint{5.237779in}{2.169009in}}%
\pgfpathlineto{\pgfqpoint{5.246586in}{1.997427in}}%
\pgfpathlineto{\pgfqpoint{5.255393in}{2.024891in}}%
\pgfpathlineto{\pgfqpoint{5.264200in}{2.072930in}}%
\pgfpathlineto{\pgfqpoint{5.273007in}{2.319987in}}%
\pgfpathlineto{\pgfqpoint{5.281813in}{2.278806in}}%
\pgfpathlineto{\pgfqpoint{5.290620in}{2.244484in}}%
\pgfpathlineto{\pgfqpoint{5.299427in}{2.066071in}}%
\pgfpathlineto{\pgfqpoint{5.308234in}{2.066071in}}%
\pgfpathlineto{\pgfqpoint{5.317041in}{1.805269in}}%
\pgfpathlineto{\pgfqpoint{5.325847in}{1.750370in}}%
\pgfpathlineto{\pgfqpoint{5.334654in}{1.606252in}}%
\pgfpathlineto{\pgfqpoint{5.343461in}{1.544494in}}%
\pgfpathlineto{\pgfqpoint{5.352268in}{1.324873in}}%
\pgfpathlineto{\pgfqpoint{5.361075in}{1.695472in}}%
\pgfpathlineto{\pgfqpoint{5.369882in}{2.114111in}}%
\pgfpathlineto{\pgfqpoint{5.378688in}{2.079789in}}%
\pgfpathlineto{\pgfqpoint{5.387495in}{1.921953in}}%
\pgfpathlineto{\pgfqpoint{5.396302in}{1.901348in}}%
\pgfpathlineto{\pgfqpoint{5.405109in}{1.846450in}}%
\pgfpathlineto{\pgfqpoint{5.413916in}{1.530777in}}%
\pgfpathlineto{\pgfqpoint{5.422722in}{1.832732in}}%
\pgfpathlineto{\pgfqpoint{5.431529in}{1.791551in}}%
\pgfpathlineto{\pgfqpoint{5.440336in}{1.873913in}}%
\pgfpathlineto{\pgfqpoint{5.449143in}{1.791551in}}%
\pgfpathlineto{\pgfqpoint{5.457950in}{1.963133in}}%
\pgfpathlineto{\pgfqpoint{5.466757in}{1.619997in}}%
\pgfpathlineto{\pgfqpoint{5.475563in}{1.976851in}}%
\pgfpathlineto{\pgfqpoint{5.484370in}{1.784692in}}%
\pgfpathlineto{\pgfqpoint{5.493177in}{1.784692in}}%
\pgfpathlineto{\pgfqpoint{5.501984in}{1.585675in}}%
\pgfpathlineto{\pgfqpoint{5.510791in}{1.825873in}}%
\pgfpathlineto{\pgfqpoint{5.519597in}{1.331760in}}%
\pgfpathlineto{\pgfqpoint{5.528404in}{1.523918in}}%
\pgfpathlineto{\pgfqpoint{5.537211in}{1.565071in}}%
\pgfpathlineto{\pgfqpoint{5.546018in}{1.276861in}}%
\pgfpathlineto{\pgfqpoint{5.554825in}{1.372912in}}%
\pgfpathlineto{\pgfqpoint{5.572438in}{1.764116in}}%
\pgfpathlineto{\pgfqpoint{5.581245in}{1.681754in}}%
\pgfpathlineto{\pgfqpoint{5.590052in}{1.530777in}}%
\pgfpathlineto{\pgfqpoint{5.598859in}{1.668037in}}%
\pgfpathlineto{\pgfqpoint{5.607666in}{1.249398in}}%
\pgfpathlineto{\pgfqpoint{5.616472in}{1.400376in}}%
\pgfpathlineto{\pgfqpoint{5.625279in}{1.585675in}}%
\pgfpathlineto{\pgfqpoint{5.634086in}{1.537636in}}%
\pgfpathlineto{\pgfqpoint{5.642893in}{1.585675in}}%
\pgfpathlineto{\pgfqpoint{5.651700in}{1.654291in}}%
\pgfpathlineto{\pgfqpoint{5.660507in}{1.949388in}}%
\pgfpathlineto{\pgfqpoint{5.669313in}{1.963133in}}%
\pgfpathlineto{\pgfqpoint{5.678120in}{2.148405in}}%
\pgfpathlineto{\pgfqpoint{5.686927in}{2.189586in}}%
\pgfpathlineto{\pgfqpoint{5.695734in}{1.908207in}}%
\pgfpathlineto{\pgfqpoint{5.704541in}{2.038608in}}%
\pgfpathlineto{\pgfqpoint{5.713347in}{1.935670in}}%
\pgfpathlineto{\pgfqpoint{5.722154in}{2.018032in}}%
\pgfpathlineto{\pgfqpoint{5.730961in}{2.038608in}}%
\pgfpathlineto{\pgfqpoint{5.739768in}{1.757229in}}%
\pgfpathlineto{\pgfqpoint{5.748575in}{1.908207in}}%
\pgfpathlineto{\pgfqpoint{5.757382in}{1.537636in}}%
\pgfpathlineto{\pgfqpoint{5.766188in}{1.668037in}}%
\pgfpathlineto{\pgfqpoint{5.774995in}{1.359195in}}%
\pgfpathlineto{\pgfqpoint{5.783802in}{1.949388in}}%
\pgfpathlineto{\pgfqpoint{5.792609in}{1.949388in}}%
\pgfpathlineto{\pgfqpoint{5.801416in}{2.162151in}}%
\pgfpathlineto{\pgfqpoint{5.810222in}{2.422925in}}%
\pgfpathlineto{\pgfqpoint{5.819029in}{2.416066in}}%
\pgfpathlineto{\pgfqpoint{5.827836in}{2.141546in}}%
\pgfpathlineto{\pgfqpoint{5.836643in}{2.072930in}}%
\pgfpathlineto{\pgfqpoint{5.845450in}{2.107252in}}%
\pgfpathlineto{\pgfqpoint{5.854257in}{2.326846in}}%
\pgfpathlineto{\pgfqpoint{5.863063in}{2.258230in}}%
\pgfpathlineto{\pgfqpoint{5.871870in}{2.416066in}}%
\pgfpathlineto{\pgfqpoint{5.889484in}{1.633715in}}%
\pgfpathlineto{\pgfqpoint{5.898291in}{1.757229in}}%
\pgfpathlineto{\pgfqpoint{5.907097in}{1.283720in}}%
\pgfpathlineto{\pgfqpoint{5.915904in}{1.716076in}}%
\pgfpathlineto{\pgfqpoint{5.924711in}{1.544494in}}%
\pgfpathlineto{\pgfqpoint{5.933518in}{1.798410in}}%
\pgfpathlineto{\pgfqpoint{5.942325in}{1.908207in}}%
\pgfpathlineto{\pgfqpoint{5.951132in}{1.791551in}}%
\pgfpathlineto{\pgfqpoint{5.959938in}{1.839591in}}%
\pgfpathlineto{\pgfqpoint{5.968745in}{1.921953in}}%
\pgfpathlineto{\pgfqpoint{5.977552in}{2.210190in}}%
\pgfpathlineto{\pgfqpoint{5.986359in}{2.354309in}}%
\pgfpathlineto{\pgfqpoint{5.995166in}{2.203303in}}%
\pgfpathlineto{\pgfqpoint{6.003972in}{1.928811in}}%
\pgfpathlineto{\pgfqpoint{6.012779in}{2.024891in}}%
\pgfpathlineto{\pgfqpoint{6.021586in}{1.963133in}}%
\pgfpathlineto{\pgfqpoint{6.030393in}{1.839591in}}%
\pgfpathlineto{\pgfqpoint{6.039200in}{2.223908in}}%
\pgfpathlineto{\pgfqpoint{6.048007in}{2.258230in}}%
\pgfpathlineto{\pgfqpoint{6.065620in}{1.640574in}}%
\pgfpathlineto{\pgfqpoint{6.074427in}{1.736653in}}%
\pgfpathlineto{\pgfqpoint{6.083234in}{1.867054in}}%
\pgfpathlineto{\pgfqpoint{6.092041in}{2.148405in}}%
\pgfpathlineto{\pgfqpoint{6.100847in}{1.983710in}}%
\pgfpathlineto{\pgfqpoint{6.109654in}{1.956246in}}%
\pgfpathlineto{\pgfqpoint{6.118461in}{1.894489in}}%
\pgfpathlineto{\pgfqpoint{6.127268in}{1.853308in}}%
\pgfpathlineto{\pgfqpoint{6.136075in}{1.846450in}}%
\pgfpathlineto{\pgfqpoint{6.144882in}{2.059213in}}%
\pgfpathlineto{\pgfqpoint{6.162495in}{1.654291in}}%
\pgfpathlineto{\pgfqpoint{6.171302in}{1.702331in}}%
\pgfpathlineto{\pgfqpoint{6.180109in}{1.915094in}}%
\pgfpathlineto{\pgfqpoint{6.188916in}{1.942529in}}%
\pgfpathlineto{\pgfqpoint{6.197722in}{1.647432in}}%
\pgfpathlineto{\pgfqpoint{6.206529in}{1.702331in}}%
\pgfpathlineto{\pgfqpoint{6.215336in}{1.832732in}}%
\pgfpathlineto{\pgfqpoint{6.224143in}{1.764116in}}%
\pgfpathlineto{\pgfqpoint{6.232950in}{1.743512in}}%
\pgfpathlineto{\pgfqpoint{6.241757in}{1.537636in}}%
\pgfpathlineto{\pgfqpoint{6.250563in}{1.503314in}}%
\pgfpathlineto{\pgfqpoint{6.259370in}{1.372912in}}%
\pgfpathlineto{\pgfqpoint{6.268177in}{1.626856in}}%
\pgfpathlineto{\pgfqpoint{6.276984in}{1.976851in}}%
\pgfpathlineto{\pgfqpoint{6.285791in}{1.585675in}}%
\pgfpathlineto{\pgfqpoint{6.294597in}{1.441556in}}%
\pgfpathlineto{\pgfqpoint{6.303404in}{1.393517in}}%
\pgfpathlineto{\pgfqpoint{6.312211in}{1.784692in}}%
\pgfpathlineto{\pgfqpoint{6.321018in}{1.764116in}}%
\pgfpathlineto{\pgfqpoint{6.329825in}{1.633715in}}%
\pgfpathlineto{\pgfqpoint{6.338632in}{1.276861in}}%
\pgfpathlineto{\pgfqpoint{6.347438in}{1.455274in}}%
\pgfpathlineto{\pgfqpoint{6.356245in}{1.489596in}}%
\pgfpathlineto{\pgfqpoint{6.365052in}{1.414093in}}%
\pgfpathlineto{\pgfqpoint{6.373859in}{1.455274in}}%
\pgfpathlineto{\pgfqpoint{6.382666in}{1.228822in}}%
\pgfpathlineto{\pgfqpoint{6.391472in}{1.160177in}}%
\pgfpathlineto{\pgfqpoint{6.400279in}{0.837646in}}%
\pgfpathlineto{\pgfqpoint{6.409086in}{1.002341in}}%
\pgfpathlineto{\pgfqpoint{6.417893in}{1.029776in}}%
\pgfpathlineto{\pgfqpoint{6.426700in}{1.098420in}}%
\pgfpathlineto{\pgfqpoint{6.435507in}{1.558212in}}%
\pgfpathlineto{\pgfqpoint{6.444313in}{2.155292in}}%
\pgfpathlineto{\pgfqpoint{6.453120in}{2.038608in}}%
\pgfpathlineto{\pgfqpoint{6.461927in}{1.956246in}}%
\pgfpathlineto{\pgfqpoint{6.470734in}{1.963133in}}%
\pgfpathlineto{\pgfqpoint{6.479541in}{2.169009in}}%
\pgfpathlineto{\pgfqpoint{6.488347in}{2.045467in}}%
\pgfpathlineto{\pgfqpoint{6.497154in}{2.127829in}}%
\pgfpathlineto{\pgfqpoint{6.505961in}{2.155292in}}%
\pgfpathlineto{\pgfqpoint{6.514768in}{2.374885in}}%
\pgfpathlineto{\pgfqpoint{6.523575in}{2.491541in}}%
\pgfpathlineto{\pgfqpoint{6.532382in}{2.567044in}}%
\pgfpathlineto{\pgfqpoint{6.541188in}{2.059213in}}%
\pgfpathlineto{\pgfqpoint{6.549995in}{2.031749in}}%
\pgfpathlineto{\pgfqpoint{6.558802in}{2.011173in}}%
\pgfpathlineto{\pgfqpoint{6.567609in}{2.031749in}}%
\pgfpathlineto{\pgfqpoint{6.576416in}{2.114111in}}%
\pgfpathlineto{\pgfqpoint{6.585222in}{2.059213in}}%
\pgfpathlineto{\pgfqpoint{6.602836in}{2.278806in}}%
\pgfpathlineto{\pgfqpoint{6.611643in}{1.709190in}}%
\pgfpathlineto{\pgfqpoint{6.620450in}{1.722935in}}%
\pgfpathlineto{\pgfqpoint{6.629257in}{1.668037in}}%
\pgfpathlineto{\pgfqpoint{6.638063in}{2.059213in}}%
\pgfpathlineto{\pgfqpoint{6.646870in}{1.839591in}}%
\pgfpathlineto{\pgfqpoint{6.655677in}{1.722935in}}%
\pgfpathlineto{\pgfqpoint{6.664484in}{1.942529in}}%
\pgfpathlineto{\pgfqpoint{6.673291in}{1.839591in}}%
\pgfpathlineto{\pgfqpoint{6.682097in}{1.825873in}}%
\pgfpathlineto{\pgfqpoint{6.690904in}{1.702331in}}%
\pgfpathlineto{\pgfqpoint{6.699711in}{1.784692in}}%
\pgfpathlineto{\pgfqpoint{6.708518in}{1.963133in}}%
\pgfpathlineto{\pgfqpoint{6.717325in}{2.079789in}}%
\pgfpathlineto{\pgfqpoint{6.726132in}{2.031749in}}%
\pgfpathlineto{\pgfqpoint{6.734938in}{2.072930in}}%
\pgfpathlineto{\pgfqpoint{6.743745in}{2.464106in}}%
\pgfpathlineto{\pgfqpoint{6.752552in}{2.477823in}}%
\pgfpathlineto{\pgfqpoint{6.761359in}{2.484682in}}%
\pgfpathlineto{\pgfqpoint{6.770166in}{2.642547in}}%
\pgfpathlineto{\pgfqpoint{6.778972in}{2.745485in}}%
\pgfpathlineto{\pgfqpoint{6.787779in}{2.059213in}}%
\pgfpathlineto{\pgfqpoint{6.796586in}{1.702331in}}%
\pgfpathlineto{\pgfqpoint{6.805393in}{1.482737in}}%
\pgfpathlineto{\pgfqpoint{6.814200in}{1.942529in}}%
\pgfpathlineto{\pgfqpoint{6.823007in}{2.237625in}}%
\pgfpathlineto{\pgfqpoint{6.840620in}{1.427839in}}%
\pgfpathlineto{\pgfqpoint{6.849427in}{1.633715in}}%
\pgfpathlineto{\pgfqpoint{6.858234in}{1.585675in}}%
\pgfpathlineto{\pgfqpoint{6.867041in}{1.517031in}}%
\pgfpathlineto{\pgfqpoint{6.875847in}{1.386658in}}%
\pgfpathlineto{\pgfqpoint{6.884654in}{1.160177in}}%
\pgfpathlineto{\pgfqpoint{6.893461in}{1.050381in}}%
\pgfpathlineto{\pgfqpoint{6.911075in}{1.661150in}}%
\pgfpathlineto{\pgfqpoint{6.919882in}{1.372912in}}%
\pgfpathlineto{\pgfqpoint{6.928688in}{1.949388in}}%
\pgfpathlineto{\pgfqpoint{6.937495in}{1.736653in}}%
\pgfpathlineto{\pgfqpoint{6.946302in}{1.873913in}}%
\pgfpathlineto{\pgfqpoint{6.955109in}{1.805269in}}%
\pgfpathlineto{\pgfqpoint{6.963916in}{1.798410in}}%
\pgfpathlineto{\pgfqpoint{6.972722in}{2.134687in}}%
\pgfpathlineto{\pgfqpoint{6.981529in}{1.921953in}}%
\pgfpathlineto{\pgfqpoint{6.990336in}{2.155292in}}%
\pgfpathlineto{\pgfqpoint{6.999143in}{2.230767in}}%
\pgfpathlineto{\pgfqpoint{7.007950in}{2.464106in}}%
\pgfpathlineto{\pgfqpoint{7.016757in}{2.532722in}}%
\pgfpathlineto{\pgfqpoint{7.025563in}{2.079789in}}%
\pgfpathlineto{\pgfqpoint{7.043177in}{1.654291in}}%
\pgfpathlineto{\pgfqpoint{7.051984in}{1.489596in}}%
\pgfpathlineto{\pgfqpoint{7.060791in}{1.221935in}}%
\pgfpathlineto{\pgfqpoint{7.069597in}{1.118997in}}%
\pgfpathlineto{\pgfqpoint{7.078404in}{1.269974in}}%
\pgfpathlineto{\pgfqpoint{7.087211in}{1.283720in}}%
\pgfpathlineto{\pgfqpoint{7.096018in}{1.221935in}}%
\pgfpathlineto{\pgfqpoint{7.104825in}{1.118997in}}%
\pgfpathlineto{\pgfqpoint{7.113632in}{1.496455in}}%
\pgfpathlineto{\pgfqpoint{7.131245in}{2.024891in}}%
\pgfpathlineto{\pgfqpoint{7.140052in}{1.784692in}}%
\pgfpathlineto{\pgfqpoint{7.148859in}{2.004286in}}%
\pgfpathlineto{\pgfqpoint{7.157666in}{2.182727in}}%
\pgfpathlineto{\pgfqpoint{7.166472in}{1.956246in}}%
\pgfpathlineto{\pgfqpoint{7.175279in}{1.832732in}}%
\pgfpathlineto{\pgfqpoint{7.184086in}{2.237625in}}%
\pgfpathlineto{\pgfqpoint{7.192893in}{2.141546in}}%
\pgfpathlineto{\pgfqpoint{7.201700in}{2.354309in}}%
\pgfpathlineto{\pgfqpoint{7.210507in}{2.374885in}}%
\pgfpathlineto{\pgfqpoint{7.219313in}{1.867054in}}%
\pgfpathlineto{\pgfqpoint{7.228120in}{2.018032in}}%
\pgfpathlineto{\pgfqpoint{7.236927in}{2.114111in}}%
\pgfpathlineto{\pgfqpoint{7.245734in}{2.285665in}}%
\pgfpathlineto{\pgfqpoint{7.254541in}{2.278806in}}%
\pgfpathlineto{\pgfqpoint{7.263347in}{2.120970in}}%
\pgfpathlineto{\pgfqpoint{7.272154in}{1.825873in}}%
\pgfpathlineto{\pgfqpoint{7.280961in}{1.770975in}}%
\pgfpathlineto{\pgfqpoint{7.289768in}{1.915094in}}%
\pgfpathlineto{\pgfqpoint{7.298575in}{1.956246in}}%
\pgfpathlineto{\pgfqpoint{7.307382in}{2.155292in}}%
\pgfpathlineto{\pgfqpoint{7.316188in}{2.285665in}}%
\pgfpathlineto{\pgfqpoint{7.324995in}{2.319987in}}%
\pgfpathlineto{\pgfqpoint{7.333802in}{2.251343in}}%
\pgfpathlineto{\pgfqpoint{7.342609in}{2.395462in}}%
\pgfpathlineto{\pgfqpoint{7.351416in}{2.196445in}}%
\pgfpathlineto{\pgfqpoint{7.360222in}{1.853308in}}%
\pgfpathlineto{\pgfqpoint{7.369029in}{2.155292in}}%
\pgfpathlineto{\pgfqpoint{7.377836in}{1.997427in}}%
\pgfpathlineto{\pgfqpoint{7.386643in}{1.935670in}}%
\pgfpathlineto{\pgfqpoint{7.395450in}{2.107252in}}%
\pgfpathlineto{\pgfqpoint{7.404257in}{2.011173in}}%
\pgfpathlineto{\pgfqpoint{7.413063in}{1.510172in}}%
\pgfpathlineto{\pgfqpoint{7.421870in}{1.482737in}}%
\pgfpathlineto{\pgfqpoint{7.430677in}{1.791551in}}%
\pgfpathlineto{\pgfqpoint{7.439484in}{2.326846in}}%
\pgfpathlineto{\pgfqpoint{7.448291in}{1.997427in}}%
\pgfpathlineto{\pgfqpoint{7.465904in}{1.846450in}}%
\pgfpathlineto{\pgfqpoint{7.474711in}{1.688613in}}%
\pgfpathlineto{\pgfqpoint{7.483518in}{1.949388in}}%
\pgfpathlineto{\pgfqpoint{7.492325in}{2.059213in}}%
\pgfpathlineto{\pgfqpoint{7.501132in}{2.018032in}}%
\pgfpathlineto{\pgfqpoint{7.509938in}{1.819015in}}%
\pgfpathlineto{\pgfqpoint{7.518745in}{1.688613in}}%
\pgfpathlineto{\pgfqpoint{7.527552in}{1.654291in}}%
\pgfpathlineto{\pgfqpoint{7.536359in}{2.258230in}}%
\pgfpathlineto{\pgfqpoint{7.545166in}{2.045467in}}%
\pgfpathlineto{\pgfqpoint{7.553972in}{2.258230in}}%
\pgfpathlineto{\pgfqpoint{7.562779in}{2.333705in}}%
\pgfpathlineto{\pgfqpoint{7.571586in}{2.169009in}}%
\pgfpathlineto{\pgfqpoint{7.580393in}{1.880772in}}%
\pgfpathlineto{\pgfqpoint{7.589200in}{1.517031in}}%
\pgfpathlineto{\pgfqpoint{7.598007in}{1.523918in}}%
\pgfpathlineto{\pgfqpoint{7.606813in}{1.544494in}}%
\pgfpathlineto{\pgfqpoint{7.615620in}{1.455274in}}%
\pgfpathlineto{\pgfqpoint{7.624427in}{1.517031in}}%
\pgfpathlineto{\pgfqpoint{7.633234in}{1.523918in}}%
\pgfpathlineto{\pgfqpoint{7.642041in}{1.400376in}}%
\pgfpathlineto{\pgfqpoint{7.650847in}{1.784692in}}%
\pgfpathlineto{\pgfqpoint{7.659654in}{1.846450in}}%
\pgfpathlineto{\pgfqpoint{7.668461in}{2.072930in}}%
\pgfpathlineto{\pgfqpoint{7.677268in}{1.743512in}}%
\pgfpathlineto{\pgfqpoint{7.686075in}{1.812128in}}%
\pgfpathlineto{\pgfqpoint{7.694882in}{2.052326in}}%
\pgfpathlineto{\pgfqpoint{7.703688in}{2.525863in}}%
\pgfpathlineto{\pgfqpoint{7.712495in}{2.546468in}}%
\pgfpathlineto{\pgfqpoint{7.721302in}{2.052326in}}%
\pgfpathlineto{\pgfqpoint{7.730109in}{1.867054in}}%
\pgfpathlineto{\pgfqpoint{7.738916in}{1.915094in}}%
\pgfpathlineto{\pgfqpoint{7.747722in}{2.100365in}}%
\pgfpathlineto{\pgfqpoint{7.756529in}{2.059213in}}%
\pgfpathlineto{\pgfqpoint{7.765336in}{1.764116in}}%
\pgfpathlineto{\pgfqpoint{7.774143in}{1.825873in}}%
\pgfpathlineto{\pgfqpoint{7.782950in}{2.045467in}}%
\pgfpathlineto{\pgfqpoint{7.791757in}{2.072930in}}%
\pgfpathlineto{\pgfqpoint{7.800563in}{1.839591in}}%
\pgfpathlineto{\pgfqpoint{7.809370in}{1.489596in}}%
\pgfpathlineto{\pgfqpoint{7.818177in}{1.599393in}}%
\pgfpathlineto{\pgfqpoint{7.826984in}{1.668037in}}%
\pgfpathlineto{\pgfqpoint{7.835791in}{1.764116in}}%
\pgfpathlineto{\pgfqpoint{7.844597in}{1.791551in}}%
\pgfpathlineto{\pgfqpoint{7.853404in}{1.585675in}}%
\pgfpathlineto{\pgfqpoint{7.862211in}{1.475878in}}%
\pgfpathlineto{\pgfqpoint{7.871018in}{1.764116in}}%
\pgfpathlineto{\pgfqpoint{7.879825in}{1.832732in}}%
\pgfpathlineto{\pgfqpoint{7.888632in}{1.702331in}}%
\pgfpathlineto{\pgfqpoint{7.897438in}{1.867054in}}%
\pgfpathlineto{\pgfqpoint{7.906245in}{1.585675in}}%
\pgfpathlineto{\pgfqpoint{7.915052in}{1.578816in}}%
\pgfpathlineto{\pgfqpoint{7.923859in}{1.894489in}}%
\pgfpathlineto{\pgfqpoint{7.932666in}{1.839591in}}%
\pgfpathlineto{\pgfqpoint{7.941472in}{1.853308in}}%
\pgfpathlineto{\pgfqpoint{7.950279in}{1.880772in}}%
\pgfpathlineto{\pgfqpoint{7.959086in}{1.791551in}}%
\pgfpathlineto{\pgfqpoint{7.967893in}{1.633715in}}%
\pgfpathlineto{\pgfqpoint{7.976700in}{1.949388in}}%
\pgfpathlineto{\pgfqpoint{7.985507in}{1.832732in}}%
\pgfpathlineto{\pgfqpoint{7.994313in}{1.942529in}}%
\pgfpathlineto{\pgfqpoint{8.003120in}{1.935670in}}%
\pgfpathlineto{\pgfqpoint{8.011927in}{1.860167in}}%
\pgfpathlineto{\pgfqpoint{8.020734in}{1.619997in}}%
\pgfpathlineto{\pgfqpoint{8.029541in}{1.819015in}}%
\pgfpathlineto{\pgfqpoint{8.038347in}{1.956246in}}%
\pgfpathlineto{\pgfqpoint{8.047154in}{1.860167in}}%
\pgfpathlineto{\pgfqpoint{8.055961in}{1.915094in}}%
\pgfpathlineto{\pgfqpoint{8.064768in}{1.743512in}}%
\pgfpathlineto{\pgfqpoint{8.073575in}{1.729794in}}%
\pgfpathlineto{\pgfqpoint{8.082382in}{1.963133in}}%
\pgfpathlineto{\pgfqpoint{8.091188in}{1.928811in}}%
\pgfpathlineto{\pgfqpoint{8.099995in}{1.880772in}}%
\pgfpathlineto{\pgfqpoint{8.108802in}{1.825873in}}%
\pgfpathlineto{\pgfqpoint{8.117609in}{2.066071in}}%
\pgfpathlineto{\pgfqpoint{8.126416in}{2.120970in}}%
\pgfpathlineto{\pgfqpoint{8.135222in}{2.148405in}}%
\pgfpathlineto{\pgfqpoint{8.144029in}{1.928811in}}%
\pgfpathlineto{\pgfqpoint{8.152836in}{1.537636in}}%
\pgfpathlineto{\pgfqpoint{8.161643in}{1.640574in}}%
\pgfpathlineto{\pgfqpoint{8.170450in}{1.805269in}}%
\pgfpathlineto{\pgfqpoint{8.179257in}{1.537636in}}%
\pgfpathlineto{\pgfqpoint{8.188063in}{1.894489in}}%
\pgfpathlineto{\pgfqpoint{8.196870in}{1.613110in}}%
\pgfpathlineto{\pgfqpoint{8.205677in}{1.901348in}}%
\pgfpathlineto{\pgfqpoint{8.214484in}{1.571958in}}%
\pgfpathlineto{\pgfqpoint{8.223291in}{1.695472in}}%
\pgfpathlineto{\pgfqpoint{8.240904in}{1.523918in}}%
\pgfpathlineto{\pgfqpoint{8.249711in}{1.709190in}}%
\pgfpathlineto{\pgfqpoint{8.258518in}{1.681754in}}%
\pgfpathlineto{\pgfqpoint{8.267325in}{2.024891in}}%
\pgfpathlineto{\pgfqpoint{8.276132in}{1.503314in}}%
\pgfpathlineto{\pgfqpoint{8.284938in}{1.407234in}}%
\pgfpathlineto{\pgfqpoint{8.293745in}{1.853308in}}%
\pgfpathlineto{\pgfqpoint{8.311359in}{2.086648in}}%
\pgfpathlineto{\pgfqpoint{8.320166in}{2.100365in}}%
\pgfpathlineto{\pgfqpoint{8.328972in}{1.901348in}}%
\pgfpathlineto{\pgfqpoint{8.337779in}{2.114111in}}%
\pgfpathlineto{\pgfqpoint{8.346586in}{1.853308in}}%
\pgfpathlineto{\pgfqpoint{8.355393in}{2.141546in}}%
\pgfpathlineto{\pgfqpoint{8.364200in}{1.969992in}}%
\pgfpathlineto{\pgfqpoint{8.373007in}{1.400376in}}%
\pgfpathlineto{\pgfqpoint{8.381813in}{1.215076in}}%
\pgfpathlineto{\pgfqpoint{8.390620in}{1.359195in}}%
\pgfpathlineto{\pgfqpoint{8.399427in}{1.462133in}}%
\pgfpathlineto{\pgfqpoint{8.408234in}{1.578816in}}%
\pgfpathlineto{\pgfqpoint{8.417041in}{1.448415in}}%
\pgfpathlineto{\pgfqpoint{8.425847in}{1.448415in}}%
\pgfpathlineto{\pgfqpoint{8.434654in}{1.517031in}}%
\pgfpathlineto{\pgfqpoint{8.443461in}{1.709190in}}%
\pgfpathlineto{\pgfqpoint{8.452268in}{1.633715in}}%
\pgfpathlineto{\pgfqpoint{8.461075in}{1.949388in}}%
\pgfpathlineto{\pgfqpoint{8.469882in}{2.141546in}}%
\pgfpathlineto{\pgfqpoint{8.478688in}{1.846450in}}%
\pgfpathlineto{\pgfqpoint{8.487495in}{1.777834in}}%
\pgfpathlineto{\pgfqpoint{8.496302in}{1.784692in}}%
\pgfpathlineto{\pgfqpoint{8.505109in}{1.949388in}}%
\pgfpathlineto{\pgfqpoint{8.513916in}{1.784692in}}%
\pgfpathlineto{\pgfqpoint{8.522722in}{1.736653in}}%
\pgfpathlineto{\pgfqpoint{8.531529in}{1.668037in}}%
\pgfpathlineto{\pgfqpoint{8.540336in}{1.846450in}}%
\pgfpathlineto{\pgfqpoint{8.549143in}{1.695472in}}%
\pgfpathlineto{\pgfqpoint{8.557950in}{1.585675in}}%
\pgfpathlineto{\pgfqpoint{8.566757in}{1.839591in}}%
\pgfpathlineto{\pgfqpoint{8.575563in}{1.942529in}}%
\pgfpathlineto{\pgfqpoint{8.584370in}{1.969992in}}%
\pgfpathlineto{\pgfqpoint{8.593177in}{2.031749in}}%
\pgfpathlineto{\pgfqpoint{8.601984in}{2.045467in}}%
\pgfpathlineto{\pgfqpoint{8.610791in}{2.004286in}}%
\pgfpathlineto{\pgfqpoint{8.619597in}{1.956246in}}%
\pgfpathlineto{\pgfqpoint{8.628404in}{2.203303in}}%
\pgfpathlineto{\pgfqpoint{8.637211in}{1.867054in}}%
\pgfpathlineto{\pgfqpoint{8.646018in}{1.777834in}}%
\pgfpathlineto{\pgfqpoint{8.654825in}{1.846450in}}%
\pgfpathlineto{\pgfqpoint{8.663632in}{2.059213in}}%
\pgfpathlineto{\pgfqpoint{8.672438in}{1.722935in}}%
\pgfpathlineto{\pgfqpoint{8.681245in}{1.777834in}}%
\pgfpathlineto{\pgfqpoint{8.690052in}{2.175868in}}%
\pgfpathlineto{\pgfqpoint{8.698859in}{2.292524in}}%
\pgfpathlineto{\pgfqpoint{8.707666in}{2.361168in}}%
\pgfpathlineto{\pgfqpoint{8.716472in}{2.217049in}}%
\pgfpathlineto{\pgfqpoint{8.725279in}{2.635660in}}%
\pgfpathlineto{\pgfqpoint{8.734086in}{2.107252in}}%
\pgfpathlineto{\pgfqpoint{8.742893in}{2.203303in}}%
\pgfpathlineto{\pgfqpoint{8.751700in}{2.519004in}}%
\pgfpathlineto{\pgfqpoint{8.760507in}{2.567044in}}%
\pgfpathlineto{\pgfqpoint{8.769313in}{2.669982in}}%
\pgfpathlineto{\pgfqpoint{8.778120in}{2.285665in}}%
\pgfpathlineto{\pgfqpoint{8.786927in}{2.395462in}}%
\pgfpathlineto{\pgfqpoint{8.795734in}{2.594507in}}%
\pgfpathlineto{\pgfqpoint{8.804541in}{2.875858in}}%
\pgfpathlineto{\pgfqpoint{8.813347in}{2.319987in}}%
\pgfpathlineto{\pgfqpoint{8.822154in}{1.949388in}}%
\pgfpathlineto{\pgfqpoint{8.830961in}{2.381744in}}%
\pgfpathlineto{\pgfqpoint{8.839768in}{1.860167in}}%
\pgfpathlineto{\pgfqpoint{8.848575in}{1.956246in}}%
\pgfpathlineto{\pgfqpoint{8.857382in}{2.169009in}}%
\pgfpathlineto{\pgfqpoint{8.866188in}{2.134687in}}%
\pgfpathlineto{\pgfqpoint{8.874995in}{1.819015in}}%
\pgfpathlineto{\pgfqpoint{8.883802in}{1.901348in}}%
\pgfpathlineto{\pgfqpoint{8.892609in}{1.688613in}}%
\pgfpathlineto{\pgfqpoint{8.901416in}{1.901348in}}%
\pgfpathlineto{\pgfqpoint{8.910222in}{1.674896in}}%
\pgfpathlineto{\pgfqpoint{8.919029in}{1.736653in}}%
\pgfpathlineto{\pgfqpoint{8.927836in}{1.736653in}}%
\pgfpathlineto{\pgfqpoint{8.936643in}{1.654291in}}%
\pgfpathlineto{\pgfqpoint{8.945450in}{1.770975in}}%
\pgfpathlineto{\pgfqpoint{8.954257in}{2.052326in}}%
\pgfpathlineto{\pgfqpoint{8.963063in}{1.695472in}}%
\pgfpathlineto{\pgfqpoint{8.971870in}{1.702331in}}%
\pgfpathlineto{\pgfqpoint{8.980677in}{1.613110in}}%
\pgfpathlineto{\pgfqpoint{8.989484in}{1.867054in}}%
\pgfpathlineto{\pgfqpoint{8.998291in}{1.853308in}}%
\pgfpathlineto{\pgfqpoint{9.015904in}{2.395462in}}%
\pgfpathlineto{\pgfqpoint{9.024711in}{2.196445in}}%
\pgfpathlineto{\pgfqpoint{9.033518in}{2.120970in}}%
\pgfpathlineto{\pgfqpoint{9.042325in}{2.079789in}}%
\pgfpathlineto{\pgfqpoint{9.051132in}{1.983710in}}%
\pgfpathlineto{\pgfqpoint{9.059938in}{1.777834in}}%
\pgfpathlineto{\pgfqpoint{9.068745in}{2.031749in}}%
\pgfpathlineto{\pgfqpoint{9.077552in}{2.326846in}}%
\pgfpathlineto{\pgfqpoint{9.086359in}{2.512146in}}%
\pgfpathlineto{\pgfqpoint{9.095166in}{2.409207in}}%
\pgfpathlineto{\pgfqpoint{9.103972in}{2.285665in}}%
\pgfpathlineto{\pgfqpoint{9.112779in}{1.921953in}}%
\pgfpathlineto{\pgfqpoint{9.121586in}{1.805269in}}%
\pgfpathlineto{\pgfqpoint{9.130393in}{2.093507in}}%
\pgfpathlineto{\pgfqpoint{9.139200in}{1.736653in}}%
\pgfpathlineto{\pgfqpoint{9.148007in}{1.969992in}}%
\pgfpathlineto{\pgfqpoint{9.156813in}{2.004286in}}%
\pgfpathlineto{\pgfqpoint{9.165620in}{2.031749in}}%
\pgfpathlineto{\pgfqpoint{9.174427in}{2.395462in}}%
\pgfpathlineto{\pgfqpoint{9.183234in}{2.704304in}}%
\pgfpathlineto{\pgfqpoint{9.192041in}{2.278806in}}%
\pgfpathlineto{\pgfqpoint{9.200847in}{2.519004in}}%
\pgfpathlineto{\pgfqpoint{9.209654in}{2.333705in}}%
\pgfpathlineto{\pgfqpoint{9.218461in}{2.306269in}}%
\pgfpathlineto{\pgfqpoint{9.227268in}{2.210190in}}%
\pgfpathlineto{\pgfqpoint{9.244882in}{2.052326in}}%
\pgfpathlineto{\pgfqpoint{9.262495in}{1.791551in}}%
\pgfpathlineto{\pgfqpoint{9.271302in}{1.716076in}}%
\pgfpathlineto{\pgfqpoint{9.280109in}{2.018032in}}%
\pgfpathlineto{\pgfqpoint{9.288916in}{1.963133in}}%
\pgfpathlineto{\pgfqpoint{9.297722in}{1.956246in}}%
\pgfpathlineto{\pgfqpoint{9.306529in}{1.503314in}}%
\pgfpathlineto{\pgfqpoint{9.315336in}{1.688613in}}%
\pgfpathlineto{\pgfqpoint{9.324143in}{1.379799in}}%
\pgfpathlineto{\pgfqpoint{9.332950in}{1.832732in}}%
\pgfpathlineto{\pgfqpoint{9.341757in}{1.915094in}}%
\pgfpathlineto{\pgfqpoint{9.350563in}{1.633715in}}%
\pgfpathlineto{\pgfqpoint{9.359370in}{1.798410in}}%
\pgfpathlineto{\pgfqpoint{9.368177in}{1.784692in}}%
\pgfpathlineto{\pgfqpoint{9.376984in}{1.798410in}}%
\pgfpathlineto{\pgfqpoint{9.385791in}{1.228822in}}%
\pgfpathlineto{\pgfqpoint{9.394597in}{1.256257in}}%
\pgfpathlineto{\pgfqpoint{9.403404in}{1.215076in}}%
\pgfpathlineto{\pgfqpoint{9.412211in}{1.517031in}}%
\pgfpathlineto{\pgfqpoint{9.421018in}{1.468992in}}%
\pgfpathlineto{\pgfqpoint{9.429825in}{1.764116in}}%
\pgfpathlineto{\pgfqpoint{9.438632in}{1.668037in}}%
\pgfpathlineto{\pgfqpoint{9.447438in}{1.819015in}}%
\pgfpathlineto{\pgfqpoint{9.456245in}{1.633715in}}%
\pgfpathlineto{\pgfqpoint{9.465052in}{1.681754in}}%
\pgfpathlineto{\pgfqpoint{9.473859in}{1.798410in}}%
\pgfpathlineto{\pgfqpoint{9.482666in}{1.489596in}}%
\pgfpathlineto{\pgfqpoint{9.491472in}{1.318014in}}%
\pgfpathlineto{\pgfqpoint{9.500279in}{1.640574in}}%
\pgfpathlineto{\pgfqpoint{9.509086in}{1.661150in}}%
\pgfpathlineto{\pgfqpoint{9.517893in}{2.319987in}}%
\pgfpathlineto{\pgfqpoint{9.526700in}{1.976851in}}%
\pgfpathlineto{\pgfqpoint{9.535507in}{1.805269in}}%
\pgfpathlineto{\pgfqpoint{9.544313in}{1.935670in}}%
\pgfpathlineto{\pgfqpoint{9.553120in}{1.805269in}}%
\pgfpathlineto{\pgfqpoint{9.561927in}{1.846450in}}%
\pgfpathlineto{\pgfqpoint{9.570734in}{2.134687in}}%
\pgfpathlineto{\pgfqpoint{9.579541in}{2.107252in}}%
\pgfpathlineto{\pgfqpoint{9.588347in}{2.271947in}}%
\pgfpathlineto{\pgfqpoint{9.597154in}{2.004286in}}%
\pgfpathlineto{\pgfqpoint{9.605961in}{1.873913in}}%
\pgfpathlineto{\pgfqpoint{9.614768in}{1.901348in}}%
\pgfpathlineto{\pgfqpoint{9.623575in}{1.956246in}}%
\pgfpathlineto{\pgfqpoint{9.632382in}{1.709190in}}%
\pgfpathlineto{\pgfqpoint{9.641188in}{1.544494in}}%
\pgfpathlineto{\pgfqpoint{9.649995in}{1.860167in}}%
\pgfpathlineto{\pgfqpoint{9.658802in}{2.066071in}}%
\pgfpathlineto{\pgfqpoint{9.667609in}{1.613110in}}%
\pgfpathlineto{\pgfqpoint{9.676416in}{1.819015in}}%
\pgfpathlineto{\pgfqpoint{9.685222in}{1.839591in}}%
\pgfpathlineto{\pgfqpoint{9.694029in}{2.024891in}}%
\pgfpathlineto{\pgfqpoint{9.702836in}{1.976851in}}%
\pgfpathlineto{\pgfqpoint{9.711643in}{1.894489in}}%
\pgfpathlineto{\pgfqpoint{9.720450in}{2.072930in}}%
\pgfpathlineto{\pgfqpoint{9.729257in}{2.223908in}}%
\pgfpathlineto{\pgfqpoint{9.738063in}{2.093507in}}%
\pgfpathlineto{\pgfqpoint{9.746870in}{2.450388in}}%
\pgfpathlineto{\pgfqpoint{9.755677in}{2.285665in}}%
\pgfpathlineto{\pgfqpoint{9.764484in}{2.374885in}}%
\pgfpathlineto{\pgfqpoint{9.773291in}{2.162151in}}%
\pgfpathlineto{\pgfqpoint{9.782097in}{1.825873in}}%
\pgfpathlineto{\pgfqpoint{9.790904in}{1.983710in}}%
\pgfpathlineto{\pgfqpoint{9.799711in}{2.470965in}}%
\pgfpathlineto{\pgfqpoint{9.808518in}{2.519004in}}%
\pgfpathlineto{\pgfqpoint{9.817325in}{2.114111in}}%
\pgfpathlineto{\pgfqpoint{9.826132in}{2.086648in}}%
\pgfpathlineto{\pgfqpoint{9.834938in}{2.319987in}}%
\pgfpathlineto{\pgfqpoint{9.843745in}{2.429784in}}%
\pgfpathlineto{\pgfqpoint{9.852552in}{2.354309in}}%
\pgfpathlineto{\pgfqpoint{9.861359in}{2.217049in}}%
\pgfpathlineto{\pgfqpoint{9.878972in}{2.244484in}}%
\pgfpathlineto{\pgfqpoint{9.887779in}{2.059213in}}%
\pgfpathlineto{\pgfqpoint{9.896586in}{2.196445in}}%
\pgfpathlineto{\pgfqpoint{9.905393in}{2.114111in}}%
\pgfpathlineto{\pgfqpoint{9.914200in}{2.230767in}}%
\pgfpathlineto{\pgfqpoint{9.923007in}{1.983710in}}%
\pgfpathlineto{\pgfqpoint{9.931813in}{2.059213in}}%
\pgfpathlineto{\pgfqpoint{9.940620in}{1.901348in}}%
\pgfpathlineto{\pgfqpoint{9.949427in}{2.114111in}}%
\pgfpathlineto{\pgfqpoint{9.949427in}{2.114111in}}%
\pgfusepath{stroke}%
\end{pgfscope}%
\begin{pgfscope}%
\pgfpathrectangle{\pgfqpoint{0.702268in}{0.521603in}}{\pgfqpoint{9.687500in}{4.235000in}}%
\pgfusepath{clip}%
\pgfsetrectcap%
\pgfsetroundjoin%
\pgfsetlinewidth{0.501875pt}%
\definecolor{currentstroke}{rgb}{0.501961,0.501961,0.501961}%
\pgfsetstrokecolor{currentstroke}%
\pgfsetstrokeopacity{0.250000}%
\pgfsetdash{}{0pt}%
\pgfpathmoveto{\pgfqpoint{1.142609in}{4.550386in}}%
\pgfpathlineto{\pgfqpoint{1.151416in}{3.486655in}}%
\pgfpathlineto{\pgfqpoint{1.160222in}{2.903321in}}%
\pgfpathlineto{\pgfqpoint{1.169029in}{2.450388in}}%
\pgfpathlineto{\pgfqpoint{1.177836in}{2.189586in}}%
\pgfpathlineto{\pgfqpoint{1.186643in}{2.141546in}}%
\pgfpathlineto{\pgfqpoint{1.195450in}{1.867054in}}%
\pgfpathlineto{\pgfqpoint{1.204257in}{1.990569in}}%
\pgfpathlineto{\pgfqpoint{1.213063in}{1.825873in}}%
\pgfpathlineto{\pgfqpoint{1.221870in}{2.155292in}}%
\pgfpathlineto{\pgfqpoint{1.230677in}{2.169009in}}%
\pgfpathlineto{\pgfqpoint{1.239484in}{1.997427in}}%
\pgfpathlineto{\pgfqpoint{1.248291in}{1.908207in}}%
\pgfpathlineto{\pgfqpoint{1.257097in}{1.942529in}}%
\pgfpathlineto{\pgfqpoint{1.265904in}{2.059213in}}%
\pgfpathlineto{\pgfqpoint{1.274711in}{1.997427in}}%
\pgfpathlineto{\pgfqpoint{1.283518in}{2.100365in}}%
\pgfpathlineto{\pgfqpoint{1.292325in}{1.963133in}}%
\pgfpathlineto{\pgfqpoint{1.301132in}{1.921953in}}%
\pgfpathlineto{\pgfqpoint{1.309938in}{2.271947in}}%
\pgfpathlineto{\pgfqpoint{1.318745in}{2.169009in}}%
\pgfpathlineto{\pgfqpoint{1.327552in}{2.052326in}}%
\pgfpathlineto{\pgfqpoint{1.336359in}{1.825873in}}%
\pgfpathlineto{\pgfqpoint{1.345166in}{1.702331in}}%
\pgfpathlineto{\pgfqpoint{1.353972in}{1.681754in}}%
\pgfpathlineto{\pgfqpoint{1.362779in}{1.880772in}}%
\pgfpathlineto{\pgfqpoint{1.371586in}{2.155292in}}%
\pgfpathlineto{\pgfqpoint{1.380393in}{2.127829in}}%
\pgfpathlineto{\pgfqpoint{1.389200in}{1.901348in}}%
\pgfpathlineto{\pgfqpoint{1.398007in}{1.867054in}}%
\pgfpathlineto{\pgfqpoint{1.406813in}{1.812128in}}%
\pgfpathlineto{\pgfqpoint{1.415620in}{1.805269in}}%
\pgfpathlineto{\pgfqpoint{1.424427in}{1.331760in}}%
\pgfpathlineto{\pgfqpoint{1.433234in}{1.338618in}}%
\pgfpathlineto{\pgfqpoint{1.442041in}{1.263115in}}%
\pgfpathlineto{\pgfqpoint{1.450847in}{1.379799in}}%
\pgfpathlineto{\pgfqpoint{1.459654in}{1.530777in}}%
\pgfpathlineto{\pgfqpoint{1.468461in}{1.722935in}}%
\pgfpathlineto{\pgfqpoint{1.477268in}{1.578816in}}%
\pgfpathlineto{\pgfqpoint{1.486075in}{1.921953in}}%
\pgfpathlineto{\pgfqpoint{1.494882in}{2.086648in}}%
\pgfpathlineto{\pgfqpoint{1.503688in}{1.976851in}}%
\pgfpathlineto{\pgfqpoint{1.512495in}{1.894489in}}%
\pgfpathlineto{\pgfqpoint{1.521302in}{1.935670in}}%
\pgfpathlineto{\pgfqpoint{1.530109in}{2.141546in}}%
\pgfpathlineto{\pgfqpoint{1.547722in}{2.340563in}}%
\pgfpathlineto{\pgfqpoint{1.556529in}{1.716076in}}%
\pgfpathlineto{\pgfqpoint{1.565336in}{1.674896in}}%
\pgfpathlineto{\pgfqpoint{1.574143in}{1.867054in}}%
\pgfpathlineto{\pgfqpoint{1.582950in}{1.489596in}}%
\pgfpathlineto{\pgfqpoint{1.591757in}{1.633715in}}%
\pgfpathlineto{\pgfqpoint{1.600563in}{1.661150in}}%
\pgfpathlineto{\pgfqpoint{1.609370in}{2.052326in}}%
\pgfpathlineto{\pgfqpoint{1.618177in}{1.722935in}}%
\pgfpathlineto{\pgfqpoint{1.626984in}{1.674896in}}%
\pgfpathlineto{\pgfqpoint{1.635791in}{1.743512in}}%
\pgfpathlineto{\pgfqpoint{1.644597in}{2.052326in}}%
\pgfpathlineto{\pgfqpoint{1.662211in}{1.853308in}}%
\pgfpathlineto{\pgfqpoint{1.671018in}{1.956246in}}%
\pgfpathlineto{\pgfqpoint{1.679825in}{1.578816in}}%
\pgfpathlineto{\pgfqpoint{1.688632in}{1.990569in}}%
\pgfpathlineto{\pgfqpoint{1.697438in}{1.963133in}}%
\pgfpathlineto{\pgfqpoint{1.706245in}{1.942529in}}%
\pgfpathlineto{\pgfqpoint{1.715052in}{1.606252in}}%
\pgfpathlineto{\pgfqpoint{1.723859in}{1.901348in}}%
\pgfpathlineto{\pgfqpoint{1.732666in}{1.784692in}}%
\pgfpathlineto{\pgfqpoint{1.741472in}{2.251343in}}%
\pgfpathlineto{\pgfqpoint{1.750279in}{2.251343in}}%
\pgfpathlineto{\pgfqpoint{1.759086in}{2.107252in}}%
\pgfpathlineto{\pgfqpoint{1.767893in}{2.347422in}}%
\pgfpathlineto{\pgfqpoint{1.776700in}{2.100365in}}%
\pgfpathlineto{\pgfqpoint{1.785507in}{2.045467in}}%
\pgfpathlineto{\pgfqpoint{1.794313in}{1.777834in}}%
\pgfpathlineto{\pgfqpoint{1.803120in}{1.414093in}}%
\pgfpathlineto{\pgfqpoint{1.811927in}{1.263115in}}%
\pgfpathlineto{\pgfqpoint{1.820734in}{1.764116in}}%
\pgfpathlineto{\pgfqpoint{1.829541in}{1.894489in}}%
\pgfpathlineto{\pgfqpoint{1.838347in}{2.059213in}}%
\pgfpathlineto{\pgfqpoint{1.847154in}{1.928811in}}%
\pgfpathlineto{\pgfqpoint{1.855961in}{1.860167in}}%
\pgfpathlineto{\pgfqpoint{1.864768in}{1.770975in}}%
\pgfpathlineto{\pgfqpoint{1.873575in}{2.066071in}}%
\pgfpathlineto{\pgfqpoint{1.891188in}{1.681754in}}%
\pgfpathlineto{\pgfqpoint{1.899995in}{1.578816in}}%
\pgfpathlineto{\pgfqpoint{1.908802in}{1.956246in}}%
\pgfpathlineto{\pgfqpoint{1.917609in}{1.571958in}}%
\pgfpathlineto{\pgfqpoint{1.926416in}{1.400376in}}%
\pgfpathlineto{\pgfqpoint{1.935222in}{1.304296in}}%
\pgfpathlineto{\pgfqpoint{1.944029in}{1.455274in}}%
\pgfpathlineto{\pgfqpoint{1.952836in}{1.091561in}}%
\pgfpathlineto{\pgfqpoint{1.961643in}{1.475878in}}%
\pgfpathlineto{\pgfqpoint{1.970450in}{1.935670in}}%
\pgfpathlineto{\pgfqpoint{1.979257in}{1.558212in}}%
\pgfpathlineto{\pgfqpoint{1.988063in}{1.921953in}}%
\pgfpathlineto{\pgfqpoint{1.996870in}{1.613110in}}%
\pgfpathlineto{\pgfqpoint{2.005677in}{1.716076in}}%
\pgfpathlineto{\pgfqpoint{2.014484in}{1.661150in}}%
\pgfpathlineto{\pgfqpoint{2.023291in}{1.503314in}}%
\pgfpathlineto{\pgfqpoint{2.032097in}{1.613110in}}%
\pgfpathlineto{\pgfqpoint{2.040904in}{1.935670in}}%
\pgfpathlineto{\pgfqpoint{2.049711in}{1.695472in}}%
\pgfpathlineto{\pgfqpoint{2.058518in}{2.072930in}}%
\pgfpathlineto{\pgfqpoint{2.076132in}{1.503314in}}%
\pgfpathlineto{\pgfqpoint{2.084938in}{1.523918in}}%
\pgfpathlineto{\pgfqpoint{2.093745in}{1.757229in}}%
\pgfpathlineto{\pgfqpoint{2.102552in}{1.894489in}}%
\pgfpathlineto{\pgfqpoint{2.111359in}{1.565071in}}%
\pgfpathlineto{\pgfqpoint{2.128972in}{1.949388in}}%
\pgfpathlineto{\pgfqpoint{2.137779in}{2.059213in}}%
\pgfpathlineto{\pgfqpoint{2.146586in}{1.709190in}}%
\pgfpathlineto{\pgfqpoint{2.155393in}{1.571958in}}%
\pgfpathlineto{\pgfqpoint{2.164200in}{1.462133in}}%
\pgfpathlineto{\pgfqpoint{2.173007in}{1.427839in}}%
\pgfpathlineto{\pgfqpoint{2.181813in}{1.654291in}}%
\pgfpathlineto{\pgfqpoint{2.190620in}{1.668037in}}%
\pgfpathlineto{\pgfqpoint{2.199427in}{1.674896in}}%
\pgfpathlineto{\pgfqpoint{2.208234in}{1.455274in}}%
\pgfpathlineto{\pgfqpoint{2.217041in}{2.004286in}}%
\pgfpathlineto{\pgfqpoint{2.225847in}{1.908207in}}%
\pgfpathlineto{\pgfqpoint{2.234654in}{2.271947in}}%
\pgfpathlineto{\pgfqpoint{2.243461in}{1.860167in}}%
\pgfpathlineto{\pgfqpoint{2.252268in}{2.004286in}}%
\pgfpathlineto{\pgfqpoint{2.261075in}{1.949388in}}%
\pgfpathlineto{\pgfqpoint{2.269882in}{1.736653in}}%
\pgfpathlineto{\pgfqpoint{2.278688in}{1.942529in}}%
\pgfpathlineto{\pgfqpoint{2.287495in}{1.784692in}}%
\pgfpathlineto{\pgfqpoint{2.296302in}{1.805269in}}%
\pgfpathlineto{\pgfqpoint{2.305109in}{1.784692in}}%
\pgfpathlineto{\pgfqpoint{2.313916in}{2.018032in}}%
\pgfpathlineto{\pgfqpoint{2.322722in}{1.825873in}}%
\pgfpathlineto{\pgfqpoint{2.331529in}{1.990569in}}%
\pgfpathlineto{\pgfqpoint{2.340336in}{2.210190in}}%
\pgfpathlineto{\pgfqpoint{2.349143in}{2.155292in}}%
\pgfpathlineto{\pgfqpoint{2.357950in}{2.031749in}}%
\pgfpathlineto{\pgfqpoint{2.366757in}{2.011173in}}%
\pgfpathlineto{\pgfqpoint{2.375563in}{1.976851in}}%
\pgfpathlineto{\pgfqpoint{2.384370in}{1.832732in}}%
\pgfpathlineto{\pgfqpoint{2.393177in}{1.510172in}}%
\pgfpathlineto{\pgfqpoint{2.401984in}{1.503314in}}%
\pgfpathlineto{\pgfqpoint{2.410791in}{1.770975in}}%
\pgfpathlineto{\pgfqpoint{2.419597in}{1.921953in}}%
\pgfpathlineto{\pgfqpoint{2.428404in}{1.750370in}}%
\pgfpathlineto{\pgfqpoint{2.437211in}{1.812128in}}%
\pgfpathlineto{\pgfqpoint{2.446018in}{1.709190in}}%
\pgfpathlineto{\pgfqpoint{2.454825in}{1.908207in}}%
\pgfpathlineto{\pgfqpoint{2.463632in}{1.976851in}}%
\pgfpathlineto{\pgfqpoint{2.472438in}{1.956246in}}%
\pgfpathlineto{\pgfqpoint{2.490052in}{1.757229in}}%
\pgfpathlineto{\pgfqpoint{2.498859in}{1.558212in}}%
\pgfpathlineto{\pgfqpoint{2.507666in}{1.427839in}}%
\pgfpathlineto{\pgfqpoint{2.516472in}{1.393517in}}%
\pgfpathlineto{\pgfqpoint{2.534086in}{1.599393in}}%
\pgfpathlineto{\pgfqpoint{2.542893in}{1.462133in}}%
\pgfpathlineto{\pgfqpoint{2.551700in}{1.407234in}}%
\pgfpathlineto{\pgfqpoint{2.560507in}{1.578816in}}%
\pgfpathlineto{\pgfqpoint{2.569313in}{1.613110in}}%
\pgfpathlineto{\pgfqpoint{2.578120in}{1.441556in}}%
\pgfpathlineto{\pgfqpoint{2.586927in}{1.777834in}}%
\pgfpathlineto{\pgfqpoint{2.595734in}{2.011173in}}%
\pgfpathlineto{\pgfqpoint{2.604541in}{1.935670in}}%
\pgfpathlineto{\pgfqpoint{2.613347in}{1.873913in}}%
\pgfpathlineto{\pgfqpoint{2.622154in}{1.880772in}}%
\pgfpathlineto{\pgfqpoint{2.630961in}{1.674896in}}%
\pgfpathlineto{\pgfqpoint{2.639768in}{1.496455in}}%
\pgfpathlineto{\pgfqpoint{2.648575in}{1.462133in}}%
\pgfpathlineto{\pgfqpoint{2.657382in}{1.585675in}}%
\pgfpathlineto{\pgfqpoint{2.666188in}{1.427839in}}%
\pgfpathlineto{\pgfqpoint{2.674995in}{1.880772in}}%
\pgfpathlineto{\pgfqpoint{2.683802in}{2.189586in}}%
\pgfpathlineto{\pgfqpoint{2.692609in}{2.162151in}}%
\pgfpathlineto{\pgfqpoint{2.701416in}{2.107252in}}%
\pgfpathlineto{\pgfqpoint{2.710222in}{2.203303in}}%
\pgfpathlineto{\pgfqpoint{2.719029in}{2.265089in}}%
\pgfpathlineto{\pgfqpoint{2.727836in}{2.292524in}}%
\pgfpathlineto{\pgfqpoint{2.736643in}{2.457247in}}%
\pgfpathlineto{\pgfqpoint{2.754257in}{2.031749in}}%
\pgfpathlineto{\pgfqpoint{2.763063in}{1.935670in}}%
\pgfpathlineto{\pgfqpoint{2.771870in}{1.901348in}}%
\pgfpathlineto{\pgfqpoint{2.780677in}{1.976851in}}%
\pgfpathlineto{\pgfqpoint{2.789484in}{1.873913in}}%
\pgfpathlineto{\pgfqpoint{2.798291in}{1.880772in}}%
\pgfpathlineto{\pgfqpoint{2.807097in}{2.004286in}}%
\pgfpathlineto{\pgfqpoint{2.815904in}{2.169009in}}%
\pgfpathlineto{\pgfqpoint{2.824711in}{2.031749in}}%
\pgfpathlineto{\pgfqpoint{2.833518in}{2.100365in}}%
\pgfpathlineto{\pgfqpoint{2.842325in}{2.519004in}}%
\pgfpathlineto{\pgfqpoint{2.851132in}{2.525863in}}%
\pgfpathlineto{\pgfqpoint{2.859938in}{2.361168in}}%
\pgfpathlineto{\pgfqpoint{2.868745in}{2.560185in}}%
\pgfpathlineto{\pgfqpoint{2.877552in}{2.251343in}}%
\pgfpathlineto{\pgfqpoint{2.886359in}{1.743512in}}%
\pgfpathlineto{\pgfqpoint{2.895166in}{1.819015in}}%
\pgfpathlineto{\pgfqpoint{2.903972in}{1.915094in}}%
\pgfpathlineto{\pgfqpoint{2.912779in}{1.674896in}}%
\pgfpathlineto{\pgfqpoint{2.921586in}{1.860167in}}%
\pgfpathlineto{\pgfqpoint{2.930393in}{1.846450in}}%
\pgfpathlineto{\pgfqpoint{2.939200in}{1.784692in}}%
\pgfpathlineto{\pgfqpoint{2.948007in}{1.956246in}}%
\pgfpathlineto{\pgfqpoint{2.956813in}{2.347422in}}%
\pgfpathlineto{\pgfqpoint{2.965620in}{2.498428in}}%
\pgfpathlineto{\pgfqpoint{2.974427in}{2.402349in}}%
\pgfpathlineto{\pgfqpoint{2.983234in}{2.093507in}}%
\pgfpathlineto{\pgfqpoint{2.992041in}{1.839591in}}%
\pgfpathlineto{\pgfqpoint{3.000847in}{1.681754in}}%
\pgfpathlineto{\pgfqpoint{3.009654in}{1.729794in}}%
\pgfpathlineto{\pgfqpoint{3.018461in}{1.743512in}}%
\pgfpathlineto{\pgfqpoint{3.027268in}{1.894489in}}%
\pgfpathlineto{\pgfqpoint{3.036075in}{1.887630in}}%
\pgfpathlineto{\pgfqpoint{3.044882in}{1.798410in}}%
\pgfpathlineto{\pgfqpoint{3.053688in}{1.544494in}}%
\pgfpathlineto{\pgfqpoint{3.062495in}{1.386658in}}%
\pgfpathlineto{\pgfqpoint{3.071302in}{1.668037in}}%
\pgfpathlineto{\pgfqpoint{3.080109in}{1.647432in}}%
\pgfpathlineto{\pgfqpoint{3.088916in}{1.530777in}}%
\pgfpathlineto{\pgfqpoint{3.097722in}{1.304296in}}%
\pgfpathlineto{\pgfqpoint{3.106529in}{1.297438in}}%
\pgfpathlineto{\pgfqpoint{3.115336in}{1.722935in}}%
\pgfpathlineto{\pgfqpoint{3.124143in}{1.420952in}}%
\pgfpathlineto{\pgfqpoint{3.132950in}{1.400376in}}%
\pgfpathlineto{\pgfqpoint{3.141757in}{1.468992in}}%
\pgfpathlineto{\pgfqpoint{3.150563in}{1.983710in}}%
\pgfpathlineto{\pgfqpoint{3.159370in}{1.688613in}}%
\pgfpathlineto{\pgfqpoint{3.168177in}{1.880772in}}%
\pgfpathlineto{\pgfqpoint{3.176984in}{2.134687in}}%
\pgfpathlineto{\pgfqpoint{3.185791in}{1.901348in}}%
\pgfpathlineto{\pgfqpoint{3.194597in}{2.038608in}}%
\pgfpathlineto{\pgfqpoint{3.203404in}{2.024891in}}%
\pgfpathlineto{\pgfqpoint{3.212211in}{1.668037in}}%
\pgfpathlineto{\pgfqpoint{3.221018in}{1.764116in}}%
\pgfpathlineto{\pgfqpoint{3.229825in}{2.066071in}}%
\pgfpathlineto{\pgfqpoint{3.247438in}{2.038608in}}%
\pgfpathlineto{\pgfqpoint{3.256245in}{1.976851in}}%
\pgfpathlineto{\pgfqpoint{3.265052in}{1.819015in}}%
\pgfpathlineto{\pgfqpoint{3.273859in}{2.120970in}}%
\pgfpathlineto{\pgfqpoint{3.282666in}{2.340563in}}%
\pgfpathlineto{\pgfqpoint{3.291472in}{1.853308in}}%
\pgfpathlineto{\pgfqpoint{3.300279in}{1.928811in}}%
\pgfpathlineto{\pgfqpoint{3.309086in}{1.846450in}}%
\pgfpathlineto{\pgfqpoint{3.317893in}{1.969992in}}%
\pgfpathlineto{\pgfqpoint{3.326700in}{2.024891in}}%
\pgfpathlineto{\pgfqpoint{3.335507in}{2.107252in}}%
\pgfpathlineto{\pgfqpoint{3.344313in}{2.100365in}}%
\pgfpathlineto{\pgfqpoint{3.361927in}{2.642547in}}%
\pgfpathlineto{\pgfqpoint{3.370734in}{2.217049in}}%
\pgfpathlineto{\pgfqpoint{3.379541in}{1.997427in}}%
\pgfpathlineto{\pgfqpoint{3.388347in}{1.949388in}}%
\pgfpathlineto{\pgfqpoint{3.397154in}{2.066071in}}%
\pgfpathlineto{\pgfqpoint{3.405961in}{2.285665in}}%
\pgfpathlineto{\pgfqpoint{3.414768in}{2.141546in}}%
\pgfpathlineto{\pgfqpoint{3.423575in}{2.018032in}}%
\pgfpathlineto{\pgfqpoint{3.441188in}{1.647432in}}%
\pgfpathlineto{\pgfqpoint{3.449995in}{1.228822in}}%
\pgfpathlineto{\pgfqpoint{3.458802in}{1.372912in}}%
\pgfpathlineto{\pgfqpoint{3.467609in}{1.743512in}}%
\pgfpathlineto{\pgfqpoint{3.476416in}{1.887630in}}%
\pgfpathlineto{\pgfqpoint{3.485222in}{2.217049in}}%
\pgfpathlineto{\pgfqpoint{3.494029in}{2.093507in}}%
\pgfpathlineto{\pgfqpoint{3.502836in}{1.908207in}}%
\pgfpathlineto{\pgfqpoint{3.511643in}{1.921953in}}%
\pgfpathlineto{\pgfqpoint{3.520450in}{1.530777in}}%
\pgfpathlineto{\pgfqpoint{3.529257in}{1.619997in}}%
\pgfpathlineto{\pgfqpoint{3.538063in}{1.722935in}}%
\pgfpathlineto{\pgfqpoint{3.546870in}{2.107252in}}%
\pgfpathlineto{\pgfqpoint{3.555677in}{2.333705in}}%
\pgfpathlineto{\pgfqpoint{3.564484in}{2.237625in}}%
\pgfpathlineto{\pgfqpoint{3.573291in}{2.368027in}}%
\pgfpathlineto{\pgfqpoint{3.582097in}{2.464106in}}%
\pgfpathlineto{\pgfqpoint{3.590904in}{2.361168in}}%
\pgfpathlineto{\pgfqpoint{3.599711in}{2.100365in}}%
\pgfpathlineto{\pgfqpoint{3.608518in}{1.956246in}}%
\pgfpathlineto{\pgfqpoint{3.617325in}{2.031749in}}%
\pgfpathlineto{\pgfqpoint{3.626132in}{1.928811in}}%
\pgfpathlineto{\pgfqpoint{3.634938in}{1.894489in}}%
\pgfpathlineto{\pgfqpoint{3.643745in}{1.887630in}}%
\pgfpathlineto{\pgfqpoint{3.652552in}{1.901348in}}%
\pgfpathlineto{\pgfqpoint{3.670166in}{1.688613in}}%
\pgfpathlineto{\pgfqpoint{3.678972in}{1.963133in}}%
\pgfpathlineto{\pgfqpoint{3.687779in}{2.416066in}}%
\pgfpathlineto{\pgfqpoint{3.696586in}{2.477823in}}%
\pgfpathlineto{\pgfqpoint{3.705393in}{2.271947in}}%
\pgfpathlineto{\pgfqpoint{3.714200in}{1.722935in}}%
\pgfpathlineto{\pgfqpoint{3.723007in}{1.915094in}}%
\pgfpathlineto{\pgfqpoint{3.731813in}{2.072930in}}%
\pgfpathlineto{\pgfqpoint{3.740620in}{1.743512in}}%
\pgfpathlineto{\pgfqpoint{3.749427in}{1.482737in}}%
\pgfpathlineto{\pgfqpoint{3.758234in}{1.400376in}}%
\pgfpathlineto{\pgfqpoint{3.767041in}{1.633715in}}%
\pgfpathlineto{\pgfqpoint{3.775847in}{1.805269in}}%
\pgfpathlineto{\pgfqpoint{3.784654in}{1.565071in}}%
\pgfpathlineto{\pgfqpoint{3.793461in}{1.832732in}}%
\pgfpathlineto{\pgfqpoint{3.802268in}{1.716076in}}%
\pgfpathlineto{\pgfqpoint{3.811075in}{2.093507in}}%
\pgfpathlineto{\pgfqpoint{3.819882in}{2.107252in}}%
\pgfpathlineto{\pgfqpoint{3.828688in}{2.155292in}}%
\pgfpathlineto{\pgfqpoint{3.837495in}{1.619997in}}%
\pgfpathlineto{\pgfqpoint{3.846302in}{1.517031in}}%
\pgfpathlineto{\pgfqpoint{3.855109in}{1.757229in}}%
\pgfpathlineto{\pgfqpoint{3.863916in}{1.304296in}}%
\pgfpathlineto{\pgfqpoint{3.872722in}{1.400376in}}%
\pgfpathlineto{\pgfqpoint{3.881529in}{1.441556in}}%
\pgfpathlineto{\pgfqpoint{3.890336in}{1.585675in}}%
\pgfpathlineto{\pgfqpoint{3.899143in}{1.784692in}}%
\pgfpathlineto{\pgfqpoint{3.907950in}{2.059213in}}%
\pgfpathlineto{\pgfqpoint{3.916757in}{2.059213in}}%
\pgfpathlineto{\pgfqpoint{3.925563in}{2.278806in}}%
\pgfpathlineto{\pgfqpoint{3.934370in}{2.244484in}}%
\pgfpathlineto{\pgfqpoint{3.943177in}{2.059213in}}%
\pgfpathlineto{\pgfqpoint{3.951984in}{2.059213in}}%
\pgfpathlineto{\pgfqpoint{3.960791in}{2.258230in}}%
\pgfpathlineto{\pgfqpoint{3.969597in}{2.072930in}}%
\pgfpathlineto{\pgfqpoint{3.978404in}{2.196445in}}%
\pgfpathlineto{\pgfqpoint{3.987211in}{2.052326in}}%
\pgfpathlineto{\pgfqpoint{3.996018in}{1.777834in}}%
\pgfpathlineto{\pgfqpoint{4.004825in}{2.072930in}}%
\pgfpathlineto{\pgfqpoint{4.013632in}{1.867054in}}%
\pgfpathlineto{\pgfqpoint{4.022438in}{1.853308in}}%
\pgfpathlineto{\pgfqpoint{4.031245in}{1.654291in}}%
\pgfpathlineto{\pgfqpoint{4.040052in}{1.908207in}}%
\pgfpathlineto{\pgfqpoint{4.048859in}{1.805269in}}%
\pgfpathlineto{\pgfqpoint{4.057666in}{1.956246in}}%
\pgfpathlineto{\pgfqpoint{4.066472in}{2.175868in}}%
\pgfpathlineto{\pgfqpoint{4.075279in}{2.072930in}}%
\pgfpathlineto{\pgfqpoint{4.084086in}{1.860167in}}%
\pgfpathlineto{\pgfqpoint{4.092893in}{1.571958in}}%
\pgfpathlineto{\pgfqpoint{4.101700in}{1.462133in}}%
\pgfpathlineto{\pgfqpoint{4.110507in}{1.558212in}}%
\pgfpathlineto{\pgfqpoint{4.119313in}{2.038608in}}%
\pgfpathlineto{\pgfqpoint{4.128120in}{2.162151in}}%
\pgfpathlineto{\pgfqpoint{4.136927in}{2.045467in}}%
\pgfpathlineto{\pgfqpoint{4.145734in}{1.997427in}}%
\pgfpathlineto{\pgfqpoint{4.154541in}{1.633715in}}%
\pgfpathlineto{\pgfqpoint{4.163347in}{1.462133in}}%
\pgfpathlineto{\pgfqpoint{4.172154in}{1.523918in}}%
\pgfpathlineto{\pgfqpoint{4.180961in}{1.551353in}}%
\pgfpathlineto{\pgfqpoint{4.189768in}{1.805269in}}%
\pgfpathlineto{\pgfqpoint{4.198575in}{1.654291in}}%
\pgfpathlineto{\pgfqpoint{4.216188in}{2.079789in}}%
\pgfpathlineto{\pgfqpoint{4.224995in}{2.155292in}}%
\pgfpathlineto{\pgfqpoint{4.233802in}{2.265089in}}%
\pgfpathlineto{\pgfqpoint{4.242609in}{2.148405in}}%
\pgfpathlineto{\pgfqpoint{4.251416in}{2.114111in}}%
\pgfpathlineto{\pgfqpoint{4.260222in}{1.613110in}}%
\pgfpathlineto{\pgfqpoint{4.269029in}{1.729794in}}%
\pgfpathlineto{\pgfqpoint{4.277836in}{1.386658in}}%
\pgfpathlineto{\pgfqpoint{4.286643in}{1.489596in}}%
\pgfpathlineto{\pgfqpoint{4.295450in}{0.974878in}}%
\pgfpathlineto{\pgfqpoint{4.304257in}{1.297438in}}%
\pgfpathlineto{\pgfqpoint{4.313063in}{1.489596in}}%
\pgfpathlineto{\pgfqpoint{4.321870in}{1.640574in}}%
\pgfpathlineto{\pgfqpoint{4.330677in}{1.565071in}}%
\pgfpathlineto{\pgfqpoint{4.339484in}{1.606252in}}%
\pgfpathlineto{\pgfqpoint{4.348291in}{1.839591in}}%
\pgfpathlineto{\pgfqpoint{4.357097in}{1.846450in}}%
\pgfpathlineto{\pgfqpoint{4.365904in}{1.887630in}}%
\pgfpathlineto{\pgfqpoint{4.374711in}{2.210190in}}%
\pgfpathlineto{\pgfqpoint{4.383518in}{2.258230in}}%
\pgfpathlineto{\pgfqpoint{4.392325in}{2.299383in}}%
\pgfpathlineto{\pgfqpoint{4.401132in}{2.127829in}}%
\pgfpathlineto{\pgfqpoint{4.409938in}{2.141546in}}%
\pgfpathlineto{\pgfqpoint{4.418745in}{2.402349in}}%
\pgfpathlineto{\pgfqpoint{4.427552in}{2.066071in}}%
\pgfpathlineto{\pgfqpoint{4.436359in}{2.093507in}}%
\pgfpathlineto{\pgfqpoint{4.445166in}{2.127829in}}%
\pgfpathlineto{\pgfqpoint{4.453972in}{2.189586in}}%
\pgfpathlineto{\pgfqpoint{4.462779in}{2.004286in}}%
\pgfpathlineto{\pgfqpoint{4.471586in}{1.654291in}}%
\pgfpathlineto{\pgfqpoint{4.480393in}{2.018032in}}%
\pgfpathlineto{\pgfqpoint{4.489200in}{2.299383in}}%
\pgfpathlineto{\pgfqpoint{4.498007in}{2.265089in}}%
\pgfpathlineto{\pgfqpoint{4.506813in}{2.422925in}}%
\pgfpathlineto{\pgfqpoint{4.515620in}{2.532722in}}%
\pgfpathlineto{\pgfqpoint{4.524427in}{2.429784in}}%
\pgfpathlineto{\pgfqpoint{4.533234in}{2.306269in}}%
\pgfpathlineto{\pgfqpoint{4.542041in}{2.395462in}}%
\pgfpathlineto{\pgfqpoint{4.559654in}{2.107252in}}%
\pgfpathlineto{\pgfqpoint{4.568461in}{2.196445in}}%
\pgfpathlineto{\pgfqpoint{4.577268in}{2.230767in}}%
\pgfpathlineto{\pgfqpoint{4.586075in}{2.470965in}}%
\pgfpathlineto{\pgfqpoint{4.594882in}{2.470965in}}%
\pgfpathlineto{\pgfqpoint{4.603688in}{2.381744in}}%
\pgfpathlineto{\pgfqpoint{4.612495in}{2.189586in}}%
\pgfpathlineto{\pgfqpoint{4.621302in}{2.134687in}}%
\pgfpathlineto{\pgfqpoint{4.630109in}{2.127829in}}%
\pgfpathlineto{\pgfqpoint{4.638916in}{1.908207in}}%
\pgfpathlineto{\pgfqpoint{4.647722in}{2.306269in}}%
\pgfpathlineto{\pgfqpoint{4.665336in}{2.443501in}}%
\pgfpathlineto{\pgfqpoint{4.674143in}{2.038608in}}%
\pgfpathlineto{\pgfqpoint{4.682950in}{1.702331in}}%
\pgfpathlineto{\pgfqpoint{4.691757in}{1.695472in}}%
\pgfpathlineto{\pgfqpoint{4.700563in}{1.777834in}}%
\pgfpathlineto{\pgfqpoint{4.709370in}{1.743512in}}%
\pgfpathlineto{\pgfqpoint{4.718177in}{1.400376in}}%
\pgfpathlineto{\pgfqpoint{4.726984in}{1.558212in}}%
\pgfpathlineto{\pgfqpoint{4.735791in}{1.681754in}}%
\pgfpathlineto{\pgfqpoint{4.753404in}{1.201358in}}%
\pgfpathlineto{\pgfqpoint{4.771018in}{1.915094in}}%
\pgfpathlineto{\pgfqpoint{4.779825in}{1.963133in}}%
\pgfpathlineto{\pgfqpoint{4.788632in}{1.729794in}}%
\pgfpathlineto{\pgfqpoint{4.797438in}{1.722935in}}%
\pgfpathlineto{\pgfqpoint{4.806245in}{1.661150in}}%
\pgfpathlineto{\pgfqpoint{4.815052in}{1.688613in}}%
\pgfpathlineto{\pgfqpoint{4.823859in}{1.585675in}}%
\pgfpathlineto{\pgfqpoint{4.832666in}{1.654291in}}%
\pgfpathlineto{\pgfqpoint{4.841472in}{1.757229in}}%
\pgfpathlineto{\pgfqpoint{4.850279in}{1.695472in}}%
\pgfpathlineto{\pgfqpoint{4.859086in}{1.578816in}}%
\pgfpathlineto{\pgfqpoint{4.867893in}{1.517031in}}%
\pgfpathlineto{\pgfqpoint{4.876700in}{1.633715in}}%
\pgfpathlineto{\pgfqpoint{4.894313in}{1.812128in}}%
\pgfpathlineto{\pgfqpoint{4.911927in}{1.633715in}}%
\pgfpathlineto{\pgfqpoint{4.920734in}{1.757229in}}%
\pgfpathlineto{\pgfqpoint{4.929541in}{1.997427in}}%
\pgfpathlineto{\pgfqpoint{4.938347in}{2.271947in}}%
\pgfpathlineto{\pgfqpoint{4.955961in}{1.647432in}}%
\pgfpathlineto{\pgfqpoint{4.964768in}{1.695472in}}%
\pgfpathlineto{\pgfqpoint{4.973575in}{1.812128in}}%
\pgfpathlineto{\pgfqpoint{4.991188in}{1.551353in}}%
\pgfpathlineto{\pgfqpoint{4.999995in}{1.290579in}}%
\pgfpathlineto{\pgfqpoint{5.008802in}{1.434698in}}%
\pgfpathlineto{\pgfqpoint{5.017609in}{1.441556in}}%
\pgfpathlineto{\pgfqpoint{5.026416in}{1.468992in}}%
\pgfpathlineto{\pgfqpoint{5.035222in}{1.530777in}}%
\pgfpathlineto{\pgfqpoint{5.044029in}{1.475878in}}%
\pgfpathlineto{\pgfqpoint{5.052836in}{1.674896in}}%
\pgfpathlineto{\pgfqpoint{5.061643in}{2.120970in}}%
\pgfpathlineto{\pgfqpoint{5.070450in}{1.901348in}}%
\pgfpathlineto{\pgfqpoint{5.079257in}{2.175868in}}%
\pgfpathlineto{\pgfqpoint{5.088063in}{2.265089in}}%
\pgfpathlineto{\pgfqpoint{5.096870in}{2.333705in}}%
\pgfpathlineto{\pgfqpoint{5.105677in}{2.251343in}}%
\pgfpathlineto{\pgfqpoint{5.114484in}{1.956246in}}%
\pgfpathlineto{\pgfqpoint{5.123291in}{1.983710in}}%
\pgfpathlineto{\pgfqpoint{5.132097in}{1.894489in}}%
\pgfpathlineto{\pgfqpoint{5.140904in}{1.983710in}}%
\pgfpathlineto{\pgfqpoint{5.149711in}{1.791551in}}%
\pgfpathlineto{\pgfqpoint{5.158518in}{1.256257in}}%
\pgfpathlineto{\pgfqpoint{5.167325in}{1.530777in}}%
\pgfpathlineto{\pgfqpoint{5.176132in}{1.578816in}}%
\pgfpathlineto{\pgfqpoint{5.184938in}{1.434698in}}%
\pgfpathlineto{\pgfqpoint{5.193745in}{1.571958in}}%
\pgfpathlineto{\pgfqpoint{5.202552in}{1.427839in}}%
\pgfpathlineto{\pgfqpoint{5.211359in}{1.331760in}}%
\pgfpathlineto{\pgfqpoint{5.220166in}{1.455274in}}%
\pgfpathlineto{\pgfqpoint{5.228972in}{1.517031in}}%
\pgfpathlineto{\pgfqpoint{5.237779in}{1.997427in}}%
\pgfpathlineto{\pgfqpoint{5.246586in}{1.928811in}}%
\pgfpathlineto{\pgfqpoint{5.255393in}{2.251343in}}%
\pgfpathlineto{\pgfqpoint{5.264200in}{2.210190in}}%
\pgfpathlineto{\pgfqpoint{5.273007in}{2.072930in}}%
\pgfpathlineto{\pgfqpoint{5.281813in}{1.997427in}}%
\pgfpathlineto{\pgfqpoint{5.290620in}{2.107252in}}%
\pgfpathlineto{\pgfqpoint{5.299427in}{2.313128in}}%
\pgfpathlineto{\pgfqpoint{5.308234in}{2.086648in}}%
\pgfpathlineto{\pgfqpoint{5.317041in}{1.997427in}}%
\pgfpathlineto{\pgfqpoint{5.325847in}{1.770975in}}%
\pgfpathlineto{\pgfqpoint{5.334654in}{1.880772in}}%
\pgfpathlineto{\pgfqpoint{5.343461in}{1.969992in}}%
\pgfpathlineto{\pgfqpoint{5.352268in}{1.777834in}}%
\pgfpathlineto{\pgfqpoint{5.361075in}{1.757229in}}%
\pgfpathlineto{\pgfqpoint{5.369882in}{2.175868in}}%
\pgfpathlineto{\pgfqpoint{5.378688in}{2.018032in}}%
\pgfpathlineto{\pgfqpoint{5.387495in}{2.182727in}}%
\pgfpathlineto{\pgfqpoint{5.396302in}{2.107252in}}%
\pgfpathlineto{\pgfqpoint{5.405109in}{1.935670in}}%
\pgfpathlineto{\pgfqpoint{5.413916in}{2.052326in}}%
\pgfpathlineto{\pgfqpoint{5.422722in}{2.258230in}}%
\pgfpathlineto{\pgfqpoint{5.431529in}{1.716076in}}%
\pgfpathlineto{\pgfqpoint{5.440336in}{1.613110in}}%
\pgfpathlineto{\pgfqpoint{5.449143in}{1.805269in}}%
\pgfpathlineto{\pgfqpoint{5.457950in}{1.928811in}}%
\pgfpathlineto{\pgfqpoint{5.466757in}{1.770975in}}%
\pgfpathlineto{\pgfqpoint{5.484370in}{2.011173in}}%
\pgfpathlineto{\pgfqpoint{5.493177in}{2.196445in}}%
\pgfpathlineto{\pgfqpoint{5.501984in}{2.093507in}}%
\pgfpathlineto{\pgfqpoint{5.510791in}{1.956246in}}%
\pgfpathlineto{\pgfqpoint{5.519597in}{1.928811in}}%
\pgfpathlineto{\pgfqpoint{5.528404in}{1.894489in}}%
\pgfpathlineto{\pgfqpoint{5.537211in}{1.901348in}}%
\pgfpathlineto{\pgfqpoint{5.546018in}{2.052326in}}%
\pgfpathlineto{\pgfqpoint{5.554825in}{1.681754in}}%
\pgfpathlineto{\pgfqpoint{5.563632in}{1.860167in}}%
\pgfpathlineto{\pgfqpoint{5.572438in}{1.867054in}}%
\pgfpathlineto{\pgfqpoint{5.581245in}{2.155292in}}%
\pgfpathlineto{\pgfqpoint{5.590052in}{1.935670in}}%
\pgfpathlineto{\pgfqpoint{5.598859in}{2.141546in}}%
\pgfpathlineto{\pgfqpoint{5.607666in}{1.832732in}}%
\pgfpathlineto{\pgfqpoint{5.616472in}{1.853308in}}%
\pgfpathlineto{\pgfqpoint{5.625279in}{1.709190in}}%
\pgfpathlineto{\pgfqpoint{5.634086in}{1.613110in}}%
\pgfpathlineto{\pgfqpoint{5.642893in}{1.379799in}}%
\pgfpathlineto{\pgfqpoint{5.651700in}{1.565071in}}%
\pgfpathlineto{\pgfqpoint{5.660507in}{1.249398in}}%
\pgfpathlineto{\pgfqpoint{5.669313in}{1.414093in}}%
\pgfpathlineto{\pgfqpoint{5.686927in}{1.812128in}}%
\pgfpathlineto{\pgfqpoint{5.695734in}{1.798410in}}%
\pgfpathlineto{\pgfqpoint{5.704541in}{1.530777in}}%
\pgfpathlineto{\pgfqpoint{5.722154in}{1.750370in}}%
\pgfpathlineto{\pgfqpoint{5.739768in}{1.537636in}}%
\pgfpathlineto{\pgfqpoint{5.748575in}{1.770975in}}%
\pgfpathlineto{\pgfqpoint{5.757382in}{2.107252in}}%
\pgfpathlineto{\pgfqpoint{5.766188in}{2.141546in}}%
\pgfpathlineto{\pgfqpoint{5.774995in}{2.470965in}}%
\pgfpathlineto{\pgfqpoint{5.783802in}{1.976851in}}%
\pgfpathlineto{\pgfqpoint{5.792609in}{2.052326in}}%
\pgfpathlineto{\pgfqpoint{5.810222in}{1.846450in}}%
\pgfpathlineto{\pgfqpoint{5.819029in}{1.372912in}}%
\pgfpathlineto{\pgfqpoint{5.827836in}{1.565071in}}%
\pgfpathlineto{\pgfqpoint{5.836643in}{1.674896in}}%
\pgfpathlineto{\pgfqpoint{5.845450in}{1.764116in}}%
\pgfpathlineto{\pgfqpoint{5.854257in}{1.276861in}}%
\pgfpathlineto{\pgfqpoint{5.863063in}{1.331760in}}%
\pgfpathlineto{\pgfqpoint{5.871870in}{1.366053in}}%
\pgfpathlineto{\pgfqpoint{5.880677in}{1.764116in}}%
\pgfpathlineto{\pgfqpoint{5.889484in}{1.969992in}}%
\pgfpathlineto{\pgfqpoint{5.898291in}{2.210190in}}%
\pgfpathlineto{\pgfqpoint{5.907097in}{2.354309in}}%
\pgfpathlineto{\pgfqpoint{5.915904in}{2.477823in}}%
\pgfpathlineto{\pgfqpoint{5.924711in}{1.990569in}}%
\pgfpathlineto{\pgfqpoint{5.933518in}{1.784692in}}%
\pgfpathlineto{\pgfqpoint{5.942325in}{2.066071in}}%
\pgfpathlineto{\pgfqpoint{5.968745in}{1.276861in}}%
\pgfpathlineto{\pgfqpoint{5.977552in}{1.613110in}}%
\pgfpathlineto{\pgfqpoint{5.986359in}{1.592534in}}%
\pgfpathlineto{\pgfqpoint{5.995166in}{1.455274in}}%
\pgfpathlineto{\pgfqpoint{6.003972in}{1.352336in}}%
\pgfpathlineto{\pgfqpoint{6.012779in}{1.571958in}}%
\pgfpathlineto{\pgfqpoint{6.021586in}{1.921953in}}%
\pgfpathlineto{\pgfqpoint{6.030393in}{2.066071in}}%
\pgfpathlineto{\pgfqpoint{6.039200in}{2.175868in}}%
\pgfpathlineto{\pgfqpoint{6.048007in}{2.319987in}}%
\pgfpathlineto{\pgfqpoint{6.056813in}{2.031749in}}%
\pgfpathlineto{\pgfqpoint{6.065620in}{2.210190in}}%
\pgfpathlineto{\pgfqpoint{6.074427in}{2.059213in}}%
\pgfpathlineto{\pgfqpoint{6.083234in}{2.155292in}}%
\pgfpathlineto{\pgfqpoint{6.092041in}{1.860167in}}%
\pgfpathlineto{\pgfqpoint{6.100847in}{2.093507in}}%
\pgfpathlineto{\pgfqpoint{6.109654in}{2.134687in}}%
\pgfpathlineto{\pgfqpoint{6.118461in}{1.784692in}}%
\pgfpathlineto{\pgfqpoint{6.127268in}{1.750370in}}%
\pgfpathlineto{\pgfqpoint{6.136075in}{1.448415in}}%
\pgfpathlineto{\pgfqpoint{6.144882in}{1.613110in}}%
\pgfpathlineto{\pgfqpoint{6.153688in}{1.331760in}}%
\pgfpathlineto{\pgfqpoint{6.162495in}{1.345477in}}%
\pgfpathlineto{\pgfqpoint{6.171302in}{1.503314in}}%
\pgfpathlineto{\pgfqpoint{6.180109in}{1.867054in}}%
\pgfpathlineto{\pgfqpoint{6.188916in}{1.750370in}}%
\pgfpathlineto{\pgfqpoint{6.197722in}{1.764116in}}%
\pgfpathlineto{\pgfqpoint{6.215336in}{2.093507in}}%
\pgfpathlineto{\pgfqpoint{6.232950in}{1.462133in}}%
\pgfpathlineto{\pgfqpoint{6.241757in}{1.674896in}}%
\pgfpathlineto{\pgfqpoint{6.250563in}{1.613110in}}%
\pgfpathlineto{\pgfqpoint{6.259370in}{1.441556in}}%
\pgfpathlineto{\pgfqpoint{6.268177in}{1.565071in}}%
\pgfpathlineto{\pgfqpoint{6.285791in}{2.052326in}}%
\pgfpathlineto{\pgfqpoint{6.294597in}{1.990569in}}%
\pgfpathlineto{\pgfqpoint{6.303404in}{1.908207in}}%
\pgfpathlineto{\pgfqpoint{6.312211in}{2.278806in}}%
\pgfpathlineto{\pgfqpoint{6.321018in}{2.244484in}}%
\pgfpathlineto{\pgfqpoint{6.329825in}{2.162151in}}%
\pgfpathlineto{\pgfqpoint{6.338632in}{1.805269in}}%
\pgfpathlineto{\pgfqpoint{6.347438in}{1.976851in}}%
\pgfpathlineto{\pgfqpoint{6.356245in}{1.846450in}}%
\pgfpathlineto{\pgfqpoint{6.365052in}{2.189586in}}%
\pgfpathlineto{\pgfqpoint{6.373859in}{2.134687in}}%
\pgfpathlineto{\pgfqpoint{6.382666in}{1.702331in}}%
\pgfpathlineto{\pgfqpoint{6.391472in}{1.722935in}}%
\pgfpathlineto{\pgfqpoint{6.400279in}{2.045467in}}%
\pgfpathlineto{\pgfqpoint{6.409086in}{1.860167in}}%
\pgfpathlineto{\pgfqpoint{6.417893in}{1.716076in}}%
\pgfpathlineto{\pgfqpoint{6.426700in}{1.901348in}}%
\pgfpathlineto{\pgfqpoint{6.435507in}{1.681754in}}%
\pgfpathlineto{\pgfqpoint{6.444313in}{2.175868in}}%
\pgfpathlineto{\pgfqpoint{6.453120in}{1.764116in}}%
\pgfpathlineto{\pgfqpoint{6.461927in}{1.729794in}}%
\pgfpathlineto{\pgfqpoint{6.470734in}{2.169009in}}%
\pgfpathlineto{\pgfqpoint{6.479541in}{2.374885in}}%
\pgfpathlineto{\pgfqpoint{6.488347in}{2.470965in}}%
\pgfpathlineto{\pgfqpoint{6.497154in}{2.086648in}}%
\pgfpathlineto{\pgfqpoint{6.505961in}{1.963133in}}%
\pgfpathlineto{\pgfqpoint{6.514768in}{2.038608in}}%
\pgfpathlineto{\pgfqpoint{6.523575in}{2.127829in}}%
\pgfpathlineto{\pgfqpoint{6.532382in}{2.333705in}}%
\pgfpathlineto{\pgfqpoint{6.541188in}{2.292524in}}%
\pgfpathlineto{\pgfqpoint{6.549995in}{2.299383in}}%
\pgfpathlineto{\pgfqpoint{6.558802in}{1.990569in}}%
\pgfpathlineto{\pgfqpoint{6.567609in}{2.354309in}}%
\pgfpathlineto{\pgfqpoint{6.576416in}{2.169009in}}%
\pgfpathlineto{\pgfqpoint{6.585222in}{1.626856in}}%
\pgfpathlineto{\pgfqpoint{6.594029in}{1.544494in}}%
\pgfpathlineto{\pgfqpoint{6.602836in}{1.805269in}}%
\pgfpathlineto{\pgfqpoint{6.611643in}{1.770975in}}%
\pgfpathlineto{\pgfqpoint{6.620450in}{2.203303in}}%
\pgfpathlineto{\pgfqpoint{6.629257in}{2.368027in}}%
\pgfpathlineto{\pgfqpoint{6.638063in}{2.093507in}}%
\pgfpathlineto{\pgfqpoint{6.646870in}{2.148405in}}%
\pgfpathlineto{\pgfqpoint{6.655677in}{2.038608in}}%
\pgfpathlineto{\pgfqpoint{6.664484in}{2.155292in}}%
\pgfpathlineto{\pgfqpoint{6.673291in}{2.114111in}}%
\pgfpathlineto{\pgfqpoint{6.682097in}{2.093507in}}%
\pgfpathlineto{\pgfqpoint{6.690904in}{2.148405in}}%
\pgfpathlineto{\pgfqpoint{6.699711in}{2.313128in}}%
\pgfpathlineto{\pgfqpoint{6.708518in}{2.210190in}}%
\pgfpathlineto{\pgfqpoint{6.717325in}{1.915094in}}%
\pgfpathlineto{\pgfqpoint{6.726132in}{1.963133in}}%
\pgfpathlineto{\pgfqpoint{6.734938in}{2.210190in}}%
\pgfpathlineto{\pgfqpoint{6.743745in}{2.326846in}}%
\pgfpathlineto{\pgfqpoint{6.752552in}{1.921953in}}%
\pgfpathlineto{\pgfqpoint{6.761359in}{1.963133in}}%
\pgfpathlineto{\pgfqpoint{6.770166in}{2.484682in}}%
\pgfpathlineto{\pgfqpoint{6.778972in}{2.059213in}}%
\pgfpathlineto{\pgfqpoint{6.787779in}{2.313128in}}%
\pgfpathlineto{\pgfqpoint{6.796586in}{2.018032in}}%
\pgfpathlineto{\pgfqpoint{6.805393in}{1.908207in}}%
\pgfpathlineto{\pgfqpoint{6.814200in}{2.059213in}}%
\pgfpathlineto{\pgfqpoint{6.823007in}{2.237625in}}%
\pgfpathlineto{\pgfqpoint{6.831813in}{2.100365in}}%
\pgfpathlineto{\pgfqpoint{6.840620in}{1.908207in}}%
\pgfpathlineto{\pgfqpoint{6.849427in}{1.983710in}}%
\pgfpathlineto{\pgfqpoint{6.858234in}{1.949388in}}%
\pgfpathlineto{\pgfqpoint{6.867041in}{2.052326in}}%
\pgfpathlineto{\pgfqpoint{6.875847in}{2.120970in}}%
\pgfpathlineto{\pgfqpoint{6.884654in}{2.066071in}}%
\pgfpathlineto{\pgfqpoint{6.893461in}{1.949388in}}%
\pgfpathlineto{\pgfqpoint{6.902268in}{2.271947in}}%
\pgfpathlineto{\pgfqpoint{6.911075in}{2.079789in}}%
\pgfpathlineto{\pgfqpoint{6.919882in}{2.066071in}}%
\pgfpathlineto{\pgfqpoint{6.928688in}{2.416066in}}%
\pgfpathlineto{\pgfqpoint{6.937495in}{1.921953in}}%
\pgfpathlineto{\pgfqpoint{6.946302in}{2.402349in}}%
\pgfpathlineto{\pgfqpoint{6.955109in}{2.285665in}}%
\pgfpathlineto{\pgfqpoint{6.963916in}{1.825873in}}%
\pgfpathlineto{\pgfqpoint{6.972722in}{1.963133in}}%
\pgfpathlineto{\pgfqpoint{6.981529in}{1.976851in}}%
\pgfpathlineto{\pgfqpoint{6.990336in}{2.024891in}}%
\pgfpathlineto{\pgfqpoint{6.999143in}{1.990569in}}%
\pgfpathlineto{\pgfqpoint{7.007950in}{2.059213in}}%
\pgfpathlineto{\pgfqpoint{7.016757in}{1.921953in}}%
\pgfpathlineto{\pgfqpoint{7.025563in}{2.031749in}}%
\pgfpathlineto{\pgfqpoint{7.034370in}{2.072930in}}%
\pgfpathlineto{\pgfqpoint{7.043177in}{1.908207in}}%
\pgfpathlineto{\pgfqpoint{7.051984in}{2.217049in}}%
\pgfpathlineto{\pgfqpoint{7.060791in}{2.100365in}}%
\pgfpathlineto{\pgfqpoint{7.069597in}{2.079789in}}%
\pgfpathlineto{\pgfqpoint{7.078404in}{1.729794in}}%
\pgfpathlineto{\pgfqpoint{7.087211in}{1.606252in}}%
\pgfpathlineto{\pgfqpoint{7.096018in}{1.908207in}}%
\pgfpathlineto{\pgfqpoint{7.104825in}{1.949388in}}%
\pgfpathlineto{\pgfqpoint{7.113632in}{2.299383in}}%
\pgfpathlineto{\pgfqpoint{7.122438in}{2.573903in}}%
\pgfpathlineto{\pgfqpoint{7.131245in}{2.100365in}}%
\pgfpathlineto{\pgfqpoint{7.140052in}{2.011173in}}%
\pgfpathlineto{\pgfqpoint{7.148859in}{2.196445in}}%
\pgfpathlineto{\pgfqpoint{7.157666in}{2.333705in}}%
\pgfpathlineto{\pgfqpoint{7.166472in}{2.368027in}}%
\pgfpathlineto{\pgfqpoint{7.175279in}{2.230767in}}%
\pgfpathlineto{\pgfqpoint{7.184086in}{2.608225in}}%
\pgfpathlineto{\pgfqpoint{7.201700in}{2.326846in}}%
\pgfpathlineto{\pgfqpoint{7.210507in}{2.223908in}}%
\pgfpathlineto{\pgfqpoint{7.219313in}{1.976851in}}%
\pgfpathlineto{\pgfqpoint{7.228120in}{2.031749in}}%
\pgfpathlineto{\pgfqpoint{7.236927in}{1.880772in}}%
\pgfpathlineto{\pgfqpoint{7.245734in}{1.757229in}}%
\pgfpathlineto{\pgfqpoint{7.254541in}{1.805269in}}%
\pgfpathlineto{\pgfqpoint{7.263347in}{1.743512in}}%
\pgfpathlineto{\pgfqpoint{7.272154in}{1.784692in}}%
\pgfpathlineto{\pgfqpoint{7.280961in}{1.908207in}}%
\pgfpathlineto{\pgfqpoint{7.289768in}{2.079789in}}%
\pgfpathlineto{\pgfqpoint{7.298575in}{2.210190in}}%
\pgfpathlineto{\pgfqpoint{7.307382in}{1.997427in}}%
\pgfpathlineto{\pgfqpoint{7.316188in}{1.722935in}}%
\pgfpathlineto{\pgfqpoint{7.324995in}{1.716076in}}%
\pgfpathlineto{\pgfqpoint{7.333802in}{1.585675in}}%
\pgfpathlineto{\pgfqpoint{7.342609in}{1.832732in}}%
\pgfpathlineto{\pgfqpoint{7.351416in}{1.942529in}}%
\pgfpathlineto{\pgfqpoint{7.360222in}{1.359195in}}%
\pgfpathlineto{\pgfqpoint{7.369029in}{1.764116in}}%
\pgfpathlineto{\pgfqpoint{7.377836in}{1.757229in}}%
\pgfpathlineto{\pgfqpoint{7.386643in}{1.880772in}}%
\pgfpathlineto{\pgfqpoint{7.395450in}{1.839591in}}%
\pgfpathlineto{\pgfqpoint{7.404257in}{1.990569in}}%
\pgfpathlineto{\pgfqpoint{7.413063in}{2.333705in}}%
\pgfpathlineto{\pgfqpoint{7.421870in}{2.313128in}}%
\pgfpathlineto{\pgfqpoint{7.430677in}{1.969992in}}%
\pgfpathlineto{\pgfqpoint{7.439484in}{1.921953in}}%
\pgfpathlineto{\pgfqpoint{7.448291in}{1.915094in}}%
\pgfpathlineto{\pgfqpoint{7.457097in}{1.654291in}}%
\pgfpathlineto{\pgfqpoint{7.465904in}{1.269974in}}%
\pgfpathlineto{\pgfqpoint{7.474711in}{1.304296in}}%
\pgfpathlineto{\pgfqpoint{7.483518in}{1.592534in}}%
\pgfpathlineto{\pgfqpoint{7.492325in}{1.400376in}}%
\pgfpathlineto{\pgfqpoint{7.501132in}{1.853308in}}%
\pgfpathlineto{\pgfqpoint{7.509938in}{2.059213in}}%
\pgfpathlineto{\pgfqpoint{7.518745in}{1.935670in}}%
\pgfpathlineto{\pgfqpoint{7.536359in}{2.422925in}}%
\pgfpathlineto{\pgfqpoint{7.545166in}{2.251343in}}%
\pgfpathlineto{\pgfqpoint{7.553972in}{1.969992in}}%
\pgfpathlineto{\pgfqpoint{7.562779in}{2.052326in}}%
\pgfpathlineto{\pgfqpoint{7.571586in}{1.681754in}}%
\pgfpathlineto{\pgfqpoint{7.580393in}{1.990569in}}%
\pgfpathlineto{\pgfqpoint{7.589200in}{2.120970in}}%
\pgfpathlineto{\pgfqpoint{7.598007in}{2.196445in}}%
\pgfpathlineto{\pgfqpoint{7.606813in}{2.217049in}}%
\pgfpathlineto{\pgfqpoint{7.615620in}{1.915094in}}%
\pgfpathlineto{\pgfqpoint{7.624427in}{2.127829in}}%
\pgfpathlineto{\pgfqpoint{7.633234in}{1.963133in}}%
\pgfpathlineto{\pgfqpoint{7.642041in}{2.052326in}}%
\pgfpathlineto{\pgfqpoint{7.650847in}{1.757229in}}%
\pgfpathlineto{\pgfqpoint{7.659654in}{1.661150in}}%
\pgfpathlineto{\pgfqpoint{7.668461in}{1.633715in}}%
\pgfpathlineto{\pgfqpoint{7.677268in}{1.613110in}}%
\pgfpathlineto{\pgfqpoint{7.686075in}{1.633715in}}%
\pgfpathlineto{\pgfqpoint{7.694882in}{1.647432in}}%
\pgfpathlineto{\pgfqpoint{7.703688in}{1.750370in}}%
\pgfpathlineto{\pgfqpoint{7.712495in}{1.578816in}}%
\pgfpathlineto{\pgfqpoint{7.721302in}{1.681754in}}%
\pgfpathlineto{\pgfqpoint{7.730109in}{1.578816in}}%
\pgfpathlineto{\pgfqpoint{7.738916in}{1.777834in}}%
\pgfpathlineto{\pgfqpoint{7.747722in}{1.928811in}}%
\pgfpathlineto{\pgfqpoint{7.756529in}{2.175868in}}%
\pgfpathlineto{\pgfqpoint{7.765336in}{2.361168in}}%
\pgfpathlineto{\pgfqpoint{7.774143in}{2.203303in}}%
\pgfpathlineto{\pgfqpoint{7.782950in}{2.525863in}}%
\pgfpathlineto{\pgfqpoint{7.791757in}{2.416066in}}%
\pgfpathlineto{\pgfqpoint{7.809370in}{1.908207in}}%
\pgfpathlineto{\pgfqpoint{7.818177in}{1.551353in}}%
\pgfpathlineto{\pgfqpoint{7.826984in}{1.482737in}}%
\pgfpathlineto{\pgfqpoint{7.835791in}{1.599393in}}%
\pgfpathlineto{\pgfqpoint{7.844597in}{1.770975in}}%
\pgfpathlineto{\pgfqpoint{7.853404in}{1.901348in}}%
\pgfpathlineto{\pgfqpoint{7.862211in}{1.674896in}}%
\pgfpathlineto{\pgfqpoint{7.871018in}{1.976851in}}%
\pgfpathlineto{\pgfqpoint{7.879825in}{1.578816in}}%
\pgfpathlineto{\pgfqpoint{7.888632in}{1.407234in}}%
\pgfpathlineto{\pgfqpoint{7.897438in}{1.503314in}}%
\pgfpathlineto{\pgfqpoint{7.906245in}{1.434698in}}%
\pgfpathlineto{\pgfqpoint{7.915052in}{1.537636in}}%
\pgfpathlineto{\pgfqpoint{7.923859in}{1.716076in}}%
\pgfpathlineto{\pgfqpoint{7.932666in}{1.571958in}}%
\pgfpathlineto{\pgfqpoint{7.941472in}{1.551353in}}%
\pgfpathlineto{\pgfqpoint{7.950279in}{1.935670in}}%
\pgfpathlineto{\pgfqpoint{7.959086in}{1.921953in}}%
\pgfpathlineto{\pgfqpoint{7.967893in}{1.928811in}}%
\pgfpathlineto{\pgfqpoint{7.976700in}{1.942529in}}%
\pgfpathlineto{\pgfqpoint{7.985507in}{2.038608in}}%
\pgfpathlineto{\pgfqpoint{7.994313in}{2.024891in}}%
\pgfpathlineto{\pgfqpoint{8.003120in}{2.217049in}}%
\pgfpathlineto{\pgfqpoint{8.011927in}{2.354309in}}%
\pgfpathlineto{\pgfqpoint{8.029541in}{2.354309in}}%
\pgfpathlineto{\pgfqpoint{8.038347in}{2.210190in}}%
\pgfpathlineto{\pgfqpoint{8.047154in}{2.388603in}}%
\pgfpathlineto{\pgfqpoint{8.055961in}{2.271947in}}%
\pgfpathlineto{\pgfqpoint{8.064768in}{2.402349in}}%
\pgfpathlineto{\pgfqpoint{8.073575in}{2.066071in}}%
\pgfpathlineto{\pgfqpoint{8.082382in}{2.230767in}}%
\pgfpathlineto{\pgfqpoint{8.091188in}{2.347422in}}%
\pgfpathlineto{\pgfqpoint{8.099995in}{2.416066in}}%
\pgfpathlineto{\pgfqpoint{8.108802in}{2.395462in}}%
\pgfpathlineto{\pgfqpoint{8.117609in}{2.278806in}}%
\pgfpathlineto{\pgfqpoint{8.126416in}{1.908207in}}%
\pgfpathlineto{\pgfqpoint{8.135222in}{2.107252in}}%
\pgfpathlineto{\pgfqpoint{8.152836in}{1.565071in}}%
\pgfpathlineto{\pgfqpoint{8.161643in}{2.079789in}}%
\pgfpathlineto{\pgfqpoint{8.170450in}{2.333705in}}%
\pgfpathlineto{\pgfqpoint{8.179257in}{2.175868in}}%
\pgfpathlineto{\pgfqpoint{8.188063in}{1.928811in}}%
\pgfpathlineto{\pgfqpoint{8.196870in}{1.901348in}}%
\pgfpathlineto{\pgfqpoint{8.205677in}{1.791551in}}%
\pgfpathlineto{\pgfqpoint{8.214484in}{2.230767in}}%
\pgfpathlineto{\pgfqpoint{8.223291in}{1.839591in}}%
\pgfpathlineto{\pgfqpoint{8.232097in}{1.668037in}}%
\pgfpathlineto{\pgfqpoint{8.240904in}{1.935670in}}%
\pgfpathlineto{\pgfqpoint{8.249711in}{1.990569in}}%
\pgfpathlineto{\pgfqpoint{8.258518in}{1.963133in}}%
\pgfpathlineto{\pgfqpoint{8.267325in}{1.798410in}}%
\pgfpathlineto{\pgfqpoint{8.276132in}{1.709190in}}%
\pgfpathlineto{\pgfqpoint{8.284938in}{2.107252in}}%
\pgfpathlineto{\pgfqpoint{8.293745in}{2.038608in}}%
\pgfpathlineto{\pgfqpoint{8.302552in}{1.819015in}}%
\pgfpathlineto{\pgfqpoint{8.311359in}{1.716076in}}%
\pgfpathlineto{\pgfqpoint{8.320166in}{2.175868in}}%
\pgfpathlineto{\pgfqpoint{8.328972in}{1.853308in}}%
\pgfpathlineto{\pgfqpoint{8.337779in}{1.681754in}}%
\pgfpathlineto{\pgfqpoint{8.346586in}{1.379799in}}%
\pgfpathlineto{\pgfqpoint{8.364200in}{1.551353in}}%
\pgfpathlineto{\pgfqpoint{8.373007in}{1.784692in}}%
\pgfpathlineto{\pgfqpoint{8.381813in}{1.750370in}}%
\pgfpathlineto{\pgfqpoint{8.390620in}{1.798410in}}%
\pgfpathlineto{\pgfqpoint{8.399427in}{1.770975in}}%
\pgfpathlineto{\pgfqpoint{8.408234in}{1.530777in}}%
\pgfpathlineto{\pgfqpoint{8.417041in}{1.523918in}}%
\pgfpathlineto{\pgfqpoint{8.425847in}{1.784692in}}%
\pgfpathlineto{\pgfqpoint{8.434654in}{1.338618in}}%
\pgfpathlineto{\pgfqpoint{8.443461in}{1.695472in}}%
\pgfpathlineto{\pgfqpoint{8.452268in}{1.496455in}}%
\pgfpathlineto{\pgfqpoint{8.461075in}{1.544494in}}%
\pgfpathlineto{\pgfqpoint{8.469882in}{1.379799in}}%
\pgfpathlineto{\pgfqpoint{8.478688in}{1.695472in}}%
\pgfpathlineto{\pgfqpoint{8.487495in}{2.052326in}}%
\pgfpathlineto{\pgfqpoint{8.496302in}{1.901348in}}%
\pgfpathlineto{\pgfqpoint{8.505109in}{1.695472in}}%
\pgfpathlineto{\pgfqpoint{8.513916in}{1.764116in}}%
\pgfpathlineto{\pgfqpoint{8.522722in}{1.716076in}}%
\pgfpathlineto{\pgfqpoint{8.531529in}{1.798410in}}%
\pgfpathlineto{\pgfqpoint{8.540336in}{1.434698in}}%
\pgfpathlineto{\pgfqpoint{8.549143in}{1.290579in}}%
\pgfpathlineto{\pgfqpoint{8.557950in}{1.839591in}}%
\pgfpathlineto{\pgfqpoint{8.566757in}{1.599393in}}%
\pgfpathlineto{\pgfqpoint{8.575563in}{1.791551in}}%
\pgfpathlineto{\pgfqpoint{8.584370in}{1.544494in}}%
\pgfpathlineto{\pgfqpoint{8.593177in}{1.674896in}}%
\pgfpathlineto{\pgfqpoint{8.601984in}{1.372912in}}%
\pgfpathlineto{\pgfqpoint{8.610791in}{1.420952in}}%
\pgfpathlineto{\pgfqpoint{8.619597in}{1.324873in}}%
\pgfpathlineto{\pgfqpoint{8.628404in}{1.592534in}}%
\pgfpathlineto{\pgfqpoint{8.637211in}{1.633715in}}%
\pgfpathlineto{\pgfqpoint{8.646018in}{1.551353in}}%
\pgfpathlineto{\pgfqpoint{8.654825in}{1.379799in}}%
\pgfpathlineto{\pgfqpoint{8.663632in}{1.462133in}}%
\pgfpathlineto{\pgfqpoint{8.672438in}{1.633715in}}%
\pgfpathlineto{\pgfqpoint{8.681245in}{1.668037in}}%
\pgfpathlineto{\pgfqpoint{8.690052in}{1.825873in}}%
\pgfpathlineto{\pgfqpoint{8.698859in}{2.175868in}}%
\pgfpathlineto{\pgfqpoint{8.707666in}{2.223908in}}%
\pgfpathlineto{\pgfqpoint{8.716472in}{2.230767in}}%
\pgfpathlineto{\pgfqpoint{8.725279in}{2.148405in}}%
\pgfpathlineto{\pgfqpoint{8.734086in}{2.230767in}}%
\pgfpathlineto{\pgfqpoint{8.742893in}{2.470965in}}%
\pgfpathlineto{\pgfqpoint{8.760507in}{2.114111in}}%
\pgfpathlineto{\pgfqpoint{8.769313in}{2.175868in}}%
\pgfpathlineto{\pgfqpoint{8.778120in}{2.093507in}}%
\pgfpathlineto{\pgfqpoint{8.786927in}{1.969992in}}%
\pgfpathlineto{\pgfqpoint{8.795734in}{1.599393in}}%
\pgfpathlineto{\pgfqpoint{8.804541in}{1.503314in}}%
\pgfpathlineto{\pgfqpoint{8.813347in}{1.750370in}}%
\pgfpathlineto{\pgfqpoint{8.822154in}{2.086648in}}%
\pgfpathlineto{\pgfqpoint{8.830961in}{2.038608in}}%
\pgfpathlineto{\pgfqpoint{8.839768in}{2.031749in}}%
\pgfpathlineto{\pgfqpoint{8.848575in}{1.880772in}}%
\pgfpathlineto{\pgfqpoint{8.857382in}{2.079789in}}%
\pgfpathlineto{\pgfqpoint{8.866188in}{1.963133in}}%
\pgfpathlineto{\pgfqpoint{8.874995in}{1.462133in}}%
\pgfpathlineto{\pgfqpoint{8.883802in}{1.558212in}}%
\pgfpathlineto{\pgfqpoint{8.892609in}{1.585675in}}%
\pgfpathlineto{\pgfqpoint{8.901416in}{1.517031in}}%
\pgfpathlineto{\pgfqpoint{8.910222in}{1.812128in}}%
\pgfpathlineto{\pgfqpoint{8.919029in}{1.530777in}}%
\pgfpathlineto{\pgfqpoint{8.927836in}{1.791551in}}%
\pgfpathlineto{\pgfqpoint{8.936643in}{1.647432in}}%
\pgfpathlineto{\pgfqpoint{8.945450in}{1.668037in}}%
\pgfpathlineto{\pgfqpoint{8.954257in}{2.024891in}}%
\pgfpathlineto{\pgfqpoint{8.963063in}{1.983710in}}%
\pgfpathlineto{\pgfqpoint{8.971870in}{1.935670in}}%
\pgfpathlineto{\pgfqpoint{8.980677in}{2.182727in}}%
\pgfpathlineto{\pgfqpoint{8.989484in}{2.107252in}}%
\pgfpathlineto{\pgfqpoint{8.998291in}{2.079789in}}%
\pgfpathlineto{\pgfqpoint{9.007097in}{1.750370in}}%
\pgfpathlineto{\pgfqpoint{9.015904in}{1.880772in}}%
\pgfpathlineto{\pgfqpoint{9.024711in}{1.894489in}}%
\pgfpathlineto{\pgfqpoint{9.033518in}{1.619997in}}%
\pgfpathlineto{\pgfqpoint{9.042325in}{1.482737in}}%
\pgfpathlineto{\pgfqpoint{9.051132in}{1.269974in}}%
\pgfpathlineto{\pgfqpoint{9.059938in}{1.599393in}}%
\pgfpathlineto{\pgfqpoint{9.068745in}{2.361168in}}%
\pgfpathlineto{\pgfqpoint{9.077552in}{2.011173in}}%
\pgfpathlineto{\pgfqpoint{9.095166in}{2.663123in}}%
\pgfpathlineto{\pgfqpoint{9.103972in}{2.601366in}}%
\pgfpathlineto{\pgfqpoint{9.112779in}{3.068016in}}%
\pgfpathlineto{\pgfqpoint{9.121586in}{2.546468in}}%
\pgfpathlineto{\pgfqpoint{9.130393in}{2.409207in}}%
\pgfpathlineto{\pgfqpoint{9.139200in}{2.210190in}}%
\pgfpathlineto{\pgfqpoint{9.148007in}{1.963133in}}%
\pgfpathlineto{\pgfqpoint{9.156813in}{1.928811in}}%
\pgfpathlineto{\pgfqpoint{9.165620in}{2.210190in}}%
\pgfpathlineto{\pgfqpoint{9.174427in}{1.791551in}}%
\pgfpathlineto{\pgfqpoint{9.183234in}{1.551353in}}%
\pgfpathlineto{\pgfqpoint{9.192041in}{1.716076in}}%
\pgfpathlineto{\pgfqpoint{9.200847in}{1.599393in}}%
\pgfpathlineto{\pgfqpoint{9.209654in}{1.743512in}}%
\pgfpathlineto{\pgfqpoint{9.218461in}{1.825873in}}%
\pgfpathlineto{\pgfqpoint{9.227268in}{1.853308in}}%
\pgfpathlineto{\pgfqpoint{9.236075in}{1.867054in}}%
\pgfpathlineto{\pgfqpoint{9.244882in}{1.839591in}}%
\pgfpathlineto{\pgfqpoint{9.262495in}{1.283720in}}%
\pgfpathlineto{\pgfqpoint{9.271302in}{1.640574in}}%
\pgfpathlineto{\pgfqpoint{9.280109in}{1.448415in}}%
\pgfpathlineto{\pgfqpoint{9.297722in}{1.894489in}}%
\pgfpathlineto{\pgfqpoint{9.306529in}{1.846450in}}%
\pgfpathlineto{\pgfqpoint{9.315336in}{2.059213in}}%
\pgfpathlineto{\pgfqpoint{9.324143in}{2.072930in}}%
\pgfpathlineto{\pgfqpoint{9.332950in}{1.928811in}}%
\pgfpathlineto{\pgfqpoint{9.341757in}{1.709190in}}%
\pgfpathlineto{\pgfqpoint{9.350563in}{1.688613in}}%
\pgfpathlineto{\pgfqpoint{9.359370in}{1.606252in}}%
\pgfpathlineto{\pgfqpoint{9.368177in}{1.448415in}}%
\pgfpathlineto{\pgfqpoint{9.376984in}{1.722935in}}%
\pgfpathlineto{\pgfqpoint{9.394597in}{2.457247in}}%
\pgfpathlineto{\pgfqpoint{9.403404in}{2.333705in}}%
\pgfpathlineto{\pgfqpoint{9.412211in}{2.608225in}}%
\pgfpathlineto{\pgfqpoint{9.421018in}{2.189586in}}%
\pgfpathlineto{\pgfqpoint{9.429825in}{2.354309in}}%
\pgfpathlineto{\pgfqpoint{9.438632in}{2.175868in}}%
\pgfpathlineto{\pgfqpoint{9.447438in}{2.299383in}}%
\pgfpathlineto{\pgfqpoint{9.456245in}{2.114111in}}%
\pgfpathlineto{\pgfqpoint{9.465052in}{2.217049in}}%
\pgfpathlineto{\pgfqpoint{9.482666in}{1.592534in}}%
\pgfpathlineto{\pgfqpoint{9.491472in}{1.812128in}}%
\pgfpathlineto{\pgfqpoint{9.500279in}{1.963133in}}%
\pgfpathlineto{\pgfqpoint{9.509086in}{2.182727in}}%
\pgfpathlineto{\pgfqpoint{9.517893in}{1.976851in}}%
\pgfpathlineto{\pgfqpoint{9.535507in}{2.278806in}}%
\pgfpathlineto{\pgfqpoint{9.544313in}{2.429784in}}%
\pgfpathlineto{\pgfqpoint{9.553120in}{2.093507in}}%
\pgfpathlineto{\pgfqpoint{9.561927in}{2.402349in}}%
\pgfpathlineto{\pgfqpoint{9.570734in}{2.354309in}}%
\pgfpathlineto{\pgfqpoint{9.588347in}{1.990569in}}%
\pgfpathlineto{\pgfqpoint{9.597154in}{1.921953in}}%
\pgfpathlineto{\pgfqpoint{9.605961in}{2.066071in}}%
\pgfpathlineto{\pgfqpoint{9.614768in}{2.066071in}}%
\pgfpathlineto{\pgfqpoint{9.623575in}{1.839591in}}%
\pgfpathlineto{\pgfqpoint{9.632382in}{1.915094in}}%
\pgfpathlineto{\pgfqpoint{9.641188in}{2.100365in}}%
\pgfpathlineto{\pgfqpoint{9.649995in}{1.681754in}}%
\pgfpathlineto{\pgfqpoint{9.658802in}{1.661150in}}%
\pgfpathlineto{\pgfqpoint{9.667609in}{1.839591in}}%
\pgfpathlineto{\pgfqpoint{9.676416in}{2.196445in}}%
\pgfpathlineto{\pgfqpoint{9.685222in}{2.086648in}}%
\pgfpathlineto{\pgfqpoint{9.694029in}{2.107252in}}%
\pgfpathlineto{\pgfqpoint{9.702836in}{1.867054in}}%
\pgfpathlineto{\pgfqpoint{9.711643in}{1.743512in}}%
\pgfpathlineto{\pgfqpoint{9.720450in}{1.565071in}}%
\pgfpathlineto{\pgfqpoint{9.729257in}{1.578816in}}%
\pgfpathlineto{\pgfqpoint{9.738063in}{1.462133in}}%
\pgfpathlineto{\pgfqpoint{9.746870in}{1.544494in}}%
\pgfpathlineto{\pgfqpoint{9.755677in}{1.860167in}}%
\pgfpathlineto{\pgfqpoint{9.764484in}{1.441556in}}%
\pgfpathlineto{\pgfqpoint{9.773291in}{1.633715in}}%
\pgfpathlineto{\pgfqpoint{9.782097in}{1.407234in}}%
\pgfpathlineto{\pgfqpoint{9.790904in}{1.688613in}}%
\pgfpathlineto{\pgfqpoint{9.799711in}{1.606252in}}%
\pgfpathlineto{\pgfqpoint{9.808518in}{1.894489in}}%
\pgfpathlineto{\pgfqpoint{9.817325in}{1.729794in}}%
\pgfpathlineto{\pgfqpoint{9.826132in}{2.100365in}}%
\pgfpathlineto{\pgfqpoint{9.834938in}{2.148405in}}%
\pgfpathlineto{\pgfqpoint{9.852552in}{2.292524in}}%
\pgfpathlineto{\pgfqpoint{9.861359in}{1.963133in}}%
\pgfpathlineto{\pgfqpoint{9.870166in}{2.059213in}}%
\pgfpathlineto{\pgfqpoint{9.878972in}{1.963133in}}%
\pgfpathlineto{\pgfqpoint{9.887779in}{2.031749in}}%
\pgfpathlineto{\pgfqpoint{9.896586in}{1.949388in}}%
\pgfpathlineto{\pgfqpoint{9.905393in}{1.997427in}}%
\pgfpathlineto{\pgfqpoint{9.914200in}{2.271947in}}%
\pgfpathlineto{\pgfqpoint{9.923007in}{2.313128in}}%
\pgfpathlineto{\pgfqpoint{9.931813in}{1.894489in}}%
\pgfpathlineto{\pgfqpoint{9.940620in}{1.921953in}}%
\pgfpathlineto{\pgfqpoint{9.949427in}{1.839591in}}%
\pgfpathlineto{\pgfqpoint{9.949427in}{1.839591in}}%
\pgfusepath{stroke}%
\end{pgfscope}%
\begin{pgfscope}%
\pgfpathrectangle{\pgfqpoint{0.702268in}{0.521603in}}{\pgfqpoint{9.687500in}{4.235000in}}%
\pgfusepath{clip}%
\pgfsetrectcap%
\pgfsetroundjoin%
\pgfsetlinewidth{0.501875pt}%
\definecolor{currentstroke}{rgb}{0.501961,0.501961,0.501961}%
\pgfsetstrokecolor{currentstroke}%
\pgfsetstrokeopacity{0.250000}%
\pgfsetdash{}{0pt}%
\pgfpathmoveto{\pgfqpoint{1.142609in}{4.138606in}}%
\pgfpathlineto{\pgfqpoint{1.160222in}{3.150378in}}%
\pgfpathlineto{\pgfqpoint{1.169029in}{2.724880in}}%
\pgfpathlineto{\pgfqpoint{1.177836in}{2.155292in}}%
\pgfpathlineto{\pgfqpoint{1.186643in}{1.722935in}}%
\pgfpathlineto{\pgfqpoint{1.195450in}{1.736653in}}%
\pgfpathlineto{\pgfqpoint{1.204257in}{1.379799in}}%
\pgfpathlineto{\pgfqpoint{1.213063in}{1.709190in}}%
\pgfpathlineto{\pgfqpoint{1.221870in}{1.716076in}}%
\pgfpathlineto{\pgfqpoint{1.230677in}{2.066071in}}%
\pgfpathlineto{\pgfqpoint{1.248291in}{1.805269in}}%
\pgfpathlineto{\pgfqpoint{1.257097in}{1.750370in}}%
\pgfpathlineto{\pgfqpoint{1.265904in}{1.599393in}}%
\pgfpathlineto{\pgfqpoint{1.274711in}{1.544494in}}%
\pgfpathlineto{\pgfqpoint{1.283518in}{1.729794in}}%
\pgfpathlineto{\pgfqpoint{1.292325in}{1.867054in}}%
\pgfpathlineto{\pgfqpoint{1.301132in}{1.640574in}}%
\pgfpathlineto{\pgfqpoint{1.309938in}{1.626856in}}%
\pgfpathlineto{\pgfqpoint{1.318745in}{1.661150in}}%
\pgfpathlineto{\pgfqpoint{1.336359in}{1.544494in}}%
\pgfpathlineto{\pgfqpoint{1.345166in}{1.599393in}}%
\pgfpathlineto{\pgfqpoint{1.353972in}{1.544494in}}%
\pgfpathlineto{\pgfqpoint{1.362779in}{1.956246in}}%
\pgfpathlineto{\pgfqpoint{1.371586in}{1.791551in}}%
\pgfpathlineto{\pgfqpoint{1.380393in}{2.031749in}}%
\pgfpathlineto{\pgfqpoint{1.389200in}{2.539581in}}%
\pgfpathlineto{\pgfqpoint{1.398007in}{2.402349in}}%
\pgfpathlineto{\pgfqpoint{1.406813in}{2.443501in}}%
\pgfpathlineto{\pgfqpoint{1.415620in}{2.271947in}}%
\pgfpathlineto{\pgfqpoint{1.424427in}{2.024891in}}%
\pgfpathlineto{\pgfqpoint{1.433234in}{1.825873in}}%
\pgfpathlineto{\pgfqpoint{1.450847in}{1.592534in}}%
\pgfpathlineto{\pgfqpoint{1.459654in}{1.496455in}}%
\pgfpathlineto{\pgfqpoint{1.468461in}{1.592534in}}%
\pgfpathlineto{\pgfqpoint{1.477268in}{1.290579in}}%
\pgfpathlineto{\pgfqpoint{1.486075in}{1.263115in}}%
\pgfpathlineto{\pgfqpoint{1.494882in}{1.681754in}}%
\pgfpathlineto{\pgfqpoint{1.503688in}{1.565071in}}%
\pgfpathlineto{\pgfqpoint{1.512495in}{1.860167in}}%
\pgfpathlineto{\pgfqpoint{1.521302in}{2.066071in}}%
\pgfpathlineto{\pgfqpoint{1.530109in}{2.114111in}}%
\pgfpathlineto{\pgfqpoint{1.538916in}{2.841564in}}%
\pgfpathlineto{\pgfqpoint{1.547722in}{2.457247in}}%
\pgfpathlineto{\pgfqpoint{1.556529in}{2.223908in}}%
\pgfpathlineto{\pgfqpoint{1.565336in}{1.757229in}}%
\pgfpathlineto{\pgfqpoint{1.574143in}{2.100365in}}%
\pgfpathlineto{\pgfqpoint{1.582950in}{1.942529in}}%
\pgfpathlineto{\pgfqpoint{1.591757in}{1.812128in}}%
\pgfpathlineto{\pgfqpoint{1.600563in}{1.661150in}}%
\pgfpathlineto{\pgfqpoint{1.609370in}{1.942529in}}%
\pgfpathlineto{\pgfqpoint{1.618177in}{1.949388in}}%
\pgfpathlineto{\pgfqpoint{1.626984in}{1.770975in}}%
\pgfpathlineto{\pgfqpoint{1.635791in}{1.867054in}}%
\pgfpathlineto{\pgfqpoint{1.644597in}{1.633715in}}%
\pgfpathlineto{\pgfqpoint{1.653404in}{1.606252in}}%
\pgfpathlineto{\pgfqpoint{1.662211in}{1.571958in}}%
\pgfpathlineto{\pgfqpoint{1.671018in}{1.407234in}}%
\pgfpathlineto{\pgfqpoint{1.679825in}{1.208217in}}%
\pgfpathlineto{\pgfqpoint{1.688632in}{1.468992in}}%
\pgfpathlineto{\pgfqpoint{1.697438in}{1.921953in}}%
\pgfpathlineto{\pgfqpoint{1.706245in}{1.901348in}}%
\pgfpathlineto{\pgfqpoint{1.715052in}{2.052326in}}%
\pgfpathlineto{\pgfqpoint{1.723859in}{1.908207in}}%
\pgfpathlineto{\pgfqpoint{1.732666in}{2.237625in}}%
\pgfpathlineto{\pgfqpoint{1.741472in}{2.416066in}}%
\pgfpathlineto{\pgfqpoint{1.750279in}{1.873913in}}%
\pgfpathlineto{\pgfqpoint{1.759086in}{1.702331in}}%
\pgfpathlineto{\pgfqpoint{1.767893in}{1.825873in}}%
\pgfpathlineto{\pgfqpoint{1.776700in}{1.757229in}}%
\pgfpathlineto{\pgfqpoint{1.785507in}{1.619997in}}%
\pgfpathlineto{\pgfqpoint{1.794313in}{1.832732in}}%
\pgfpathlineto{\pgfqpoint{1.803120in}{1.880772in}}%
\pgfpathlineto{\pgfqpoint{1.811927in}{1.887630in}}%
\pgfpathlineto{\pgfqpoint{1.820734in}{2.004286in}}%
\pgfpathlineto{\pgfqpoint{1.829541in}{1.935670in}}%
\pgfpathlineto{\pgfqpoint{1.838347in}{2.251343in}}%
\pgfpathlineto{\pgfqpoint{1.847154in}{2.093507in}}%
\pgfpathlineto{\pgfqpoint{1.855961in}{2.018032in}}%
\pgfpathlineto{\pgfqpoint{1.864768in}{2.567044in}}%
\pgfpathlineto{\pgfqpoint{1.873575in}{2.594507in}}%
\pgfpathlineto{\pgfqpoint{1.882382in}{2.388603in}}%
\pgfpathlineto{\pgfqpoint{1.891188in}{2.340563in}}%
\pgfpathlineto{\pgfqpoint{1.899995in}{2.361168in}}%
\pgfpathlineto{\pgfqpoint{1.908802in}{2.354309in}}%
\pgfpathlineto{\pgfqpoint{1.917609in}{2.278806in}}%
\pgfpathlineto{\pgfqpoint{1.926416in}{2.189586in}}%
\pgfpathlineto{\pgfqpoint{1.935222in}{2.079789in}}%
\pgfpathlineto{\pgfqpoint{1.944029in}{2.251343in}}%
\pgfpathlineto{\pgfqpoint{1.952836in}{1.908207in}}%
\pgfpathlineto{\pgfqpoint{1.961643in}{1.839591in}}%
\pgfpathlineto{\pgfqpoint{1.970450in}{2.196445in}}%
\pgfpathlineto{\pgfqpoint{1.979257in}{2.368027in}}%
\pgfpathlineto{\pgfqpoint{1.988063in}{2.217049in}}%
\pgfpathlineto{\pgfqpoint{1.996870in}{2.162151in}}%
\pgfpathlineto{\pgfqpoint{2.005677in}{1.928811in}}%
\pgfpathlineto{\pgfqpoint{2.014484in}{1.969992in}}%
\pgfpathlineto{\pgfqpoint{2.023291in}{2.299383in}}%
\pgfpathlineto{\pgfqpoint{2.032097in}{2.038608in}}%
\pgfpathlineto{\pgfqpoint{2.040904in}{1.482737in}}%
\pgfpathlineto{\pgfqpoint{2.049711in}{1.221935in}}%
\pgfpathlineto{\pgfqpoint{2.058518in}{1.530777in}}%
\pgfpathlineto{\pgfqpoint{2.067325in}{1.544494in}}%
\pgfpathlineto{\pgfqpoint{2.076132in}{1.949388in}}%
\pgfpathlineto{\pgfqpoint{2.084938in}{2.066071in}}%
\pgfpathlineto{\pgfqpoint{2.093745in}{1.764116in}}%
\pgfpathlineto{\pgfqpoint{2.120166in}{2.299383in}}%
\pgfpathlineto{\pgfqpoint{2.128972in}{2.299383in}}%
\pgfpathlineto{\pgfqpoint{2.137779in}{2.182727in}}%
\pgfpathlineto{\pgfqpoint{2.146586in}{2.333705in}}%
\pgfpathlineto{\pgfqpoint{2.155393in}{2.422925in}}%
\pgfpathlineto{\pgfqpoint{2.164200in}{1.942529in}}%
\pgfpathlineto{\pgfqpoint{2.173007in}{1.867054in}}%
\pgfpathlineto{\pgfqpoint{2.181813in}{1.846450in}}%
\pgfpathlineto{\pgfqpoint{2.190620in}{1.702331in}}%
\pgfpathlineto{\pgfqpoint{2.199427in}{1.729794in}}%
\pgfpathlineto{\pgfqpoint{2.208234in}{1.736653in}}%
\pgfpathlineto{\pgfqpoint{2.217041in}{1.661150in}}%
\pgfpathlineto{\pgfqpoint{2.225847in}{1.716076in}}%
\pgfpathlineto{\pgfqpoint{2.234654in}{1.819015in}}%
\pgfpathlineto{\pgfqpoint{2.243461in}{1.908207in}}%
\pgfpathlineto{\pgfqpoint{2.252268in}{2.182727in}}%
\pgfpathlineto{\pgfqpoint{2.261075in}{2.299383in}}%
\pgfpathlineto{\pgfqpoint{2.269882in}{2.278806in}}%
\pgfpathlineto{\pgfqpoint{2.278688in}{2.416066in}}%
\pgfpathlineto{\pgfqpoint{2.287495in}{2.361168in}}%
\pgfpathlineto{\pgfqpoint{2.296302in}{2.271947in}}%
\pgfpathlineto{\pgfqpoint{2.305109in}{1.867054in}}%
\pgfpathlineto{\pgfqpoint{2.313916in}{1.688613in}}%
\pgfpathlineto{\pgfqpoint{2.322722in}{1.455274in}}%
\pgfpathlineto{\pgfqpoint{2.331529in}{1.345477in}}%
\pgfpathlineto{\pgfqpoint{2.340336in}{1.592534in}}%
\pgfpathlineto{\pgfqpoint{2.349143in}{2.107252in}}%
\pgfpathlineto{\pgfqpoint{2.357950in}{1.805269in}}%
\pgfpathlineto{\pgfqpoint{2.366757in}{2.217049in}}%
\pgfpathlineto{\pgfqpoint{2.375563in}{2.278806in}}%
\pgfpathlineto{\pgfqpoint{2.384370in}{2.155292in}}%
\pgfpathlineto{\pgfqpoint{2.393177in}{1.812128in}}%
\pgfpathlineto{\pgfqpoint{2.401984in}{1.702331in}}%
\pgfpathlineto{\pgfqpoint{2.410791in}{1.523918in}}%
\pgfpathlineto{\pgfqpoint{2.419597in}{1.695472in}}%
\pgfpathlineto{\pgfqpoint{2.428404in}{1.613110in}}%
\pgfpathlineto{\pgfqpoint{2.437211in}{1.661150in}}%
\pgfpathlineto{\pgfqpoint{2.446018in}{1.798410in}}%
\pgfpathlineto{\pgfqpoint{2.454825in}{1.812128in}}%
\pgfpathlineto{\pgfqpoint{2.463632in}{1.983710in}}%
\pgfpathlineto{\pgfqpoint{2.472438in}{1.894489in}}%
\pgfpathlineto{\pgfqpoint{2.481245in}{1.969992in}}%
\pgfpathlineto{\pgfqpoint{2.490052in}{2.059213in}}%
\pgfpathlineto{\pgfqpoint{2.498859in}{2.127829in}}%
\pgfpathlineto{\pgfqpoint{2.507666in}{2.306269in}}%
\pgfpathlineto{\pgfqpoint{2.516472in}{2.210190in}}%
\pgfpathlineto{\pgfqpoint{2.525279in}{1.997427in}}%
\pgfpathlineto{\pgfqpoint{2.534086in}{1.853308in}}%
\pgfpathlineto{\pgfqpoint{2.542893in}{1.963133in}}%
\pgfpathlineto{\pgfqpoint{2.551700in}{1.839591in}}%
\pgfpathlineto{\pgfqpoint{2.560507in}{2.031749in}}%
\pgfpathlineto{\pgfqpoint{2.569313in}{2.114111in}}%
\pgfpathlineto{\pgfqpoint{2.578120in}{1.805269in}}%
\pgfpathlineto{\pgfqpoint{2.586927in}{2.155292in}}%
\pgfpathlineto{\pgfqpoint{2.595734in}{2.148405in}}%
\pgfpathlineto{\pgfqpoint{2.604541in}{2.223908in}}%
\pgfpathlineto{\pgfqpoint{2.613347in}{2.189586in}}%
\pgfpathlineto{\pgfqpoint{2.622154in}{2.381744in}}%
\pgfpathlineto{\pgfqpoint{2.630961in}{2.251343in}}%
\pgfpathlineto{\pgfqpoint{2.639768in}{2.093507in}}%
\pgfpathlineto{\pgfqpoint{2.648575in}{1.956246in}}%
\pgfpathlineto{\pgfqpoint{2.657382in}{1.983710in}}%
\pgfpathlineto{\pgfqpoint{2.666188in}{2.189586in}}%
\pgfpathlineto{\pgfqpoint{2.674995in}{2.120970in}}%
\pgfpathlineto{\pgfqpoint{2.683802in}{1.709190in}}%
\pgfpathlineto{\pgfqpoint{2.692609in}{1.887630in}}%
\pgfpathlineto{\pgfqpoint{2.701416in}{1.928811in}}%
\pgfpathlineto{\pgfqpoint{2.710222in}{1.750370in}}%
\pgfpathlineto{\pgfqpoint{2.719029in}{1.764116in}}%
\pgfpathlineto{\pgfqpoint{2.727836in}{1.956246in}}%
\pgfpathlineto{\pgfqpoint{2.736643in}{2.018032in}}%
\pgfpathlineto{\pgfqpoint{2.745450in}{1.963133in}}%
\pgfpathlineto{\pgfqpoint{2.754257in}{1.935670in}}%
\pgfpathlineto{\pgfqpoint{2.763063in}{2.038608in}}%
\pgfpathlineto{\pgfqpoint{2.771870in}{1.729794in}}%
\pgfpathlineto{\pgfqpoint{2.780677in}{1.331760in}}%
\pgfpathlineto{\pgfqpoint{2.789484in}{1.736653in}}%
\pgfpathlineto{\pgfqpoint{2.798291in}{1.613110in}}%
\pgfpathlineto{\pgfqpoint{2.807097in}{1.839591in}}%
\pgfpathlineto{\pgfqpoint{2.815904in}{1.949388in}}%
\pgfpathlineto{\pgfqpoint{2.824711in}{2.038608in}}%
\pgfpathlineto{\pgfqpoint{2.833518in}{1.990569in}}%
\pgfpathlineto{\pgfqpoint{2.842325in}{1.956246in}}%
\pgfpathlineto{\pgfqpoint{2.851132in}{1.997427in}}%
\pgfpathlineto{\pgfqpoint{2.859938in}{1.599393in}}%
\pgfpathlineto{\pgfqpoint{2.868745in}{1.496455in}}%
\pgfpathlineto{\pgfqpoint{2.877552in}{1.887630in}}%
\pgfpathlineto{\pgfqpoint{2.886359in}{1.819015in}}%
\pgfpathlineto{\pgfqpoint{2.895166in}{1.908207in}}%
\pgfpathlineto{\pgfqpoint{2.903972in}{2.066071in}}%
\pgfpathlineto{\pgfqpoint{2.912779in}{1.784692in}}%
\pgfpathlineto{\pgfqpoint{2.921586in}{1.825873in}}%
\pgfpathlineto{\pgfqpoint{2.930393in}{1.722935in}}%
\pgfpathlineto{\pgfqpoint{2.939200in}{1.674896in}}%
\pgfpathlineto{\pgfqpoint{2.948007in}{1.921953in}}%
\pgfpathlineto{\pgfqpoint{2.956813in}{1.956246in}}%
\pgfpathlineto{\pgfqpoint{2.965620in}{1.956246in}}%
\pgfpathlineto{\pgfqpoint{2.974427in}{2.024891in}}%
\pgfpathlineto{\pgfqpoint{2.983234in}{1.427839in}}%
\pgfpathlineto{\pgfqpoint{2.992041in}{1.221935in}}%
\pgfpathlineto{\pgfqpoint{3.000847in}{1.427839in}}%
\pgfpathlineto{\pgfqpoint{3.009654in}{1.510172in}}%
\pgfpathlineto{\pgfqpoint{3.018461in}{1.544494in}}%
\pgfpathlineto{\pgfqpoint{3.027268in}{1.791551in}}%
\pgfpathlineto{\pgfqpoint{3.036075in}{1.345477in}}%
\pgfpathlineto{\pgfqpoint{3.044882in}{1.606252in}}%
\pgfpathlineto{\pgfqpoint{3.053688in}{1.956246in}}%
\pgfpathlineto{\pgfqpoint{3.062495in}{1.942529in}}%
\pgfpathlineto{\pgfqpoint{3.080109in}{2.306269in}}%
\pgfpathlineto{\pgfqpoint{3.088916in}{2.464106in}}%
\pgfpathlineto{\pgfqpoint{3.097722in}{2.271947in}}%
\pgfpathlineto{\pgfqpoint{3.106529in}{2.333705in}}%
\pgfpathlineto{\pgfqpoint{3.115336in}{2.464106in}}%
\pgfpathlineto{\pgfqpoint{3.124143in}{2.326846in}}%
\pgfpathlineto{\pgfqpoint{3.132950in}{2.031749in}}%
\pgfpathlineto{\pgfqpoint{3.141757in}{2.361168in}}%
\pgfpathlineto{\pgfqpoint{3.150563in}{2.120970in}}%
\pgfpathlineto{\pgfqpoint{3.159370in}{1.764116in}}%
\pgfpathlineto{\pgfqpoint{3.168177in}{1.846450in}}%
\pgfpathlineto{\pgfqpoint{3.176984in}{1.770975in}}%
\pgfpathlineto{\pgfqpoint{3.185791in}{1.901348in}}%
\pgfpathlineto{\pgfqpoint{3.194597in}{1.599393in}}%
\pgfpathlineto{\pgfqpoint{3.203404in}{1.407234in}}%
\pgfpathlineto{\pgfqpoint{3.212211in}{1.764116in}}%
\pgfpathlineto{\pgfqpoint{3.221018in}{1.619997in}}%
\pgfpathlineto{\pgfqpoint{3.229825in}{1.853308in}}%
\pgfpathlineto{\pgfqpoint{3.238632in}{2.189586in}}%
\pgfpathlineto{\pgfqpoint{3.247438in}{2.203303in}}%
\pgfpathlineto{\pgfqpoint{3.256245in}{2.429784in}}%
\pgfpathlineto{\pgfqpoint{3.265052in}{2.477823in}}%
\pgfpathlineto{\pgfqpoint{3.273859in}{1.990569in}}%
\pgfpathlineto{\pgfqpoint{3.282666in}{2.086648in}}%
\pgfpathlineto{\pgfqpoint{3.291472in}{1.640574in}}%
\pgfpathlineto{\pgfqpoint{3.300279in}{1.668037in}}%
\pgfpathlineto{\pgfqpoint{3.309086in}{1.654291in}}%
\pgfpathlineto{\pgfqpoint{3.317893in}{2.086648in}}%
\pgfpathlineto{\pgfqpoint{3.326700in}{1.983710in}}%
\pgfpathlineto{\pgfqpoint{3.344313in}{1.619997in}}%
\pgfpathlineto{\pgfqpoint{3.361927in}{1.269974in}}%
\pgfpathlineto{\pgfqpoint{3.370734in}{1.372912in}}%
\pgfpathlineto{\pgfqpoint{3.379541in}{1.798410in}}%
\pgfpathlineto{\pgfqpoint{3.388347in}{1.963133in}}%
\pgfpathlineto{\pgfqpoint{3.397154in}{2.031749in}}%
\pgfpathlineto{\pgfqpoint{3.405961in}{1.963133in}}%
\pgfpathlineto{\pgfqpoint{3.414768in}{1.928811in}}%
\pgfpathlineto{\pgfqpoint{3.423575in}{1.544494in}}%
\pgfpathlineto{\pgfqpoint{3.432382in}{1.915094in}}%
\pgfpathlineto{\pgfqpoint{3.441188in}{1.784692in}}%
\pgfpathlineto{\pgfqpoint{3.449995in}{1.537636in}}%
\pgfpathlineto{\pgfqpoint{3.458802in}{1.359195in}}%
\pgfpathlineto{\pgfqpoint{3.467609in}{1.420952in}}%
\pgfpathlineto{\pgfqpoint{3.476416in}{1.242539in}}%
\pgfpathlineto{\pgfqpoint{3.485222in}{1.894489in}}%
\pgfpathlineto{\pgfqpoint{3.502836in}{2.141546in}}%
\pgfpathlineto{\pgfqpoint{3.511643in}{2.388603in}}%
\pgfpathlineto{\pgfqpoint{3.520450in}{2.162151in}}%
\pgfpathlineto{\pgfqpoint{3.529257in}{2.079789in}}%
\pgfpathlineto{\pgfqpoint{3.538063in}{1.736653in}}%
\pgfpathlineto{\pgfqpoint{3.546870in}{1.448415in}}%
\pgfpathlineto{\pgfqpoint{3.555677in}{1.880772in}}%
\pgfpathlineto{\pgfqpoint{3.564484in}{2.100365in}}%
\pgfpathlineto{\pgfqpoint{3.573291in}{2.251343in}}%
\pgfpathlineto{\pgfqpoint{3.582097in}{2.169009in}}%
\pgfpathlineto{\pgfqpoint{3.590904in}{1.990569in}}%
\pgfpathlineto{\pgfqpoint{3.599711in}{2.223908in}}%
\pgfpathlineto{\pgfqpoint{3.608518in}{2.532722in}}%
\pgfpathlineto{\pgfqpoint{3.617325in}{2.306269in}}%
\pgfpathlineto{\pgfqpoint{3.626132in}{2.560185in}}%
\pgfpathlineto{\pgfqpoint{3.634938in}{2.484682in}}%
\pgfpathlineto{\pgfqpoint{3.643745in}{2.663123in}}%
\pgfpathlineto{\pgfqpoint{3.661359in}{2.189586in}}%
\pgfpathlineto{\pgfqpoint{3.670166in}{2.319987in}}%
\pgfpathlineto{\pgfqpoint{3.678972in}{2.127829in}}%
\pgfpathlineto{\pgfqpoint{3.687779in}{2.018032in}}%
\pgfpathlineto{\pgfqpoint{3.696586in}{2.018032in}}%
\pgfpathlineto{\pgfqpoint{3.705393in}{2.306269in}}%
\pgfpathlineto{\pgfqpoint{3.714200in}{2.093507in}}%
\pgfpathlineto{\pgfqpoint{3.723007in}{2.196445in}}%
\pgfpathlineto{\pgfqpoint{3.731813in}{1.921953in}}%
\pgfpathlineto{\pgfqpoint{3.740620in}{2.107252in}}%
\pgfpathlineto{\pgfqpoint{3.749427in}{1.942529in}}%
\pgfpathlineto{\pgfqpoint{3.758234in}{1.530777in}}%
\pgfpathlineto{\pgfqpoint{3.767041in}{1.839591in}}%
\pgfpathlineto{\pgfqpoint{3.775847in}{1.722935in}}%
\pgfpathlineto{\pgfqpoint{3.784654in}{1.880772in}}%
\pgfpathlineto{\pgfqpoint{3.793461in}{1.894489in}}%
\pgfpathlineto{\pgfqpoint{3.802268in}{1.372912in}}%
\pgfpathlineto{\pgfqpoint{3.811075in}{1.606252in}}%
\pgfpathlineto{\pgfqpoint{3.819882in}{1.537636in}}%
\pgfpathlineto{\pgfqpoint{3.828688in}{1.626856in}}%
\pgfpathlineto{\pgfqpoint{3.837495in}{1.825873in}}%
\pgfpathlineto{\pgfqpoint{3.846302in}{1.633715in}}%
\pgfpathlineto{\pgfqpoint{3.863916in}{2.402349in}}%
\pgfpathlineto{\pgfqpoint{3.872722in}{2.024891in}}%
\pgfpathlineto{\pgfqpoint{3.881529in}{2.347422in}}%
\pgfpathlineto{\pgfqpoint{3.890336in}{2.169009in}}%
\pgfpathlineto{\pgfqpoint{3.899143in}{1.681754in}}%
\pgfpathlineto{\pgfqpoint{3.907950in}{1.674896in}}%
\pgfpathlineto{\pgfqpoint{3.916757in}{1.544494in}}%
\pgfpathlineto{\pgfqpoint{3.925563in}{1.695472in}}%
\pgfpathlineto{\pgfqpoint{3.934370in}{1.983710in}}%
\pgfpathlineto{\pgfqpoint{3.943177in}{1.894489in}}%
\pgfpathlineto{\pgfqpoint{3.951984in}{1.770975in}}%
\pgfpathlineto{\pgfqpoint{3.960791in}{1.887630in}}%
\pgfpathlineto{\pgfqpoint{3.969597in}{1.695472in}}%
\pgfpathlineto{\pgfqpoint{3.978404in}{1.750370in}}%
\pgfpathlineto{\pgfqpoint{3.987211in}{1.496455in}}%
\pgfpathlineto{\pgfqpoint{3.996018in}{1.578816in}}%
\pgfpathlineto{\pgfqpoint{4.004825in}{1.736653in}}%
\pgfpathlineto{\pgfqpoint{4.013632in}{1.969992in}}%
\pgfpathlineto{\pgfqpoint{4.022438in}{2.271947in}}%
\pgfpathlineto{\pgfqpoint{4.031245in}{1.997427in}}%
\pgfpathlineto{\pgfqpoint{4.040052in}{2.031749in}}%
\pgfpathlineto{\pgfqpoint{4.048859in}{1.901348in}}%
\pgfpathlineto{\pgfqpoint{4.057666in}{2.052326in}}%
\pgfpathlineto{\pgfqpoint{4.066472in}{1.915094in}}%
\pgfpathlineto{\pgfqpoint{4.075279in}{2.086648in}}%
\pgfpathlineto{\pgfqpoint{4.084086in}{2.011173in}}%
\pgfpathlineto{\pgfqpoint{4.092893in}{2.059213in}}%
\pgfpathlineto{\pgfqpoint{4.101700in}{1.750370in}}%
\pgfpathlineto{\pgfqpoint{4.110507in}{1.791551in}}%
\pgfpathlineto{\pgfqpoint{4.119313in}{1.887630in}}%
\pgfpathlineto{\pgfqpoint{4.128120in}{1.956246in}}%
\pgfpathlineto{\pgfqpoint{4.136927in}{2.004286in}}%
\pgfpathlineto{\pgfqpoint{4.145734in}{1.853308in}}%
\pgfpathlineto{\pgfqpoint{4.154541in}{1.969992in}}%
\pgfpathlineto{\pgfqpoint{4.163347in}{1.915094in}}%
\pgfpathlineto{\pgfqpoint{4.172154in}{1.908207in}}%
\pgfpathlineto{\pgfqpoint{4.180961in}{1.997427in}}%
\pgfpathlineto{\pgfqpoint{4.189768in}{1.908207in}}%
\pgfpathlineto{\pgfqpoint{4.198575in}{1.901348in}}%
\pgfpathlineto{\pgfqpoint{4.207382in}{1.764116in}}%
\pgfpathlineto{\pgfqpoint{4.216188in}{2.059213in}}%
\pgfpathlineto{\pgfqpoint{4.224995in}{2.313128in}}%
\pgfpathlineto{\pgfqpoint{4.233802in}{2.114111in}}%
\pgfpathlineto{\pgfqpoint{4.242609in}{1.873913in}}%
\pgfpathlineto{\pgfqpoint{4.251416in}{2.031749in}}%
\pgfpathlineto{\pgfqpoint{4.260222in}{2.292524in}}%
\pgfpathlineto{\pgfqpoint{4.269029in}{2.443501in}}%
\pgfpathlineto{\pgfqpoint{4.277836in}{2.738598in}}%
\pgfpathlineto{\pgfqpoint{4.286643in}{2.251343in}}%
\pgfpathlineto{\pgfqpoint{4.295450in}{2.258230in}}%
\pgfpathlineto{\pgfqpoint{4.304257in}{2.100365in}}%
\pgfpathlineto{\pgfqpoint{4.313063in}{2.230767in}}%
\pgfpathlineto{\pgfqpoint{4.321870in}{2.230767in}}%
\pgfpathlineto{\pgfqpoint{4.330677in}{2.285665in}}%
\pgfpathlineto{\pgfqpoint{4.339484in}{2.100365in}}%
\pgfpathlineto{\pgfqpoint{4.348291in}{2.120970in}}%
\pgfpathlineto{\pgfqpoint{4.357097in}{2.031749in}}%
\pgfpathlineto{\pgfqpoint{4.365904in}{2.086648in}}%
\pgfpathlineto{\pgfqpoint{4.374711in}{1.921953in}}%
\pgfpathlineto{\pgfqpoint{4.383518in}{1.846450in}}%
\pgfpathlineto{\pgfqpoint{4.392325in}{2.141546in}}%
\pgfpathlineto{\pgfqpoint{4.401132in}{2.127829in}}%
\pgfpathlineto{\pgfqpoint{4.409938in}{1.908207in}}%
\pgfpathlineto{\pgfqpoint{4.418745in}{1.908207in}}%
\pgfpathlineto{\pgfqpoint{4.427552in}{1.997427in}}%
\pgfpathlineto{\pgfqpoint{4.436359in}{1.819015in}}%
\pgfpathlineto{\pgfqpoint{4.445166in}{1.592534in}}%
\pgfpathlineto{\pgfqpoint{4.453972in}{1.880772in}}%
\pgfpathlineto{\pgfqpoint{4.462779in}{1.764116in}}%
\pgfpathlineto{\pgfqpoint{4.471586in}{1.702331in}}%
\pgfpathlineto{\pgfqpoint{4.480393in}{1.894489in}}%
\pgfpathlineto{\pgfqpoint{4.489200in}{1.963133in}}%
\pgfpathlineto{\pgfqpoint{4.498007in}{1.764116in}}%
\pgfpathlineto{\pgfqpoint{4.506813in}{1.716076in}}%
\pgfpathlineto{\pgfqpoint{4.515620in}{1.537636in}}%
\pgfpathlineto{\pgfqpoint{4.524427in}{1.434698in}}%
\pgfpathlineto{\pgfqpoint{4.533234in}{1.674896in}}%
\pgfpathlineto{\pgfqpoint{4.542041in}{1.420952in}}%
\pgfpathlineto{\pgfqpoint{4.550847in}{1.729794in}}%
\pgfpathlineto{\pgfqpoint{4.559654in}{1.599393in}}%
\pgfpathlineto{\pgfqpoint{4.577268in}{2.141546in}}%
\pgfpathlineto{\pgfqpoint{4.586075in}{1.894489in}}%
\pgfpathlineto{\pgfqpoint{4.594882in}{1.798410in}}%
\pgfpathlineto{\pgfqpoint{4.603688in}{1.791551in}}%
\pgfpathlineto{\pgfqpoint{4.612495in}{1.647432in}}%
\pgfpathlineto{\pgfqpoint{4.621302in}{1.750370in}}%
\pgfpathlineto{\pgfqpoint{4.630109in}{1.626856in}}%
\pgfpathlineto{\pgfqpoint{4.638916in}{1.716076in}}%
\pgfpathlineto{\pgfqpoint{4.647722in}{1.777834in}}%
\pgfpathlineto{\pgfqpoint{4.656529in}{1.578816in}}%
\pgfpathlineto{\pgfqpoint{4.665336in}{1.537636in}}%
\pgfpathlineto{\pgfqpoint{4.674143in}{1.640574in}}%
\pgfpathlineto{\pgfqpoint{4.682950in}{1.983710in}}%
\pgfpathlineto{\pgfqpoint{4.691757in}{1.825873in}}%
\pgfpathlineto{\pgfqpoint{4.700563in}{1.709190in}}%
\pgfpathlineto{\pgfqpoint{4.709370in}{1.777834in}}%
\pgfpathlineto{\pgfqpoint{4.718177in}{1.674896in}}%
\pgfpathlineto{\pgfqpoint{4.726984in}{2.004286in}}%
\pgfpathlineto{\pgfqpoint{4.735791in}{1.901348in}}%
\pgfpathlineto{\pgfqpoint{4.744597in}{1.825873in}}%
\pgfpathlineto{\pgfqpoint{4.753404in}{1.976851in}}%
\pgfpathlineto{\pgfqpoint{4.762211in}{1.901348in}}%
\pgfpathlineto{\pgfqpoint{4.771018in}{1.654291in}}%
\pgfpathlineto{\pgfqpoint{4.779825in}{1.839591in}}%
\pgfpathlineto{\pgfqpoint{4.788632in}{1.791551in}}%
\pgfpathlineto{\pgfqpoint{4.797438in}{1.880772in}}%
\pgfpathlineto{\pgfqpoint{4.806245in}{2.011173in}}%
\pgfpathlineto{\pgfqpoint{4.815052in}{2.038608in}}%
\pgfpathlineto{\pgfqpoint{4.823859in}{2.120970in}}%
\pgfpathlineto{\pgfqpoint{4.832666in}{1.983710in}}%
\pgfpathlineto{\pgfqpoint{4.841472in}{1.887630in}}%
\pgfpathlineto{\pgfqpoint{4.850279in}{2.093507in}}%
\pgfpathlineto{\pgfqpoint{4.867893in}{2.079789in}}%
\pgfpathlineto{\pgfqpoint{4.876700in}{1.901348in}}%
\pgfpathlineto{\pgfqpoint{4.885507in}{1.860167in}}%
\pgfpathlineto{\pgfqpoint{4.894313in}{1.949388in}}%
\pgfpathlineto{\pgfqpoint{4.903120in}{1.887630in}}%
\pgfpathlineto{\pgfqpoint{4.911927in}{1.544494in}}%
\pgfpathlineto{\pgfqpoint{4.920734in}{1.400376in}}%
\pgfpathlineto{\pgfqpoint{4.929541in}{1.585675in}}%
\pgfpathlineto{\pgfqpoint{4.938347in}{1.558212in}}%
\pgfpathlineto{\pgfqpoint{4.955961in}{2.203303in}}%
\pgfpathlineto{\pgfqpoint{4.964768in}{1.798410in}}%
\pgfpathlineto{\pgfqpoint{4.973575in}{1.702331in}}%
\pgfpathlineto{\pgfqpoint{4.982382in}{1.407234in}}%
\pgfpathlineto{\pgfqpoint{4.991188in}{1.496455in}}%
\pgfpathlineto{\pgfqpoint{4.999995in}{1.386658in}}%
\pgfpathlineto{\pgfqpoint{5.008802in}{1.503314in}}%
\pgfpathlineto{\pgfqpoint{5.017609in}{1.544494in}}%
\pgfpathlineto{\pgfqpoint{5.026416in}{1.208217in}}%
\pgfpathlineto{\pgfqpoint{5.035222in}{1.249398in}}%
\pgfpathlineto{\pgfqpoint{5.044029in}{1.016059in}}%
\pgfpathlineto{\pgfqpoint{5.052836in}{1.269974in}}%
\pgfpathlineto{\pgfqpoint{5.061643in}{1.359195in}}%
\pgfpathlineto{\pgfqpoint{5.070450in}{1.517031in}}%
\pgfpathlineto{\pgfqpoint{5.079257in}{1.626856in}}%
\pgfpathlineto{\pgfqpoint{5.088063in}{1.585675in}}%
\pgfpathlineto{\pgfqpoint{5.096870in}{1.722935in}}%
\pgfpathlineto{\pgfqpoint{5.105677in}{1.819015in}}%
\pgfpathlineto{\pgfqpoint{5.114484in}{1.942529in}}%
\pgfpathlineto{\pgfqpoint{5.123291in}{1.825873in}}%
\pgfpathlineto{\pgfqpoint{5.132097in}{1.757229in}}%
\pgfpathlineto{\pgfqpoint{5.140904in}{1.921953in}}%
\pgfpathlineto{\pgfqpoint{5.149711in}{2.162151in}}%
\pgfpathlineto{\pgfqpoint{5.158518in}{1.544494in}}%
\pgfpathlineto{\pgfqpoint{5.167325in}{1.592534in}}%
\pgfpathlineto{\pgfqpoint{5.176132in}{1.839591in}}%
\pgfpathlineto{\pgfqpoint{5.184938in}{1.915094in}}%
\pgfpathlineto{\pgfqpoint{5.193745in}{1.880772in}}%
\pgfpathlineto{\pgfqpoint{5.202552in}{1.860167in}}%
\pgfpathlineto{\pgfqpoint{5.211359in}{1.640574in}}%
\pgfpathlineto{\pgfqpoint{5.220166in}{1.736653in}}%
\pgfpathlineto{\pgfqpoint{5.228972in}{1.743512in}}%
\pgfpathlineto{\pgfqpoint{5.237779in}{1.853308in}}%
\pgfpathlineto{\pgfqpoint{5.246586in}{1.873913in}}%
\pgfpathlineto{\pgfqpoint{5.255393in}{2.244484in}}%
\pgfpathlineto{\pgfqpoint{5.264200in}{2.326846in}}%
\pgfpathlineto{\pgfqpoint{5.273007in}{2.635660in}}%
\pgfpathlineto{\pgfqpoint{5.290620in}{1.894489in}}%
\pgfpathlineto{\pgfqpoint{5.299427in}{1.592534in}}%
\pgfpathlineto{\pgfqpoint{5.308234in}{1.462133in}}%
\pgfpathlineto{\pgfqpoint{5.317041in}{1.366053in}}%
\pgfpathlineto{\pgfqpoint{5.325847in}{1.084703in}}%
\pgfpathlineto{\pgfqpoint{5.334654in}{0.940584in}}%
\pgfpathlineto{\pgfqpoint{5.352268in}{1.695472in}}%
\pgfpathlineto{\pgfqpoint{5.361075in}{1.674896in}}%
\pgfpathlineto{\pgfqpoint{5.369882in}{1.880772in}}%
\pgfpathlineto{\pgfqpoint{5.378688in}{1.942529in}}%
\pgfpathlineto{\pgfqpoint{5.387495in}{1.764116in}}%
\pgfpathlineto{\pgfqpoint{5.396302in}{1.791551in}}%
\pgfpathlineto{\pgfqpoint{5.405109in}{1.791551in}}%
\pgfpathlineto{\pgfqpoint{5.413916in}{1.880772in}}%
\pgfpathlineto{\pgfqpoint{5.422722in}{2.162151in}}%
\pgfpathlineto{\pgfqpoint{5.431529in}{2.265089in}}%
\pgfpathlineto{\pgfqpoint{5.440336in}{2.395462in}}%
\pgfpathlineto{\pgfqpoint{5.449143in}{2.347422in}}%
\pgfpathlineto{\pgfqpoint{5.457950in}{2.004286in}}%
\pgfpathlineto{\pgfqpoint{5.466757in}{2.361168in}}%
\pgfpathlineto{\pgfqpoint{5.475563in}{2.223908in}}%
\pgfpathlineto{\pgfqpoint{5.484370in}{1.963133in}}%
\pgfpathlineto{\pgfqpoint{5.493177in}{1.750370in}}%
\pgfpathlineto{\pgfqpoint{5.510791in}{2.278806in}}%
\pgfpathlineto{\pgfqpoint{5.519597in}{2.038608in}}%
\pgfpathlineto{\pgfqpoint{5.528404in}{1.853308in}}%
\pgfpathlineto{\pgfqpoint{5.537211in}{1.798410in}}%
\pgfpathlineto{\pgfqpoint{5.546018in}{2.038608in}}%
\pgfpathlineto{\pgfqpoint{5.554825in}{2.402349in}}%
\pgfpathlineto{\pgfqpoint{5.563632in}{2.512146in}}%
\pgfpathlineto{\pgfqpoint{5.572438in}{2.100365in}}%
\pgfpathlineto{\pgfqpoint{5.581245in}{2.175868in}}%
\pgfpathlineto{\pgfqpoint{5.590052in}{2.018032in}}%
\pgfpathlineto{\pgfqpoint{5.598859in}{1.585675in}}%
\pgfpathlineto{\pgfqpoint{5.607666in}{1.928811in}}%
\pgfpathlineto{\pgfqpoint{5.616472in}{1.969992in}}%
\pgfpathlineto{\pgfqpoint{5.625279in}{2.018032in}}%
\pgfpathlineto{\pgfqpoint{5.634086in}{1.846450in}}%
\pgfpathlineto{\pgfqpoint{5.642893in}{1.770975in}}%
\pgfpathlineto{\pgfqpoint{5.651700in}{2.313128in}}%
\pgfpathlineto{\pgfqpoint{5.660507in}{2.155292in}}%
\pgfpathlineto{\pgfqpoint{5.669313in}{1.949388in}}%
\pgfpathlineto{\pgfqpoint{5.678120in}{1.805269in}}%
\pgfpathlineto{\pgfqpoint{5.686927in}{2.011173in}}%
\pgfpathlineto{\pgfqpoint{5.695734in}{2.271947in}}%
\pgfpathlineto{\pgfqpoint{5.704541in}{2.347422in}}%
\pgfpathlineto{\pgfqpoint{5.713347in}{2.278806in}}%
\pgfpathlineto{\pgfqpoint{5.722154in}{2.018032in}}%
\pgfpathlineto{\pgfqpoint{5.730961in}{1.963133in}}%
\pgfpathlineto{\pgfqpoint{5.739768in}{1.983710in}}%
\pgfpathlineto{\pgfqpoint{5.748575in}{1.709190in}}%
\pgfpathlineto{\pgfqpoint{5.757382in}{1.880772in}}%
\pgfpathlineto{\pgfqpoint{5.774995in}{2.093507in}}%
\pgfpathlineto{\pgfqpoint{5.783802in}{1.949388in}}%
\pgfpathlineto{\pgfqpoint{5.792609in}{1.599393in}}%
\pgfpathlineto{\pgfqpoint{5.801416in}{1.784692in}}%
\pgfpathlineto{\pgfqpoint{5.810222in}{1.496455in}}%
\pgfpathlineto{\pgfqpoint{5.819029in}{1.578816in}}%
\pgfpathlineto{\pgfqpoint{5.827836in}{1.798410in}}%
\pgfpathlineto{\pgfqpoint{5.836643in}{1.736653in}}%
\pgfpathlineto{\pgfqpoint{5.845450in}{1.736653in}}%
\pgfpathlineto{\pgfqpoint{5.854257in}{1.969992in}}%
\pgfpathlineto{\pgfqpoint{5.863063in}{1.956246in}}%
\pgfpathlineto{\pgfqpoint{5.871870in}{2.169009in}}%
\pgfpathlineto{\pgfqpoint{5.880677in}{1.812128in}}%
\pgfpathlineto{\pgfqpoint{5.889484in}{2.333705in}}%
\pgfpathlineto{\pgfqpoint{5.898291in}{2.484682in}}%
\pgfpathlineto{\pgfqpoint{5.907097in}{2.038608in}}%
\pgfpathlineto{\pgfqpoint{5.915904in}{1.997427in}}%
\pgfpathlineto{\pgfqpoint{5.924711in}{1.887630in}}%
\pgfpathlineto{\pgfqpoint{5.933518in}{2.093507in}}%
\pgfpathlineto{\pgfqpoint{5.942325in}{1.969992in}}%
\pgfpathlineto{\pgfqpoint{5.951132in}{1.915094in}}%
\pgfpathlineto{\pgfqpoint{5.959938in}{1.921953in}}%
\pgfpathlineto{\pgfqpoint{5.968745in}{1.619997in}}%
\pgfpathlineto{\pgfqpoint{5.977552in}{1.873913in}}%
\pgfpathlineto{\pgfqpoint{5.986359in}{1.805269in}}%
\pgfpathlineto{\pgfqpoint{5.995166in}{1.592534in}}%
\pgfpathlineto{\pgfqpoint{6.003972in}{1.414093in}}%
\pgfpathlineto{\pgfqpoint{6.012779in}{1.366053in}}%
\pgfpathlineto{\pgfqpoint{6.021586in}{1.468992in}}%
\pgfpathlineto{\pgfqpoint{6.030393in}{1.812128in}}%
\pgfpathlineto{\pgfqpoint{6.039200in}{1.798410in}}%
\pgfpathlineto{\pgfqpoint{6.056813in}{1.434698in}}%
\pgfpathlineto{\pgfqpoint{6.065620in}{1.372912in}}%
\pgfpathlineto{\pgfqpoint{6.074427in}{1.160177in}}%
\pgfpathlineto{\pgfqpoint{6.083234in}{1.537636in}}%
\pgfpathlineto{\pgfqpoint{6.092041in}{1.716076in}}%
\pgfpathlineto{\pgfqpoint{6.100847in}{1.736653in}}%
\pgfpathlineto{\pgfqpoint{6.109654in}{1.805269in}}%
\pgfpathlineto{\pgfqpoint{6.118461in}{1.722935in}}%
\pgfpathlineto{\pgfqpoint{6.127268in}{1.743512in}}%
\pgfpathlineto{\pgfqpoint{6.136075in}{1.290579in}}%
\pgfpathlineto{\pgfqpoint{6.144882in}{1.256257in}}%
\pgfpathlineto{\pgfqpoint{6.153688in}{1.201358in}}%
\pgfpathlineto{\pgfqpoint{6.162495in}{1.283720in}}%
\pgfpathlineto{\pgfqpoint{6.171302in}{1.331760in}}%
\pgfpathlineto{\pgfqpoint{6.180109in}{1.448415in}}%
\pgfpathlineto{\pgfqpoint{6.188916in}{1.448415in}}%
\pgfpathlineto{\pgfqpoint{6.206529in}{1.619997in}}%
\pgfpathlineto{\pgfqpoint{6.215336in}{1.805269in}}%
\pgfpathlineto{\pgfqpoint{6.224143in}{1.867054in}}%
\pgfpathlineto{\pgfqpoint{6.232950in}{2.072930in}}%
\pgfpathlineto{\pgfqpoint{6.241757in}{2.086648in}}%
\pgfpathlineto{\pgfqpoint{6.250563in}{1.853308in}}%
\pgfpathlineto{\pgfqpoint{6.259370in}{2.018032in}}%
\pgfpathlineto{\pgfqpoint{6.268177in}{2.093507in}}%
\pgfpathlineto{\pgfqpoint{6.276984in}{2.189586in}}%
\pgfpathlineto{\pgfqpoint{6.285791in}{2.011173in}}%
\pgfpathlineto{\pgfqpoint{6.294597in}{1.963133in}}%
\pgfpathlineto{\pgfqpoint{6.303404in}{1.990569in}}%
\pgfpathlineto{\pgfqpoint{6.312211in}{1.901348in}}%
\pgfpathlineto{\pgfqpoint{6.321018in}{1.592534in}}%
\pgfpathlineto{\pgfqpoint{6.338632in}{2.148405in}}%
\pgfpathlineto{\pgfqpoint{6.347438in}{1.969992in}}%
\pgfpathlineto{\pgfqpoint{6.356245in}{2.203303in}}%
\pgfpathlineto{\pgfqpoint{6.365052in}{2.127829in}}%
\pgfpathlineto{\pgfqpoint{6.373859in}{2.182727in}}%
\pgfpathlineto{\pgfqpoint{6.382666in}{2.203303in}}%
\pgfpathlineto{\pgfqpoint{6.391472in}{2.004286in}}%
\pgfpathlineto{\pgfqpoint{6.400279in}{2.196445in}}%
\pgfpathlineto{\pgfqpoint{6.409086in}{2.539581in}}%
\pgfpathlineto{\pgfqpoint{6.417893in}{2.381744in}}%
\pgfpathlineto{\pgfqpoint{6.426700in}{2.052326in}}%
\pgfpathlineto{\pgfqpoint{6.435507in}{2.066071in}}%
\pgfpathlineto{\pgfqpoint{6.453120in}{1.709190in}}%
\pgfpathlineto{\pgfqpoint{6.461927in}{1.757229in}}%
\pgfpathlineto{\pgfqpoint{6.470734in}{1.798410in}}%
\pgfpathlineto{\pgfqpoint{6.479541in}{1.743512in}}%
\pgfpathlineto{\pgfqpoint{6.488347in}{1.812128in}}%
\pgfpathlineto{\pgfqpoint{6.497154in}{1.942529in}}%
\pgfpathlineto{\pgfqpoint{6.505961in}{2.217049in}}%
\pgfpathlineto{\pgfqpoint{6.514768in}{1.887630in}}%
\pgfpathlineto{\pgfqpoint{6.523575in}{2.107252in}}%
\pgfpathlineto{\pgfqpoint{6.532382in}{2.004286in}}%
\pgfpathlineto{\pgfqpoint{6.541188in}{1.887630in}}%
\pgfpathlineto{\pgfqpoint{6.549995in}{2.340563in}}%
\pgfpathlineto{\pgfqpoint{6.558802in}{1.935670in}}%
\pgfpathlineto{\pgfqpoint{6.567609in}{2.086648in}}%
\pgfpathlineto{\pgfqpoint{6.576416in}{1.681754in}}%
\pgfpathlineto{\pgfqpoint{6.585222in}{1.571958in}}%
\pgfpathlineto{\pgfqpoint{6.594029in}{1.558212in}}%
\pgfpathlineto{\pgfqpoint{6.602836in}{2.086648in}}%
\pgfpathlineto{\pgfqpoint{6.611643in}{2.251343in}}%
\pgfpathlineto{\pgfqpoint{6.620450in}{1.976851in}}%
\pgfpathlineto{\pgfqpoint{6.629257in}{2.120970in}}%
\pgfpathlineto{\pgfqpoint{6.638063in}{2.072930in}}%
\pgfpathlineto{\pgfqpoint{6.646870in}{2.052326in}}%
\pgfpathlineto{\pgfqpoint{6.655677in}{1.770975in}}%
\pgfpathlineto{\pgfqpoint{6.664484in}{1.873913in}}%
\pgfpathlineto{\pgfqpoint{6.673291in}{1.716076in}}%
\pgfpathlineto{\pgfqpoint{6.682097in}{1.825873in}}%
\pgfpathlineto{\pgfqpoint{6.690904in}{1.736653in}}%
\pgfpathlineto{\pgfqpoint{6.699711in}{1.825873in}}%
\pgfpathlineto{\pgfqpoint{6.708518in}{2.258230in}}%
\pgfpathlineto{\pgfqpoint{6.717325in}{1.839591in}}%
\pgfpathlineto{\pgfqpoint{6.726132in}{2.251343in}}%
\pgfpathlineto{\pgfqpoint{6.734938in}{2.120970in}}%
\pgfpathlineto{\pgfqpoint{6.743745in}{2.045467in}}%
\pgfpathlineto{\pgfqpoint{6.752552in}{1.928811in}}%
\pgfpathlineto{\pgfqpoint{6.761359in}{1.777834in}}%
\pgfpathlineto{\pgfqpoint{6.770166in}{1.805269in}}%
\pgfpathlineto{\pgfqpoint{6.778972in}{1.709190in}}%
\pgfpathlineto{\pgfqpoint{6.796586in}{1.853308in}}%
\pgfpathlineto{\pgfqpoint{6.805393in}{1.894489in}}%
\pgfpathlineto{\pgfqpoint{6.814200in}{1.867054in}}%
\pgfpathlineto{\pgfqpoint{6.823007in}{1.832732in}}%
\pgfpathlineto{\pgfqpoint{6.831813in}{1.908207in}}%
\pgfpathlineto{\pgfqpoint{6.840620in}{2.004286in}}%
\pgfpathlineto{\pgfqpoint{6.849427in}{2.079789in}}%
\pgfpathlineto{\pgfqpoint{6.858234in}{1.729794in}}%
\pgfpathlineto{\pgfqpoint{6.867041in}{1.805269in}}%
\pgfpathlineto{\pgfqpoint{6.875847in}{1.750370in}}%
\pgfpathlineto{\pgfqpoint{6.884654in}{2.120970in}}%
\pgfpathlineto{\pgfqpoint{6.893461in}{2.230767in}}%
\pgfpathlineto{\pgfqpoint{6.902268in}{2.175868in}}%
\pgfpathlineto{\pgfqpoint{6.911075in}{1.990569in}}%
\pgfpathlineto{\pgfqpoint{6.919882in}{1.942529in}}%
\pgfpathlineto{\pgfqpoint{6.928688in}{1.647432in}}%
\pgfpathlineto{\pgfqpoint{6.937495in}{1.853308in}}%
\pgfpathlineto{\pgfqpoint{6.946302in}{1.791551in}}%
\pgfpathlineto{\pgfqpoint{6.955109in}{1.757229in}}%
\pgfpathlineto{\pgfqpoint{6.963916in}{1.867054in}}%
\pgfpathlineto{\pgfqpoint{6.972722in}{1.880772in}}%
\pgfpathlineto{\pgfqpoint{6.981529in}{2.127829in}}%
\pgfpathlineto{\pgfqpoint{6.990336in}{1.894489in}}%
\pgfpathlineto{\pgfqpoint{7.007950in}{2.059213in}}%
\pgfpathlineto{\pgfqpoint{7.016757in}{2.011173in}}%
\pgfpathlineto{\pgfqpoint{7.025563in}{1.976851in}}%
\pgfpathlineto{\pgfqpoint{7.034370in}{2.271947in}}%
\pgfpathlineto{\pgfqpoint{7.043177in}{2.162151in}}%
\pgfpathlineto{\pgfqpoint{7.051984in}{2.539581in}}%
\pgfpathlineto{\pgfqpoint{7.060791in}{2.786638in}}%
\pgfpathlineto{\pgfqpoint{7.069597in}{2.965078in}}%
\pgfpathlineto{\pgfqpoint{7.078404in}{2.704304in}}%
\pgfpathlineto{\pgfqpoint{7.087211in}{2.690558in}}%
\pgfpathlineto{\pgfqpoint{7.096018in}{2.155292in}}%
\pgfpathlineto{\pgfqpoint{7.113632in}{2.169009in}}%
\pgfpathlineto{\pgfqpoint{7.122438in}{2.416066in}}%
\pgfpathlineto{\pgfqpoint{7.131245in}{2.402349in}}%
\pgfpathlineto{\pgfqpoint{7.148859in}{1.633715in}}%
\pgfpathlineto{\pgfqpoint{7.157666in}{1.571958in}}%
\pgfpathlineto{\pgfqpoint{7.166472in}{1.496455in}}%
\pgfpathlineto{\pgfqpoint{7.175279in}{1.661150in}}%
\pgfpathlineto{\pgfqpoint{7.184086in}{2.189586in}}%
\pgfpathlineto{\pgfqpoint{7.192893in}{2.299383in}}%
\pgfpathlineto{\pgfqpoint{7.201700in}{1.860167in}}%
\pgfpathlineto{\pgfqpoint{7.210507in}{1.860167in}}%
\pgfpathlineto{\pgfqpoint{7.219313in}{1.908207in}}%
\pgfpathlineto{\pgfqpoint{7.228120in}{1.915094in}}%
\pgfpathlineto{\pgfqpoint{7.236927in}{1.963133in}}%
\pgfpathlineto{\pgfqpoint{7.245734in}{1.812128in}}%
\pgfpathlineto{\pgfqpoint{7.254541in}{1.935670in}}%
\pgfpathlineto{\pgfqpoint{7.263347in}{1.846450in}}%
\pgfpathlineto{\pgfqpoint{7.272154in}{2.031749in}}%
\pgfpathlineto{\pgfqpoint{7.280961in}{1.915094in}}%
\pgfpathlineto{\pgfqpoint{7.289768in}{1.915094in}}%
\pgfpathlineto{\pgfqpoint{7.298575in}{2.045467in}}%
\pgfpathlineto{\pgfqpoint{7.307382in}{2.011173in}}%
\pgfpathlineto{\pgfqpoint{7.316188in}{1.764116in}}%
\pgfpathlineto{\pgfqpoint{7.324995in}{1.777834in}}%
\pgfpathlineto{\pgfqpoint{7.333802in}{1.647432in}}%
\pgfpathlineto{\pgfqpoint{7.342609in}{1.695472in}}%
\pgfpathlineto{\pgfqpoint{7.351416in}{1.558212in}}%
\pgfpathlineto{\pgfqpoint{7.360222in}{1.928811in}}%
\pgfpathlineto{\pgfqpoint{7.369029in}{1.695472in}}%
\pgfpathlineto{\pgfqpoint{7.377836in}{1.633715in}}%
\pgfpathlineto{\pgfqpoint{7.386643in}{1.880772in}}%
\pgfpathlineto{\pgfqpoint{7.395450in}{1.853308in}}%
\pgfpathlineto{\pgfqpoint{7.404257in}{1.819015in}}%
\pgfpathlineto{\pgfqpoint{7.413063in}{2.169009in}}%
\pgfpathlineto{\pgfqpoint{7.421870in}{2.251343in}}%
\pgfpathlineto{\pgfqpoint{7.430677in}{2.018032in}}%
\pgfpathlineto{\pgfqpoint{7.439484in}{1.963133in}}%
\pgfpathlineto{\pgfqpoint{7.457097in}{2.004286in}}%
\pgfpathlineto{\pgfqpoint{7.465904in}{1.873913in}}%
\pgfpathlineto{\pgfqpoint{7.474711in}{1.921953in}}%
\pgfpathlineto{\pgfqpoint{7.483518in}{1.963133in}}%
\pgfpathlineto{\pgfqpoint{7.492325in}{1.942529in}}%
\pgfpathlineto{\pgfqpoint{7.509938in}{1.695472in}}%
\pgfpathlineto{\pgfqpoint{7.518745in}{1.770975in}}%
\pgfpathlineto{\pgfqpoint{7.527552in}{1.915094in}}%
\pgfpathlineto{\pgfqpoint{7.536359in}{1.873913in}}%
\pgfpathlineto{\pgfqpoint{7.545166in}{2.306269in}}%
\pgfpathlineto{\pgfqpoint{7.553972in}{1.812128in}}%
\pgfpathlineto{\pgfqpoint{7.562779in}{1.599393in}}%
\pgfpathlineto{\pgfqpoint{7.571586in}{1.510172in}}%
\pgfpathlineto{\pgfqpoint{7.580393in}{1.654291in}}%
\pgfpathlineto{\pgfqpoint{7.589200in}{2.045467in}}%
\pgfpathlineto{\pgfqpoint{7.606813in}{1.654291in}}%
\pgfpathlineto{\pgfqpoint{7.615620in}{1.551353in}}%
\pgfpathlineto{\pgfqpoint{7.624427in}{1.523918in}}%
\pgfpathlineto{\pgfqpoint{7.633234in}{1.599393in}}%
\pgfpathlineto{\pgfqpoint{7.650847in}{2.052326in}}%
\pgfpathlineto{\pgfqpoint{7.659654in}{2.114111in}}%
\pgfpathlineto{\pgfqpoint{7.668461in}{2.107252in}}%
\pgfpathlineto{\pgfqpoint{7.677268in}{1.839591in}}%
\pgfpathlineto{\pgfqpoint{7.686075in}{1.654291in}}%
\pgfpathlineto{\pgfqpoint{7.694882in}{1.517031in}}%
\pgfpathlineto{\pgfqpoint{7.703688in}{1.969992in}}%
\pgfpathlineto{\pgfqpoint{7.721302in}{2.464106in}}%
\pgfpathlineto{\pgfqpoint{7.730109in}{2.422925in}}%
\pgfpathlineto{\pgfqpoint{7.738916in}{2.354309in}}%
\pgfpathlineto{\pgfqpoint{7.747722in}{2.148405in}}%
\pgfpathlineto{\pgfqpoint{7.756529in}{2.217049in}}%
\pgfpathlineto{\pgfqpoint{7.765336in}{1.990569in}}%
\pgfpathlineto{\pgfqpoint{7.774143in}{1.990569in}}%
\pgfpathlineto{\pgfqpoint{7.782950in}{2.217049in}}%
\pgfpathlineto{\pgfqpoint{7.791757in}{1.983710in}}%
\pgfpathlineto{\pgfqpoint{7.800563in}{1.921953in}}%
\pgfpathlineto{\pgfqpoint{7.809370in}{1.873913in}}%
\pgfpathlineto{\pgfqpoint{7.818177in}{2.024891in}}%
\pgfpathlineto{\pgfqpoint{7.826984in}{1.915094in}}%
\pgfpathlineto{\pgfqpoint{7.835791in}{1.880772in}}%
\pgfpathlineto{\pgfqpoint{7.844597in}{1.805269in}}%
\pgfpathlineto{\pgfqpoint{7.853404in}{1.969992in}}%
\pgfpathlineto{\pgfqpoint{7.862211in}{2.072930in}}%
\pgfpathlineto{\pgfqpoint{7.871018in}{1.860167in}}%
\pgfpathlineto{\pgfqpoint{7.879825in}{2.100365in}}%
\pgfpathlineto{\pgfqpoint{7.888632in}{2.086648in}}%
\pgfpathlineto{\pgfqpoint{7.897438in}{2.237625in}}%
\pgfpathlineto{\pgfqpoint{7.906245in}{2.347422in}}%
\pgfpathlineto{\pgfqpoint{7.915052in}{2.539581in}}%
\pgfpathlineto{\pgfqpoint{7.923859in}{2.381744in}}%
\pgfpathlineto{\pgfqpoint{7.932666in}{2.354309in}}%
\pgfpathlineto{\pgfqpoint{7.941472in}{1.757229in}}%
\pgfpathlineto{\pgfqpoint{7.950279in}{1.976851in}}%
\pgfpathlineto{\pgfqpoint{7.959086in}{2.093507in}}%
\pgfpathlineto{\pgfqpoint{7.967893in}{2.141546in}}%
\pgfpathlineto{\pgfqpoint{7.976700in}{2.004286in}}%
\pgfpathlineto{\pgfqpoint{7.985507in}{1.729794in}}%
\pgfpathlineto{\pgfqpoint{7.994313in}{1.619997in}}%
\pgfpathlineto{\pgfqpoint{8.003120in}{1.455274in}}%
\pgfpathlineto{\pgfqpoint{8.011927in}{1.743512in}}%
\pgfpathlineto{\pgfqpoint{8.020734in}{1.420952in}}%
\pgfpathlineto{\pgfqpoint{8.029541in}{1.901348in}}%
\pgfpathlineto{\pgfqpoint{8.038347in}{1.722935in}}%
\pgfpathlineto{\pgfqpoint{8.047154in}{1.764116in}}%
\pgfpathlineto{\pgfqpoint{8.055961in}{1.853308in}}%
\pgfpathlineto{\pgfqpoint{8.064768in}{1.482737in}}%
\pgfpathlineto{\pgfqpoint{8.073575in}{1.201358in}}%
\pgfpathlineto{\pgfqpoint{8.082382in}{1.187641in}}%
\pgfpathlineto{\pgfqpoint{8.091188in}{1.372912in}}%
\pgfpathlineto{\pgfqpoint{8.099995in}{1.475878in}}%
\pgfpathlineto{\pgfqpoint{8.108802in}{1.496455in}}%
\pgfpathlineto{\pgfqpoint{8.117609in}{1.523918in}}%
\pgfpathlineto{\pgfqpoint{8.126416in}{1.832732in}}%
\pgfpathlineto{\pgfqpoint{8.135222in}{1.626856in}}%
\pgfpathlineto{\pgfqpoint{8.144029in}{1.661150in}}%
\pgfpathlineto{\pgfqpoint{8.152836in}{2.024891in}}%
\pgfpathlineto{\pgfqpoint{8.161643in}{1.942529in}}%
\pgfpathlineto{\pgfqpoint{8.170450in}{2.059213in}}%
\pgfpathlineto{\pgfqpoint{8.179257in}{2.361168in}}%
\pgfpathlineto{\pgfqpoint{8.188063in}{2.429784in}}%
\pgfpathlineto{\pgfqpoint{8.196870in}{2.203303in}}%
\pgfpathlineto{\pgfqpoint{8.205677in}{2.354309in}}%
\pgfpathlineto{\pgfqpoint{8.223291in}{1.983710in}}%
\pgfpathlineto{\pgfqpoint{8.232097in}{1.949388in}}%
\pgfpathlineto{\pgfqpoint{8.249711in}{1.530777in}}%
\pgfpathlineto{\pgfqpoint{8.258518in}{1.949388in}}%
\pgfpathlineto{\pgfqpoint{8.267325in}{1.956246in}}%
\pgfpathlineto{\pgfqpoint{8.276132in}{1.709190in}}%
\pgfpathlineto{\pgfqpoint{8.293745in}{2.031749in}}%
\pgfpathlineto{\pgfqpoint{8.302552in}{2.086648in}}%
\pgfpathlineto{\pgfqpoint{8.311359in}{2.114111in}}%
\pgfpathlineto{\pgfqpoint{8.320166in}{1.846450in}}%
\pgfpathlineto{\pgfqpoint{8.328972in}{2.004286in}}%
\pgfpathlineto{\pgfqpoint{8.337779in}{2.189586in}}%
\pgfpathlineto{\pgfqpoint{8.346586in}{2.429784in}}%
\pgfpathlineto{\pgfqpoint{8.355393in}{2.340563in}}%
\pgfpathlineto{\pgfqpoint{8.364200in}{2.155292in}}%
\pgfpathlineto{\pgfqpoint{8.373007in}{2.189586in}}%
\pgfpathlineto{\pgfqpoint{8.381813in}{2.285665in}}%
\pgfpathlineto{\pgfqpoint{8.390620in}{2.306269in}}%
\pgfpathlineto{\pgfqpoint{8.399427in}{2.196445in}}%
\pgfpathlineto{\pgfqpoint{8.408234in}{1.908207in}}%
\pgfpathlineto{\pgfqpoint{8.417041in}{1.832732in}}%
\pgfpathlineto{\pgfqpoint{8.425847in}{1.709190in}}%
\pgfpathlineto{\pgfqpoint{8.434654in}{1.928811in}}%
\pgfpathlineto{\pgfqpoint{8.443461in}{2.045467in}}%
\pgfpathlineto{\pgfqpoint{8.452268in}{1.839591in}}%
\pgfpathlineto{\pgfqpoint{8.461075in}{1.695472in}}%
\pgfpathlineto{\pgfqpoint{8.469882in}{1.770975in}}%
\pgfpathlineto{\pgfqpoint{8.478688in}{1.626856in}}%
\pgfpathlineto{\pgfqpoint{8.487495in}{1.448415in}}%
\pgfpathlineto{\pgfqpoint{8.496302in}{1.647432in}}%
\pgfpathlineto{\pgfqpoint{8.505109in}{1.468992in}}%
\pgfpathlineto{\pgfqpoint{8.513916in}{1.386658in}}%
\pgfpathlineto{\pgfqpoint{8.522722in}{1.448415in}}%
\pgfpathlineto{\pgfqpoint{8.531529in}{1.379799in}}%
\pgfpathlineto{\pgfqpoint{8.549143in}{2.285665in}}%
\pgfpathlineto{\pgfqpoint{8.557950in}{2.182727in}}%
\pgfpathlineto{\pgfqpoint{8.566757in}{2.148405in}}%
\pgfpathlineto{\pgfqpoint{8.575563in}{2.299383in}}%
\pgfpathlineto{\pgfqpoint{8.584370in}{2.059213in}}%
\pgfpathlineto{\pgfqpoint{8.593177in}{2.086648in}}%
\pgfpathlineto{\pgfqpoint{8.601984in}{1.819015in}}%
\pgfpathlineto{\pgfqpoint{8.610791in}{1.853308in}}%
\pgfpathlineto{\pgfqpoint{8.619597in}{2.093507in}}%
\pgfpathlineto{\pgfqpoint{8.628404in}{2.038608in}}%
\pgfpathlineto{\pgfqpoint{8.637211in}{1.819015in}}%
\pgfpathlineto{\pgfqpoint{8.646018in}{1.969992in}}%
\pgfpathlineto{\pgfqpoint{8.654825in}{1.743512in}}%
\pgfpathlineto{\pgfqpoint{8.663632in}{1.702331in}}%
\pgfpathlineto{\pgfqpoint{8.672438in}{1.592534in}}%
\pgfpathlineto{\pgfqpoint{8.681245in}{1.578816in}}%
\pgfpathlineto{\pgfqpoint{8.690052in}{1.434698in}}%
\pgfpathlineto{\pgfqpoint{8.707666in}{1.489596in}}%
\pgfpathlineto{\pgfqpoint{8.716472in}{1.551353in}}%
\pgfpathlineto{\pgfqpoint{8.725279in}{1.743512in}}%
\pgfpathlineto{\pgfqpoint{8.734086in}{1.606252in}}%
\pgfpathlineto{\pgfqpoint{8.742893in}{1.770975in}}%
\pgfpathlineto{\pgfqpoint{8.751700in}{1.510172in}}%
\pgfpathlineto{\pgfqpoint{8.760507in}{1.510172in}}%
\pgfpathlineto{\pgfqpoint{8.769313in}{1.359195in}}%
\pgfpathlineto{\pgfqpoint{8.778120in}{1.695472in}}%
\pgfpathlineto{\pgfqpoint{8.786927in}{1.839591in}}%
\pgfpathlineto{\pgfqpoint{8.804541in}{2.093507in}}%
\pgfpathlineto{\pgfqpoint{8.813347in}{2.100365in}}%
\pgfpathlineto{\pgfqpoint{8.830961in}{1.496455in}}%
\pgfpathlineto{\pgfqpoint{8.839768in}{1.455274in}}%
\pgfpathlineto{\pgfqpoint{8.848575in}{1.805269in}}%
\pgfpathlineto{\pgfqpoint{8.857382in}{1.585675in}}%
\pgfpathlineto{\pgfqpoint{8.866188in}{1.736653in}}%
\pgfpathlineto{\pgfqpoint{8.874995in}{1.949388in}}%
\pgfpathlineto{\pgfqpoint{8.883802in}{1.908207in}}%
\pgfpathlineto{\pgfqpoint{8.892609in}{1.956246in}}%
\pgfpathlineto{\pgfqpoint{8.901416in}{2.072930in}}%
\pgfpathlineto{\pgfqpoint{8.910222in}{1.716076in}}%
\pgfpathlineto{\pgfqpoint{8.919029in}{1.571958in}}%
\pgfpathlineto{\pgfqpoint{8.927836in}{1.215076in}}%
\pgfpathlineto{\pgfqpoint{8.945450in}{1.928811in}}%
\pgfpathlineto{\pgfqpoint{8.954257in}{1.805269in}}%
\pgfpathlineto{\pgfqpoint{8.963063in}{1.860167in}}%
\pgfpathlineto{\pgfqpoint{8.971870in}{1.825873in}}%
\pgfpathlineto{\pgfqpoint{8.980677in}{1.681754in}}%
\pgfpathlineto{\pgfqpoint{8.989484in}{1.867054in}}%
\pgfpathlineto{\pgfqpoint{8.998291in}{2.155292in}}%
\pgfpathlineto{\pgfqpoint{9.007097in}{1.901348in}}%
\pgfpathlineto{\pgfqpoint{9.015904in}{1.585675in}}%
\pgfpathlineto{\pgfqpoint{9.024711in}{1.496455in}}%
\pgfpathlineto{\pgfqpoint{9.033518in}{1.729794in}}%
\pgfpathlineto{\pgfqpoint{9.042325in}{1.908207in}}%
\pgfpathlineto{\pgfqpoint{9.051132in}{1.908207in}}%
\pgfpathlineto{\pgfqpoint{9.059938in}{1.702331in}}%
\pgfpathlineto{\pgfqpoint{9.068745in}{1.867054in}}%
\pgfpathlineto{\pgfqpoint{9.077552in}{1.839591in}}%
\pgfpathlineto{\pgfqpoint{9.086359in}{1.599393in}}%
\pgfpathlineto{\pgfqpoint{9.095166in}{2.217049in}}%
\pgfpathlineto{\pgfqpoint{9.103972in}{2.134687in}}%
\pgfpathlineto{\pgfqpoint{9.112779in}{1.777834in}}%
\pgfpathlineto{\pgfqpoint{9.121586in}{1.915094in}}%
\pgfpathlineto{\pgfqpoint{9.130393in}{2.313128in}}%
\pgfpathlineto{\pgfqpoint{9.139200in}{2.374885in}}%
\pgfpathlineto{\pgfqpoint{9.148007in}{2.162151in}}%
\pgfpathlineto{\pgfqpoint{9.156813in}{2.251343in}}%
\pgfpathlineto{\pgfqpoint{9.165620in}{2.196445in}}%
\pgfpathlineto{\pgfqpoint{9.174427in}{2.018032in}}%
\pgfpathlineto{\pgfqpoint{9.183234in}{2.066071in}}%
\pgfpathlineto{\pgfqpoint{9.192041in}{2.031749in}}%
\pgfpathlineto{\pgfqpoint{9.200847in}{2.120970in}}%
\pgfpathlineto{\pgfqpoint{9.209654in}{1.757229in}}%
\pgfpathlineto{\pgfqpoint{9.218461in}{1.695472in}}%
\pgfpathlineto{\pgfqpoint{9.227268in}{2.354309in}}%
\pgfpathlineto{\pgfqpoint{9.236075in}{2.107252in}}%
\pgfpathlineto{\pgfqpoint{9.244882in}{2.052326in}}%
\pgfpathlineto{\pgfqpoint{9.253688in}{1.825873in}}%
\pgfpathlineto{\pgfqpoint{9.262495in}{2.072930in}}%
\pgfpathlineto{\pgfqpoint{9.271302in}{2.429784in}}%
\pgfpathlineto{\pgfqpoint{9.280109in}{2.505287in}}%
\pgfpathlineto{\pgfqpoint{9.288916in}{2.292524in}}%
\pgfpathlineto{\pgfqpoint{9.297722in}{1.935670in}}%
\pgfpathlineto{\pgfqpoint{9.306529in}{2.086648in}}%
\pgfpathlineto{\pgfqpoint{9.315336in}{2.141546in}}%
\pgfpathlineto{\pgfqpoint{9.324143in}{1.908207in}}%
\pgfpathlineto{\pgfqpoint{9.332950in}{1.963133in}}%
\pgfpathlineto{\pgfqpoint{9.341757in}{2.038608in}}%
\pgfpathlineto{\pgfqpoint{9.350563in}{2.265089in}}%
\pgfpathlineto{\pgfqpoint{9.359370in}{2.251343in}}%
\pgfpathlineto{\pgfqpoint{9.368177in}{2.024891in}}%
\pgfpathlineto{\pgfqpoint{9.376984in}{1.983710in}}%
\pgfpathlineto{\pgfqpoint{9.385791in}{1.956246in}}%
\pgfpathlineto{\pgfqpoint{9.394597in}{2.004286in}}%
\pgfpathlineto{\pgfqpoint{9.403404in}{2.093507in}}%
\pgfpathlineto{\pgfqpoint{9.412211in}{2.100365in}}%
\pgfpathlineto{\pgfqpoint{9.421018in}{1.770975in}}%
\pgfpathlineto{\pgfqpoint{9.429825in}{1.839591in}}%
\pgfpathlineto{\pgfqpoint{9.438632in}{2.175868in}}%
\pgfpathlineto{\pgfqpoint{9.447438in}{2.134687in}}%
\pgfpathlineto{\pgfqpoint{9.456245in}{2.127829in}}%
\pgfpathlineto{\pgfqpoint{9.465052in}{2.100365in}}%
\pgfpathlineto{\pgfqpoint{9.473859in}{1.990569in}}%
\pgfpathlineto{\pgfqpoint{9.482666in}{1.619997in}}%
\pgfpathlineto{\pgfqpoint{9.491472in}{1.311155in}}%
\pgfpathlineto{\pgfqpoint{9.500279in}{1.276861in}}%
\pgfpathlineto{\pgfqpoint{9.509086in}{1.798410in}}%
\pgfpathlineto{\pgfqpoint{9.517893in}{1.969992in}}%
\pgfpathlineto{\pgfqpoint{9.526700in}{2.278806in}}%
\pgfpathlineto{\pgfqpoint{9.535507in}{2.100365in}}%
\pgfpathlineto{\pgfqpoint{9.544313in}{2.457247in}}%
\pgfpathlineto{\pgfqpoint{9.553120in}{2.265089in}}%
\pgfpathlineto{\pgfqpoint{9.561927in}{2.258230in}}%
\pgfpathlineto{\pgfqpoint{9.570734in}{2.175868in}}%
\pgfpathlineto{\pgfqpoint{9.579541in}{1.853308in}}%
\pgfpathlineto{\pgfqpoint{9.588347in}{1.770975in}}%
\pgfpathlineto{\pgfqpoint{9.597154in}{1.798410in}}%
\pgfpathlineto{\pgfqpoint{9.605961in}{1.963133in}}%
\pgfpathlineto{\pgfqpoint{9.614768in}{2.313128in}}%
\pgfpathlineto{\pgfqpoint{9.632382in}{1.455274in}}%
\pgfpathlineto{\pgfqpoint{9.641188in}{1.496455in}}%
\pgfpathlineto{\pgfqpoint{9.649995in}{1.722935in}}%
\pgfpathlineto{\pgfqpoint{9.658802in}{1.867054in}}%
\pgfpathlineto{\pgfqpoint{9.667609in}{2.045467in}}%
\pgfpathlineto{\pgfqpoint{9.676416in}{1.956246in}}%
\pgfpathlineto{\pgfqpoint{9.685222in}{1.983710in}}%
\pgfpathlineto{\pgfqpoint{9.694029in}{1.997427in}}%
\pgfpathlineto{\pgfqpoint{9.702836in}{1.983710in}}%
\pgfpathlineto{\pgfqpoint{9.711643in}{1.997427in}}%
\pgfpathlineto{\pgfqpoint{9.720450in}{2.086648in}}%
\pgfpathlineto{\pgfqpoint{9.729257in}{2.100365in}}%
\pgfpathlineto{\pgfqpoint{9.738063in}{2.072930in}}%
\pgfpathlineto{\pgfqpoint{9.746870in}{2.292524in}}%
\pgfpathlineto{\pgfqpoint{9.755677in}{2.169009in}}%
\pgfpathlineto{\pgfqpoint{9.764484in}{1.777834in}}%
\pgfpathlineto{\pgfqpoint{9.773291in}{2.100365in}}%
\pgfpathlineto{\pgfqpoint{9.782097in}{2.196445in}}%
\pgfpathlineto{\pgfqpoint{9.790904in}{2.024891in}}%
\pgfpathlineto{\pgfqpoint{9.799711in}{1.798410in}}%
\pgfpathlineto{\pgfqpoint{9.808518in}{1.688613in}}%
\pgfpathlineto{\pgfqpoint{9.817325in}{1.695472in}}%
\pgfpathlineto{\pgfqpoint{9.826132in}{1.681754in}}%
\pgfpathlineto{\pgfqpoint{9.834938in}{1.812128in}}%
\pgfpathlineto{\pgfqpoint{9.843745in}{2.011173in}}%
\pgfpathlineto{\pgfqpoint{9.852552in}{1.812128in}}%
\pgfpathlineto{\pgfqpoint{9.861359in}{1.963133in}}%
\pgfpathlineto{\pgfqpoint{9.870166in}{1.819015in}}%
\pgfpathlineto{\pgfqpoint{9.878972in}{1.791551in}}%
\pgfpathlineto{\pgfqpoint{9.887779in}{1.551353in}}%
\pgfpathlineto{\pgfqpoint{9.896586in}{1.894489in}}%
\pgfpathlineto{\pgfqpoint{9.905393in}{2.141546in}}%
\pgfpathlineto{\pgfqpoint{9.914200in}{2.203303in}}%
\pgfpathlineto{\pgfqpoint{9.923007in}{2.086648in}}%
\pgfpathlineto{\pgfqpoint{9.931813in}{2.450388in}}%
\pgfpathlineto{\pgfqpoint{9.940620in}{2.203303in}}%
\pgfpathlineto{\pgfqpoint{9.949427in}{2.223908in}}%
\pgfpathlineto{\pgfqpoint{9.949427in}{2.223908in}}%
\pgfusepath{stroke}%
\end{pgfscope}%
\begin{pgfscope}%
\pgfpathrectangle{\pgfqpoint{0.702268in}{0.521603in}}{\pgfqpoint{9.687500in}{4.235000in}}%
\pgfusepath{clip}%
\pgfsetrectcap%
\pgfsetroundjoin%
\pgfsetlinewidth{0.501875pt}%
\definecolor{currentstroke}{rgb}{0.501961,0.501961,0.501961}%
\pgfsetstrokecolor{currentstroke}%
\pgfsetstrokeopacity{0.250000}%
\pgfsetdash{}{0pt}%
\pgfpathmoveto{\pgfqpoint{1.142609in}{4.564103in}}%
\pgfpathlineto{\pgfqpoint{1.151416in}{3.706249in}}%
\pgfpathlineto{\pgfqpoint{1.160222in}{3.665068in}}%
\pgfpathlineto{\pgfqpoint{1.169029in}{3.239599in}}%
\pgfpathlineto{\pgfqpoint{1.177836in}{2.539581in}}%
\pgfpathlineto{\pgfqpoint{1.186643in}{2.223908in}}%
\pgfpathlineto{\pgfqpoint{1.195450in}{2.182727in}}%
\pgfpathlineto{\pgfqpoint{1.204257in}{2.244484in}}%
\pgfpathlineto{\pgfqpoint{1.213063in}{2.285665in}}%
\pgfpathlineto{\pgfqpoint{1.221870in}{2.024891in}}%
\pgfpathlineto{\pgfqpoint{1.230677in}{1.976851in}}%
\pgfpathlineto{\pgfqpoint{1.239484in}{1.997427in}}%
\pgfpathlineto{\pgfqpoint{1.248291in}{1.613110in}}%
\pgfpathlineto{\pgfqpoint{1.257097in}{1.709190in}}%
\pgfpathlineto{\pgfqpoint{1.265904in}{1.613110in}}%
\pgfpathlineto{\pgfqpoint{1.274711in}{1.963133in}}%
\pgfpathlineto{\pgfqpoint{1.283518in}{2.086648in}}%
\pgfpathlineto{\pgfqpoint{1.292325in}{2.093507in}}%
\pgfpathlineto{\pgfqpoint{1.301132in}{2.011173in}}%
\pgfpathlineto{\pgfqpoint{1.309938in}{1.729794in}}%
\pgfpathlineto{\pgfqpoint{1.318745in}{1.722935in}}%
\pgfpathlineto{\pgfqpoint{1.327552in}{1.764116in}}%
\pgfpathlineto{\pgfqpoint{1.336359in}{1.640574in}}%
\pgfpathlineto{\pgfqpoint{1.345166in}{1.860167in}}%
\pgfpathlineto{\pgfqpoint{1.353972in}{1.791551in}}%
\pgfpathlineto{\pgfqpoint{1.371586in}{1.544494in}}%
\pgfpathlineto{\pgfqpoint{1.380393in}{1.530777in}}%
\pgfpathlineto{\pgfqpoint{1.389200in}{1.846450in}}%
\pgfpathlineto{\pgfqpoint{1.398007in}{1.880772in}}%
\pgfpathlineto{\pgfqpoint{1.406813in}{1.613110in}}%
\pgfpathlineto{\pgfqpoint{1.415620in}{1.654291in}}%
\pgfpathlineto{\pgfqpoint{1.424427in}{1.633715in}}%
\pgfpathlineto{\pgfqpoint{1.433234in}{1.736653in}}%
\pgfpathlineto{\pgfqpoint{1.442041in}{1.743512in}}%
\pgfpathlineto{\pgfqpoint{1.450847in}{1.839591in}}%
\pgfpathlineto{\pgfqpoint{1.468461in}{2.079789in}}%
\pgfpathlineto{\pgfqpoint{1.477268in}{2.066071in}}%
\pgfpathlineto{\pgfqpoint{1.486075in}{1.901348in}}%
\pgfpathlineto{\pgfqpoint{1.494882in}{1.647432in}}%
\pgfpathlineto{\pgfqpoint{1.503688in}{1.825873in}}%
\pgfpathlineto{\pgfqpoint{1.512495in}{2.169009in}}%
\pgfpathlineto{\pgfqpoint{1.521302in}{2.120970in}}%
\pgfpathlineto{\pgfqpoint{1.530109in}{2.498428in}}%
\pgfpathlineto{\pgfqpoint{1.538916in}{2.230767in}}%
\pgfpathlineto{\pgfqpoint{1.547722in}{2.313128in}}%
\pgfpathlineto{\pgfqpoint{1.556529in}{2.292524in}}%
\pgfpathlineto{\pgfqpoint{1.565336in}{2.306269in}}%
\pgfpathlineto{\pgfqpoint{1.574143in}{2.093507in}}%
\pgfpathlineto{\pgfqpoint{1.582950in}{2.416066in}}%
\pgfpathlineto{\pgfqpoint{1.591757in}{2.072930in}}%
\pgfpathlineto{\pgfqpoint{1.600563in}{2.196445in}}%
\pgfpathlineto{\pgfqpoint{1.609370in}{1.832732in}}%
\pgfpathlineto{\pgfqpoint{1.618177in}{1.613110in}}%
\pgfpathlineto{\pgfqpoint{1.626984in}{1.619997in}}%
\pgfpathlineto{\pgfqpoint{1.635791in}{1.496455in}}%
\pgfpathlineto{\pgfqpoint{1.644597in}{1.496455in}}%
\pgfpathlineto{\pgfqpoint{1.653404in}{1.215076in}}%
\pgfpathlineto{\pgfqpoint{1.662211in}{1.791551in}}%
\pgfpathlineto{\pgfqpoint{1.671018in}{1.606252in}}%
\pgfpathlineto{\pgfqpoint{1.679825in}{1.640574in}}%
\pgfpathlineto{\pgfqpoint{1.688632in}{1.489596in}}%
\pgfpathlineto{\pgfqpoint{1.697438in}{1.654291in}}%
\pgfpathlineto{\pgfqpoint{1.706245in}{1.757229in}}%
\pgfpathlineto{\pgfqpoint{1.715052in}{1.448415in}}%
\pgfpathlineto{\pgfqpoint{1.723859in}{1.324873in}}%
\pgfpathlineto{\pgfqpoint{1.732666in}{1.173895in}}%
\pgfpathlineto{\pgfqpoint{1.741472in}{1.496455in}}%
\pgfpathlineto{\pgfqpoint{1.750279in}{1.551353in}}%
\pgfpathlineto{\pgfqpoint{1.759086in}{1.551353in}}%
\pgfpathlineto{\pgfqpoint{1.767893in}{1.263115in}}%
\pgfpathlineto{\pgfqpoint{1.776700in}{1.420952in}}%
\pgfpathlineto{\pgfqpoint{1.785507in}{1.668037in}}%
\pgfpathlineto{\pgfqpoint{1.794313in}{1.729794in}}%
\pgfpathlineto{\pgfqpoint{1.803120in}{1.757229in}}%
\pgfpathlineto{\pgfqpoint{1.811927in}{2.011173in}}%
\pgfpathlineto{\pgfqpoint{1.820734in}{1.867054in}}%
\pgfpathlineto{\pgfqpoint{1.829541in}{1.764116in}}%
\pgfpathlineto{\pgfqpoint{1.838347in}{1.709190in}}%
\pgfpathlineto{\pgfqpoint{1.847154in}{1.619997in}}%
\pgfpathlineto{\pgfqpoint{1.855961in}{1.805269in}}%
\pgfpathlineto{\pgfqpoint{1.864768in}{1.839591in}}%
\pgfpathlineto{\pgfqpoint{1.873575in}{2.120970in}}%
\pgfpathlineto{\pgfqpoint{1.882382in}{2.182727in}}%
\pgfpathlineto{\pgfqpoint{1.891188in}{1.983710in}}%
\pgfpathlineto{\pgfqpoint{1.899995in}{1.757229in}}%
\pgfpathlineto{\pgfqpoint{1.908802in}{1.736653in}}%
\pgfpathlineto{\pgfqpoint{1.917609in}{2.120970in}}%
\pgfpathlineto{\pgfqpoint{1.926416in}{2.182727in}}%
\pgfpathlineto{\pgfqpoint{1.935222in}{2.086648in}}%
\pgfpathlineto{\pgfqpoint{1.944029in}{2.237625in}}%
\pgfpathlineto{\pgfqpoint{1.952836in}{2.018032in}}%
\pgfpathlineto{\pgfqpoint{1.961643in}{2.203303in}}%
\pgfpathlineto{\pgfqpoint{1.970450in}{1.825873in}}%
\pgfpathlineto{\pgfqpoint{1.979257in}{2.059213in}}%
\pgfpathlineto{\pgfqpoint{1.988063in}{1.819015in}}%
\pgfpathlineto{\pgfqpoint{1.996870in}{1.832732in}}%
\pgfpathlineto{\pgfqpoint{2.014484in}{2.388603in}}%
\pgfpathlineto{\pgfqpoint{2.023291in}{1.880772in}}%
\pgfpathlineto{\pgfqpoint{2.032097in}{1.661150in}}%
\pgfpathlineto{\pgfqpoint{2.040904in}{1.860167in}}%
\pgfpathlineto{\pgfqpoint{2.049711in}{1.997427in}}%
\pgfpathlineto{\pgfqpoint{2.058518in}{1.832732in}}%
\pgfpathlineto{\pgfqpoint{2.076132in}{2.004286in}}%
\pgfpathlineto{\pgfqpoint{2.084938in}{2.532722in}}%
\pgfpathlineto{\pgfqpoint{2.093745in}{1.784692in}}%
\pgfpathlineto{\pgfqpoint{2.102552in}{2.066071in}}%
\pgfpathlineto{\pgfqpoint{2.111359in}{1.873913in}}%
\pgfpathlineto{\pgfqpoint{2.120166in}{1.880772in}}%
\pgfpathlineto{\pgfqpoint{2.128972in}{1.798410in}}%
\pgfpathlineto{\pgfqpoint{2.137779in}{1.963133in}}%
\pgfpathlineto{\pgfqpoint{2.146586in}{1.832732in}}%
\pgfpathlineto{\pgfqpoint{2.155393in}{1.558212in}}%
\pgfpathlineto{\pgfqpoint{2.164200in}{1.462133in}}%
\pgfpathlineto{\pgfqpoint{2.173007in}{1.853308in}}%
\pgfpathlineto{\pgfqpoint{2.181813in}{2.155292in}}%
\pgfpathlineto{\pgfqpoint{2.190620in}{2.299383in}}%
\pgfpathlineto{\pgfqpoint{2.208234in}{1.860167in}}%
\pgfpathlineto{\pgfqpoint{2.217041in}{1.716076in}}%
\pgfpathlineto{\pgfqpoint{2.225847in}{1.770975in}}%
\pgfpathlineto{\pgfqpoint{2.234654in}{1.757229in}}%
\pgfpathlineto{\pgfqpoint{2.243461in}{1.468992in}}%
\pgfpathlineto{\pgfqpoint{2.252268in}{1.832732in}}%
\pgfpathlineto{\pgfqpoint{2.261075in}{1.942529in}}%
\pgfpathlineto{\pgfqpoint{2.269882in}{2.018032in}}%
\pgfpathlineto{\pgfqpoint{2.278688in}{1.956246in}}%
\pgfpathlineto{\pgfqpoint{2.287495in}{1.990569in}}%
\pgfpathlineto{\pgfqpoint{2.296302in}{2.011173in}}%
\pgfpathlineto{\pgfqpoint{2.305109in}{2.196445in}}%
\pgfpathlineto{\pgfqpoint{2.313916in}{2.086648in}}%
\pgfpathlineto{\pgfqpoint{2.322722in}{1.805269in}}%
\pgfpathlineto{\pgfqpoint{2.331529in}{1.619997in}}%
\pgfpathlineto{\pgfqpoint{2.340336in}{1.544494in}}%
\pgfpathlineto{\pgfqpoint{2.349143in}{1.661150in}}%
\pgfpathlineto{\pgfqpoint{2.357950in}{1.551353in}}%
\pgfpathlineto{\pgfqpoint{2.366757in}{1.462133in}}%
\pgfpathlineto{\pgfqpoint{2.375563in}{1.352336in}}%
\pgfpathlineto{\pgfqpoint{2.384370in}{1.585675in}}%
\pgfpathlineto{\pgfqpoint{2.393177in}{1.503314in}}%
\pgfpathlineto{\pgfqpoint{2.401984in}{1.867054in}}%
\pgfpathlineto{\pgfqpoint{2.410791in}{1.825873in}}%
\pgfpathlineto{\pgfqpoint{2.419597in}{1.853308in}}%
\pgfpathlineto{\pgfqpoint{2.428404in}{1.839591in}}%
\pgfpathlineto{\pgfqpoint{2.437211in}{1.695472in}}%
\pgfpathlineto{\pgfqpoint{2.446018in}{1.928811in}}%
\pgfpathlineto{\pgfqpoint{2.454825in}{1.880772in}}%
\pgfpathlineto{\pgfqpoint{2.463632in}{1.956246in}}%
\pgfpathlineto{\pgfqpoint{2.472438in}{1.887630in}}%
\pgfpathlineto{\pgfqpoint{2.481245in}{1.997427in}}%
\pgfpathlineto{\pgfqpoint{2.490052in}{2.079789in}}%
\pgfpathlineto{\pgfqpoint{2.498859in}{2.422925in}}%
\pgfpathlineto{\pgfqpoint{2.507666in}{2.532722in}}%
\pgfpathlineto{\pgfqpoint{2.516472in}{2.299383in}}%
\pgfpathlineto{\pgfqpoint{2.525279in}{1.990569in}}%
\pgfpathlineto{\pgfqpoint{2.534086in}{1.736653in}}%
\pgfpathlineto{\pgfqpoint{2.542893in}{1.736653in}}%
\pgfpathlineto{\pgfqpoint{2.551700in}{1.928811in}}%
\pgfpathlineto{\pgfqpoint{2.560507in}{2.182727in}}%
\pgfpathlineto{\pgfqpoint{2.569313in}{1.997427in}}%
\pgfpathlineto{\pgfqpoint{2.578120in}{1.770975in}}%
\pgfpathlineto{\pgfqpoint{2.586927in}{1.853308in}}%
\pgfpathlineto{\pgfqpoint{2.595734in}{1.743512in}}%
\pgfpathlineto{\pgfqpoint{2.604541in}{1.729794in}}%
\pgfpathlineto{\pgfqpoint{2.613347in}{1.915094in}}%
\pgfpathlineto{\pgfqpoint{2.622154in}{2.203303in}}%
\pgfpathlineto{\pgfqpoint{2.630961in}{2.635660in}}%
\pgfpathlineto{\pgfqpoint{2.639768in}{2.203303in}}%
\pgfpathlineto{\pgfqpoint{2.648575in}{2.072930in}}%
\pgfpathlineto{\pgfqpoint{2.657382in}{1.764116in}}%
\pgfpathlineto{\pgfqpoint{2.666188in}{1.777834in}}%
\pgfpathlineto{\pgfqpoint{2.674995in}{2.086648in}}%
\pgfpathlineto{\pgfqpoint{2.683802in}{2.203303in}}%
\pgfpathlineto{\pgfqpoint{2.692609in}{2.127829in}}%
\pgfpathlineto{\pgfqpoint{2.701416in}{2.100365in}}%
\pgfpathlineto{\pgfqpoint{2.710222in}{1.894489in}}%
\pgfpathlineto{\pgfqpoint{2.719029in}{1.578816in}}%
\pgfpathlineto{\pgfqpoint{2.727836in}{1.777834in}}%
\pgfpathlineto{\pgfqpoint{2.736643in}{2.100365in}}%
\pgfpathlineto{\pgfqpoint{2.745450in}{1.969992in}}%
\pgfpathlineto{\pgfqpoint{2.754257in}{2.066071in}}%
\pgfpathlineto{\pgfqpoint{2.763063in}{2.031749in}}%
\pgfpathlineto{\pgfqpoint{2.771870in}{2.011173in}}%
\pgfpathlineto{\pgfqpoint{2.780677in}{2.313128in}}%
\pgfpathlineto{\pgfqpoint{2.798291in}{2.093507in}}%
\pgfpathlineto{\pgfqpoint{2.807097in}{1.867054in}}%
\pgfpathlineto{\pgfqpoint{2.815904in}{1.887630in}}%
\pgfpathlineto{\pgfqpoint{2.824711in}{2.162151in}}%
\pgfpathlineto{\pgfqpoint{2.833518in}{1.798410in}}%
\pgfpathlineto{\pgfqpoint{2.842325in}{1.908207in}}%
\pgfpathlineto{\pgfqpoint{2.851132in}{1.812128in}}%
\pgfpathlineto{\pgfqpoint{2.859938in}{2.196445in}}%
\pgfpathlineto{\pgfqpoint{2.868745in}{2.244484in}}%
\pgfpathlineto{\pgfqpoint{2.877552in}{2.011173in}}%
\pgfpathlineto{\pgfqpoint{2.886359in}{2.292524in}}%
\pgfpathlineto{\pgfqpoint{2.895166in}{2.368027in}}%
\pgfpathlineto{\pgfqpoint{2.903972in}{2.306269in}}%
\pgfpathlineto{\pgfqpoint{2.912779in}{2.333705in}}%
\pgfpathlineto{\pgfqpoint{2.921586in}{2.319987in}}%
\pgfpathlineto{\pgfqpoint{2.930393in}{2.319987in}}%
\pgfpathlineto{\pgfqpoint{2.939200in}{2.210190in}}%
\pgfpathlineto{\pgfqpoint{2.948007in}{1.853308in}}%
\pgfpathlineto{\pgfqpoint{2.956813in}{1.571958in}}%
\pgfpathlineto{\pgfqpoint{2.965620in}{1.503314in}}%
\pgfpathlineto{\pgfqpoint{2.974427in}{1.585675in}}%
\pgfpathlineto{\pgfqpoint{2.983234in}{2.223908in}}%
\pgfpathlineto{\pgfqpoint{2.992041in}{2.265089in}}%
\pgfpathlineto{\pgfqpoint{3.009654in}{2.525863in}}%
\pgfpathlineto{\pgfqpoint{3.018461in}{2.251343in}}%
\pgfpathlineto{\pgfqpoint{3.027268in}{1.736653in}}%
\pgfpathlineto{\pgfqpoint{3.036075in}{2.134687in}}%
\pgfpathlineto{\pgfqpoint{3.044882in}{1.846450in}}%
\pgfpathlineto{\pgfqpoint{3.053688in}{1.777834in}}%
\pgfpathlineto{\pgfqpoint{3.062495in}{1.647432in}}%
\pgfpathlineto{\pgfqpoint{3.071302in}{1.894489in}}%
\pgfpathlineto{\pgfqpoint{3.080109in}{2.230767in}}%
\pgfpathlineto{\pgfqpoint{3.088916in}{2.031749in}}%
\pgfpathlineto{\pgfqpoint{3.097722in}{1.784692in}}%
\pgfpathlineto{\pgfqpoint{3.106529in}{1.757229in}}%
\pgfpathlineto{\pgfqpoint{3.115336in}{2.011173in}}%
\pgfpathlineto{\pgfqpoint{3.124143in}{1.894489in}}%
\pgfpathlineto{\pgfqpoint{3.132950in}{1.908207in}}%
\pgfpathlineto{\pgfqpoint{3.141757in}{2.210190in}}%
\pgfpathlineto{\pgfqpoint{3.150563in}{2.340563in}}%
\pgfpathlineto{\pgfqpoint{3.159370in}{2.436643in}}%
\pgfpathlineto{\pgfqpoint{3.168177in}{2.594507in}}%
\pgfpathlineto{\pgfqpoint{3.176984in}{2.381744in}}%
\pgfpathlineto{\pgfqpoint{3.185791in}{2.436643in}}%
\pgfpathlineto{\pgfqpoint{3.194597in}{2.319987in}}%
\pgfpathlineto{\pgfqpoint{3.203404in}{2.045467in}}%
\pgfpathlineto{\pgfqpoint{3.212211in}{2.120970in}}%
\pgfpathlineto{\pgfqpoint{3.221018in}{2.079789in}}%
\pgfpathlineto{\pgfqpoint{3.229825in}{2.285665in}}%
\pgfpathlineto{\pgfqpoint{3.238632in}{1.942529in}}%
\pgfpathlineto{\pgfqpoint{3.247438in}{1.873913in}}%
\pgfpathlineto{\pgfqpoint{3.256245in}{2.079789in}}%
\pgfpathlineto{\pgfqpoint{3.265052in}{2.223908in}}%
\pgfpathlineto{\pgfqpoint{3.273859in}{2.162151in}}%
\pgfpathlineto{\pgfqpoint{3.282666in}{1.832732in}}%
\pgfpathlineto{\pgfqpoint{3.291472in}{1.647432in}}%
\pgfpathlineto{\pgfqpoint{3.300279in}{1.530777in}}%
\pgfpathlineto{\pgfqpoint{3.309086in}{1.654291in}}%
\pgfpathlineto{\pgfqpoint{3.317893in}{1.949388in}}%
\pgfpathlineto{\pgfqpoint{3.326700in}{1.921953in}}%
\pgfpathlineto{\pgfqpoint{3.335507in}{2.217049in}}%
\pgfpathlineto{\pgfqpoint{3.344313in}{1.983710in}}%
\pgfpathlineto{\pgfqpoint{3.353120in}{2.319987in}}%
\pgfpathlineto{\pgfqpoint{3.361927in}{2.086648in}}%
\pgfpathlineto{\pgfqpoint{3.370734in}{2.038608in}}%
\pgfpathlineto{\pgfqpoint{3.379541in}{1.839591in}}%
\pgfpathlineto{\pgfqpoint{3.388347in}{1.798410in}}%
\pgfpathlineto{\pgfqpoint{3.397154in}{1.722935in}}%
\pgfpathlineto{\pgfqpoint{3.405961in}{1.537636in}}%
\pgfpathlineto{\pgfqpoint{3.414768in}{1.661150in}}%
\pgfpathlineto{\pgfqpoint{3.423575in}{2.059213in}}%
\pgfpathlineto{\pgfqpoint{3.432382in}{1.894489in}}%
\pgfpathlineto{\pgfqpoint{3.441188in}{1.606252in}}%
\pgfpathlineto{\pgfqpoint{3.449995in}{1.510172in}}%
\pgfpathlineto{\pgfqpoint{3.458802in}{1.489596in}}%
\pgfpathlineto{\pgfqpoint{3.467609in}{1.681754in}}%
\pgfpathlineto{\pgfqpoint{3.476416in}{1.839591in}}%
\pgfpathlineto{\pgfqpoint{3.485222in}{1.441556in}}%
\pgfpathlineto{\pgfqpoint{3.494029in}{1.565071in}}%
\pgfpathlineto{\pgfqpoint{3.502836in}{1.613110in}}%
\pgfpathlineto{\pgfqpoint{3.511643in}{1.722935in}}%
\pgfpathlineto{\pgfqpoint{3.520450in}{1.887630in}}%
\pgfpathlineto{\pgfqpoint{3.529257in}{1.860167in}}%
\pgfpathlineto{\pgfqpoint{3.538063in}{1.626856in}}%
\pgfpathlineto{\pgfqpoint{3.546870in}{1.510172in}}%
\pgfpathlineto{\pgfqpoint{3.555677in}{1.455274in}}%
\pgfpathlineto{\pgfqpoint{3.564484in}{1.729794in}}%
\pgfpathlineto{\pgfqpoint{3.573291in}{1.736653in}}%
\pgfpathlineto{\pgfqpoint{3.582097in}{1.846450in}}%
\pgfpathlineto{\pgfqpoint{3.590904in}{1.613110in}}%
\pgfpathlineto{\pgfqpoint{3.599711in}{1.674896in}}%
\pgfpathlineto{\pgfqpoint{3.608518in}{1.702331in}}%
\pgfpathlineto{\pgfqpoint{3.617325in}{1.819015in}}%
\pgfpathlineto{\pgfqpoint{3.626132in}{1.661150in}}%
\pgfpathlineto{\pgfqpoint{3.634938in}{1.585675in}}%
\pgfpathlineto{\pgfqpoint{3.643745in}{1.716076in}}%
\pgfpathlineto{\pgfqpoint{3.652552in}{1.743512in}}%
\pgfpathlineto{\pgfqpoint{3.661359in}{2.045467in}}%
\pgfpathlineto{\pgfqpoint{3.670166in}{2.134687in}}%
\pgfpathlineto{\pgfqpoint{3.678972in}{2.553326in}}%
\pgfpathlineto{\pgfqpoint{3.687779in}{2.203303in}}%
\pgfpathlineto{\pgfqpoint{3.696586in}{2.175868in}}%
\pgfpathlineto{\pgfqpoint{3.705393in}{2.464106in}}%
\pgfpathlineto{\pgfqpoint{3.714200in}{2.169009in}}%
\pgfpathlineto{\pgfqpoint{3.723007in}{2.470965in}}%
\pgfpathlineto{\pgfqpoint{3.731813in}{2.368027in}}%
\pgfpathlineto{\pgfqpoint{3.749427in}{2.093507in}}%
\pgfpathlineto{\pgfqpoint{3.758234in}{1.983710in}}%
\pgfpathlineto{\pgfqpoint{3.767041in}{2.271947in}}%
\pgfpathlineto{\pgfqpoint{3.775847in}{2.210190in}}%
\pgfpathlineto{\pgfqpoint{3.784654in}{2.340563in}}%
\pgfpathlineto{\pgfqpoint{3.793461in}{2.443501in}}%
\pgfpathlineto{\pgfqpoint{3.802268in}{2.114111in}}%
\pgfpathlineto{\pgfqpoint{3.811075in}{2.059213in}}%
\pgfpathlineto{\pgfqpoint{3.819882in}{2.079789in}}%
\pgfpathlineto{\pgfqpoint{3.828688in}{2.079789in}}%
\pgfpathlineto{\pgfqpoint{3.837495in}{1.551353in}}%
\pgfpathlineto{\pgfqpoint{3.846302in}{1.716076in}}%
\pgfpathlineto{\pgfqpoint{3.855109in}{1.757229in}}%
\pgfpathlineto{\pgfqpoint{3.863916in}{1.613110in}}%
\pgfpathlineto{\pgfqpoint{3.872722in}{1.606252in}}%
\pgfpathlineto{\pgfqpoint{3.881529in}{1.517031in}}%
\pgfpathlineto{\pgfqpoint{3.890336in}{1.558212in}}%
\pgfpathlineto{\pgfqpoint{3.899143in}{1.722935in}}%
\pgfpathlineto{\pgfqpoint{3.907950in}{1.578816in}}%
\pgfpathlineto{\pgfqpoint{3.916757in}{1.571958in}}%
\pgfpathlineto{\pgfqpoint{3.925563in}{1.674896in}}%
\pgfpathlineto{\pgfqpoint{3.934370in}{2.141546in}}%
\pgfpathlineto{\pgfqpoint{3.943177in}{2.182727in}}%
\pgfpathlineto{\pgfqpoint{3.951984in}{2.004286in}}%
\pgfpathlineto{\pgfqpoint{3.960791in}{2.285665in}}%
\pgfpathlineto{\pgfqpoint{3.969597in}{2.258230in}}%
\pgfpathlineto{\pgfqpoint{3.978404in}{2.573903in}}%
\pgfpathlineto{\pgfqpoint{3.987211in}{2.347422in}}%
\pgfpathlineto{\pgfqpoint{3.996018in}{2.299383in}}%
\pgfpathlineto{\pgfqpoint{4.004825in}{2.498428in}}%
\pgfpathlineto{\pgfqpoint{4.013632in}{2.409207in}}%
\pgfpathlineto{\pgfqpoint{4.022438in}{2.175868in}}%
\pgfpathlineto{\pgfqpoint{4.031245in}{2.189586in}}%
\pgfpathlineto{\pgfqpoint{4.040052in}{2.196445in}}%
\pgfpathlineto{\pgfqpoint{4.048859in}{2.354309in}}%
\pgfpathlineto{\pgfqpoint{4.057666in}{2.169009in}}%
\pgfpathlineto{\pgfqpoint{4.066472in}{2.066071in}}%
\pgfpathlineto{\pgfqpoint{4.075279in}{2.086648in}}%
\pgfpathlineto{\pgfqpoint{4.084086in}{1.880772in}}%
\pgfpathlineto{\pgfqpoint{4.092893in}{1.832732in}}%
\pgfpathlineto{\pgfqpoint{4.101700in}{1.585675in}}%
\pgfpathlineto{\pgfqpoint{4.110507in}{1.825873in}}%
\pgfpathlineto{\pgfqpoint{4.119313in}{1.915094in}}%
\pgfpathlineto{\pgfqpoint{4.128120in}{2.285665in}}%
\pgfpathlineto{\pgfqpoint{4.136927in}{2.278806in}}%
\pgfpathlineto{\pgfqpoint{4.145734in}{2.333705in}}%
\pgfpathlineto{\pgfqpoint{4.154541in}{2.127829in}}%
\pgfpathlineto{\pgfqpoint{4.163347in}{1.997427in}}%
\pgfpathlineto{\pgfqpoint{4.172154in}{1.963133in}}%
\pgfpathlineto{\pgfqpoint{4.180961in}{2.134687in}}%
\pgfpathlineto{\pgfqpoint{4.189768in}{2.127829in}}%
\pgfpathlineto{\pgfqpoint{4.198575in}{1.867054in}}%
\pgfpathlineto{\pgfqpoint{4.207382in}{2.031749in}}%
\pgfpathlineto{\pgfqpoint{4.216188in}{2.251343in}}%
\pgfpathlineto{\pgfqpoint{4.224995in}{2.354309in}}%
\pgfpathlineto{\pgfqpoint{4.233802in}{2.223908in}}%
\pgfpathlineto{\pgfqpoint{4.251416in}{2.436643in}}%
\pgfpathlineto{\pgfqpoint{4.260222in}{2.148405in}}%
\pgfpathlineto{\pgfqpoint{4.269029in}{1.990569in}}%
\pgfpathlineto{\pgfqpoint{4.277836in}{2.217049in}}%
\pgfpathlineto{\pgfqpoint{4.286643in}{2.477823in}}%
\pgfpathlineto{\pgfqpoint{4.295450in}{2.546468in}}%
\pgfpathlineto{\pgfqpoint{4.304257in}{2.663123in}}%
\pgfpathlineto{\pgfqpoint{4.313063in}{2.601366in}}%
\pgfpathlineto{\pgfqpoint{4.321870in}{2.066071in}}%
\pgfpathlineto{\pgfqpoint{4.330677in}{1.935670in}}%
\pgfpathlineto{\pgfqpoint{4.339484in}{2.120970in}}%
\pgfpathlineto{\pgfqpoint{4.348291in}{2.374885in}}%
\pgfpathlineto{\pgfqpoint{4.357097in}{2.148405in}}%
\pgfpathlineto{\pgfqpoint{4.365904in}{2.210190in}}%
\pgfpathlineto{\pgfqpoint{4.374711in}{2.175868in}}%
\pgfpathlineto{\pgfqpoint{4.383518in}{2.066071in}}%
\pgfpathlineto{\pgfqpoint{4.392325in}{1.846450in}}%
\pgfpathlineto{\pgfqpoint{4.401132in}{2.155292in}}%
\pgfpathlineto{\pgfqpoint{4.409938in}{2.066071in}}%
\pgfpathlineto{\pgfqpoint{4.418745in}{2.175868in}}%
\pgfpathlineto{\pgfqpoint{4.427552in}{2.223908in}}%
\pgfpathlineto{\pgfqpoint{4.436359in}{2.333705in}}%
\pgfpathlineto{\pgfqpoint{4.445166in}{1.956246in}}%
\pgfpathlineto{\pgfqpoint{4.453972in}{1.846450in}}%
\pgfpathlineto{\pgfqpoint{4.462779in}{2.217049in}}%
\pgfpathlineto{\pgfqpoint{4.471586in}{2.134687in}}%
\pgfpathlineto{\pgfqpoint{4.480393in}{1.764116in}}%
\pgfpathlineto{\pgfqpoint{4.489200in}{1.722935in}}%
\pgfpathlineto{\pgfqpoint{4.498007in}{1.921953in}}%
\pgfpathlineto{\pgfqpoint{4.506813in}{1.846450in}}%
\pgfpathlineto{\pgfqpoint{4.515620in}{2.072930in}}%
\pgfpathlineto{\pgfqpoint{4.524427in}{2.175868in}}%
\pgfpathlineto{\pgfqpoint{4.533234in}{2.182727in}}%
\pgfpathlineto{\pgfqpoint{4.542041in}{2.340563in}}%
\pgfpathlineto{\pgfqpoint{4.550847in}{1.990569in}}%
\pgfpathlineto{\pgfqpoint{4.559654in}{1.537636in}}%
\pgfpathlineto{\pgfqpoint{4.568461in}{1.887630in}}%
\pgfpathlineto{\pgfqpoint{4.577268in}{1.736653in}}%
\pgfpathlineto{\pgfqpoint{4.586075in}{1.805269in}}%
\pgfpathlineto{\pgfqpoint{4.594882in}{1.736653in}}%
\pgfpathlineto{\pgfqpoint{4.603688in}{1.764116in}}%
\pgfpathlineto{\pgfqpoint{4.612495in}{1.585675in}}%
\pgfpathlineto{\pgfqpoint{4.621302in}{2.031749in}}%
\pgfpathlineto{\pgfqpoint{4.630109in}{2.210190in}}%
\pgfpathlineto{\pgfqpoint{4.638916in}{2.251343in}}%
\pgfpathlineto{\pgfqpoint{4.647722in}{2.217049in}}%
\pgfpathlineto{\pgfqpoint{4.656529in}{1.935670in}}%
\pgfpathlineto{\pgfqpoint{4.665336in}{2.038608in}}%
\pgfpathlineto{\pgfqpoint{4.674143in}{2.230767in}}%
\pgfpathlineto{\pgfqpoint{4.682950in}{1.990569in}}%
\pgfpathlineto{\pgfqpoint{4.691757in}{2.031749in}}%
\pgfpathlineto{\pgfqpoint{4.700563in}{2.189586in}}%
\pgfpathlineto{\pgfqpoint{4.709370in}{1.812128in}}%
\pgfpathlineto{\pgfqpoint{4.718177in}{1.702331in}}%
\pgfpathlineto{\pgfqpoint{4.726984in}{1.825873in}}%
\pgfpathlineto{\pgfqpoint{4.735791in}{1.661150in}}%
\pgfpathlineto{\pgfqpoint{4.744597in}{1.901348in}}%
\pgfpathlineto{\pgfqpoint{4.753404in}{1.770975in}}%
\pgfpathlineto{\pgfqpoint{4.762211in}{2.093507in}}%
\pgfpathlineto{\pgfqpoint{4.771018in}{2.223908in}}%
\pgfpathlineto{\pgfqpoint{4.779825in}{2.299383in}}%
\pgfpathlineto{\pgfqpoint{4.788632in}{1.880772in}}%
\pgfpathlineto{\pgfqpoint{4.797438in}{1.571958in}}%
\pgfpathlineto{\pgfqpoint{4.806245in}{1.420952in}}%
\pgfpathlineto{\pgfqpoint{4.815052in}{1.441556in}}%
\pgfpathlineto{\pgfqpoint{4.823859in}{1.599393in}}%
\pgfpathlineto{\pgfqpoint{4.832666in}{1.544494in}}%
\pgfpathlineto{\pgfqpoint{4.841472in}{1.839591in}}%
\pgfpathlineto{\pgfqpoint{4.850279in}{1.729794in}}%
\pgfpathlineto{\pgfqpoint{4.859086in}{1.997427in}}%
\pgfpathlineto{\pgfqpoint{4.867893in}{2.114111in}}%
\pgfpathlineto{\pgfqpoint{4.876700in}{2.031749in}}%
\pgfpathlineto{\pgfqpoint{4.885507in}{1.839591in}}%
\pgfpathlineto{\pgfqpoint{4.894313in}{2.278806in}}%
\pgfpathlineto{\pgfqpoint{4.903120in}{2.203303in}}%
\pgfpathlineto{\pgfqpoint{4.911927in}{2.285665in}}%
\pgfpathlineto{\pgfqpoint{4.920734in}{2.182727in}}%
\pgfpathlineto{\pgfqpoint{4.929541in}{2.368027in}}%
\pgfpathlineto{\pgfqpoint{4.938347in}{2.429784in}}%
\pgfpathlineto{\pgfqpoint{4.947154in}{2.573903in}}%
\pgfpathlineto{\pgfqpoint{4.955961in}{1.825873in}}%
\pgfpathlineto{\pgfqpoint{4.964768in}{1.613110in}}%
\pgfpathlineto{\pgfqpoint{4.973575in}{1.537636in}}%
\pgfpathlineto{\pgfqpoint{4.982382in}{1.729794in}}%
\pgfpathlineto{\pgfqpoint{4.991188in}{1.805269in}}%
\pgfpathlineto{\pgfqpoint{4.999995in}{2.141546in}}%
\pgfpathlineto{\pgfqpoint{5.008802in}{2.265089in}}%
\pgfpathlineto{\pgfqpoint{5.017609in}{2.175868in}}%
\pgfpathlineto{\pgfqpoint{5.026416in}{2.340563in}}%
\pgfpathlineto{\pgfqpoint{5.035222in}{2.244484in}}%
\pgfpathlineto{\pgfqpoint{5.044029in}{1.976851in}}%
\pgfpathlineto{\pgfqpoint{5.052836in}{1.894489in}}%
\pgfpathlineto{\pgfqpoint{5.061643in}{1.722935in}}%
\pgfpathlineto{\pgfqpoint{5.070450in}{1.434698in}}%
\pgfpathlineto{\pgfqpoint{5.079257in}{1.716076in}}%
\pgfpathlineto{\pgfqpoint{5.088063in}{1.613110in}}%
\pgfpathlineto{\pgfqpoint{5.096870in}{2.484682in}}%
\pgfpathlineto{\pgfqpoint{5.105677in}{2.210190in}}%
\pgfpathlineto{\pgfqpoint{5.114484in}{2.718022in}}%
\pgfpathlineto{\pgfqpoint{5.123291in}{2.285665in}}%
\pgfpathlineto{\pgfqpoint{5.132097in}{2.244484in}}%
\pgfpathlineto{\pgfqpoint{5.149711in}{1.963133in}}%
\pgfpathlineto{\pgfqpoint{5.158518in}{1.949388in}}%
\pgfpathlineto{\pgfqpoint{5.167325in}{2.052326in}}%
\pgfpathlineto{\pgfqpoint{5.176132in}{1.921953in}}%
\pgfpathlineto{\pgfqpoint{5.184938in}{1.750370in}}%
\pgfpathlineto{\pgfqpoint{5.193745in}{2.011173in}}%
\pgfpathlineto{\pgfqpoint{5.202552in}{2.059213in}}%
\pgfpathlineto{\pgfqpoint{5.211359in}{2.196445in}}%
\pgfpathlineto{\pgfqpoint{5.220166in}{2.141546in}}%
\pgfpathlineto{\pgfqpoint{5.228972in}{1.791551in}}%
\pgfpathlineto{\pgfqpoint{5.237779in}{2.004286in}}%
\pgfpathlineto{\pgfqpoint{5.246586in}{2.340563in}}%
\pgfpathlineto{\pgfqpoint{5.255393in}{2.271947in}}%
\pgfpathlineto{\pgfqpoint{5.264200in}{1.503314in}}%
\pgfpathlineto{\pgfqpoint{5.273007in}{1.448415in}}%
\pgfpathlineto{\pgfqpoint{5.281813in}{1.523918in}}%
\pgfpathlineto{\pgfqpoint{5.290620in}{1.613110in}}%
\pgfpathlineto{\pgfqpoint{5.299427in}{1.825873in}}%
\pgfpathlineto{\pgfqpoint{5.308234in}{1.496455in}}%
\pgfpathlineto{\pgfqpoint{5.317041in}{1.434698in}}%
\pgfpathlineto{\pgfqpoint{5.325847in}{1.743512in}}%
\pgfpathlineto{\pgfqpoint{5.334654in}{1.668037in}}%
\pgfpathlineto{\pgfqpoint{5.352268in}{2.134687in}}%
\pgfpathlineto{\pgfqpoint{5.361075in}{2.519004in}}%
\pgfpathlineto{\pgfqpoint{5.369882in}{2.512146in}}%
\pgfpathlineto{\pgfqpoint{5.378688in}{2.416066in}}%
\pgfpathlineto{\pgfqpoint{5.387495in}{2.374885in}}%
\pgfpathlineto{\pgfqpoint{5.396302in}{1.963133in}}%
\pgfpathlineto{\pgfqpoint{5.405109in}{2.546468in}}%
\pgfpathlineto{\pgfqpoint{5.413916in}{2.244484in}}%
\pgfpathlineto{\pgfqpoint{5.422722in}{2.107252in}}%
\pgfpathlineto{\pgfqpoint{5.431529in}{2.011173in}}%
\pgfpathlineto{\pgfqpoint{5.440336in}{2.429784in}}%
\pgfpathlineto{\pgfqpoint{5.449143in}{2.498428in}}%
\pgfpathlineto{\pgfqpoint{5.457950in}{2.649406in}}%
\pgfpathlineto{\pgfqpoint{5.466757in}{2.512146in}}%
\pgfpathlineto{\pgfqpoint{5.475563in}{2.203303in}}%
\pgfpathlineto{\pgfqpoint{5.484370in}{2.299383in}}%
\pgfpathlineto{\pgfqpoint{5.493177in}{1.928811in}}%
\pgfpathlineto{\pgfqpoint{5.501984in}{1.956246in}}%
\pgfpathlineto{\pgfqpoint{5.510791in}{2.018032in}}%
\pgfpathlineto{\pgfqpoint{5.519597in}{1.791551in}}%
\pgfpathlineto{\pgfqpoint{5.528404in}{1.626856in}}%
\pgfpathlineto{\pgfqpoint{5.546018in}{1.613110in}}%
\pgfpathlineto{\pgfqpoint{5.554825in}{1.894489in}}%
\pgfpathlineto{\pgfqpoint{5.563632in}{1.908207in}}%
\pgfpathlineto{\pgfqpoint{5.572438in}{2.093507in}}%
\pgfpathlineto{\pgfqpoint{5.581245in}{1.997427in}}%
\pgfpathlineto{\pgfqpoint{5.590052in}{1.359195in}}%
\pgfpathlineto{\pgfqpoint{5.598859in}{1.839591in}}%
\pgfpathlineto{\pgfqpoint{5.607666in}{1.585675in}}%
\pgfpathlineto{\pgfqpoint{5.616472in}{1.716076in}}%
\pgfpathlineto{\pgfqpoint{5.625279in}{1.565071in}}%
\pgfpathlineto{\pgfqpoint{5.634086in}{1.558212in}}%
\pgfpathlineto{\pgfqpoint{5.642893in}{1.523918in}}%
\pgfpathlineto{\pgfqpoint{5.651700in}{1.599393in}}%
\pgfpathlineto{\pgfqpoint{5.660507in}{1.819015in}}%
\pgfpathlineto{\pgfqpoint{5.669313in}{1.812128in}}%
\pgfpathlineto{\pgfqpoint{5.678120in}{1.722935in}}%
\pgfpathlineto{\pgfqpoint{5.686927in}{2.079789in}}%
\pgfpathlineto{\pgfqpoint{5.695734in}{1.997427in}}%
\pgfpathlineto{\pgfqpoint{5.704541in}{2.107252in}}%
\pgfpathlineto{\pgfqpoint{5.713347in}{2.189586in}}%
\pgfpathlineto{\pgfqpoint{5.722154in}{2.079789in}}%
\pgfpathlineto{\pgfqpoint{5.730961in}{1.894489in}}%
\pgfpathlineto{\pgfqpoint{5.739768in}{2.120970in}}%
\pgfpathlineto{\pgfqpoint{5.748575in}{1.668037in}}%
\pgfpathlineto{\pgfqpoint{5.757382in}{1.791551in}}%
\pgfpathlineto{\pgfqpoint{5.766188in}{1.468992in}}%
\pgfpathlineto{\pgfqpoint{5.774995in}{1.963133in}}%
\pgfpathlineto{\pgfqpoint{5.783802in}{1.674896in}}%
\pgfpathlineto{\pgfqpoint{5.792609in}{1.681754in}}%
\pgfpathlineto{\pgfqpoint{5.801416in}{1.544494in}}%
\pgfpathlineto{\pgfqpoint{5.810222in}{1.475878in}}%
\pgfpathlineto{\pgfqpoint{5.819029in}{1.743512in}}%
\pgfpathlineto{\pgfqpoint{5.827836in}{1.784692in}}%
\pgfpathlineto{\pgfqpoint{5.845450in}{2.196445in}}%
\pgfpathlineto{\pgfqpoint{5.854257in}{2.059213in}}%
\pgfpathlineto{\pgfqpoint{5.863063in}{2.271947in}}%
\pgfpathlineto{\pgfqpoint{5.871870in}{1.942529in}}%
\pgfpathlineto{\pgfqpoint{5.880677in}{1.928811in}}%
\pgfpathlineto{\pgfqpoint{5.889484in}{1.867054in}}%
\pgfpathlineto{\pgfqpoint{5.898291in}{1.887630in}}%
\pgfpathlineto{\pgfqpoint{5.907097in}{1.935670in}}%
\pgfpathlineto{\pgfqpoint{5.915904in}{2.196445in}}%
\pgfpathlineto{\pgfqpoint{5.924711in}{2.196445in}}%
\pgfpathlineto{\pgfqpoint{5.933518in}{2.079789in}}%
\pgfpathlineto{\pgfqpoint{5.942325in}{2.079789in}}%
\pgfpathlineto{\pgfqpoint{5.951132in}{2.388603in}}%
\pgfpathlineto{\pgfqpoint{5.968745in}{1.770975in}}%
\pgfpathlineto{\pgfqpoint{5.977552in}{2.196445in}}%
\pgfpathlineto{\pgfqpoint{5.986359in}{2.333705in}}%
\pgfpathlineto{\pgfqpoint{5.995166in}{2.031749in}}%
\pgfpathlineto{\pgfqpoint{6.003972in}{2.155292in}}%
\pgfpathlineto{\pgfqpoint{6.012779in}{1.990569in}}%
\pgfpathlineto{\pgfqpoint{6.021586in}{2.011173in}}%
\pgfpathlineto{\pgfqpoint{6.030393in}{2.162151in}}%
\pgfpathlineto{\pgfqpoint{6.039200in}{1.908207in}}%
\pgfpathlineto{\pgfqpoint{6.048007in}{1.757229in}}%
\pgfpathlineto{\pgfqpoint{6.056813in}{1.297438in}}%
\pgfpathlineto{\pgfqpoint{6.065620in}{1.194499in}}%
\pgfpathlineto{\pgfqpoint{6.074427in}{1.441556in}}%
\pgfpathlineto{\pgfqpoint{6.083234in}{1.750370in}}%
\pgfpathlineto{\pgfqpoint{6.092041in}{1.393517in}}%
\pgfpathlineto{\pgfqpoint{6.109654in}{1.908207in}}%
\pgfpathlineto{\pgfqpoint{6.118461in}{1.784692in}}%
\pgfpathlineto{\pgfqpoint{6.127268in}{1.722935in}}%
\pgfpathlineto{\pgfqpoint{6.136075in}{1.551353in}}%
\pgfpathlineto{\pgfqpoint{6.144882in}{1.729794in}}%
\pgfpathlineto{\pgfqpoint{6.153688in}{1.819015in}}%
\pgfpathlineto{\pgfqpoint{6.162495in}{1.688613in}}%
\pgfpathlineto{\pgfqpoint{6.171302in}{1.640574in}}%
\pgfpathlineto{\pgfqpoint{6.188916in}{1.839591in}}%
\pgfpathlineto{\pgfqpoint{6.197722in}{2.031749in}}%
\pgfpathlineto{\pgfqpoint{6.206529in}{1.969992in}}%
\pgfpathlineto{\pgfqpoint{6.215336in}{1.990569in}}%
\pgfpathlineto{\pgfqpoint{6.224143in}{1.853308in}}%
\pgfpathlineto{\pgfqpoint{6.232950in}{2.299383in}}%
\pgfpathlineto{\pgfqpoint{6.241757in}{2.278806in}}%
\pgfpathlineto{\pgfqpoint{6.250563in}{2.127829in}}%
\pgfpathlineto{\pgfqpoint{6.259370in}{2.381744in}}%
\pgfpathlineto{\pgfqpoint{6.268177in}{2.505287in}}%
\pgfpathlineto{\pgfqpoint{6.276984in}{2.834677in}}%
\pgfpathlineto{\pgfqpoint{6.285791in}{2.820960in}}%
\pgfpathlineto{\pgfqpoint{6.294597in}{2.498428in}}%
\pgfpathlineto{\pgfqpoint{6.303404in}{2.374885in}}%
\pgfpathlineto{\pgfqpoint{6.312211in}{2.381744in}}%
\pgfpathlineto{\pgfqpoint{6.321018in}{2.244484in}}%
\pgfpathlineto{\pgfqpoint{6.329825in}{1.921953in}}%
\pgfpathlineto{\pgfqpoint{6.338632in}{2.072930in}}%
\pgfpathlineto{\pgfqpoint{6.347438in}{2.052326in}}%
\pgfpathlineto{\pgfqpoint{6.356245in}{2.237625in}}%
\pgfpathlineto{\pgfqpoint{6.365052in}{2.470965in}}%
\pgfpathlineto{\pgfqpoint{6.373859in}{2.237625in}}%
\pgfpathlineto{\pgfqpoint{6.382666in}{2.354309in}}%
\pgfpathlineto{\pgfqpoint{6.391472in}{2.011173in}}%
\pgfpathlineto{\pgfqpoint{6.400279in}{2.162151in}}%
\pgfpathlineto{\pgfqpoint{6.409086in}{2.134687in}}%
\pgfpathlineto{\pgfqpoint{6.417893in}{2.196445in}}%
\pgfpathlineto{\pgfqpoint{6.426700in}{2.484682in}}%
\pgfpathlineto{\pgfqpoint{6.435507in}{2.368027in}}%
\pgfpathlineto{\pgfqpoint{6.444313in}{2.381744in}}%
\pgfpathlineto{\pgfqpoint{6.453120in}{2.416066in}}%
\pgfpathlineto{\pgfqpoint{6.461927in}{2.066071in}}%
\pgfpathlineto{\pgfqpoint{6.470734in}{2.045467in}}%
\pgfpathlineto{\pgfqpoint{6.479541in}{2.114111in}}%
\pgfpathlineto{\pgfqpoint{6.488347in}{2.107252in}}%
\pgfpathlineto{\pgfqpoint{6.497154in}{1.997427in}}%
\pgfpathlineto{\pgfqpoint{6.505961in}{2.011173in}}%
\pgfpathlineto{\pgfqpoint{6.514768in}{2.059213in}}%
\pgfpathlineto{\pgfqpoint{6.523575in}{1.798410in}}%
\pgfpathlineto{\pgfqpoint{6.532382in}{1.825873in}}%
\pgfpathlineto{\pgfqpoint{6.541188in}{2.072930in}}%
\pgfpathlineto{\pgfqpoint{6.549995in}{1.743512in}}%
\pgfpathlineto{\pgfqpoint{6.558802in}{1.702331in}}%
\pgfpathlineto{\pgfqpoint{6.567609in}{1.633715in}}%
\pgfpathlineto{\pgfqpoint{6.576416in}{1.626856in}}%
\pgfpathlineto{\pgfqpoint{6.585222in}{1.949388in}}%
\pgfpathlineto{\pgfqpoint{6.594029in}{1.921953in}}%
\pgfpathlineto{\pgfqpoint{6.602836in}{1.489596in}}%
\pgfpathlineto{\pgfqpoint{6.611643in}{1.702331in}}%
\pgfpathlineto{\pgfqpoint{6.620450in}{1.860167in}}%
\pgfpathlineto{\pgfqpoint{6.629257in}{2.052326in}}%
\pgfpathlineto{\pgfqpoint{6.638063in}{1.736653in}}%
\pgfpathlineto{\pgfqpoint{6.646870in}{1.928811in}}%
\pgfpathlineto{\pgfqpoint{6.655677in}{1.770975in}}%
\pgfpathlineto{\pgfqpoint{6.664484in}{1.757229in}}%
\pgfpathlineto{\pgfqpoint{6.673291in}{2.024891in}}%
\pgfpathlineto{\pgfqpoint{6.682097in}{2.217049in}}%
\pgfpathlineto{\pgfqpoint{6.690904in}{2.203303in}}%
\pgfpathlineto{\pgfqpoint{6.699711in}{2.011173in}}%
\pgfpathlineto{\pgfqpoint{6.717325in}{1.681754in}}%
\pgfpathlineto{\pgfqpoint{6.726132in}{1.976851in}}%
\pgfpathlineto{\pgfqpoint{6.734938in}{2.148405in}}%
\pgfpathlineto{\pgfqpoint{6.743745in}{1.983710in}}%
\pgfpathlineto{\pgfqpoint{6.752552in}{1.894489in}}%
\pgfpathlineto{\pgfqpoint{6.761359in}{1.743512in}}%
\pgfpathlineto{\pgfqpoint{6.770166in}{1.819015in}}%
\pgfpathlineto{\pgfqpoint{6.778972in}{1.455274in}}%
\pgfpathlineto{\pgfqpoint{6.787779in}{1.544494in}}%
\pgfpathlineto{\pgfqpoint{6.796586in}{1.551353in}}%
\pgfpathlineto{\pgfqpoint{6.805393in}{1.517031in}}%
\pgfpathlineto{\pgfqpoint{6.814200in}{1.901348in}}%
\pgfpathlineto{\pgfqpoint{6.823007in}{1.606252in}}%
\pgfpathlineto{\pgfqpoint{6.831813in}{1.386658in}}%
\pgfpathlineto{\pgfqpoint{6.840620in}{1.304296in}}%
\pgfpathlineto{\pgfqpoint{6.849427in}{1.702331in}}%
\pgfpathlineto{\pgfqpoint{6.858234in}{1.537636in}}%
\pgfpathlineto{\pgfqpoint{6.867041in}{1.757229in}}%
\pgfpathlineto{\pgfqpoint{6.875847in}{1.619997in}}%
\pgfpathlineto{\pgfqpoint{6.884654in}{2.045467in}}%
\pgfpathlineto{\pgfqpoint{6.893461in}{1.770975in}}%
\pgfpathlineto{\pgfqpoint{6.902268in}{1.812128in}}%
\pgfpathlineto{\pgfqpoint{6.911075in}{1.462133in}}%
\pgfpathlineto{\pgfqpoint{6.919882in}{1.716076in}}%
\pgfpathlineto{\pgfqpoint{6.928688in}{1.722935in}}%
\pgfpathlineto{\pgfqpoint{6.937495in}{1.784692in}}%
\pgfpathlineto{\pgfqpoint{6.946302in}{2.169009in}}%
\pgfpathlineto{\pgfqpoint{6.955109in}{1.949388in}}%
\pgfpathlineto{\pgfqpoint{6.963916in}{2.422925in}}%
\pgfpathlineto{\pgfqpoint{6.972722in}{2.285665in}}%
\pgfpathlineto{\pgfqpoint{6.981529in}{2.436643in}}%
\pgfpathlineto{\pgfqpoint{6.990336in}{2.841564in}}%
\pgfpathlineto{\pgfqpoint{6.999143in}{2.621942in}}%
\pgfpathlineto{\pgfqpoint{7.007950in}{2.278806in}}%
\pgfpathlineto{\pgfqpoint{7.025563in}{1.894489in}}%
\pgfpathlineto{\pgfqpoint{7.034370in}{1.729794in}}%
\pgfpathlineto{\pgfqpoint{7.043177in}{1.770975in}}%
\pgfpathlineto{\pgfqpoint{7.051984in}{1.558212in}}%
\pgfpathlineto{\pgfqpoint{7.060791in}{1.770975in}}%
\pgfpathlineto{\pgfqpoint{7.069597in}{1.510172in}}%
\pgfpathlineto{\pgfqpoint{7.078404in}{1.949388in}}%
\pgfpathlineto{\pgfqpoint{7.087211in}{1.839591in}}%
\pgfpathlineto{\pgfqpoint{7.096018in}{1.990569in}}%
\pgfpathlineto{\pgfqpoint{7.104825in}{2.409207in}}%
\pgfpathlineto{\pgfqpoint{7.113632in}{2.642547in}}%
\pgfpathlineto{\pgfqpoint{7.122438in}{2.230767in}}%
\pgfpathlineto{\pgfqpoint{7.131245in}{2.429784in}}%
\pgfpathlineto{\pgfqpoint{7.140052in}{2.443501in}}%
\pgfpathlineto{\pgfqpoint{7.148859in}{2.155292in}}%
\pgfpathlineto{\pgfqpoint{7.157666in}{1.812128in}}%
\pgfpathlineto{\pgfqpoint{7.175279in}{2.546468in}}%
\pgfpathlineto{\pgfqpoint{7.184086in}{2.258230in}}%
\pgfpathlineto{\pgfqpoint{7.192893in}{2.306269in}}%
\pgfpathlineto{\pgfqpoint{7.201700in}{1.853308in}}%
\pgfpathlineto{\pgfqpoint{7.210507in}{1.571958in}}%
\pgfpathlineto{\pgfqpoint{7.219313in}{1.517031in}}%
\pgfpathlineto{\pgfqpoint{7.228120in}{1.592534in}}%
\pgfpathlineto{\pgfqpoint{7.236927in}{1.654291in}}%
\pgfpathlineto{\pgfqpoint{7.245734in}{1.537636in}}%
\pgfpathlineto{\pgfqpoint{7.254541in}{1.530777in}}%
\pgfpathlineto{\pgfqpoint{7.263347in}{1.668037in}}%
\pgfpathlineto{\pgfqpoint{7.272154in}{1.661150in}}%
\pgfpathlineto{\pgfqpoint{7.280961in}{1.839591in}}%
\pgfpathlineto{\pgfqpoint{7.289768in}{1.846450in}}%
\pgfpathlineto{\pgfqpoint{7.298575in}{1.743512in}}%
\pgfpathlineto{\pgfqpoint{7.307382in}{2.052326in}}%
\pgfpathlineto{\pgfqpoint{7.316188in}{2.024891in}}%
\pgfpathlineto{\pgfqpoint{7.324995in}{1.640574in}}%
\pgfpathlineto{\pgfqpoint{7.342609in}{1.880772in}}%
\pgfpathlineto{\pgfqpoint{7.360222in}{1.736653in}}%
\pgfpathlineto{\pgfqpoint{7.369029in}{1.750370in}}%
\pgfpathlineto{\pgfqpoint{7.377836in}{1.791551in}}%
\pgfpathlineto{\pgfqpoint{7.386643in}{1.688613in}}%
\pgfpathlineto{\pgfqpoint{7.395450in}{1.654291in}}%
\pgfpathlineto{\pgfqpoint{7.404257in}{1.819015in}}%
\pgfpathlineto{\pgfqpoint{7.413063in}{2.189586in}}%
\pgfpathlineto{\pgfqpoint{7.421870in}{2.416066in}}%
\pgfpathlineto{\pgfqpoint{7.430677in}{2.079789in}}%
\pgfpathlineto{\pgfqpoint{7.439484in}{1.791551in}}%
\pgfpathlineto{\pgfqpoint{7.448291in}{2.059213in}}%
\pgfpathlineto{\pgfqpoint{7.457097in}{2.004286in}}%
\pgfpathlineto{\pgfqpoint{7.465904in}{2.059213in}}%
\pgfpathlineto{\pgfqpoint{7.474711in}{2.079789in}}%
\pgfpathlineto{\pgfqpoint{7.483518in}{1.949388in}}%
\pgfpathlineto{\pgfqpoint{7.492325in}{2.223908in}}%
\pgfpathlineto{\pgfqpoint{7.501132in}{1.578816in}}%
\pgfpathlineto{\pgfqpoint{7.509938in}{1.482737in}}%
\pgfpathlineto{\pgfqpoint{7.518745in}{1.791551in}}%
\pgfpathlineto{\pgfqpoint{7.527552in}{1.777834in}}%
\pgfpathlineto{\pgfqpoint{7.536359in}{1.681754in}}%
\pgfpathlineto{\pgfqpoint{7.545166in}{1.956246in}}%
\pgfpathlineto{\pgfqpoint{7.553972in}{1.791551in}}%
\pgfpathlineto{\pgfqpoint{7.562779in}{1.880772in}}%
\pgfpathlineto{\pgfqpoint{7.571586in}{1.709190in}}%
\pgfpathlineto{\pgfqpoint{7.580393in}{1.729794in}}%
\pgfpathlineto{\pgfqpoint{7.598007in}{2.203303in}}%
\pgfpathlineto{\pgfqpoint{7.606813in}{2.265089in}}%
\pgfpathlineto{\pgfqpoint{7.615620in}{1.942529in}}%
\pgfpathlineto{\pgfqpoint{7.624427in}{2.285665in}}%
\pgfpathlineto{\pgfqpoint{7.642041in}{1.668037in}}%
\pgfpathlineto{\pgfqpoint{7.650847in}{1.585675in}}%
\pgfpathlineto{\pgfqpoint{7.659654in}{1.764116in}}%
\pgfpathlineto{\pgfqpoint{7.668461in}{1.722935in}}%
\pgfpathlineto{\pgfqpoint{7.677268in}{1.825873in}}%
\pgfpathlineto{\pgfqpoint{7.686075in}{1.736653in}}%
\pgfpathlineto{\pgfqpoint{7.694882in}{1.626856in}}%
\pgfpathlineto{\pgfqpoint{7.703688in}{1.434698in}}%
\pgfpathlineto{\pgfqpoint{7.712495in}{1.537636in}}%
\pgfpathlineto{\pgfqpoint{7.721302in}{1.462133in}}%
\pgfpathlineto{\pgfqpoint{7.730109in}{1.420952in}}%
\pgfpathlineto{\pgfqpoint{7.738916in}{1.118997in}}%
\pgfpathlineto{\pgfqpoint{7.747722in}{1.366053in}}%
\pgfpathlineto{\pgfqpoint{7.756529in}{1.414093in}}%
\pgfpathlineto{\pgfqpoint{7.774143in}{1.812128in}}%
\pgfpathlineto{\pgfqpoint{7.782950in}{2.052326in}}%
\pgfpathlineto{\pgfqpoint{7.791757in}{2.155292in}}%
\pgfpathlineto{\pgfqpoint{7.800563in}{1.915094in}}%
\pgfpathlineto{\pgfqpoint{7.809370in}{2.045467in}}%
\pgfpathlineto{\pgfqpoint{7.818177in}{2.107252in}}%
\pgfpathlineto{\pgfqpoint{7.826984in}{1.983710in}}%
\pgfpathlineto{\pgfqpoint{7.835791in}{2.368027in}}%
\pgfpathlineto{\pgfqpoint{7.844597in}{2.203303in}}%
\pgfpathlineto{\pgfqpoint{7.853404in}{1.990569in}}%
\pgfpathlineto{\pgfqpoint{7.871018in}{2.107252in}}%
\pgfpathlineto{\pgfqpoint{7.879825in}{1.496455in}}%
\pgfpathlineto{\pgfqpoint{7.888632in}{1.544494in}}%
\pgfpathlineto{\pgfqpoint{7.897438in}{1.640574in}}%
\pgfpathlineto{\pgfqpoint{7.906245in}{1.770975in}}%
\pgfpathlineto{\pgfqpoint{7.915052in}{1.482737in}}%
\pgfpathlineto{\pgfqpoint{7.923859in}{1.537636in}}%
\pgfpathlineto{\pgfqpoint{7.932666in}{1.743512in}}%
\pgfpathlineto{\pgfqpoint{7.941472in}{1.681754in}}%
\pgfpathlineto{\pgfqpoint{7.950279in}{1.695472in}}%
\pgfpathlineto{\pgfqpoint{7.959086in}{1.873913in}}%
\pgfpathlineto{\pgfqpoint{7.967893in}{2.011173in}}%
\pgfpathlineto{\pgfqpoint{7.976700in}{1.963133in}}%
\pgfpathlineto{\pgfqpoint{7.985507in}{2.491541in}}%
\pgfpathlineto{\pgfqpoint{7.994313in}{2.539581in}}%
\pgfpathlineto{\pgfqpoint{8.003120in}{2.319987in}}%
\pgfpathlineto{\pgfqpoint{8.011927in}{2.340563in}}%
\pgfpathlineto{\pgfqpoint{8.020734in}{2.251343in}}%
\pgfpathlineto{\pgfqpoint{8.029541in}{2.237625in}}%
\pgfpathlineto{\pgfqpoint{8.038347in}{2.347422in}}%
\pgfpathlineto{\pgfqpoint{8.047154in}{2.052326in}}%
\pgfpathlineto{\pgfqpoint{8.055961in}{2.107252in}}%
\pgfpathlineto{\pgfqpoint{8.064768in}{1.791551in}}%
\pgfpathlineto{\pgfqpoint{8.073575in}{1.990569in}}%
\pgfpathlineto{\pgfqpoint{8.082382in}{1.853308in}}%
\pgfpathlineto{\pgfqpoint{8.091188in}{1.757229in}}%
\pgfpathlineto{\pgfqpoint{8.099995in}{1.558212in}}%
\pgfpathlineto{\pgfqpoint{8.108802in}{1.839591in}}%
\pgfpathlineto{\pgfqpoint{8.117609in}{1.908207in}}%
\pgfpathlineto{\pgfqpoint{8.126416in}{1.839591in}}%
\pgfpathlineto{\pgfqpoint{8.135222in}{2.196445in}}%
\pgfpathlineto{\pgfqpoint{8.144029in}{2.011173in}}%
\pgfpathlineto{\pgfqpoint{8.152836in}{1.963133in}}%
\pgfpathlineto{\pgfqpoint{8.161643in}{1.846450in}}%
\pgfpathlineto{\pgfqpoint{8.170450in}{1.928811in}}%
\pgfpathlineto{\pgfqpoint{8.179257in}{1.825873in}}%
\pgfpathlineto{\pgfqpoint{8.188063in}{1.736653in}}%
\pgfpathlineto{\pgfqpoint{8.196870in}{1.873913in}}%
\pgfpathlineto{\pgfqpoint{8.205677in}{1.846450in}}%
\pgfpathlineto{\pgfqpoint{8.214484in}{1.613110in}}%
\pgfpathlineto{\pgfqpoint{8.223291in}{1.510172in}}%
\pgfpathlineto{\pgfqpoint{8.232097in}{1.709190in}}%
\pgfpathlineto{\pgfqpoint{8.240904in}{1.819015in}}%
\pgfpathlineto{\pgfqpoint{8.249711in}{1.764116in}}%
\pgfpathlineto{\pgfqpoint{8.258518in}{1.468992in}}%
\pgfpathlineto{\pgfqpoint{8.267325in}{1.565071in}}%
\pgfpathlineto{\pgfqpoint{8.276132in}{1.798410in}}%
\pgfpathlineto{\pgfqpoint{8.284938in}{2.066071in}}%
\pgfpathlineto{\pgfqpoint{8.293745in}{1.949388in}}%
\pgfpathlineto{\pgfqpoint{8.302552in}{2.169009in}}%
\pgfpathlineto{\pgfqpoint{8.311359in}{2.052326in}}%
\pgfpathlineto{\pgfqpoint{8.320166in}{2.038608in}}%
\pgfpathlineto{\pgfqpoint{8.328972in}{2.313128in}}%
\pgfpathlineto{\pgfqpoint{8.337779in}{2.251343in}}%
\pgfpathlineto{\pgfqpoint{8.346586in}{1.832732in}}%
\pgfpathlineto{\pgfqpoint{8.355393in}{1.496455in}}%
\pgfpathlineto{\pgfqpoint{8.364200in}{1.599393in}}%
\pgfpathlineto{\pgfqpoint{8.373007in}{1.983710in}}%
\pgfpathlineto{\pgfqpoint{8.381813in}{2.196445in}}%
\pgfpathlineto{\pgfqpoint{8.390620in}{2.340563in}}%
\pgfpathlineto{\pgfqpoint{8.399427in}{2.313128in}}%
\pgfpathlineto{\pgfqpoint{8.408234in}{2.093507in}}%
\pgfpathlineto{\pgfqpoint{8.417041in}{2.086648in}}%
\pgfpathlineto{\pgfqpoint{8.425847in}{2.169009in}}%
\pgfpathlineto{\pgfqpoint{8.434654in}{2.512146in}}%
\pgfpathlineto{\pgfqpoint{8.443461in}{2.169009in}}%
\pgfpathlineto{\pgfqpoint{8.461075in}{2.038608in}}%
\pgfpathlineto{\pgfqpoint{8.469882in}{2.251343in}}%
\pgfpathlineto{\pgfqpoint{8.478688in}{2.251343in}}%
\pgfpathlineto{\pgfqpoint{8.487495in}{2.210190in}}%
\pgfpathlineto{\pgfqpoint{8.496302in}{2.155292in}}%
\pgfpathlineto{\pgfqpoint{8.505109in}{2.175868in}}%
\pgfpathlineto{\pgfqpoint{8.513916in}{2.594507in}}%
\pgfpathlineto{\pgfqpoint{8.522722in}{2.450388in}}%
\pgfpathlineto{\pgfqpoint{8.531529in}{2.388603in}}%
\pgfpathlineto{\pgfqpoint{8.540336in}{2.155292in}}%
\pgfpathlineto{\pgfqpoint{8.549143in}{2.210190in}}%
\pgfpathlineto{\pgfqpoint{8.557950in}{2.031749in}}%
\pgfpathlineto{\pgfqpoint{8.566757in}{2.093507in}}%
\pgfpathlineto{\pgfqpoint{8.575563in}{1.757229in}}%
\pgfpathlineto{\pgfqpoint{8.584370in}{1.729794in}}%
\pgfpathlineto{\pgfqpoint{8.593177in}{2.299383in}}%
\pgfpathlineto{\pgfqpoint{8.601984in}{2.361168in}}%
\pgfpathlineto{\pgfqpoint{8.610791in}{2.162151in}}%
\pgfpathlineto{\pgfqpoint{8.619597in}{2.066071in}}%
\pgfpathlineto{\pgfqpoint{8.628404in}{2.251343in}}%
\pgfpathlineto{\pgfqpoint{8.637211in}{2.155292in}}%
\pgfpathlineto{\pgfqpoint{8.646018in}{2.100365in}}%
\pgfpathlineto{\pgfqpoint{8.654825in}{1.853308in}}%
\pgfpathlineto{\pgfqpoint{8.663632in}{2.093507in}}%
\pgfpathlineto{\pgfqpoint{8.672438in}{2.024891in}}%
\pgfpathlineto{\pgfqpoint{8.681245in}{1.928811in}}%
\pgfpathlineto{\pgfqpoint{8.690052in}{1.757229in}}%
\pgfpathlineto{\pgfqpoint{8.698859in}{1.819015in}}%
\pgfpathlineto{\pgfqpoint{8.707666in}{1.599393in}}%
\pgfpathlineto{\pgfqpoint{8.716472in}{1.510172in}}%
\pgfpathlineto{\pgfqpoint{8.725279in}{1.578816in}}%
\pgfpathlineto{\pgfqpoint{8.734086in}{1.510172in}}%
\pgfpathlineto{\pgfqpoint{8.742893in}{1.873913in}}%
\pgfpathlineto{\pgfqpoint{8.751700in}{2.052326in}}%
\pgfpathlineto{\pgfqpoint{8.760507in}{1.921953in}}%
\pgfpathlineto{\pgfqpoint{8.769313in}{2.429784in}}%
\pgfpathlineto{\pgfqpoint{8.778120in}{2.326846in}}%
\pgfpathlineto{\pgfqpoint{8.786927in}{1.976851in}}%
\pgfpathlineto{\pgfqpoint{8.795734in}{1.867054in}}%
\pgfpathlineto{\pgfqpoint{8.813347in}{1.523918in}}%
\pgfpathlineto{\pgfqpoint{8.822154in}{1.551353in}}%
\pgfpathlineto{\pgfqpoint{8.830961in}{1.448415in}}%
\pgfpathlineto{\pgfqpoint{8.839768in}{1.729794in}}%
\pgfpathlineto{\pgfqpoint{8.848575in}{1.668037in}}%
\pgfpathlineto{\pgfqpoint{8.857382in}{1.976851in}}%
\pgfpathlineto{\pgfqpoint{8.866188in}{1.777834in}}%
\pgfpathlineto{\pgfqpoint{8.874995in}{1.873913in}}%
\pgfpathlineto{\pgfqpoint{8.883802in}{1.997427in}}%
\pgfpathlineto{\pgfqpoint{8.892609in}{1.551353in}}%
\pgfpathlineto{\pgfqpoint{8.901416in}{1.688613in}}%
\pgfpathlineto{\pgfqpoint{8.910222in}{1.764116in}}%
\pgfpathlineto{\pgfqpoint{8.919029in}{2.313128in}}%
\pgfpathlineto{\pgfqpoint{8.927836in}{1.983710in}}%
\pgfpathlineto{\pgfqpoint{8.936643in}{1.592534in}}%
\pgfpathlineto{\pgfqpoint{8.945450in}{1.798410in}}%
\pgfpathlineto{\pgfqpoint{8.954257in}{1.764116in}}%
\pgfpathlineto{\pgfqpoint{8.963063in}{1.935670in}}%
\pgfpathlineto{\pgfqpoint{8.971870in}{1.537636in}}%
\pgfpathlineto{\pgfqpoint{8.980677in}{1.496455in}}%
\pgfpathlineto{\pgfqpoint{8.989484in}{1.263115in}}%
\pgfpathlineto{\pgfqpoint{8.998291in}{1.565071in}}%
\pgfpathlineto{\pgfqpoint{9.007097in}{1.791551in}}%
\pgfpathlineto{\pgfqpoint{9.015904in}{1.867054in}}%
\pgfpathlineto{\pgfqpoint{9.024711in}{2.189586in}}%
\pgfpathlineto{\pgfqpoint{9.033518in}{2.134687in}}%
\pgfpathlineto{\pgfqpoint{9.042325in}{2.052326in}}%
\pgfpathlineto{\pgfqpoint{9.051132in}{1.990569in}}%
\pgfpathlineto{\pgfqpoint{9.059938in}{1.867054in}}%
\pgfpathlineto{\pgfqpoint{9.068745in}{1.764116in}}%
\pgfpathlineto{\pgfqpoint{9.077552in}{2.120970in}}%
\pgfpathlineto{\pgfqpoint{9.086359in}{1.661150in}}%
\pgfpathlineto{\pgfqpoint{9.095166in}{1.551353in}}%
\pgfpathlineto{\pgfqpoint{9.103972in}{1.880772in}}%
\pgfpathlineto{\pgfqpoint{9.112779in}{1.722935in}}%
\pgfpathlineto{\pgfqpoint{9.121586in}{1.743512in}}%
\pgfpathlineto{\pgfqpoint{9.130393in}{1.633715in}}%
\pgfpathlineto{\pgfqpoint{9.139200in}{1.640574in}}%
\pgfpathlineto{\pgfqpoint{9.148007in}{1.565071in}}%
\pgfpathlineto{\pgfqpoint{9.156813in}{1.517031in}}%
\pgfpathlineto{\pgfqpoint{9.165620in}{1.441556in}}%
\pgfpathlineto{\pgfqpoint{9.174427in}{1.496455in}}%
\pgfpathlineto{\pgfqpoint{9.183234in}{1.592534in}}%
\pgfpathlineto{\pgfqpoint{9.192041in}{1.908207in}}%
\pgfpathlineto{\pgfqpoint{9.200847in}{1.997427in}}%
\pgfpathlineto{\pgfqpoint{9.209654in}{2.120970in}}%
\pgfpathlineto{\pgfqpoint{9.218461in}{2.196445in}}%
\pgfpathlineto{\pgfqpoint{9.227268in}{2.148405in}}%
\pgfpathlineto{\pgfqpoint{9.236075in}{1.860167in}}%
\pgfpathlineto{\pgfqpoint{9.244882in}{1.695472in}}%
\pgfpathlineto{\pgfqpoint{9.253688in}{1.661150in}}%
\pgfpathlineto{\pgfqpoint{9.262495in}{1.702331in}}%
\pgfpathlineto{\pgfqpoint{9.271302in}{1.819015in}}%
\pgfpathlineto{\pgfqpoint{9.280109in}{1.578816in}}%
\pgfpathlineto{\pgfqpoint{9.288916in}{1.702331in}}%
\pgfpathlineto{\pgfqpoint{9.297722in}{1.770975in}}%
\pgfpathlineto{\pgfqpoint{9.306529in}{1.798410in}}%
\pgfpathlineto{\pgfqpoint{9.315336in}{1.668037in}}%
\pgfpathlineto{\pgfqpoint{9.324143in}{1.976851in}}%
\pgfpathlineto{\pgfqpoint{9.332950in}{1.935670in}}%
\pgfpathlineto{\pgfqpoint{9.341757in}{2.114111in}}%
\pgfpathlineto{\pgfqpoint{9.350563in}{1.681754in}}%
\pgfpathlineto{\pgfqpoint{9.359370in}{1.606252in}}%
\pgfpathlineto{\pgfqpoint{9.368177in}{1.805269in}}%
\pgfpathlineto{\pgfqpoint{9.385791in}{1.935670in}}%
\pgfpathlineto{\pgfqpoint{9.394597in}{1.805269in}}%
\pgfpathlineto{\pgfqpoint{9.403404in}{1.942529in}}%
\pgfpathlineto{\pgfqpoint{9.412211in}{1.805269in}}%
\pgfpathlineto{\pgfqpoint{9.421018in}{1.873913in}}%
\pgfpathlineto{\pgfqpoint{9.429825in}{1.873913in}}%
\pgfpathlineto{\pgfqpoint{9.438632in}{1.825873in}}%
\pgfpathlineto{\pgfqpoint{9.447438in}{1.674896in}}%
\pgfpathlineto{\pgfqpoint{9.456245in}{1.599393in}}%
\pgfpathlineto{\pgfqpoint{9.465052in}{1.544494in}}%
\pgfpathlineto{\pgfqpoint{9.473859in}{1.791551in}}%
\pgfpathlineto{\pgfqpoint{9.482666in}{1.867054in}}%
\pgfpathlineto{\pgfqpoint{9.491472in}{1.901348in}}%
\pgfpathlineto{\pgfqpoint{9.500279in}{1.915094in}}%
\pgfpathlineto{\pgfqpoint{9.509086in}{1.942529in}}%
\pgfpathlineto{\pgfqpoint{9.517893in}{2.162151in}}%
\pgfpathlineto{\pgfqpoint{9.526700in}{2.148405in}}%
\pgfpathlineto{\pgfqpoint{9.535507in}{2.258230in}}%
\pgfpathlineto{\pgfqpoint{9.544313in}{2.573903in}}%
\pgfpathlineto{\pgfqpoint{9.553120in}{2.642547in}}%
\pgfpathlineto{\pgfqpoint{9.561927in}{2.676841in}}%
\pgfpathlineto{\pgfqpoint{9.570734in}{2.464106in}}%
\pgfpathlineto{\pgfqpoint{9.579541in}{2.210190in}}%
\pgfpathlineto{\pgfqpoint{9.588347in}{2.422925in}}%
\pgfpathlineto{\pgfqpoint{9.597154in}{2.470965in}}%
\pgfpathlineto{\pgfqpoint{9.605961in}{2.477823in}}%
\pgfpathlineto{\pgfqpoint{9.614768in}{2.340563in}}%
\pgfpathlineto{\pgfqpoint{9.623575in}{2.347422in}}%
\pgfpathlineto{\pgfqpoint{9.632382in}{2.107252in}}%
\pgfpathlineto{\pgfqpoint{9.641188in}{2.148405in}}%
\pgfpathlineto{\pgfqpoint{9.649995in}{1.777834in}}%
\pgfpathlineto{\pgfqpoint{9.658802in}{1.503314in}}%
\pgfpathlineto{\pgfqpoint{9.667609in}{1.084703in}}%
\pgfpathlineto{\pgfqpoint{9.676416in}{1.462133in}}%
\pgfpathlineto{\pgfqpoint{9.685222in}{1.420952in}}%
\pgfpathlineto{\pgfqpoint{9.694029in}{1.523918in}}%
\pgfpathlineto{\pgfqpoint{9.702836in}{1.729794in}}%
\pgfpathlineto{\pgfqpoint{9.711643in}{1.468992in}}%
\pgfpathlineto{\pgfqpoint{9.720450in}{1.503314in}}%
\pgfpathlineto{\pgfqpoint{9.729257in}{1.132742in}}%
\pgfpathlineto{\pgfqpoint{9.738063in}{1.269974in}}%
\pgfpathlineto{\pgfqpoint{9.746870in}{1.269974in}}%
\pgfpathlineto{\pgfqpoint{9.755677in}{1.908207in}}%
\pgfpathlineto{\pgfqpoint{9.764484in}{2.093507in}}%
\pgfpathlineto{\pgfqpoint{9.773291in}{1.963133in}}%
\pgfpathlineto{\pgfqpoint{9.782097in}{1.565071in}}%
\pgfpathlineto{\pgfqpoint{9.790904in}{1.544494in}}%
\pgfpathlineto{\pgfqpoint{9.799711in}{1.489596in}}%
\pgfpathlineto{\pgfqpoint{9.808518in}{1.448415in}}%
\pgfpathlineto{\pgfqpoint{9.817325in}{1.530777in}}%
\pgfpathlineto{\pgfqpoint{9.826132in}{1.475878in}}%
\pgfpathlineto{\pgfqpoint{9.834938in}{1.345477in}}%
\pgfpathlineto{\pgfqpoint{9.843745in}{1.716076in}}%
\pgfpathlineto{\pgfqpoint{9.852552in}{1.750370in}}%
\pgfpathlineto{\pgfqpoint{9.861359in}{2.100365in}}%
\pgfpathlineto{\pgfqpoint{9.870166in}{2.162151in}}%
\pgfpathlineto{\pgfqpoint{9.878972in}{2.114111in}}%
\pgfpathlineto{\pgfqpoint{9.896586in}{2.059213in}}%
\pgfpathlineto{\pgfqpoint{9.905393in}{2.155292in}}%
\pgfpathlineto{\pgfqpoint{9.914200in}{1.853308in}}%
\pgfpathlineto{\pgfqpoint{9.923007in}{2.086648in}}%
\pgfpathlineto{\pgfqpoint{9.931813in}{2.223908in}}%
\pgfpathlineto{\pgfqpoint{9.940620in}{2.031749in}}%
\pgfpathlineto{\pgfqpoint{9.949427in}{2.182727in}}%
\pgfpathlineto{\pgfqpoint{9.949427in}{2.182727in}}%
\pgfusepath{stroke}%
\end{pgfscope}%
\begin{pgfscope}%
\pgfpathrectangle{\pgfqpoint{0.702268in}{0.521603in}}{\pgfqpoint{9.687500in}{4.235000in}}%
\pgfusepath{clip}%
\pgfsetrectcap%
\pgfsetroundjoin%
\pgfsetlinewidth{1.003750pt}%
\definecolor{currentstroke}{rgb}{0.000000,0.000000,0.000000}%
\pgfsetstrokecolor{currentstroke}%
\pgfsetstrokeopacity{0.750000}%
\pgfsetdash{}{0pt}%
\pgfpathmoveto{\pgfqpoint{1.142609in}{4.280335in}}%
\pgfpathlineto{\pgfqpoint{1.151416in}{3.493177in}}%
\pgfpathlineto{\pgfqpoint{1.160222in}{2.903996in}}%
\pgfpathlineto{\pgfqpoint{1.169029in}{2.552286in}}%
\pgfpathlineto{\pgfqpoint{1.177836in}{2.266466in}}%
\pgfpathlineto{\pgfqpoint{1.186643in}{2.170724in}}%
\pgfpathlineto{\pgfqpoint{1.204257in}{2.076697in}}%
\pgfpathlineto{\pgfqpoint{1.221870in}{1.911299in}}%
\pgfpathlineto{\pgfqpoint{1.230677in}{1.925045in}}%
\pgfpathlineto{\pgfqpoint{1.239484in}{1.866351in}}%
\pgfpathlineto{\pgfqpoint{1.248291in}{1.908572in}}%
\pgfpathlineto{\pgfqpoint{1.257097in}{1.875965in}}%
\pgfpathlineto{\pgfqpoint{1.265904in}{1.929823in}}%
\pgfpathlineto{\pgfqpoint{1.274711in}{1.887630in}}%
\pgfpathlineto{\pgfqpoint{1.283518in}{1.868769in}}%
\pgfpathlineto{\pgfqpoint{1.292325in}{1.932915in}}%
\pgfpathlineto{\pgfqpoint{1.301132in}{1.843020in}}%
\pgfpathlineto{\pgfqpoint{1.309938in}{1.939100in}}%
\pgfpathlineto{\pgfqpoint{1.318745in}{1.947673in}}%
\pgfpathlineto{\pgfqpoint{1.327552in}{1.992283in}}%
\pgfpathlineto{\pgfqpoint{1.336359in}{1.960041in}}%
\pgfpathlineto{\pgfqpoint{1.345166in}{1.944244in}}%
\pgfpathlineto{\pgfqpoint{1.353972in}{1.946296in}}%
\pgfpathlineto{\pgfqpoint{1.362779in}{1.932241in}}%
\pgfpathlineto{\pgfqpoint{1.371586in}{1.897244in}}%
\pgfpathlineto{\pgfqpoint{1.380393in}{1.916106in}}%
\pgfpathlineto{\pgfqpoint{1.389200in}{1.929823in}}%
\pgfpathlineto{\pgfqpoint{1.398007in}{1.963471in}}%
\pgfpathlineto{\pgfqpoint{1.406813in}{1.975811in}}%
\pgfpathlineto{\pgfqpoint{1.415620in}{1.956612in}}%
\pgfpathlineto{\pgfqpoint{1.424427in}{1.883161in}}%
\pgfpathlineto{\pgfqpoint{1.433234in}{1.837539in}}%
\pgfpathlineto{\pgfqpoint{1.442041in}{1.906492in}}%
\pgfpathlineto{\pgfqpoint{1.450847in}{1.877342in}}%
\pgfpathlineto{\pgfqpoint{1.459654in}{1.935333in}}%
\pgfpathlineto{\pgfqpoint{1.468461in}{1.874250in}}%
\pgfpathlineto{\pgfqpoint{1.477268in}{1.846450in}}%
\pgfpathlineto{\pgfqpoint{1.486075in}{1.870484in}}%
\pgfpathlineto{\pgfqpoint{1.494882in}{1.910962in}}%
\pgfpathlineto{\pgfqpoint{1.503688in}{1.963471in}}%
\pgfpathlineto{\pgfqpoint{1.512495in}{1.950090in}}%
\pgfpathlineto{\pgfqpoint{1.521302in}{2.009796in}}%
\pgfpathlineto{\pgfqpoint{1.530109in}{2.057835in}}%
\pgfpathlineto{\pgfqpoint{1.538916in}{2.082881in}}%
\pgfpathlineto{\pgfqpoint{1.547722in}{2.024891in}}%
\pgfpathlineto{\pgfqpoint{1.565336in}{1.925719in}}%
\pgfpathlineto{\pgfqpoint{1.574143in}{1.996415in}}%
\pgfpathlineto{\pgfqpoint{1.582950in}{1.995375in}}%
\pgfpathlineto{\pgfqpoint{1.591757in}{1.968615in}}%
\pgfpathlineto{\pgfqpoint{1.600563in}{1.876302in}}%
\pgfpathlineto{\pgfqpoint{1.609370in}{1.870821in}}%
\pgfpathlineto{\pgfqpoint{1.618177in}{1.883526in}}%
\pgfpathlineto{\pgfqpoint{1.626984in}{1.891763in}}%
\pgfpathlineto{\pgfqpoint{1.644597in}{1.837876in}}%
\pgfpathlineto{\pgfqpoint{1.653404in}{1.889008in}}%
\pgfpathlineto{\pgfqpoint{1.662211in}{1.969655in}}%
\pgfpathlineto{\pgfqpoint{1.671018in}{1.945284in}}%
\pgfpathlineto{\pgfqpoint{1.679825in}{1.901011in}}%
\pgfpathlineto{\pgfqpoint{1.688632in}{1.891060in}}%
\pgfpathlineto{\pgfqpoint{1.697438in}{1.964848in}}%
\pgfpathlineto{\pgfqpoint{1.706245in}{2.016992in}}%
\pgfpathlineto{\pgfqpoint{1.715052in}{1.948376in}}%
\pgfpathlineto{\pgfqpoint{1.723859in}{1.909247in}}%
\pgfpathlineto{\pgfqpoint{1.732666in}{1.910287in}}%
\pgfpathlineto{\pgfqpoint{1.741472in}{1.952142in}}%
\pgfpathlineto{\pgfqpoint{1.750279in}{2.006001in}}%
\pgfpathlineto{\pgfqpoint{1.759086in}{1.951468in}}%
\pgfpathlineto{\pgfqpoint{1.767893in}{1.988854in}}%
\pgfpathlineto{\pgfqpoint{1.776700in}{1.944946in}}%
\pgfpathlineto{\pgfqpoint{1.785507in}{1.873210in}}%
\pgfpathlineto{\pgfqpoint{1.794313in}{1.910287in}}%
\pgfpathlineto{\pgfqpoint{1.803120in}{1.873576in}}%
\pgfpathlineto{\pgfqpoint{1.811927in}{1.898256in}}%
\pgfpathlineto{\pgfqpoint{1.820734in}{1.941152in}}%
\pgfpathlineto{\pgfqpoint{1.829541in}{1.926057in}}%
\pgfpathlineto{\pgfqpoint{1.838347in}{1.928474in}}%
\pgfpathlineto{\pgfqpoint{1.847154in}{1.903063in}}%
\pgfpathlineto{\pgfqpoint{1.855961in}{1.914728in}}%
\pgfpathlineto{\pgfqpoint{1.864768in}{1.953183in}}%
\pgfpathlineto{\pgfqpoint{1.873575in}{1.966197in}}%
\pgfpathlineto{\pgfqpoint{1.882382in}{1.945621in}}%
\pgfpathlineto{\pgfqpoint{1.891188in}{1.938762in}}%
\pgfpathlineto{\pgfqpoint{1.899995in}{1.955235in}}%
\pgfpathlineto{\pgfqpoint{1.908802in}{1.920912in}}%
\pgfpathlineto{\pgfqpoint{1.917609in}{1.921953in}}%
\pgfpathlineto{\pgfqpoint{1.926416in}{1.905480in}}%
\pgfpathlineto{\pgfqpoint{1.935222in}{1.880434in}}%
\pgfpathlineto{\pgfqpoint{1.944029in}{1.963808in}}%
\pgfpathlineto{\pgfqpoint{1.952836in}{1.966900in}}%
\pgfpathlineto{\pgfqpoint{1.961643in}{1.977526in}}%
\pgfpathlineto{\pgfqpoint{1.970450in}{2.022136in}}%
\pgfpathlineto{\pgfqpoint{1.979257in}{1.986802in}}%
\pgfpathlineto{\pgfqpoint{1.988063in}{1.990906in}}%
\pgfpathlineto{\pgfqpoint{1.996870in}{1.887293in}}%
\pgfpathlineto{\pgfqpoint{2.005677in}{1.890723in}}%
\pgfpathlineto{\pgfqpoint{2.014484in}{1.984384in}}%
\pgfpathlineto{\pgfqpoint{2.023291in}{1.910287in}}%
\pgfpathlineto{\pgfqpoint{2.032097in}{1.890020in}}%
\pgfpathlineto{\pgfqpoint{2.040904in}{1.897581in}}%
\pgfpathlineto{\pgfqpoint{2.049711in}{1.909584in}}%
\pgfpathlineto{\pgfqpoint{2.058518in}{1.854349in}}%
\pgfpathlineto{\pgfqpoint{2.067325in}{1.881812in}}%
\pgfpathlineto{\pgfqpoint{2.076132in}{1.939100in}}%
\pgfpathlineto{\pgfqpoint{2.084938in}{1.974096in}}%
\pgfpathlineto{\pgfqpoint{2.093745in}{1.908207in}}%
\pgfpathlineto{\pgfqpoint{2.102552in}{1.973422in}}%
\pgfpathlineto{\pgfqpoint{2.111359in}{1.885916in}}%
\pgfpathlineto{\pgfqpoint{2.120166in}{1.916106in}}%
\pgfpathlineto{\pgfqpoint{2.128972in}{1.941854in}}%
\pgfpathlineto{\pgfqpoint{2.146586in}{1.912339in}}%
\pgfpathlineto{\pgfqpoint{2.155393in}{1.918158in}}%
\pgfpathlineto{\pgfqpoint{2.164200in}{1.883864in}}%
\pgfpathlineto{\pgfqpoint{2.173007in}{1.954194in}}%
\pgfpathlineto{\pgfqpoint{2.181813in}{1.943569in}}%
\pgfpathlineto{\pgfqpoint{2.190620in}{1.918860in}}%
\pgfpathlineto{\pgfqpoint{2.199427in}{1.939802in}}%
\pgfpathlineto{\pgfqpoint{2.208234in}{1.943232in}}%
\pgfpathlineto{\pgfqpoint{2.217041in}{1.928109in}}%
\pgfpathlineto{\pgfqpoint{2.225847in}{1.874250in}}%
\pgfpathlineto{\pgfqpoint{2.234654in}{1.938425in}}%
\pgfpathlineto{\pgfqpoint{2.243461in}{1.896204in}}%
\pgfpathlineto{\pgfqpoint{2.252268in}{1.980280in}}%
\pgfpathlineto{\pgfqpoint{2.261075in}{2.006366in}}%
\pgfpathlineto{\pgfqpoint{2.269882in}{2.037933in}}%
\pgfpathlineto{\pgfqpoint{2.278688in}{2.022473in}}%
\pgfpathlineto{\pgfqpoint{2.287495in}{1.948713in}}%
\pgfpathlineto{\pgfqpoint{2.296302in}{1.981320in}}%
\pgfpathlineto{\pgfqpoint{2.305109in}{1.938762in}}%
\pgfpathlineto{\pgfqpoint{2.313916in}{1.914728in}}%
\pgfpathlineto{\pgfqpoint{2.322722in}{1.942866in}}%
\pgfpathlineto{\pgfqpoint{2.331529in}{1.924005in}}%
\pgfpathlineto{\pgfqpoint{2.340336in}{2.010470in}}%
\pgfpathlineto{\pgfqpoint{2.349143in}{1.950765in}}%
\pgfpathlineto{\pgfqpoint{2.357950in}{1.927771in}}%
\pgfpathlineto{\pgfqpoint{2.366757in}{1.896541in}}%
\pgfpathlineto{\pgfqpoint{2.384370in}{1.950765in}}%
\pgfpathlineto{\pgfqpoint{2.393177in}{1.846112in}}%
\pgfpathlineto{\pgfqpoint{2.401984in}{1.845438in}}%
\pgfpathlineto{\pgfqpoint{2.410791in}{1.788122in}}%
\pgfpathlineto{\pgfqpoint{2.419597in}{1.855389in}}%
\pgfpathlineto{\pgfqpoint{2.428404in}{1.866689in}}%
\pgfpathlineto{\pgfqpoint{2.437211in}{1.827251in}}%
\pgfpathlineto{\pgfqpoint{2.446018in}{1.844735in}}%
\pgfpathlineto{\pgfqpoint{2.454825in}{1.886590in}}%
\pgfpathlineto{\pgfqpoint{2.463632in}{1.873913in}}%
\pgfpathlineto{\pgfqpoint{2.472438in}{1.894827in}}%
\pgfpathlineto{\pgfqpoint{2.481245in}{1.927434in}}%
\pgfpathlineto{\pgfqpoint{2.490052in}{1.903063in}}%
\pgfpathlineto{\pgfqpoint{2.498859in}{1.960379in}}%
\pgfpathlineto{\pgfqpoint{2.507666in}{1.915094in}}%
\pgfpathlineto{\pgfqpoint{2.516472in}{1.944581in}}%
\pgfpathlineto{\pgfqpoint{2.534086in}{1.921615in}}%
\pgfpathlineto{\pgfqpoint{2.542893in}{1.906492in}}%
\pgfpathlineto{\pgfqpoint{2.551700in}{1.950765in}}%
\pgfpathlineto{\pgfqpoint{2.560507in}{2.053703in}}%
\pgfpathlineto{\pgfqpoint{2.569313in}{1.967912in}}%
\pgfpathlineto{\pgfqpoint{2.578120in}{1.953857in}}%
\pgfpathlineto{\pgfqpoint{2.586927in}{1.990906in}}%
\pgfpathlineto{\pgfqpoint{2.595734in}{1.923330in}}%
\pgfpathlineto{\pgfqpoint{2.604541in}{1.904777in}}%
\pgfpathlineto{\pgfqpoint{2.613347in}{1.878720in}}%
\pgfpathlineto{\pgfqpoint{2.622154in}{1.942866in}}%
\pgfpathlineto{\pgfqpoint{2.630961in}{1.913716in}}%
\pgfpathlineto{\pgfqpoint{2.639768in}{1.873210in}}%
\pgfpathlineto{\pgfqpoint{2.648575in}{1.862922in}}%
\pgfpathlineto{\pgfqpoint{2.657382in}{1.848867in}}%
\pgfpathlineto{\pgfqpoint{2.666188in}{1.868769in}}%
\pgfpathlineto{\pgfqpoint{2.674995in}{1.914054in}}%
\pgfpathlineto{\pgfqpoint{2.683802in}{1.891763in}}%
\pgfpathlineto{\pgfqpoint{2.692609in}{1.875290in}}%
\pgfpathlineto{\pgfqpoint{2.701416in}{1.897919in}}%
\pgfpathlineto{\pgfqpoint{2.710222in}{1.868066in}}%
\pgfpathlineto{\pgfqpoint{2.719029in}{1.912676in}}%
\pgfpathlineto{\pgfqpoint{2.727836in}{1.979240in}}%
\pgfpathlineto{\pgfqpoint{2.736643in}{2.084933in}}%
\pgfpathlineto{\pgfqpoint{2.745450in}{2.010133in}}%
\pgfpathlineto{\pgfqpoint{2.763063in}{1.944946in}}%
\pgfpathlineto{\pgfqpoint{2.771870in}{1.930863in}}%
\pgfpathlineto{\pgfqpoint{2.780677in}{1.967575in}}%
\pgfpathlineto{\pgfqpoint{2.789484in}{1.926394in}}%
\pgfpathlineto{\pgfqpoint{2.798291in}{1.960716in}}%
\pgfpathlineto{\pgfqpoint{2.807097in}{1.912002in}}%
\pgfpathlineto{\pgfqpoint{2.815904in}{1.953520in}}%
\pgfpathlineto{\pgfqpoint{2.824711in}{1.959001in}}%
\pgfpathlineto{\pgfqpoint{2.833518in}{1.877680in}}%
\pgfpathlineto{\pgfqpoint{2.842325in}{1.948010in}}%
\pgfpathlineto{\pgfqpoint{2.851132in}{2.044792in}}%
\pgfpathlineto{\pgfqpoint{2.859938in}{2.023850in}}%
\pgfpathlineto{\pgfqpoint{2.868745in}{1.927771in}}%
\pgfpathlineto{\pgfqpoint{2.877552in}{1.867391in}}%
\pgfpathlineto{\pgfqpoint{2.886359in}{1.842683in}}%
\pgfpathlineto{\pgfqpoint{2.895166in}{1.886253in}}%
\pgfpathlineto{\pgfqpoint{2.903972in}{1.960716in}}%
\pgfpathlineto{\pgfqpoint{2.912779in}{1.984750in}}%
\pgfpathlineto{\pgfqpoint{2.921586in}{1.949753in}}%
\pgfpathlineto{\pgfqpoint{2.930393in}{1.874588in}}%
\pgfpathlineto{\pgfqpoint{2.939200in}{1.881812in}}%
\pgfpathlineto{\pgfqpoint{2.948007in}{1.916106in}}%
\pgfpathlineto{\pgfqpoint{2.956813in}{1.993323in}}%
\pgfpathlineto{\pgfqpoint{2.965620in}{2.004989in}}%
\pgfpathlineto{\pgfqpoint{2.974427in}{2.034504in}}%
\pgfpathlineto{\pgfqpoint{2.983234in}{1.916808in}}%
\pgfpathlineto{\pgfqpoint{2.992041in}{1.882149in}}%
\pgfpathlineto{\pgfqpoint{3.000847in}{1.862585in}}%
\pgfpathlineto{\pgfqpoint{3.009654in}{1.866689in}}%
\pgfpathlineto{\pgfqpoint{3.018461in}{1.849204in}}%
\pgfpathlineto{\pgfqpoint{3.027268in}{1.840968in}}%
\pgfpathlineto{\pgfqpoint{3.044882in}{1.838551in}}%
\pgfpathlineto{\pgfqpoint{3.053688in}{1.806309in}}%
\pgfpathlineto{\pgfqpoint{3.062495in}{1.801502in}}%
\pgfpathlineto{\pgfqpoint{3.071302in}{1.855023in}}%
\pgfpathlineto{\pgfqpoint{3.080109in}{1.858115in}}%
\pgfpathlineto{\pgfqpoint{3.088916in}{1.826211in}}%
\pgfpathlineto{\pgfqpoint{3.097722in}{1.783652in}}%
\pgfpathlineto{\pgfqpoint{3.106529in}{1.835487in}}%
\pgfpathlineto{\pgfqpoint{3.115336in}{1.897581in}}%
\pgfpathlineto{\pgfqpoint{3.124143in}{1.848502in}}%
\pgfpathlineto{\pgfqpoint{3.132950in}{1.782978in}}%
\pgfpathlineto{\pgfqpoint{3.141757in}{1.775416in}}%
\pgfpathlineto{\pgfqpoint{3.150563in}{1.791214in}}%
\pgfpathlineto{\pgfqpoint{3.159370in}{1.817300in}}%
\pgfpathlineto{\pgfqpoint{3.168177in}{1.837539in}}%
\pgfpathlineto{\pgfqpoint{3.176984in}{1.915094in}}%
\pgfpathlineto{\pgfqpoint{3.185791in}{1.961053in}}%
\pgfpathlineto{\pgfqpoint{3.194597in}{1.985762in}}%
\pgfpathlineto{\pgfqpoint{3.203404in}{1.956612in}}%
\pgfpathlineto{\pgfqpoint{3.212211in}{1.964848in}}%
\pgfpathlineto{\pgfqpoint{3.221018in}{1.942529in}}%
\pgfpathlineto{\pgfqpoint{3.229825in}{1.967237in}}%
\pgfpathlineto{\pgfqpoint{3.238632in}{2.035179in}}%
\pgfpathlineto{\pgfqpoint{3.247438in}{2.011510in}}%
\pgfpathlineto{\pgfqpoint{3.256245in}{1.894152in}}%
\pgfpathlineto{\pgfqpoint{3.265052in}{1.929823in}}%
\pgfpathlineto{\pgfqpoint{3.273859in}{1.852971in}}%
\pgfpathlineto{\pgfqpoint{3.282666in}{1.882486in}}%
\pgfpathlineto{\pgfqpoint{3.291472in}{1.809738in}}%
\pgfpathlineto{\pgfqpoint{3.300279in}{1.919198in}}%
\pgfpathlineto{\pgfqpoint{3.309086in}{1.942529in}}%
\pgfpathlineto{\pgfqpoint{3.317893in}{2.037596in}}%
\pgfpathlineto{\pgfqpoint{3.326700in}{2.025903in}}%
\pgfpathlineto{\pgfqpoint{3.335507in}{2.070878in}}%
\pgfpathlineto{\pgfqpoint{3.344313in}{2.091117in}}%
\pgfpathlineto{\pgfqpoint{3.353120in}{2.095559in}}%
\pgfpathlineto{\pgfqpoint{3.370734in}{2.014237in}}%
\pgfpathlineto{\pgfqpoint{3.379541in}{2.065369in}}%
\pgfpathlineto{\pgfqpoint{3.388347in}{2.023513in}}%
\pgfpathlineto{\pgfqpoint{3.397154in}{2.005326in}}%
\pgfpathlineto{\pgfqpoint{3.405961in}{2.057498in}}%
\pgfpathlineto{\pgfqpoint{3.414768in}{1.959001in}}%
\pgfpathlineto{\pgfqpoint{3.423575in}{1.967575in}}%
\pgfpathlineto{\pgfqpoint{3.432382in}{1.898256in}}%
\pgfpathlineto{\pgfqpoint{3.441188in}{1.891763in}}%
\pgfpathlineto{\pgfqpoint{3.449995in}{1.808361in}}%
\pgfpathlineto{\pgfqpoint{3.458802in}{1.788122in}}%
\pgfpathlineto{\pgfqpoint{3.467609in}{1.831692in}}%
\pgfpathlineto{\pgfqpoint{3.476416in}{1.837202in}}%
\pgfpathlineto{\pgfqpoint{3.485222in}{1.817974in}}%
\pgfpathlineto{\pgfqpoint{3.494029in}{1.892100in}}%
\pgfpathlineto{\pgfqpoint{3.502836in}{1.905480in}}%
\pgfpathlineto{\pgfqpoint{3.511643in}{1.915768in}}%
\pgfpathlineto{\pgfqpoint{3.520450in}{1.867729in}}%
\pgfpathlineto{\pgfqpoint{3.529257in}{2.043078in}}%
\pgfpathlineto{\pgfqpoint{3.538063in}{1.913716in}}%
\pgfpathlineto{\pgfqpoint{3.546870in}{1.868403in}}%
\pgfpathlineto{\pgfqpoint{3.555677in}{1.949050in}}%
\pgfpathlineto{\pgfqpoint{3.564484in}{2.013225in}}%
\pgfpathlineto{\pgfqpoint{3.573291in}{2.004286in}}%
\pgfpathlineto{\pgfqpoint{3.582097in}{2.014940in}}%
\pgfpathlineto{\pgfqpoint{3.590904in}{1.956246in}}%
\pgfpathlineto{\pgfqpoint{3.599711in}{1.914391in}}%
\pgfpathlineto{\pgfqpoint{3.608518in}{1.898959in}}%
\pgfpathlineto{\pgfqpoint{3.617325in}{1.902388in}}%
\pgfpathlineto{\pgfqpoint{3.626132in}{1.987139in}}%
\pgfpathlineto{\pgfqpoint{3.634938in}{1.934293in}}%
\pgfpathlineto{\pgfqpoint{3.643745in}{1.979943in}}%
\pgfpathlineto{\pgfqpoint{3.652552in}{1.924005in}}%
\pgfpathlineto{\pgfqpoint{3.661359in}{1.928474in}}%
\pgfpathlineto{\pgfqpoint{3.670166in}{1.907532in}}%
\pgfpathlineto{\pgfqpoint{3.678972in}{1.928474in}}%
\pgfpathlineto{\pgfqpoint{3.687779in}{1.920575in}}%
\pgfpathlineto{\pgfqpoint{3.696586in}{1.991946in}}%
\pgfpathlineto{\pgfqpoint{3.705393in}{2.026943in}}%
\pgfpathlineto{\pgfqpoint{3.714200in}{1.905480in}}%
\pgfpathlineto{\pgfqpoint{3.723007in}{1.912676in}}%
\pgfpathlineto{\pgfqpoint{3.731813in}{1.987814in}}%
\pgfpathlineto{\pgfqpoint{3.740620in}{2.027983in}}%
\pgfpathlineto{\pgfqpoint{3.749427in}{2.006704in}}%
\pgfpathlineto{\pgfqpoint{3.758234in}{1.996753in}}%
\pgfpathlineto{\pgfqpoint{3.767041in}{1.996415in}}%
\pgfpathlineto{\pgfqpoint{3.775847in}{1.877342in}}%
\pgfpathlineto{\pgfqpoint{3.793461in}{1.922627in}}%
\pgfpathlineto{\pgfqpoint{3.802268in}{1.824159in}}%
\pgfpathlineto{\pgfqpoint{3.811075in}{1.793603in}}%
\pgfpathlineto{\pgfqpoint{3.819882in}{1.818649in}}%
\pgfpathlineto{\pgfqpoint{3.828688in}{1.852296in}}%
\pgfpathlineto{\pgfqpoint{3.837495in}{1.837202in}}%
\pgfpathlineto{\pgfqpoint{3.846302in}{1.877680in}}%
\pgfpathlineto{\pgfqpoint{3.855109in}{1.857103in}}%
\pgfpathlineto{\pgfqpoint{3.863916in}{1.899971in}}%
\pgfpathlineto{\pgfqpoint{3.872722in}{1.859493in}}%
\pgfpathlineto{\pgfqpoint{3.881529in}{1.855023in}}%
\pgfpathlineto{\pgfqpoint{3.890336in}{1.952480in}}%
\pgfpathlineto{\pgfqpoint{3.899143in}{1.880069in}}%
\pgfpathlineto{\pgfqpoint{3.907950in}{1.892437in}}%
\pgfpathlineto{\pgfqpoint{3.916757in}{1.854686in}}%
\pgfpathlineto{\pgfqpoint{3.925563in}{1.960041in}}%
\pgfpathlineto{\pgfqpoint{3.934370in}{1.924679in}}%
\pgfpathlineto{\pgfqpoint{3.943177in}{1.969992in}}%
\pgfpathlineto{\pgfqpoint{3.951984in}{1.887968in}}%
\pgfpathlineto{\pgfqpoint{3.960791in}{2.006704in}}%
\pgfpathlineto{\pgfqpoint{3.969597in}{1.946661in}}%
\pgfpathlineto{\pgfqpoint{3.978404in}{1.950090in}}%
\pgfpathlineto{\pgfqpoint{3.987211in}{1.896204in}}%
\pgfpathlineto{\pgfqpoint{3.996018in}{1.906492in}}%
\pgfpathlineto{\pgfqpoint{4.004825in}{1.990569in}}%
\pgfpathlineto{\pgfqpoint{4.013632in}{1.954897in}}%
\pgfpathlineto{\pgfqpoint{4.022438in}{1.976148in}}%
\pgfpathlineto{\pgfqpoint{4.031245in}{1.955909in}}%
\pgfpathlineto{\pgfqpoint{4.040052in}{1.994701in}}%
\pgfpathlineto{\pgfqpoint{4.048859in}{2.009458in}}%
\pgfpathlineto{\pgfqpoint{4.057666in}{1.974433in}}%
\pgfpathlineto{\pgfqpoint{4.066472in}{1.910962in}}%
\pgfpathlineto{\pgfqpoint{4.075279in}{1.898621in}}%
\pgfpathlineto{\pgfqpoint{4.084086in}{1.868066in}}%
\pgfpathlineto{\pgfqpoint{4.092893in}{1.938059in}}%
\pgfpathlineto{\pgfqpoint{4.101700in}{1.843358in}}%
\pgfpathlineto{\pgfqpoint{4.110507in}{1.868066in}}%
\pgfpathlineto{\pgfqpoint{4.119313in}{1.878720in}}%
\pgfpathlineto{\pgfqpoint{4.128120in}{1.949753in}}%
\pgfpathlineto{\pgfqpoint{4.136927in}{1.872536in}}%
\pgfpathlineto{\pgfqpoint{4.145734in}{1.886590in}}%
\pgfpathlineto{\pgfqpoint{4.154541in}{1.916808in}}%
\pgfpathlineto{\pgfqpoint{4.163347in}{1.904440in}}%
\pgfpathlineto{\pgfqpoint{4.172154in}{1.861910in}}%
\pgfpathlineto{\pgfqpoint{4.180961in}{1.915768in}}%
\pgfpathlineto{\pgfqpoint{4.189768in}{1.942529in}}%
\pgfpathlineto{\pgfqpoint{4.198575in}{1.889345in}}%
\pgfpathlineto{\pgfqpoint{4.207382in}{1.891397in}}%
\pgfpathlineto{\pgfqpoint{4.216188in}{1.967575in}}%
\pgfpathlineto{\pgfqpoint{4.224995in}{1.972381in}}%
\pgfpathlineto{\pgfqpoint{4.233802in}{1.978566in}}%
\pgfpathlineto{\pgfqpoint{4.242609in}{1.979943in}}%
\pgfpathlineto{\pgfqpoint{4.251416in}{2.019746in}}%
\pgfpathlineto{\pgfqpoint{4.260222in}{2.047182in}}%
\pgfpathlineto{\pgfqpoint{4.277836in}{1.986802in}}%
\pgfpathlineto{\pgfqpoint{4.286643in}{1.896907in}}%
\pgfpathlineto{\pgfqpoint{4.295450in}{1.858818in}}%
\pgfpathlineto{\pgfqpoint{4.304257in}{1.909922in}}%
\pgfpathlineto{\pgfqpoint{4.313063in}{1.951468in}}%
\pgfpathlineto{\pgfqpoint{4.321870in}{1.867391in}}%
\pgfpathlineto{\pgfqpoint{4.330677in}{1.843723in}}%
\pgfpathlineto{\pgfqpoint{4.339484in}{1.865677in}}%
\pgfpathlineto{\pgfqpoint{4.348291in}{1.908207in}}%
\pgfpathlineto{\pgfqpoint{4.357097in}{1.911299in}}%
\pgfpathlineto{\pgfqpoint{4.365904in}{1.942192in}}%
\pgfpathlineto{\pgfqpoint{4.374711in}{2.034504in}}%
\pgfpathlineto{\pgfqpoint{4.383518in}{1.985762in}}%
\pgfpathlineto{\pgfqpoint{4.392325in}{1.947336in}}%
\pgfpathlineto{\pgfqpoint{4.401132in}{1.899633in}}%
\pgfpathlineto{\pgfqpoint{4.409938in}{1.925719in}}%
\pgfpathlineto{\pgfqpoint{4.418745in}{2.009093in}}%
\pgfpathlineto{\pgfqpoint{4.427552in}{2.053703in}}%
\pgfpathlineto{\pgfqpoint{4.436359in}{2.082544in}}%
\pgfpathlineto{\pgfqpoint{4.445166in}{2.041026in}}%
\pgfpathlineto{\pgfqpoint{4.453972in}{2.014940in}}%
\pgfpathlineto{\pgfqpoint{4.471586in}{1.923330in}}%
\pgfpathlineto{\pgfqpoint{4.480393in}{1.930526in}}%
\pgfpathlineto{\pgfqpoint{4.489200in}{1.909922in}}%
\pgfpathlineto{\pgfqpoint{4.498007in}{1.867729in}}%
\pgfpathlineto{\pgfqpoint{4.506813in}{1.883864in}}%
\pgfpathlineto{\pgfqpoint{4.515620in}{1.926394in}}%
\pgfpathlineto{\pgfqpoint{4.524427in}{1.877680in}}%
\pgfpathlineto{\pgfqpoint{4.533234in}{1.879394in}}%
\pgfpathlineto{\pgfqpoint{4.542041in}{1.897244in}}%
\pgfpathlineto{\pgfqpoint{4.550847in}{1.882824in}}%
\pgfpathlineto{\pgfqpoint{4.559654in}{1.906155in}}%
\pgfpathlineto{\pgfqpoint{4.568461in}{1.926057in}}%
\pgfpathlineto{\pgfqpoint{4.577268in}{1.976148in}}%
\pgfpathlineto{\pgfqpoint{4.586075in}{1.987476in}}%
\pgfpathlineto{\pgfqpoint{4.594882in}{1.891060in}}%
\pgfpathlineto{\pgfqpoint{4.603688in}{1.944581in}}%
\pgfpathlineto{\pgfqpoint{4.612495in}{1.894827in}}%
\pgfpathlineto{\pgfqpoint{4.621302in}{1.860870in}}%
\pgfpathlineto{\pgfqpoint{4.630109in}{1.893815in}}%
\pgfpathlineto{\pgfqpoint{4.638916in}{1.906858in}}%
\pgfpathlineto{\pgfqpoint{4.647722in}{1.989894in}}%
\pgfpathlineto{\pgfqpoint{4.656529in}{1.996415in}}%
\pgfpathlineto{\pgfqpoint{4.665336in}{1.980955in}}%
\pgfpathlineto{\pgfqpoint{4.674143in}{1.930863in}}%
\pgfpathlineto{\pgfqpoint{4.682950in}{1.810413in}}%
\pgfpathlineto{\pgfqpoint{4.691757in}{1.843723in}}%
\pgfpathlineto{\pgfqpoint{4.700563in}{1.839254in}}%
\pgfpathlineto{\pgfqpoint{4.709370in}{1.806309in}}%
\pgfpathlineto{\pgfqpoint{4.718177in}{1.701656in}}%
\pgfpathlineto{\pgfqpoint{4.726984in}{1.707138in}}%
\pgfpathlineto{\pgfqpoint{4.735791in}{1.806984in}}%
\pgfpathlineto{\pgfqpoint{4.744597in}{1.854011in}}%
\pgfpathlineto{\pgfqpoint{4.753404in}{1.932241in}}%
\pgfpathlineto{\pgfqpoint{4.762211in}{1.945284in}}%
\pgfpathlineto{\pgfqpoint{4.771018in}{1.951102in}}%
\pgfpathlineto{\pgfqpoint{4.779825in}{1.988854in}}%
\pgfpathlineto{\pgfqpoint{4.788632in}{1.950090in}}%
\pgfpathlineto{\pgfqpoint{4.797438in}{1.905480in}}%
\pgfpathlineto{\pgfqpoint{4.806245in}{1.878720in}}%
\pgfpathlineto{\pgfqpoint{4.815052in}{1.882824in}}%
\pgfpathlineto{\pgfqpoint{4.823859in}{1.883526in}}%
\pgfpathlineto{\pgfqpoint{4.832666in}{1.816934in}}%
\pgfpathlineto{\pgfqpoint{4.841472in}{1.808024in}}%
\pgfpathlineto{\pgfqpoint{4.850279in}{1.848867in}}%
\pgfpathlineto{\pgfqpoint{4.859086in}{1.910962in}}%
\pgfpathlineto{\pgfqpoint{4.867893in}{1.937047in}}%
\pgfpathlineto{\pgfqpoint{4.876700in}{1.905143in}}%
\pgfpathlineto{\pgfqpoint{4.885507in}{1.908572in}}%
\pgfpathlineto{\pgfqpoint{4.894313in}{1.983035in}}%
\pgfpathlineto{\pgfqpoint{4.903120in}{1.951468in}}%
\pgfpathlineto{\pgfqpoint{4.911927in}{1.906155in}}%
\pgfpathlineto{\pgfqpoint{4.920734in}{1.921250in}}%
\pgfpathlineto{\pgfqpoint{4.929541in}{1.918158in}}%
\pgfpathlineto{\pgfqpoint{4.938347in}{1.977863in}}%
\pgfpathlineto{\pgfqpoint{4.947154in}{1.987139in}}%
\pgfpathlineto{\pgfqpoint{4.955961in}{1.908572in}}%
\pgfpathlineto{\pgfqpoint{4.964768in}{1.886590in}}%
\pgfpathlineto{\pgfqpoint{4.973575in}{1.922964in}}%
\pgfpathlineto{\pgfqpoint{4.982382in}{1.891060in}}%
\pgfpathlineto{\pgfqpoint{4.999995in}{1.883161in}}%
\pgfpathlineto{\pgfqpoint{5.008802in}{1.936007in}}%
\pgfpathlineto{\pgfqpoint{5.017609in}{1.932915in}}%
\pgfpathlineto{\pgfqpoint{5.026416in}{1.904103in}}%
\pgfpathlineto{\pgfqpoint{5.035222in}{1.906858in}}%
\pgfpathlineto{\pgfqpoint{5.044029in}{1.854686in}}%
\pgfpathlineto{\pgfqpoint{5.052836in}{1.900673in}}%
\pgfpathlineto{\pgfqpoint{5.061643in}{1.867054in}}%
\pgfpathlineto{\pgfqpoint{5.070450in}{1.875628in}}%
\pgfpathlineto{\pgfqpoint{5.079257in}{1.942866in}}%
\pgfpathlineto{\pgfqpoint{5.088063in}{1.903428in}}%
\pgfpathlineto{\pgfqpoint{5.096870in}{1.957961in}}%
\pgfpathlineto{\pgfqpoint{5.105677in}{1.993661in}}%
\pgfpathlineto{\pgfqpoint{5.114484in}{2.004989in}}%
\pgfpathlineto{\pgfqpoint{5.123291in}{1.964145in}}%
\pgfpathlineto{\pgfqpoint{5.132097in}{1.873210in}}%
\pgfpathlineto{\pgfqpoint{5.140904in}{1.905143in}}%
\pgfpathlineto{\pgfqpoint{5.149711in}{1.950428in}}%
\pgfpathlineto{\pgfqpoint{5.158518in}{1.905818in}}%
\pgfpathlineto{\pgfqpoint{5.167325in}{1.905480in}}%
\pgfpathlineto{\pgfqpoint{5.176132in}{1.939437in}}%
\pgfpathlineto{\pgfqpoint{5.184938in}{1.934995in}}%
\pgfpathlineto{\pgfqpoint{5.193745in}{1.940140in}}%
\pgfpathlineto{\pgfqpoint{5.202552in}{1.932241in}}%
\pgfpathlineto{\pgfqpoint{5.211359in}{2.000519in}}%
\pgfpathlineto{\pgfqpoint{5.228972in}{1.942866in}}%
\pgfpathlineto{\pgfqpoint{5.237779in}{1.884538in}}%
\pgfpathlineto{\pgfqpoint{5.246586in}{1.884538in}}%
\pgfpathlineto{\pgfqpoint{5.255393in}{1.935333in}}%
\pgfpathlineto{\pgfqpoint{5.264200in}{1.894489in}}%
\pgfpathlineto{\pgfqpoint{5.273007in}{1.961419in}}%
\pgfpathlineto{\pgfqpoint{5.281813in}{1.931566in}}%
\pgfpathlineto{\pgfqpoint{5.290620in}{1.953857in}}%
\pgfpathlineto{\pgfqpoint{5.299427in}{1.843723in}}%
\pgfpathlineto{\pgfqpoint{5.308234in}{1.783315in}}%
\pgfpathlineto{\pgfqpoint{5.317041in}{1.786407in}}%
\pgfpathlineto{\pgfqpoint{5.325847in}{1.841643in}}%
\pgfpathlineto{\pgfqpoint{5.334654in}{1.819352in}}%
\pgfpathlineto{\pgfqpoint{5.343461in}{1.889008in}}%
\pgfpathlineto{\pgfqpoint{5.352268in}{1.927097in}}%
\pgfpathlineto{\pgfqpoint{5.361075in}{1.897581in}}%
\pgfpathlineto{\pgfqpoint{5.369882in}{1.943232in}}%
\pgfpathlineto{\pgfqpoint{5.378688in}{1.914728in}}%
\pgfpathlineto{\pgfqpoint{5.387495in}{1.904103in}}%
\pgfpathlineto{\pgfqpoint{5.396302in}{1.952817in}}%
\pgfpathlineto{\pgfqpoint{5.405109in}{1.947673in}}%
\pgfpathlineto{\pgfqpoint{5.413916in}{2.013225in}}%
\pgfpathlineto{\pgfqpoint{5.422722in}{1.981320in}}%
\pgfpathlineto{\pgfqpoint{5.431529in}{1.974096in}}%
\pgfpathlineto{\pgfqpoint{5.440336in}{2.014940in}}%
\pgfpathlineto{\pgfqpoint{5.449143in}{1.881109in}}%
\pgfpathlineto{\pgfqpoint{5.457950in}{1.874588in}}%
\pgfpathlineto{\pgfqpoint{5.466757in}{1.887968in}}%
\pgfpathlineto{\pgfqpoint{5.475563in}{1.921250in}}%
\pgfpathlineto{\pgfqpoint{5.484370in}{1.921250in}}%
\pgfpathlineto{\pgfqpoint{5.493177in}{1.954897in}}%
\pgfpathlineto{\pgfqpoint{5.501984in}{1.945958in}}%
\pgfpathlineto{\pgfqpoint{5.510791in}{1.946998in}}%
\pgfpathlineto{\pgfqpoint{5.519597in}{1.866351in}}%
\pgfpathlineto{\pgfqpoint{5.528404in}{1.887630in}}%
\pgfpathlineto{\pgfqpoint{5.546018in}{1.862585in}}%
\pgfpathlineto{\pgfqpoint{5.563632in}{1.904103in}}%
\pgfpathlineto{\pgfqpoint{5.572438in}{1.874925in}}%
\pgfpathlineto{\pgfqpoint{5.581245in}{1.914054in}}%
\pgfpathlineto{\pgfqpoint{5.590052in}{1.861910in}}%
\pgfpathlineto{\pgfqpoint{5.598859in}{1.891060in}}%
\pgfpathlineto{\pgfqpoint{5.607666in}{1.853674in}}%
\pgfpathlineto{\pgfqpoint{5.616472in}{1.841643in}}%
\pgfpathlineto{\pgfqpoint{5.625279in}{1.934293in}}%
\pgfpathlineto{\pgfqpoint{5.634086in}{1.840266in}}%
\pgfpathlineto{\pgfqpoint{5.642893in}{1.831017in}}%
\pgfpathlineto{\pgfqpoint{5.651700in}{1.905143in}}%
\pgfpathlineto{\pgfqpoint{5.660507in}{1.890723in}}%
\pgfpathlineto{\pgfqpoint{5.669313in}{1.859155in}}%
\pgfpathlineto{\pgfqpoint{5.678120in}{1.875965in}}%
\pgfpathlineto{\pgfqpoint{5.686927in}{1.913716in}}%
\pgfpathlineto{\pgfqpoint{5.695734in}{1.871158in}}%
\pgfpathlineto{\pgfqpoint{5.704541in}{1.960379in}}%
\pgfpathlineto{\pgfqpoint{5.713347in}{1.955235in}}%
\pgfpathlineto{\pgfqpoint{5.730961in}{1.920238in}}%
\pgfpathlineto{\pgfqpoint{5.739768in}{1.898256in}}%
\pgfpathlineto{\pgfqpoint{5.748575in}{1.869443in}}%
\pgfpathlineto{\pgfqpoint{5.757382in}{1.846450in}}%
\pgfpathlineto{\pgfqpoint{5.766188in}{1.811790in}}%
\pgfpathlineto{\pgfqpoint{5.774995in}{1.849542in}}%
\pgfpathlineto{\pgfqpoint{5.783802in}{1.879394in}}%
\pgfpathlineto{\pgfqpoint{5.792609in}{1.840968in}}%
\pgfpathlineto{\pgfqpoint{5.801416in}{1.904777in}}%
\pgfpathlineto{\pgfqpoint{5.810222in}{1.925719in}}%
\pgfpathlineto{\pgfqpoint{5.819029in}{1.872536in}}%
\pgfpathlineto{\pgfqpoint{5.827836in}{1.884538in}}%
\pgfpathlineto{\pgfqpoint{5.836643in}{1.847152in}}%
\pgfpathlineto{\pgfqpoint{5.845450in}{1.956949in}}%
\pgfpathlineto{\pgfqpoint{5.854257in}{1.996050in}}%
\pgfpathlineto{\pgfqpoint{5.863063in}{2.042038in}}%
\pgfpathlineto{\pgfqpoint{5.871870in}{2.101068in}}%
\pgfpathlineto{\pgfqpoint{5.880677in}{2.081166in}}%
\pgfpathlineto{\pgfqpoint{5.889484in}{2.014602in}}%
\pgfpathlineto{\pgfqpoint{5.898291in}{2.036219in}}%
\pgfpathlineto{\pgfqpoint{5.907097in}{1.960379in}}%
\pgfpathlineto{\pgfqpoint{5.915904in}{1.995038in}}%
\pgfpathlineto{\pgfqpoint{5.924711in}{2.012185in}}%
\pgfpathlineto{\pgfqpoint{5.933518in}{1.999507in}}%
\pgfpathlineto{\pgfqpoint{5.942325in}{2.047182in}}%
\pgfpathlineto{\pgfqpoint{5.951132in}{2.086985in}}%
\pgfpathlineto{\pgfqpoint{5.959938in}{2.014940in}}%
\pgfpathlineto{\pgfqpoint{5.968745in}{1.930526in}}%
\pgfpathlineto{\pgfqpoint{5.977552in}{1.969289in}}%
\pgfpathlineto{\pgfqpoint{5.986359in}{1.972719in}}%
\pgfpathlineto{\pgfqpoint{5.995166in}{1.956246in}}%
\pgfpathlineto{\pgfqpoint{6.003972in}{1.920575in}}%
\pgfpathlineto{\pgfqpoint{6.012779in}{1.863962in}}%
\pgfpathlineto{\pgfqpoint{6.021586in}{1.906155in}}%
\pgfpathlineto{\pgfqpoint{6.030393in}{2.010470in}}%
\pgfpathlineto{\pgfqpoint{6.039200in}{2.017666in}}%
\pgfpathlineto{\pgfqpoint{6.048007in}{1.955572in}}%
\pgfpathlineto{\pgfqpoint{6.056813in}{1.882486in}}%
\pgfpathlineto{\pgfqpoint{6.065620in}{1.832732in}}%
\pgfpathlineto{\pgfqpoint{6.074427in}{1.861910in}}%
\pgfpathlineto{\pgfqpoint{6.083234in}{1.935333in}}%
\pgfpathlineto{\pgfqpoint{6.092041in}{1.897244in}}%
\pgfpathlineto{\pgfqpoint{6.100847in}{1.984384in}}%
\pgfpathlineto{\pgfqpoint{6.109654in}{1.931903in}}%
\pgfpathlineto{\pgfqpoint{6.118461in}{1.937047in}}%
\pgfpathlineto{\pgfqpoint{6.136075in}{1.816597in}}%
\pgfpathlineto{\pgfqpoint{6.144882in}{1.901011in}}%
\pgfpathlineto{\pgfqpoint{6.153688in}{1.869781in}}%
\pgfpathlineto{\pgfqpoint{6.162495in}{1.825171in}}%
\pgfpathlineto{\pgfqpoint{6.171302in}{1.925045in}}%
\pgfpathlineto{\pgfqpoint{6.180109in}{1.965523in}}%
\pgfpathlineto{\pgfqpoint{6.188916in}{1.932241in}}%
\pgfpathlineto{\pgfqpoint{6.197722in}{1.921250in}}%
\pgfpathlineto{\pgfqpoint{6.206529in}{1.938762in}}%
\pgfpathlineto{\pgfqpoint{6.215336in}{1.859830in}}%
\pgfpathlineto{\pgfqpoint{6.224143in}{1.885578in}}%
\pgfpathlineto{\pgfqpoint{6.232950in}{1.895529in}}%
\pgfpathlineto{\pgfqpoint{6.241757in}{1.865339in}}%
\pgfpathlineto{\pgfqpoint{6.250563in}{1.868403in}}%
\pgfpathlineto{\pgfqpoint{6.259370in}{1.879732in}}%
\pgfpathlineto{\pgfqpoint{6.276984in}{2.014940in}}%
\pgfpathlineto{\pgfqpoint{6.285791in}{1.926057in}}%
\pgfpathlineto{\pgfqpoint{6.294597in}{1.938059in}}%
\pgfpathlineto{\pgfqpoint{6.303404in}{1.913014in}}%
\pgfpathlineto{\pgfqpoint{6.312211in}{1.962768in}}%
\pgfpathlineto{\pgfqpoint{6.321018in}{1.930526in}}%
\pgfpathlineto{\pgfqpoint{6.329825in}{1.903766in}}%
\pgfpathlineto{\pgfqpoint{6.338632in}{1.906155in}}%
\pgfpathlineto{\pgfqpoint{6.347438in}{1.940814in}}%
\pgfpathlineto{\pgfqpoint{6.356245in}{1.924342in}}%
\pgfpathlineto{\pgfqpoint{6.365052in}{1.987814in}}%
\pgfpathlineto{\pgfqpoint{6.373859in}{1.956246in}}%
\pgfpathlineto{\pgfqpoint{6.382666in}{1.976148in}}%
\pgfpathlineto{\pgfqpoint{6.391472in}{1.893477in}}%
\pgfpathlineto{\pgfqpoint{6.400279in}{1.905143in}}%
\pgfpathlineto{\pgfqpoint{6.409086in}{1.869781in}}%
\pgfpathlineto{\pgfqpoint{6.417893in}{1.875290in}}%
\pgfpathlineto{\pgfqpoint{6.426700in}{1.934630in}}%
\pgfpathlineto{\pgfqpoint{6.435507in}{1.920575in}}%
\pgfpathlineto{\pgfqpoint{6.444313in}{1.972719in}}%
\pgfpathlineto{\pgfqpoint{6.453120in}{1.932915in}}%
\pgfpathlineto{\pgfqpoint{6.461927in}{1.930863in}}%
\pgfpathlineto{\pgfqpoint{6.479541in}{2.009458in}}%
\pgfpathlineto{\pgfqpoint{6.488347in}{2.146690in}}%
\pgfpathlineto{\pgfqpoint{6.497154in}{2.111019in}}%
\pgfpathlineto{\pgfqpoint{6.505961in}{2.122347in}}%
\pgfpathlineto{\pgfqpoint{6.514768in}{2.065369in}}%
\pgfpathlineto{\pgfqpoint{6.523575in}{2.062642in}}%
\pgfpathlineto{\pgfqpoint{6.532382in}{1.998130in}}%
\pgfpathlineto{\pgfqpoint{6.541188in}{1.955909in}}%
\pgfpathlineto{\pgfqpoint{6.549995in}{1.991271in}}%
\pgfpathlineto{\pgfqpoint{6.558802in}{2.006366in}}%
\pgfpathlineto{\pgfqpoint{6.567609in}{1.995713in}}%
\pgfpathlineto{\pgfqpoint{6.576416in}{1.904103in}}%
\pgfpathlineto{\pgfqpoint{6.585222in}{1.930863in}}%
\pgfpathlineto{\pgfqpoint{6.594029in}{1.972381in}}%
\pgfpathlineto{\pgfqpoint{6.602836in}{1.903428in}}%
\pgfpathlineto{\pgfqpoint{6.611643in}{1.868066in}}%
\pgfpathlineto{\pgfqpoint{6.620450in}{1.884538in}}%
\pgfpathlineto{\pgfqpoint{6.629257in}{1.878354in}}%
\pgfpathlineto{\pgfqpoint{6.638063in}{1.852971in}}%
\pgfpathlineto{\pgfqpoint{6.646870in}{1.964483in}}%
\pgfpathlineto{\pgfqpoint{6.655677in}{1.889683in}}%
\pgfpathlineto{\pgfqpoint{6.664484in}{1.947336in}}%
\pgfpathlineto{\pgfqpoint{6.673291in}{1.871495in}}%
\pgfpathlineto{\pgfqpoint{6.682097in}{1.885578in}}%
\pgfpathlineto{\pgfqpoint{6.690904in}{1.865339in}}%
\pgfpathlineto{\pgfqpoint{6.699711in}{1.804932in}}%
\pgfpathlineto{\pgfqpoint{6.708518in}{1.891060in}}%
\pgfpathlineto{\pgfqpoint{6.717325in}{1.868403in}}%
\pgfpathlineto{\pgfqpoint{6.726132in}{1.960379in}}%
\pgfpathlineto{\pgfqpoint{6.734938in}{1.974799in}}%
\pgfpathlineto{\pgfqpoint{6.743745in}{2.125777in}}%
\pgfpathlineto{\pgfqpoint{6.752552in}{2.112031in}}%
\pgfpathlineto{\pgfqpoint{6.761359in}{2.059887in}}%
\pgfpathlineto{\pgfqpoint{6.770166in}{2.052691in}}%
\pgfpathlineto{\pgfqpoint{6.778972in}{1.977188in}}%
\pgfpathlineto{\pgfqpoint{6.787779in}{1.949753in}}%
\pgfpathlineto{\pgfqpoint{6.796586in}{1.878354in}}%
\pgfpathlineto{\pgfqpoint{6.805393in}{1.827588in}}%
\pgfpathlineto{\pgfqpoint{6.814200in}{1.786407in}}%
\pgfpathlineto{\pgfqpoint{6.823007in}{1.844398in}}%
\pgfpathlineto{\pgfqpoint{6.831813in}{1.871861in}}%
\pgfpathlineto{\pgfqpoint{6.840620in}{1.815922in}}%
\pgfpathlineto{\pgfqpoint{6.849427in}{1.946998in}}%
\pgfpathlineto{\pgfqpoint{6.858234in}{1.976851in}}%
\pgfpathlineto{\pgfqpoint{6.867041in}{1.957287in}}%
\pgfpathlineto{\pgfqpoint{6.875847in}{1.909584in}}%
\pgfpathlineto{\pgfqpoint{6.884654in}{1.896907in}}%
\pgfpathlineto{\pgfqpoint{6.893461in}{1.917146in}}%
\pgfpathlineto{\pgfqpoint{6.902268in}{1.954897in}}%
\pgfpathlineto{\pgfqpoint{6.911075in}{1.912339in}}%
\pgfpathlineto{\pgfqpoint{6.919882in}{1.875628in}}%
\pgfpathlineto{\pgfqpoint{6.928688in}{1.917820in}}%
\pgfpathlineto{\pgfqpoint{6.937495in}{1.875290in}}%
\pgfpathlineto{\pgfqpoint{6.946302in}{2.008081in}}%
\pgfpathlineto{\pgfqpoint{6.955109in}{1.985762in}}%
\pgfpathlineto{\pgfqpoint{6.963916in}{2.008756in}}%
\pgfpathlineto{\pgfqpoint{6.972722in}{2.074308in}}%
\pgfpathlineto{\pgfqpoint{6.981529in}{2.060225in}}%
\pgfpathlineto{\pgfqpoint{6.990336in}{2.058847in}}%
\pgfpathlineto{\pgfqpoint{6.999143in}{1.963471in}}%
\pgfpathlineto{\pgfqpoint{7.007950in}{1.984047in}}%
\pgfpathlineto{\pgfqpoint{7.016757in}{1.907195in}}%
\pgfpathlineto{\pgfqpoint{7.025563in}{1.939437in}}%
\pgfpathlineto{\pgfqpoint{7.034370in}{1.902051in}}%
\pgfpathlineto{\pgfqpoint{7.043177in}{1.819352in}}%
\pgfpathlineto{\pgfqpoint{7.051984in}{1.884201in}}%
\pgfpathlineto{\pgfqpoint{7.060791in}{1.908572in}}%
\pgfpathlineto{\pgfqpoint{7.069597in}{1.951805in}}%
\pgfpathlineto{\pgfqpoint{7.078404in}{1.964483in}}%
\pgfpathlineto{\pgfqpoint{7.087211in}{1.959339in}}%
\pgfpathlineto{\pgfqpoint{7.096018in}{1.971004in}}%
\pgfpathlineto{\pgfqpoint{7.104825in}{2.012522in}}%
\pgfpathlineto{\pgfqpoint{7.113632in}{2.047547in}}%
\pgfpathlineto{\pgfqpoint{7.122438in}{2.046844in}}%
\pgfpathlineto{\pgfqpoint{7.131245in}{2.054743in}}%
\pgfpathlineto{\pgfqpoint{7.140052in}{1.971707in}}%
\pgfpathlineto{\pgfqpoint{7.148859in}{1.968277in}}%
\pgfpathlineto{\pgfqpoint{7.157666in}{1.981658in}}%
\pgfpathlineto{\pgfqpoint{7.166472in}{2.019409in}}%
\pgfpathlineto{\pgfqpoint{7.175279in}{1.964145in}}%
\pgfpathlineto{\pgfqpoint{7.184086in}{1.999845in}}%
\pgfpathlineto{\pgfqpoint{7.192893in}{1.983710in}}%
\pgfpathlineto{\pgfqpoint{7.201700in}{1.943906in}}%
\pgfpathlineto{\pgfqpoint{7.210507in}{1.932241in}}%
\pgfpathlineto{\pgfqpoint{7.219313in}{1.899296in}}%
\pgfpathlineto{\pgfqpoint{7.228120in}{1.844398in}}%
\pgfpathlineto{\pgfqpoint{7.236927in}{1.875965in}}%
\pgfpathlineto{\pgfqpoint{7.245734in}{1.916106in}}%
\pgfpathlineto{\pgfqpoint{7.254541in}{1.931201in}}%
\pgfpathlineto{\pgfqpoint{7.263347in}{1.858453in}}%
\pgfpathlineto{\pgfqpoint{7.272154in}{2.013225in}}%
\pgfpathlineto{\pgfqpoint{7.280961in}{1.873913in}}%
\pgfpathlineto{\pgfqpoint{7.289768in}{1.921953in}}%
\pgfpathlineto{\pgfqpoint{7.298575in}{1.916106in}}%
\pgfpathlineto{\pgfqpoint{7.307382in}{1.847827in}}%
\pgfpathlineto{\pgfqpoint{7.316188in}{1.883161in}}%
\pgfpathlineto{\pgfqpoint{7.324995in}{1.926394in}}%
\pgfpathlineto{\pgfqpoint{7.333802in}{1.928474in}}%
\pgfpathlineto{\pgfqpoint{7.342609in}{1.982670in}}%
\pgfpathlineto{\pgfqpoint{7.351416in}{1.995038in}}%
\pgfpathlineto{\pgfqpoint{7.360222in}{1.872873in}}%
\pgfpathlineto{\pgfqpoint{7.369029in}{1.935333in}}%
\pgfpathlineto{\pgfqpoint{7.377836in}{1.886956in}}%
\pgfpathlineto{\pgfqpoint{7.386643in}{1.925719in}}%
\pgfpathlineto{\pgfqpoint{7.395450in}{1.893815in}}%
\pgfpathlineto{\pgfqpoint{7.404257in}{1.908910in}}%
\pgfpathlineto{\pgfqpoint{7.413063in}{1.975136in}}%
\pgfpathlineto{\pgfqpoint{7.421870in}{1.984384in}}%
\pgfpathlineto{\pgfqpoint{7.430677in}{1.874588in}}%
\pgfpathlineto{\pgfqpoint{7.439484in}{1.900673in}}%
\pgfpathlineto{\pgfqpoint{7.448291in}{1.886590in}}%
\pgfpathlineto{\pgfqpoint{7.457097in}{1.875628in}}%
\pgfpathlineto{\pgfqpoint{7.465904in}{1.828965in}}%
\pgfpathlineto{\pgfqpoint{7.474711in}{1.839928in}}%
\pgfpathlineto{\pgfqpoint{7.483518in}{1.890723in}}%
\pgfpathlineto{\pgfqpoint{7.492325in}{1.898621in}}%
\pgfpathlineto{\pgfqpoint{7.501132in}{1.889683in}}%
\pgfpathlineto{\pgfqpoint{7.509938in}{1.815585in}}%
\pgfpathlineto{\pgfqpoint{7.518745in}{1.816934in}}%
\pgfpathlineto{\pgfqpoint{7.527552in}{1.804932in}}%
\pgfpathlineto{\pgfqpoint{7.536359in}{1.873913in}}%
\pgfpathlineto{\pgfqpoint{7.545166in}{1.860870in}}%
\pgfpathlineto{\pgfqpoint{7.553972in}{1.893815in}}%
\pgfpathlineto{\pgfqpoint{7.562779in}{1.863259in}}%
\pgfpathlineto{\pgfqpoint{7.571586in}{1.792591in}}%
\pgfpathlineto{\pgfqpoint{7.580393in}{1.830315in}}%
\pgfpathlineto{\pgfqpoint{7.589200in}{1.894152in}}%
\pgfpathlineto{\pgfqpoint{7.598007in}{1.899971in}}%
\pgfpathlineto{\pgfqpoint{7.606813in}{1.865677in}}%
\pgfpathlineto{\pgfqpoint{7.615620in}{1.940140in}}%
\pgfpathlineto{\pgfqpoint{7.624427in}{2.027280in}}%
\pgfpathlineto{\pgfqpoint{7.633234in}{2.025228in}}%
\pgfpathlineto{\pgfqpoint{7.642041in}{1.986802in}}%
\pgfpathlineto{\pgfqpoint{7.650847in}{1.933618in}}%
\pgfpathlineto{\pgfqpoint{7.659654in}{1.948376in}}%
\pgfpathlineto{\pgfqpoint{7.668461in}{1.983372in}}%
\pgfpathlineto{\pgfqpoint{7.686075in}{1.900336in}}%
\pgfpathlineto{\pgfqpoint{7.694882in}{1.890723in}}%
\pgfpathlineto{\pgfqpoint{7.703688in}{1.890385in}}%
\pgfpathlineto{\pgfqpoint{7.712495in}{1.878017in}}%
\pgfpathlineto{\pgfqpoint{7.721302in}{1.863259in}}%
\pgfpathlineto{\pgfqpoint{7.730109in}{1.913014in}}%
\pgfpathlineto{\pgfqpoint{7.738916in}{1.919535in}}%
\pgfpathlineto{\pgfqpoint{7.747722in}{1.992986in}}%
\pgfpathlineto{\pgfqpoint{7.756529in}{1.981995in}}%
\pgfpathlineto{\pgfqpoint{7.765336in}{2.024188in}}%
\pgfpathlineto{\pgfqpoint{7.774143in}{2.003949in}}%
\pgfpathlineto{\pgfqpoint{7.782950in}{2.004652in}}%
\pgfpathlineto{\pgfqpoint{7.791757in}{1.944581in}}%
\pgfpathlineto{\pgfqpoint{7.800563in}{1.941517in}}%
\pgfpathlineto{\pgfqpoint{7.809370in}{1.906492in}}%
\pgfpathlineto{\pgfqpoint{7.818177in}{1.865677in}}%
\pgfpathlineto{\pgfqpoint{7.826984in}{1.859155in}}%
\pgfpathlineto{\pgfqpoint{7.835791in}{1.946296in}}%
\pgfpathlineto{\pgfqpoint{7.844597in}{1.857778in}}%
\pgfpathlineto{\pgfqpoint{7.853404in}{1.890020in}}%
\pgfpathlineto{\pgfqpoint{7.862211in}{1.862585in}}%
\pgfpathlineto{\pgfqpoint{7.871018in}{1.868769in}}%
\pgfpathlineto{\pgfqpoint{7.879825in}{1.861207in}}%
\pgfpathlineto{\pgfqpoint{7.888632in}{1.800125in}}%
\pgfpathlineto{\pgfqpoint{7.897438in}{1.787082in}}%
\pgfpathlineto{\pgfqpoint{7.906245in}{1.747278in}}%
\pgfpathlineto{\pgfqpoint{7.915052in}{1.758607in}}%
\pgfpathlineto{\pgfqpoint{7.923859in}{1.822078in}}%
\pgfpathlineto{\pgfqpoint{7.932666in}{1.815585in}}%
\pgfpathlineto{\pgfqpoint{7.941472in}{1.713996in}}%
\pgfpathlineto{\pgfqpoint{7.950279in}{1.781263in}}%
\pgfpathlineto{\pgfqpoint{7.959086in}{1.810778in}}%
\pgfpathlineto{\pgfqpoint{7.967893in}{1.820364in}}%
\pgfpathlineto{\pgfqpoint{7.976700in}{1.835121in}}%
\pgfpathlineto{\pgfqpoint{7.985507in}{1.902725in}}%
\pgfpathlineto{\pgfqpoint{7.994313in}{1.945621in}}%
\pgfpathlineto{\pgfqpoint{8.003120in}{1.962768in}}%
\pgfpathlineto{\pgfqpoint{8.011927in}{1.961419in}}%
\pgfpathlineto{\pgfqpoint{8.020734in}{1.852634in}}%
\pgfpathlineto{\pgfqpoint{8.029541in}{1.921953in}}%
\pgfpathlineto{\pgfqpoint{8.038347in}{1.881812in}}%
\pgfpathlineto{\pgfqpoint{8.047154in}{1.903766in}}%
\pgfpathlineto{\pgfqpoint{8.055961in}{1.891397in}}%
\pgfpathlineto{\pgfqpoint{8.064768in}{1.846450in}}%
\pgfpathlineto{\pgfqpoint{8.073575in}{1.841643in}}%
\pgfpathlineto{\pgfqpoint{8.082382in}{1.858115in}}%
\pgfpathlineto{\pgfqpoint{8.091188in}{1.852971in}}%
\pgfpathlineto{\pgfqpoint{8.099995in}{1.919872in}}%
\pgfpathlineto{\pgfqpoint{8.108802in}{1.953857in}}%
\pgfpathlineto{\pgfqpoint{8.117609in}{1.955572in}}%
\pgfpathlineto{\pgfqpoint{8.126416in}{1.989191in}}%
\pgfpathlineto{\pgfqpoint{8.135222in}{1.934293in}}%
\pgfpathlineto{\pgfqpoint{8.144029in}{1.896907in}}%
\pgfpathlineto{\pgfqpoint{8.152836in}{1.878354in}}%
\pgfpathlineto{\pgfqpoint{8.161643in}{1.917146in}}%
\pgfpathlineto{\pgfqpoint{8.170450in}{1.990906in}}%
\pgfpathlineto{\pgfqpoint{8.179257in}{1.935333in}}%
\pgfpathlineto{\pgfqpoint{8.188063in}{1.859493in}}%
\pgfpathlineto{\pgfqpoint{8.196870in}{1.840266in}}%
\pgfpathlineto{\pgfqpoint{8.205677in}{1.829303in}}%
\pgfpathlineto{\pgfqpoint{8.214484in}{1.839254in}}%
\pgfpathlineto{\pgfqpoint{8.223291in}{1.833407in}}%
\pgfpathlineto{\pgfqpoint{8.232097in}{1.882149in}}%
\pgfpathlineto{\pgfqpoint{8.240904in}{1.866689in}}%
\pgfpathlineto{\pgfqpoint{8.249711in}{1.841643in}}%
\pgfpathlineto{\pgfqpoint{8.258518in}{1.855726in}}%
\pgfpathlineto{\pgfqpoint{8.276132in}{1.802879in}}%
\pgfpathlineto{\pgfqpoint{8.284938in}{1.884201in}}%
\pgfpathlineto{\pgfqpoint{8.293745in}{1.869781in}}%
\pgfpathlineto{\pgfqpoint{8.302552in}{1.870821in}}%
\pgfpathlineto{\pgfqpoint{8.320166in}{1.765831in}}%
\pgfpathlineto{\pgfqpoint{8.328972in}{1.836499in}}%
\pgfpathlineto{\pgfqpoint{8.337779in}{1.804932in}}%
\pgfpathlineto{\pgfqpoint{8.346586in}{1.762373in}}%
\pgfpathlineto{\pgfqpoint{8.355393in}{1.768895in}}%
\pgfpathlineto{\pgfqpoint{8.364200in}{1.807349in}}%
\pgfpathlineto{\pgfqpoint{8.373007in}{1.796695in}}%
\pgfpathlineto{\pgfqpoint{8.381813in}{1.821404in}}%
\pgfpathlineto{\pgfqpoint{8.390620in}{1.948376in}}%
\pgfpathlineto{\pgfqpoint{8.399427in}{1.907870in}}%
\pgfpathlineto{\pgfqpoint{8.408234in}{1.848502in}}%
\pgfpathlineto{\pgfqpoint{8.417041in}{1.803554in}}%
\pgfpathlineto{\pgfqpoint{8.425847in}{1.803920in}}%
\pgfpathlineto{\pgfqpoint{8.434654in}{1.815220in}}%
\pgfpathlineto{\pgfqpoint{8.443461in}{1.893815in}}%
\pgfpathlineto{\pgfqpoint{8.452268in}{1.880434in}}%
\pgfpathlineto{\pgfqpoint{8.461075in}{1.864299in}}%
\pgfpathlineto{\pgfqpoint{8.469882in}{1.870821in}}%
\pgfpathlineto{\pgfqpoint{8.478688in}{1.923330in}}%
\pgfpathlineto{\pgfqpoint{8.487495in}{1.953183in}}%
\pgfpathlineto{\pgfqpoint{8.496302in}{1.998130in}}%
\pgfpathlineto{\pgfqpoint{8.505109in}{1.935333in}}%
\pgfpathlineto{\pgfqpoint{8.513916in}{1.917820in}}%
\pgfpathlineto{\pgfqpoint{8.522722in}{1.925719in}}%
\pgfpathlineto{\pgfqpoint{8.531529in}{1.917483in}}%
\pgfpathlineto{\pgfqpoint{8.540336in}{1.906155in}}%
\pgfpathlineto{\pgfqpoint{8.549143in}{1.963808in}}%
\pgfpathlineto{\pgfqpoint{8.557950in}{1.960041in}}%
\pgfpathlineto{\pgfqpoint{8.566757in}{1.969992in}}%
\pgfpathlineto{\pgfqpoint{8.575563in}{1.932578in}}%
\pgfpathlineto{\pgfqpoint{8.584370in}{1.835121in}}%
\pgfpathlineto{\pgfqpoint{8.593177in}{1.873210in}}%
\pgfpathlineto{\pgfqpoint{8.601984in}{1.932578in}}%
\pgfpathlineto{\pgfqpoint{8.610791in}{1.926057in}}%
\pgfpathlineto{\pgfqpoint{8.619597in}{1.873576in}}%
\pgfpathlineto{\pgfqpoint{8.628404in}{1.936710in}}%
\pgfpathlineto{\pgfqpoint{8.637211in}{1.926759in}}%
\pgfpathlineto{\pgfqpoint{8.646018in}{1.911299in}}%
\pgfpathlineto{\pgfqpoint{8.654825in}{1.873913in}}%
\pgfpathlineto{\pgfqpoint{8.663632in}{1.850216in}}%
\pgfpathlineto{\pgfqpoint{8.672438in}{1.804594in}}%
\pgfpathlineto{\pgfqpoint{8.681245in}{1.827925in}}%
\pgfpathlineto{\pgfqpoint{8.690052in}{1.877005in}}%
\pgfpathlineto{\pgfqpoint{8.698859in}{1.805972in}}%
\pgfpathlineto{\pgfqpoint{8.707666in}{1.793969in}}%
\pgfpathlineto{\pgfqpoint{8.716472in}{1.793603in}}%
\pgfpathlineto{\pgfqpoint{8.725279in}{1.834109in}}%
\pgfpathlineto{\pgfqpoint{8.734086in}{1.851931in}}%
\pgfpathlineto{\pgfqpoint{8.742893in}{1.900673in}}%
\pgfpathlineto{\pgfqpoint{8.751700in}{1.921953in}}%
\pgfpathlineto{\pgfqpoint{8.760507in}{1.922290in}}%
\pgfpathlineto{\pgfqpoint{8.769313in}{1.964145in}}%
\pgfpathlineto{\pgfqpoint{8.778120in}{1.908207in}}%
\pgfpathlineto{\pgfqpoint{8.786927in}{1.960716in}}%
\pgfpathlineto{\pgfqpoint{8.795734in}{1.965523in}}%
\pgfpathlineto{\pgfqpoint{8.804541in}{1.926394in}}%
\pgfpathlineto{\pgfqpoint{8.813347in}{1.925382in}}%
\pgfpathlineto{\pgfqpoint{8.822154in}{1.897581in}}%
\pgfpathlineto{\pgfqpoint{8.830961in}{1.909922in}}%
\pgfpathlineto{\pgfqpoint{8.839768in}{1.946998in}}%
\pgfpathlineto{\pgfqpoint{8.848575in}{2.023176in}}%
\pgfpathlineto{\pgfqpoint{8.857382in}{1.988516in}}%
\pgfpathlineto{\pgfqpoint{8.866188in}{2.011510in}}%
\pgfpathlineto{\pgfqpoint{8.874995in}{2.024188in}}%
\pgfpathlineto{\pgfqpoint{8.883802in}{2.026943in}}%
\pgfpathlineto{\pgfqpoint{8.892609in}{2.067449in}}%
\pgfpathlineto{\pgfqpoint{8.919029in}{1.928109in}}%
\pgfpathlineto{\pgfqpoint{8.927836in}{1.906858in}}%
\pgfpathlineto{\pgfqpoint{8.936643in}{1.822078in}}%
\pgfpathlineto{\pgfqpoint{8.945450in}{1.874925in}}%
\pgfpathlineto{\pgfqpoint{8.954257in}{1.944244in}}%
\pgfpathlineto{\pgfqpoint{8.963063in}{1.937047in}}%
\pgfpathlineto{\pgfqpoint{8.971870in}{1.892775in}}%
\pgfpathlineto{\pgfqpoint{8.980677in}{1.917820in}}%
\pgfpathlineto{\pgfqpoint{8.989484in}{1.815220in}}%
\pgfpathlineto{\pgfqpoint{8.998291in}{1.864637in}}%
\pgfpathlineto{\pgfqpoint{9.007097in}{1.878017in}}%
\pgfpathlineto{\pgfqpoint{9.015904in}{1.863962in}}%
\pgfpathlineto{\pgfqpoint{9.024711in}{1.911664in}}%
\pgfpathlineto{\pgfqpoint{9.033518in}{1.932578in}}%
\pgfpathlineto{\pgfqpoint{9.042325in}{1.882149in}}%
\pgfpathlineto{\pgfqpoint{9.051132in}{1.853674in}}%
\pgfpathlineto{\pgfqpoint{9.059938in}{1.804932in}}%
\pgfpathlineto{\pgfqpoint{9.068745in}{1.879732in}}%
\pgfpathlineto{\pgfqpoint{9.077552in}{1.816597in}}%
\pgfpathlineto{\pgfqpoint{9.086359in}{1.807349in}}%
\pgfpathlineto{\pgfqpoint{9.095166in}{1.894152in}}%
\pgfpathlineto{\pgfqpoint{9.103972in}{1.899971in}}%
\pgfpathlineto{\pgfqpoint{9.112779in}{1.859493in}}%
\pgfpathlineto{\pgfqpoint{9.121586in}{1.896541in}}%
\pgfpathlineto{\pgfqpoint{9.130393in}{1.915094in}}%
\pgfpathlineto{\pgfqpoint{9.139200in}{1.883161in}}%
\pgfpathlineto{\pgfqpoint{9.148007in}{1.885578in}}%
\pgfpathlineto{\pgfqpoint{9.156813in}{1.884876in}}%
\pgfpathlineto{\pgfqpoint{9.165620in}{1.889345in}}%
\pgfpathlineto{\pgfqpoint{9.174427in}{1.815220in}}%
\pgfpathlineto{\pgfqpoint{9.183234in}{1.834447in}}%
\pgfpathlineto{\pgfqpoint{9.192041in}{1.912002in}}%
\pgfpathlineto{\pgfqpoint{9.200847in}{1.933618in}}%
\pgfpathlineto{\pgfqpoint{9.209654in}{1.946998in}}%
\pgfpathlineto{\pgfqpoint{9.218461in}{1.984384in}}%
\pgfpathlineto{\pgfqpoint{9.227268in}{2.104835in}}%
\pgfpathlineto{\pgfqpoint{9.244882in}{2.011510in}}%
\pgfpathlineto{\pgfqpoint{9.253688in}{1.983710in}}%
\pgfpathlineto{\pgfqpoint{9.262495in}{1.893112in}}%
\pgfpathlineto{\pgfqpoint{9.271302in}{1.956949in}}%
\pgfpathlineto{\pgfqpoint{9.280109in}{1.960379in}}%
\pgfpathlineto{\pgfqpoint{9.288916in}{1.974433in}}%
\pgfpathlineto{\pgfqpoint{9.297722in}{1.990906in}}%
\pgfpathlineto{\pgfqpoint{9.306529in}{1.950765in}}%
\pgfpathlineto{\pgfqpoint{9.315336in}{1.936345in}}%
\pgfpathlineto{\pgfqpoint{9.324143in}{1.961053in}}%
\pgfpathlineto{\pgfqpoint{9.332950in}{1.966197in}}%
\pgfpathlineto{\pgfqpoint{9.341757in}{1.964848in}}%
\pgfpathlineto{\pgfqpoint{9.350563in}{1.962093in}}%
\pgfpathlineto{\pgfqpoint{9.368177in}{1.921250in}}%
\pgfpathlineto{\pgfqpoint{9.376984in}{1.888305in}}%
\pgfpathlineto{\pgfqpoint{9.385791in}{1.958664in}}%
\pgfpathlineto{\pgfqpoint{9.394597in}{1.896907in}}%
\pgfpathlineto{\pgfqpoint{9.412211in}{1.830680in}}%
\pgfpathlineto{\pgfqpoint{9.421018in}{1.788459in}}%
\pgfpathlineto{\pgfqpoint{9.429825in}{1.857441in}}%
\pgfpathlineto{\pgfqpoint{9.438632in}{1.898621in}}%
\pgfpathlineto{\pgfqpoint{9.447438in}{1.855389in}}%
\pgfpathlineto{\pgfqpoint{9.456245in}{1.802879in}}%
\pgfpathlineto{\pgfqpoint{9.465052in}{1.887293in}}%
\pgfpathlineto{\pgfqpoint{9.473859in}{1.922964in}}%
\pgfpathlineto{\pgfqpoint{9.482666in}{1.923667in}}%
\pgfpathlineto{\pgfqpoint{9.491472in}{1.920238in}}%
\pgfpathlineto{\pgfqpoint{9.500279in}{1.938762in}}%
\pgfpathlineto{\pgfqpoint{9.509086in}{1.946661in}}%
\pgfpathlineto{\pgfqpoint{9.517893in}{1.928109in}}%
\pgfpathlineto{\pgfqpoint{9.535507in}{1.922964in}}%
\pgfpathlineto{\pgfqpoint{9.544313in}{1.997427in}}%
\pgfpathlineto{\pgfqpoint{9.553120in}{1.935670in}}%
\pgfpathlineto{\pgfqpoint{9.561927in}{1.890723in}}%
\pgfpathlineto{\pgfqpoint{9.570734in}{1.925719in}}%
\pgfpathlineto{\pgfqpoint{9.579541in}{1.951805in}}%
\pgfpathlineto{\pgfqpoint{9.588347in}{1.973759in}}%
\pgfpathlineto{\pgfqpoint{9.597154in}{1.913716in}}%
\pgfpathlineto{\pgfqpoint{9.605961in}{1.903428in}}%
\pgfpathlineto{\pgfqpoint{9.614768in}{1.916808in}}%
\pgfpathlineto{\pgfqpoint{9.623575in}{1.899296in}}%
\pgfpathlineto{\pgfqpoint{9.632382in}{1.928474in}}%
\pgfpathlineto{\pgfqpoint{9.641188in}{1.858818in}}%
\pgfpathlineto{\pgfqpoint{9.649995in}{1.864974in}}%
\pgfpathlineto{\pgfqpoint{9.658802in}{1.902388in}}%
\pgfpathlineto{\pgfqpoint{9.667609in}{1.866014in}}%
\pgfpathlineto{\pgfqpoint{9.676416in}{1.931903in}}%
\pgfpathlineto{\pgfqpoint{9.685222in}{1.937047in}}%
\pgfpathlineto{\pgfqpoint{9.694029in}{1.897919in}}%
\pgfpathlineto{\pgfqpoint{9.702836in}{1.960041in}}%
\pgfpathlineto{\pgfqpoint{9.711643in}{1.902051in}}%
\pgfpathlineto{\pgfqpoint{9.720450in}{1.886590in}}%
\pgfpathlineto{\pgfqpoint{9.729257in}{1.919872in}}%
\pgfpathlineto{\pgfqpoint{9.738063in}{1.938762in}}%
\pgfpathlineto{\pgfqpoint{9.746870in}{1.908910in}}%
\pgfpathlineto{\pgfqpoint{9.755677in}{1.955235in}}%
\pgfpathlineto{\pgfqpoint{9.764484in}{1.861207in}}%
\pgfpathlineto{\pgfqpoint{9.773291in}{1.802177in}}%
\pgfpathlineto{\pgfqpoint{9.782097in}{1.856401in}}%
\pgfpathlineto{\pgfqpoint{9.790904in}{1.824496in}}%
\pgfpathlineto{\pgfqpoint{9.799711in}{1.829977in}}%
\pgfpathlineto{\pgfqpoint{9.808518in}{1.880069in}}%
\pgfpathlineto{\pgfqpoint{9.817325in}{1.831017in}}%
\pgfpathlineto{\pgfqpoint{9.826132in}{1.788459in}}%
\pgfpathlineto{\pgfqpoint{9.834938in}{1.832029in}}%
\pgfpathlineto{\pgfqpoint{9.843745in}{1.890020in}}%
\pgfpathlineto{\pgfqpoint{9.852552in}{1.900673in}}%
\pgfpathlineto{\pgfqpoint{9.861359in}{1.927434in}}%
\pgfpathlineto{\pgfqpoint{9.870166in}{1.906155in}}%
\pgfpathlineto{\pgfqpoint{9.878972in}{1.898256in}}%
\pgfpathlineto{\pgfqpoint{9.887779in}{1.796358in}}%
\pgfpathlineto{\pgfqpoint{9.896586in}{1.892775in}}%
\pgfpathlineto{\pgfqpoint{9.905393in}{1.952817in}}%
\pgfpathlineto{\pgfqpoint{9.914200in}{1.910624in}}%
\pgfpathlineto{\pgfqpoint{9.923007in}{1.887630in}}%
\pgfpathlineto{\pgfqpoint{9.931813in}{1.875965in}}%
\pgfpathlineto{\pgfqpoint{9.940620in}{1.881109in}}%
\pgfpathlineto{\pgfqpoint{9.949427in}{1.931201in}}%
\pgfpathlineto{\pgfqpoint{9.949427in}{1.931201in}}%
\pgfusepath{stroke}%
\end{pgfscope}%
\begin{pgfscope}%
\pgfsetrectcap%
\pgfsetmiterjoin%
\pgfsetlinewidth{0.803000pt}%
\definecolor{currentstroke}{rgb}{0.000000,0.000000,0.000000}%
\pgfsetstrokecolor{currentstroke}%
\pgfsetdash{}{0pt}%
\pgfpathmoveto{\pgfqpoint{0.702268in}{0.521603in}}%
\pgfpathlineto{\pgfqpoint{0.702268in}{4.756603in}}%
\pgfusepath{stroke}%
\end{pgfscope}%
\begin{pgfscope}%
\pgfsetrectcap%
\pgfsetmiterjoin%
\pgfsetlinewidth{0.803000pt}%
\definecolor{currentstroke}{rgb}{0.000000,0.000000,0.000000}%
\pgfsetstrokecolor{currentstroke}%
\pgfsetdash{}{0pt}%
\pgfpathmoveto{\pgfqpoint{10.389768in}{0.521603in}}%
\pgfpathlineto{\pgfqpoint{10.389768in}{4.756603in}}%
\pgfusepath{stroke}%
\end{pgfscope}%
\begin{pgfscope}%
\pgfsetrectcap%
\pgfsetmiterjoin%
\pgfsetlinewidth{0.803000pt}%
\definecolor{currentstroke}{rgb}{0.000000,0.000000,0.000000}%
\pgfsetstrokecolor{currentstroke}%
\pgfsetdash{}{0pt}%
\pgfpathmoveto{\pgfqpoint{0.702268in}{0.521603in}}%
\pgfpathlineto{\pgfqpoint{10.389768in}{0.521603in}}%
\pgfusepath{stroke}%
\end{pgfscope}%
\begin{pgfscope}%
\pgfsetrectcap%
\pgfsetmiterjoin%
\pgfsetlinewidth{0.803000pt}%
\definecolor{currentstroke}{rgb}{0.000000,0.000000,0.000000}%
\pgfsetstrokecolor{currentstroke}%
\pgfsetdash{}{0pt}%
\pgfpathmoveto{\pgfqpoint{0.702268in}{4.756603in}}%
\pgfpathlineto{\pgfqpoint{10.389768in}{4.756603in}}%
\pgfusepath{stroke}%
\end{pgfscope}%
\begin{pgfscope}%
\definecolor{textcolor}{rgb}{0.000000,0.000000,0.000000}%
\pgfsetstrokecolor{textcolor}%
\pgfsetfillcolor{textcolor}%
\pgftext[x=5.546018in,y=4.839937in,,base]{\color{textcolor}\sffamily\fontsize{12.000000}{14.400000}\selectfont time evolution of \(\displaystyle \langle I\rangle_t\) for \(\displaystyle T=1000\) simulation steps and \(\displaystyle L=64\) with \(\displaystyle p_1=p_2=p_3=0.5\)}%
\end{pgfscope}%
\begin{pgfscope}%
\pgfsetbuttcap%
\pgfsetmiterjoin%
\definecolor{currentfill}{rgb}{1.000000,1.000000,1.000000}%
\pgfsetfillcolor{currentfill}%
\pgfsetfillopacity{0.800000}%
\pgfsetlinewidth{1.003750pt}%
\definecolor{currentstroke}{rgb}{0.800000,0.800000,0.800000}%
\pgfsetstrokecolor{currentstroke}%
\pgfsetstrokeopacity{0.800000}%
\pgfsetdash{}{0pt}%
\pgfpathmoveto{\pgfqpoint{8.219139in}{3.975342in}}%
\pgfpathlineto{\pgfqpoint{10.292546in}{3.975342in}}%
\pgfpathquadraticcurveto{\pgfqpoint{10.320323in}{3.975342in}}{\pgfqpoint{10.320323in}{4.003120in}}%
\pgfpathlineto{\pgfqpoint{10.320323in}{4.659381in}}%
\pgfpathquadraticcurveto{\pgfqpoint{10.320323in}{4.687159in}}{\pgfqpoint{10.292546in}{4.687159in}}%
\pgfpathlineto{\pgfqpoint{8.219139in}{4.687159in}}%
\pgfpathquadraticcurveto{\pgfqpoint{8.191362in}{4.687159in}}{\pgfqpoint{8.191362in}{4.659381in}}%
\pgfpathlineto{\pgfqpoint{8.191362in}{4.003120in}}%
\pgfpathquadraticcurveto{\pgfqpoint{8.191362in}{3.975342in}}{\pgfqpoint{8.219139in}{3.975342in}}%
\pgfpathlineto{\pgfqpoint{8.219139in}{3.975342in}}%
\pgfpathclose%
\pgfusepath{stroke,fill}%
\end{pgfscope}%
\begin{pgfscope}%
\pgfsetrectcap%
\pgfsetroundjoin%
\pgfsetlinewidth{0.501875pt}%
\definecolor{currentstroke}{rgb}{0.501961,0.501961,0.501961}%
\pgfsetstrokecolor{currentstroke}%
\pgfsetstrokeopacity{0.250000}%
\pgfsetdash{}{0pt}%
\pgfpathmoveto{\pgfqpoint{8.246917in}{4.574691in}}%
\pgfpathlineto{\pgfqpoint{8.385806in}{4.574691in}}%
\pgfpathlineto{\pgfqpoint{8.524695in}{4.574691in}}%
\pgfusepath{stroke}%
\end{pgfscope}%
\begin{pgfscope}%
\definecolor{textcolor}{rgb}{0.000000,0.000000,0.000000}%
\pgfsetstrokecolor{textcolor}%
\pgfsetfillcolor{textcolor}%
\pgftext[x=8.635806in,y=4.526080in,left,base]{\color{textcolor}\sffamily\fontsize{10.000000}{12.000000}\selectfont individual samples \(\displaystyle \langle I\rangle_t\)}%
\end{pgfscope}%
\begin{pgfscope}%
\pgfsetrectcap%
\pgfsetroundjoin%
\pgfsetlinewidth{1.003750pt}%
\definecolor{currentstroke}{rgb}{0.000000,0.000000,0.000000}%
\pgfsetstrokecolor{currentstroke}%
\pgfsetstrokeopacity{0.750000}%
\pgfsetdash{}{0pt}%
\pgfpathmoveto{\pgfqpoint{8.246917in}{4.338582in}}%
\pgfpathlineto{\pgfqpoint{8.385806in}{4.338582in}}%
\pgfpathlineto{\pgfqpoint{8.524695in}{4.338582in}}%
\pgfusepath{stroke}%
\end{pgfscope}%
\begin{pgfscope}%
\definecolor{textcolor}{rgb}{0.000000,0.000000,0.000000}%
\pgfsetstrokecolor{textcolor}%
\pgfsetfillcolor{textcolor}%
\pgftext[x=8.635806in,y=4.289971in,left,base]{\color{textcolor}\sffamily\fontsize{10.000000}{12.000000}\selectfont mean \(\displaystyle \overline{\langle I\rangle_t}\)}%
\end{pgfscope}%
\begin{pgfscope}%
\pgfsetbuttcap%
\pgfsetmiterjoin%
\definecolor{currentfill}{rgb}{0.980392,0.164706,0.333333}%
\pgfsetfillcolor{currentfill}%
\pgfsetfillopacity{0.300000}%
\pgfsetlinewidth{1.003750pt}%
\definecolor{currentstroke}{rgb}{0.980392,0.164706,0.333333}%
\pgfsetstrokecolor{currentstroke}%
\pgfsetstrokeopacity{0.300000}%
\pgfsetdash{}{0pt}%
\pgfpathmoveto{\pgfqpoint{8.246917in}{4.080281in}}%
\pgfpathlineto{\pgfqpoint{8.524695in}{4.080281in}}%
\pgfpathlineto{\pgfqpoint{8.524695in}{4.177503in}}%
\pgfpathlineto{\pgfqpoint{8.246917in}{4.177503in}}%
\pgfpathlineto{\pgfqpoint{8.246917in}{4.080281in}}%
\pgfpathclose%
\pgfusepath{stroke,fill}%
\end{pgfscope}%
\begin{pgfscope}%
\definecolor{textcolor}{rgb}{0.000000,0.000000,0.000000}%
\pgfsetstrokecolor{textcolor}%
\pgfsetfillcolor{textcolor}%
\pgftext[x=8.635806in,y=4.080281in,left,base]{\color{textcolor}\sffamily\fontsize{10.000000}{12.000000}\selectfont standard deviation \(\displaystyle \sigma_{\langle I\rangle_t}\)}%
\end{pgfscope}%
\end{pgfpicture}%
\makeatother%
\endgroup%
}
    \caption{Time evolution of the infection rate $\langle I\rangle_t$ over each of the $T=1000$ simulation steps for a total of 20 samples. At each timestep mean $\overline{\langle I\rangle_t}$
    and standard deviation $\sigma_{\langle I\rangle_t}$ of the infection rate were calculated and are also displayed in the plot. The grid size was chosen as $L=64$, the turnover probabilities
    as $p_1=p_2=p_3=0.5$, no vaccinated induviduals were used.}\label{fig:Avg_Inf_over_t}
\end{figure}
