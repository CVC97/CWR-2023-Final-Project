\section{Results and Discussion}

\subsection{Model for the Spread of Infectious Diseases}

\subsection{Expected Ratio of Infected People averaged over Time}


\begin{figure}[ht]
    \centering
    \resizebox{\textwidth}{!}{%% Creator: Matplotlib, PGF backend
%%
%% To include the figure in your LaTeX document, write
%%   \input{<filename>.pgf}
%%
%% Make sure the required packages are loaded in your preamble
%%   \usepackage{pgf}
%%
%% Also ensure that all the required font packages are loaded; for instance,
%% the lmodern package is sometimes necessary when using math font.
%%   \usepackage{lmodern}
%%
%% Figures using additional raster images can only be included by \input if
%% they are in the same directory as the main LaTeX file. For loading figures
%% from other directories you can use the `import` package
%%   \usepackage{import}
%%
%% and then include the figures with
%%   \import{<path to file>}{<filename>.pgf}
%%
%% Matplotlib used the following preamble
%%   
%%   \usepackage{fontspec}
%%   \setmainfont{DejaVuSerif.ttf}[Path=\detokenize{/home/carlo/.local/lib/python3.10/site-packages/matplotlib/mpl-data/fonts/ttf/}]
%%   \setsansfont{DejaVuSans.ttf}[Path=\detokenize{/home/carlo/.local/lib/python3.10/site-packages/matplotlib/mpl-data/fonts/ttf/}]
%%   \setmonofont{DejaVuSansMono.ttf}[Path=\detokenize{/home/carlo/.local/lib/python3.10/site-packages/matplotlib/mpl-data/fonts/ttf/}]
%%   \makeatletter\@ifpackageloaded{underscore}{}{\usepackage[strings]{underscore}}\makeatother
%%
\begingroup%
\makeatletter%
\begin{pgfpicture}%
\pgfpathrectangle{\pgfpointorigin}{\pgfqpoint{10.427823in}{13.953546in}}%
\pgfusepath{use as bounding box, clip}%
\begin{pgfscope}%
\pgfsetbuttcap%
\pgfsetmiterjoin%
\definecolor{currentfill}{rgb}{1.000000,1.000000,1.000000}%
\pgfsetfillcolor{currentfill}%
\pgfsetlinewidth{0.000000pt}%
\definecolor{currentstroke}{rgb}{1.000000,1.000000,1.000000}%
\pgfsetstrokecolor{currentstroke}%
\pgfsetdash{}{0pt}%
\pgfpathmoveto{\pgfqpoint{0.000000in}{0.000000in}}%
\pgfpathlineto{\pgfqpoint{10.427823in}{0.000000in}}%
\pgfpathlineto{\pgfqpoint{10.427823in}{13.953546in}}%
\pgfpathlineto{\pgfqpoint{0.000000in}{13.953546in}}%
\pgfpathlineto{\pgfqpoint{0.000000in}{0.000000in}}%
\pgfpathclose%
\pgfusepath{fill}%
\end{pgfscope}%
\begin{pgfscope}%
\pgfsetbuttcap%
\pgfsetmiterjoin%
\definecolor{currentfill}{rgb}{1.000000,1.000000,1.000000}%
\pgfsetfillcolor{currentfill}%
\pgfsetlinewidth{0.000000pt}%
\definecolor{currentstroke}{rgb}{0.000000,0.000000,0.000000}%
\pgfsetstrokecolor{currentstroke}%
\pgfsetstrokeopacity{0.000000}%
\pgfsetdash{}{0pt}%
\pgfpathmoveto{\pgfqpoint{0.640323in}{9.767436in}}%
\pgfpathlineto{\pgfqpoint{10.327822in}{9.767436in}}%
\pgfpathlineto{\pgfqpoint{10.327822in}{13.617436in}}%
\pgfpathlineto{\pgfqpoint{0.640323in}{13.617436in}}%
\pgfpathlineto{\pgfqpoint{0.640323in}{9.767436in}}%
\pgfpathclose%
\pgfusepath{fill}%
\end{pgfscope}%
\begin{pgfscope}%
\pgfpathrectangle{\pgfqpoint{0.640323in}{9.767436in}}{\pgfqpoint{9.687500in}{3.850000in}}%
\pgfusepath{clip}%
\pgfsetbuttcap%
\pgfsetroundjoin%
\definecolor{currentfill}{rgb}{0.000000,0.000000,1.000000}%
\pgfsetfillcolor{currentfill}%
\pgfsetfillopacity{0.500000}%
\pgfsetlinewidth{1.003750pt}%
\definecolor{currentstroke}{rgb}{0.000000,0.000000,1.000000}%
\pgfsetstrokecolor{currentstroke}%
\pgfsetstrokeopacity{0.500000}%
\pgfsetdash{}{0pt}%
\pgfsys@defobject{currentmarker}{\pgfqpoint{-0.021960in}{-0.021960in}}{\pgfqpoint{0.021960in}{0.021960in}}{%
\pgfpathmoveto{\pgfqpoint{0.000000in}{-0.021960in}}%
\pgfpathcurveto{\pgfqpoint{0.005824in}{-0.021960in}}{\pgfqpoint{0.011410in}{-0.019646in}}{\pgfqpoint{0.015528in}{-0.015528in}}%
\pgfpathcurveto{\pgfqpoint{0.019646in}{-0.011410in}}{\pgfqpoint{0.021960in}{-0.005824in}}{\pgfqpoint{0.021960in}{0.000000in}}%
\pgfpathcurveto{\pgfqpoint{0.021960in}{0.005824in}}{\pgfqpoint{0.019646in}{0.011410in}}{\pgfqpoint{0.015528in}{0.015528in}}%
\pgfpathcurveto{\pgfqpoint{0.011410in}{0.019646in}}{\pgfqpoint{0.005824in}{0.021960in}}{\pgfqpoint{0.000000in}{0.021960in}}%
\pgfpathcurveto{\pgfqpoint{-0.005824in}{0.021960in}}{\pgfqpoint{-0.011410in}{0.019646in}}{\pgfqpoint{-0.015528in}{0.015528in}}%
\pgfpathcurveto{\pgfqpoint{-0.019646in}{0.011410in}}{\pgfqpoint{-0.021960in}{0.005824in}}{\pgfqpoint{-0.021960in}{0.000000in}}%
\pgfpathcurveto{\pgfqpoint{-0.021960in}{-0.005824in}}{\pgfqpoint{-0.019646in}{-0.011410in}}{\pgfqpoint{-0.015528in}{-0.015528in}}%
\pgfpathcurveto{\pgfqpoint{-0.011410in}{-0.019646in}}{\pgfqpoint{-0.005824in}{-0.021960in}}{\pgfqpoint{0.000000in}{-0.021960in}}%
\pgfpathlineto{\pgfqpoint{0.000000in}{-0.021960in}}%
\pgfpathclose%
\pgfusepath{stroke,fill}%
}%
\begin{pgfscope}%
\pgfsys@transformshift{1.080663in}{9.899391in}%
\pgfsys@useobject{currentmarker}{}%
\end{pgfscope}%
\begin{pgfscope}%
\pgfsys@transformshift{1.260394in}{9.900022in}%
\pgfsys@useobject{currentmarker}{}%
\end{pgfscope}%
\begin{pgfscope}%
\pgfsys@transformshift{1.440125in}{9.902399in}%
\pgfsys@useobject{currentmarker}{}%
\end{pgfscope}%
\begin{pgfscope}%
\pgfsys@transformshift{1.619856in}{9.905553in}%
\pgfsys@useobject{currentmarker}{}%
\end{pgfscope}%
\begin{pgfscope}%
\pgfsys@transformshift{1.799587in}{9.905262in}%
\pgfsys@useobject{currentmarker}{}%
\end{pgfscope}%
\begin{pgfscope}%
\pgfsys@transformshift{1.979318in}{9.913363in}%
\pgfsys@useobject{currentmarker}{}%
\end{pgfscope}%
\begin{pgfscope}%
\pgfsys@transformshift{2.159049in}{9.912830in}%
\pgfsys@useobject{currentmarker}{}%
\end{pgfscope}%
\begin{pgfscope}%
\pgfsys@transformshift{2.338780in}{9.914503in}%
\pgfsys@useobject{currentmarker}{}%
\end{pgfscope}%
\begin{pgfscope}%
\pgfsys@transformshift{2.518511in}{9.923842in}%
\pgfsys@useobject{currentmarker}{}%
\end{pgfscope}%
\begin{pgfscope}%
\pgfsys@transformshift{2.698242in}{9.949142in}%
\pgfsys@useobject{currentmarker}{}%
\end{pgfscope}%
\begin{pgfscope}%
\pgfsys@transformshift{2.877973in}{9.994186in}%
\pgfsys@useobject{currentmarker}{}%
\end{pgfscope}%
\begin{pgfscope}%
\pgfsys@transformshift{3.057704in}{10.385080in}%
\pgfsys@useobject{currentmarker}{}%
\end{pgfscope}%
\begin{pgfscope}%
\pgfsys@transformshift{3.237435in}{10.869287in}%
\pgfsys@useobject{currentmarker}{}%
\end{pgfscope}%
\begin{pgfscope}%
\pgfsys@transformshift{3.417166in}{10.938860in}%
\pgfsys@useobject{currentmarker}{}%
\end{pgfscope}%
\begin{pgfscope}%
\pgfsys@transformshift{3.596897in}{11.149381in}%
\pgfsys@useobject{currentmarker}{}%
\end{pgfscope}%
\begin{pgfscope}%
\pgfsys@transformshift{3.776628in}{11.263968in}%
\pgfsys@useobject{currentmarker}{}%
\end{pgfscope}%
\begin{pgfscope}%
\pgfsys@transformshift{3.956359in}{11.293029in}%
\pgfsys@useobject{currentmarker}{}%
\end{pgfscope}%
\begin{pgfscope}%
\pgfsys@transformshift{4.136090in}{11.472067in}%
\pgfsys@useobject{currentmarker}{}%
\end{pgfscope}%
\begin{pgfscope}%
\pgfsys@transformshift{4.315821in}{11.621192in}%
\pgfsys@useobject{currentmarker}{}%
\end{pgfscope}%
\begin{pgfscope}%
\pgfsys@transformshift{4.495552in}{11.671621in}%
\pgfsys@useobject{currentmarker}{}%
\end{pgfscope}%
\begin{pgfscope}%
\pgfsys@transformshift{4.675283in}{11.706842in}%
\pgfsys@useobject{currentmarker}{}%
\end{pgfscope}%
\begin{pgfscope}%
\pgfsys@transformshift{4.855014in}{11.764064in}%
\pgfsys@useobject{currentmarker}{}%
\end{pgfscope}%
\begin{pgfscope}%
\pgfsys@transformshift{5.034745in}{11.845157in}%
\pgfsys@useobject{currentmarker}{}%
\end{pgfscope}%
\begin{pgfscope}%
\pgfsys@transformshift{5.214476in}{11.890152in}%
\pgfsys@useobject{currentmarker}{}%
\end{pgfscope}%
\begin{pgfscope}%
\pgfsys@transformshift{5.394207in}{11.924113in}%
\pgfsys@useobject{currentmarker}{}%
\end{pgfscope}%
\begin{pgfscope}%
\pgfsys@transformshift{5.573938in}{11.921467in}%
\pgfsys@useobject{currentmarker}{}%
\end{pgfscope}%
\begin{pgfscope}%
\pgfsys@transformshift{5.753669in}{12.042022in}%
\pgfsys@useobject{currentmarker}{}%
\end{pgfscope}%
\begin{pgfscope}%
\pgfsys@transformshift{5.933400in}{12.024703in}%
\pgfsys@useobject{currentmarker}{}%
\end{pgfscope}%
\begin{pgfscope}%
\pgfsys@transformshift{6.113131in}{12.074163in}%
\pgfsys@useobject{currentmarker}{}%
\end{pgfscope}%
\begin{pgfscope}%
\pgfsys@transformshift{6.292862in}{12.109720in}%
\pgfsys@useobject{currentmarker}{}%
\end{pgfscope}%
\begin{pgfscope}%
\pgfsys@transformshift{6.472593in}{12.101331in}%
\pgfsys@useobject{currentmarker}{}%
\end{pgfscope}%
\begin{pgfscope}%
\pgfsys@transformshift{6.652324in}{12.120587in}%
\pgfsys@useobject{currentmarker}{}%
\end{pgfscope}%
\begin{pgfscope}%
\pgfsys@transformshift{6.832055in}{12.167985in}%
\pgfsys@useobject{currentmarker}{}%
\end{pgfscope}%
\begin{pgfscope}%
\pgfsys@transformshift{7.011786in}{12.193190in}%
\pgfsys@useobject{currentmarker}{}%
\end{pgfscope}%
\begin{pgfscope}%
\pgfsys@transformshift{7.191517in}{12.211112in}%
\pgfsys@useobject{currentmarker}{}%
\end{pgfscope}%
\begin{pgfscope}%
\pgfsys@transformshift{7.371248in}{12.236900in}%
\pgfsys@useobject{currentmarker}{}%
\end{pgfscope}%
\begin{pgfscope}%
\pgfsys@transformshift{7.550979in}{12.269427in}%
\pgfsys@useobject{currentmarker}{}%
\end{pgfscope}%
\begin{pgfscope}%
\pgfsys@transformshift{7.730710in}{12.262391in}%
\pgfsys@useobject{currentmarker}{}%
\end{pgfscope}%
\begin{pgfscope}%
\pgfsys@transformshift{7.910441in}{12.257274in}%
\pgfsys@useobject{currentmarker}{}%
\end{pgfscope}%
\begin{pgfscope}%
\pgfsys@transformshift{8.090172in}{12.295333in}%
\pgfsys@useobject{currentmarker}{}%
\end{pgfscope}%
\begin{pgfscope}%
\pgfsys@transformshift{8.269903in}{12.305741in}%
\pgfsys@useobject{currentmarker}{}%
\end{pgfscope}%
\begin{pgfscope}%
\pgfsys@transformshift{8.449634in}{12.347991in}%
\pgfsys@useobject{currentmarker}{}%
\end{pgfscope}%
\begin{pgfscope}%
\pgfsys@transformshift{8.629365in}{12.322861in}%
\pgfsys@useobject{currentmarker}{}%
\end{pgfscope}%
\begin{pgfscope}%
\pgfsys@transformshift{8.809096in}{12.368129in}%
\pgfsys@useobject{currentmarker}{}%
\end{pgfscope}%
\begin{pgfscope}%
\pgfsys@transformshift{8.988827in}{12.380766in}%
\pgfsys@useobject{currentmarker}{}%
\end{pgfscope}%
\begin{pgfscope}%
\pgfsys@transformshift{9.168558in}{12.397284in}%
\pgfsys@useobject{currentmarker}{}%
\end{pgfscope}%
\begin{pgfscope}%
\pgfsys@transformshift{9.348289in}{12.388913in}%
\pgfsys@useobject{currentmarker}{}%
\end{pgfscope}%
\begin{pgfscope}%
\pgfsys@transformshift{9.528020in}{12.391968in}%
\pgfsys@useobject{currentmarker}{}%
\end{pgfscope}%
\begin{pgfscope}%
\pgfsys@transformshift{9.707751in}{12.422924in}%
\pgfsys@useobject{currentmarker}{}%
\end{pgfscope}%
\begin{pgfscope}%
\pgfsys@transformshift{9.887482in}{12.423259in}%
\pgfsys@useobject{currentmarker}{}%
\end{pgfscope}%
\end{pgfscope}%
\begin{pgfscope}%
\pgfpathrectangle{\pgfqpoint{0.640323in}{9.767436in}}{\pgfqpoint{9.687500in}{3.850000in}}%
\pgfusepath{clip}%
\pgfsetbuttcap%
\pgfsetroundjoin%
\definecolor{currentfill}{rgb}{0.980392,0.164706,0.333333}%
\pgfsetfillcolor{currentfill}%
\pgfsetfillopacity{0.500000}%
\pgfsetlinewidth{1.003750pt}%
\definecolor{currentstroke}{rgb}{0.980392,0.164706,0.333333}%
\pgfsetstrokecolor{currentstroke}%
\pgfsetstrokeopacity{0.500000}%
\pgfsetdash{}{0pt}%
\pgfsys@defobject{currentmarker}{\pgfqpoint{-0.021960in}{-0.021960in}}{\pgfqpoint{0.021960in}{0.021960in}}{%
\pgfpathmoveto{\pgfqpoint{0.000000in}{-0.021960in}}%
\pgfpathcurveto{\pgfqpoint{0.005824in}{-0.021960in}}{\pgfqpoint{0.011410in}{-0.019646in}}{\pgfqpoint{0.015528in}{-0.015528in}}%
\pgfpathcurveto{\pgfqpoint{0.019646in}{-0.011410in}}{\pgfqpoint{0.021960in}{-0.005824in}}{\pgfqpoint{0.021960in}{0.000000in}}%
\pgfpathcurveto{\pgfqpoint{0.021960in}{0.005824in}}{\pgfqpoint{0.019646in}{0.011410in}}{\pgfqpoint{0.015528in}{0.015528in}}%
\pgfpathcurveto{\pgfqpoint{0.011410in}{0.019646in}}{\pgfqpoint{0.005824in}{0.021960in}}{\pgfqpoint{0.000000in}{0.021960in}}%
\pgfpathcurveto{\pgfqpoint{-0.005824in}{0.021960in}}{\pgfqpoint{-0.011410in}{0.019646in}}{\pgfqpoint{-0.015528in}{0.015528in}}%
\pgfpathcurveto{\pgfqpoint{-0.019646in}{0.011410in}}{\pgfqpoint{-0.021960in}{0.005824in}}{\pgfqpoint{-0.021960in}{0.000000in}}%
\pgfpathcurveto{\pgfqpoint{-0.021960in}{-0.005824in}}{\pgfqpoint{-0.019646in}{-0.011410in}}{\pgfqpoint{-0.015528in}{-0.015528in}}%
\pgfpathcurveto{\pgfqpoint{-0.011410in}{-0.019646in}}{\pgfqpoint{-0.005824in}{-0.021960in}}{\pgfqpoint{0.000000in}{-0.021960in}}%
\pgfpathlineto{\pgfqpoint{0.000000in}{-0.021960in}}%
\pgfpathclose%
\pgfusepath{stroke,fill}%
}%
\begin{pgfscope}%
\pgfsys@transformshift{1.080663in}{9.899119in}%
\pgfsys@useobject{currentmarker}{}%
\end{pgfscope}%
\begin{pgfscope}%
\pgfsys@transformshift{1.260394in}{9.901562in}%
\pgfsys@useobject{currentmarker}{}%
\end{pgfscope}%
\begin{pgfscope}%
\pgfsys@transformshift{1.440125in}{9.903036in}%
\pgfsys@useobject{currentmarker}{}%
\end{pgfscope}%
\begin{pgfscope}%
\pgfsys@transformshift{1.619856in}{9.903576in}%
\pgfsys@useobject{currentmarker}{}%
\end{pgfscope}%
\begin{pgfscope}%
\pgfsys@transformshift{1.799587in}{9.905353in}%
\pgfsys@useobject{currentmarker}{}%
\end{pgfscope}%
\begin{pgfscope}%
\pgfsys@transformshift{1.979318in}{9.910307in}%
\pgfsys@useobject{currentmarker}{}%
\end{pgfscope}%
\begin{pgfscope}%
\pgfsys@transformshift{2.159049in}{9.910531in}%
\pgfsys@useobject{currentmarker}{}%
\end{pgfscope}%
\begin{pgfscope}%
\pgfsys@transformshift{2.338780in}{9.931271in}%
\pgfsys@useobject{currentmarker}{}%
\end{pgfscope}%
\begin{pgfscope}%
\pgfsys@transformshift{2.518511in}{9.930488in}%
\pgfsys@useobject{currentmarker}{}%
\end{pgfscope}%
\begin{pgfscope}%
\pgfsys@transformshift{2.698242in}{9.961937in}%
\pgfsys@useobject{currentmarker}{}%
\end{pgfscope}%
\begin{pgfscope}%
\pgfsys@transformshift{2.877973in}{10.346506in}%
\pgfsys@useobject{currentmarker}{}%
\end{pgfscope}%
\begin{pgfscope}%
\pgfsys@transformshift{3.057704in}{10.660375in}%
\pgfsys@useobject{currentmarker}{}%
\end{pgfscope}%
\begin{pgfscope}%
\pgfsys@transformshift{3.237435in}{10.928155in}%
\pgfsys@useobject{currentmarker}{}%
\end{pgfscope}%
\begin{pgfscope}%
\pgfsys@transformshift{3.417166in}{11.155131in}%
\pgfsys@useobject{currentmarker}{}%
\end{pgfscope}%
\begin{pgfscope}%
\pgfsys@transformshift{3.596897in}{11.258044in}%
\pgfsys@useobject{currentmarker}{}%
\end{pgfscope}%
\begin{pgfscope}%
\pgfsys@transformshift{3.776628in}{11.367067in}%
\pgfsys@useobject{currentmarker}{}%
\end{pgfscope}%
\begin{pgfscope}%
\pgfsys@transformshift{3.956359in}{11.492161in}%
\pgfsys@useobject{currentmarker}{}%
\end{pgfscope}%
\begin{pgfscope}%
\pgfsys@transformshift{4.136090in}{11.565516in}%
\pgfsys@useobject{currentmarker}{}%
\end{pgfscope}%
\begin{pgfscope}%
\pgfsys@transformshift{4.315821in}{11.655749in}%
\pgfsys@useobject{currentmarker}{}%
\end{pgfscope}%
\begin{pgfscope}%
\pgfsys@transformshift{4.495552in}{11.729986in}%
\pgfsys@useobject{currentmarker}{}%
\end{pgfscope}%
\begin{pgfscope}%
\pgfsys@transformshift{4.675283in}{11.783159in}%
\pgfsys@useobject{currentmarker}{}%
\end{pgfscope}%
\begin{pgfscope}%
\pgfsys@transformshift{4.855014in}{11.813022in}%
\pgfsys@useobject{currentmarker}{}%
\end{pgfscope}%
\begin{pgfscope}%
\pgfsys@transformshift{5.034745in}{11.889425in}%
\pgfsys@useobject{currentmarker}{}%
\end{pgfscope}%
\begin{pgfscope}%
\pgfsys@transformshift{5.214476in}{11.921523in}%
\pgfsys@useobject{currentmarker}{}%
\end{pgfscope}%
\begin{pgfscope}%
\pgfsys@transformshift{5.394207in}{11.974063in}%
\pgfsys@useobject{currentmarker}{}%
\end{pgfscope}%
\begin{pgfscope}%
\pgfsys@transformshift{5.573938in}{12.006136in}%
\pgfsys@useobject{currentmarker}{}%
\end{pgfscope}%
\begin{pgfscope}%
\pgfsys@transformshift{5.753669in}{12.026697in}%
\pgfsys@useobject{currentmarker}{}%
\end{pgfscope}%
\begin{pgfscope}%
\pgfsys@transformshift{5.933400in}{12.072760in}%
\pgfsys@useobject{currentmarker}{}%
\end{pgfscope}%
\begin{pgfscope}%
\pgfsys@transformshift{6.113131in}{12.105355in}%
\pgfsys@useobject{currentmarker}{}%
\end{pgfscope}%
\begin{pgfscope}%
\pgfsys@transformshift{6.292862in}{12.138036in}%
\pgfsys@useobject{currentmarker}{}%
\end{pgfscope}%
\begin{pgfscope}%
\pgfsys@transformshift{6.472593in}{12.162440in}%
\pgfsys@useobject{currentmarker}{}%
\end{pgfscope}%
\begin{pgfscope}%
\pgfsys@transformshift{6.652324in}{12.186956in}%
\pgfsys@useobject{currentmarker}{}%
\end{pgfscope}%
\begin{pgfscope}%
\pgfsys@transformshift{6.832055in}{12.201580in}%
\pgfsys@useobject{currentmarker}{}%
\end{pgfscope}%
\begin{pgfscope}%
\pgfsys@transformshift{7.011786in}{12.224034in}%
\pgfsys@useobject{currentmarker}{}%
\end{pgfscope}%
\begin{pgfscope}%
\pgfsys@transformshift{7.191517in}{12.242893in}%
\pgfsys@useobject{currentmarker}{}%
\end{pgfscope}%
\begin{pgfscope}%
\pgfsys@transformshift{7.371248in}{12.274550in}%
\pgfsys@useobject{currentmarker}{}%
\end{pgfscope}%
\begin{pgfscope}%
\pgfsys@transformshift{7.550979in}{12.296240in}%
\pgfsys@useobject{currentmarker}{}%
\end{pgfscope}%
\begin{pgfscope}%
\pgfsys@transformshift{7.730710in}{12.299488in}%
\pgfsys@useobject{currentmarker}{}%
\end{pgfscope}%
\begin{pgfscope}%
\pgfsys@transformshift{7.910441in}{12.302909in}%
\pgfsys@useobject{currentmarker}{}%
\end{pgfscope}%
\begin{pgfscope}%
\pgfsys@transformshift{8.090172in}{12.331325in}%
\pgfsys@useobject{currentmarker}{}%
\end{pgfscope}%
\begin{pgfscope}%
\pgfsys@transformshift{8.269903in}{12.356039in}%
\pgfsys@useobject{currentmarker}{}%
\end{pgfscope}%
\begin{pgfscope}%
\pgfsys@transformshift{8.449634in}{12.350047in}%
\pgfsys@useobject{currentmarker}{}%
\end{pgfscope}%
\begin{pgfscope}%
\pgfsys@transformshift{8.629365in}{12.374147in}%
\pgfsys@useobject{currentmarker}{}%
\end{pgfscope}%
\begin{pgfscope}%
\pgfsys@transformshift{8.809096in}{12.396669in}%
\pgfsys@useobject{currentmarker}{}%
\end{pgfscope}%
\begin{pgfscope}%
\pgfsys@transformshift{8.988827in}{12.390826in}%
\pgfsys@useobject{currentmarker}{}%
\end{pgfscope}%
\begin{pgfscope}%
\pgfsys@transformshift{9.168558in}{12.424544in}%
\pgfsys@useobject{currentmarker}{}%
\end{pgfscope}%
\begin{pgfscope}%
\pgfsys@transformshift{9.348289in}{12.424190in}%
\pgfsys@useobject{currentmarker}{}%
\end{pgfscope}%
\begin{pgfscope}%
\pgfsys@transformshift{9.528020in}{12.436231in}%
\pgfsys@useobject{currentmarker}{}%
\end{pgfscope}%
\begin{pgfscope}%
\pgfsys@transformshift{9.707751in}{12.444862in}%
\pgfsys@useobject{currentmarker}{}%
\end{pgfscope}%
\begin{pgfscope}%
\pgfsys@transformshift{9.887482in}{12.469304in}%
\pgfsys@useobject{currentmarker}{}%
\end{pgfscope}%
\end{pgfscope}%
\begin{pgfscope}%
\pgfpathrectangle{\pgfqpoint{0.640323in}{9.767436in}}{\pgfqpoint{9.687500in}{3.850000in}}%
\pgfusepath{clip}%
\pgfsetbuttcap%
\pgfsetroundjoin%
\definecolor{currentfill}{rgb}{0.239216,0.478431,0.992157}%
\pgfsetfillcolor{currentfill}%
\pgfsetfillopacity{0.500000}%
\pgfsetlinewidth{1.003750pt}%
\definecolor{currentstroke}{rgb}{0.239216,0.478431,0.992157}%
\pgfsetstrokecolor{currentstroke}%
\pgfsetstrokeopacity{0.500000}%
\pgfsetdash{}{0pt}%
\pgfsys@defobject{currentmarker}{\pgfqpoint{-0.021960in}{-0.021960in}}{\pgfqpoint{0.021960in}{0.021960in}}{%
\pgfpathmoveto{\pgfqpoint{0.000000in}{-0.021960in}}%
\pgfpathcurveto{\pgfqpoint{0.005824in}{-0.021960in}}{\pgfqpoint{0.011410in}{-0.019646in}}{\pgfqpoint{0.015528in}{-0.015528in}}%
\pgfpathcurveto{\pgfqpoint{0.019646in}{-0.011410in}}{\pgfqpoint{0.021960in}{-0.005824in}}{\pgfqpoint{0.021960in}{0.000000in}}%
\pgfpathcurveto{\pgfqpoint{0.021960in}{0.005824in}}{\pgfqpoint{0.019646in}{0.011410in}}{\pgfqpoint{0.015528in}{0.015528in}}%
\pgfpathcurveto{\pgfqpoint{0.011410in}{0.019646in}}{\pgfqpoint{0.005824in}{0.021960in}}{\pgfqpoint{0.000000in}{0.021960in}}%
\pgfpathcurveto{\pgfqpoint{-0.005824in}{0.021960in}}{\pgfqpoint{-0.011410in}{0.019646in}}{\pgfqpoint{-0.015528in}{0.015528in}}%
\pgfpathcurveto{\pgfqpoint{-0.019646in}{0.011410in}}{\pgfqpoint{-0.021960in}{0.005824in}}{\pgfqpoint{-0.021960in}{0.000000in}}%
\pgfpathcurveto{\pgfqpoint{-0.021960in}{-0.005824in}}{\pgfqpoint{-0.019646in}{-0.011410in}}{\pgfqpoint{-0.015528in}{-0.015528in}}%
\pgfpathcurveto{\pgfqpoint{-0.011410in}{-0.019646in}}{\pgfqpoint{-0.005824in}{-0.021960in}}{\pgfqpoint{0.000000in}{-0.021960in}}%
\pgfpathlineto{\pgfqpoint{0.000000in}{-0.021960in}}%
\pgfpathclose%
\pgfusepath{stroke,fill}%
}%
\begin{pgfscope}%
\pgfsys@transformshift{1.080663in}{9.899819in}%
\pgfsys@useobject{currentmarker}{}%
\end{pgfscope}%
\begin{pgfscope}%
\pgfsys@transformshift{1.260394in}{9.900851in}%
\pgfsys@useobject{currentmarker}{}%
\end{pgfscope}%
\begin{pgfscope}%
\pgfsys@transformshift{1.440125in}{9.902135in}%
\pgfsys@useobject{currentmarker}{}%
\end{pgfscope}%
\begin{pgfscope}%
\pgfsys@transformshift{1.619856in}{9.903297in}%
\pgfsys@useobject{currentmarker}{}%
\end{pgfscope}%
\begin{pgfscope}%
\pgfsys@transformshift{1.799587in}{9.905530in}%
\pgfsys@useobject{currentmarker}{}%
\end{pgfscope}%
\begin{pgfscope}%
\pgfsys@transformshift{1.979318in}{9.906911in}%
\pgfsys@useobject{currentmarker}{}%
\end{pgfscope}%
\begin{pgfscope}%
\pgfsys@transformshift{2.159049in}{9.914432in}%
\pgfsys@useobject{currentmarker}{}%
\end{pgfscope}%
\begin{pgfscope}%
\pgfsys@transformshift{2.338780in}{9.921424in}%
\pgfsys@useobject{currentmarker}{}%
\end{pgfscope}%
\begin{pgfscope}%
\pgfsys@transformshift{2.518511in}{9.928775in}%
\pgfsys@useobject{currentmarker}{}%
\end{pgfscope}%
\begin{pgfscope}%
\pgfsys@transformshift{2.698242in}{10.018474in}%
\pgfsys@useobject{currentmarker}{}%
\end{pgfscope}%
\begin{pgfscope}%
\pgfsys@transformshift{2.877973in}{10.446193in}%
\pgfsys@useobject{currentmarker}{}%
\end{pgfscope}%
\begin{pgfscope}%
\pgfsys@transformshift{3.057704in}{10.781520in}%
\pgfsys@useobject{currentmarker}{}%
\end{pgfscope}%
\begin{pgfscope}%
\pgfsys@transformshift{3.237435in}{11.000641in}%
\pgfsys@useobject{currentmarker}{}%
\end{pgfscope}%
\begin{pgfscope}%
\pgfsys@transformshift{3.417166in}{11.181789in}%
\pgfsys@useobject{currentmarker}{}%
\end{pgfscope}%
\begin{pgfscope}%
\pgfsys@transformshift{3.596897in}{11.302996in}%
\pgfsys@useobject{currentmarker}{}%
\end{pgfscope}%
\begin{pgfscope}%
\pgfsys@transformshift{3.776628in}{11.438634in}%
\pgfsys@useobject{currentmarker}{}%
\end{pgfscope}%
\begin{pgfscope}%
\pgfsys@transformshift{3.956359in}{11.531605in}%
\pgfsys@useobject{currentmarker}{}%
\end{pgfscope}%
\begin{pgfscope}%
\pgfsys@transformshift{4.136090in}{11.611542in}%
\pgfsys@useobject{currentmarker}{}%
\end{pgfscope}%
\begin{pgfscope}%
\pgfsys@transformshift{4.315821in}{11.691697in}%
\pgfsys@useobject{currentmarker}{}%
\end{pgfscope}%
\begin{pgfscope}%
\pgfsys@transformshift{4.495552in}{11.753551in}%
\pgfsys@useobject{currentmarker}{}%
\end{pgfscope}%
\begin{pgfscope}%
\pgfsys@transformshift{4.675283in}{11.807035in}%
\pgfsys@useobject{currentmarker}{}%
\end{pgfscope}%
\begin{pgfscope}%
\pgfsys@transformshift{4.855014in}{11.859116in}%
\pgfsys@useobject{currentmarker}{}%
\end{pgfscope}%
\begin{pgfscope}%
\pgfsys@transformshift{5.034745in}{11.908011in}%
\pgfsys@useobject{currentmarker}{}%
\end{pgfscope}%
\begin{pgfscope}%
\pgfsys@transformshift{5.214476in}{11.956993in}%
\pgfsys@useobject{currentmarker}{}%
\end{pgfscope}%
\begin{pgfscope}%
\pgfsys@transformshift{5.394207in}{11.990519in}%
\pgfsys@useobject{currentmarker}{}%
\end{pgfscope}%
\begin{pgfscope}%
\pgfsys@transformshift{5.573938in}{12.034204in}%
\pgfsys@useobject{currentmarker}{}%
\end{pgfscope}%
\begin{pgfscope}%
\pgfsys@transformshift{5.753669in}{12.060341in}%
\pgfsys@useobject{currentmarker}{}%
\end{pgfscope}%
\begin{pgfscope}%
\pgfsys@transformshift{5.933400in}{12.094457in}%
\pgfsys@useobject{currentmarker}{}%
\end{pgfscope}%
\begin{pgfscope}%
\pgfsys@transformshift{6.113131in}{12.126294in}%
\pgfsys@useobject{currentmarker}{}%
\end{pgfscope}%
\begin{pgfscope}%
\pgfsys@transformshift{6.292862in}{12.151679in}%
\pgfsys@useobject{currentmarker}{}%
\end{pgfscope}%
\begin{pgfscope}%
\pgfsys@transformshift{6.472593in}{12.183336in}%
\pgfsys@useobject{currentmarker}{}%
\end{pgfscope}%
\begin{pgfscope}%
\pgfsys@transformshift{6.652324in}{12.204020in}%
\pgfsys@useobject{currentmarker}{}%
\end{pgfscope}%
\begin{pgfscope}%
\pgfsys@transformshift{6.832055in}{12.222357in}%
\pgfsys@useobject{currentmarker}{}%
\end{pgfscope}%
\begin{pgfscope}%
\pgfsys@transformshift{7.011786in}{12.241806in}%
\pgfsys@useobject{currentmarker}{}%
\end{pgfscope}%
\begin{pgfscope}%
\pgfsys@transformshift{7.191517in}{12.267825in}%
\pgfsys@useobject{currentmarker}{}%
\end{pgfscope}%
\begin{pgfscope}%
\pgfsys@transformshift{7.371248in}{12.285156in}%
\pgfsys@useobject{currentmarker}{}%
\end{pgfscope}%
\begin{pgfscope}%
\pgfsys@transformshift{7.550979in}{12.303611in}%
\pgfsys@useobject{currentmarker}{}%
\end{pgfscope}%
\begin{pgfscope}%
\pgfsys@transformshift{7.730710in}{12.314826in}%
\pgfsys@useobject{currentmarker}{}%
\end{pgfscope}%
\begin{pgfscope}%
\pgfsys@transformshift{7.910441in}{12.333802in}%
\pgfsys@useobject{currentmarker}{}%
\end{pgfscope}%
\begin{pgfscope}%
\pgfsys@transformshift{8.090172in}{12.342788in}%
\pgfsys@useobject{currentmarker}{}%
\end{pgfscope}%
\begin{pgfscope}%
\pgfsys@transformshift{8.269903in}{12.363267in}%
\pgfsys@useobject{currentmarker}{}%
\end{pgfscope}%
\begin{pgfscope}%
\pgfsys@transformshift{8.449634in}{12.373190in}%
\pgfsys@useobject{currentmarker}{}%
\end{pgfscope}%
\begin{pgfscope}%
\pgfsys@transformshift{8.629365in}{12.393806in}%
\pgfsys@useobject{currentmarker}{}%
\end{pgfscope}%
\begin{pgfscope}%
\pgfsys@transformshift{8.809096in}{12.406996in}%
\pgfsys@useobject{currentmarker}{}%
\end{pgfscope}%
\begin{pgfscope}%
\pgfsys@transformshift{8.988827in}{12.417900in}%
\pgfsys@useobject{currentmarker}{}%
\end{pgfscope}%
\begin{pgfscope}%
\pgfsys@transformshift{9.168558in}{12.433250in}%
\pgfsys@useobject{currentmarker}{}%
\end{pgfscope}%
\begin{pgfscope}%
\pgfsys@transformshift{9.348289in}{12.442900in}%
\pgfsys@useobject{currentmarker}{}%
\end{pgfscope}%
\begin{pgfscope}%
\pgfsys@transformshift{9.528020in}{12.449694in}%
\pgfsys@useobject{currentmarker}{}%
\end{pgfscope}%
\begin{pgfscope}%
\pgfsys@transformshift{9.707751in}{12.460871in}%
\pgfsys@useobject{currentmarker}{}%
\end{pgfscope}%
\begin{pgfscope}%
\pgfsys@transformshift{9.887482in}{12.477637in}%
\pgfsys@useobject{currentmarker}{}%
\end{pgfscope}%
\end{pgfscope}%
\begin{pgfscope}%
\pgfpathrectangle{\pgfqpoint{0.640323in}{9.767436in}}{\pgfqpoint{9.687500in}{3.850000in}}%
\pgfusepath{clip}%
\pgfsetbuttcap%
\pgfsetroundjoin%
\definecolor{currentfill}{rgb}{0.000000,0.000000,0.000000}%
\pgfsetfillcolor{currentfill}%
\pgfsetfillopacity{0.500000}%
\pgfsetlinewidth{1.003750pt}%
\definecolor{currentstroke}{rgb}{0.000000,0.000000,0.000000}%
\pgfsetstrokecolor{currentstroke}%
\pgfsetstrokeopacity{0.500000}%
\pgfsetdash{}{0pt}%
\pgfsys@defobject{currentmarker}{\pgfqpoint{-0.021960in}{-0.021960in}}{\pgfqpoint{0.021960in}{0.021960in}}{%
\pgfpathmoveto{\pgfqpoint{0.000000in}{-0.021960in}}%
\pgfpathcurveto{\pgfqpoint{0.005824in}{-0.021960in}}{\pgfqpoint{0.011410in}{-0.019646in}}{\pgfqpoint{0.015528in}{-0.015528in}}%
\pgfpathcurveto{\pgfqpoint{0.019646in}{-0.011410in}}{\pgfqpoint{0.021960in}{-0.005824in}}{\pgfqpoint{0.021960in}{0.000000in}}%
\pgfpathcurveto{\pgfqpoint{0.021960in}{0.005824in}}{\pgfqpoint{0.019646in}{0.011410in}}{\pgfqpoint{0.015528in}{0.015528in}}%
\pgfpathcurveto{\pgfqpoint{0.011410in}{0.019646in}}{\pgfqpoint{0.005824in}{0.021960in}}{\pgfqpoint{0.000000in}{0.021960in}}%
\pgfpathcurveto{\pgfqpoint{-0.005824in}{0.021960in}}{\pgfqpoint{-0.011410in}{0.019646in}}{\pgfqpoint{-0.015528in}{0.015528in}}%
\pgfpathcurveto{\pgfqpoint{-0.019646in}{0.011410in}}{\pgfqpoint{-0.021960in}{0.005824in}}{\pgfqpoint{-0.021960in}{0.000000in}}%
\pgfpathcurveto{\pgfqpoint{-0.021960in}{-0.005824in}}{\pgfqpoint{-0.019646in}{-0.011410in}}{\pgfqpoint{-0.015528in}{-0.015528in}}%
\pgfpathcurveto{\pgfqpoint{-0.011410in}{-0.019646in}}{\pgfqpoint{-0.005824in}{-0.021960in}}{\pgfqpoint{0.000000in}{-0.021960in}}%
\pgfpathlineto{\pgfqpoint{0.000000in}{-0.021960in}}%
\pgfpathclose%
\pgfusepath{stroke,fill}%
}%
\begin{pgfscope}%
\pgfsys@transformshift{1.080663in}{9.899636in}%
\pgfsys@useobject{currentmarker}{}%
\end{pgfscope}%
\begin{pgfscope}%
\pgfsys@transformshift{1.260394in}{9.900663in}%
\pgfsys@useobject{currentmarker}{}%
\end{pgfscope}%
\begin{pgfscope}%
\pgfsys@transformshift{1.440125in}{9.901686in}%
\pgfsys@useobject{currentmarker}{}%
\end{pgfscope}%
\begin{pgfscope}%
\pgfsys@transformshift{1.619856in}{9.903556in}%
\pgfsys@useobject{currentmarker}{}%
\end{pgfscope}%
\begin{pgfscope}%
\pgfsys@transformshift{1.799587in}{9.904833in}%
\pgfsys@useobject{currentmarker}{}%
\end{pgfscope}%
\begin{pgfscope}%
\pgfsys@transformshift{1.979318in}{9.908198in}%
\pgfsys@useobject{currentmarker}{}%
\end{pgfscope}%
\begin{pgfscope}%
\pgfsys@transformshift{2.159049in}{9.913532in}%
\pgfsys@useobject{currentmarker}{}%
\end{pgfscope}%
\begin{pgfscope}%
\pgfsys@transformshift{2.338780in}{9.920951in}%
\pgfsys@useobject{currentmarker}{}%
\end{pgfscope}%
\begin{pgfscope}%
\pgfsys@transformshift{2.518511in}{9.939267in}%
\pgfsys@useobject{currentmarker}{}%
\end{pgfscope}%
\begin{pgfscope}%
\pgfsys@transformshift{2.698242in}{10.084847in}%
\pgfsys@useobject{currentmarker}{}%
\end{pgfscope}%
\begin{pgfscope}%
\pgfsys@transformshift{2.877973in}{10.523607in}%
\pgfsys@useobject{currentmarker}{}%
\end{pgfscope}%
\begin{pgfscope}%
\pgfsys@transformshift{3.057704in}{10.826707in}%
\pgfsys@useobject{currentmarker}{}%
\end{pgfscope}%
\begin{pgfscope}%
\pgfsys@transformshift{3.237435in}{11.037321in}%
\pgfsys@useobject{currentmarker}{}%
\end{pgfscope}%
\begin{pgfscope}%
\pgfsys@transformshift{3.417166in}{11.207379in}%
\pgfsys@useobject{currentmarker}{}%
\end{pgfscope}%
\begin{pgfscope}%
\pgfsys@transformshift{3.596897in}{11.347308in}%
\pgfsys@useobject{currentmarker}{}%
\end{pgfscope}%
\begin{pgfscope}%
\pgfsys@transformshift{3.776628in}{11.451761in}%
\pgfsys@useobject{currentmarker}{}%
\end{pgfscope}%
\begin{pgfscope}%
\pgfsys@transformshift{3.956359in}{11.552339in}%
\pgfsys@useobject{currentmarker}{}%
\end{pgfscope}%
\begin{pgfscope}%
\pgfsys@transformshift{4.136090in}{11.631910in}%
\pgfsys@useobject{currentmarker}{}%
\end{pgfscope}%
\begin{pgfscope}%
\pgfsys@transformshift{4.315821in}{11.706060in}%
\pgfsys@useobject{currentmarker}{}%
\end{pgfscope}%
\begin{pgfscope}%
\pgfsys@transformshift{4.495552in}{11.767666in}%
\pgfsys@useobject{currentmarker}{}%
\end{pgfscope}%
\begin{pgfscope}%
\pgfsys@transformshift{4.675283in}{11.823696in}%
\pgfsys@useobject{currentmarker}{}%
\end{pgfscope}%
\begin{pgfscope}%
\pgfsys@transformshift{4.855014in}{11.877820in}%
\pgfsys@useobject{currentmarker}{}%
\end{pgfscope}%
\begin{pgfscope}%
\pgfsys@transformshift{5.034745in}{11.921436in}%
\pgfsys@useobject{currentmarker}{}%
\end{pgfscope}%
\begin{pgfscope}%
\pgfsys@transformshift{5.214476in}{11.968543in}%
\pgfsys@useobject{currentmarker}{}%
\end{pgfscope}%
\begin{pgfscope}%
\pgfsys@transformshift{5.394207in}{12.006813in}%
\pgfsys@useobject{currentmarker}{}%
\end{pgfscope}%
\begin{pgfscope}%
\pgfsys@transformshift{5.573938in}{12.045121in}%
\pgfsys@useobject{currentmarker}{}%
\end{pgfscope}%
\begin{pgfscope}%
\pgfsys@transformshift{5.753669in}{12.074294in}%
\pgfsys@useobject{currentmarker}{}%
\end{pgfscope}%
\begin{pgfscope}%
\pgfsys@transformshift{5.933400in}{12.102281in}%
\pgfsys@useobject{currentmarker}{}%
\end{pgfscope}%
\begin{pgfscope}%
\pgfsys@transformshift{6.113131in}{12.135099in}%
\pgfsys@useobject{currentmarker}{}%
\end{pgfscope}%
\begin{pgfscope}%
\pgfsys@transformshift{6.292862in}{12.156907in}%
\pgfsys@useobject{currentmarker}{}%
\end{pgfscope}%
\begin{pgfscope}%
\pgfsys@transformshift{6.472593in}{12.187496in}%
\pgfsys@useobject{currentmarker}{}%
\end{pgfscope}%
\begin{pgfscope}%
\pgfsys@transformshift{6.652324in}{12.211484in}%
\pgfsys@useobject{currentmarker}{}%
\end{pgfscope}%
\begin{pgfscope}%
\pgfsys@transformshift{6.832055in}{12.234019in}%
\pgfsys@useobject{currentmarker}{}%
\end{pgfscope}%
\begin{pgfscope}%
\pgfsys@transformshift{7.011786in}{12.253480in}%
\pgfsys@useobject{currentmarker}{}%
\end{pgfscope}%
\begin{pgfscope}%
\pgfsys@transformshift{7.191517in}{12.277207in}%
\pgfsys@useobject{currentmarker}{}%
\end{pgfscope}%
\begin{pgfscope}%
\pgfsys@transformshift{7.371248in}{12.291862in}%
\pgfsys@useobject{currentmarker}{}%
\end{pgfscope}%
\begin{pgfscope}%
\pgfsys@transformshift{7.550979in}{12.306946in}%
\pgfsys@useobject{currentmarker}{}%
\end{pgfscope}%
\begin{pgfscope}%
\pgfsys@transformshift{7.730710in}{12.324848in}%
\pgfsys@useobject{currentmarker}{}%
\end{pgfscope}%
\begin{pgfscope}%
\pgfsys@transformshift{7.910441in}{12.338280in}%
\pgfsys@useobject{currentmarker}{}%
\end{pgfscope}%
\begin{pgfscope}%
\pgfsys@transformshift{8.090172in}{12.357163in}%
\pgfsys@useobject{currentmarker}{}%
\end{pgfscope}%
\begin{pgfscope}%
\pgfsys@transformshift{8.269903in}{12.370160in}%
\pgfsys@useobject{currentmarker}{}%
\end{pgfscope}%
\begin{pgfscope}%
\pgfsys@transformshift{8.449634in}{12.383151in}%
\pgfsys@useobject{currentmarker}{}%
\end{pgfscope}%
\begin{pgfscope}%
\pgfsys@transformshift{8.629365in}{12.397725in}%
\pgfsys@useobject{currentmarker}{}%
\end{pgfscope}%
\begin{pgfscope}%
\pgfsys@transformshift{8.809096in}{12.408387in}%
\pgfsys@useobject{currentmarker}{}%
\end{pgfscope}%
\begin{pgfscope}%
\pgfsys@transformshift{8.988827in}{12.418701in}%
\pgfsys@useobject{currentmarker}{}%
\end{pgfscope}%
\begin{pgfscope}%
\pgfsys@transformshift{9.168558in}{12.434908in}%
\pgfsys@useobject{currentmarker}{}%
\end{pgfscope}%
\begin{pgfscope}%
\pgfsys@transformshift{9.348289in}{12.445719in}%
\pgfsys@useobject{currentmarker}{}%
\end{pgfscope}%
\begin{pgfscope}%
\pgfsys@transformshift{9.528020in}{12.454338in}%
\pgfsys@useobject{currentmarker}{}%
\end{pgfscope}%
\begin{pgfscope}%
\pgfsys@transformshift{9.707751in}{12.465168in}%
\pgfsys@useobject{currentmarker}{}%
\end{pgfscope}%
\begin{pgfscope}%
\pgfsys@transformshift{9.887482in}{12.471502in}%
\pgfsys@useobject{currentmarker}{}%
\end{pgfscope}%
\end{pgfscope}%
\begin{pgfscope}%
\pgfpathrectangle{\pgfqpoint{0.640323in}{9.767436in}}{\pgfqpoint{9.687500in}{3.850000in}}%
\pgfusepath{clip}%
\pgfsetrectcap%
\pgfsetroundjoin%
\pgfsetlinewidth{0.803000pt}%
\definecolor{currentstroke}{rgb}{0.690196,0.690196,0.690196}%
\pgfsetstrokecolor{currentstroke}%
\pgfsetdash{}{0pt}%
\pgfpathmoveto{\pgfqpoint{1.080663in}{9.767436in}}%
\pgfpathlineto{\pgfqpoint{1.080663in}{13.617436in}}%
\pgfusepath{stroke}%
\end{pgfscope}%
\begin{pgfscope}%
\pgfsetbuttcap%
\pgfsetroundjoin%
\definecolor{currentfill}{rgb}{0.000000,0.000000,0.000000}%
\pgfsetfillcolor{currentfill}%
\pgfsetlinewidth{0.803000pt}%
\definecolor{currentstroke}{rgb}{0.000000,0.000000,0.000000}%
\pgfsetstrokecolor{currentstroke}%
\pgfsetdash{}{0pt}%
\pgfsys@defobject{currentmarker}{\pgfqpoint{0.000000in}{-0.048611in}}{\pgfqpoint{0.000000in}{0.000000in}}{%
\pgfpathmoveto{\pgfqpoint{0.000000in}{0.000000in}}%
\pgfpathlineto{\pgfqpoint{0.000000in}{-0.048611in}}%
\pgfusepath{stroke,fill}%
}%
\begin{pgfscope}%
\pgfsys@transformshift{1.080663in}{9.767436in}%
\pgfsys@useobject{currentmarker}{}%
\end{pgfscope}%
\end{pgfscope}%
\begin{pgfscope}%
\definecolor{textcolor}{rgb}{0.000000,0.000000,0.000000}%
\pgfsetstrokecolor{textcolor}%
\pgfsetfillcolor{textcolor}%
\pgftext[x=1.080663in,y=9.670214in,,top]{\color{textcolor}\sffamily\fontsize{10.000000}{12.000000}\selectfont 0.0}%
\end{pgfscope}%
\begin{pgfscope}%
\pgfpathrectangle{\pgfqpoint{0.640323in}{9.767436in}}{\pgfqpoint{9.687500in}{3.850000in}}%
\pgfusepath{clip}%
\pgfsetrectcap%
\pgfsetroundjoin%
\pgfsetlinewidth{0.803000pt}%
\definecolor{currentstroke}{rgb}{0.690196,0.690196,0.690196}%
\pgfsetstrokecolor{currentstroke}%
\pgfsetdash{}{0pt}%
\pgfpathmoveto{\pgfqpoint{2.877973in}{9.767436in}}%
\pgfpathlineto{\pgfqpoint{2.877973in}{13.617436in}}%
\pgfusepath{stroke}%
\end{pgfscope}%
\begin{pgfscope}%
\pgfsetbuttcap%
\pgfsetroundjoin%
\definecolor{currentfill}{rgb}{0.000000,0.000000,0.000000}%
\pgfsetfillcolor{currentfill}%
\pgfsetlinewidth{0.803000pt}%
\definecolor{currentstroke}{rgb}{0.000000,0.000000,0.000000}%
\pgfsetstrokecolor{currentstroke}%
\pgfsetdash{}{0pt}%
\pgfsys@defobject{currentmarker}{\pgfqpoint{0.000000in}{-0.048611in}}{\pgfqpoint{0.000000in}{0.000000in}}{%
\pgfpathmoveto{\pgfqpoint{0.000000in}{0.000000in}}%
\pgfpathlineto{\pgfqpoint{0.000000in}{-0.048611in}}%
\pgfusepath{stroke,fill}%
}%
\begin{pgfscope}%
\pgfsys@transformshift{2.877973in}{9.767436in}%
\pgfsys@useobject{currentmarker}{}%
\end{pgfscope}%
\end{pgfscope}%
\begin{pgfscope}%
\definecolor{textcolor}{rgb}{0.000000,0.000000,0.000000}%
\pgfsetstrokecolor{textcolor}%
\pgfsetfillcolor{textcolor}%
\pgftext[x=2.877973in,y=9.670214in,,top]{\color{textcolor}\sffamily\fontsize{10.000000}{12.000000}\selectfont 0.2}%
\end{pgfscope}%
\begin{pgfscope}%
\pgfpathrectangle{\pgfqpoint{0.640323in}{9.767436in}}{\pgfqpoint{9.687500in}{3.850000in}}%
\pgfusepath{clip}%
\pgfsetrectcap%
\pgfsetroundjoin%
\pgfsetlinewidth{0.803000pt}%
\definecolor{currentstroke}{rgb}{0.690196,0.690196,0.690196}%
\pgfsetstrokecolor{currentstroke}%
\pgfsetdash{}{0pt}%
\pgfpathmoveto{\pgfqpoint{4.675283in}{9.767436in}}%
\pgfpathlineto{\pgfqpoint{4.675283in}{13.617436in}}%
\pgfusepath{stroke}%
\end{pgfscope}%
\begin{pgfscope}%
\pgfsetbuttcap%
\pgfsetroundjoin%
\definecolor{currentfill}{rgb}{0.000000,0.000000,0.000000}%
\pgfsetfillcolor{currentfill}%
\pgfsetlinewidth{0.803000pt}%
\definecolor{currentstroke}{rgb}{0.000000,0.000000,0.000000}%
\pgfsetstrokecolor{currentstroke}%
\pgfsetdash{}{0pt}%
\pgfsys@defobject{currentmarker}{\pgfqpoint{0.000000in}{-0.048611in}}{\pgfqpoint{0.000000in}{0.000000in}}{%
\pgfpathmoveto{\pgfqpoint{0.000000in}{0.000000in}}%
\pgfpathlineto{\pgfqpoint{0.000000in}{-0.048611in}}%
\pgfusepath{stroke,fill}%
}%
\begin{pgfscope}%
\pgfsys@transformshift{4.675283in}{9.767436in}%
\pgfsys@useobject{currentmarker}{}%
\end{pgfscope}%
\end{pgfscope}%
\begin{pgfscope}%
\definecolor{textcolor}{rgb}{0.000000,0.000000,0.000000}%
\pgfsetstrokecolor{textcolor}%
\pgfsetfillcolor{textcolor}%
\pgftext[x=4.675283in,y=9.670214in,,top]{\color{textcolor}\sffamily\fontsize{10.000000}{12.000000}\selectfont 0.4}%
\end{pgfscope}%
\begin{pgfscope}%
\pgfpathrectangle{\pgfqpoint{0.640323in}{9.767436in}}{\pgfqpoint{9.687500in}{3.850000in}}%
\pgfusepath{clip}%
\pgfsetrectcap%
\pgfsetroundjoin%
\pgfsetlinewidth{0.803000pt}%
\definecolor{currentstroke}{rgb}{0.690196,0.690196,0.690196}%
\pgfsetstrokecolor{currentstroke}%
\pgfsetdash{}{0pt}%
\pgfpathmoveto{\pgfqpoint{6.472593in}{9.767436in}}%
\pgfpathlineto{\pgfqpoint{6.472593in}{13.617436in}}%
\pgfusepath{stroke}%
\end{pgfscope}%
\begin{pgfscope}%
\pgfsetbuttcap%
\pgfsetroundjoin%
\definecolor{currentfill}{rgb}{0.000000,0.000000,0.000000}%
\pgfsetfillcolor{currentfill}%
\pgfsetlinewidth{0.803000pt}%
\definecolor{currentstroke}{rgb}{0.000000,0.000000,0.000000}%
\pgfsetstrokecolor{currentstroke}%
\pgfsetdash{}{0pt}%
\pgfsys@defobject{currentmarker}{\pgfqpoint{0.000000in}{-0.048611in}}{\pgfqpoint{0.000000in}{0.000000in}}{%
\pgfpathmoveto{\pgfqpoint{0.000000in}{0.000000in}}%
\pgfpathlineto{\pgfqpoint{0.000000in}{-0.048611in}}%
\pgfusepath{stroke,fill}%
}%
\begin{pgfscope}%
\pgfsys@transformshift{6.472593in}{9.767436in}%
\pgfsys@useobject{currentmarker}{}%
\end{pgfscope}%
\end{pgfscope}%
\begin{pgfscope}%
\definecolor{textcolor}{rgb}{0.000000,0.000000,0.000000}%
\pgfsetstrokecolor{textcolor}%
\pgfsetfillcolor{textcolor}%
\pgftext[x=6.472593in,y=9.670214in,,top]{\color{textcolor}\sffamily\fontsize{10.000000}{12.000000}\selectfont 0.6}%
\end{pgfscope}%
\begin{pgfscope}%
\pgfpathrectangle{\pgfqpoint{0.640323in}{9.767436in}}{\pgfqpoint{9.687500in}{3.850000in}}%
\pgfusepath{clip}%
\pgfsetrectcap%
\pgfsetroundjoin%
\pgfsetlinewidth{0.803000pt}%
\definecolor{currentstroke}{rgb}{0.690196,0.690196,0.690196}%
\pgfsetstrokecolor{currentstroke}%
\pgfsetdash{}{0pt}%
\pgfpathmoveto{\pgfqpoint{8.269903in}{9.767436in}}%
\pgfpathlineto{\pgfqpoint{8.269903in}{13.617436in}}%
\pgfusepath{stroke}%
\end{pgfscope}%
\begin{pgfscope}%
\pgfsetbuttcap%
\pgfsetroundjoin%
\definecolor{currentfill}{rgb}{0.000000,0.000000,0.000000}%
\pgfsetfillcolor{currentfill}%
\pgfsetlinewidth{0.803000pt}%
\definecolor{currentstroke}{rgb}{0.000000,0.000000,0.000000}%
\pgfsetstrokecolor{currentstroke}%
\pgfsetdash{}{0pt}%
\pgfsys@defobject{currentmarker}{\pgfqpoint{0.000000in}{-0.048611in}}{\pgfqpoint{0.000000in}{0.000000in}}{%
\pgfpathmoveto{\pgfqpoint{0.000000in}{0.000000in}}%
\pgfpathlineto{\pgfqpoint{0.000000in}{-0.048611in}}%
\pgfusepath{stroke,fill}%
}%
\begin{pgfscope}%
\pgfsys@transformshift{8.269903in}{9.767436in}%
\pgfsys@useobject{currentmarker}{}%
\end{pgfscope}%
\end{pgfscope}%
\begin{pgfscope}%
\definecolor{textcolor}{rgb}{0.000000,0.000000,0.000000}%
\pgfsetstrokecolor{textcolor}%
\pgfsetfillcolor{textcolor}%
\pgftext[x=8.269903in,y=9.670214in,,top]{\color{textcolor}\sffamily\fontsize{10.000000}{12.000000}\selectfont 0.8}%
\end{pgfscope}%
\begin{pgfscope}%
\pgfpathrectangle{\pgfqpoint{0.640323in}{9.767436in}}{\pgfqpoint{9.687500in}{3.850000in}}%
\pgfusepath{clip}%
\pgfsetrectcap%
\pgfsetroundjoin%
\pgfsetlinewidth{0.803000pt}%
\definecolor{currentstroke}{rgb}{0.690196,0.690196,0.690196}%
\pgfsetstrokecolor{currentstroke}%
\pgfsetdash{}{0pt}%
\pgfpathmoveto{\pgfqpoint{10.067213in}{9.767436in}}%
\pgfpathlineto{\pgfqpoint{10.067213in}{13.617436in}}%
\pgfusepath{stroke}%
\end{pgfscope}%
\begin{pgfscope}%
\pgfsetbuttcap%
\pgfsetroundjoin%
\definecolor{currentfill}{rgb}{0.000000,0.000000,0.000000}%
\pgfsetfillcolor{currentfill}%
\pgfsetlinewidth{0.803000pt}%
\definecolor{currentstroke}{rgb}{0.000000,0.000000,0.000000}%
\pgfsetstrokecolor{currentstroke}%
\pgfsetdash{}{0pt}%
\pgfsys@defobject{currentmarker}{\pgfqpoint{0.000000in}{-0.048611in}}{\pgfqpoint{0.000000in}{0.000000in}}{%
\pgfpathmoveto{\pgfqpoint{0.000000in}{0.000000in}}%
\pgfpathlineto{\pgfqpoint{0.000000in}{-0.048611in}}%
\pgfusepath{stroke,fill}%
}%
\begin{pgfscope}%
\pgfsys@transformshift{10.067213in}{9.767436in}%
\pgfsys@useobject{currentmarker}{}%
\end{pgfscope}%
\end{pgfscope}%
\begin{pgfscope}%
\definecolor{textcolor}{rgb}{0.000000,0.000000,0.000000}%
\pgfsetstrokecolor{textcolor}%
\pgfsetfillcolor{textcolor}%
\pgftext[x=10.067213in,y=9.670214in,,top]{\color{textcolor}\sffamily\fontsize{10.000000}{12.000000}\selectfont 1.0}%
\end{pgfscope}%
\begin{pgfscope}%
\pgfpathrectangle{\pgfqpoint{0.640323in}{9.767436in}}{\pgfqpoint{9.687500in}{3.850000in}}%
\pgfusepath{clip}%
\pgfsetrectcap%
\pgfsetroundjoin%
\pgfsetlinewidth{0.803000pt}%
\definecolor{currentstroke}{rgb}{0.600000,0.600000,0.600000}%
\pgfsetstrokecolor{currentstroke}%
\pgfsetstrokeopacity{0.200000}%
\pgfsetdash{}{0pt}%
\pgfpathmoveto{\pgfqpoint{1.529991in}{9.767436in}}%
\pgfpathlineto{\pgfqpoint{1.529991in}{13.617436in}}%
\pgfusepath{stroke}%
\end{pgfscope}%
\begin{pgfscope}%
\pgfsetbuttcap%
\pgfsetroundjoin%
\definecolor{currentfill}{rgb}{0.000000,0.000000,0.000000}%
\pgfsetfillcolor{currentfill}%
\pgfsetlinewidth{0.602250pt}%
\definecolor{currentstroke}{rgb}{0.000000,0.000000,0.000000}%
\pgfsetstrokecolor{currentstroke}%
\pgfsetdash{}{0pt}%
\pgfsys@defobject{currentmarker}{\pgfqpoint{0.000000in}{-0.027778in}}{\pgfqpoint{0.000000in}{0.000000in}}{%
\pgfpathmoveto{\pgfqpoint{0.000000in}{0.000000in}}%
\pgfpathlineto{\pgfqpoint{0.000000in}{-0.027778in}}%
\pgfusepath{stroke,fill}%
}%
\begin{pgfscope}%
\pgfsys@transformshift{1.529991in}{9.767436in}%
\pgfsys@useobject{currentmarker}{}%
\end{pgfscope}%
\end{pgfscope}%
\begin{pgfscope}%
\pgfpathrectangle{\pgfqpoint{0.640323in}{9.767436in}}{\pgfqpoint{9.687500in}{3.850000in}}%
\pgfusepath{clip}%
\pgfsetrectcap%
\pgfsetroundjoin%
\pgfsetlinewidth{0.803000pt}%
\definecolor{currentstroke}{rgb}{0.600000,0.600000,0.600000}%
\pgfsetstrokecolor{currentstroke}%
\pgfsetstrokeopacity{0.200000}%
\pgfsetdash{}{0pt}%
\pgfpathmoveto{\pgfqpoint{1.979318in}{9.767436in}}%
\pgfpathlineto{\pgfqpoint{1.979318in}{13.617436in}}%
\pgfusepath{stroke}%
\end{pgfscope}%
\begin{pgfscope}%
\pgfsetbuttcap%
\pgfsetroundjoin%
\definecolor{currentfill}{rgb}{0.000000,0.000000,0.000000}%
\pgfsetfillcolor{currentfill}%
\pgfsetlinewidth{0.602250pt}%
\definecolor{currentstroke}{rgb}{0.000000,0.000000,0.000000}%
\pgfsetstrokecolor{currentstroke}%
\pgfsetdash{}{0pt}%
\pgfsys@defobject{currentmarker}{\pgfqpoint{0.000000in}{-0.027778in}}{\pgfqpoint{0.000000in}{0.000000in}}{%
\pgfpathmoveto{\pgfqpoint{0.000000in}{0.000000in}}%
\pgfpathlineto{\pgfqpoint{0.000000in}{-0.027778in}}%
\pgfusepath{stroke,fill}%
}%
\begin{pgfscope}%
\pgfsys@transformshift{1.979318in}{9.767436in}%
\pgfsys@useobject{currentmarker}{}%
\end{pgfscope}%
\end{pgfscope}%
\begin{pgfscope}%
\pgfpathrectangle{\pgfqpoint{0.640323in}{9.767436in}}{\pgfqpoint{9.687500in}{3.850000in}}%
\pgfusepath{clip}%
\pgfsetrectcap%
\pgfsetroundjoin%
\pgfsetlinewidth{0.803000pt}%
\definecolor{currentstroke}{rgb}{0.600000,0.600000,0.600000}%
\pgfsetstrokecolor{currentstroke}%
\pgfsetstrokeopacity{0.200000}%
\pgfsetdash{}{0pt}%
\pgfpathmoveto{\pgfqpoint{2.428646in}{9.767436in}}%
\pgfpathlineto{\pgfqpoint{2.428646in}{13.617436in}}%
\pgfusepath{stroke}%
\end{pgfscope}%
\begin{pgfscope}%
\pgfsetbuttcap%
\pgfsetroundjoin%
\definecolor{currentfill}{rgb}{0.000000,0.000000,0.000000}%
\pgfsetfillcolor{currentfill}%
\pgfsetlinewidth{0.602250pt}%
\definecolor{currentstroke}{rgb}{0.000000,0.000000,0.000000}%
\pgfsetstrokecolor{currentstroke}%
\pgfsetdash{}{0pt}%
\pgfsys@defobject{currentmarker}{\pgfqpoint{0.000000in}{-0.027778in}}{\pgfqpoint{0.000000in}{0.000000in}}{%
\pgfpathmoveto{\pgfqpoint{0.000000in}{0.000000in}}%
\pgfpathlineto{\pgfqpoint{0.000000in}{-0.027778in}}%
\pgfusepath{stroke,fill}%
}%
\begin{pgfscope}%
\pgfsys@transformshift{2.428646in}{9.767436in}%
\pgfsys@useobject{currentmarker}{}%
\end{pgfscope}%
\end{pgfscope}%
\begin{pgfscope}%
\pgfpathrectangle{\pgfqpoint{0.640323in}{9.767436in}}{\pgfqpoint{9.687500in}{3.850000in}}%
\pgfusepath{clip}%
\pgfsetrectcap%
\pgfsetroundjoin%
\pgfsetlinewidth{0.803000pt}%
\definecolor{currentstroke}{rgb}{0.600000,0.600000,0.600000}%
\pgfsetstrokecolor{currentstroke}%
\pgfsetstrokeopacity{0.200000}%
\pgfsetdash{}{0pt}%
\pgfpathmoveto{\pgfqpoint{3.327301in}{9.767436in}}%
\pgfpathlineto{\pgfqpoint{3.327301in}{13.617436in}}%
\pgfusepath{stroke}%
\end{pgfscope}%
\begin{pgfscope}%
\pgfsetbuttcap%
\pgfsetroundjoin%
\definecolor{currentfill}{rgb}{0.000000,0.000000,0.000000}%
\pgfsetfillcolor{currentfill}%
\pgfsetlinewidth{0.602250pt}%
\definecolor{currentstroke}{rgb}{0.000000,0.000000,0.000000}%
\pgfsetstrokecolor{currentstroke}%
\pgfsetdash{}{0pt}%
\pgfsys@defobject{currentmarker}{\pgfqpoint{0.000000in}{-0.027778in}}{\pgfqpoint{0.000000in}{0.000000in}}{%
\pgfpathmoveto{\pgfqpoint{0.000000in}{0.000000in}}%
\pgfpathlineto{\pgfqpoint{0.000000in}{-0.027778in}}%
\pgfusepath{stroke,fill}%
}%
\begin{pgfscope}%
\pgfsys@transformshift{3.327301in}{9.767436in}%
\pgfsys@useobject{currentmarker}{}%
\end{pgfscope}%
\end{pgfscope}%
\begin{pgfscope}%
\pgfpathrectangle{\pgfqpoint{0.640323in}{9.767436in}}{\pgfqpoint{9.687500in}{3.850000in}}%
\pgfusepath{clip}%
\pgfsetrectcap%
\pgfsetroundjoin%
\pgfsetlinewidth{0.803000pt}%
\definecolor{currentstroke}{rgb}{0.600000,0.600000,0.600000}%
\pgfsetstrokecolor{currentstroke}%
\pgfsetstrokeopacity{0.200000}%
\pgfsetdash{}{0pt}%
\pgfpathmoveto{\pgfqpoint{3.776628in}{9.767436in}}%
\pgfpathlineto{\pgfqpoint{3.776628in}{13.617436in}}%
\pgfusepath{stroke}%
\end{pgfscope}%
\begin{pgfscope}%
\pgfsetbuttcap%
\pgfsetroundjoin%
\definecolor{currentfill}{rgb}{0.000000,0.000000,0.000000}%
\pgfsetfillcolor{currentfill}%
\pgfsetlinewidth{0.602250pt}%
\definecolor{currentstroke}{rgb}{0.000000,0.000000,0.000000}%
\pgfsetstrokecolor{currentstroke}%
\pgfsetdash{}{0pt}%
\pgfsys@defobject{currentmarker}{\pgfqpoint{0.000000in}{-0.027778in}}{\pgfqpoint{0.000000in}{0.000000in}}{%
\pgfpathmoveto{\pgfqpoint{0.000000in}{0.000000in}}%
\pgfpathlineto{\pgfqpoint{0.000000in}{-0.027778in}}%
\pgfusepath{stroke,fill}%
}%
\begin{pgfscope}%
\pgfsys@transformshift{3.776628in}{9.767436in}%
\pgfsys@useobject{currentmarker}{}%
\end{pgfscope}%
\end{pgfscope}%
\begin{pgfscope}%
\pgfpathrectangle{\pgfqpoint{0.640323in}{9.767436in}}{\pgfqpoint{9.687500in}{3.850000in}}%
\pgfusepath{clip}%
\pgfsetrectcap%
\pgfsetroundjoin%
\pgfsetlinewidth{0.803000pt}%
\definecolor{currentstroke}{rgb}{0.600000,0.600000,0.600000}%
\pgfsetstrokecolor{currentstroke}%
\pgfsetstrokeopacity{0.200000}%
\pgfsetdash{}{0pt}%
\pgfpathmoveto{\pgfqpoint{4.225956in}{9.767436in}}%
\pgfpathlineto{\pgfqpoint{4.225956in}{13.617436in}}%
\pgfusepath{stroke}%
\end{pgfscope}%
\begin{pgfscope}%
\pgfsetbuttcap%
\pgfsetroundjoin%
\definecolor{currentfill}{rgb}{0.000000,0.000000,0.000000}%
\pgfsetfillcolor{currentfill}%
\pgfsetlinewidth{0.602250pt}%
\definecolor{currentstroke}{rgb}{0.000000,0.000000,0.000000}%
\pgfsetstrokecolor{currentstroke}%
\pgfsetdash{}{0pt}%
\pgfsys@defobject{currentmarker}{\pgfqpoint{0.000000in}{-0.027778in}}{\pgfqpoint{0.000000in}{0.000000in}}{%
\pgfpathmoveto{\pgfqpoint{0.000000in}{0.000000in}}%
\pgfpathlineto{\pgfqpoint{0.000000in}{-0.027778in}}%
\pgfusepath{stroke,fill}%
}%
\begin{pgfscope}%
\pgfsys@transformshift{4.225956in}{9.767436in}%
\pgfsys@useobject{currentmarker}{}%
\end{pgfscope}%
\end{pgfscope}%
\begin{pgfscope}%
\pgfpathrectangle{\pgfqpoint{0.640323in}{9.767436in}}{\pgfqpoint{9.687500in}{3.850000in}}%
\pgfusepath{clip}%
\pgfsetrectcap%
\pgfsetroundjoin%
\pgfsetlinewidth{0.803000pt}%
\definecolor{currentstroke}{rgb}{0.600000,0.600000,0.600000}%
\pgfsetstrokecolor{currentstroke}%
\pgfsetstrokeopacity{0.200000}%
\pgfsetdash{}{0pt}%
\pgfpathmoveto{\pgfqpoint{5.124611in}{9.767436in}}%
\pgfpathlineto{\pgfqpoint{5.124611in}{13.617436in}}%
\pgfusepath{stroke}%
\end{pgfscope}%
\begin{pgfscope}%
\pgfsetbuttcap%
\pgfsetroundjoin%
\definecolor{currentfill}{rgb}{0.000000,0.000000,0.000000}%
\pgfsetfillcolor{currentfill}%
\pgfsetlinewidth{0.602250pt}%
\definecolor{currentstroke}{rgb}{0.000000,0.000000,0.000000}%
\pgfsetstrokecolor{currentstroke}%
\pgfsetdash{}{0pt}%
\pgfsys@defobject{currentmarker}{\pgfqpoint{0.000000in}{-0.027778in}}{\pgfqpoint{0.000000in}{0.000000in}}{%
\pgfpathmoveto{\pgfqpoint{0.000000in}{0.000000in}}%
\pgfpathlineto{\pgfqpoint{0.000000in}{-0.027778in}}%
\pgfusepath{stroke,fill}%
}%
\begin{pgfscope}%
\pgfsys@transformshift{5.124611in}{9.767436in}%
\pgfsys@useobject{currentmarker}{}%
\end{pgfscope}%
\end{pgfscope}%
\begin{pgfscope}%
\pgfpathrectangle{\pgfqpoint{0.640323in}{9.767436in}}{\pgfqpoint{9.687500in}{3.850000in}}%
\pgfusepath{clip}%
\pgfsetrectcap%
\pgfsetroundjoin%
\pgfsetlinewidth{0.803000pt}%
\definecolor{currentstroke}{rgb}{0.600000,0.600000,0.600000}%
\pgfsetstrokecolor{currentstroke}%
\pgfsetstrokeopacity{0.200000}%
\pgfsetdash{}{0pt}%
\pgfpathmoveto{\pgfqpoint{5.573938in}{9.767436in}}%
\pgfpathlineto{\pgfqpoint{5.573938in}{13.617436in}}%
\pgfusepath{stroke}%
\end{pgfscope}%
\begin{pgfscope}%
\pgfsetbuttcap%
\pgfsetroundjoin%
\definecolor{currentfill}{rgb}{0.000000,0.000000,0.000000}%
\pgfsetfillcolor{currentfill}%
\pgfsetlinewidth{0.602250pt}%
\definecolor{currentstroke}{rgb}{0.000000,0.000000,0.000000}%
\pgfsetstrokecolor{currentstroke}%
\pgfsetdash{}{0pt}%
\pgfsys@defobject{currentmarker}{\pgfqpoint{0.000000in}{-0.027778in}}{\pgfqpoint{0.000000in}{0.000000in}}{%
\pgfpathmoveto{\pgfqpoint{0.000000in}{0.000000in}}%
\pgfpathlineto{\pgfqpoint{0.000000in}{-0.027778in}}%
\pgfusepath{stroke,fill}%
}%
\begin{pgfscope}%
\pgfsys@transformshift{5.573938in}{9.767436in}%
\pgfsys@useobject{currentmarker}{}%
\end{pgfscope}%
\end{pgfscope}%
\begin{pgfscope}%
\pgfpathrectangle{\pgfqpoint{0.640323in}{9.767436in}}{\pgfqpoint{9.687500in}{3.850000in}}%
\pgfusepath{clip}%
\pgfsetrectcap%
\pgfsetroundjoin%
\pgfsetlinewidth{0.803000pt}%
\definecolor{currentstroke}{rgb}{0.600000,0.600000,0.600000}%
\pgfsetstrokecolor{currentstroke}%
\pgfsetstrokeopacity{0.200000}%
\pgfsetdash{}{0pt}%
\pgfpathmoveto{\pgfqpoint{6.023265in}{9.767436in}}%
\pgfpathlineto{\pgfqpoint{6.023265in}{13.617436in}}%
\pgfusepath{stroke}%
\end{pgfscope}%
\begin{pgfscope}%
\pgfsetbuttcap%
\pgfsetroundjoin%
\definecolor{currentfill}{rgb}{0.000000,0.000000,0.000000}%
\pgfsetfillcolor{currentfill}%
\pgfsetlinewidth{0.602250pt}%
\definecolor{currentstroke}{rgb}{0.000000,0.000000,0.000000}%
\pgfsetstrokecolor{currentstroke}%
\pgfsetdash{}{0pt}%
\pgfsys@defobject{currentmarker}{\pgfqpoint{0.000000in}{-0.027778in}}{\pgfqpoint{0.000000in}{0.000000in}}{%
\pgfpathmoveto{\pgfqpoint{0.000000in}{0.000000in}}%
\pgfpathlineto{\pgfqpoint{0.000000in}{-0.027778in}}%
\pgfusepath{stroke,fill}%
}%
\begin{pgfscope}%
\pgfsys@transformshift{6.023265in}{9.767436in}%
\pgfsys@useobject{currentmarker}{}%
\end{pgfscope}%
\end{pgfscope}%
\begin{pgfscope}%
\pgfpathrectangle{\pgfqpoint{0.640323in}{9.767436in}}{\pgfqpoint{9.687500in}{3.850000in}}%
\pgfusepath{clip}%
\pgfsetrectcap%
\pgfsetroundjoin%
\pgfsetlinewidth{0.803000pt}%
\definecolor{currentstroke}{rgb}{0.600000,0.600000,0.600000}%
\pgfsetstrokecolor{currentstroke}%
\pgfsetstrokeopacity{0.200000}%
\pgfsetdash{}{0pt}%
\pgfpathmoveto{\pgfqpoint{6.921920in}{9.767436in}}%
\pgfpathlineto{\pgfqpoint{6.921920in}{13.617436in}}%
\pgfusepath{stroke}%
\end{pgfscope}%
\begin{pgfscope}%
\pgfsetbuttcap%
\pgfsetroundjoin%
\definecolor{currentfill}{rgb}{0.000000,0.000000,0.000000}%
\pgfsetfillcolor{currentfill}%
\pgfsetlinewidth{0.602250pt}%
\definecolor{currentstroke}{rgb}{0.000000,0.000000,0.000000}%
\pgfsetstrokecolor{currentstroke}%
\pgfsetdash{}{0pt}%
\pgfsys@defobject{currentmarker}{\pgfqpoint{0.000000in}{-0.027778in}}{\pgfqpoint{0.000000in}{0.000000in}}{%
\pgfpathmoveto{\pgfqpoint{0.000000in}{0.000000in}}%
\pgfpathlineto{\pgfqpoint{0.000000in}{-0.027778in}}%
\pgfusepath{stroke,fill}%
}%
\begin{pgfscope}%
\pgfsys@transformshift{6.921920in}{9.767436in}%
\pgfsys@useobject{currentmarker}{}%
\end{pgfscope}%
\end{pgfscope}%
\begin{pgfscope}%
\pgfpathrectangle{\pgfqpoint{0.640323in}{9.767436in}}{\pgfqpoint{9.687500in}{3.850000in}}%
\pgfusepath{clip}%
\pgfsetrectcap%
\pgfsetroundjoin%
\pgfsetlinewidth{0.803000pt}%
\definecolor{currentstroke}{rgb}{0.600000,0.600000,0.600000}%
\pgfsetstrokecolor{currentstroke}%
\pgfsetstrokeopacity{0.200000}%
\pgfsetdash{}{0pt}%
\pgfpathmoveto{\pgfqpoint{7.371248in}{9.767436in}}%
\pgfpathlineto{\pgfqpoint{7.371248in}{13.617436in}}%
\pgfusepath{stroke}%
\end{pgfscope}%
\begin{pgfscope}%
\pgfsetbuttcap%
\pgfsetroundjoin%
\definecolor{currentfill}{rgb}{0.000000,0.000000,0.000000}%
\pgfsetfillcolor{currentfill}%
\pgfsetlinewidth{0.602250pt}%
\definecolor{currentstroke}{rgb}{0.000000,0.000000,0.000000}%
\pgfsetstrokecolor{currentstroke}%
\pgfsetdash{}{0pt}%
\pgfsys@defobject{currentmarker}{\pgfqpoint{0.000000in}{-0.027778in}}{\pgfqpoint{0.000000in}{0.000000in}}{%
\pgfpathmoveto{\pgfqpoint{0.000000in}{0.000000in}}%
\pgfpathlineto{\pgfqpoint{0.000000in}{-0.027778in}}%
\pgfusepath{stroke,fill}%
}%
\begin{pgfscope}%
\pgfsys@transformshift{7.371248in}{9.767436in}%
\pgfsys@useobject{currentmarker}{}%
\end{pgfscope}%
\end{pgfscope}%
\begin{pgfscope}%
\pgfpathrectangle{\pgfqpoint{0.640323in}{9.767436in}}{\pgfqpoint{9.687500in}{3.850000in}}%
\pgfusepath{clip}%
\pgfsetrectcap%
\pgfsetroundjoin%
\pgfsetlinewidth{0.803000pt}%
\definecolor{currentstroke}{rgb}{0.600000,0.600000,0.600000}%
\pgfsetstrokecolor{currentstroke}%
\pgfsetstrokeopacity{0.200000}%
\pgfsetdash{}{0pt}%
\pgfpathmoveto{\pgfqpoint{7.820575in}{9.767436in}}%
\pgfpathlineto{\pgfqpoint{7.820575in}{13.617436in}}%
\pgfusepath{stroke}%
\end{pgfscope}%
\begin{pgfscope}%
\pgfsetbuttcap%
\pgfsetroundjoin%
\definecolor{currentfill}{rgb}{0.000000,0.000000,0.000000}%
\pgfsetfillcolor{currentfill}%
\pgfsetlinewidth{0.602250pt}%
\definecolor{currentstroke}{rgb}{0.000000,0.000000,0.000000}%
\pgfsetstrokecolor{currentstroke}%
\pgfsetdash{}{0pt}%
\pgfsys@defobject{currentmarker}{\pgfqpoint{0.000000in}{-0.027778in}}{\pgfqpoint{0.000000in}{0.000000in}}{%
\pgfpathmoveto{\pgfqpoint{0.000000in}{0.000000in}}%
\pgfpathlineto{\pgfqpoint{0.000000in}{-0.027778in}}%
\pgfusepath{stroke,fill}%
}%
\begin{pgfscope}%
\pgfsys@transformshift{7.820575in}{9.767436in}%
\pgfsys@useobject{currentmarker}{}%
\end{pgfscope}%
\end{pgfscope}%
\begin{pgfscope}%
\pgfpathrectangle{\pgfqpoint{0.640323in}{9.767436in}}{\pgfqpoint{9.687500in}{3.850000in}}%
\pgfusepath{clip}%
\pgfsetrectcap%
\pgfsetroundjoin%
\pgfsetlinewidth{0.803000pt}%
\definecolor{currentstroke}{rgb}{0.600000,0.600000,0.600000}%
\pgfsetstrokecolor{currentstroke}%
\pgfsetstrokeopacity{0.200000}%
\pgfsetdash{}{0pt}%
\pgfpathmoveto{\pgfqpoint{8.719230in}{9.767436in}}%
\pgfpathlineto{\pgfqpoint{8.719230in}{13.617436in}}%
\pgfusepath{stroke}%
\end{pgfscope}%
\begin{pgfscope}%
\pgfsetbuttcap%
\pgfsetroundjoin%
\definecolor{currentfill}{rgb}{0.000000,0.000000,0.000000}%
\pgfsetfillcolor{currentfill}%
\pgfsetlinewidth{0.602250pt}%
\definecolor{currentstroke}{rgb}{0.000000,0.000000,0.000000}%
\pgfsetstrokecolor{currentstroke}%
\pgfsetdash{}{0pt}%
\pgfsys@defobject{currentmarker}{\pgfqpoint{0.000000in}{-0.027778in}}{\pgfqpoint{0.000000in}{0.000000in}}{%
\pgfpathmoveto{\pgfqpoint{0.000000in}{0.000000in}}%
\pgfpathlineto{\pgfqpoint{0.000000in}{-0.027778in}}%
\pgfusepath{stroke,fill}%
}%
\begin{pgfscope}%
\pgfsys@transformshift{8.719230in}{9.767436in}%
\pgfsys@useobject{currentmarker}{}%
\end{pgfscope}%
\end{pgfscope}%
\begin{pgfscope}%
\pgfpathrectangle{\pgfqpoint{0.640323in}{9.767436in}}{\pgfqpoint{9.687500in}{3.850000in}}%
\pgfusepath{clip}%
\pgfsetrectcap%
\pgfsetroundjoin%
\pgfsetlinewidth{0.803000pt}%
\definecolor{currentstroke}{rgb}{0.600000,0.600000,0.600000}%
\pgfsetstrokecolor{currentstroke}%
\pgfsetstrokeopacity{0.200000}%
\pgfsetdash{}{0pt}%
\pgfpathmoveto{\pgfqpoint{9.168558in}{9.767436in}}%
\pgfpathlineto{\pgfqpoint{9.168558in}{13.617436in}}%
\pgfusepath{stroke}%
\end{pgfscope}%
\begin{pgfscope}%
\pgfsetbuttcap%
\pgfsetroundjoin%
\definecolor{currentfill}{rgb}{0.000000,0.000000,0.000000}%
\pgfsetfillcolor{currentfill}%
\pgfsetlinewidth{0.602250pt}%
\definecolor{currentstroke}{rgb}{0.000000,0.000000,0.000000}%
\pgfsetstrokecolor{currentstroke}%
\pgfsetdash{}{0pt}%
\pgfsys@defobject{currentmarker}{\pgfqpoint{0.000000in}{-0.027778in}}{\pgfqpoint{0.000000in}{0.000000in}}{%
\pgfpathmoveto{\pgfqpoint{0.000000in}{0.000000in}}%
\pgfpathlineto{\pgfqpoint{0.000000in}{-0.027778in}}%
\pgfusepath{stroke,fill}%
}%
\begin{pgfscope}%
\pgfsys@transformshift{9.168558in}{9.767436in}%
\pgfsys@useobject{currentmarker}{}%
\end{pgfscope}%
\end{pgfscope}%
\begin{pgfscope}%
\pgfpathrectangle{\pgfqpoint{0.640323in}{9.767436in}}{\pgfqpoint{9.687500in}{3.850000in}}%
\pgfusepath{clip}%
\pgfsetrectcap%
\pgfsetroundjoin%
\pgfsetlinewidth{0.803000pt}%
\definecolor{currentstroke}{rgb}{0.600000,0.600000,0.600000}%
\pgfsetstrokecolor{currentstroke}%
\pgfsetstrokeopacity{0.200000}%
\pgfsetdash{}{0pt}%
\pgfpathmoveto{\pgfqpoint{9.617885in}{9.767436in}}%
\pgfpathlineto{\pgfqpoint{9.617885in}{13.617436in}}%
\pgfusepath{stroke}%
\end{pgfscope}%
\begin{pgfscope}%
\pgfsetbuttcap%
\pgfsetroundjoin%
\definecolor{currentfill}{rgb}{0.000000,0.000000,0.000000}%
\pgfsetfillcolor{currentfill}%
\pgfsetlinewidth{0.602250pt}%
\definecolor{currentstroke}{rgb}{0.000000,0.000000,0.000000}%
\pgfsetstrokecolor{currentstroke}%
\pgfsetdash{}{0pt}%
\pgfsys@defobject{currentmarker}{\pgfqpoint{0.000000in}{-0.027778in}}{\pgfqpoint{0.000000in}{0.000000in}}{%
\pgfpathmoveto{\pgfqpoint{0.000000in}{0.000000in}}%
\pgfpathlineto{\pgfqpoint{0.000000in}{-0.027778in}}%
\pgfusepath{stroke,fill}%
}%
\begin{pgfscope}%
\pgfsys@transformshift{9.617885in}{9.767436in}%
\pgfsys@useobject{currentmarker}{}%
\end{pgfscope}%
\end{pgfscope}%
\begin{pgfscope}%
\definecolor{textcolor}{rgb}{0.000000,0.000000,0.000000}%
\pgfsetstrokecolor{textcolor}%
\pgfsetfillcolor{textcolor}%
\pgftext[x=5.484072in,y=9.480245in,,top]{\color{textcolor}\sffamily\fontsize{10.000000}{12.000000}\selectfont turnover probability \(\displaystyle p_1\,(S\rightarrow I\,)\)}%
\end{pgfscope}%
\begin{pgfscope}%
\pgfpathrectangle{\pgfqpoint{0.640323in}{9.767436in}}{\pgfqpoint{9.687500in}{3.850000in}}%
\pgfusepath{clip}%
\pgfsetrectcap%
\pgfsetroundjoin%
\pgfsetlinewidth{0.803000pt}%
\definecolor{currentstroke}{rgb}{0.690196,0.690196,0.690196}%
\pgfsetstrokecolor{currentstroke}%
\pgfsetdash{}{0pt}%
\pgfpathmoveto{\pgfqpoint{0.640323in}{9.891629in}}%
\pgfpathlineto{\pgfqpoint{10.327822in}{9.891629in}}%
\pgfusepath{stroke}%
\end{pgfscope}%
\begin{pgfscope}%
\pgfsetbuttcap%
\pgfsetroundjoin%
\definecolor{currentfill}{rgb}{0.000000,0.000000,0.000000}%
\pgfsetfillcolor{currentfill}%
\pgfsetlinewidth{0.803000pt}%
\definecolor{currentstroke}{rgb}{0.000000,0.000000,0.000000}%
\pgfsetstrokecolor{currentstroke}%
\pgfsetdash{}{0pt}%
\pgfsys@defobject{currentmarker}{\pgfqpoint{-0.048611in}{0.000000in}}{\pgfqpoint{-0.000000in}{0.000000in}}{%
\pgfpathmoveto{\pgfqpoint{-0.000000in}{0.000000in}}%
\pgfpathlineto{\pgfqpoint{-0.048611in}{0.000000in}}%
\pgfusepath{stroke,fill}%
}%
\begin{pgfscope}%
\pgfsys@transformshift{0.640323in}{9.891629in}%
\pgfsys@useobject{currentmarker}{}%
\end{pgfscope}%
\end{pgfscope}%
\begin{pgfscope}%
\definecolor{textcolor}{rgb}{0.000000,0.000000,0.000000}%
\pgfsetstrokecolor{textcolor}%
\pgfsetfillcolor{textcolor}%
\pgftext[x=0.322221in, y=9.838868in, left, base]{\color{textcolor}\sffamily\fontsize{10.000000}{12.000000}\selectfont 0.0}%
\end{pgfscope}%
\begin{pgfscope}%
\pgfpathrectangle{\pgfqpoint{0.640323in}{9.767436in}}{\pgfqpoint{9.687500in}{3.850000in}}%
\pgfusepath{clip}%
\pgfsetrectcap%
\pgfsetroundjoin%
\pgfsetlinewidth{0.803000pt}%
\definecolor{currentstroke}{rgb}{0.690196,0.690196,0.690196}%
\pgfsetstrokecolor{currentstroke}%
\pgfsetdash{}{0pt}%
\pgfpathmoveto{\pgfqpoint{0.640323in}{10.512597in}}%
\pgfpathlineto{\pgfqpoint{10.327822in}{10.512597in}}%
\pgfusepath{stroke}%
\end{pgfscope}%
\begin{pgfscope}%
\pgfsetbuttcap%
\pgfsetroundjoin%
\definecolor{currentfill}{rgb}{0.000000,0.000000,0.000000}%
\pgfsetfillcolor{currentfill}%
\pgfsetlinewidth{0.803000pt}%
\definecolor{currentstroke}{rgb}{0.000000,0.000000,0.000000}%
\pgfsetstrokecolor{currentstroke}%
\pgfsetdash{}{0pt}%
\pgfsys@defobject{currentmarker}{\pgfqpoint{-0.048611in}{0.000000in}}{\pgfqpoint{-0.000000in}{0.000000in}}{%
\pgfpathmoveto{\pgfqpoint{-0.000000in}{0.000000in}}%
\pgfpathlineto{\pgfqpoint{-0.048611in}{0.000000in}}%
\pgfusepath{stroke,fill}%
}%
\begin{pgfscope}%
\pgfsys@transformshift{0.640323in}{10.512597in}%
\pgfsys@useobject{currentmarker}{}%
\end{pgfscope}%
\end{pgfscope}%
\begin{pgfscope}%
\definecolor{textcolor}{rgb}{0.000000,0.000000,0.000000}%
\pgfsetstrokecolor{textcolor}%
\pgfsetfillcolor{textcolor}%
\pgftext[x=0.322221in, y=10.459836in, left, base]{\color{textcolor}\sffamily\fontsize{10.000000}{12.000000}\selectfont 0.1}%
\end{pgfscope}%
\begin{pgfscope}%
\pgfpathrectangle{\pgfqpoint{0.640323in}{9.767436in}}{\pgfqpoint{9.687500in}{3.850000in}}%
\pgfusepath{clip}%
\pgfsetrectcap%
\pgfsetroundjoin%
\pgfsetlinewidth{0.803000pt}%
\definecolor{currentstroke}{rgb}{0.690196,0.690196,0.690196}%
\pgfsetstrokecolor{currentstroke}%
\pgfsetdash{}{0pt}%
\pgfpathmoveto{\pgfqpoint{0.640323in}{11.133565in}}%
\pgfpathlineto{\pgfqpoint{10.327822in}{11.133565in}}%
\pgfusepath{stroke}%
\end{pgfscope}%
\begin{pgfscope}%
\pgfsetbuttcap%
\pgfsetroundjoin%
\definecolor{currentfill}{rgb}{0.000000,0.000000,0.000000}%
\pgfsetfillcolor{currentfill}%
\pgfsetlinewidth{0.803000pt}%
\definecolor{currentstroke}{rgb}{0.000000,0.000000,0.000000}%
\pgfsetstrokecolor{currentstroke}%
\pgfsetdash{}{0pt}%
\pgfsys@defobject{currentmarker}{\pgfqpoint{-0.048611in}{0.000000in}}{\pgfqpoint{-0.000000in}{0.000000in}}{%
\pgfpathmoveto{\pgfqpoint{-0.000000in}{0.000000in}}%
\pgfpathlineto{\pgfqpoint{-0.048611in}{0.000000in}}%
\pgfusepath{stroke,fill}%
}%
\begin{pgfscope}%
\pgfsys@transformshift{0.640323in}{11.133565in}%
\pgfsys@useobject{currentmarker}{}%
\end{pgfscope}%
\end{pgfscope}%
\begin{pgfscope}%
\definecolor{textcolor}{rgb}{0.000000,0.000000,0.000000}%
\pgfsetstrokecolor{textcolor}%
\pgfsetfillcolor{textcolor}%
\pgftext[x=0.322221in, y=11.080803in, left, base]{\color{textcolor}\sffamily\fontsize{10.000000}{12.000000}\selectfont 0.2}%
\end{pgfscope}%
\begin{pgfscope}%
\pgfpathrectangle{\pgfqpoint{0.640323in}{9.767436in}}{\pgfqpoint{9.687500in}{3.850000in}}%
\pgfusepath{clip}%
\pgfsetrectcap%
\pgfsetroundjoin%
\pgfsetlinewidth{0.803000pt}%
\definecolor{currentstroke}{rgb}{0.690196,0.690196,0.690196}%
\pgfsetstrokecolor{currentstroke}%
\pgfsetdash{}{0pt}%
\pgfpathmoveto{\pgfqpoint{0.640323in}{11.754533in}}%
\pgfpathlineto{\pgfqpoint{10.327822in}{11.754533in}}%
\pgfusepath{stroke}%
\end{pgfscope}%
\begin{pgfscope}%
\pgfsetbuttcap%
\pgfsetroundjoin%
\definecolor{currentfill}{rgb}{0.000000,0.000000,0.000000}%
\pgfsetfillcolor{currentfill}%
\pgfsetlinewidth{0.803000pt}%
\definecolor{currentstroke}{rgb}{0.000000,0.000000,0.000000}%
\pgfsetstrokecolor{currentstroke}%
\pgfsetdash{}{0pt}%
\pgfsys@defobject{currentmarker}{\pgfqpoint{-0.048611in}{0.000000in}}{\pgfqpoint{-0.000000in}{0.000000in}}{%
\pgfpathmoveto{\pgfqpoint{-0.000000in}{0.000000in}}%
\pgfpathlineto{\pgfqpoint{-0.048611in}{0.000000in}}%
\pgfusepath{stroke,fill}%
}%
\begin{pgfscope}%
\pgfsys@transformshift{0.640323in}{11.754533in}%
\pgfsys@useobject{currentmarker}{}%
\end{pgfscope}%
\end{pgfscope}%
\begin{pgfscope}%
\definecolor{textcolor}{rgb}{0.000000,0.000000,0.000000}%
\pgfsetstrokecolor{textcolor}%
\pgfsetfillcolor{textcolor}%
\pgftext[x=0.322221in, y=11.701771in, left, base]{\color{textcolor}\sffamily\fontsize{10.000000}{12.000000}\selectfont 0.3}%
\end{pgfscope}%
\begin{pgfscope}%
\pgfpathrectangle{\pgfqpoint{0.640323in}{9.767436in}}{\pgfqpoint{9.687500in}{3.850000in}}%
\pgfusepath{clip}%
\pgfsetrectcap%
\pgfsetroundjoin%
\pgfsetlinewidth{0.803000pt}%
\definecolor{currentstroke}{rgb}{0.690196,0.690196,0.690196}%
\pgfsetstrokecolor{currentstroke}%
\pgfsetdash{}{0pt}%
\pgfpathmoveto{\pgfqpoint{0.640323in}{12.375500in}}%
\pgfpathlineto{\pgfqpoint{10.327822in}{12.375500in}}%
\pgfusepath{stroke}%
\end{pgfscope}%
\begin{pgfscope}%
\pgfsetbuttcap%
\pgfsetroundjoin%
\definecolor{currentfill}{rgb}{0.000000,0.000000,0.000000}%
\pgfsetfillcolor{currentfill}%
\pgfsetlinewidth{0.803000pt}%
\definecolor{currentstroke}{rgb}{0.000000,0.000000,0.000000}%
\pgfsetstrokecolor{currentstroke}%
\pgfsetdash{}{0pt}%
\pgfsys@defobject{currentmarker}{\pgfqpoint{-0.048611in}{0.000000in}}{\pgfqpoint{-0.000000in}{0.000000in}}{%
\pgfpathmoveto{\pgfqpoint{-0.000000in}{0.000000in}}%
\pgfpathlineto{\pgfqpoint{-0.048611in}{0.000000in}}%
\pgfusepath{stroke,fill}%
}%
\begin{pgfscope}%
\pgfsys@transformshift{0.640323in}{12.375500in}%
\pgfsys@useobject{currentmarker}{}%
\end{pgfscope}%
\end{pgfscope}%
\begin{pgfscope}%
\definecolor{textcolor}{rgb}{0.000000,0.000000,0.000000}%
\pgfsetstrokecolor{textcolor}%
\pgfsetfillcolor{textcolor}%
\pgftext[x=0.322221in, y=12.322739in, left, base]{\color{textcolor}\sffamily\fontsize{10.000000}{12.000000}\selectfont 0.4}%
\end{pgfscope}%
\begin{pgfscope}%
\pgfpathrectangle{\pgfqpoint{0.640323in}{9.767436in}}{\pgfqpoint{9.687500in}{3.850000in}}%
\pgfusepath{clip}%
\pgfsetrectcap%
\pgfsetroundjoin%
\pgfsetlinewidth{0.803000pt}%
\definecolor{currentstroke}{rgb}{0.690196,0.690196,0.690196}%
\pgfsetstrokecolor{currentstroke}%
\pgfsetdash{}{0pt}%
\pgfpathmoveto{\pgfqpoint{0.640323in}{12.996468in}}%
\pgfpathlineto{\pgfqpoint{10.327822in}{12.996468in}}%
\pgfusepath{stroke}%
\end{pgfscope}%
\begin{pgfscope}%
\pgfsetbuttcap%
\pgfsetroundjoin%
\definecolor{currentfill}{rgb}{0.000000,0.000000,0.000000}%
\pgfsetfillcolor{currentfill}%
\pgfsetlinewidth{0.803000pt}%
\definecolor{currentstroke}{rgb}{0.000000,0.000000,0.000000}%
\pgfsetstrokecolor{currentstroke}%
\pgfsetdash{}{0pt}%
\pgfsys@defobject{currentmarker}{\pgfqpoint{-0.048611in}{0.000000in}}{\pgfqpoint{-0.000000in}{0.000000in}}{%
\pgfpathmoveto{\pgfqpoint{-0.000000in}{0.000000in}}%
\pgfpathlineto{\pgfqpoint{-0.048611in}{0.000000in}}%
\pgfusepath{stroke,fill}%
}%
\begin{pgfscope}%
\pgfsys@transformshift{0.640323in}{12.996468in}%
\pgfsys@useobject{currentmarker}{}%
\end{pgfscope}%
\end{pgfscope}%
\begin{pgfscope}%
\definecolor{textcolor}{rgb}{0.000000,0.000000,0.000000}%
\pgfsetstrokecolor{textcolor}%
\pgfsetfillcolor{textcolor}%
\pgftext[x=0.322221in, y=12.943707in, left, base]{\color{textcolor}\sffamily\fontsize{10.000000}{12.000000}\selectfont 0.5}%
\end{pgfscope}%
\begin{pgfscope}%
\pgfpathrectangle{\pgfqpoint{0.640323in}{9.767436in}}{\pgfqpoint{9.687500in}{3.850000in}}%
\pgfusepath{clip}%
\pgfsetrectcap%
\pgfsetroundjoin%
\pgfsetlinewidth{0.803000pt}%
\definecolor{currentstroke}{rgb}{0.690196,0.690196,0.690196}%
\pgfsetstrokecolor{currentstroke}%
\pgfsetdash{}{0pt}%
\pgfpathmoveto{\pgfqpoint{0.640323in}{13.617436in}}%
\pgfpathlineto{\pgfqpoint{10.327822in}{13.617436in}}%
\pgfusepath{stroke}%
\end{pgfscope}%
\begin{pgfscope}%
\pgfsetbuttcap%
\pgfsetroundjoin%
\definecolor{currentfill}{rgb}{0.000000,0.000000,0.000000}%
\pgfsetfillcolor{currentfill}%
\pgfsetlinewidth{0.803000pt}%
\definecolor{currentstroke}{rgb}{0.000000,0.000000,0.000000}%
\pgfsetstrokecolor{currentstroke}%
\pgfsetdash{}{0pt}%
\pgfsys@defobject{currentmarker}{\pgfqpoint{-0.048611in}{0.000000in}}{\pgfqpoint{-0.000000in}{0.000000in}}{%
\pgfpathmoveto{\pgfqpoint{-0.000000in}{0.000000in}}%
\pgfpathlineto{\pgfqpoint{-0.048611in}{0.000000in}}%
\pgfusepath{stroke,fill}%
}%
\begin{pgfscope}%
\pgfsys@transformshift{0.640323in}{13.617436in}%
\pgfsys@useobject{currentmarker}{}%
\end{pgfscope}%
\end{pgfscope}%
\begin{pgfscope}%
\definecolor{textcolor}{rgb}{0.000000,0.000000,0.000000}%
\pgfsetstrokecolor{textcolor}%
\pgfsetfillcolor{textcolor}%
\pgftext[x=0.322221in, y=13.564674in, left, base]{\color{textcolor}\sffamily\fontsize{10.000000}{12.000000}\selectfont 0.6}%
\end{pgfscope}%
\begin{pgfscope}%
\pgfpathrectangle{\pgfqpoint{0.640323in}{9.767436in}}{\pgfqpoint{9.687500in}{3.850000in}}%
\pgfusepath{clip}%
\pgfsetrectcap%
\pgfsetroundjoin%
\pgfsetlinewidth{0.803000pt}%
\definecolor{currentstroke}{rgb}{0.600000,0.600000,0.600000}%
\pgfsetstrokecolor{currentstroke}%
\pgfsetstrokeopacity{0.200000}%
\pgfsetdash{}{0pt}%
\pgfpathmoveto{\pgfqpoint{0.640323in}{10.015823in}}%
\pgfpathlineto{\pgfqpoint{10.327822in}{10.015823in}}%
\pgfusepath{stroke}%
\end{pgfscope}%
\begin{pgfscope}%
\pgfsetbuttcap%
\pgfsetroundjoin%
\definecolor{currentfill}{rgb}{0.000000,0.000000,0.000000}%
\pgfsetfillcolor{currentfill}%
\pgfsetlinewidth{0.602250pt}%
\definecolor{currentstroke}{rgb}{0.000000,0.000000,0.000000}%
\pgfsetstrokecolor{currentstroke}%
\pgfsetdash{}{0pt}%
\pgfsys@defobject{currentmarker}{\pgfqpoint{-0.027778in}{0.000000in}}{\pgfqpoint{-0.000000in}{0.000000in}}{%
\pgfpathmoveto{\pgfqpoint{-0.000000in}{0.000000in}}%
\pgfpathlineto{\pgfqpoint{-0.027778in}{0.000000in}}%
\pgfusepath{stroke,fill}%
}%
\begin{pgfscope}%
\pgfsys@transformshift{0.640323in}{10.015823in}%
\pgfsys@useobject{currentmarker}{}%
\end{pgfscope}%
\end{pgfscope}%
\begin{pgfscope}%
\pgfpathrectangle{\pgfqpoint{0.640323in}{9.767436in}}{\pgfqpoint{9.687500in}{3.850000in}}%
\pgfusepath{clip}%
\pgfsetrectcap%
\pgfsetroundjoin%
\pgfsetlinewidth{0.803000pt}%
\definecolor{currentstroke}{rgb}{0.600000,0.600000,0.600000}%
\pgfsetstrokecolor{currentstroke}%
\pgfsetstrokeopacity{0.200000}%
\pgfsetdash{}{0pt}%
\pgfpathmoveto{\pgfqpoint{0.640323in}{10.140016in}}%
\pgfpathlineto{\pgfqpoint{10.327822in}{10.140016in}}%
\pgfusepath{stroke}%
\end{pgfscope}%
\begin{pgfscope}%
\pgfsetbuttcap%
\pgfsetroundjoin%
\definecolor{currentfill}{rgb}{0.000000,0.000000,0.000000}%
\pgfsetfillcolor{currentfill}%
\pgfsetlinewidth{0.602250pt}%
\definecolor{currentstroke}{rgb}{0.000000,0.000000,0.000000}%
\pgfsetstrokecolor{currentstroke}%
\pgfsetdash{}{0pt}%
\pgfsys@defobject{currentmarker}{\pgfqpoint{-0.027778in}{0.000000in}}{\pgfqpoint{-0.000000in}{0.000000in}}{%
\pgfpathmoveto{\pgfqpoint{-0.000000in}{0.000000in}}%
\pgfpathlineto{\pgfqpoint{-0.027778in}{0.000000in}}%
\pgfusepath{stroke,fill}%
}%
\begin{pgfscope}%
\pgfsys@transformshift{0.640323in}{10.140016in}%
\pgfsys@useobject{currentmarker}{}%
\end{pgfscope}%
\end{pgfscope}%
\begin{pgfscope}%
\pgfpathrectangle{\pgfqpoint{0.640323in}{9.767436in}}{\pgfqpoint{9.687500in}{3.850000in}}%
\pgfusepath{clip}%
\pgfsetrectcap%
\pgfsetroundjoin%
\pgfsetlinewidth{0.803000pt}%
\definecolor{currentstroke}{rgb}{0.600000,0.600000,0.600000}%
\pgfsetstrokecolor{currentstroke}%
\pgfsetstrokeopacity{0.200000}%
\pgfsetdash{}{0pt}%
\pgfpathmoveto{\pgfqpoint{0.640323in}{10.264210in}}%
\pgfpathlineto{\pgfqpoint{10.327822in}{10.264210in}}%
\pgfusepath{stroke}%
\end{pgfscope}%
\begin{pgfscope}%
\pgfsetbuttcap%
\pgfsetroundjoin%
\definecolor{currentfill}{rgb}{0.000000,0.000000,0.000000}%
\pgfsetfillcolor{currentfill}%
\pgfsetlinewidth{0.602250pt}%
\definecolor{currentstroke}{rgb}{0.000000,0.000000,0.000000}%
\pgfsetstrokecolor{currentstroke}%
\pgfsetdash{}{0pt}%
\pgfsys@defobject{currentmarker}{\pgfqpoint{-0.027778in}{0.000000in}}{\pgfqpoint{-0.000000in}{0.000000in}}{%
\pgfpathmoveto{\pgfqpoint{-0.000000in}{0.000000in}}%
\pgfpathlineto{\pgfqpoint{-0.027778in}{0.000000in}}%
\pgfusepath{stroke,fill}%
}%
\begin{pgfscope}%
\pgfsys@transformshift{0.640323in}{10.264210in}%
\pgfsys@useobject{currentmarker}{}%
\end{pgfscope}%
\end{pgfscope}%
\begin{pgfscope}%
\pgfpathrectangle{\pgfqpoint{0.640323in}{9.767436in}}{\pgfqpoint{9.687500in}{3.850000in}}%
\pgfusepath{clip}%
\pgfsetrectcap%
\pgfsetroundjoin%
\pgfsetlinewidth{0.803000pt}%
\definecolor{currentstroke}{rgb}{0.600000,0.600000,0.600000}%
\pgfsetstrokecolor{currentstroke}%
\pgfsetstrokeopacity{0.200000}%
\pgfsetdash{}{0pt}%
\pgfpathmoveto{\pgfqpoint{0.640323in}{10.388404in}}%
\pgfpathlineto{\pgfqpoint{10.327822in}{10.388404in}}%
\pgfusepath{stroke}%
\end{pgfscope}%
\begin{pgfscope}%
\pgfsetbuttcap%
\pgfsetroundjoin%
\definecolor{currentfill}{rgb}{0.000000,0.000000,0.000000}%
\pgfsetfillcolor{currentfill}%
\pgfsetlinewidth{0.602250pt}%
\definecolor{currentstroke}{rgb}{0.000000,0.000000,0.000000}%
\pgfsetstrokecolor{currentstroke}%
\pgfsetdash{}{0pt}%
\pgfsys@defobject{currentmarker}{\pgfqpoint{-0.027778in}{0.000000in}}{\pgfqpoint{-0.000000in}{0.000000in}}{%
\pgfpathmoveto{\pgfqpoint{-0.000000in}{0.000000in}}%
\pgfpathlineto{\pgfqpoint{-0.027778in}{0.000000in}}%
\pgfusepath{stroke,fill}%
}%
\begin{pgfscope}%
\pgfsys@transformshift{0.640323in}{10.388404in}%
\pgfsys@useobject{currentmarker}{}%
\end{pgfscope}%
\end{pgfscope}%
\begin{pgfscope}%
\pgfpathrectangle{\pgfqpoint{0.640323in}{9.767436in}}{\pgfqpoint{9.687500in}{3.850000in}}%
\pgfusepath{clip}%
\pgfsetrectcap%
\pgfsetroundjoin%
\pgfsetlinewidth{0.803000pt}%
\definecolor{currentstroke}{rgb}{0.600000,0.600000,0.600000}%
\pgfsetstrokecolor{currentstroke}%
\pgfsetstrokeopacity{0.200000}%
\pgfsetdash{}{0pt}%
\pgfpathmoveto{\pgfqpoint{0.640323in}{10.636791in}}%
\pgfpathlineto{\pgfqpoint{10.327822in}{10.636791in}}%
\pgfusepath{stroke}%
\end{pgfscope}%
\begin{pgfscope}%
\pgfsetbuttcap%
\pgfsetroundjoin%
\definecolor{currentfill}{rgb}{0.000000,0.000000,0.000000}%
\pgfsetfillcolor{currentfill}%
\pgfsetlinewidth{0.602250pt}%
\definecolor{currentstroke}{rgb}{0.000000,0.000000,0.000000}%
\pgfsetstrokecolor{currentstroke}%
\pgfsetdash{}{0pt}%
\pgfsys@defobject{currentmarker}{\pgfqpoint{-0.027778in}{0.000000in}}{\pgfqpoint{-0.000000in}{0.000000in}}{%
\pgfpathmoveto{\pgfqpoint{-0.000000in}{0.000000in}}%
\pgfpathlineto{\pgfqpoint{-0.027778in}{0.000000in}}%
\pgfusepath{stroke,fill}%
}%
\begin{pgfscope}%
\pgfsys@transformshift{0.640323in}{10.636791in}%
\pgfsys@useobject{currentmarker}{}%
\end{pgfscope}%
\end{pgfscope}%
\begin{pgfscope}%
\pgfpathrectangle{\pgfqpoint{0.640323in}{9.767436in}}{\pgfqpoint{9.687500in}{3.850000in}}%
\pgfusepath{clip}%
\pgfsetrectcap%
\pgfsetroundjoin%
\pgfsetlinewidth{0.803000pt}%
\definecolor{currentstroke}{rgb}{0.600000,0.600000,0.600000}%
\pgfsetstrokecolor{currentstroke}%
\pgfsetstrokeopacity{0.200000}%
\pgfsetdash{}{0pt}%
\pgfpathmoveto{\pgfqpoint{0.640323in}{10.760984in}}%
\pgfpathlineto{\pgfqpoint{10.327822in}{10.760984in}}%
\pgfusepath{stroke}%
\end{pgfscope}%
\begin{pgfscope}%
\pgfsetbuttcap%
\pgfsetroundjoin%
\definecolor{currentfill}{rgb}{0.000000,0.000000,0.000000}%
\pgfsetfillcolor{currentfill}%
\pgfsetlinewidth{0.602250pt}%
\definecolor{currentstroke}{rgb}{0.000000,0.000000,0.000000}%
\pgfsetstrokecolor{currentstroke}%
\pgfsetdash{}{0pt}%
\pgfsys@defobject{currentmarker}{\pgfqpoint{-0.027778in}{0.000000in}}{\pgfqpoint{-0.000000in}{0.000000in}}{%
\pgfpathmoveto{\pgfqpoint{-0.000000in}{0.000000in}}%
\pgfpathlineto{\pgfqpoint{-0.027778in}{0.000000in}}%
\pgfusepath{stroke,fill}%
}%
\begin{pgfscope}%
\pgfsys@transformshift{0.640323in}{10.760984in}%
\pgfsys@useobject{currentmarker}{}%
\end{pgfscope}%
\end{pgfscope}%
\begin{pgfscope}%
\pgfpathrectangle{\pgfqpoint{0.640323in}{9.767436in}}{\pgfqpoint{9.687500in}{3.850000in}}%
\pgfusepath{clip}%
\pgfsetrectcap%
\pgfsetroundjoin%
\pgfsetlinewidth{0.803000pt}%
\definecolor{currentstroke}{rgb}{0.600000,0.600000,0.600000}%
\pgfsetstrokecolor{currentstroke}%
\pgfsetstrokeopacity{0.200000}%
\pgfsetdash{}{0pt}%
\pgfpathmoveto{\pgfqpoint{0.640323in}{10.885178in}}%
\pgfpathlineto{\pgfqpoint{10.327822in}{10.885178in}}%
\pgfusepath{stroke}%
\end{pgfscope}%
\begin{pgfscope}%
\pgfsetbuttcap%
\pgfsetroundjoin%
\definecolor{currentfill}{rgb}{0.000000,0.000000,0.000000}%
\pgfsetfillcolor{currentfill}%
\pgfsetlinewidth{0.602250pt}%
\definecolor{currentstroke}{rgb}{0.000000,0.000000,0.000000}%
\pgfsetstrokecolor{currentstroke}%
\pgfsetdash{}{0pt}%
\pgfsys@defobject{currentmarker}{\pgfqpoint{-0.027778in}{0.000000in}}{\pgfqpoint{-0.000000in}{0.000000in}}{%
\pgfpathmoveto{\pgfqpoint{-0.000000in}{0.000000in}}%
\pgfpathlineto{\pgfqpoint{-0.027778in}{0.000000in}}%
\pgfusepath{stroke,fill}%
}%
\begin{pgfscope}%
\pgfsys@transformshift{0.640323in}{10.885178in}%
\pgfsys@useobject{currentmarker}{}%
\end{pgfscope}%
\end{pgfscope}%
\begin{pgfscope}%
\pgfpathrectangle{\pgfqpoint{0.640323in}{9.767436in}}{\pgfqpoint{9.687500in}{3.850000in}}%
\pgfusepath{clip}%
\pgfsetrectcap%
\pgfsetroundjoin%
\pgfsetlinewidth{0.803000pt}%
\definecolor{currentstroke}{rgb}{0.600000,0.600000,0.600000}%
\pgfsetstrokecolor{currentstroke}%
\pgfsetstrokeopacity{0.200000}%
\pgfsetdash{}{0pt}%
\pgfpathmoveto{\pgfqpoint{0.640323in}{11.009371in}}%
\pgfpathlineto{\pgfqpoint{10.327822in}{11.009371in}}%
\pgfusepath{stroke}%
\end{pgfscope}%
\begin{pgfscope}%
\pgfsetbuttcap%
\pgfsetroundjoin%
\definecolor{currentfill}{rgb}{0.000000,0.000000,0.000000}%
\pgfsetfillcolor{currentfill}%
\pgfsetlinewidth{0.602250pt}%
\definecolor{currentstroke}{rgb}{0.000000,0.000000,0.000000}%
\pgfsetstrokecolor{currentstroke}%
\pgfsetdash{}{0pt}%
\pgfsys@defobject{currentmarker}{\pgfqpoint{-0.027778in}{0.000000in}}{\pgfqpoint{-0.000000in}{0.000000in}}{%
\pgfpathmoveto{\pgfqpoint{-0.000000in}{0.000000in}}%
\pgfpathlineto{\pgfqpoint{-0.027778in}{0.000000in}}%
\pgfusepath{stroke,fill}%
}%
\begin{pgfscope}%
\pgfsys@transformshift{0.640323in}{11.009371in}%
\pgfsys@useobject{currentmarker}{}%
\end{pgfscope}%
\end{pgfscope}%
\begin{pgfscope}%
\pgfpathrectangle{\pgfqpoint{0.640323in}{9.767436in}}{\pgfqpoint{9.687500in}{3.850000in}}%
\pgfusepath{clip}%
\pgfsetrectcap%
\pgfsetroundjoin%
\pgfsetlinewidth{0.803000pt}%
\definecolor{currentstroke}{rgb}{0.600000,0.600000,0.600000}%
\pgfsetstrokecolor{currentstroke}%
\pgfsetstrokeopacity{0.200000}%
\pgfsetdash{}{0pt}%
\pgfpathmoveto{\pgfqpoint{0.640323in}{11.257758in}}%
\pgfpathlineto{\pgfqpoint{10.327822in}{11.257758in}}%
\pgfusepath{stroke}%
\end{pgfscope}%
\begin{pgfscope}%
\pgfsetbuttcap%
\pgfsetroundjoin%
\definecolor{currentfill}{rgb}{0.000000,0.000000,0.000000}%
\pgfsetfillcolor{currentfill}%
\pgfsetlinewidth{0.602250pt}%
\definecolor{currentstroke}{rgb}{0.000000,0.000000,0.000000}%
\pgfsetstrokecolor{currentstroke}%
\pgfsetdash{}{0pt}%
\pgfsys@defobject{currentmarker}{\pgfqpoint{-0.027778in}{0.000000in}}{\pgfqpoint{-0.000000in}{0.000000in}}{%
\pgfpathmoveto{\pgfqpoint{-0.000000in}{0.000000in}}%
\pgfpathlineto{\pgfqpoint{-0.027778in}{0.000000in}}%
\pgfusepath{stroke,fill}%
}%
\begin{pgfscope}%
\pgfsys@transformshift{0.640323in}{11.257758in}%
\pgfsys@useobject{currentmarker}{}%
\end{pgfscope}%
\end{pgfscope}%
\begin{pgfscope}%
\pgfpathrectangle{\pgfqpoint{0.640323in}{9.767436in}}{\pgfqpoint{9.687500in}{3.850000in}}%
\pgfusepath{clip}%
\pgfsetrectcap%
\pgfsetroundjoin%
\pgfsetlinewidth{0.803000pt}%
\definecolor{currentstroke}{rgb}{0.600000,0.600000,0.600000}%
\pgfsetstrokecolor{currentstroke}%
\pgfsetstrokeopacity{0.200000}%
\pgfsetdash{}{0pt}%
\pgfpathmoveto{\pgfqpoint{0.640323in}{11.381952in}}%
\pgfpathlineto{\pgfqpoint{10.327822in}{11.381952in}}%
\pgfusepath{stroke}%
\end{pgfscope}%
\begin{pgfscope}%
\pgfsetbuttcap%
\pgfsetroundjoin%
\definecolor{currentfill}{rgb}{0.000000,0.000000,0.000000}%
\pgfsetfillcolor{currentfill}%
\pgfsetlinewidth{0.602250pt}%
\definecolor{currentstroke}{rgb}{0.000000,0.000000,0.000000}%
\pgfsetstrokecolor{currentstroke}%
\pgfsetdash{}{0pt}%
\pgfsys@defobject{currentmarker}{\pgfqpoint{-0.027778in}{0.000000in}}{\pgfqpoint{-0.000000in}{0.000000in}}{%
\pgfpathmoveto{\pgfqpoint{-0.000000in}{0.000000in}}%
\pgfpathlineto{\pgfqpoint{-0.027778in}{0.000000in}}%
\pgfusepath{stroke,fill}%
}%
\begin{pgfscope}%
\pgfsys@transformshift{0.640323in}{11.381952in}%
\pgfsys@useobject{currentmarker}{}%
\end{pgfscope}%
\end{pgfscope}%
\begin{pgfscope}%
\pgfpathrectangle{\pgfqpoint{0.640323in}{9.767436in}}{\pgfqpoint{9.687500in}{3.850000in}}%
\pgfusepath{clip}%
\pgfsetrectcap%
\pgfsetroundjoin%
\pgfsetlinewidth{0.803000pt}%
\definecolor{currentstroke}{rgb}{0.600000,0.600000,0.600000}%
\pgfsetstrokecolor{currentstroke}%
\pgfsetstrokeopacity{0.200000}%
\pgfsetdash{}{0pt}%
\pgfpathmoveto{\pgfqpoint{0.640323in}{11.506146in}}%
\pgfpathlineto{\pgfqpoint{10.327822in}{11.506146in}}%
\pgfusepath{stroke}%
\end{pgfscope}%
\begin{pgfscope}%
\pgfsetbuttcap%
\pgfsetroundjoin%
\definecolor{currentfill}{rgb}{0.000000,0.000000,0.000000}%
\pgfsetfillcolor{currentfill}%
\pgfsetlinewidth{0.602250pt}%
\definecolor{currentstroke}{rgb}{0.000000,0.000000,0.000000}%
\pgfsetstrokecolor{currentstroke}%
\pgfsetdash{}{0pt}%
\pgfsys@defobject{currentmarker}{\pgfqpoint{-0.027778in}{0.000000in}}{\pgfqpoint{-0.000000in}{0.000000in}}{%
\pgfpathmoveto{\pgfqpoint{-0.000000in}{0.000000in}}%
\pgfpathlineto{\pgfqpoint{-0.027778in}{0.000000in}}%
\pgfusepath{stroke,fill}%
}%
\begin{pgfscope}%
\pgfsys@transformshift{0.640323in}{11.506146in}%
\pgfsys@useobject{currentmarker}{}%
\end{pgfscope}%
\end{pgfscope}%
\begin{pgfscope}%
\pgfpathrectangle{\pgfqpoint{0.640323in}{9.767436in}}{\pgfqpoint{9.687500in}{3.850000in}}%
\pgfusepath{clip}%
\pgfsetrectcap%
\pgfsetroundjoin%
\pgfsetlinewidth{0.803000pt}%
\definecolor{currentstroke}{rgb}{0.600000,0.600000,0.600000}%
\pgfsetstrokecolor{currentstroke}%
\pgfsetstrokeopacity{0.200000}%
\pgfsetdash{}{0pt}%
\pgfpathmoveto{\pgfqpoint{0.640323in}{11.630339in}}%
\pgfpathlineto{\pgfqpoint{10.327822in}{11.630339in}}%
\pgfusepath{stroke}%
\end{pgfscope}%
\begin{pgfscope}%
\pgfsetbuttcap%
\pgfsetroundjoin%
\definecolor{currentfill}{rgb}{0.000000,0.000000,0.000000}%
\pgfsetfillcolor{currentfill}%
\pgfsetlinewidth{0.602250pt}%
\definecolor{currentstroke}{rgb}{0.000000,0.000000,0.000000}%
\pgfsetstrokecolor{currentstroke}%
\pgfsetdash{}{0pt}%
\pgfsys@defobject{currentmarker}{\pgfqpoint{-0.027778in}{0.000000in}}{\pgfqpoint{-0.000000in}{0.000000in}}{%
\pgfpathmoveto{\pgfqpoint{-0.000000in}{0.000000in}}%
\pgfpathlineto{\pgfqpoint{-0.027778in}{0.000000in}}%
\pgfusepath{stroke,fill}%
}%
\begin{pgfscope}%
\pgfsys@transformshift{0.640323in}{11.630339in}%
\pgfsys@useobject{currentmarker}{}%
\end{pgfscope}%
\end{pgfscope}%
\begin{pgfscope}%
\pgfpathrectangle{\pgfqpoint{0.640323in}{9.767436in}}{\pgfqpoint{9.687500in}{3.850000in}}%
\pgfusepath{clip}%
\pgfsetrectcap%
\pgfsetroundjoin%
\pgfsetlinewidth{0.803000pt}%
\definecolor{currentstroke}{rgb}{0.600000,0.600000,0.600000}%
\pgfsetstrokecolor{currentstroke}%
\pgfsetstrokeopacity{0.200000}%
\pgfsetdash{}{0pt}%
\pgfpathmoveto{\pgfqpoint{0.640323in}{11.878726in}}%
\pgfpathlineto{\pgfqpoint{10.327822in}{11.878726in}}%
\pgfusepath{stroke}%
\end{pgfscope}%
\begin{pgfscope}%
\pgfsetbuttcap%
\pgfsetroundjoin%
\definecolor{currentfill}{rgb}{0.000000,0.000000,0.000000}%
\pgfsetfillcolor{currentfill}%
\pgfsetlinewidth{0.602250pt}%
\definecolor{currentstroke}{rgb}{0.000000,0.000000,0.000000}%
\pgfsetstrokecolor{currentstroke}%
\pgfsetdash{}{0pt}%
\pgfsys@defobject{currentmarker}{\pgfqpoint{-0.027778in}{0.000000in}}{\pgfqpoint{-0.000000in}{0.000000in}}{%
\pgfpathmoveto{\pgfqpoint{-0.000000in}{0.000000in}}%
\pgfpathlineto{\pgfqpoint{-0.027778in}{0.000000in}}%
\pgfusepath{stroke,fill}%
}%
\begin{pgfscope}%
\pgfsys@transformshift{0.640323in}{11.878726in}%
\pgfsys@useobject{currentmarker}{}%
\end{pgfscope}%
\end{pgfscope}%
\begin{pgfscope}%
\pgfpathrectangle{\pgfqpoint{0.640323in}{9.767436in}}{\pgfqpoint{9.687500in}{3.850000in}}%
\pgfusepath{clip}%
\pgfsetrectcap%
\pgfsetroundjoin%
\pgfsetlinewidth{0.803000pt}%
\definecolor{currentstroke}{rgb}{0.600000,0.600000,0.600000}%
\pgfsetstrokecolor{currentstroke}%
\pgfsetstrokeopacity{0.200000}%
\pgfsetdash{}{0pt}%
\pgfpathmoveto{\pgfqpoint{0.640323in}{12.002920in}}%
\pgfpathlineto{\pgfqpoint{10.327822in}{12.002920in}}%
\pgfusepath{stroke}%
\end{pgfscope}%
\begin{pgfscope}%
\pgfsetbuttcap%
\pgfsetroundjoin%
\definecolor{currentfill}{rgb}{0.000000,0.000000,0.000000}%
\pgfsetfillcolor{currentfill}%
\pgfsetlinewidth{0.602250pt}%
\definecolor{currentstroke}{rgb}{0.000000,0.000000,0.000000}%
\pgfsetstrokecolor{currentstroke}%
\pgfsetdash{}{0pt}%
\pgfsys@defobject{currentmarker}{\pgfqpoint{-0.027778in}{0.000000in}}{\pgfqpoint{-0.000000in}{0.000000in}}{%
\pgfpathmoveto{\pgfqpoint{-0.000000in}{0.000000in}}%
\pgfpathlineto{\pgfqpoint{-0.027778in}{0.000000in}}%
\pgfusepath{stroke,fill}%
}%
\begin{pgfscope}%
\pgfsys@transformshift{0.640323in}{12.002920in}%
\pgfsys@useobject{currentmarker}{}%
\end{pgfscope}%
\end{pgfscope}%
\begin{pgfscope}%
\pgfpathrectangle{\pgfqpoint{0.640323in}{9.767436in}}{\pgfqpoint{9.687500in}{3.850000in}}%
\pgfusepath{clip}%
\pgfsetrectcap%
\pgfsetroundjoin%
\pgfsetlinewidth{0.803000pt}%
\definecolor{currentstroke}{rgb}{0.600000,0.600000,0.600000}%
\pgfsetstrokecolor{currentstroke}%
\pgfsetstrokeopacity{0.200000}%
\pgfsetdash{}{0pt}%
\pgfpathmoveto{\pgfqpoint{0.640323in}{12.127113in}}%
\pgfpathlineto{\pgfqpoint{10.327822in}{12.127113in}}%
\pgfusepath{stroke}%
\end{pgfscope}%
\begin{pgfscope}%
\pgfsetbuttcap%
\pgfsetroundjoin%
\definecolor{currentfill}{rgb}{0.000000,0.000000,0.000000}%
\pgfsetfillcolor{currentfill}%
\pgfsetlinewidth{0.602250pt}%
\definecolor{currentstroke}{rgb}{0.000000,0.000000,0.000000}%
\pgfsetstrokecolor{currentstroke}%
\pgfsetdash{}{0pt}%
\pgfsys@defobject{currentmarker}{\pgfqpoint{-0.027778in}{0.000000in}}{\pgfqpoint{-0.000000in}{0.000000in}}{%
\pgfpathmoveto{\pgfqpoint{-0.000000in}{0.000000in}}%
\pgfpathlineto{\pgfqpoint{-0.027778in}{0.000000in}}%
\pgfusepath{stroke,fill}%
}%
\begin{pgfscope}%
\pgfsys@transformshift{0.640323in}{12.127113in}%
\pgfsys@useobject{currentmarker}{}%
\end{pgfscope}%
\end{pgfscope}%
\begin{pgfscope}%
\pgfpathrectangle{\pgfqpoint{0.640323in}{9.767436in}}{\pgfqpoint{9.687500in}{3.850000in}}%
\pgfusepath{clip}%
\pgfsetrectcap%
\pgfsetroundjoin%
\pgfsetlinewidth{0.803000pt}%
\definecolor{currentstroke}{rgb}{0.600000,0.600000,0.600000}%
\pgfsetstrokecolor{currentstroke}%
\pgfsetstrokeopacity{0.200000}%
\pgfsetdash{}{0pt}%
\pgfpathmoveto{\pgfqpoint{0.640323in}{12.251307in}}%
\pgfpathlineto{\pgfqpoint{10.327822in}{12.251307in}}%
\pgfusepath{stroke}%
\end{pgfscope}%
\begin{pgfscope}%
\pgfsetbuttcap%
\pgfsetroundjoin%
\definecolor{currentfill}{rgb}{0.000000,0.000000,0.000000}%
\pgfsetfillcolor{currentfill}%
\pgfsetlinewidth{0.602250pt}%
\definecolor{currentstroke}{rgb}{0.000000,0.000000,0.000000}%
\pgfsetstrokecolor{currentstroke}%
\pgfsetdash{}{0pt}%
\pgfsys@defobject{currentmarker}{\pgfqpoint{-0.027778in}{0.000000in}}{\pgfqpoint{-0.000000in}{0.000000in}}{%
\pgfpathmoveto{\pgfqpoint{-0.000000in}{0.000000in}}%
\pgfpathlineto{\pgfqpoint{-0.027778in}{0.000000in}}%
\pgfusepath{stroke,fill}%
}%
\begin{pgfscope}%
\pgfsys@transformshift{0.640323in}{12.251307in}%
\pgfsys@useobject{currentmarker}{}%
\end{pgfscope}%
\end{pgfscope}%
\begin{pgfscope}%
\pgfpathrectangle{\pgfqpoint{0.640323in}{9.767436in}}{\pgfqpoint{9.687500in}{3.850000in}}%
\pgfusepath{clip}%
\pgfsetrectcap%
\pgfsetroundjoin%
\pgfsetlinewidth{0.803000pt}%
\definecolor{currentstroke}{rgb}{0.600000,0.600000,0.600000}%
\pgfsetstrokecolor{currentstroke}%
\pgfsetstrokeopacity{0.200000}%
\pgfsetdash{}{0pt}%
\pgfpathmoveto{\pgfqpoint{0.640323in}{12.499694in}}%
\pgfpathlineto{\pgfqpoint{10.327822in}{12.499694in}}%
\pgfusepath{stroke}%
\end{pgfscope}%
\begin{pgfscope}%
\pgfsetbuttcap%
\pgfsetroundjoin%
\definecolor{currentfill}{rgb}{0.000000,0.000000,0.000000}%
\pgfsetfillcolor{currentfill}%
\pgfsetlinewidth{0.602250pt}%
\definecolor{currentstroke}{rgb}{0.000000,0.000000,0.000000}%
\pgfsetstrokecolor{currentstroke}%
\pgfsetdash{}{0pt}%
\pgfsys@defobject{currentmarker}{\pgfqpoint{-0.027778in}{0.000000in}}{\pgfqpoint{-0.000000in}{0.000000in}}{%
\pgfpathmoveto{\pgfqpoint{-0.000000in}{0.000000in}}%
\pgfpathlineto{\pgfqpoint{-0.027778in}{0.000000in}}%
\pgfusepath{stroke,fill}%
}%
\begin{pgfscope}%
\pgfsys@transformshift{0.640323in}{12.499694in}%
\pgfsys@useobject{currentmarker}{}%
\end{pgfscope}%
\end{pgfscope}%
\begin{pgfscope}%
\pgfpathrectangle{\pgfqpoint{0.640323in}{9.767436in}}{\pgfqpoint{9.687500in}{3.850000in}}%
\pgfusepath{clip}%
\pgfsetrectcap%
\pgfsetroundjoin%
\pgfsetlinewidth{0.803000pt}%
\definecolor{currentstroke}{rgb}{0.600000,0.600000,0.600000}%
\pgfsetstrokecolor{currentstroke}%
\pgfsetstrokeopacity{0.200000}%
\pgfsetdash{}{0pt}%
\pgfpathmoveto{\pgfqpoint{0.640323in}{12.623887in}}%
\pgfpathlineto{\pgfqpoint{10.327822in}{12.623887in}}%
\pgfusepath{stroke}%
\end{pgfscope}%
\begin{pgfscope}%
\pgfsetbuttcap%
\pgfsetroundjoin%
\definecolor{currentfill}{rgb}{0.000000,0.000000,0.000000}%
\pgfsetfillcolor{currentfill}%
\pgfsetlinewidth{0.602250pt}%
\definecolor{currentstroke}{rgb}{0.000000,0.000000,0.000000}%
\pgfsetstrokecolor{currentstroke}%
\pgfsetdash{}{0pt}%
\pgfsys@defobject{currentmarker}{\pgfqpoint{-0.027778in}{0.000000in}}{\pgfqpoint{-0.000000in}{0.000000in}}{%
\pgfpathmoveto{\pgfqpoint{-0.000000in}{0.000000in}}%
\pgfpathlineto{\pgfqpoint{-0.027778in}{0.000000in}}%
\pgfusepath{stroke,fill}%
}%
\begin{pgfscope}%
\pgfsys@transformshift{0.640323in}{12.623887in}%
\pgfsys@useobject{currentmarker}{}%
\end{pgfscope}%
\end{pgfscope}%
\begin{pgfscope}%
\pgfpathrectangle{\pgfqpoint{0.640323in}{9.767436in}}{\pgfqpoint{9.687500in}{3.850000in}}%
\pgfusepath{clip}%
\pgfsetrectcap%
\pgfsetroundjoin%
\pgfsetlinewidth{0.803000pt}%
\definecolor{currentstroke}{rgb}{0.600000,0.600000,0.600000}%
\pgfsetstrokecolor{currentstroke}%
\pgfsetstrokeopacity{0.200000}%
\pgfsetdash{}{0pt}%
\pgfpathmoveto{\pgfqpoint{0.640323in}{12.748081in}}%
\pgfpathlineto{\pgfqpoint{10.327822in}{12.748081in}}%
\pgfusepath{stroke}%
\end{pgfscope}%
\begin{pgfscope}%
\pgfsetbuttcap%
\pgfsetroundjoin%
\definecolor{currentfill}{rgb}{0.000000,0.000000,0.000000}%
\pgfsetfillcolor{currentfill}%
\pgfsetlinewidth{0.602250pt}%
\definecolor{currentstroke}{rgb}{0.000000,0.000000,0.000000}%
\pgfsetstrokecolor{currentstroke}%
\pgfsetdash{}{0pt}%
\pgfsys@defobject{currentmarker}{\pgfqpoint{-0.027778in}{0.000000in}}{\pgfqpoint{-0.000000in}{0.000000in}}{%
\pgfpathmoveto{\pgfqpoint{-0.000000in}{0.000000in}}%
\pgfpathlineto{\pgfqpoint{-0.027778in}{0.000000in}}%
\pgfusepath{stroke,fill}%
}%
\begin{pgfscope}%
\pgfsys@transformshift{0.640323in}{12.748081in}%
\pgfsys@useobject{currentmarker}{}%
\end{pgfscope}%
\end{pgfscope}%
\begin{pgfscope}%
\pgfpathrectangle{\pgfqpoint{0.640323in}{9.767436in}}{\pgfqpoint{9.687500in}{3.850000in}}%
\pgfusepath{clip}%
\pgfsetrectcap%
\pgfsetroundjoin%
\pgfsetlinewidth{0.803000pt}%
\definecolor{currentstroke}{rgb}{0.600000,0.600000,0.600000}%
\pgfsetstrokecolor{currentstroke}%
\pgfsetstrokeopacity{0.200000}%
\pgfsetdash{}{0pt}%
\pgfpathmoveto{\pgfqpoint{0.640323in}{12.872275in}}%
\pgfpathlineto{\pgfqpoint{10.327822in}{12.872275in}}%
\pgfusepath{stroke}%
\end{pgfscope}%
\begin{pgfscope}%
\pgfsetbuttcap%
\pgfsetroundjoin%
\definecolor{currentfill}{rgb}{0.000000,0.000000,0.000000}%
\pgfsetfillcolor{currentfill}%
\pgfsetlinewidth{0.602250pt}%
\definecolor{currentstroke}{rgb}{0.000000,0.000000,0.000000}%
\pgfsetstrokecolor{currentstroke}%
\pgfsetdash{}{0pt}%
\pgfsys@defobject{currentmarker}{\pgfqpoint{-0.027778in}{0.000000in}}{\pgfqpoint{-0.000000in}{0.000000in}}{%
\pgfpathmoveto{\pgfqpoint{-0.000000in}{0.000000in}}%
\pgfpathlineto{\pgfqpoint{-0.027778in}{0.000000in}}%
\pgfusepath{stroke,fill}%
}%
\begin{pgfscope}%
\pgfsys@transformshift{0.640323in}{12.872275in}%
\pgfsys@useobject{currentmarker}{}%
\end{pgfscope}%
\end{pgfscope}%
\begin{pgfscope}%
\pgfpathrectangle{\pgfqpoint{0.640323in}{9.767436in}}{\pgfqpoint{9.687500in}{3.850000in}}%
\pgfusepath{clip}%
\pgfsetrectcap%
\pgfsetroundjoin%
\pgfsetlinewidth{0.803000pt}%
\definecolor{currentstroke}{rgb}{0.600000,0.600000,0.600000}%
\pgfsetstrokecolor{currentstroke}%
\pgfsetstrokeopacity{0.200000}%
\pgfsetdash{}{0pt}%
\pgfpathmoveto{\pgfqpoint{0.640323in}{13.120662in}}%
\pgfpathlineto{\pgfqpoint{10.327822in}{13.120662in}}%
\pgfusepath{stroke}%
\end{pgfscope}%
\begin{pgfscope}%
\pgfsetbuttcap%
\pgfsetroundjoin%
\definecolor{currentfill}{rgb}{0.000000,0.000000,0.000000}%
\pgfsetfillcolor{currentfill}%
\pgfsetlinewidth{0.602250pt}%
\definecolor{currentstroke}{rgb}{0.000000,0.000000,0.000000}%
\pgfsetstrokecolor{currentstroke}%
\pgfsetdash{}{0pt}%
\pgfsys@defobject{currentmarker}{\pgfqpoint{-0.027778in}{0.000000in}}{\pgfqpoint{-0.000000in}{0.000000in}}{%
\pgfpathmoveto{\pgfqpoint{-0.000000in}{0.000000in}}%
\pgfpathlineto{\pgfqpoint{-0.027778in}{0.000000in}}%
\pgfusepath{stroke,fill}%
}%
\begin{pgfscope}%
\pgfsys@transformshift{0.640323in}{13.120662in}%
\pgfsys@useobject{currentmarker}{}%
\end{pgfscope}%
\end{pgfscope}%
\begin{pgfscope}%
\pgfpathrectangle{\pgfqpoint{0.640323in}{9.767436in}}{\pgfqpoint{9.687500in}{3.850000in}}%
\pgfusepath{clip}%
\pgfsetrectcap%
\pgfsetroundjoin%
\pgfsetlinewidth{0.803000pt}%
\definecolor{currentstroke}{rgb}{0.600000,0.600000,0.600000}%
\pgfsetstrokecolor{currentstroke}%
\pgfsetstrokeopacity{0.200000}%
\pgfsetdash{}{0pt}%
\pgfpathmoveto{\pgfqpoint{0.640323in}{13.244855in}}%
\pgfpathlineto{\pgfqpoint{10.327822in}{13.244855in}}%
\pgfusepath{stroke}%
\end{pgfscope}%
\begin{pgfscope}%
\pgfsetbuttcap%
\pgfsetroundjoin%
\definecolor{currentfill}{rgb}{0.000000,0.000000,0.000000}%
\pgfsetfillcolor{currentfill}%
\pgfsetlinewidth{0.602250pt}%
\definecolor{currentstroke}{rgb}{0.000000,0.000000,0.000000}%
\pgfsetstrokecolor{currentstroke}%
\pgfsetdash{}{0pt}%
\pgfsys@defobject{currentmarker}{\pgfqpoint{-0.027778in}{0.000000in}}{\pgfqpoint{-0.000000in}{0.000000in}}{%
\pgfpathmoveto{\pgfqpoint{-0.000000in}{0.000000in}}%
\pgfpathlineto{\pgfqpoint{-0.027778in}{0.000000in}}%
\pgfusepath{stroke,fill}%
}%
\begin{pgfscope}%
\pgfsys@transformshift{0.640323in}{13.244855in}%
\pgfsys@useobject{currentmarker}{}%
\end{pgfscope}%
\end{pgfscope}%
\begin{pgfscope}%
\pgfpathrectangle{\pgfqpoint{0.640323in}{9.767436in}}{\pgfqpoint{9.687500in}{3.850000in}}%
\pgfusepath{clip}%
\pgfsetrectcap%
\pgfsetroundjoin%
\pgfsetlinewidth{0.803000pt}%
\definecolor{currentstroke}{rgb}{0.600000,0.600000,0.600000}%
\pgfsetstrokecolor{currentstroke}%
\pgfsetstrokeopacity{0.200000}%
\pgfsetdash{}{0pt}%
\pgfpathmoveto{\pgfqpoint{0.640323in}{13.369049in}}%
\pgfpathlineto{\pgfqpoint{10.327822in}{13.369049in}}%
\pgfusepath{stroke}%
\end{pgfscope}%
\begin{pgfscope}%
\pgfsetbuttcap%
\pgfsetroundjoin%
\definecolor{currentfill}{rgb}{0.000000,0.000000,0.000000}%
\pgfsetfillcolor{currentfill}%
\pgfsetlinewidth{0.602250pt}%
\definecolor{currentstroke}{rgb}{0.000000,0.000000,0.000000}%
\pgfsetstrokecolor{currentstroke}%
\pgfsetdash{}{0pt}%
\pgfsys@defobject{currentmarker}{\pgfqpoint{-0.027778in}{0.000000in}}{\pgfqpoint{-0.000000in}{0.000000in}}{%
\pgfpathmoveto{\pgfqpoint{-0.000000in}{0.000000in}}%
\pgfpathlineto{\pgfqpoint{-0.027778in}{0.000000in}}%
\pgfusepath{stroke,fill}%
}%
\begin{pgfscope}%
\pgfsys@transformshift{0.640323in}{13.369049in}%
\pgfsys@useobject{currentmarker}{}%
\end{pgfscope}%
\end{pgfscope}%
\begin{pgfscope}%
\pgfpathrectangle{\pgfqpoint{0.640323in}{9.767436in}}{\pgfqpoint{9.687500in}{3.850000in}}%
\pgfusepath{clip}%
\pgfsetrectcap%
\pgfsetroundjoin%
\pgfsetlinewidth{0.803000pt}%
\definecolor{currentstroke}{rgb}{0.600000,0.600000,0.600000}%
\pgfsetstrokecolor{currentstroke}%
\pgfsetstrokeopacity{0.200000}%
\pgfsetdash{}{0pt}%
\pgfpathmoveto{\pgfqpoint{0.640323in}{13.493242in}}%
\pgfpathlineto{\pgfqpoint{10.327822in}{13.493242in}}%
\pgfusepath{stroke}%
\end{pgfscope}%
\begin{pgfscope}%
\pgfsetbuttcap%
\pgfsetroundjoin%
\definecolor{currentfill}{rgb}{0.000000,0.000000,0.000000}%
\pgfsetfillcolor{currentfill}%
\pgfsetlinewidth{0.602250pt}%
\definecolor{currentstroke}{rgb}{0.000000,0.000000,0.000000}%
\pgfsetstrokecolor{currentstroke}%
\pgfsetdash{}{0pt}%
\pgfsys@defobject{currentmarker}{\pgfqpoint{-0.027778in}{0.000000in}}{\pgfqpoint{-0.000000in}{0.000000in}}{%
\pgfpathmoveto{\pgfqpoint{-0.000000in}{0.000000in}}%
\pgfpathlineto{\pgfqpoint{-0.027778in}{0.000000in}}%
\pgfusepath{stroke,fill}%
}%
\begin{pgfscope}%
\pgfsys@transformshift{0.640323in}{13.493242in}%
\pgfsys@useobject{currentmarker}{}%
\end{pgfscope}%
\end{pgfscope}%
\begin{pgfscope}%
\definecolor{textcolor}{rgb}{0.000000,0.000000,0.000000}%
\pgfsetstrokecolor{textcolor}%
\pgfsetfillcolor{textcolor}%
\pgftext[x=0.266665in,y=11.692436in,,bottom,rotate=90.000000]{\color{textcolor}\sffamily\fontsize{10.000000}{12.000000}\selectfont avg. infection rate \(\displaystyle \overline{\langle I\rangle}\)}%
\end{pgfscope}%
\begin{pgfscope}%
\pgfpathrectangle{\pgfqpoint{0.640323in}{9.767436in}}{\pgfqpoint{9.687500in}{3.850000in}}%
\pgfusepath{clip}%
\pgfsetbuttcap%
\pgfsetroundjoin%
\pgfsetlinewidth{1.003750pt}%
\definecolor{currentstroke}{rgb}{0.000000,0.000000,1.000000}%
\pgfsetstrokecolor{currentstroke}%
\pgfsetstrokeopacity{0.500000}%
\pgfsetdash{{3.700000pt}{1.600000pt}}{0.000000pt}%
\pgfpathmoveto{\pgfqpoint{1.080663in}{9.899391in}}%
\pgfpathlineto{\pgfqpoint{1.260394in}{9.900022in}}%
\pgfpathlineto{\pgfqpoint{1.440125in}{9.902399in}}%
\pgfpathlineto{\pgfqpoint{1.619856in}{9.905553in}}%
\pgfpathlineto{\pgfqpoint{1.799587in}{9.905262in}}%
\pgfpathlineto{\pgfqpoint{1.979318in}{9.913363in}}%
\pgfpathlineto{\pgfqpoint{2.159049in}{9.912830in}}%
\pgfpathlineto{\pgfqpoint{2.338780in}{9.914503in}}%
\pgfpathlineto{\pgfqpoint{2.518511in}{9.923842in}}%
\pgfpathlineto{\pgfqpoint{2.698242in}{9.949142in}}%
\pgfpathlineto{\pgfqpoint{2.877973in}{9.994186in}}%
\pgfpathlineto{\pgfqpoint{3.057704in}{10.385080in}}%
\pgfpathlineto{\pgfqpoint{3.237435in}{10.869287in}}%
\pgfpathlineto{\pgfqpoint{3.417166in}{10.938860in}}%
\pgfpathlineto{\pgfqpoint{3.596897in}{11.149381in}}%
\pgfpathlineto{\pgfqpoint{3.776628in}{11.263968in}}%
\pgfpathlineto{\pgfqpoint{3.956359in}{11.293029in}}%
\pgfpathlineto{\pgfqpoint{4.136090in}{11.472067in}}%
\pgfpathlineto{\pgfqpoint{4.315821in}{11.621192in}}%
\pgfpathlineto{\pgfqpoint{4.495552in}{11.671621in}}%
\pgfpathlineto{\pgfqpoint{4.675283in}{11.706842in}}%
\pgfpathlineto{\pgfqpoint{4.855014in}{11.764064in}}%
\pgfpathlineto{\pgfqpoint{5.034745in}{11.845157in}}%
\pgfpathlineto{\pgfqpoint{5.214476in}{11.890152in}}%
\pgfpathlineto{\pgfqpoint{5.394207in}{11.924113in}}%
\pgfpathlineto{\pgfqpoint{5.573938in}{11.921467in}}%
\pgfpathlineto{\pgfqpoint{5.753669in}{12.042022in}}%
\pgfpathlineto{\pgfqpoint{5.933400in}{12.024703in}}%
\pgfpathlineto{\pgfqpoint{6.113131in}{12.074163in}}%
\pgfpathlineto{\pgfqpoint{6.292862in}{12.109720in}}%
\pgfpathlineto{\pgfqpoint{6.472593in}{12.101331in}}%
\pgfpathlineto{\pgfqpoint{6.652324in}{12.120587in}}%
\pgfpathlineto{\pgfqpoint{6.832055in}{12.167985in}}%
\pgfpathlineto{\pgfqpoint{7.011786in}{12.193190in}}%
\pgfpathlineto{\pgfqpoint{7.191517in}{12.211112in}}%
\pgfpathlineto{\pgfqpoint{7.371248in}{12.236900in}}%
\pgfpathlineto{\pgfqpoint{7.550979in}{12.269427in}}%
\pgfpathlineto{\pgfqpoint{7.730710in}{12.262391in}}%
\pgfpathlineto{\pgfqpoint{7.910441in}{12.257274in}}%
\pgfpathlineto{\pgfqpoint{8.090172in}{12.295333in}}%
\pgfpathlineto{\pgfqpoint{8.269903in}{12.305741in}}%
\pgfpathlineto{\pgfqpoint{8.449634in}{12.347991in}}%
\pgfpathlineto{\pgfqpoint{8.629365in}{12.322861in}}%
\pgfpathlineto{\pgfqpoint{8.809096in}{12.368129in}}%
\pgfpathlineto{\pgfqpoint{8.988827in}{12.380766in}}%
\pgfpathlineto{\pgfqpoint{9.168558in}{12.397284in}}%
\pgfpathlineto{\pgfqpoint{9.348289in}{12.388913in}}%
\pgfpathlineto{\pgfqpoint{9.528020in}{12.391968in}}%
\pgfpathlineto{\pgfqpoint{9.707751in}{12.422924in}}%
\pgfpathlineto{\pgfqpoint{9.887482in}{12.423259in}}%
\pgfusepath{stroke}%
\end{pgfscope}%
\begin{pgfscope}%
\pgfpathrectangle{\pgfqpoint{0.640323in}{9.767436in}}{\pgfqpoint{9.687500in}{3.850000in}}%
\pgfusepath{clip}%
\pgfsetbuttcap%
\pgfsetroundjoin%
\pgfsetlinewidth{1.003750pt}%
\definecolor{currentstroke}{rgb}{0.980392,0.164706,0.333333}%
\pgfsetstrokecolor{currentstroke}%
\pgfsetstrokeopacity{0.500000}%
\pgfsetdash{{3.700000pt}{1.600000pt}}{0.000000pt}%
\pgfpathmoveto{\pgfqpoint{1.080663in}{9.899119in}}%
\pgfpathlineto{\pgfqpoint{1.260394in}{9.901562in}}%
\pgfpathlineto{\pgfqpoint{1.440125in}{9.903036in}}%
\pgfpathlineto{\pgfqpoint{1.619856in}{9.903576in}}%
\pgfpathlineto{\pgfqpoint{1.799587in}{9.905353in}}%
\pgfpathlineto{\pgfqpoint{1.979318in}{9.910307in}}%
\pgfpathlineto{\pgfqpoint{2.159049in}{9.910531in}}%
\pgfpathlineto{\pgfqpoint{2.338780in}{9.931271in}}%
\pgfpathlineto{\pgfqpoint{2.518511in}{9.930488in}}%
\pgfpathlineto{\pgfqpoint{2.698242in}{9.961937in}}%
\pgfpathlineto{\pgfqpoint{2.877973in}{10.346506in}}%
\pgfpathlineto{\pgfqpoint{3.057704in}{10.660375in}}%
\pgfpathlineto{\pgfqpoint{3.237435in}{10.928155in}}%
\pgfpathlineto{\pgfqpoint{3.417166in}{11.155131in}}%
\pgfpathlineto{\pgfqpoint{3.596897in}{11.258044in}}%
\pgfpathlineto{\pgfqpoint{3.776628in}{11.367067in}}%
\pgfpathlineto{\pgfqpoint{3.956359in}{11.492161in}}%
\pgfpathlineto{\pgfqpoint{4.136090in}{11.565516in}}%
\pgfpathlineto{\pgfqpoint{4.315821in}{11.655749in}}%
\pgfpathlineto{\pgfqpoint{4.495552in}{11.729986in}}%
\pgfpathlineto{\pgfqpoint{4.675283in}{11.783159in}}%
\pgfpathlineto{\pgfqpoint{4.855014in}{11.813022in}}%
\pgfpathlineto{\pgfqpoint{5.034745in}{11.889425in}}%
\pgfpathlineto{\pgfqpoint{5.214476in}{11.921523in}}%
\pgfpathlineto{\pgfqpoint{5.394207in}{11.974063in}}%
\pgfpathlineto{\pgfqpoint{5.573938in}{12.006136in}}%
\pgfpathlineto{\pgfqpoint{5.753669in}{12.026697in}}%
\pgfpathlineto{\pgfqpoint{5.933400in}{12.072760in}}%
\pgfpathlineto{\pgfqpoint{6.113131in}{12.105355in}}%
\pgfpathlineto{\pgfqpoint{6.292862in}{12.138036in}}%
\pgfpathlineto{\pgfqpoint{6.472593in}{12.162440in}}%
\pgfpathlineto{\pgfqpoint{6.652324in}{12.186956in}}%
\pgfpathlineto{\pgfqpoint{6.832055in}{12.201580in}}%
\pgfpathlineto{\pgfqpoint{7.011786in}{12.224034in}}%
\pgfpathlineto{\pgfqpoint{7.191517in}{12.242893in}}%
\pgfpathlineto{\pgfqpoint{7.371248in}{12.274550in}}%
\pgfpathlineto{\pgfqpoint{7.550979in}{12.296240in}}%
\pgfpathlineto{\pgfqpoint{7.730710in}{12.299488in}}%
\pgfpathlineto{\pgfqpoint{7.910441in}{12.302909in}}%
\pgfpathlineto{\pgfqpoint{8.090172in}{12.331325in}}%
\pgfpathlineto{\pgfqpoint{8.269903in}{12.356039in}}%
\pgfpathlineto{\pgfqpoint{8.449634in}{12.350047in}}%
\pgfpathlineto{\pgfqpoint{8.629365in}{12.374147in}}%
\pgfpathlineto{\pgfqpoint{8.809096in}{12.396669in}}%
\pgfpathlineto{\pgfqpoint{8.988827in}{12.390826in}}%
\pgfpathlineto{\pgfqpoint{9.168558in}{12.424544in}}%
\pgfpathlineto{\pgfqpoint{9.348289in}{12.424190in}}%
\pgfpathlineto{\pgfqpoint{9.528020in}{12.436231in}}%
\pgfpathlineto{\pgfqpoint{9.707751in}{12.444862in}}%
\pgfpathlineto{\pgfqpoint{9.887482in}{12.469304in}}%
\pgfusepath{stroke}%
\end{pgfscope}%
\begin{pgfscope}%
\pgfpathrectangle{\pgfqpoint{0.640323in}{9.767436in}}{\pgfqpoint{9.687500in}{3.850000in}}%
\pgfusepath{clip}%
\pgfsetbuttcap%
\pgfsetroundjoin%
\pgfsetlinewidth{1.003750pt}%
\definecolor{currentstroke}{rgb}{0.239216,0.478431,0.992157}%
\pgfsetstrokecolor{currentstroke}%
\pgfsetstrokeopacity{0.500000}%
\pgfsetdash{{3.700000pt}{1.600000pt}}{0.000000pt}%
\pgfpathmoveto{\pgfqpoint{1.080663in}{9.899819in}}%
\pgfpathlineto{\pgfqpoint{1.260394in}{9.900851in}}%
\pgfpathlineto{\pgfqpoint{1.440125in}{9.902135in}}%
\pgfpathlineto{\pgfqpoint{1.619856in}{9.903297in}}%
\pgfpathlineto{\pgfqpoint{1.799587in}{9.905530in}}%
\pgfpathlineto{\pgfqpoint{1.979318in}{9.906911in}}%
\pgfpathlineto{\pgfqpoint{2.159049in}{9.914432in}}%
\pgfpathlineto{\pgfqpoint{2.338780in}{9.921424in}}%
\pgfpathlineto{\pgfqpoint{2.518511in}{9.928775in}}%
\pgfpathlineto{\pgfqpoint{2.698242in}{10.018474in}}%
\pgfpathlineto{\pgfqpoint{2.877973in}{10.446193in}}%
\pgfpathlineto{\pgfqpoint{3.057704in}{10.781520in}}%
\pgfpathlineto{\pgfqpoint{3.237435in}{11.000641in}}%
\pgfpathlineto{\pgfqpoint{3.417166in}{11.181789in}}%
\pgfpathlineto{\pgfqpoint{3.596897in}{11.302996in}}%
\pgfpathlineto{\pgfqpoint{3.776628in}{11.438634in}}%
\pgfpathlineto{\pgfqpoint{3.956359in}{11.531605in}}%
\pgfpathlineto{\pgfqpoint{4.136090in}{11.611542in}}%
\pgfpathlineto{\pgfqpoint{4.315821in}{11.691697in}}%
\pgfpathlineto{\pgfqpoint{4.495552in}{11.753551in}}%
\pgfpathlineto{\pgfqpoint{4.675283in}{11.807035in}}%
\pgfpathlineto{\pgfqpoint{4.855014in}{11.859116in}}%
\pgfpathlineto{\pgfqpoint{5.034745in}{11.908011in}}%
\pgfpathlineto{\pgfqpoint{5.214476in}{11.956993in}}%
\pgfpathlineto{\pgfqpoint{5.394207in}{11.990519in}}%
\pgfpathlineto{\pgfqpoint{5.573938in}{12.034204in}}%
\pgfpathlineto{\pgfqpoint{5.753669in}{12.060341in}}%
\pgfpathlineto{\pgfqpoint{5.933400in}{12.094457in}}%
\pgfpathlineto{\pgfqpoint{6.113131in}{12.126294in}}%
\pgfpathlineto{\pgfqpoint{6.292862in}{12.151679in}}%
\pgfpathlineto{\pgfqpoint{6.472593in}{12.183336in}}%
\pgfpathlineto{\pgfqpoint{6.652324in}{12.204020in}}%
\pgfpathlineto{\pgfqpoint{6.832055in}{12.222357in}}%
\pgfpathlineto{\pgfqpoint{7.011786in}{12.241806in}}%
\pgfpathlineto{\pgfqpoint{7.191517in}{12.267825in}}%
\pgfpathlineto{\pgfqpoint{7.371248in}{12.285156in}}%
\pgfpathlineto{\pgfqpoint{7.550979in}{12.303611in}}%
\pgfpathlineto{\pgfqpoint{7.730710in}{12.314826in}}%
\pgfpathlineto{\pgfqpoint{7.910441in}{12.333802in}}%
\pgfpathlineto{\pgfqpoint{8.090172in}{12.342788in}}%
\pgfpathlineto{\pgfqpoint{8.269903in}{12.363267in}}%
\pgfpathlineto{\pgfqpoint{8.449634in}{12.373190in}}%
\pgfpathlineto{\pgfqpoint{8.629365in}{12.393806in}}%
\pgfpathlineto{\pgfqpoint{8.809096in}{12.406996in}}%
\pgfpathlineto{\pgfqpoint{8.988827in}{12.417900in}}%
\pgfpathlineto{\pgfqpoint{9.168558in}{12.433250in}}%
\pgfpathlineto{\pgfqpoint{9.348289in}{12.442900in}}%
\pgfpathlineto{\pgfqpoint{9.528020in}{12.449694in}}%
\pgfpathlineto{\pgfqpoint{9.707751in}{12.460871in}}%
\pgfpathlineto{\pgfqpoint{9.887482in}{12.477637in}}%
\pgfusepath{stroke}%
\end{pgfscope}%
\begin{pgfscope}%
\pgfpathrectangle{\pgfqpoint{0.640323in}{9.767436in}}{\pgfqpoint{9.687500in}{3.850000in}}%
\pgfusepath{clip}%
\pgfsetbuttcap%
\pgfsetroundjoin%
\pgfsetlinewidth{1.003750pt}%
\definecolor{currentstroke}{rgb}{0.000000,0.000000,0.000000}%
\pgfsetstrokecolor{currentstroke}%
\pgfsetstrokeopacity{0.500000}%
\pgfsetdash{{3.700000pt}{1.600000pt}}{0.000000pt}%
\pgfpathmoveto{\pgfqpoint{1.080663in}{9.899636in}}%
\pgfpathlineto{\pgfqpoint{1.260394in}{9.900663in}}%
\pgfpathlineto{\pgfqpoint{1.440125in}{9.901686in}}%
\pgfpathlineto{\pgfqpoint{1.619856in}{9.903556in}}%
\pgfpathlineto{\pgfqpoint{1.799587in}{9.904833in}}%
\pgfpathlineto{\pgfqpoint{1.979318in}{9.908198in}}%
\pgfpathlineto{\pgfqpoint{2.159049in}{9.913532in}}%
\pgfpathlineto{\pgfqpoint{2.338780in}{9.920951in}}%
\pgfpathlineto{\pgfqpoint{2.518511in}{9.939267in}}%
\pgfpathlineto{\pgfqpoint{2.698242in}{10.084847in}}%
\pgfpathlineto{\pgfqpoint{2.877973in}{10.523607in}}%
\pgfpathlineto{\pgfqpoint{3.057704in}{10.826707in}}%
\pgfpathlineto{\pgfqpoint{3.237435in}{11.037321in}}%
\pgfpathlineto{\pgfqpoint{3.417166in}{11.207379in}}%
\pgfpathlineto{\pgfqpoint{3.596897in}{11.347308in}}%
\pgfpathlineto{\pgfqpoint{3.776628in}{11.451761in}}%
\pgfpathlineto{\pgfqpoint{3.956359in}{11.552339in}}%
\pgfpathlineto{\pgfqpoint{4.136090in}{11.631910in}}%
\pgfpathlineto{\pgfqpoint{4.315821in}{11.706060in}}%
\pgfpathlineto{\pgfqpoint{4.495552in}{11.767666in}}%
\pgfpathlineto{\pgfqpoint{4.675283in}{11.823696in}}%
\pgfpathlineto{\pgfqpoint{4.855014in}{11.877820in}}%
\pgfpathlineto{\pgfqpoint{5.034745in}{11.921436in}}%
\pgfpathlineto{\pgfqpoint{5.214476in}{11.968543in}}%
\pgfpathlineto{\pgfqpoint{5.394207in}{12.006813in}}%
\pgfpathlineto{\pgfqpoint{5.573938in}{12.045121in}}%
\pgfpathlineto{\pgfqpoint{5.753669in}{12.074294in}}%
\pgfpathlineto{\pgfqpoint{5.933400in}{12.102281in}}%
\pgfpathlineto{\pgfqpoint{6.113131in}{12.135099in}}%
\pgfpathlineto{\pgfqpoint{6.292862in}{12.156907in}}%
\pgfpathlineto{\pgfqpoint{6.472593in}{12.187496in}}%
\pgfpathlineto{\pgfqpoint{6.652324in}{12.211484in}}%
\pgfpathlineto{\pgfqpoint{6.832055in}{12.234019in}}%
\pgfpathlineto{\pgfqpoint{7.011786in}{12.253480in}}%
\pgfpathlineto{\pgfqpoint{7.191517in}{12.277207in}}%
\pgfpathlineto{\pgfqpoint{7.371248in}{12.291862in}}%
\pgfpathlineto{\pgfqpoint{7.550979in}{12.306946in}}%
\pgfpathlineto{\pgfqpoint{7.730710in}{12.324848in}}%
\pgfpathlineto{\pgfqpoint{7.910441in}{12.338280in}}%
\pgfpathlineto{\pgfqpoint{8.090172in}{12.357163in}}%
\pgfpathlineto{\pgfqpoint{8.269903in}{12.370160in}}%
\pgfpathlineto{\pgfqpoint{8.449634in}{12.383151in}}%
\pgfpathlineto{\pgfqpoint{8.629365in}{12.397725in}}%
\pgfpathlineto{\pgfqpoint{8.809096in}{12.408387in}}%
\pgfpathlineto{\pgfqpoint{8.988827in}{12.418701in}}%
\pgfpathlineto{\pgfqpoint{9.168558in}{12.434908in}}%
\pgfpathlineto{\pgfqpoint{9.348289in}{12.445719in}}%
\pgfpathlineto{\pgfqpoint{9.528020in}{12.454338in}}%
\pgfpathlineto{\pgfqpoint{9.707751in}{12.465168in}}%
\pgfpathlineto{\pgfqpoint{9.887482in}{12.471502in}}%
\pgfusepath{stroke}%
\end{pgfscope}%
\begin{pgfscope}%
\pgfsetrectcap%
\pgfsetmiterjoin%
\pgfsetlinewidth{0.803000pt}%
\definecolor{currentstroke}{rgb}{0.000000,0.000000,0.000000}%
\pgfsetstrokecolor{currentstroke}%
\pgfsetdash{}{0pt}%
\pgfpathmoveto{\pgfqpoint{0.640323in}{9.767436in}}%
\pgfpathlineto{\pgfqpoint{0.640323in}{13.617436in}}%
\pgfusepath{stroke}%
\end{pgfscope}%
\begin{pgfscope}%
\pgfsetrectcap%
\pgfsetmiterjoin%
\pgfsetlinewidth{0.803000pt}%
\definecolor{currentstroke}{rgb}{0.000000,0.000000,0.000000}%
\pgfsetstrokecolor{currentstroke}%
\pgfsetdash{}{0pt}%
\pgfpathmoveto{\pgfqpoint{10.327822in}{9.767436in}}%
\pgfpathlineto{\pgfqpoint{10.327822in}{13.617436in}}%
\pgfusepath{stroke}%
\end{pgfscope}%
\begin{pgfscope}%
\pgfsetrectcap%
\pgfsetmiterjoin%
\pgfsetlinewidth{0.803000pt}%
\definecolor{currentstroke}{rgb}{0.000000,0.000000,0.000000}%
\pgfsetstrokecolor{currentstroke}%
\pgfsetdash{}{0pt}%
\pgfpathmoveto{\pgfqpoint{0.640322in}{9.767436in}}%
\pgfpathlineto{\pgfqpoint{10.327823in}{9.767436in}}%
\pgfusepath{stroke}%
\end{pgfscope}%
\begin{pgfscope}%
\pgfsetrectcap%
\pgfsetmiterjoin%
\pgfsetlinewidth{0.803000pt}%
\definecolor{currentstroke}{rgb}{0.000000,0.000000,0.000000}%
\pgfsetstrokecolor{currentstroke}%
\pgfsetdash{}{0pt}%
\pgfpathmoveto{\pgfqpoint{0.640322in}{13.617436in}}%
\pgfpathlineto{\pgfqpoint{10.327823in}{13.617436in}}%
\pgfusepath{stroke}%
\end{pgfscope}%
\begin{pgfscope}%
\definecolor{textcolor}{rgb}{0.000000,0.000000,0.000000}%
\pgfsetstrokecolor{textcolor}%
\pgfsetfillcolor{textcolor}%
\pgftext[x=5.484072in,y=13.700769in,,base]{\color{textcolor}\sffamily\fontsize{12.000000}{14.400000}\selectfont \(\displaystyle \overline{\langle I\rangle}\) over \(\displaystyle p_1\) for \(\displaystyle T=1000\) with \(\displaystyle p_2=0.3\), \(\displaystyle p_3=0.3\)}%
\end{pgfscope}%
\begin{pgfscope}%
\pgfsetbuttcap%
\pgfsetmiterjoin%
\definecolor{currentfill}{rgb}{1.000000,1.000000,1.000000}%
\pgfsetfillcolor{currentfill}%
\pgfsetfillopacity{0.800000}%
\pgfsetlinewidth{1.003750pt}%
\definecolor{currentstroke}{rgb}{0.800000,0.800000,0.800000}%
\pgfsetstrokecolor{currentstroke}%
\pgfsetstrokeopacity{0.800000}%
\pgfsetdash{}{0pt}%
\pgfpathmoveto{\pgfqpoint{0.737545in}{12.690896in}}%
\pgfpathlineto{\pgfqpoint{1.670029in}{12.690896in}}%
\pgfpathquadraticcurveto{\pgfqpoint{1.697806in}{12.690896in}}{\pgfqpoint{1.697806in}{12.718674in}}%
\pgfpathlineto{\pgfqpoint{1.697806in}{13.520214in}}%
\pgfpathquadraticcurveto{\pgfqpoint{1.697806in}{13.547991in}}{\pgfqpoint{1.670029in}{13.547991in}}%
\pgfpathlineto{\pgfqpoint{0.737545in}{13.547991in}}%
\pgfpathquadraticcurveto{\pgfqpoint{0.709767in}{13.547991in}}{\pgfqpoint{0.709767in}{13.520214in}}%
\pgfpathlineto{\pgfqpoint{0.709767in}{12.718674in}}%
\pgfpathquadraticcurveto{\pgfqpoint{0.709767in}{12.690896in}}{\pgfqpoint{0.737545in}{12.690896in}}%
\pgfpathlineto{\pgfqpoint{0.737545in}{12.690896in}}%
\pgfpathclose%
\pgfusepath{stroke,fill}%
\end{pgfscope}%
\begin{pgfscope}%
\pgfsetbuttcap%
\pgfsetroundjoin%
\definecolor{currentfill}{rgb}{0.000000,0.000000,1.000000}%
\pgfsetfillcolor{currentfill}%
\pgfsetfillopacity{0.500000}%
\pgfsetlinewidth{1.003750pt}%
\definecolor{currentstroke}{rgb}{0.000000,0.000000,1.000000}%
\pgfsetstrokecolor{currentstroke}%
\pgfsetstrokeopacity{0.500000}%
\pgfsetdash{}{0pt}%
\pgfsys@defobject{currentmarker}{\pgfqpoint{-0.021960in}{-0.021960in}}{\pgfqpoint{0.021960in}{0.021960in}}{%
\pgfpathmoveto{\pgfqpoint{0.000000in}{-0.021960in}}%
\pgfpathcurveto{\pgfqpoint{0.005824in}{-0.021960in}}{\pgfqpoint{0.011410in}{-0.019646in}}{\pgfqpoint{0.015528in}{-0.015528in}}%
\pgfpathcurveto{\pgfqpoint{0.019646in}{-0.011410in}}{\pgfqpoint{0.021960in}{-0.005824in}}{\pgfqpoint{0.021960in}{0.000000in}}%
\pgfpathcurveto{\pgfqpoint{0.021960in}{0.005824in}}{\pgfqpoint{0.019646in}{0.011410in}}{\pgfqpoint{0.015528in}{0.015528in}}%
\pgfpathcurveto{\pgfqpoint{0.011410in}{0.019646in}}{\pgfqpoint{0.005824in}{0.021960in}}{\pgfqpoint{0.000000in}{0.021960in}}%
\pgfpathcurveto{\pgfqpoint{-0.005824in}{0.021960in}}{\pgfqpoint{-0.011410in}{0.019646in}}{\pgfqpoint{-0.015528in}{0.015528in}}%
\pgfpathcurveto{\pgfqpoint{-0.019646in}{0.011410in}}{\pgfqpoint{-0.021960in}{0.005824in}}{\pgfqpoint{-0.021960in}{0.000000in}}%
\pgfpathcurveto{\pgfqpoint{-0.021960in}{-0.005824in}}{\pgfqpoint{-0.019646in}{-0.011410in}}{\pgfqpoint{-0.015528in}{-0.015528in}}%
\pgfpathcurveto{\pgfqpoint{-0.011410in}{-0.019646in}}{\pgfqpoint{-0.005824in}{-0.021960in}}{\pgfqpoint{0.000000in}{-0.021960in}}%
\pgfpathlineto{\pgfqpoint{0.000000in}{-0.021960in}}%
\pgfpathclose%
\pgfusepath{stroke,fill}%
}%
\begin{pgfscope}%
\pgfsys@transformshift{0.904211in}{13.423371in}%
\pgfsys@useobject{currentmarker}{}%
\end{pgfscope}%
\end{pgfscope}%
\begin{pgfscope}%
\definecolor{textcolor}{rgb}{0.000000,0.000000,0.000000}%
\pgfsetstrokecolor{textcolor}%
\pgfsetfillcolor{textcolor}%
\pgftext[x=1.154211in,y=13.386913in,left,base]{\color{textcolor}\sffamily\fontsize{10.000000}{12.000000}\selectfont \(\displaystyle L=16\)}%
\end{pgfscope}%
\begin{pgfscope}%
\pgfsetbuttcap%
\pgfsetroundjoin%
\definecolor{currentfill}{rgb}{0.980392,0.164706,0.333333}%
\pgfsetfillcolor{currentfill}%
\pgfsetfillopacity{0.500000}%
\pgfsetlinewidth{1.003750pt}%
\definecolor{currentstroke}{rgb}{0.980392,0.164706,0.333333}%
\pgfsetstrokecolor{currentstroke}%
\pgfsetstrokeopacity{0.500000}%
\pgfsetdash{}{0pt}%
\pgfsys@defobject{currentmarker}{\pgfqpoint{-0.021960in}{-0.021960in}}{\pgfqpoint{0.021960in}{0.021960in}}{%
\pgfpathmoveto{\pgfqpoint{0.000000in}{-0.021960in}}%
\pgfpathcurveto{\pgfqpoint{0.005824in}{-0.021960in}}{\pgfqpoint{0.011410in}{-0.019646in}}{\pgfqpoint{0.015528in}{-0.015528in}}%
\pgfpathcurveto{\pgfqpoint{0.019646in}{-0.011410in}}{\pgfqpoint{0.021960in}{-0.005824in}}{\pgfqpoint{0.021960in}{0.000000in}}%
\pgfpathcurveto{\pgfqpoint{0.021960in}{0.005824in}}{\pgfqpoint{0.019646in}{0.011410in}}{\pgfqpoint{0.015528in}{0.015528in}}%
\pgfpathcurveto{\pgfqpoint{0.011410in}{0.019646in}}{\pgfqpoint{0.005824in}{0.021960in}}{\pgfqpoint{0.000000in}{0.021960in}}%
\pgfpathcurveto{\pgfqpoint{-0.005824in}{0.021960in}}{\pgfqpoint{-0.011410in}{0.019646in}}{\pgfqpoint{-0.015528in}{0.015528in}}%
\pgfpathcurveto{\pgfqpoint{-0.019646in}{0.011410in}}{\pgfqpoint{-0.021960in}{0.005824in}}{\pgfqpoint{-0.021960in}{0.000000in}}%
\pgfpathcurveto{\pgfqpoint{-0.021960in}{-0.005824in}}{\pgfqpoint{-0.019646in}{-0.011410in}}{\pgfqpoint{-0.015528in}{-0.015528in}}%
\pgfpathcurveto{\pgfqpoint{-0.011410in}{-0.019646in}}{\pgfqpoint{-0.005824in}{-0.021960in}}{\pgfqpoint{0.000000in}{-0.021960in}}%
\pgfpathlineto{\pgfqpoint{0.000000in}{-0.021960in}}%
\pgfpathclose%
\pgfusepath{stroke,fill}%
}%
\begin{pgfscope}%
\pgfsys@transformshift{0.904211in}{13.219514in}%
\pgfsys@useobject{currentmarker}{}%
\end{pgfscope}%
\end{pgfscope}%
\begin{pgfscope}%
\definecolor{textcolor}{rgb}{0.000000,0.000000,0.000000}%
\pgfsetstrokecolor{textcolor}%
\pgfsetfillcolor{textcolor}%
\pgftext[x=1.154211in,y=13.183056in,left,base]{\color{textcolor}\sffamily\fontsize{10.000000}{12.000000}\selectfont \(\displaystyle L=32\)}%
\end{pgfscope}%
\begin{pgfscope}%
\pgfsetbuttcap%
\pgfsetroundjoin%
\definecolor{currentfill}{rgb}{0.239216,0.478431,0.992157}%
\pgfsetfillcolor{currentfill}%
\pgfsetfillopacity{0.500000}%
\pgfsetlinewidth{1.003750pt}%
\definecolor{currentstroke}{rgb}{0.239216,0.478431,0.992157}%
\pgfsetstrokecolor{currentstroke}%
\pgfsetstrokeopacity{0.500000}%
\pgfsetdash{}{0pt}%
\pgfsys@defobject{currentmarker}{\pgfqpoint{-0.021960in}{-0.021960in}}{\pgfqpoint{0.021960in}{0.021960in}}{%
\pgfpathmoveto{\pgfqpoint{0.000000in}{-0.021960in}}%
\pgfpathcurveto{\pgfqpoint{0.005824in}{-0.021960in}}{\pgfqpoint{0.011410in}{-0.019646in}}{\pgfqpoint{0.015528in}{-0.015528in}}%
\pgfpathcurveto{\pgfqpoint{0.019646in}{-0.011410in}}{\pgfqpoint{0.021960in}{-0.005824in}}{\pgfqpoint{0.021960in}{0.000000in}}%
\pgfpathcurveto{\pgfqpoint{0.021960in}{0.005824in}}{\pgfqpoint{0.019646in}{0.011410in}}{\pgfqpoint{0.015528in}{0.015528in}}%
\pgfpathcurveto{\pgfqpoint{0.011410in}{0.019646in}}{\pgfqpoint{0.005824in}{0.021960in}}{\pgfqpoint{0.000000in}{0.021960in}}%
\pgfpathcurveto{\pgfqpoint{-0.005824in}{0.021960in}}{\pgfqpoint{-0.011410in}{0.019646in}}{\pgfqpoint{-0.015528in}{0.015528in}}%
\pgfpathcurveto{\pgfqpoint{-0.019646in}{0.011410in}}{\pgfqpoint{-0.021960in}{0.005824in}}{\pgfqpoint{-0.021960in}{0.000000in}}%
\pgfpathcurveto{\pgfqpoint{-0.021960in}{-0.005824in}}{\pgfqpoint{-0.019646in}{-0.011410in}}{\pgfqpoint{-0.015528in}{-0.015528in}}%
\pgfpathcurveto{\pgfqpoint{-0.011410in}{-0.019646in}}{\pgfqpoint{-0.005824in}{-0.021960in}}{\pgfqpoint{0.000000in}{-0.021960in}}%
\pgfpathlineto{\pgfqpoint{0.000000in}{-0.021960in}}%
\pgfpathclose%
\pgfusepath{stroke,fill}%
}%
\begin{pgfscope}%
\pgfsys@transformshift{0.904211in}{13.015657in}%
\pgfsys@useobject{currentmarker}{}%
\end{pgfscope}%
\end{pgfscope}%
\begin{pgfscope}%
\definecolor{textcolor}{rgb}{0.000000,0.000000,0.000000}%
\pgfsetstrokecolor{textcolor}%
\pgfsetfillcolor{textcolor}%
\pgftext[x=1.154211in,y=12.979198in,left,base]{\color{textcolor}\sffamily\fontsize{10.000000}{12.000000}\selectfont \(\displaystyle L=64\)}%
\end{pgfscope}%
\begin{pgfscope}%
\pgfsetbuttcap%
\pgfsetroundjoin%
\definecolor{currentfill}{rgb}{0.000000,0.000000,0.000000}%
\pgfsetfillcolor{currentfill}%
\pgfsetfillopacity{0.500000}%
\pgfsetlinewidth{1.003750pt}%
\definecolor{currentstroke}{rgb}{0.000000,0.000000,0.000000}%
\pgfsetstrokecolor{currentstroke}%
\pgfsetstrokeopacity{0.500000}%
\pgfsetdash{}{0pt}%
\pgfsys@defobject{currentmarker}{\pgfqpoint{-0.021960in}{-0.021960in}}{\pgfqpoint{0.021960in}{0.021960in}}{%
\pgfpathmoveto{\pgfqpoint{0.000000in}{-0.021960in}}%
\pgfpathcurveto{\pgfqpoint{0.005824in}{-0.021960in}}{\pgfqpoint{0.011410in}{-0.019646in}}{\pgfqpoint{0.015528in}{-0.015528in}}%
\pgfpathcurveto{\pgfqpoint{0.019646in}{-0.011410in}}{\pgfqpoint{0.021960in}{-0.005824in}}{\pgfqpoint{0.021960in}{0.000000in}}%
\pgfpathcurveto{\pgfqpoint{0.021960in}{0.005824in}}{\pgfqpoint{0.019646in}{0.011410in}}{\pgfqpoint{0.015528in}{0.015528in}}%
\pgfpathcurveto{\pgfqpoint{0.011410in}{0.019646in}}{\pgfqpoint{0.005824in}{0.021960in}}{\pgfqpoint{0.000000in}{0.021960in}}%
\pgfpathcurveto{\pgfqpoint{-0.005824in}{0.021960in}}{\pgfqpoint{-0.011410in}{0.019646in}}{\pgfqpoint{-0.015528in}{0.015528in}}%
\pgfpathcurveto{\pgfqpoint{-0.019646in}{0.011410in}}{\pgfqpoint{-0.021960in}{0.005824in}}{\pgfqpoint{-0.021960in}{0.000000in}}%
\pgfpathcurveto{\pgfqpoint{-0.021960in}{-0.005824in}}{\pgfqpoint{-0.019646in}{-0.011410in}}{\pgfqpoint{-0.015528in}{-0.015528in}}%
\pgfpathcurveto{\pgfqpoint{-0.011410in}{-0.019646in}}{\pgfqpoint{-0.005824in}{-0.021960in}}{\pgfqpoint{0.000000in}{-0.021960in}}%
\pgfpathlineto{\pgfqpoint{0.000000in}{-0.021960in}}%
\pgfpathclose%
\pgfusepath{stroke,fill}%
}%
\begin{pgfscope}%
\pgfsys@transformshift{0.904211in}{12.811799in}%
\pgfsys@useobject{currentmarker}{}%
\end{pgfscope}%
\end{pgfscope}%
\begin{pgfscope}%
\definecolor{textcolor}{rgb}{0.000000,0.000000,0.000000}%
\pgfsetstrokecolor{textcolor}%
\pgfsetfillcolor{textcolor}%
\pgftext[x=1.154211in,y=12.775341in,left,base]{\color{textcolor}\sffamily\fontsize{10.000000}{12.000000}\selectfont \(\displaystyle L=128\)}%
\end{pgfscope}%
\begin{pgfscope}%
\pgfsetbuttcap%
\pgfsetmiterjoin%
\definecolor{currentfill}{rgb}{1.000000,1.000000,1.000000}%
\pgfsetfillcolor{currentfill}%
\pgfsetlinewidth{0.000000pt}%
\definecolor{currentstroke}{rgb}{0.000000,0.000000,0.000000}%
\pgfsetstrokecolor{currentstroke}%
\pgfsetstrokeopacity{0.000000}%
\pgfsetdash{}{0pt}%
\pgfpathmoveto{\pgfqpoint{0.640323in}{5.147436in}}%
\pgfpathlineto{\pgfqpoint{10.327822in}{5.147436in}}%
\pgfpathlineto{\pgfqpoint{10.327822in}{8.997436in}}%
\pgfpathlineto{\pgfqpoint{0.640323in}{8.997436in}}%
\pgfpathlineto{\pgfqpoint{0.640323in}{5.147436in}}%
\pgfpathclose%
\pgfusepath{fill}%
\end{pgfscope}%
\begin{pgfscope}%
\pgfpathrectangle{\pgfqpoint{0.640323in}{5.147436in}}{\pgfqpoint{9.687500in}{3.850000in}}%
\pgfusepath{clip}%
\pgfsetbuttcap%
\pgfsetroundjoin%
\definecolor{currentfill}{rgb}{0.000000,0.000000,1.000000}%
\pgfsetfillcolor{currentfill}%
\pgfsetfillopacity{0.500000}%
\pgfsetlinewidth{1.003750pt}%
\definecolor{currentstroke}{rgb}{0.000000,0.000000,1.000000}%
\pgfsetstrokecolor{currentstroke}%
\pgfsetstrokeopacity{0.500000}%
\pgfsetdash{}{0pt}%
\pgfsys@defobject{currentmarker}{\pgfqpoint{-0.021960in}{-0.021960in}}{\pgfqpoint{0.021960in}{0.021960in}}{%
\pgfpathmoveto{\pgfqpoint{0.000000in}{-0.021960in}}%
\pgfpathcurveto{\pgfqpoint{0.005824in}{-0.021960in}}{\pgfqpoint{0.011410in}{-0.019646in}}{\pgfqpoint{0.015528in}{-0.015528in}}%
\pgfpathcurveto{\pgfqpoint{0.019646in}{-0.011410in}}{\pgfqpoint{0.021960in}{-0.005824in}}{\pgfqpoint{0.021960in}{0.000000in}}%
\pgfpathcurveto{\pgfqpoint{0.021960in}{0.005824in}}{\pgfqpoint{0.019646in}{0.011410in}}{\pgfqpoint{0.015528in}{0.015528in}}%
\pgfpathcurveto{\pgfqpoint{0.011410in}{0.019646in}}{\pgfqpoint{0.005824in}{0.021960in}}{\pgfqpoint{0.000000in}{0.021960in}}%
\pgfpathcurveto{\pgfqpoint{-0.005824in}{0.021960in}}{\pgfqpoint{-0.011410in}{0.019646in}}{\pgfqpoint{-0.015528in}{0.015528in}}%
\pgfpathcurveto{\pgfqpoint{-0.019646in}{0.011410in}}{\pgfqpoint{-0.021960in}{0.005824in}}{\pgfqpoint{-0.021960in}{0.000000in}}%
\pgfpathcurveto{\pgfqpoint{-0.021960in}{-0.005824in}}{\pgfqpoint{-0.019646in}{-0.011410in}}{\pgfqpoint{-0.015528in}{-0.015528in}}%
\pgfpathcurveto{\pgfqpoint{-0.011410in}{-0.019646in}}{\pgfqpoint{-0.005824in}{-0.021960in}}{\pgfqpoint{0.000000in}{-0.021960in}}%
\pgfpathlineto{\pgfqpoint{0.000000in}{-0.021960in}}%
\pgfpathclose%
\pgfusepath{stroke,fill}%
}%
\begin{pgfscope}%
\pgfsys@transformshift{1.080663in}{5.275389in}%
\pgfsys@useobject{currentmarker}{}%
\end{pgfscope}%
\begin{pgfscope}%
\pgfsys@transformshift{1.260394in}{5.275607in}%
\pgfsys@useobject{currentmarker}{}%
\end{pgfscope}%
\begin{pgfscope}%
\pgfsys@transformshift{1.440125in}{5.276505in}%
\pgfsys@useobject{currentmarker}{}%
\end{pgfscope}%
\begin{pgfscope}%
\pgfsys@transformshift{1.619856in}{5.276626in}%
\pgfsys@useobject{currentmarker}{}%
\end{pgfscope}%
\begin{pgfscope}%
\pgfsys@transformshift{1.799587in}{5.277742in}%
\pgfsys@useobject{currentmarker}{}%
\end{pgfscope}%
\begin{pgfscope}%
\pgfsys@transformshift{1.979318in}{5.277378in}%
\pgfsys@useobject{currentmarker}{}%
\end{pgfscope}%
\begin{pgfscope}%
\pgfsys@transformshift{2.159049in}{5.279489in}%
\pgfsys@useobject{currentmarker}{}%
\end{pgfscope}%
\begin{pgfscope}%
\pgfsys@transformshift{2.338780in}{5.278809in}%
\pgfsys@useobject{currentmarker}{}%
\end{pgfscope}%
\begin{pgfscope}%
\pgfsys@transformshift{2.518511in}{5.276917in}%
\pgfsys@useobject{currentmarker}{}%
\end{pgfscope}%
\begin{pgfscope}%
\pgfsys@transformshift{2.698242in}{5.278057in}%
\pgfsys@useobject{currentmarker}{}%
\end{pgfscope}%
\begin{pgfscope}%
\pgfsys@transformshift{2.877973in}{5.277475in}%
\pgfsys@useobject{currentmarker}{}%
\end{pgfscope}%
\begin{pgfscope}%
\pgfsys@transformshift{3.057704in}{5.281162in}%
\pgfsys@useobject{currentmarker}{}%
\end{pgfscope}%
\begin{pgfscope}%
\pgfsys@transformshift{3.237435in}{5.278518in}%
\pgfsys@useobject{currentmarker}{}%
\end{pgfscope}%
\begin{pgfscope}%
\pgfsys@transformshift{3.417166in}{5.280968in}%
\pgfsys@useobject{currentmarker}{}%
\end{pgfscope}%
\begin{pgfscope}%
\pgfsys@transformshift{3.596897in}{5.278785in}%
\pgfsys@useobject{currentmarker}{}%
\end{pgfscope}%
\begin{pgfscope}%
\pgfsys@transformshift{3.776628in}{5.283733in}%
\pgfsys@useobject{currentmarker}{}%
\end{pgfscope}%
\begin{pgfscope}%
\pgfsys@transformshift{3.956359in}{5.283345in}%
\pgfsys@useobject{currentmarker}{}%
\end{pgfscope}%
\begin{pgfscope}%
\pgfsys@transformshift{4.136090in}{5.281235in}%
\pgfsys@useobject{currentmarker}{}%
\end{pgfscope}%
\begin{pgfscope}%
\pgfsys@transformshift{4.315821in}{5.290477in}%
\pgfsys@useobject{currentmarker}{}%
\end{pgfscope}%
\begin{pgfscope}%
\pgfsys@transformshift{4.495552in}{5.279901in}%
\pgfsys@useobject{currentmarker}{}%
\end{pgfscope}%
\begin{pgfscope}%
\pgfsys@transformshift{4.675283in}{5.310513in}%
\pgfsys@useobject{currentmarker}{}%
\end{pgfscope}%
\begin{pgfscope}%
\pgfsys@transformshift{4.855014in}{5.287687in}%
\pgfsys@useobject{currentmarker}{}%
\end{pgfscope}%
\begin{pgfscope}%
\pgfsys@transformshift{5.034745in}{5.279537in}%
\pgfsys@useobject{currentmarker}{}%
\end{pgfscope}%
\begin{pgfscope}%
\pgfsys@transformshift{5.214476in}{5.323684in}%
\pgfsys@useobject{currentmarker}{}%
\end{pgfscope}%
\begin{pgfscope}%
\pgfsys@transformshift{5.394207in}{5.299646in}%
\pgfsys@useobject{currentmarker}{}%
\end{pgfscope}%
\begin{pgfscope}%
\pgfsys@transformshift{5.573938in}{5.309664in}%
\pgfsys@useobject{currentmarker}{}%
\end{pgfscope}%
\begin{pgfscope}%
\pgfsys@transformshift{5.753669in}{5.343817in}%
\pgfsys@useobject{currentmarker}{}%
\end{pgfscope}%
\begin{pgfscope}%
\pgfsys@transformshift{5.933400in}{5.524043in}%
\pgfsys@useobject{currentmarker}{}%
\end{pgfscope}%
\begin{pgfscope}%
\pgfsys@transformshift{6.113131in}{5.310319in}%
\pgfsys@useobject{currentmarker}{}%
\end{pgfscope}%
\begin{pgfscope}%
\pgfsys@transformshift{6.292862in}{5.909891in}%
\pgfsys@useobject{currentmarker}{}%
\end{pgfscope}%
\begin{pgfscope}%
\pgfsys@transformshift{6.472593in}{5.552738in}%
\pgfsys@useobject{currentmarker}{}%
\end{pgfscope}%
\begin{pgfscope}%
\pgfsys@transformshift{6.652324in}{5.463984in}%
\pgfsys@useobject{currentmarker}{}%
\end{pgfscope}%
\begin{pgfscope}%
\pgfsys@transformshift{6.832055in}{5.505948in}%
\pgfsys@useobject{currentmarker}{}%
\end{pgfscope}%
\begin{pgfscope}%
\pgfsys@transformshift{7.011786in}{5.527730in}%
\pgfsys@useobject{currentmarker}{}%
\end{pgfscope}%
\begin{pgfscope}%
\pgfsys@transformshift{7.191517in}{6.071175in}%
\pgfsys@useobject{currentmarker}{}%
\end{pgfscope}%
\begin{pgfscope}%
\pgfsys@transformshift{7.371248in}{5.485451in}%
\pgfsys@useobject{currentmarker}{}%
\end{pgfscope}%
\begin{pgfscope}%
\pgfsys@transformshift{7.550979in}{6.155900in}%
\pgfsys@useobject{currentmarker}{}%
\end{pgfscope}%
\begin{pgfscope}%
\pgfsys@transformshift{7.730710in}{6.217078in}%
\pgfsys@useobject{currentmarker}{}%
\end{pgfscope}%
\begin{pgfscope}%
\pgfsys@transformshift{7.910441in}{6.219506in}%
\pgfsys@useobject{currentmarker}{}%
\end{pgfscope}%
\begin{pgfscope}%
\pgfsys@transformshift{8.090172in}{6.263967in}%
\pgfsys@useobject{currentmarker}{}%
\end{pgfscope}%
\begin{pgfscope}%
\pgfsys@transformshift{8.269903in}{6.319388in}%
\pgfsys@useobject{currentmarker}{}%
\end{pgfscope}%
\begin{pgfscope}%
\pgfsys@transformshift{8.449634in}{6.361304in}%
\pgfsys@useobject{currentmarker}{}%
\end{pgfscope}%
\begin{pgfscope}%
\pgfsys@transformshift{8.629365in}{6.346872in}%
\pgfsys@useobject{currentmarker}{}%
\end{pgfscope}%
\begin{pgfscope}%
\pgfsys@transformshift{8.809096in}{6.368538in}%
\pgfsys@useobject{currentmarker}{}%
\end{pgfscope}%
\begin{pgfscope}%
\pgfsys@transformshift{8.988827in}{6.367929in}%
\pgfsys@useobject{currentmarker}{}%
\end{pgfscope}%
\begin{pgfscope}%
\pgfsys@transformshift{9.168558in}{6.419861in}%
\pgfsys@useobject{currentmarker}{}%
\end{pgfscope}%
\begin{pgfscope}%
\pgfsys@transformshift{9.348289in}{6.462018in}%
\pgfsys@useobject{currentmarker}{}%
\end{pgfscope}%
\begin{pgfscope}%
\pgfsys@transformshift{9.528020in}{6.480647in}%
\pgfsys@useobject{currentmarker}{}%
\end{pgfscope}%
\begin{pgfscope}%
\pgfsys@transformshift{9.707751in}{6.471674in}%
\pgfsys@useobject{currentmarker}{}%
\end{pgfscope}%
\begin{pgfscope}%
\pgfsys@transformshift{9.887482in}{6.488943in}%
\pgfsys@useobject{currentmarker}{}%
\end{pgfscope}%
\end{pgfscope}%
\begin{pgfscope}%
\pgfpathrectangle{\pgfqpoint{0.640323in}{5.147436in}}{\pgfqpoint{9.687500in}{3.850000in}}%
\pgfusepath{clip}%
\pgfsetbuttcap%
\pgfsetroundjoin%
\definecolor{currentfill}{rgb}{0.980392,0.164706,0.333333}%
\pgfsetfillcolor{currentfill}%
\pgfsetfillopacity{0.500000}%
\pgfsetlinewidth{1.003750pt}%
\definecolor{currentstroke}{rgb}{0.980392,0.164706,0.333333}%
\pgfsetstrokecolor{currentstroke}%
\pgfsetstrokeopacity{0.500000}%
\pgfsetdash{}{0pt}%
\pgfsys@defobject{currentmarker}{\pgfqpoint{-0.021960in}{-0.021960in}}{\pgfqpoint{0.021960in}{0.021960in}}{%
\pgfpathmoveto{\pgfqpoint{0.000000in}{-0.021960in}}%
\pgfpathcurveto{\pgfqpoint{0.005824in}{-0.021960in}}{\pgfqpoint{0.011410in}{-0.019646in}}{\pgfqpoint{0.015528in}{-0.015528in}}%
\pgfpathcurveto{\pgfqpoint{0.019646in}{-0.011410in}}{\pgfqpoint{0.021960in}{-0.005824in}}{\pgfqpoint{0.021960in}{0.000000in}}%
\pgfpathcurveto{\pgfqpoint{0.021960in}{0.005824in}}{\pgfqpoint{0.019646in}{0.011410in}}{\pgfqpoint{0.015528in}{0.015528in}}%
\pgfpathcurveto{\pgfqpoint{0.011410in}{0.019646in}}{\pgfqpoint{0.005824in}{0.021960in}}{\pgfqpoint{0.000000in}{0.021960in}}%
\pgfpathcurveto{\pgfqpoint{-0.005824in}{0.021960in}}{\pgfqpoint{-0.011410in}{0.019646in}}{\pgfqpoint{-0.015528in}{0.015528in}}%
\pgfpathcurveto{\pgfqpoint{-0.019646in}{0.011410in}}{\pgfqpoint{-0.021960in}{0.005824in}}{\pgfqpoint{-0.021960in}{0.000000in}}%
\pgfpathcurveto{\pgfqpoint{-0.021960in}{-0.005824in}}{\pgfqpoint{-0.019646in}{-0.011410in}}{\pgfqpoint{-0.015528in}{-0.015528in}}%
\pgfpathcurveto{\pgfqpoint{-0.011410in}{-0.019646in}}{\pgfqpoint{-0.005824in}{-0.021960in}}{\pgfqpoint{0.000000in}{-0.021960in}}%
\pgfpathlineto{\pgfqpoint{0.000000in}{-0.021960in}}%
\pgfpathclose%
\pgfusepath{stroke,fill}%
}%
\begin{pgfscope}%
\pgfsys@transformshift{1.080663in}{5.276705in}%
\pgfsys@useobject{currentmarker}{}%
\end{pgfscope}%
\begin{pgfscope}%
\pgfsys@transformshift{1.260394in}{5.276469in}%
\pgfsys@useobject{currentmarker}{}%
\end{pgfscope}%
\begin{pgfscope}%
\pgfsys@transformshift{1.440125in}{5.276523in}%
\pgfsys@useobject{currentmarker}{}%
\end{pgfscope}%
\begin{pgfscope}%
\pgfsys@transformshift{1.619856in}{5.276893in}%
\pgfsys@useobject{currentmarker}{}%
\end{pgfscope}%
\begin{pgfscope}%
\pgfsys@transformshift{1.799587in}{5.277433in}%
\pgfsys@useobject{currentmarker}{}%
\end{pgfscope}%
\begin{pgfscope}%
\pgfsys@transformshift{1.979318in}{5.277633in}%
\pgfsys@useobject{currentmarker}{}%
\end{pgfscope}%
\begin{pgfscope}%
\pgfsys@transformshift{2.159049in}{5.277524in}%
\pgfsys@useobject{currentmarker}{}%
\end{pgfscope}%
\begin{pgfscope}%
\pgfsys@transformshift{2.338780in}{5.277809in}%
\pgfsys@useobject{currentmarker}{}%
\end{pgfscope}%
\begin{pgfscope}%
\pgfsys@transformshift{2.518511in}{5.277851in}%
\pgfsys@useobject{currentmarker}{}%
\end{pgfscope}%
\begin{pgfscope}%
\pgfsys@transformshift{2.698242in}{5.278730in}%
\pgfsys@useobject{currentmarker}{}%
\end{pgfscope}%
\begin{pgfscope}%
\pgfsys@transformshift{2.877973in}{5.278785in}%
\pgfsys@useobject{currentmarker}{}%
\end{pgfscope}%
\begin{pgfscope}%
\pgfsys@transformshift{3.057704in}{5.278542in}%
\pgfsys@useobject{currentmarker}{}%
\end{pgfscope}%
\begin{pgfscope}%
\pgfsys@transformshift{3.237435in}{5.280980in}%
\pgfsys@useobject{currentmarker}{}%
\end{pgfscope}%
\begin{pgfscope}%
\pgfsys@transformshift{3.417166in}{5.281532in}%
\pgfsys@useobject{currentmarker}{}%
\end{pgfscope}%
\begin{pgfscope}%
\pgfsys@transformshift{3.596897in}{5.279246in}%
\pgfsys@useobject{currentmarker}{}%
\end{pgfscope}%
\begin{pgfscope}%
\pgfsys@transformshift{3.776628in}{5.284643in}%
\pgfsys@useobject{currentmarker}{}%
\end{pgfscope}%
\begin{pgfscope}%
\pgfsys@transformshift{3.956359in}{5.280544in}%
\pgfsys@useobject{currentmarker}{}%
\end{pgfscope}%
\begin{pgfscope}%
\pgfsys@transformshift{4.136090in}{5.287057in}%
\pgfsys@useobject{currentmarker}{}%
\end{pgfscope}%
\begin{pgfscope}%
\pgfsys@transformshift{4.315821in}{5.283927in}%
\pgfsys@useobject{currentmarker}{}%
\end{pgfscope}%
\begin{pgfscope}%
\pgfsys@transformshift{4.495552in}{5.285977in}%
\pgfsys@useobject{currentmarker}{}%
\end{pgfscope}%
\begin{pgfscope}%
\pgfsys@transformshift{4.675283in}{5.298706in}%
\pgfsys@useobject{currentmarker}{}%
\end{pgfscope}%
\begin{pgfscope}%
\pgfsys@transformshift{4.855014in}{5.304655in}%
\pgfsys@useobject{currentmarker}{}%
\end{pgfscope}%
\begin{pgfscope}%
\pgfsys@transformshift{5.034745in}{5.314369in}%
\pgfsys@useobject{currentmarker}{}%
\end{pgfscope}%
\begin{pgfscope}%
\pgfsys@transformshift{5.214476in}{5.376624in}%
\pgfsys@useobject{currentmarker}{}%
\end{pgfscope}%
\begin{pgfscope}%
\pgfsys@transformshift{5.394207in}{5.585012in}%
\pgfsys@useobject{currentmarker}{}%
\end{pgfscope}%
\begin{pgfscope}%
\pgfsys@transformshift{5.573938in}{5.684949in}%
\pgfsys@useobject{currentmarker}{}%
\end{pgfscope}%
\begin{pgfscope}%
\pgfsys@transformshift{5.753669in}{5.807293in}%
\pgfsys@useobject{currentmarker}{}%
\end{pgfscope}%
\begin{pgfscope}%
\pgfsys@transformshift{5.933400in}{5.842271in}%
\pgfsys@useobject{currentmarker}{}%
\end{pgfscope}%
\begin{pgfscope}%
\pgfsys@transformshift{6.113131in}{5.905811in}%
\pgfsys@useobject{currentmarker}{}%
\end{pgfscope}%
\begin{pgfscope}%
\pgfsys@transformshift{6.292862in}{6.021479in}%
\pgfsys@useobject{currentmarker}{}%
\end{pgfscope}%
\begin{pgfscope}%
\pgfsys@transformshift{6.472593in}{6.038841in}%
\pgfsys@useobject{currentmarker}{}%
\end{pgfscope}%
\begin{pgfscope}%
\pgfsys@transformshift{6.652324in}{6.079912in}%
\pgfsys@useobject{currentmarker}{}%
\end{pgfscope}%
\begin{pgfscope}%
\pgfsys@transformshift{6.832055in}{6.124516in}%
\pgfsys@useobject{currentmarker}{}%
\end{pgfscope}%
\begin{pgfscope}%
\pgfsys@transformshift{7.011786in}{6.162743in}%
\pgfsys@useobject{currentmarker}{}%
\end{pgfscope}%
\begin{pgfscope}%
\pgfsys@transformshift{7.191517in}{6.221120in}%
\pgfsys@useobject{currentmarker}{}%
\end{pgfscope}%
\begin{pgfscope}%
\pgfsys@transformshift{7.371248in}{6.276051in}%
\pgfsys@useobject{currentmarker}{}%
\end{pgfscope}%
\begin{pgfscope}%
\pgfsys@transformshift{7.550979in}{6.268382in}%
\pgfsys@useobject{currentmarker}{}%
\end{pgfscope}%
\begin{pgfscope}%
\pgfsys@transformshift{7.730710in}{6.312328in}%
\pgfsys@useobject{currentmarker}{}%
\end{pgfscope}%
\begin{pgfscope}%
\pgfsys@transformshift{7.910441in}{6.350170in}%
\pgfsys@useobject{currentmarker}{}%
\end{pgfscope}%
\begin{pgfscope}%
\pgfsys@transformshift{8.090172in}{6.373388in}%
\pgfsys@useobject{currentmarker}{}%
\end{pgfscope}%
\begin{pgfscope}%
\pgfsys@transformshift{8.269903in}{6.400598in}%
\pgfsys@useobject{currentmarker}{}%
\end{pgfscope}%
\begin{pgfscope}%
\pgfsys@transformshift{8.449634in}{6.409627in}%
\pgfsys@useobject{currentmarker}{}%
\end{pgfscope}%
\begin{pgfscope}%
\pgfsys@transformshift{8.629365in}{6.434385in}%
\pgfsys@useobject{currentmarker}{}%
\end{pgfscope}%
\begin{pgfscope}%
\pgfsys@transformshift{8.809096in}{6.445445in}%
\pgfsys@useobject{currentmarker}{}%
\end{pgfscope}%
\begin{pgfscope}%
\pgfsys@transformshift{8.988827in}{6.475251in}%
\pgfsys@useobject{currentmarker}{}%
\end{pgfscope}%
\begin{pgfscope}%
\pgfsys@transformshift{9.168558in}{6.490061in}%
\pgfsys@useobject{currentmarker}{}%
\end{pgfscope}%
\begin{pgfscope}%
\pgfsys@transformshift{9.348289in}{6.506995in}%
\pgfsys@useobject{currentmarker}{}%
\end{pgfscope}%
\begin{pgfscope}%
\pgfsys@transformshift{9.528020in}{6.520184in}%
\pgfsys@useobject{currentmarker}{}%
\end{pgfscope}%
\begin{pgfscope}%
\pgfsys@transformshift{9.707751in}{6.544303in}%
\pgfsys@useobject{currentmarker}{}%
\end{pgfscope}%
\begin{pgfscope}%
\pgfsys@transformshift{9.887482in}{6.554959in}%
\pgfsys@useobject{currentmarker}{}%
\end{pgfscope}%
\end{pgfscope}%
\begin{pgfscope}%
\pgfpathrectangle{\pgfqpoint{0.640323in}{5.147436in}}{\pgfqpoint{9.687500in}{3.850000in}}%
\pgfusepath{clip}%
\pgfsetbuttcap%
\pgfsetroundjoin%
\definecolor{currentfill}{rgb}{0.239216,0.478431,0.992157}%
\pgfsetfillcolor{currentfill}%
\pgfsetfillopacity{0.500000}%
\pgfsetlinewidth{1.003750pt}%
\definecolor{currentstroke}{rgb}{0.239216,0.478431,0.992157}%
\pgfsetstrokecolor{currentstroke}%
\pgfsetstrokeopacity{0.500000}%
\pgfsetdash{}{0pt}%
\pgfsys@defobject{currentmarker}{\pgfqpoint{-0.021960in}{-0.021960in}}{\pgfqpoint{0.021960in}{0.021960in}}{%
\pgfpathmoveto{\pgfqpoint{0.000000in}{-0.021960in}}%
\pgfpathcurveto{\pgfqpoint{0.005824in}{-0.021960in}}{\pgfqpoint{0.011410in}{-0.019646in}}{\pgfqpoint{0.015528in}{-0.015528in}}%
\pgfpathcurveto{\pgfqpoint{0.019646in}{-0.011410in}}{\pgfqpoint{0.021960in}{-0.005824in}}{\pgfqpoint{0.021960in}{0.000000in}}%
\pgfpathcurveto{\pgfqpoint{0.021960in}{0.005824in}}{\pgfqpoint{0.019646in}{0.011410in}}{\pgfqpoint{0.015528in}{0.015528in}}%
\pgfpathcurveto{\pgfqpoint{0.011410in}{0.019646in}}{\pgfqpoint{0.005824in}{0.021960in}}{\pgfqpoint{0.000000in}{0.021960in}}%
\pgfpathcurveto{\pgfqpoint{-0.005824in}{0.021960in}}{\pgfqpoint{-0.011410in}{0.019646in}}{\pgfqpoint{-0.015528in}{0.015528in}}%
\pgfpathcurveto{\pgfqpoint{-0.019646in}{0.011410in}}{\pgfqpoint{-0.021960in}{0.005824in}}{\pgfqpoint{-0.021960in}{0.000000in}}%
\pgfpathcurveto{\pgfqpoint{-0.021960in}{-0.005824in}}{\pgfqpoint{-0.019646in}{-0.011410in}}{\pgfqpoint{-0.015528in}{-0.015528in}}%
\pgfpathcurveto{\pgfqpoint{-0.011410in}{-0.019646in}}{\pgfqpoint{-0.005824in}{-0.021960in}}{\pgfqpoint{0.000000in}{-0.021960in}}%
\pgfpathlineto{\pgfqpoint{0.000000in}{-0.021960in}}%
\pgfpathclose%
\pgfusepath{stroke,fill}%
}%
\begin{pgfscope}%
\pgfsys@transformshift{1.080663in}{5.276250in}%
\pgfsys@useobject{currentmarker}{}%
\end{pgfscope}%
\begin{pgfscope}%
\pgfsys@transformshift{1.260394in}{5.276256in}%
\pgfsys@useobject{currentmarker}{}%
\end{pgfscope}%
\begin{pgfscope}%
\pgfsys@transformshift{1.440125in}{5.276743in}%
\pgfsys@useobject{currentmarker}{}%
\end{pgfscope}%
\begin{pgfscope}%
\pgfsys@transformshift{1.619856in}{5.276764in}%
\pgfsys@useobject{currentmarker}{}%
\end{pgfscope}%
\begin{pgfscope}%
\pgfsys@transformshift{1.799587in}{5.277131in}%
\pgfsys@useobject{currentmarker}{}%
\end{pgfscope}%
\begin{pgfscope}%
\pgfsys@transformshift{1.979318in}{5.277202in}%
\pgfsys@useobject{currentmarker}{}%
\end{pgfscope}%
\begin{pgfscope}%
\pgfsys@transformshift{2.159049in}{5.277747in}%
\pgfsys@useobject{currentmarker}{}%
\end{pgfscope}%
\begin{pgfscope}%
\pgfsys@transformshift{2.338780in}{5.277872in}%
\pgfsys@useobject{currentmarker}{}%
\end{pgfscope}%
\begin{pgfscope}%
\pgfsys@transformshift{2.518511in}{5.278391in}%
\pgfsys@useobject{currentmarker}{}%
\end{pgfscope}%
\begin{pgfscope}%
\pgfsys@transformshift{2.698242in}{5.278438in}%
\pgfsys@useobject{currentmarker}{}%
\end{pgfscope}%
\begin{pgfscope}%
\pgfsys@transformshift{2.877973in}{5.279081in}%
\pgfsys@useobject{currentmarker}{}%
\end{pgfscope}%
\begin{pgfscope}%
\pgfsys@transformshift{3.057704in}{5.279279in}%
\pgfsys@useobject{currentmarker}{}%
\end{pgfscope}%
\begin{pgfscope}%
\pgfsys@transformshift{3.237435in}{5.279899in}%
\pgfsys@useobject{currentmarker}{}%
\end{pgfscope}%
\begin{pgfscope}%
\pgfsys@transformshift{3.417166in}{5.281779in}%
\pgfsys@useobject{currentmarker}{}%
\end{pgfscope}%
\begin{pgfscope}%
\pgfsys@transformshift{3.596897in}{5.282287in}%
\pgfsys@useobject{currentmarker}{}%
\end{pgfscope}%
\begin{pgfscope}%
\pgfsys@transformshift{3.776628in}{5.282116in}%
\pgfsys@useobject{currentmarker}{}%
\end{pgfscope}%
\begin{pgfscope}%
\pgfsys@transformshift{3.956359in}{5.283697in}%
\pgfsys@useobject{currentmarker}{}%
\end{pgfscope}%
\begin{pgfscope}%
\pgfsys@transformshift{4.136090in}{5.284625in}%
\pgfsys@useobject{currentmarker}{}%
\end{pgfscope}%
\begin{pgfscope}%
\pgfsys@transformshift{4.315821in}{5.287903in}%
\pgfsys@useobject{currentmarker}{}%
\end{pgfscope}%
\begin{pgfscope}%
\pgfsys@transformshift{4.495552in}{5.292931in}%
\pgfsys@useobject{currentmarker}{}%
\end{pgfscope}%
\begin{pgfscope}%
\pgfsys@transformshift{4.675283in}{5.300599in}%
\pgfsys@useobject{currentmarker}{}%
\end{pgfscope}%
\begin{pgfscope}%
\pgfsys@transformshift{4.855014in}{5.322623in}%
\pgfsys@useobject{currentmarker}{}%
\end{pgfscope}%
\begin{pgfscope}%
\pgfsys@transformshift{5.034745in}{5.420359in}%
\pgfsys@useobject{currentmarker}{}%
\end{pgfscope}%
\begin{pgfscope}%
\pgfsys@transformshift{5.214476in}{5.518179in}%
\pgfsys@useobject{currentmarker}{}%
\end{pgfscope}%
\begin{pgfscope}%
\pgfsys@transformshift{5.394207in}{5.674770in}%
\pgfsys@useobject{currentmarker}{}%
\end{pgfscope}%
\begin{pgfscope}%
\pgfsys@transformshift{5.573938in}{5.769451in}%
\pgfsys@useobject{currentmarker}{}%
\end{pgfscope}%
\begin{pgfscope}%
\pgfsys@transformshift{5.753669in}{5.871621in}%
\pgfsys@useobject{currentmarker}{}%
\end{pgfscope}%
\begin{pgfscope}%
\pgfsys@transformshift{5.933400in}{5.959183in}%
\pgfsys@useobject{currentmarker}{}%
\end{pgfscope}%
\begin{pgfscope}%
\pgfsys@transformshift{6.113131in}{6.000292in}%
\pgfsys@useobject{currentmarker}{}%
\end{pgfscope}%
\begin{pgfscope}%
\pgfsys@transformshift{6.292862in}{6.066928in}%
\pgfsys@useobject{currentmarker}{}%
\end{pgfscope}%
\begin{pgfscope}%
\pgfsys@transformshift{6.472593in}{6.103111in}%
\pgfsys@useobject{currentmarker}{}%
\end{pgfscope}%
\begin{pgfscope}%
\pgfsys@transformshift{6.652324in}{6.141369in}%
\pgfsys@useobject{currentmarker}{}%
\end{pgfscope}%
\begin{pgfscope}%
\pgfsys@transformshift{6.832055in}{6.191748in}%
\pgfsys@useobject{currentmarker}{}%
\end{pgfscope}%
\begin{pgfscope}%
\pgfsys@transformshift{7.011786in}{6.245009in}%
\pgfsys@useobject{currentmarker}{}%
\end{pgfscope}%
\begin{pgfscope}%
\pgfsys@transformshift{7.191517in}{6.266047in}%
\pgfsys@useobject{currentmarker}{}%
\end{pgfscope}%
\begin{pgfscope}%
\pgfsys@transformshift{7.371248in}{6.301194in}%
\pgfsys@useobject{currentmarker}{}%
\end{pgfscope}%
\begin{pgfscope}%
\pgfsys@transformshift{7.550979in}{6.332093in}%
\pgfsys@useobject{currentmarker}{}%
\end{pgfscope}%
\begin{pgfscope}%
\pgfsys@transformshift{7.730710in}{6.362396in}%
\pgfsys@useobject{currentmarker}{}%
\end{pgfscope}%
\begin{pgfscope}%
\pgfsys@transformshift{7.910441in}{6.373996in}%
\pgfsys@useobject{currentmarker}{}%
\end{pgfscope}%
\begin{pgfscope}%
\pgfsys@transformshift{8.090172in}{6.410559in}%
\pgfsys@useobject{currentmarker}{}%
\end{pgfscope}%
\begin{pgfscope}%
\pgfsys@transformshift{8.269903in}{6.434006in}%
\pgfsys@useobject{currentmarker}{}%
\end{pgfscope}%
\begin{pgfscope}%
\pgfsys@transformshift{8.449634in}{6.460640in}%
\pgfsys@useobject{currentmarker}{}%
\end{pgfscope}%
\begin{pgfscope}%
\pgfsys@transformshift{8.629365in}{6.481896in}%
\pgfsys@useobject{currentmarker}{}%
\end{pgfscope}%
\begin{pgfscope}%
\pgfsys@transformshift{8.809096in}{6.502561in}%
\pgfsys@useobject{currentmarker}{}%
\end{pgfscope}%
\begin{pgfscope}%
\pgfsys@transformshift{8.988827in}{6.510205in}%
\pgfsys@useobject{currentmarker}{}%
\end{pgfscope}%
\begin{pgfscope}%
\pgfsys@transformshift{9.168558in}{6.531380in}%
\pgfsys@useobject{currentmarker}{}%
\end{pgfscope}%
\begin{pgfscope}%
\pgfsys@transformshift{9.348289in}{6.541744in}%
\pgfsys@useobject{currentmarker}{}%
\end{pgfscope}%
\begin{pgfscope}%
\pgfsys@transformshift{9.528020in}{6.555753in}%
\pgfsys@useobject{currentmarker}{}%
\end{pgfscope}%
\begin{pgfscope}%
\pgfsys@transformshift{9.707751in}{6.575891in}%
\pgfsys@useobject{currentmarker}{}%
\end{pgfscope}%
\begin{pgfscope}%
\pgfsys@transformshift{9.887482in}{6.591403in}%
\pgfsys@useobject{currentmarker}{}%
\end{pgfscope}%
\end{pgfscope}%
\begin{pgfscope}%
\pgfpathrectangle{\pgfqpoint{0.640323in}{5.147436in}}{\pgfqpoint{9.687500in}{3.850000in}}%
\pgfusepath{clip}%
\pgfsetbuttcap%
\pgfsetroundjoin%
\definecolor{currentfill}{rgb}{0.000000,0.000000,0.000000}%
\pgfsetfillcolor{currentfill}%
\pgfsetfillopacity{0.500000}%
\pgfsetlinewidth{1.003750pt}%
\definecolor{currentstroke}{rgb}{0.000000,0.000000,0.000000}%
\pgfsetstrokecolor{currentstroke}%
\pgfsetstrokeopacity{0.500000}%
\pgfsetdash{}{0pt}%
\pgfsys@defobject{currentmarker}{\pgfqpoint{-0.021960in}{-0.021960in}}{\pgfqpoint{0.021960in}{0.021960in}}{%
\pgfpathmoveto{\pgfqpoint{0.000000in}{-0.021960in}}%
\pgfpathcurveto{\pgfqpoint{0.005824in}{-0.021960in}}{\pgfqpoint{0.011410in}{-0.019646in}}{\pgfqpoint{0.015528in}{-0.015528in}}%
\pgfpathcurveto{\pgfqpoint{0.019646in}{-0.011410in}}{\pgfqpoint{0.021960in}{-0.005824in}}{\pgfqpoint{0.021960in}{0.000000in}}%
\pgfpathcurveto{\pgfqpoint{0.021960in}{0.005824in}}{\pgfqpoint{0.019646in}{0.011410in}}{\pgfqpoint{0.015528in}{0.015528in}}%
\pgfpathcurveto{\pgfqpoint{0.011410in}{0.019646in}}{\pgfqpoint{0.005824in}{0.021960in}}{\pgfqpoint{0.000000in}{0.021960in}}%
\pgfpathcurveto{\pgfqpoint{-0.005824in}{0.021960in}}{\pgfqpoint{-0.011410in}{0.019646in}}{\pgfqpoint{-0.015528in}{0.015528in}}%
\pgfpathcurveto{\pgfqpoint{-0.019646in}{0.011410in}}{\pgfqpoint{-0.021960in}{0.005824in}}{\pgfqpoint{-0.021960in}{0.000000in}}%
\pgfpathcurveto{\pgfqpoint{-0.021960in}{-0.005824in}}{\pgfqpoint{-0.019646in}{-0.011410in}}{\pgfqpoint{-0.015528in}{-0.015528in}}%
\pgfpathcurveto{\pgfqpoint{-0.011410in}{-0.019646in}}{\pgfqpoint{-0.005824in}{-0.021960in}}{\pgfqpoint{0.000000in}{-0.021960in}}%
\pgfpathlineto{\pgfqpoint{0.000000in}{-0.021960in}}%
\pgfpathclose%
\pgfusepath{stroke,fill}%
}%
\begin{pgfscope}%
\pgfsys@transformshift{1.080663in}{5.276317in}%
\pgfsys@useobject{currentmarker}{}%
\end{pgfscope}%
\begin{pgfscope}%
\pgfsys@transformshift{1.260394in}{5.276409in}%
\pgfsys@useobject{currentmarker}{}%
\end{pgfscope}%
\begin{pgfscope}%
\pgfsys@transformshift{1.440125in}{5.276751in}%
\pgfsys@useobject{currentmarker}{}%
\end{pgfscope}%
\begin{pgfscope}%
\pgfsys@transformshift{1.619856in}{5.276889in}%
\pgfsys@useobject{currentmarker}{}%
\end{pgfscope}%
\begin{pgfscope}%
\pgfsys@transformshift{1.799587in}{5.277083in}%
\pgfsys@useobject{currentmarker}{}%
\end{pgfscope}%
\begin{pgfscope}%
\pgfsys@transformshift{1.979318in}{5.277261in}%
\pgfsys@useobject{currentmarker}{}%
\end{pgfscope}%
\begin{pgfscope}%
\pgfsys@transformshift{2.159049in}{5.277601in}%
\pgfsys@useobject{currentmarker}{}%
\end{pgfscope}%
\begin{pgfscope}%
\pgfsys@transformshift{2.338780in}{5.277905in}%
\pgfsys@useobject{currentmarker}{}%
\end{pgfscope}%
\begin{pgfscope}%
\pgfsys@transformshift{2.518511in}{5.278229in}%
\pgfsys@useobject{currentmarker}{}%
\end{pgfscope}%
\begin{pgfscope}%
\pgfsys@transformshift{2.698242in}{5.278620in}%
\pgfsys@useobject{currentmarker}{}%
\end{pgfscope}%
\begin{pgfscope}%
\pgfsys@transformshift{2.877973in}{5.278832in}%
\pgfsys@useobject{currentmarker}{}%
\end{pgfscope}%
\begin{pgfscope}%
\pgfsys@transformshift{3.057704in}{5.279708in}%
\pgfsys@useobject{currentmarker}{}%
\end{pgfscope}%
\begin{pgfscope}%
\pgfsys@transformshift{3.237435in}{5.280171in}%
\pgfsys@useobject{currentmarker}{}%
\end{pgfscope}%
\begin{pgfscope}%
\pgfsys@transformshift{3.417166in}{5.280771in}%
\pgfsys@useobject{currentmarker}{}%
\end{pgfscope}%
\begin{pgfscope}%
\pgfsys@transformshift{3.596897in}{5.282038in}%
\pgfsys@useobject{currentmarker}{}%
\end{pgfscope}%
\begin{pgfscope}%
\pgfsys@transformshift{3.776628in}{5.282917in}%
\pgfsys@useobject{currentmarker}{}%
\end{pgfscope}%
\begin{pgfscope}%
\pgfsys@transformshift{3.956359in}{5.283727in}%
\pgfsys@useobject{currentmarker}{}%
\end{pgfscope}%
\begin{pgfscope}%
\pgfsys@transformshift{4.136090in}{5.285470in}%
\pgfsys@useobject{currentmarker}{}%
\end{pgfscope}%
\begin{pgfscope}%
\pgfsys@transformshift{4.315821in}{5.288475in}%
\pgfsys@useobject{currentmarker}{}%
\end{pgfscope}%
\begin{pgfscope}%
\pgfsys@transformshift{4.495552in}{5.291315in}%
\pgfsys@useobject{currentmarker}{}%
\end{pgfscope}%
\begin{pgfscope}%
\pgfsys@transformshift{4.675283in}{5.296769in}%
\pgfsys@useobject{currentmarker}{}%
\end{pgfscope}%
\begin{pgfscope}%
\pgfsys@transformshift{4.855014in}{5.323733in}%
\pgfsys@useobject{currentmarker}{}%
\end{pgfscope}%
\begin{pgfscope}%
\pgfsys@transformshift{5.034745in}{5.428259in}%
\pgfsys@useobject{currentmarker}{}%
\end{pgfscope}%
\begin{pgfscope}%
\pgfsys@transformshift{5.214476in}{5.593213in}%
\pgfsys@useobject{currentmarker}{}%
\end{pgfscope}%
\begin{pgfscope}%
\pgfsys@transformshift{5.394207in}{5.708419in}%
\pgfsys@useobject{currentmarker}{}%
\end{pgfscope}%
\begin{pgfscope}%
\pgfsys@transformshift{5.573938in}{5.825315in}%
\pgfsys@useobject{currentmarker}{}%
\end{pgfscope}%
\begin{pgfscope}%
\pgfsys@transformshift{5.753669in}{5.898707in}%
\pgfsys@useobject{currentmarker}{}%
\end{pgfscope}%
\begin{pgfscope}%
\pgfsys@transformshift{5.933400in}{5.968647in}%
\pgfsys@useobject{currentmarker}{}%
\end{pgfscope}%
\begin{pgfscope}%
\pgfsys@transformshift{6.113131in}{6.037630in}%
\pgfsys@useobject{currentmarker}{}%
\end{pgfscope}%
\begin{pgfscope}%
\pgfsys@transformshift{6.292862in}{6.083787in}%
\pgfsys@useobject{currentmarker}{}%
\end{pgfscope}%
\begin{pgfscope}%
\pgfsys@transformshift{6.472593in}{6.138016in}%
\pgfsys@useobject{currentmarker}{}%
\end{pgfscope}%
\begin{pgfscope}%
\pgfsys@transformshift{6.652324in}{6.177137in}%
\pgfsys@useobject{currentmarker}{}%
\end{pgfscope}%
\begin{pgfscope}%
\pgfsys@transformshift{6.832055in}{6.222561in}%
\pgfsys@useobject{currentmarker}{}%
\end{pgfscope}%
\begin{pgfscope}%
\pgfsys@transformshift{7.011786in}{6.256372in}%
\pgfsys@useobject{currentmarker}{}%
\end{pgfscope}%
\begin{pgfscope}%
\pgfsys@transformshift{7.191517in}{6.291171in}%
\pgfsys@useobject{currentmarker}{}%
\end{pgfscope}%
\begin{pgfscope}%
\pgfsys@transformshift{7.371248in}{6.327728in}%
\pgfsys@useobject{currentmarker}{}%
\end{pgfscope}%
\begin{pgfscope}%
\pgfsys@transformshift{7.550979in}{6.351095in}%
\pgfsys@useobject{currentmarker}{}%
\end{pgfscope}%
\begin{pgfscope}%
\pgfsys@transformshift{7.730710in}{6.377424in}%
\pgfsys@useobject{currentmarker}{}%
\end{pgfscope}%
\begin{pgfscope}%
\pgfsys@transformshift{7.910441in}{6.405759in}%
\pgfsys@useobject{currentmarker}{}%
\end{pgfscope}%
\begin{pgfscope}%
\pgfsys@transformshift{8.090172in}{6.427561in}%
\pgfsys@useobject{currentmarker}{}%
\end{pgfscope}%
\begin{pgfscope}%
\pgfsys@transformshift{8.269903in}{6.450990in}%
\pgfsys@useobject{currentmarker}{}%
\end{pgfscope}%
\begin{pgfscope}%
\pgfsys@transformshift{8.449634in}{6.472084in}%
\pgfsys@useobject{currentmarker}{}%
\end{pgfscope}%
\begin{pgfscope}%
\pgfsys@transformshift{8.629365in}{6.491545in}%
\pgfsys@useobject{currentmarker}{}%
\end{pgfscope}%
\begin{pgfscope}%
\pgfsys@transformshift{8.809096in}{6.511776in}%
\pgfsys@useobject{currentmarker}{}%
\end{pgfscope}%
\begin{pgfscope}%
\pgfsys@transformshift{8.988827in}{6.530474in}%
\pgfsys@useobject{currentmarker}{}%
\end{pgfscope}%
\begin{pgfscope}%
\pgfsys@transformshift{9.168558in}{6.545656in}%
\pgfsys@useobject{currentmarker}{}%
\end{pgfscope}%
\begin{pgfscope}%
\pgfsys@transformshift{9.348289in}{6.564708in}%
\pgfsys@useobject{currentmarker}{}%
\end{pgfscope}%
\begin{pgfscope}%
\pgfsys@transformshift{9.528020in}{6.576997in}%
\pgfsys@useobject{currentmarker}{}%
\end{pgfscope}%
\begin{pgfscope}%
\pgfsys@transformshift{9.707751in}{6.591416in}%
\pgfsys@useobject{currentmarker}{}%
\end{pgfscope}%
\begin{pgfscope}%
\pgfsys@transformshift{9.887482in}{6.607548in}%
\pgfsys@useobject{currentmarker}{}%
\end{pgfscope}%
\end{pgfscope}%
\begin{pgfscope}%
\pgfpathrectangle{\pgfqpoint{0.640323in}{5.147436in}}{\pgfqpoint{9.687500in}{3.850000in}}%
\pgfusepath{clip}%
\pgfsetrectcap%
\pgfsetroundjoin%
\pgfsetlinewidth{0.803000pt}%
\definecolor{currentstroke}{rgb}{0.690196,0.690196,0.690196}%
\pgfsetstrokecolor{currentstroke}%
\pgfsetdash{}{0pt}%
\pgfpathmoveto{\pgfqpoint{1.080663in}{5.147436in}}%
\pgfpathlineto{\pgfqpoint{1.080663in}{8.997436in}}%
\pgfusepath{stroke}%
\end{pgfscope}%
\begin{pgfscope}%
\pgfsetbuttcap%
\pgfsetroundjoin%
\definecolor{currentfill}{rgb}{0.000000,0.000000,0.000000}%
\pgfsetfillcolor{currentfill}%
\pgfsetlinewidth{0.803000pt}%
\definecolor{currentstroke}{rgb}{0.000000,0.000000,0.000000}%
\pgfsetstrokecolor{currentstroke}%
\pgfsetdash{}{0pt}%
\pgfsys@defobject{currentmarker}{\pgfqpoint{0.000000in}{-0.048611in}}{\pgfqpoint{0.000000in}{0.000000in}}{%
\pgfpathmoveto{\pgfqpoint{0.000000in}{0.000000in}}%
\pgfpathlineto{\pgfqpoint{0.000000in}{-0.048611in}}%
\pgfusepath{stroke,fill}%
}%
\begin{pgfscope}%
\pgfsys@transformshift{1.080663in}{5.147436in}%
\pgfsys@useobject{currentmarker}{}%
\end{pgfscope}%
\end{pgfscope}%
\begin{pgfscope}%
\definecolor{textcolor}{rgb}{0.000000,0.000000,0.000000}%
\pgfsetstrokecolor{textcolor}%
\pgfsetfillcolor{textcolor}%
\pgftext[x=1.080663in,y=5.050214in,,top]{\color{textcolor}\sffamily\fontsize{10.000000}{12.000000}\selectfont 0.0}%
\end{pgfscope}%
\begin{pgfscope}%
\pgfpathrectangle{\pgfqpoint{0.640323in}{5.147436in}}{\pgfqpoint{9.687500in}{3.850000in}}%
\pgfusepath{clip}%
\pgfsetrectcap%
\pgfsetroundjoin%
\pgfsetlinewidth{0.803000pt}%
\definecolor{currentstroke}{rgb}{0.690196,0.690196,0.690196}%
\pgfsetstrokecolor{currentstroke}%
\pgfsetdash{}{0pt}%
\pgfpathmoveto{\pgfqpoint{2.877973in}{5.147436in}}%
\pgfpathlineto{\pgfqpoint{2.877973in}{8.997436in}}%
\pgfusepath{stroke}%
\end{pgfscope}%
\begin{pgfscope}%
\pgfsetbuttcap%
\pgfsetroundjoin%
\definecolor{currentfill}{rgb}{0.000000,0.000000,0.000000}%
\pgfsetfillcolor{currentfill}%
\pgfsetlinewidth{0.803000pt}%
\definecolor{currentstroke}{rgb}{0.000000,0.000000,0.000000}%
\pgfsetstrokecolor{currentstroke}%
\pgfsetdash{}{0pt}%
\pgfsys@defobject{currentmarker}{\pgfqpoint{0.000000in}{-0.048611in}}{\pgfqpoint{0.000000in}{0.000000in}}{%
\pgfpathmoveto{\pgfqpoint{0.000000in}{0.000000in}}%
\pgfpathlineto{\pgfqpoint{0.000000in}{-0.048611in}}%
\pgfusepath{stroke,fill}%
}%
\begin{pgfscope}%
\pgfsys@transformshift{2.877973in}{5.147436in}%
\pgfsys@useobject{currentmarker}{}%
\end{pgfscope}%
\end{pgfscope}%
\begin{pgfscope}%
\definecolor{textcolor}{rgb}{0.000000,0.000000,0.000000}%
\pgfsetstrokecolor{textcolor}%
\pgfsetfillcolor{textcolor}%
\pgftext[x=2.877973in,y=5.050214in,,top]{\color{textcolor}\sffamily\fontsize{10.000000}{12.000000}\selectfont 0.2}%
\end{pgfscope}%
\begin{pgfscope}%
\pgfpathrectangle{\pgfqpoint{0.640323in}{5.147436in}}{\pgfqpoint{9.687500in}{3.850000in}}%
\pgfusepath{clip}%
\pgfsetrectcap%
\pgfsetroundjoin%
\pgfsetlinewidth{0.803000pt}%
\definecolor{currentstroke}{rgb}{0.690196,0.690196,0.690196}%
\pgfsetstrokecolor{currentstroke}%
\pgfsetdash{}{0pt}%
\pgfpathmoveto{\pgfqpoint{4.675283in}{5.147436in}}%
\pgfpathlineto{\pgfqpoint{4.675283in}{8.997436in}}%
\pgfusepath{stroke}%
\end{pgfscope}%
\begin{pgfscope}%
\pgfsetbuttcap%
\pgfsetroundjoin%
\definecolor{currentfill}{rgb}{0.000000,0.000000,0.000000}%
\pgfsetfillcolor{currentfill}%
\pgfsetlinewidth{0.803000pt}%
\definecolor{currentstroke}{rgb}{0.000000,0.000000,0.000000}%
\pgfsetstrokecolor{currentstroke}%
\pgfsetdash{}{0pt}%
\pgfsys@defobject{currentmarker}{\pgfqpoint{0.000000in}{-0.048611in}}{\pgfqpoint{0.000000in}{0.000000in}}{%
\pgfpathmoveto{\pgfqpoint{0.000000in}{0.000000in}}%
\pgfpathlineto{\pgfqpoint{0.000000in}{-0.048611in}}%
\pgfusepath{stroke,fill}%
}%
\begin{pgfscope}%
\pgfsys@transformshift{4.675283in}{5.147436in}%
\pgfsys@useobject{currentmarker}{}%
\end{pgfscope}%
\end{pgfscope}%
\begin{pgfscope}%
\definecolor{textcolor}{rgb}{0.000000,0.000000,0.000000}%
\pgfsetstrokecolor{textcolor}%
\pgfsetfillcolor{textcolor}%
\pgftext[x=4.675283in,y=5.050214in,,top]{\color{textcolor}\sffamily\fontsize{10.000000}{12.000000}\selectfont 0.4}%
\end{pgfscope}%
\begin{pgfscope}%
\pgfpathrectangle{\pgfqpoint{0.640323in}{5.147436in}}{\pgfqpoint{9.687500in}{3.850000in}}%
\pgfusepath{clip}%
\pgfsetrectcap%
\pgfsetroundjoin%
\pgfsetlinewidth{0.803000pt}%
\definecolor{currentstroke}{rgb}{0.690196,0.690196,0.690196}%
\pgfsetstrokecolor{currentstroke}%
\pgfsetdash{}{0pt}%
\pgfpathmoveto{\pgfqpoint{6.472593in}{5.147436in}}%
\pgfpathlineto{\pgfqpoint{6.472593in}{8.997436in}}%
\pgfusepath{stroke}%
\end{pgfscope}%
\begin{pgfscope}%
\pgfsetbuttcap%
\pgfsetroundjoin%
\definecolor{currentfill}{rgb}{0.000000,0.000000,0.000000}%
\pgfsetfillcolor{currentfill}%
\pgfsetlinewidth{0.803000pt}%
\definecolor{currentstroke}{rgb}{0.000000,0.000000,0.000000}%
\pgfsetstrokecolor{currentstroke}%
\pgfsetdash{}{0pt}%
\pgfsys@defobject{currentmarker}{\pgfqpoint{0.000000in}{-0.048611in}}{\pgfqpoint{0.000000in}{0.000000in}}{%
\pgfpathmoveto{\pgfqpoint{0.000000in}{0.000000in}}%
\pgfpathlineto{\pgfqpoint{0.000000in}{-0.048611in}}%
\pgfusepath{stroke,fill}%
}%
\begin{pgfscope}%
\pgfsys@transformshift{6.472593in}{5.147436in}%
\pgfsys@useobject{currentmarker}{}%
\end{pgfscope}%
\end{pgfscope}%
\begin{pgfscope}%
\definecolor{textcolor}{rgb}{0.000000,0.000000,0.000000}%
\pgfsetstrokecolor{textcolor}%
\pgfsetfillcolor{textcolor}%
\pgftext[x=6.472593in,y=5.050214in,,top]{\color{textcolor}\sffamily\fontsize{10.000000}{12.000000}\selectfont 0.6}%
\end{pgfscope}%
\begin{pgfscope}%
\pgfpathrectangle{\pgfqpoint{0.640323in}{5.147436in}}{\pgfqpoint{9.687500in}{3.850000in}}%
\pgfusepath{clip}%
\pgfsetrectcap%
\pgfsetroundjoin%
\pgfsetlinewidth{0.803000pt}%
\definecolor{currentstroke}{rgb}{0.690196,0.690196,0.690196}%
\pgfsetstrokecolor{currentstroke}%
\pgfsetdash{}{0pt}%
\pgfpathmoveto{\pgfqpoint{8.269903in}{5.147436in}}%
\pgfpathlineto{\pgfqpoint{8.269903in}{8.997436in}}%
\pgfusepath{stroke}%
\end{pgfscope}%
\begin{pgfscope}%
\pgfsetbuttcap%
\pgfsetroundjoin%
\definecolor{currentfill}{rgb}{0.000000,0.000000,0.000000}%
\pgfsetfillcolor{currentfill}%
\pgfsetlinewidth{0.803000pt}%
\definecolor{currentstroke}{rgb}{0.000000,0.000000,0.000000}%
\pgfsetstrokecolor{currentstroke}%
\pgfsetdash{}{0pt}%
\pgfsys@defobject{currentmarker}{\pgfqpoint{0.000000in}{-0.048611in}}{\pgfqpoint{0.000000in}{0.000000in}}{%
\pgfpathmoveto{\pgfqpoint{0.000000in}{0.000000in}}%
\pgfpathlineto{\pgfqpoint{0.000000in}{-0.048611in}}%
\pgfusepath{stroke,fill}%
}%
\begin{pgfscope}%
\pgfsys@transformshift{8.269903in}{5.147436in}%
\pgfsys@useobject{currentmarker}{}%
\end{pgfscope}%
\end{pgfscope}%
\begin{pgfscope}%
\definecolor{textcolor}{rgb}{0.000000,0.000000,0.000000}%
\pgfsetstrokecolor{textcolor}%
\pgfsetfillcolor{textcolor}%
\pgftext[x=8.269903in,y=5.050214in,,top]{\color{textcolor}\sffamily\fontsize{10.000000}{12.000000}\selectfont 0.8}%
\end{pgfscope}%
\begin{pgfscope}%
\pgfpathrectangle{\pgfqpoint{0.640323in}{5.147436in}}{\pgfqpoint{9.687500in}{3.850000in}}%
\pgfusepath{clip}%
\pgfsetrectcap%
\pgfsetroundjoin%
\pgfsetlinewidth{0.803000pt}%
\definecolor{currentstroke}{rgb}{0.690196,0.690196,0.690196}%
\pgfsetstrokecolor{currentstroke}%
\pgfsetdash{}{0pt}%
\pgfpathmoveto{\pgfqpoint{10.067213in}{5.147436in}}%
\pgfpathlineto{\pgfqpoint{10.067213in}{8.997436in}}%
\pgfusepath{stroke}%
\end{pgfscope}%
\begin{pgfscope}%
\pgfsetbuttcap%
\pgfsetroundjoin%
\definecolor{currentfill}{rgb}{0.000000,0.000000,0.000000}%
\pgfsetfillcolor{currentfill}%
\pgfsetlinewidth{0.803000pt}%
\definecolor{currentstroke}{rgb}{0.000000,0.000000,0.000000}%
\pgfsetstrokecolor{currentstroke}%
\pgfsetdash{}{0pt}%
\pgfsys@defobject{currentmarker}{\pgfqpoint{0.000000in}{-0.048611in}}{\pgfqpoint{0.000000in}{0.000000in}}{%
\pgfpathmoveto{\pgfqpoint{0.000000in}{0.000000in}}%
\pgfpathlineto{\pgfqpoint{0.000000in}{-0.048611in}}%
\pgfusepath{stroke,fill}%
}%
\begin{pgfscope}%
\pgfsys@transformshift{10.067213in}{5.147436in}%
\pgfsys@useobject{currentmarker}{}%
\end{pgfscope}%
\end{pgfscope}%
\begin{pgfscope}%
\definecolor{textcolor}{rgb}{0.000000,0.000000,0.000000}%
\pgfsetstrokecolor{textcolor}%
\pgfsetfillcolor{textcolor}%
\pgftext[x=10.067213in,y=5.050214in,,top]{\color{textcolor}\sffamily\fontsize{10.000000}{12.000000}\selectfont 1.0}%
\end{pgfscope}%
\begin{pgfscope}%
\pgfpathrectangle{\pgfqpoint{0.640323in}{5.147436in}}{\pgfqpoint{9.687500in}{3.850000in}}%
\pgfusepath{clip}%
\pgfsetrectcap%
\pgfsetroundjoin%
\pgfsetlinewidth{0.803000pt}%
\definecolor{currentstroke}{rgb}{0.600000,0.600000,0.600000}%
\pgfsetstrokecolor{currentstroke}%
\pgfsetstrokeopacity{0.200000}%
\pgfsetdash{}{0pt}%
\pgfpathmoveto{\pgfqpoint{1.529991in}{5.147436in}}%
\pgfpathlineto{\pgfqpoint{1.529991in}{8.997436in}}%
\pgfusepath{stroke}%
\end{pgfscope}%
\begin{pgfscope}%
\pgfsetbuttcap%
\pgfsetroundjoin%
\definecolor{currentfill}{rgb}{0.000000,0.000000,0.000000}%
\pgfsetfillcolor{currentfill}%
\pgfsetlinewidth{0.602250pt}%
\definecolor{currentstroke}{rgb}{0.000000,0.000000,0.000000}%
\pgfsetstrokecolor{currentstroke}%
\pgfsetdash{}{0pt}%
\pgfsys@defobject{currentmarker}{\pgfqpoint{0.000000in}{-0.027778in}}{\pgfqpoint{0.000000in}{0.000000in}}{%
\pgfpathmoveto{\pgfqpoint{0.000000in}{0.000000in}}%
\pgfpathlineto{\pgfqpoint{0.000000in}{-0.027778in}}%
\pgfusepath{stroke,fill}%
}%
\begin{pgfscope}%
\pgfsys@transformshift{1.529991in}{5.147436in}%
\pgfsys@useobject{currentmarker}{}%
\end{pgfscope}%
\end{pgfscope}%
\begin{pgfscope}%
\pgfpathrectangle{\pgfqpoint{0.640323in}{5.147436in}}{\pgfqpoint{9.687500in}{3.850000in}}%
\pgfusepath{clip}%
\pgfsetrectcap%
\pgfsetroundjoin%
\pgfsetlinewidth{0.803000pt}%
\definecolor{currentstroke}{rgb}{0.600000,0.600000,0.600000}%
\pgfsetstrokecolor{currentstroke}%
\pgfsetstrokeopacity{0.200000}%
\pgfsetdash{}{0pt}%
\pgfpathmoveto{\pgfqpoint{1.979318in}{5.147436in}}%
\pgfpathlineto{\pgfqpoint{1.979318in}{8.997436in}}%
\pgfusepath{stroke}%
\end{pgfscope}%
\begin{pgfscope}%
\pgfsetbuttcap%
\pgfsetroundjoin%
\definecolor{currentfill}{rgb}{0.000000,0.000000,0.000000}%
\pgfsetfillcolor{currentfill}%
\pgfsetlinewidth{0.602250pt}%
\definecolor{currentstroke}{rgb}{0.000000,0.000000,0.000000}%
\pgfsetstrokecolor{currentstroke}%
\pgfsetdash{}{0pt}%
\pgfsys@defobject{currentmarker}{\pgfqpoint{0.000000in}{-0.027778in}}{\pgfqpoint{0.000000in}{0.000000in}}{%
\pgfpathmoveto{\pgfqpoint{0.000000in}{0.000000in}}%
\pgfpathlineto{\pgfqpoint{0.000000in}{-0.027778in}}%
\pgfusepath{stroke,fill}%
}%
\begin{pgfscope}%
\pgfsys@transformshift{1.979318in}{5.147436in}%
\pgfsys@useobject{currentmarker}{}%
\end{pgfscope}%
\end{pgfscope}%
\begin{pgfscope}%
\pgfpathrectangle{\pgfqpoint{0.640323in}{5.147436in}}{\pgfqpoint{9.687500in}{3.850000in}}%
\pgfusepath{clip}%
\pgfsetrectcap%
\pgfsetroundjoin%
\pgfsetlinewidth{0.803000pt}%
\definecolor{currentstroke}{rgb}{0.600000,0.600000,0.600000}%
\pgfsetstrokecolor{currentstroke}%
\pgfsetstrokeopacity{0.200000}%
\pgfsetdash{}{0pt}%
\pgfpathmoveto{\pgfqpoint{2.428646in}{5.147436in}}%
\pgfpathlineto{\pgfqpoint{2.428646in}{8.997436in}}%
\pgfusepath{stroke}%
\end{pgfscope}%
\begin{pgfscope}%
\pgfsetbuttcap%
\pgfsetroundjoin%
\definecolor{currentfill}{rgb}{0.000000,0.000000,0.000000}%
\pgfsetfillcolor{currentfill}%
\pgfsetlinewidth{0.602250pt}%
\definecolor{currentstroke}{rgb}{0.000000,0.000000,0.000000}%
\pgfsetstrokecolor{currentstroke}%
\pgfsetdash{}{0pt}%
\pgfsys@defobject{currentmarker}{\pgfqpoint{0.000000in}{-0.027778in}}{\pgfqpoint{0.000000in}{0.000000in}}{%
\pgfpathmoveto{\pgfqpoint{0.000000in}{0.000000in}}%
\pgfpathlineto{\pgfqpoint{0.000000in}{-0.027778in}}%
\pgfusepath{stroke,fill}%
}%
\begin{pgfscope}%
\pgfsys@transformshift{2.428646in}{5.147436in}%
\pgfsys@useobject{currentmarker}{}%
\end{pgfscope}%
\end{pgfscope}%
\begin{pgfscope}%
\pgfpathrectangle{\pgfqpoint{0.640323in}{5.147436in}}{\pgfqpoint{9.687500in}{3.850000in}}%
\pgfusepath{clip}%
\pgfsetrectcap%
\pgfsetroundjoin%
\pgfsetlinewidth{0.803000pt}%
\definecolor{currentstroke}{rgb}{0.600000,0.600000,0.600000}%
\pgfsetstrokecolor{currentstroke}%
\pgfsetstrokeopacity{0.200000}%
\pgfsetdash{}{0pt}%
\pgfpathmoveto{\pgfqpoint{3.327301in}{5.147436in}}%
\pgfpathlineto{\pgfqpoint{3.327301in}{8.997436in}}%
\pgfusepath{stroke}%
\end{pgfscope}%
\begin{pgfscope}%
\pgfsetbuttcap%
\pgfsetroundjoin%
\definecolor{currentfill}{rgb}{0.000000,0.000000,0.000000}%
\pgfsetfillcolor{currentfill}%
\pgfsetlinewidth{0.602250pt}%
\definecolor{currentstroke}{rgb}{0.000000,0.000000,0.000000}%
\pgfsetstrokecolor{currentstroke}%
\pgfsetdash{}{0pt}%
\pgfsys@defobject{currentmarker}{\pgfqpoint{0.000000in}{-0.027778in}}{\pgfqpoint{0.000000in}{0.000000in}}{%
\pgfpathmoveto{\pgfqpoint{0.000000in}{0.000000in}}%
\pgfpathlineto{\pgfqpoint{0.000000in}{-0.027778in}}%
\pgfusepath{stroke,fill}%
}%
\begin{pgfscope}%
\pgfsys@transformshift{3.327301in}{5.147436in}%
\pgfsys@useobject{currentmarker}{}%
\end{pgfscope}%
\end{pgfscope}%
\begin{pgfscope}%
\pgfpathrectangle{\pgfqpoint{0.640323in}{5.147436in}}{\pgfqpoint{9.687500in}{3.850000in}}%
\pgfusepath{clip}%
\pgfsetrectcap%
\pgfsetroundjoin%
\pgfsetlinewidth{0.803000pt}%
\definecolor{currentstroke}{rgb}{0.600000,0.600000,0.600000}%
\pgfsetstrokecolor{currentstroke}%
\pgfsetstrokeopacity{0.200000}%
\pgfsetdash{}{0pt}%
\pgfpathmoveto{\pgfqpoint{3.776628in}{5.147436in}}%
\pgfpathlineto{\pgfqpoint{3.776628in}{8.997436in}}%
\pgfusepath{stroke}%
\end{pgfscope}%
\begin{pgfscope}%
\pgfsetbuttcap%
\pgfsetroundjoin%
\definecolor{currentfill}{rgb}{0.000000,0.000000,0.000000}%
\pgfsetfillcolor{currentfill}%
\pgfsetlinewidth{0.602250pt}%
\definecolor{currentstroke}{rgb}{0.000000,0.000000,0.000000}%
\pgfsetstrokecolor{currentstroke}%
\pgfsetdash{}{0pt}%
\pgfsys@defobject{currentmarker}{\pgfqpoint{0.000000in}{-0.027778in}}{\pgfqpoint{0.000000in}{0.000000in}}{%
\pgfpathmoveto{\pgfqpoint{0.000000in}{0.000000in}}%
\pgfpathlineto{\pgfqpoint{0.000000in}{-0.027778in}}%
\pgfusepath{stroke,fill}%
}%
\begin{pgfscope}%
\pgfsys@transformshift{3.776628in}{5.147436in}%
\pgfsys@useobject{currentmarker}{}%
\end{pgfscope}%
\end{pgfscope}%
\begin{pgfscope}%
\pgfpathrectangle{\pgfqpoint{0.640323in}{5.147436in}}{\pgfqpoint{9.687500in}{3.850000in}}%
\pgfusepath{clip}%
\pgfsetrectcap%
\pgfsetroundjoin%
\pgfsetlinewidth{0.803000pt}%
\definecolor{currentstroke}{rgb}{0.600000,0.600000,0.600000}%
\pgfsetstrokecolor{currentstroke}%
\pgfsetstrokeopacity{0.200000}%
\pgfsetdash{}{0pt}%
\pgfpathmoveto{\pgfqpoint{4.225956in}{5.147436in}}%
\pgfpathlineto{\pgfqpoint{4.225956in}{8.997436in}}%
\pgfusepath{stroke}%
\end{pgfscope}%
\begin{pgfscope}%
\pgfsetbuttcap%
\pgfsetroundjoin%
\definecolor{currentfill}{rgb}{0.000000,0.000000,0.000000}%
\pgfsetfillcolor{currentfill}%
\pgfsetlinewidth{0.602250pt}%
\definecolor{currentstroke}{rgb}{0.000000,0.000000,0.000000}%
\pgfsetstrokecolor{currentstroke}%
\pgfsetdash{}{0pt}%
\pgfsys@defobject{currentmarker}{\pgfqpoint{0.000000in}{-0.027778in}}{\pgfqpoint{0.000000in}{0.000000in}}{%
\pgfpathmoveto{\pgfqpoint{0.000000in}{0.000000in}}%
\pgfpathlineto{\pgfqpoint{0.000000in}{-0.027778in}}%
\pgfusepath{stroke,fill}%
}%
\begin{pgfscope}%
\pgfsys@transformshift{4.225956in}{5.147436in}%
\pgfsys@useobject{currentmarker}{}%
\end{pgfscope}%
\end{pgfscope}%
\begin{pgfscope}%
\pgfpathrectangle{\pgfqpoint{0.640323in}{5.147436in}}{\pgfqpoint{9.687500in}{3.850000in}}%
\pgfusepath{clip}%
\pgfsetrectcap%
\pgfsetroundjoin%
\pgfsetlinewidth{0.803000pt}%
\definecolor{currentstroke}{rgb}{0.600000,0.600000,0.600000}%
\pgfsetstrokecolor{currentstroke}%
\pgfsetstrokeopacity{0.200000}%
\pgfsetdash{}{0pt}%
\pgfpathmoveto{\pgfqpoint{5.124611in}{5.147436in}}%
\pgfpathlineto{\pgfqpoint{5.124611in}{8.997436in}}%
\pgfusepath{stroke}%
\end{pgfscope}%
\begin{pgfscope}%
\pgfsetbuttcap%
\pgfsetroundjoin%
\definecolor{currentfill}{rgb}{0.000000,0.000000,0.000000}%
\pgfsetfillcolor{currentfill}%
\pgfsetlinewidth{0.602250pt}%
\definecolor{currentstroke}{rgb}{0.000000,0.000000,0.000000}%
\pgfsetstrokecolor{currentstroke}%
\pgfsetdash{}{0pt}%
\pgfsys@defobject{currentmarker}{\pgfqpoint{0.000000in}{-0.027778in}}{\pgfqpoint{0.000000in}{0.000000in}}{%
\pgfpathmoveto{\pgfqpoint{0.000000in}{0.000000in}}%
\pgfpathlineto{\pgfqpoint{0.000000in}{-0.027778in}}%
\pgfusepath{stroke,fill}%
}%
\begin{pgfscope}%
\pgfsys@transformshift{5.124611in}{5.147436in}%
\pgfsys@useobject{currentmarker}{}%
\end{pgfscope}%
\end{pgfscope}%
\begin{pgfscope}%
\pgfpathrectangle{\pgfqpoint{0.640323in}{5.147436in}}{\pgfqpoint{9.687500in}{3.850000in}}%
\pgfusepath{clip}%
\pgfsetrectcap%
\pgfsetroundjoin%
\pgfsetlinewidth{0.803000pt}%
\definecolor{currentstroke}{rgb}{0.600000,0.600000,0.600000}%
\pgfsetstrokecolor{currentstroke}%
\pgfsetstrokeopacity{0.200000}%
\pgfsetdash{}{0pt}%
\pgfpathmoveto{\pgfqpoint{5.573938in}{5.147436in}}%
\pgfpathlineto{\pgfqpoint{5.573938in}{8.997436in}}%
\pgfusepath{stroke}%
\end{pgfscope}%
\begin{pgfscope}%
\pgfsetbuttcap%
\pgfsetroundjoin%
\definecolor{currentfill}{rgb}{0.000000,0.000000,0.000000}%
\pgfsetfillcolor{currentfill}%
\pgfsetlinewidth{0.602250pt}%
\definecolor{currentstroke}{rgb}{0.000000,0.000000,0.000000}%
\pgfsetstrokecolor{currentstroke}%
\pgfsetdash{}{0pt}%
\pgfsys@defobject{currentmarker}{\pgfqpoint{0.000000in}{-0.027778in}}{\pgfqpoint{0.000000in}{0.000000in}}{%
\pgfpathmoveto{\pgfqpoint{0.000000in}{0.000000in}}%
\pgfpathlineto{\pgfqpoint{0.000000in}{-0.027778in}}%
\pgfusepath{stroke,fill}%
}%
\begin{pgfscope}%
\pgfsys@transformshift{5.573938in}{5.147436in}%
\pgfsys@useobject{currentmarker}{}%
\end{pgfscope}%
\end{pgfscope}%
\begin{pgfscope}%
\pgfpathrectangle{\pgfqpoint{0.640323in}{5.147436in}}{\pgfqpoint{9.687500in}{3.850000in}}%
\pgfusepath{clip}%
\pgfsetrectcap%
\pgfsetroundjoin%
\pgfsetlinewidth{0.803000pt}%
\definecolor{currentstroke}{rgb}{0.600000,0.600000,0.600000}%
\pgfsetstrokecolor{currentstroke}%
\pgfsetstrokeopacity{0.200000}%
\pgfsetdash{}{0pt}%
\pgfpathmoveto{\pgfqpoint{6.023265in}{5.147436in}}%
\pgfpathlineto{\pgfqpoint{6.023265in}{8.997436in}}%
\pgfusepath{stroke}%
\end{pgfscope}%
\begin{pgfscope}%
\pgfsetbuttcap%
\pgfsetroundjoin%
\definecolor{currentfill}{rgb}{0.000000,0.000000,0.000000}%
\pgfsetfillcolor{currentfill}%
\pgfsetlinewidth{0.602250pt}%
\definecolor{currentstroke}{rgb}{0.000000,0.000000,0.000000}%
\pgfsetstrokecolor{currentstroke}%
\pgfsetdash{}{0pt}%
\pgfsys@defobject{currentmarker}{\pgfqpoint{0.000000in}{-0.027778in}}{\pgfqpoint{0.000000in}{0.000000in}}{%
\pgfpathmoveto{\pgfqpoint{0.000000in}{0.000000in}}%
\pgfpathlineto{\pgfqpoint{0.000000in}{-0.027778in}}%
\pgfusepath{stroke,fill}%
}%
\begin{pgfscope}%
\pgfsys@transformshift{6.023265in}{5.147436in}%
\pgfsys@useobject{currentmarker}{}%
\end{pgfscope}%
\end{pgfscope}%
\begin{pgfscope}%
\pgfpathrectangle{\pgfqpoint{0.640323in}{5.147436in}}{\pgfqpoint{9.687500in}{3.850000in}}%
\pgfusepath{clip}%
\pgfsetrectcap%
\pgfsetroundjoin%
\pgfsetlinewidth{0.803000pt}%
\definecolor{currentstroke}{rgb}{0.600000,0.600000,0.600000}%
\pgfsetstrokecolor{currentstroke}%
\pgfsetstrokeopacity{0.200000}%
\pgfsetdash{}{0pt}%
\pgfpathmoveto{\pgfqpoint{6.921920in}{5.147436in}}%
\pgfpathlineto{\pgfqpoint{6.921920in}{8.997436in}}%
\pgfusepath{stroke}%
\end{pgfscope}%
\begin{pgfscope}%
\pgfsetbuttcap%
\pgfsetroundjoin%
\definecolor{currentfill}{rgb}{0.000000,0.000000,0.000000}%
\pgfsetfillcolor{currentfill}%
\pgfsetlinewidth{0.602250pt}%
\definecolor{currentstroke}{rgb}{0.000000,0.000000,0.000000}%
\pgfsetstrokecolor{currentstroke}%
\pgfsetdash{}{0pt}%
\pgfsys@defobject{currentmarker}{\pgfqpoint{0.000000in}{-0.027778in}}{\pgfqpoint{0.000000in}{0.000000in}}{%
\pgfpathmoveto{\pgfqpoint{0.000000in}{0.000000in}}%
\pgfpathlineto{\pgfqpoint{0.000000in}{-0.027778in}}%
\pgfusepath{stroke,fill}%
}%
\begin{pgfscope}%
\pgfsys@transformshift{6.921920in}{5.147436in}%
\pgfsys@useobject{currentmarker}{}%
\end{pgfscope}%
\end{pgfscope}%
\begin{pgfscope}%
\pgfpathrectangle{\pgfqpoint{0.640323in}{5.147436in}}{\pgfqpoint{9.687500in}{3.850000in}}%
\pgfusepath{clip}%
\pgfsetrectcap%
\pgfsetroundjoin%
\pgfsetlinewidth{0.803000pt}%
\definecolor{currentstroke}{rgb}{0.600000,0.600000,0.600000}%
\pgfsetstrokecolor{currentstroke}%
\pgfsetstrokeopacity{0.200000}%
\pgfsetdash{}{0pt}%
\pgfpathmoveto{\pgfqpoint{7.371248in}{5.147436in}}%
\pgfpathlineto{\pgfqpoint{7.371248in}{8.997436in}}%
\pgfusepath{stroke}%
\end{pgfscope}%
\begin{pgfscope}%
\pgfsetbuttcap%
\pgfsetroundjoin%
\definecolor{currentfill}{rgb}{0.000000,0.000000,0.000000}%
\pgfsetfillcolor{currentfill}%
\pgfsetlinewidth{0.602250pt}%
\definecolor{currentstroke}{rgb}{0.000000,0.000000,0.000000}%
\pgfsetstrokecolor{currentstroke}%
\pgfsetdash{}{0pt}%
\pgfsys@defobject{currentmarker}{\pgfqpoint{0.000000in}{-0.027778in}}{\pgfqpoint{0.000000in}{0.000000in}}{%
\pgfpathmoveto{\pgfqpoint{0.000000in}{0.000000in}}%
\pgfpathlineto{\pgfqpoint{0.000000in}{-0.027778in}}%
\pgfusepath{stroke,fill}%
}%
\begin{pgfscope}%
\pgfsys@transformshift{7.371248in}{5.147436in}%
\pgfsys@useobject{currentmarker}{}%
\end{pgfscope}%
\end{pgfscope}%
\begin{pgfscope}%
\pgfpathrectangle{\pgfqpoint{0.640323in}{5.147436in}}{\pgfqpoint{9.687500in}{3.850000in}}%
\pgfusepath{clip}%
\pgfsetrectcap%
\pgfsetroundjoin%
\pgfsetlinewidth{0.803000pt}%
\definecolor{currentstroke}{rgb}{0.600000,0.600000,0.600000}%
\pgfsetstrokecolor{currentstroke}%
\pgfsetstrokeopacity{0.200000}%
\pgfsetdash{}{0pt}%
\pgfpathmoveto{\pgfqpoint{7.820575in}{5.147436in}}%
\pgfpathlineto{\pgfqpoint{7.820575in}{8.997436in}}%
\pgfusepath{stroke}%
\end{pgfscope}%
\begin{pgfscope}%
\pgfsetbuttcap%
\pgfsetroundjoin%
\definecolor{currentfill}{rgb}{0.000000,0.000000,0.000000}%
\pgfsetfillcolor{currentfill}%
\pgfsetlinewidth{0.602250pt}%
\definecolor{currentstroke}{rgb}{0.000000,0.000000,0.000000}%
\pgfsetstrokecolor{currentstroke}%
\pgfsetdash{}{0pt}%
\pgfsys@defobject{currentmarker}{\pgfqpoint{0.000000in}{-0.027778in}}{\pgfqpoint{0.000000in}{0.000000in}}{%
\pgfpathmoveto{\pgfqpoint{0.000000in}{0.000000in}}%
\pgfpathlineto{\pgfqpoint{0.000000in}{-0.027778in}}%
\pgfusepath{stroke,fill}%
}%
\begin{pgfscope}%
\pgfsys@transformshift{7.820575in}{5.147436in}%
\pgfsys@useobject{currentmarker}{}%
\end{pgfscope}%
\end{pgfscope}%
\begin{pgfscope}%
\pgfpathrectangle{\pgfqpoint{0.640323in}{5.147436in}}{\pgfqpoint{9.687500in}{3.850000in}}%
\pgfusepath{clip}%
\pgfsetrectcap%
\pgfsetroundjoin%
\pgfsetlinewidth{0.803000pt}%
\definecolor{currentstroke}{rgb}{0.600000,0.600000,0.600000}%
\pgfsetstrokecolor{currentstroke}%
\pgfsetstrokeopacity{0.200000}%
\pgfsetdash{}{0pt}%
\pgfpathmoveto{\pgfqpoint{8.719230in}{5.147436in}}%
\pgfpathlineto{\pgfqpoint{8.719230in}{8.997436in}}%
\pgfusepath{stroke}%
\end{pgfscope}%
\begin{pgfscope}%
\pgfsetbuttcap%
\pgfsetroundjoin%
\definecolor{currentfill}{rgb}{0.000000,0.000000,0.000000}%
\pgfsetfillcolor{currentfill}%
\pgfsetlinewidth{0.602250pt}%
\definecolor{currentstroke}{rgb}{0.000000,0.000000,0.000000}%
\pgfsetstrokecolor{currentstroke}%
\pgfsetdash{}{0pt}%
\pgfsys@defobject{currentmarker}{\pgfqpoint{0.000000in}{-0.027778in}}{\pgfqpoint{0.000000in}{0.000000in}}{%
\pgfpathmoveto{\pgfqpoint{0.000000in}{0.000000in}}%
\pgfpathlineto{\pgfqpoint{0.000000in}{-0.027778in}}%
\pgfusepath{stroke,fill}%
}%
\begin{pgfscope}%
\pgfsys@transformshift{8.719230in}{5.147436in}%
\pgfsys@useobject{currentmarker}{}%
\end{pgfscope}%
\end{pgfscope}%
\begin{pgfscope}%
\pgfpathrectangle{\pgfqpoint{0.640323in}{5.147436in}}{\pgfqpoint{9.687500in}{3.850000in}}%
\pgfusepath{clip}%
\pgfsetrectcap%
\pgfsetroundjoin%
\pgfsetlinewidth{0.803000pt}%
\definecolor{currentstroke}{rgb}{0.600000,0.600000,0.600000}%
\pgfsetstrokecolor{currentstroke}%
\pgfsetstrokeopacity{0.200000}%
\pgfsetdash{}{0pt}%
\pgfpathmoveto{\pgfqpoint{9.168558in}{5.147436in}}%
\pgfpathlineto{\pgfqpoint{9.168558in}{8.997436in}}%
\pgfusepath{stroke}%
\end{pgfscope}%
\begin{pgfscope}%
\pgfsetbuttcap%
\pgfsetroundjoin%
\definecolor{currentfill}{rgb}{0.000000,0.000000,0.000000}%
\pgfsetfillcolor{currentfill}%
\pgfsetlinewidth{0.602250pt}%
\definecolor{currentstroke}{rgb}{0.000000,0.000000,0.000000}%
\pgfsetstrokecolor{currentstroke}%
\pgfsetdash{}{0pt}%
\pgfsys@defobject{currentmarker}{\pgfqpoint{0.000000in}{-0.027778in}}{\pgfqpoint{0.000000in}{0.000000in}}{%
\pgfpathmoveto{\pgfqpoint{0.000000in}{0.000000in}}%
\pgfpathlineto{\pgfqpoint{0.000000in}{-0.027778in}}%
\pgfusepath{stroke,fill}%
}%
\begin{pgfscope}%
\pgfsys@transformshift{9.168558in}{5.147436in}%
\pgfsys@useobject{currentmarker}{}%
\end{pgfscope}%
\end{pgfscope}%
\begin{pgfscope}%
\pgfpathrectangle{\pgfqpoint{0.640323in}{5.147436in}}{\pgfqpoint{9.687500in}{3.850000in}}%
\pgfusepath{clip}%
\pgfsetrectcap%
\pgfsetroundjoin%
\pgfsetlinewidth{0.803000pt}%
\definecolor{currentstroke}{rgb}{0.600000,0.600000,0.600000}%
\pgfsetstrokecolor{currentstroke}%
\pgfsetstrokeopacity{0.200000}%
\pgfsetdash{}{0pt}%
\pgfpathmoveto{\pgfqpoint{9.617885in}{5.147436in}}%
\pgfpathlineto{\pgfqpoint{9.617885in}{8.997436in}}%
\pgfusepath{stroke}%
\end{pgfscope}%
\begin{pgfscope}%
\pgfsetbuttcap%
\pgfsetroundjoin%
\definecolor{currentfill}{rgb}{0.000000,0.000000,0.000000}%
\pgfsetfillcolor{currentfill}%
\pgfsetlinewidth{0.602250pt}%
\definecolor{currentstroke}{rgb}{0.000000,0.000000,0.000000}%
\pgfsetstrokecolor{currentstroke}%
\pgfsetdash{}{0pt}%
\pgfsys@defobject{currentmarker}{\pgfqpoint{0.000000in}{-0.027778in}}{\pgfqpoint{0.000000in}{0.000000in}}{%
\pgfpathmoveto{\pgfqpoint{0.000000in}{0.000000in}}%
\pgfpathlineto{\pgfqpoint{0.000000in}{-0.027778in}}%
\pgfusepath{stroke,fill}%
}%
\begin{pgfscope}%
\pgfsys@transformshift{9.617885in}{5.147436in}%
\pgfsys@useobject{currentmarker}{}%
\end{pgfscope}%
\end{pgfscope}%
\begin{pgfscope}%
\definecolor{textcolor}{rgb}{0.000000,0.000000,0.000000}%
\pgfsetstrokecolor{textcolor}%
\pgfsetfillcolor{textcolor}%
\pgftext[x=5.484072in,y=4.860245in,,top]{\color{textcolor}\sffamily\fontsize{10.000000}{12.000000}\selectfont turnover probability \(\displaystyle p_1\,(S\rightarrow I\,)\)}%
\end{pgfscope}%
\begin{pgfscope}%
\pgfpathrectangle{\pgfqpoint{0.640323in}{5.147436in}}{\pgfqpoint{9.687500in}{3.850000in}}%
\pgfusepath{clip}%
\pgfsetrectcap%
\pgfsetroundjoin%
\pgfsetlinewidth{0.803000pt}%
\definecolor{currentstroke}{rgb}{0.690196,0.690196,0.690196}%
\pgfsetstrokecolor{currentstroke}%
\pgfsetdash{}{0pt}%
\pgfpathmoveto{\pgfqpoint{0.640323in}{5.271629in}}%
\pgfpathlineto{\pgfqpoint{10.327822in}{5.271629in}}%
\pgfusepath{stroke}%
\end{pgfscope}%
\begin{pgfscope}%
\pgfsetbuttcap%
\pgfsetroundjoin%
\definecolor{currentfill}{rgb}{0.000000,0.000000,0.000000}%
\pgfsetfillcolor{currentfill}%
\pgfsetlinewidth{0.803000pt}%
\definecolor{currentstroke}{rgb}{0.000000,0.000000,0.000000}%
\pgfsetstrokecolor{currentstroke}%
\pgfsetdash{}{0pt}%
\pgfsys@defobject{currentmarker}{\pgfqpoint{-0.048611in}{0.000000in}}{\pgfqpoint{-0.000000in}{0.000000in}}{%
\pgfpathmoveto{\pgfqpoint{-0.000000in}{0.000000in}}%
\pgfpathlineto{\pgfqpoint{-0.048611in}{0.000000in}}%
\pgfusepath{stroke,fill}%
}%
\begin{pgfscope}%
\pgfsys@transformshift{0.640323in}{5.271629in}%
\pgfsys@useobject{currentmarker}{}%
\end{pgfscope}%
\end{pgfscope}%
\begin{pgfscope}%
\definecolor{textcolor}{rgb}{0.000000,0.000000,0.000000}%
\pgfsetstrokecolor{textcolor}%
\pgfsetfillcolor{textcolor}%
\pgftext[x=0.322221in, y=5.218868in, left, base]{\color{textcolor}\sffamily\fontsize{10.000000}{12.000000}\selectfont 0.0}%
\end{pgfscope}%
\begin{pgfscope}%
\pgfpathrectangle{\pgfqpoint{0.640323in}{5.147436in}}{\pgfqpoint{9.687500in}{3.850000in}}%
\pgfusepath{clip}%
\pgfsetrectcap%
\pgfsetroundjoin%
\pgfsetlinewidth{0.803000pt}%
\definecolor{currentstroke}{rgb}{0.690196,0.690196,0.690196}%
\pgfsetstrokecolor{currentstroke}%
\pgfsetdash{}{0pt}%
\pgfpathmoveto{\pgfqpoint{0.640323in}{5.892597in}}%
\pgfpathlineto{\pgfqpoint{10.327822in}{5.892597in}}%
\pgfusepath{stroke}%
\end{pgfscope}%
\begin{pgfscope}%
\pgfsetbuttcap%
\pgfsetroundjoin%
\definecolor{currentfill}{rgb}{0.000000,0.000000,0.000000}%
\pgfsetfillcolor{currentfill}%
\pgfsetlinewidth{0.803000pt}%
\definecolor{currentstroke}{rgb}{0.000000,0.000000,0.000000}%
\pgfsetstrokecolor{currentstroke}%
\pgfsetdash{}{0pt}%
\pgfsys@defobject{currentmarker}{\pgfqpoint{-0.048611in}{0.000000in}}{\pgfqpoint{-0.000000in}{0.000000in}}{%
\pgfpathmoveto{\pgfqpoint{-0.000000in}{0.000000in}}%
\pgfpathlineto{\pgfqpoint{-0.048611in}{0.000000in}}%
\pgfusepath{stroke,fill}%
}%
\begin{pgfscope}%
\pgfsys@transformshift{0.640323in}{5.892597in}%
\pgfsys@useobject{currentmarker}{}%
\end{pgfscope}%
\end{pgfscope}%
\begin{pgfscope}%
\definecolor{textcolor}{rgb}{0.000000,0.000000,0.000000}%
\pgfsetstrokecolor{textcolor}%
\pgfsetfillcolor{textcolor}%
\pgftext[x=0.322221in, y=5.839836in, left, base]{\color{textcolor}\sffamily\fontsize{10.000000}{12.000000}\selectfont 0.1}%
\end{pgfscope}%
\begin{pgfscope}%
\pgfpathrectangle{\pgfqpoint{0.640323in}{5.147436in}}{\pgfqpoint{9.687500in}{3.850000in}}%
\pgfusepath{clip}%
\pgfsetrectcap%
\pgfsetroundjoin%
\pgfsetlinewidth{0.803000pt}%
\definecolor{currentstroke}{rgb}{0.690196,0.690196,0.690196}%
\pgfsetstrokecolor{currentstroke}%
\pgfsetdash{}{0pt}%
\pgfpathmoveto{\pgfqpoint{0.640323in}{6.513565in}}%
\pgfpathlineto{\pgfqpoint{10.327822in}{6.513565in}}%
\pgfusepath{stroke}%
\end{pgfscope}%
\begin{pgfscope}%
\pgfsetbuttcap%
\pgfsetroundjoin%
\definecolor{currentfill}{rgb}{0.000000,0.000000,0.000000}%
\pgfsetfillcolor{currentfill}%
\pgfsetlinewidth{0.803000pt}%
\definecolor{currentstroke}{rgb}{0.000000,0.000000,0.000000}%
\pgfsetstrokecolor{currentstroke}%
\pgfsetdash{}{0pt}%
\pgfsys@defobject{currentmarker}{\pgfqpoint{-0.048611in}{0.000000in}}{\pgfqpoint{-0.000000in}{0.000000in}}{%
\pgfpathmoveto{\pgfqpoint{-0.000000in}{0.000000in}}%
\pgfpathlineto{\pgfqpoint{-0.048611in}{0.000000in}}%
\pgfusepath{stroke,fill}%
}%
\begin{pgfscope}%
\pgfsys@transformshift{0.640323in}{6.513565in}%
\pgfsys@useobject{currentmarker}{}%
\end{pgfscope}%
\end{pgfscope}%
\begin{pgfscope}%
\definecolor{textcolor}{rgb}{0.000000,0.000000,0.000000}%
\pgfsetstrokecolor{textcolor}%
\pgfsetfillcolor{textcolor}%
\pgftext[x=0.322221in, y=6.460803in, left, base]{\color{textcolor}\sffamily\fontsize{10.000000}{12.000000}\selectfont 0.2}%
\end{pgfscope}%
\begin{pgfscope}%
\pgfpathrectangle{\pgfqpoint{0.640323in}{5.147436in}}{\pgfqpoint{9.687500in}{3.850000in}}%
\pgfusepath{clip}%
\pgfsetrectcap%
\pgfsetroundjoin%
\pgfsetlinewidth{0.803000pt}%
\definecolor{currentstroke}{rgb}{0.690196,0.690196,0.690196}%
\pgfsetstrokecolor{currentstroke}%
\pgfsetdash{}{0pt}%
\pgfpathmoveto{\pgfqpoint{0.640323in}{7.134533in}}%
\pgfpathlineto{\pgfqpoint{10.327822in}{7.134533in}}%
\pgfusepath{stroke}%
\end{pgfscope}%
\begin{pgfscope}%
\pgfsetbuttcap%
\pgfsetroundjoin%
\definecolor{currentfill}{rgb}{0.000000,0.000000,0.000000}%
\pgfsetfillcolor{currentfill}%
\pgfsetlinewidth{0.803000pt}%
\definecolor{currentstroke}{rgb}{0.000000,0.000000,0.000000}%
\pgfsetstrokecolor{currentstroke}%
\pgfsetdash{}{0pt}%
\pgfsys@defobject{currentmarker}{\pgfqpoint{-0.048611in}{0.000000in}}{\pgfqpoint{-0.000000in}{0.000000in}}{%
\pgfpathmoveto{\pgfqpoint{-0.000000in}{0.000000in}}%
\pgfpathlineto{\pgfqpoint{-0.048611in}{0.000000in}}%
\pgfusepath{stroke,fill}%
}%
\begin{pgfscope}%
\pgfsys@transformshift{0.640323in}{7.134533in}%
\pgfsys@useobject{currentmarker}{}%
\end{pgfscope}%
\end{pgfscope}%
\begin{pgfscope}%
\definecolor{textcolor}{rgb}{0.000000,0.000000,0.000000}%
\pgfsetstrokecolor{textcolor}%
\pgfsetfillcolor{textcolor}%
\pgftext[x=0.322221in, y=7.081771in, left, base]{\color{textcolor}\sffamily\fontsize{10.000000}{12.000000}\selectfont 0.3}%
\end{pgfscope}%
\begin{pgfscope}%
\pgfpathrectangle{\pgfqpoint{0.640323in}{5.147436in}}{\pgfqpoint{9.687500in}{3.850000in}}%
\pgfusepath{clip}%
\pgfsetrectcap%
\pgfsetroundjoin%
\pgfsetlinewidth{0.803000pt}%
\definecolor{currentstroke}{rgb}{0.690196,0.690196,0.690196}%
\pgfsetstrokecolor{currentstroke}%
\pgfsetdash{}{0pt}%
\pgfpathmoveto{\pgfqpoint{0.640323in}{7.755500in}}%
\pgfpathlineto{\pgfqpoint{10.327822in}{7.755500in}}%
\pgfusepath{stroke}%
\end{pgfscope}%
\begin{pgfscope}%
\pgfsetbuttcap%
\pgfsetroundjoin%
\definecolor{currentfill}{rgb}{0.000000,0.000000,0.000000}%
\pgfsetfillcolor{currentfill}%
\pgfsetlinewidth{0.803000pt}%
\definecolor{currentstroke}{rgb}{0.000000,0.000000,0.000000}%
\pgfsetstrokecolor{currentstroke}%
\pgfsetdash{}{0pt}%
\pgfsys@defobject{currentmarker}{\pgfqpoint{-0.048611in}{0.000000in}}{\pgfqpoint{-0.000000in}{0.000000in}}{%
\pgfpathmoveto{\pgfqpoint{-0.000000in}{0.000000in}}%
\pgfpathlineto{\pgfqpoint{-0.048611in}{0.000000in}}%
\pgfusepath{stroke,fill}%
}%
\begin{pgfscope}%
\pgfsys@transformshift{0.640323in}{7.755500in}%
\pgfsys@useobject{currentmarker}{}%
\end{pgfscope}%
\end{pgfscope}%
\begin{pgfscope}%
\definecolor{textcolor}{rgb}{0.000000,0.000000,0.000000}%
\pgfsetstrokecolor{textcolor}%
\pgfsetfillcolor{textcolor}%
\pgftext[x=0.322221in, y=7.702739in, left, base]{\color{textcolor}\sffamily\fontsize{10.000000}{12.000000}\selectfont 0.4}%
\end{pgfscope}%
\begin{pgfscope}%
\pgfpathrectangle{\pgfqpoint{0.640323in}{5.147436in}}{\pgfqpoint{9.687500in}{3.850000in}}%
\pgfusepath{clip}%
\pgfsetrectcap%
\pgfsetroundjoin%
\pgfsetlinewidth{0.803000pt}%
\definecolor{currentstroke}{rgb}{0.690196,0.690196,0.690196}%
\pgfsetstrokecolor{currentstroke}%
\pgfsetdash{}{0pt}%
\pgfpathmoveto{\pgfqpoint{0.640323in}{8.376468in}}%
\pgfpathlineto{\pgfqpoint{10.327822in}{8.376468in}}%
\pgfusepath{stroke}%
\end{pgfscope}%
\begin{pgfscope}%
\pgfsetbuttcap%
\pgfsetroundjoin%
\definecolor{currentfill}{rgb}{0.000000,0.000000,0.000000}%
\pgfsetfillcolor{currentfill}%
\pgfsetlinewidth{0.803000pt}%
\definecolor{currentstroke}{rgb}{0.000000,0.000000,0.000000}%
\pgfsetstrokecolor{currentstroke}%
\pgfsetdash{}{0pt}%
\pgfsys@defobject{currentmarker}{\pgfqpoint{-0.048611in}{0.000000in}}{\pgfqpoint{-0.000000in}{0.000000in}}{%
\pgfpathmoveto{\pgfqpoint{-0.000000in}{0.000000in}}%
\pgfpathlineto{\pgfqpoint{-0.048611in}{0.000000in}}%
\pgfusepath{stroke,fill}%
}%
\begin{pgfscope}%
\pgfsys@transformshift{0.640323in}{8.376468in}%
\pgfsys@useobject{currentmarker}{}%
\end{pgfscope}%
\end{pgfscope}%
\begin{pgfscope}%
\definecolor{textcolor}{rgb}{0.000000,0.000000,0.000000}%
\pgfsetstrokecolor{textcolor}%
\pgfsetfillcolor{textcolor}%
\pgftext[x=0.322221in, y=8.323707in, left, base]{\color{textcolor}\sffamily\fontsize{10.000000}{12.000000}\selectfont 0.5}%
\end{pgfscope}%
\begin{pgfscope}%
\pgfpathrectangle{\pgfqpoint{0.640323in}{5.147436in}}{\pgfqpoint{9.687500in}{3.850000in}}%
\pgfusepath{clip}%
\pgfsetrectcap%
\pgfsetroundjoin%
\pgfsetlinewidth{0.803000pt}%
\definecolor{currentstroke}{rgb}{0.690196,0.690196,0.690196}%
\pgfsetstrokecolor{currentstroke}%
\pgfsetdash{}{0pt}%
\pgfpathmoveto{\pgfqpoint{0.640323in}{8.997436in}}%
\pgfpathlineto{\pgfqpoint{10.327822in}{8.997436in}}%
\pgfusepath{stroke}%
\end{pgfscope}%
\begin{pgfscope}%
\pgfsetbuttcap%
\pgfsetroundjoin%
\definecolor{currentfill}{rgb}{0.000000,0.000000,0.000000}%
\pgfsetfillcolor{currentfill}%
\pgfsetlinewidth{0.803000pt}%
\definecolor{currentstroke}{rgb}{0.000000,0.000000,0.000000}%
\pgfsetstrokecolor{currentstroke}%
\pgfsetdash{}{0pt}%
\pgfsys@defobject{currentmarker}{\pgfqpoint{-0.048611in}{0.000000in}}{\pgfqpoint{-0.000000in}{0.000000in}}{%
\pgfpathmoveto{\pgfqpoint{-0.000000in}{0.000000in}}%
\pgfpathlineto{\pgfqpoint{-0.048611in}{0.000000in}}%
\pgfusepath{stroke,fill}%
}%
\begin{pgfscope}%
\pgfsys@transformshift{0.640323in}{8.997436in}%
\pgfsys@useobject{currentmarker}{}%
\end{pgfscope}%
\end{pgfscope}%
\begin{pgfscope}%
\definecolor{textcolor}{rgb}{0.000000,0.000000,0.000000}%
\pgfsetstrokecolor{textcolor}%
\pgfsetfillcolor{textcolor}%
\pgftext[x=0.322221in, y=8.944674in, left, base]{\color{textcolor}\sffamily\fontsize{10.000000}{12.000000}\selectfont 0.6}%
\end{pgfscope}%
\begin{pgfscope}%
\pgfpathrectangle{\pgfqpoint{0.640323in}{5.147436in}}{\pgfqpoint{9.687500in}{3.850000in}}%
\pgfusepath{clip}%
\pgfsetrectcap%
\pgfsetroundjoin%
\pgfsetlinewidth{0.803000pt}%
\definecolor{currentstroke}{rgb}{0.600000,0.600000,0.600000}%
\pgfsetstrokecolor{currentstroke}%
\pgfsetstrokeopacity{0.200000}%
\pgfsetdash{}{0pt}%
\pgfpathmoveto{\pgfqpoint{0.640323in}{5.395823in}}%
\pgfpathlineto{\pgfqpoint{10.327822in}{5.395823in}}%
\pgfusepath{stroke}%
\end{pgfscope}%
\begin{pgfscope}%
\pgfsetbuttcap%
\pgfsetroundjoin%
\definecolor{currentfill}{rgb}{0.000000,0.000000,0.000000}%
\pgfsetfillcolor{currentfill}%
\pgfsetlinewidth{0.602250pt}%
\definecolor{currentstroke}{rgb}{0.000000,0.000000,0.000000}%
\pgfsetstrokecolor{currentstroke}%
\pgfsetdash{}{0pt}%
\pgfsys@defobject{currentmarker}{\pgfqpoint{-0.027778in}{0.000000in}}{\pgfqpoint{-0.000000in}{0.000000in}}{%
\pgfpathmoveto{\pgfqpoint{-0.000000in}{0.000000in}}%
\pgfpathlineto{\pgfqpoint{-0.027778in}{0.000000in}}%
\pgfusepath{stroke,fill}%
}%
\begin{pgfscope}%
\pgfsys@transformshift{0.640323in}{5.395823in}%
\pgfsys@useobject{currentmarker}{}%
\end{pgfscope}%
\end{pgfscope}%
\begin{pgfscope}%
\pgfpathrectangle{\pgfqpoint{0.640323in}{5.147436in}}{\pgfqpoint{9.687500in}{3.850000in}}%
\pgfusepath{clip}%
\pgfsetrectcap%
\pgfsetroundjoin%
\pgfsetlinewidth{0.803000pt}%
\definecolor{currentstroke}{rgb}{0.600000,0.600000,0.600000}%
\pgfsetstrokecolor{currentstroke}%
\pgfsetstrokeopacity{0.200000}%
\pgfsetdash{}{0pt}%
\pgfpathmoveto{\pgfqpoint{0.640323in}{5.520016in}}%
\pgfpathlineto{\pgfqpoint{10.327822in}{5.520016in}}%
\pgfusepath{stroke}%
\end{pgfscope}%
\begin{pgfscope}%
\pgfsetbuttcap%
\pgfsetroundjoin%
\definecolor{currentfill}{rgb}{0.000000,0.000000,0.000000}%
\pgfsetfillcolor{currentfill}%
\pgfsetlinewidth{0.602250pt}%
\definecolor{currentstroke}{rgb}{0.000000,0.000000,0.000000}%
\pgfsetstrokecolor{currentstroke}%
\pgfsetdash{}{0pt}%
\pgfsys@defobject{currentmarker}{\pgfqpoint{-0.027778in}{0.000000in}}{\pgfqpoint{-0.000000in}{0.000000in}}{%
\pgfpathmoveto{\pgfqpoint{-0.000000in}{0.000000in}}%
\pgfpathlineto{\pgfqpoint{-0.027778in}{0.000000in}}%
\pgfusepath{stroke,fill}%
}%
\begin{pgfscope}%
\pgfsys@transformshift{0.640323in}{5.520016in}%
\pgfsys@useobject{currentmarker}{}%
\end{pgfscope}%
\end{pgfscope}%
\begin{pgfscope}%
\pgfpathrectangle{\pgfqpoint{0.640323in}{5.147436in}}{\pgfqpoint{9.687500in}{3.850000in}}%
\pgfusepath{clip}%
\pgfsetrectcap%
\pgfsetroundjoin%
\pgfsetlinewidth{0.803000pt}%
\definecolor{currentstroke}{rgb}{0.600000,0.600000,0.600000}%
\pgfsetstrokecolor{currentstroke}%
\pgfsetstrokeopacity{0.200000}%
\pgfsetdash{}{0pt}%
\pgfpathmoveto{\pgfqpoint{0.640323in}{5.644210in}}%
\pgfpathlineto{\pgfqpoint{10.327822in}{5.644210in}}%
\pgfusepath{stroke}%
\end{pgfscope}%
\begin{pgfscope}%
\pgfsetbuttcap%
\pgfsetroundjoin%
\definecolor{currentfill}{rgb}{0.000000,0.000000,0.000000}%
\pgfsetfillcolor{currentfill}%
\pgfsetlinewidth{0.602250pt}%
\definecolor{currentstroke}{rgb}{0.000000,0.000000,0.000000}%
\pgfsetstrokecolor{currentstroke}%
\pgfsetdash{}{0pt}%
\pgfsys@defobject{currentmarker}{\pgfqpoint{-0.027778in}{0.000000in}}{\pgfqpoint{-0.000000in}{0.000000in}}{%
\pgfpathmoveto{\pgfqpoint{-0.000000in}{0.000000in}}%
\pgfpathlineto{\pgfqpoint{-0.027778in}{0.000000in}}%
\pgfusepath{stroke,fill}%
}%
\begin{pgfscope}%
\pgfsys@transformshift{0.640323in}{5.644210in}%
\pgfsys@useobject{currentmarker}{}%
\end{pgfscope}%
\end{pgfscope}%
\begin{pgfscope}%
\pgfpathrectangle{\pgfqpoint{0.640323in}{5.147436in}}{\pgfqpoint{9.687500in}{3.850000in}}%
\pgfusepath{clip}%
\pgfsetrectcap%
\pgfsetroundjoin%
\pgfsetlinewidth{0.803000pt}%
\definecolor{currentstroke}{rgb}{0.600000,0.600000,0.600000}%
\pgfsetstrokecolor{currentstroke}%
\pgfsetstrokeopacity{0.200000}%
\pgfsetdash{}{0pt}%
\pgfpathmoveto{\pgfqpoint{0.640323in}{5.768404in}}%
\pgfpathlineto{\pgfqpoint{10.327822in}{5.768404in}}%
\pgfusepath{stroke}%
\end{pgfscope}%
\begin{pgfscope}%
\pgfsetbuttcap%
\pgfsetroundjoin%
\definecolor{currentfill}{rgb}{0.000000,0.000000,0.000000}%
\pgfsetfillcolor{currentfill}%
\pgfsetlinewidth{0.602250pt}%
\definecolor{currentstroke}{rgb}{0.000000,0.000000,0.000000}%
\pgfsetstrokecolor{currentstroke}%
\pgfsetdash{}{0pt}%
\pgfsys@defobject{currentmarker}{\pgfqpoint{-0.027778in}{0.000000in}}{\pgfqpoint{-0.000000in}{0.000000in}}{%
\pgfpathmoveto{\pgfqpoint{-0.000000in}{0.000000in}}%
\pgfpathlineto{\pgfqpoint{-0.027778in}{0.000000in}}%
\pgfusepath{stroke,fill}%
}%
\begin{pgfscope}%
\pgfsys@transformshift{0.640323in}{5.768404in}%
\pgfsys@useobject{currentmarker}{}%
\end{pgfscope}%
\end{pgfscope}%
\begin{pgfscope}%
\pgfpathrectangle{\pgfqpoint{0.640323in}{5.147436in}}{\pgfqpoint{9.687500in}{3.850000in}}%
\pgfusepath{clip}%
\pgfsetrectcap%
\pgfsetroundjoin%
\pgfsetlinewidth{0.803000pt}%
\definecolor{currentstroke}{rgb}{0.600000,0.600000,0.600000}%
\pgfsetstrokecolor{currentstroke}%
\pgfsetstrokeopacity{0.200000}%
\pgfsetdash{}{0pt}%
\pgfpathmoveto{\pgfqpoint{0.640323in}{6.016791in}}%
\pgfpathlineto{\pgfqpoint{10.327822in}{6.016791in}}%
\pgfusepath{stroke}%
\end{pgfscope}%
\begin{pgfscope}%
\pgfsetbuttcap%
\pgfsetroundjoin%
\definecolor{currentfill}{rgb}{0.000000,0.000000,0.000000}%
\pgfsetfillcolor{currentfill}%
\pgfsetlinewidth{0.602250pt}%
\definecolor{currentstroke}{rgb}{0.000000,0.000000,0.000000}%
\pgfsetstrokecolor{currentstroke}%
\pgfsetdash{}{0pt}%
\pgfsys@defobject{currentmarker}{\pgfqpoint{-0.027778in}{0.000000in}}{\pgfqpoint{-0.000000in}{0.000000in}}{%
\pgfpathmoveto{\pgfqpoint{-0.000000in}{0.000000in}}%
\pgfpathlineto{\pgfqpoint{-0.027778in}{0.000000in}}%
\pgfusepath{stroke,fill}%
}%
\begin{pgfscope}%
\pgfsys@transformshift{0.640323in}{6.016791in}%
\pgfsys@useobject{currentmarker}{}%
\end{pgfscope}%
\end{pgfscope}%
\begin{pgfscope}%
\pgfpathrectangle{\pgfqpoint{0.640323in}{5.147436in}}{\pgfqpoint{9.687500in}{3.850000in}}%
\pgfusepath{clip}%
\pgfsetrectcap%
\pgfsetroundjoin%
\pgfsetlinewidth{0.803000pt}%
\definecolor{currentstroke}{rgb}{0.600000,0.600000,0.600000}%
\pgfsetstrokecolor{currentstroke}%
\pgfsetstrokeopacity{0.200000}%
\pgfsetdash{}{0pt}%
\pgfpathmoveto{\pgfqpoint{0.640323in}{6.140984in}}%
\pgfpathlineto{\pgfqpoint{10.327822in}{6.140984in}}%
\pgfusepath{stroke}%
\end{pgfscope}%
\begin{pgfscope}%
\pgfsetbuttcap%
\pgfsetroundjoin%
\definecolor{currentfill}{rgb}{0.000000,0.000000,0.000000}%
\pgfsetfillcolor{currentfill}%
\pgfsetlinewidth{0.602250pt}%
\definecolor{currentstroke}{rgb}{0.000000,0.000000,0.000000}%
\pgfsetstrokecolor{currentstroke}%
\pgfsetdash{}{0pt}%
\pgfsys@defobject{currentmarker}{\pgfqpoint{-0.027778in}{0.000000in}}{\pgfqpoint{-0.000000in}{0.000000in}}{%
\pgfpathmoveto{\pgfqpoint{-0.000000in}{0.000000in}}%
\pgfpathlineto{\pgfqpoint{-0.027778in}{0.000000in}}%
\pgfusepath{stroke,fill}%
}%
\begin{pgfscope}%
\pgfsys@transformshift{0.640323in}{6.140984in}%
\pgfsys@useobject{currentmarker}{}%
\end{pgfscope}%
\end{pgfscope}%
\begin{pgfscope}%
\pgfpathrectangle{\pgfqpoint{0.640323in}{5.147436in}}{\pgfqpoint{9.687500in}{3.850000in}}%
\pgfusepath{clip}%
\pgfsetrectcap%
\pgfsetroundjoin%
\pgfsetlinewidth{0.803000pt}%
\definecolor{currentstroke}{rgb}{0.600000,0.600000,0.600000}%
\pgfsetstrokecolor{currentstroke}%
\pgfsetstrokeopacity{0.200000}%
\pgfsetdash{}{0pt}%
\pgfpathmoveto{\pgfqpoint{0.640323in}{6.265178in}}%
\pgfpathlineto{\pgfqpoint{10.327822in}{6.265178in}}%
\pgfusepath{stroke}%
\end{pgfscope}%
\begin{pgfscope}%
\pgfsetbuttcap%
\pgfsetroundjoin%
\definecolor{currentfill}{rgb}{0.000000,0.000000,0.000000}%
\pgfsetfillcolor{currentfill}%
\pgfsetlinewidth{0.602250pt}%
\definecolor{currentstroke}{rgb}{0.000000,0.000000,0.000000}%
\pgfsetstrokecolor{currentstroke}%
\pgfsetdash{}{0pt}%
\pgfsys@defobject{currentmarker}{\pgfqpoint{-0.027778in}{0.000000in}}{\pgfqpoint{-0.000000in}{0.000000in}}{%
\pgfpathmoveto{\pgfqpoint{-0.000000in}{0.000000in}}%
\pgfpathlineto{\pgfqpoint{-0.027778in}{0.000000in}}%
\pgfusepath{stroke,fill}%
}%
\begin{pgfscope}%
\pgfsys@transformshift{0.640323in}{6.265178in}%
\pgfsys@useobject{currentmarker}{}%
\end{pgfscope}%
\end{pgfscope}%
\begin{pgfscope}%
\pgfpathrectangle{\pgfqpoint{0.640323in}{5.147436in}}{\pgfqpoint{9.687500in}{3.850000in}}%
\pgfusepath{clip}%
\pgfsetrectcap%
\pgfsetroundjoin%
\pgfsetlinewidth{0.803000pt}%
\definecolor{currentstroke}{rgb}{0.600000,0.600000,0.600000}%
\pgfsetstrokecolor{currentstroke}%
\pgfsetstrokeopacity{0.200000}%
\pgfsetdash{}{0pt}%
\pgfpathmoveto{\pgfqpoint{0.640323in}{6.389371in}}%
\pgfpathlineto{\pgfqpoint{10.327822in}{6.389371in}}%
\pgfusepath{stroke}%
\end{pgfscope}%
\begin{pgfscope}%
\pgfsetbuttcap%
\pgfsetroundjoin%
\definecolor{currentfill}{rgb}{0.000000,0.000000,0.000000}%
\pgfsetfillcolor{currentfill}%
\pgfsetlinewidth{0.602250pt}%
\definecolor{currentstroke}{rgb}{0.000000,0.000000,0.000000}%
\pgfsetstrokecolor{currentstroke}%
\pgfsetdash{}{0pt}%
\pgfsys@defobject{currentmarker}{\pgfqpoint{-0.027778in}{0.000000in}}{\pgfqpoint{-0.000000in}{0.000000in}}{%
\pgfpathmoveto{\pgfqpoint{-0.000000in}{0.000000in}}%
\pgfpathlineto{\pgfqpoint{-0.027778in}{0.000000in}}%
\pgfusepath{stroke,fill}%
}%
\begin{pgfscope}%
\pgfsys@transformshift{0.640323in}{6.389371in}%
\pgfsys@useobject{currentmarker}{}%
\end{pgfscope}%
\end{pgfscope}%
\begin{pgfscope}%
\pgfpathrectangle{\pgfqpoint{0.640323in}{5.147436in}}{\pgfqpoint{9.687500in}{3.850000in}}%
\pgfusepath{clip}%
\pgfsetrectcap%
\pgfsetroundjoin%
\pgfsetlinewidth{0.803000pt}%
\definecolor{currentstroke}{rgb}{0.600000,0.600000,0.600000}%
\pgfsetstrokecolor{currentstroke}%
\pgfsetstrokeopacity{0.200000}%
\pgfsetdash{}{0pt}%
\pgfpathmoveto{\pgfqpoint{0.640323in}{6.637758in}}%
\pgfpathlineto{\pgfqpoint{10.327822in}{6.637758in}}%
\pgfusepath{stroke}%
\end{pgfscope}%
\begin{pgfscope}%
\pgfsetbuttcap%
\pgfsetroundjoin%
\definecolor{currentfill}{rgb}{0.000000,0.000000,0.000000}%
\pgfsetfillcolor{currentfill}%
\pgfsetlinewidth{0.602250pt}%
\definecolor{currentstroke}{rgb}{0.000000,0.000000,0.000000}%
\pgfsetstrokecolor{currentstroke}%
\pgfsetdash{}{0pt}%
\pgfsys@defobject{currentmarker}{\pgfqpoint{-0.027778in}{0.000000in}}{\pgfqpoint{-0.000000in}{0.000000in}}{%
\pgfpathmoveto{\pgfqpoint{-0.000000in}{0.000000in}}%
\pgfpathlineto{\pgfqpoint{-0.027778in}{0.000000in}}%
\pgfusepath{stroke,fill}%
}%
\begin{pgfscope}%
\pgfsys@transformshift{0.640323in}{6.637758in}%
\pgfsys@useobject{currentmarker}{}%
\end{pgfscope}%
\end{pgfscope}%
\begin{pgfscope}%
\pgfpathrectangle{\pgfqpoint{0.640323in}{5.147436in}}{\pgfqpoint{9.687500in}{3.850000in}}%
\pgfusepath{clip}%
\pgfsetrectcap%
\pgfsetroundjoin%
\pgfsetlinewidth{0.803000pt}%
\definecolor{currentstroke}{rgb}{0.600000,0.600000,0.600000}%
\pgfsetstrokecolor{currentstroke}%
\pgfsetstrokeopacity{0.200000}%
\pgfsetdash{}{0pt}%
\pgfpathmoveto{\pgfqpoint{0.640323in}{6.761952in}}%
\pgfpathlineto{\pgfqpoint{10.327822in}{6.761952in}}%
\pgfusepath{stroke}%
\end{pgfscope}%
\begin{pgfscope}%
\pgfsetbuttcap%
\pgfsetroundjoin%
\definecolor{currentfill}{rgb}{0.000000,0.000000,0.000000}%
\pgfsetfillcolor{currentfill}%
\pgfsetlinewidth{0.602250pt}%
\definecolor{currentstroke}{rgb}{0.000000,0.000000,0.000000}%
\pgfsetstrokecolor{currentstroke}%
\pgfsetdash{}{0pt}%
\pgfsys@defobject{currentmarker}{\pgfqpoint{-0.027778in}{0.000000in}}{\pgfqpoint{-0.000000in}{0.000000in}}{%
\pgfpathmoveto{\pgfqpoint{-0.000000in}{0.000000in}}%
\pgfpathlineto{\pgfqpoint{-0.027778in}{0.000000in}}%
\pgfusepath{stroke,fill}%
}%
\begin{pgfscope}%
\pgfsys@transformshift{0.640323in}{6.761952in}%
\pgfsys@useobject{currentmarker}{}%
\end{pgfscope}%
\end{pgfscope}%
\begin{pgfscope}%
\pgfpathrectangle{\pgfqpoint{0.640323in}{5.147436in}}{\pgfqpoint{9.687500in}{3.850000in}}%
\pgfusepath{clip}%
\pgfsetrectcap%
\pgfsetroundjoin%
\pgfsetlinewidth{0.803000pt}%
\definecolor{currentstroke}{rgb}{0.600000,0.600000,0.600000}%
\pgfsetstrokecolor{currentstroke}%
\pgfsetstrokeopacity{0.200000}%
\pgfsetdash{}{0pt}%
\pgfpathmoveto{\pgfqpoint{0.640323in}{6.886146in}}%
\pgfpathlineto{\pgfqpoint{10.327822in}{6.886146in}}%
\pgfusepath{stroke}%
\end{pgfscope}%
\begin{pgfscope}%
\pgfsetbuttcap%
\pgfsetroundjoin%
\definecolor{currentfill}{rgb}{0.000000,0.000000,0.000000}%
\pgfsetfillcolor{currentfill}%
\pgfsetlinewidth{0.602250pt}%
\definecolor{currentstroke}{rgb}{0.000000,0.000000,0.000000}%
\pgfsetstrokecolor{currentstroke}%
\pgfsetdash{}{0pt}%
\pgfsys@defobject{currentmarker}{\pgfqpoint{-0.027778in}{0.000000in}}{\pgfqpoint{-0.000000in}{0.000000in}}{%
\pgfpathmoveto{\pgfqpoint{-0.000000in}{0.000000in}}%
\pgfpathlineto{\pgfqpoint{-0.027778in}{0.000000in}}%
\pgfusepath{stroke,fill}%
}%
\begin{pgfscope}%
\pgfsys@transformshift{0.640323in}{6.886146in}%
\pgfsys@useobject{currentmarker}{}%
\end{pgfscope}%
\end{pgfscope}%
\begin{pgfscope}%
\pgfpathrectangle{\pgfqpoint{0.640323in}{5.147436in}}{\pgfqpoint{9.687500in}{3.850000in}}%
\pgfusepath{clip}%
\pgfsetrectcap%
\pgfsetroundjoin%
\pgfsetlinewidth{0.803000pt}%
\definecolor{currentstroke}{rgb}{0.600000,0.600000,0.600000}%
\pgfsetstrokecolor{currentstroke}%
\pgfsetstrokeopacity{0.200000}%
\pgfsetdash{}{0pt}%
\pgfpathmoveto{\pgfqpoint{0.640323in}{7.010339in}}%
\pgfpathlineto{\pgfqpoint{10.327822in}{7.010339in}}%
\pgfusepath{stroke}%
\end{pgfscope}%
\begin{pgfscope}%
\pgfsetbuttcap%
\pgfsetroundjoin%
\definecolor{currentfill}{rgb}{0.000000,0.000000,0.000000}%
\pgfsetfillcolor{currentfill}%
\pgfsetlinewidth{0.602250pt}%
\definecolor{currentstroke}{rgb}{0.000000,0.000000,0.000000}%
\pgfsetstrokecolor{currentstroke}%
\pgfsetdash{}{0pt}%
\pgfsys@defobject{currentmarker}{\pgfqpoint{-0.027778in}{0.000000in}}{\pgfqpoint{-0.000000in}{0.000000in}}{%
\pgfpathmoveto{\pgfqpoint{-0.000000in}{0.000000in}}%
\pgfpathlineto{\pgfqpoint{-0.027778in}{0.000000in}}%
\pgfusepath{stroke,fill}%
}%
\begin{pgfscope}%
\pgfsys@transformshift{0.640323in}{7.010339in}%
\pgfsys@useobject{currentmarker}{}%
\end{pgfscope}%
\end{pgfscope}%
\begin{pgfscope}%
\pgfpathrectangle{\pgfqpoint{0.640323in}{5.147436in}}{\pgfqpoint{9.687500in}{3.850000in}}%
\pgfusepath{clip}%
\pgfsetrectcap%
\pgfsetroundjoin%
\pgfsetlinewidth{0.803000pt}%
\definecolor{currentstroke}{rgb}{0.600000,0.600000,0.600000}%
\pgfsetstrokecolor{currentstroke}%
\pgfsetstrokeopacity{0.200000}%
\pgfsetdash{}{0pt}%
\pgfpathmoveto{\pgfqpoint{0.640323in}{7.258726in}}%
\pgfpathlineto{\pgfqpoint{10.327822in}{7.258726in}}%
\pgfusepath{stroke}%
\end{pgfscope}%
\begin{pgfscope}%
\pgfsetbuttcap%
\pgfsetroundjoin%
\definecolor{currentfill}{rgb}{0.000000,0.000000,0.000000}%
\pgfsetfillcolor{currentfill}%
\pgfsetlinewidth{0.602250pt}%
\definecolor{currentstroke}{rgb}{0.000000,0.000000,0.000000}%
\pgfsetstrokecolor{currentstroke}%
\pgfsetdash{}{0pt}%
\pgfsys@defobject{currentmarker}{\pgfqpoint{-0.027778in}{0.000000in}}{\pgfqpoint{-0.000000in}{0.000000in}}{%
\pgfpathmoveto{\pgfqpoint{-0.000000in}{0.000000in}}%
\pgfpathlineto{\pgfqpoint{-0.027778in}{0.000000in}}%
\pgfusepath{stroke,fill}%
}%
\begin{pgfscope}%
\pgfsys@transformshift{0.640323in}{7.258726in}%
\pgfsys@useobject{currentmarker}{}%
\end{pgfscope}%
\end{pgfscope}%
\begin{pgfscope}%
\pgfpathrectangle{\pgfqpoint{0.640323in}{5.147436in}}{\pgfqpoint{9.687500in}{3.850000in}}%
\pgfusepath{clip}%
\pgfsetrectcap%
\pgfsetroundjoin%
\pgfsetlinewidth{0.803000pt}%
\definecolor{currentstroke}{rgb}{0.600000,0.600000,0.600000}%
\pgfsetstrokecolor{currentstroke}%
\pgfsetstrokeopacity{0.200000}%
\pgfsetdash{}{0pt}%
\pgfpathmoveto{\pgfqpoint{0.640323in}{7.382920in}}%
\pgfpathlineto{\pgfqpoint{10.327822in}{7.382920in}}%
\pgfusepath{stroke}%
\end{pgfscope}%
\begin{pgfscope}%
\pgfsetbuttcap%
\pgfsetroundjoin%
\definecolor{currentfill}{rgb}{0.000000,0.000000,0.000000}%
\pgfsetfillcolor{currentfill}%
\pgfsetlinewidth{0.602250pt}%
\definecolor{currentstroke}{rgb}{0.000000,0.000000,0.000000}%
\pgfsetstrokecolor{currentstroke}%
\pgfsetdash{}{0pt}%
\pgfsys@defobject{currentmarker}{\pgfqpoint{-0.027778in}{0.000000in}}{\pgfqpoint{-0.000000in}{0.000000in}}{%
\pgfpathmoveto{\pgfqpoint{-0.000000in}{0.000000in}}%
\pgfpathlineto{\pgfqpoint{-0.027778in}{0.000000in}}%
\pgfusepath{stroke,fill}%
}%
\begin{pgfscope}%
\pgfsys@transformshift{0.640323in}{7.382920in}%
\pgfsys@useobject{currentmarker}{}%
\end{pgfscope}%
\end{pgfscope}%
\begin{pgfscope}%
\pgfpathrectangle{\pgfqpoint{0.640323in}{5.147436in}}{\pgfqpoint{9.687500in}{3.850000in}}%
\pgfusepath{clip}%
\pgfsetrectcap%
\pgfsetroundjoin%
\pgfsetlinewidth{0.803000pt}%
\definecolor{currentstroke}{rgb}{0.600000,0.600000,0.600000}%
\pgfsetstrokecolor{currentstroke}%
\pgfsetstrokeopacity{0.200000}%
\pgfsetdash{}{0pt}%
\pgfpathmoveto{\pgfqpoint{0.640323in}{7.507113in}}%
\pgfpathlineto{\pgfqpoint{10.327822in}{7.507113in}}%
\pgfusepath{stroke}%
\end{pgfscope}%
\begin{pgfscope}%
\pgfsetbuttcap%
\pgfsetroundjoin%
\definecolor{currentfill}{rgb}{0.000000,0.000000,0.000000}%
\pgfsetfillcolor{currentfill}%
\pgfsetlinewidth{0.602250pt}%
\definecolor{currentstroke}{rgb}{0.000000,0.000000,0.000000}%
\pgfsetstrokecolor{currentstroke}%
\pgfsetdash{}{0pt}%
\pgfsys@defobject{currentmarker}{\pgfqpoint{-0.027778in}{0.000000in}}{\pgfqpoint{-0.000000in}{0.000000in}}{%
\pgfpathmoveto{\pgfqpoint{-0.000000in}{0.000000in}}%
\pgfpathlineto{\pgfqpoint{-0.027778in}{0.000000in}}%
\pgfusepath{stroke,fill}%
}%
\begin{pgfscope}%
\pgfsys@transformshift{0.640323in}{7.507113in}%
\pgfsys@useobject{currentmarker}{}%
\end{pgfscope}%
\end{pgfscope}%
\begin{pgfscope}%
\pgfpathrectangle{\pgfqpoint{0.640323in}{5.147436in}}{\pgfqpoint{9.687500in}{3.850000in}}%
\pgfusepath{clip}%
\pgfsetrectcap%
\pgfsetroundjoin%
\pgfsetlinewidth{0.803000pt}%
\definecolor{currentstroke}{rgb}{0.600000,0.600000,0.600000}%
\pgfsetstrokecolor{currentstroke}%
\pgfsetstrokeopacity{0.200000}%
\pgfsetdash{}{0pt}%
\pgfpathmoveto{\pgfqpoint{0.640323in}{7.631307in}}%
\pgfpathlineto{\pgfqpoint{10.327822in}{7.631307in}}%
\pgfusepath{stroke}%
\end{pgfscope}%
\begin{pgfscope}%
\pgfsetbuttcap%
\pgfsetroundjoin%
\definecolor{currentfill}{rgb}{0.000000,0.000000,0.000000}%
\pgfsetfillcolor{currentfill}%
\pgfsetlinewidth{0.602250pt}%
\definecolor{currentstroke}{rgb}{0.000000,0.000000,0.000000}%
\pgfsetstrokecolor{currentstroke}%
\pgfsetdash{}{0pt}%
\pgfsys@defobject{currentmarker}{\pgfqpoint{-0.027778in}{0.000000in}}{\pgfqpoint{-0.000000in}{0.000000in}}{%
\pgfpathmoveto{\pgfqpoint{-0.000000in}{0.000000in}}%
\pgfpathlineto{\pgfqpoint{-0.027778in}{0.000000in}}%
\pgfusepath{stroke,fill}%
}%
\begin{pgfscope}%
\pgfsys@transformshift{0.640323in}{7.631307in}%
\pgfsys@useobject{currentmarker}{}%
\end{pgfscope}%
\end{pgfscope}%
\begin{pgfscope}%
\pgfpathrectangle{\pgfqpoint{0.640323in}{5.147436in}}{\pgfqpoint{9.687500in}{3.850000in}}%
\pgfusepath{clip}%
\pgfsetrectcap%
\pgfsetroundjoin%
\pgfsetlinewidth{0.803000pt}%
\definecolor{currentstroke}{rgb}{0.600000,0.600000,0.600000}%
\pgfsetstrokecolor{currentstroke}%
\pgfsetstrokeopacity{0.200000}%
\pgfsetdash{}{0pt}%
\pgfpathmoveto{\pgfqpoint{0.640323in}{7.879694in}}%
\pgfpathlineto{\pgfqpoint{10.327822in}{7.879694in}}%
\pgfusepath{stroke}%
\end{pgfscope}%
\begin{pgfscope}%
\pgfsetbuttcap%
\pgfsetroundjoin%
\definecolor{currentfill}{rgb}{0.000000,0.000000,0.000000}%
\pgfsetfillcolor{currentfill}%
\pgfsetlinewidth{0.602250pt}%
\definecolor{currentstroke}{rgb}{0.000000,0.000000,0.000000}%
\pgfsetstrokecolor{currentstroke}%
\pgfsetdash{}{0pt}%
\pgfsys@defobject{currentmarker}{\pgfqpoint{-0.027778in}{0.000000in}}{\pgfqpoint{-0.000000in}{0.000000in}}{%
\pgfpathmoveto{\pgfqpoint{-0.000000in}{0.000000in}}%
\pgfpathlineto{\pgfqpoint{-0.027778in}{0.000000in}}%
\pgfusepath{stroke,fill}%
}%
\begin{pgfscope}%
\pgfsys@transformshift{0.640323in}{7.879694in}%
\pgfsys@useobject{currentmarker}{}%
\end{pgfscope}%
\end{pgfscope}%
\begin{pgfscope}%
\pgfpathrectangle{\pgfqpoint{0.640323in}{5.147436in}}{\pgfqpoint{9.687500in}{3.850000in}}%
\pgfusepath{clip}%
\pgfsetrectcap%
\pgfsetroundjoin%
\pgfsetlinewidth{0.803000pt}%
\definecolor{currentstroke}{rgb}{0.600000,0.600000,0.600000}%
\pgfsetstrokecolor{currentstroke}%
\pgfsetstrokeopacity{0.200000}%
\pgfsetdash{}{0pt}%
\pgfpathmoveto{\pgfqpoint{0.640323in}{8.003887in}}%
\pgfpathlineto{\pgfqpoint{10.327822in}{8.003887in}}%
\pgfusepath{stroke}%
\end{pgfscope}%
\begin{pgfscope}%
\pgfsetbuttcap%
\pgfsetroundjoin%
\definecolor{currentfill}{rgb}{0.000000,0.000000,0.000000}%
\pgfsetfillcolor{currentfill}%
\pgfsetlinewidth{0.602250pt}%
\definecolor{currentstroke}{rgb}{0.000000,0.000000,0.000000}%
\pgfsetstrokecolor{currentstroke}%
\pgfsetdash{}{0pt}%
\pgfsys@defobject{currentmarker}{\pgfqpoint{-0.027778in}{0.000000in}}{\pgfqpoint{-0.000000in}{0.000000in}}{%
\pgfpathmoveto{\pgfqpoint{-0.000000in}{0.000000in}}%
\pgfpathlineto{\pgfqpoint{-0.027778in}{0.000000in}}%
\pgfusepath{stroke,fill}%
}%
\begin{pgfscope}%
\pgfsys@transformshift{0.640323in}{8.003887in}%
\pgfsys@useobject{currentmarker}{}%
\end{pgfscope}%
\end{pgfscope}%
\begin{pgfscope}%
\pgfpathrectangle{\pgfqpoint{0.640323in}{5.147436in}}{\pgfqpoint{9.687500in}{3.850000in}}%
\pgfusepath{clip}%
\pgfsetrectcap%
\pgfsetroundjoin%
\pgfsetlinewidth{0.803000pt}%
\definecolor{currentstroke}{rgb}{0.600000,0.600000,0.600000}%
\pgfsetstrokecolor{currentstroke}%
\pgfsetstrokeopacity{0.200000}%
\pgfsetdash{}{0pt}%
\pgfpathmoveto{\pgfqpoint{0.640323in}{8.128081in}}%
\pgfpathlineto{\pgfqpoint{10.327822in}{8.128081in}}%
\pgfusepath{stroke}%
\end{pgfscope}%
\begin{pgfscope}%
\pgfsetbuttcap%
\pgfsetroundjoin%
\definecolor{currentfill}{rgb}{0.000000,0.000000,0.000000}%
\pgfsetfillcolor{currentfill}%
\pgfsetlinewidth{0.602250pt}%
\definecolor{currentstroke}{rgb}{0.000000,0.000000,0.000000}%
\pgfsetstrokecolor{currentstroke}%
\pgfsetdash{}{0pt}%
\pgfsys@defobject{currentmarker}{\pgfqpoint{-0.027778in}{0.000000in}}{\pgfqpoint{-0.000000in}{0.000000in}}{%
\pgfpathmoveto{\pgfqpoint{-0.000000in}{0.000000in}}%
\pgfpathlineto{\pgfqpoint{-0.027778in}{0.000000in}}%
\pgfusepath{stroke,fill}%
}%
\begin{pgfscope}%
\pgfsys@transformshift{0.640323in}{8.128081in}%
\pgfsys@useobject{currentmarker}{}%
\end{pgfscope}%
\end{pgfscope}%
\begin{pgfscope}%
\pgfpathrectangle{\pgfqpoint{0.640323in}{5.147436in}}{\pgfqpoint{9.687500in}{3.850000in}}%
\pgfusepath{clip}%
\pgfsetrectcap%
\pgfsetroundjoin%
\pgfsetlinewidth{0.803000pt}%
\definecolor{currentstroke}{rgb}{0.600000,0.600000,0.600000}%
\pgfsetstrokecolor{currentstroke}%
\pgfsetstrokeopacity{0.200000}%
\pgfsetdash{}{0pt}%
\pgfpathmoveto{\pgfqpoint{0.640323in}{8.252275in}}%
\pgfpathlineto{\pgfqpoint{10.327822in}{8.252275in}}%
\pgfusepath{stroke}%
\end{pgfscope}%
\begin{pgfscope}%
\pgfsetbuttcap%
\pgfsetroundjoin%
\definecolor{currentfill}{rgb}{0.000000,0.000000,0.000000}%
\pgfsetfillcolor{currentfill}%
\pgfsetlinewidth{0.602250pt}%
\definecolor{currentstroke}{rgb}{0.000000,0.000000,0.000000}%
\pgfsetstrokecolor{currentstroke}%
\pgfsetdash{}{0pt}%
\pgfsys@defobject{currentmarker}{\pgfqpoint{-0.027778in}{0.000000in}}{\pgfqpoint{-0.000000in}{0.000000in}}{%
\pgfpathmoveto{\pgfqpoint{-0.000000in}{0.000000in}}%
\pgfpathlineto{\pgfqpoint{-0.027778in}{0.000000in}}%
\pgfusepath{stroke,fill}%
}%
\begin{pgfscope}%
\pgfsys@transformshift{0.640323in}{8.252275in}%
\pgfsys@useobject{currentmarker}{}%
\end{pgfscope}%
\end{pgfscope}%
\begin{pgfscope}%
\pgfpathrectangle{\pgfqpoint{0.640323in}{5.147436in}}{\pgfqpoint{9.687500in}{3.850000in}}%
\pgfusepath{clip}%
\pgfsetrectcap%
\pgfsetroundjoin%
\pgfsetlinewidth{0.803000pt}%
\definecolor{currentstroke}{rgb}{0.600000,0.600000,0.600000}%
\pgfsetstrokecolor{currentstroke}%
\pgfsetstrokeopacity{0.200000}%
\pgfsetdash{}{0pt}%
\pgfpathmoveto{\pgfqpoint{0.640323in}{8.500662in}}%
\pgfpathlineto{\pgfqpoint{10.327822in}{8.500662in}}%
\pgfusepath{stroke}%
\end{pgfscope}%
\begin{pgfscope}%
\pgfsetbuttcap%
\pgfsetroundjoin%
\definecolor{currentfill}{rgb}{0.000000,0.000000,0.000000}%
\pgfsetfillcolor{currentfill}%
\pgfsetlinewidth{0.602250pt}%
\definecolor{currentstroke}{rgb}{0.000000,0.000000,0.000000}%
\pgfsetstrokecolor{currentstroke}%
\pgfsetdash{}{0pt}%
\pgfsys@defobject{currentmarker}{\pgfqpoint{-0.027778in}{0.000000in}}{\pgfqpoint{-0.000000in}{0.000000in}}{%
\pgfpathmoveto{\pgfqpoint{-0.000000in}{0.000000in}}%
\pgfpathlineto{\pgfqpoint{-0.027778in}{0.000000in}}%
\pgfusepath{stroke,fill}%
}%
\begin{pgfscope}%
\pgfsys@transformshift{0.640323in}{8.500662in}%
\pgfsys@useobject{currentmarker}{}%
\end{pgfscope}%
\end{pgfscope}%
\begin{pgfscope}%
\pgfpathrectangle{\pgfqpoint{0.640323in}{5.147436in}}{\pgfqpoint{9.687500in}{3.850000in}}%
\pgfusepath{clip}%
\pgfsetrectcap%
\pgfsetroundjoin%
\pgfsetlinewidth{0.803000pt}%
\definecolor{currentstroke}{rgb}{0.600000,0.600000,0.600000}%
\pgfsetstrokecolor{currentstroke}%
\pgfsetstrokeopacity{0.200000}%
\pgfsetdash{}{0pt}%
\pgfpathmoveto{\pgfqpoint{0.640323in}{8.624855in}}%
\pgfpathlineto{\pgfqpoint{10.327822in}{8.624855in}}%
\pgfusepath{stroke}%
\end{pgfscope}%
\begin{pgfscope}%
\pgfsetbuttcap%
\pgfsetroundjoin%
\definecolor{currentfill}{rgb}{0.000000,0.000000,0.000000}%
\pgfsetfillcolor{currentfill}%
\pgfsetlinewidth{0.602250pt}%
\definecolor{currentstroke}{rgb}{0.000000,0.000000,0.000000}%
\pgfsetstrokecolor{currentstroke}%
\pgfsetdash{}{0pt}%
\pgfsys@defobject{currentmarker}{\pgfqpoint{-0.027778in}{0.000000in}}{\pgfqpoint{-0.000000in}{0.000000in}}{%
\pgfpathmoveto{\pgfqpoint{-0.000000in}{0.000000in}}%
\pgfpathlineto{\pgfqpoint{-0.027778in}{0.000000in}}%
\pgfusepath{stroke,fill}%
}%
\begin{pgfscope}%
\pgfsys@transformshift{0.640323in}{8.624855in}%
\pgfsys@useobject{currentmarker}{}%
\end{pgfscope}%
\end{pgfscope}%
\begin{pgfscope}%
\pgfpathrectangle{\pgfqpoint{0.640323in}{5.147436in}}{\pgfqpoint{9.687500in}{3.850000in}}%
\pgfusepath{clip}%
\pgfsetrectcap%
\pgfsetroundjoin%
\pgfsetlinewidth{0.803000pt}%
\definecolor{currentstroke}{rgb}{0.600000,0.600000,0.600000}%
\pgfsetstrokecolor{currentstroke}%
\pgfsetstrokeopacity{0.200000}%
\pgfsetdash{}{0pt}%
\pgfpathmoveto{\pgfqpoint{0.640323in}{8.749049in}}%
\pgfpathlineto{\pgfqpoint{10.327822in}{8.749049in}}%
\pgfusepath{stroke}%
\end{pgfscope}%
\begin{pgfscope}%
\pgfsetbuttcap%
\pgfsetroundjoin%
\definecolor{currentfill}{rgb}{0.000000,0.000000,0.000000}%
\pgfsetfillcolor{currentfill}%
\pgfsetlinewidth{0.602250pt}%
\definecolor{currentstroke}{rgb}{0.000000,0.000000,0.000000}%
\pgfsetstrokecolor{currentstroke}%
\pgfsetdash{}{0pt}%
\pgfsys@defobject{currentmarker}{\pgfqpoint{-0.027778in}{0.000000in}}{\pgfqpoint{-0.000000in}{0.000000in}}{%
\pgfpathmoveto{\pgfqpoint{-0.000000in}{0.000000in}}%
\pgfpathlineto{\pgfqpoint{-0.027778in}{0.000000in}}%
\pgfusepath{stroke,fill}%
}%
\begin{pgfscope}%
\pgfsys@transformshift{0.640323in}{8.749049in}%
\pgfsys@useobject{currentmarker}{}%
\end{pgfscope}%
\end{pgfscope}%
\begin{pgfscope}%
\pgfpathrectangle{\pgfqpoint{0.640323in}{5.147436in}}{\pgfqpoint{9.687500in}{3.850000in}}%
\pgfusepath{clip}%
\pgfsetrectcap%
\pgfsetroundjoin%
\pgfsetlinewidth{0.803000pt}%
\definecolor{currentstroke}{rgb}{0.600000,0.600000,0.600000}%
\pgfsetstrokecolor{currentstroke}%
\pgfsetstrokeopacity{0.200000}%
\pgfsetdash{}{0pt}%
\pgfpathmoveto{\pgfqpoint{0.640323in}{8.873242in}}%
\pgfpathlineto{\pgfqpoint{10.327822in}{8.873242in}}%
\pgfusepath{stroke}%
\end{pgfscope}%
\begin{pgfscope}%
\pgfsetbuttcap%
\pgfsetroundjoin%
\definecolor{currentfill}{rgb}{0.000000,0.000000,0.000000}%
\pgfsetfillcolor{currentfill}%
\pgfsetlinewidth{0.602250pt}%
\definecolor{currentstroke}{rgb}{0.000000,0.000000,0.000000}%
\pgfsetstrokecolor{currentstroke}%
\pgfsetdash{}{0pt}%
\pgfsys@defobject{currentmarker}{\pgfqpoint{-0.027778in}{0.000000in}}{\pgfqpoint{-0.000000in}{0.000000in}}{%
\pgfpathmoveto{\pgfqpoint{-0.000000in}{0.000000in}}%
\pgfpathlineto{\pgfqpoint{-0.027778in}{0.000000in}}%
\pgfusepath{stroke,fill}%
}%
\begin{pgfscope}%
\pgfsys@transformshift{0.640323in}{8.873242in}%
\pgfsys@useobject{currentmarker}{}%
\end{pgfscope}%
\end{pgfscope}%
\begin{pgfscope}%
\definecolor{textcolor}{rgb}{0.000000,0.000000,0.000000}%
\pgfsetstrokecolor{textcolor}%
\pgfsetfillcolor{textcolor}%
\pgftext[x=0.266665in,y=7.072436in,,bottom,rotate=90.000000]{\color{textcolor}\sffamily\fontsize{10.000000}{12.000000}\selectfont avg. infection rate \(\displaystyle \overline{\langle I\rangle}\)}%
\end{pgfscope}%
\begin{pgfscope}%
\pgfpathrectangle{\pgfqpoint{0.640323in}{5.147436in}}{\pgfqpoint{9.687500in}{3.850000in}}%
\pgfusepath{clip}%
\pgfsetbuttcap%
\pgfsetroundjoin%
\pgfsetlinewidth{1.003750pt}%
\definecolor{currentstroke}{rgb}{0.000000,0.000000,1.000000}%
\pgfsetstrokecolor{currentstroke}%
\pgfsetstrokeopacity{0.500000}%
\pgfsetdash{{3.700000pt}{1.600000pt}}{0.000000pt}%
\pgfpathmoveto{\pgfqpoint{1.080663in}{5.275389in}}%
\pgfpathlineto{\pgfqpoint{1.260394in}{5.275607in}}%
\pgfpathlineto{\pgfqpoint{1.440125in}{5.276505in}}%
\pgfpathlineto{\pgfqpoint{1.619856in}{5.276626in}}%
\pgfpathlineto{\pgfqpoint{1.799587in}{5.277742in}}%
\pgfpathlineto{\pgfqpoint{1.979318in}{5.277378in}}%
\pgfpathlineto{\pgfqpoint{2.159049in}{5.279489in}}%
\pgfpathlineto{\pgfqpoint{2.338780in}{5.278809in}}%
\pgfpathlineto{\pgfqpoint{2.518511in}{5.276917in}}%
\pgfpathlineto{\pgfqpoint{2.698242in}{5.278057in}}%
\pgfpathlineto{\pgfqpoint{2.877973in}{5.277475in}}%
\pgfpathlineto{\pgfqpoint{3.057704in}{5.281162in}}%
\pgfpathlineto{\pgfqpoint{3.237435in}{5.278518in}}%
\pgfpathlineto{\pgfqpoint{3.417166in}{5.280968in}}%
\pgfpathlineto{\pgfqpoint{3.596897in}{5.278785in}}%
\pgfpathlineto{\pgfqpoint{3.776628in}{5.283733in}}%
\pgfpathlineto{\pgfqpoint{3.956359in}{5.283345in}}%
\pgfpathlineto{\pgfqpoint{4.136090in}{5.281235in}}%
\pgfpathlineto{\pgfqpoint{4.315821in}{5.290477in}}%
\pgfpathlineto{\pgfqpoint{4.495552in}{5.279901in}}%
\pgfpathlineto{\pgfqpoint{4.675283in}{5.310513in}}%
\pgfpathlineto{\pgfqpoint{4.855014in}{5.287687in}}%
\pgfpathlineto{\pgfqpoint{5.034745in}{5.279537in}}%
\pgfpathlineto{\pgfqpoint{5.214476in}{5.323684in}}%
\pgfpathlineto{\pgfqpoint{5.394207in}{5.299646in}}%
\pgfpathlineto{\pgfqpoint{5.573938in}{5.309664in}}%
\pgfpathlineto{\pgfqpoint{5.753669in}{5.343817in}}%
\pgfpathlineto{\pgfqpoint{5.933400in}{5.524043in}}%
\pgfpathlineto{\pgfqpoint{6.113131in}{5.310319in}}%
\pgfpathlineto{\pgfqpoint{6.292862in}{5.909891in}}%
\pgfpathlineto{\pgfqpoint{6.472593in}{5.552738in}}%
\pgfpathlineto{\pgfqpoint{6.652324in}{5.463984in}}%
\pgfpathlineto{\pgfqpoint{6.832055in}{5.505948in}}%
\pgfpathlineto{\pgfqpoint{7.011786in}{5.527730in}}%
\pgfpathlineto{\pgfqpoint{7.191517in}{6.071175in}}%
\pgfpathlineto{\pgfqpoint{7.371248in}{5.485451in}}%
\pgfpathlineto{\pgfqpoint{7.550979in}{6.155900in}}%
\pgfpathlineto{\pgfqpoint{7.730710in}{6.217078in}}%
\pgfpathlineto{\pgfqpoint{7.910441in}{6.219506in}}%
\pgfpathlineto{\pgfqpoint{8.090172in}{6.263967in}}%
\pgfpathlineto{\pgfqpoint{8.269903in}{6.319388in}}%
\pgfpathlineto{\pgfqpoint{8.449634in}{6.361304in}}%
\pgfpathlineto{\pgfqpoint{8.629365in}{6.346872in}}%
\pgfpathlineto{\pgfqpoint{8.809096in}{6.368538in}}%
\pgfpathlineto{\pgfqpoint{8.988827in}{6.367929in}}%
\pgfpathlineto{\pgfqpoint{9.168558in}{6.419861in}}%
\pgfpathlineto{\pgfqpoint{9.348289in}{6.462018in}}%
\pgfpathlineto{\pgfqpoint{9.528020in}{6.480647in}}%
\pgfpathlineto{\pgfqpoint{9.707751in}{6.471674in}}%
\pgfpathlineto{\pgfqpoint{9.887482in}{6.488943in}}%
\pgfusepath{stroke}%
\end{pgfscope}%
\begin{pgfscope}%
\pgfpathrectangle{\pgfqpoint{0.640323in}{5.147436in}}{\pgfqpoint{9.687500in}{3.850000in}}%
\pgfusepath{clip}%
\pgfsetbuttcap%
\pgfsetroundjoin%
\pgfsetlinewidth{1.003750pt}%
\definecolor{currentstroke}{rgb}{0.980392,0.164706,0.333333}%
\pgfsetstrokecolor{currentstroke}%
\pgfsetstrokeopacity{0.500000}%
\pgfsetdash{{3.700000pt}{1.600000pt}}{0.000000pt}%
\pgfpathmoveto{\pgfqpoint{1.080663in}{5.276705in}}%
\pgfpathlineto{\pgfqpoint{1.260394in}{5.276469in}}%
\pgfpathlineto{\pgfqpoint{1.440125in}{5.276523in}}%
\pgfpathlineto{\pgfqpoint{1.619856in}{5.276893in}}%
\pgfpathlineto{\pgfqpoint{1.799587in}{5.277433in}}%
\pgfpathlineto{\pgfqpoint{1.979318in}{5.277633in}}%
\pgfpathlineto{\pgfqpoint{2.159049in}{5.277524in}}%
\pgfpathlineto{\pgfqpoint{2.338780in}{5.277809in}}%
\pgfpathlineto{\pgfqpoint{2.518511in}{5.277851in}}%
\pgfpathlineto{\pgfqpoint{2.698242in}{5.278730in}}%
\pgfpathlineto{\pgfqpoint{2.877973in}{5.278785in}}%
\pgfpathlineto{\pgfqpoint{3.057704in}{5.278542in}}%
\pgfpathlineto{\pgfqpoint{3.237435in}{5.280980in}}%
\pgfpathlineto{\pgfqpoint{3.417166in}{5.281532in}}%
\pgfpathlineto{\pgfqpoint{3.596897in}{5.279246in}}%
\pgfpathlineto{\pgfqpoint{3.776628in}{5.284643in}}%
\pgfpathlineto{\pgfqpoint{3.956359in}{5.280544in}}%
\pgfpathlineto{\pgfqpoint{4.136090in}{5.287057in}}%
\pgfpathlineto{\pgfqpoint{4.315821in}{5.283927in}}%
\pgfpathlineto{\pgfqpoint{4.495552in}{5.285977in}}%
\pgfpathlineto{\pgfqpoint{4.675283in}{5.298706in}}%
\pgfpathlineto{\pgfqpoint{4.855014in}{5.304655in}}%
\pgfpathlineto{\pgfqpoint{5.034745in}{5.314369in}}%
\pgfpathlineto{\pgfqpoint{5.214476in}{5.376624in}}%
\pgfpathlineto{\pgfqpoint{5.394207in}{5.585012in}}%
\pgfpathlineto{\pgfqpoint{5.573938in}{5.684949in}}%
\pgfpathlineto{\pgfqpoint{5.753669in}{5.807293in}}%
\pgfpathlineto{\pgfqpoint{5.933400in}{5.842271in}}%
\pgfpathlineto{\pgfqpoint{6.113131in}{5.905811in}}%
\pgfpathlineto{\pgfqpoint{6.292862in}{6.021479in}}%
\pgfpathlineto{\pgfqpoint{6.472593in}{6.038841in}}%
\pgfpathlineto{\pgfqpoint{6.652324in}{6.079912in}}%
\pgfpathlineto{\pgfqpoint{6.832055in}{6.124516in}}%
\pgfpathlineto{\pgfqpoint{7.011786in}{6.162743in}}%
\pgfpathlineto{\pgfqpoint{7.191517in}{6.221120in}}%
\pgfpathlineto{\pgfqpoint{7.371248in}{6.276051in}}%
\pgfpathlineto{\pgfqpoint{7.550979in}{6.268382in}}%
\pgfpathlineto{\pgfqpoint{7.730710in}{6.312328in}}%
\pgfpathlineto{\pgfqpoint{7.910441in}{6.350170in}}%
\pgfpathlineto{\pgfqpoint{8.090172in}{6.373388in}}%
\pgfpathlineto{\pgfqpoint{8.269903in}{6.400598in}}%
\pgfpathlineto{\pgfqpoint{8.449634in}{6.409627in}}%
\pgfpathlineto{\pgfqpoint{8.629365in}{6.434385in}}%
\pgfpathlineto{\pgfqpoint{8.809096in}{6.445445in}}%
\pgfpathlineto{\pgfqpoint{8.988827in}{6.475251in}}%
\pgfpathlineto{\pgfqpoint{9.168558in}{6.490061in}}%
\pgfpathlineto{\pgfqpoint{9.348289in}{6.506995in}}%
\pgfpathlineto{\pgfqpoint{9.528020in}{6.520184in}}%
\pgfpathlineto{\pgfqpoint{9.707751in}{6.544303in}}%
\pgfpathlineto{\pgfqpoint{9.887482in}{6.554959in}}%
\pgfusepath{stroke}%
\end{pgfscope}%
\begin{pgfscope}%
\pgfpathrectangle{\pgfqpoint{0.640323in}{5.147436in}}{\pgfqpoint{9.687500in}{3.850000in}}%
\pgfusepath{clip}%
\pgfsetbuttcap%
\pgfsetroundjoin%
\pgfsetlinewidth{1.003750pt}%
\definecolor{currentstroke}{rgb}{0.239216,0.478431,0.992157}%
\pgfsetstrokecolor{currentstroke}%
\pgfsetstrokeopacity{0.500000}%
\pgfsetdash{{3.700000pt}{1.600000pt}}{0.000000pt}%
\pgfpathmoveto{\pgfqpoint{1.080663in}{5.276250in}}%
\pgfpathlineto{\pgfqpoint{1.260394in}{5.276256in}}%
\pgfpathlineto{\pgfqpoint{1.440125in}{5.276743in}}%
\pgfpathlineto{\pgfqpoint{1.619856in}{5.276764in}}%
\pgfpathlineto{\pgfqpoint{1.799587in}{5.277131in}}%
\pgfpathlineto{\pgfqpoint{1.979318in}{5.277202in}}%
\pgfpathlineto{\pgfqpoint{2.159049in}{5.277747in}}%
\pgfpathlineto{\pgfqpoint{2.338780in}{5.277872in}}%
\pgfpathlineto{\pgfqpoint{2.518511in}{5.278391in}}%
\pgfpathlineto{\pgfqpoint{2.698242in}{5.278438in}}%
\pgfpathlineto{\pgfqpoint{2.877973in}{5.279081in}}%
\pgfpathlineto{\pgfqpoint{3.057704in}{5.279279in}}%
\pgfpathlineto{\pgfqpoint{3.237435in}{5.279899in}}%
\pgfpathlineto{\pgfqpoint{3.417166in}{5.281779in}}%
\pgfpathlineto{\pgfqpoint{3.596897in}{5.282287in}}%
\pgfpathlineto{\pgfqpoint{3.776628in}{5.282116in}}%
\pgfpathlineto{\pgfqpoint{3.956359in}{5.283697in}}%
\pgfpathlineto{\pgfqpoint{4.136090in}{5.284625in}}%
\pgfpathlineto{\pgfqpoint{4.315821in}{5.287903in}}%
\pgfpathlineto{\pgfqpoint{4.495552in}{5.292931in}}%
\pgfpathlineto{\pgfqpoint{4.675283in}{5.300599in}}%
\pgfpathlineto{\pgfqpoint{4.855014in}{5.322623in}}%
\pgfpathlineto{\pgfqpoint{5.034745in}{5.420359in}}%
\pgfpathlineto{\pgfqpoint{5.214476in}{5.518179in}}%
\pgfpathlineto{\pgfqpoint{5.394207in}{5.674770in}}%
\pgfpathlineto{\pgfqpoint{5.573938in}{5.769451in}}%
\pgfpathlineto{\pgfqpoint{5.753669in}{5.871621in}}%
\pgfpathlineto{\pgfqpoint{5.933400in}{5.959183in}}%
\pgfpathlineto{\pgfqpoint{6.113131in}{6.000292in}}%
\pgfpathlineto{\pgfqpoint{6.292862in}{6.066928in}}%
\pgfpathlineto{\pgfqpoint{6.472593in}{6.103111in}}%
\pgfpathlineto{\pgfqpoint{6.652324in}{6.141369in}}%
\pgfpathlineto{\pgfqpoint{6.832055in}{6.191748in}}%
\pgfpathlineto{\pgfqpoint{7.011786in}{6.245009in}}%
\pgfpathlineto{\pgfqpoint{7.191517in}{6.266047in}}%
\pgfpathlineto{\pgfqpoint{7.371248in}{6.301194in}}%
\pgfpathlineto{\pgfqpoint{7.550979in}{6.332093in}}%
\pgfpathlineto{\pgfqpoint{7.730710in}{6.362396in}}%
\pgfpathlineto{\pgfqpoint{7.910441in}{6.373996in}}%
\pgfpathlineto{\pgfqpoint{8.090172in}{6.410559in}}%
\pgfpathlineto{\pgfqpoint{8.269903in}{6.434006in}}%
\pgfpathlineto{\pgfqpoint{8.449634in}{6.460640in}}%
\pgfpathlineto{\pgfqpoint{8.629365in}{6.481896in}}%
\pgfpathlineto{\pgfqpoint{8.809096in}{6.502561in}}%
\pgfpathlineto{\pgfqpoint{8.988827in}{6.510205in}}%
\pgfpathlineto{\pgfqpoint{9.168558in}{6.531380in}}%
\pgfpathlineto{\pgfqpoint{9.348289in}{6.541744in}}%
\pgfpathlineto{\pgfqpoint{9.528020in}{6.555753in}}%
\pgfpathlineto{\pgfqpoint{9.707751in}{6.575891in}}%
\pgfpathlineto{\pgfqpoint{9.887482in}{6.591403in}}%
\pgfusepath{stroke}%
\end{pgfscope}%
\begin{pgfscope}%
\pgfpathrectangle{\pgfqpoint{0.640323in}{5.147436in}}{\pgfqpoint{9.687500in}{3.850000in}}%
\pgfusepath{clip}%
\pgfsetbuttcap%
\pgfsetroundjoin%
\pgfsetlinewidth{1.003750pt}%
\definecolor{currentstroke}{rgb}{0.000000,0.000000,0.000000}%
\pgfsetstrokecolor{currentstroke}%
\pgfsetstrokeopacity{0.500000}%
\pgfsetdash{{3.700000pt}{1.600000pt}}{0.000000pt}%
\pgfpathmoveto{\pgfqpoint{1.080663in}{5.276317in}}%
\pgfpathlineto{\pgfqpoint{1.260394in}{5.276409in}}%
\pgfpathlineto{\pgfqpoint{1.440125in}{5.276751in}}%
\pgfpathlineto{\pgfqpoint{1.619856in}{5.276889in}}%
\pgfpathlineto{\pgfqpoint{1.799587in}{5.277083in}}%
\pgfpathlineto{\pgfqpoint{1.979318in}{5.277261in}}%
\pgfpathlineto{\pgfqpoint{2.159049in}{5.277601in}}%
\pgfpathlineto{\pgfqpoint{2.338780in}{5.277905in}}%
\pgfpathlineto{\pgfqpoint{2.518511in}{5.278229in}}%
\pgfpathlineto{\pgfqpoint{2.698242in}{5.278620in}}%
\pgfpathlineto{\pgfqpoint{2.877973in}{5.278832in}}%
\pgfpathlineto{\pgfqpoint{3.057704in}{5.279708in}}%
\pgfpathlineto{\pgfqpoint{3.237435in}{5.280171in}}%
\pgfpathlineto{\pgfqpoint{3.417166in}{5.280771in}}%
\pgfpathlineto{\pgfqpoint{3.596897in}{5.282038in}}%
\pgfpathlineto{\pgfqpoint{3.776628in}{5.282917in}}%
\pgfpathlineto{\pgfqpoint{3.956359in}{5.283727in}}%
\pgfpathlineto{\pgfqpoint{4.136090in}{5.285470in}}%
\pgfpathlineto{\pgfqpoint{4.315821in}{5.288475in}}%
\pgfpathlineto{\pgfqpoint{4.495552in}{5.291315in}}%
\pgfpathlineto{\pgfqpoint{4.675283in}{5.296769in}}%
\pgfpathlineto{\pgfqpoint{4.855014in}{5.323733in}}%
\pgfpathlineto{\pgfqpoint{5.034745in}{5.428259in}}%
\pgfpathlineto{\pgfqpoint{5.214476in}{5.593213in}}%
\pgfpathlineto{\pgfqpoint{5.394207in}{5.708419in}}%
\pgfpathlineto{\pgfqpoint{5.573938in}{5.825315in}}%
\pgfpathlineto{\pgfqpoint{5.753669in}{5.898707in}}%
\pgfpathlineto{\pgfqpoint{5.933400in}{5.968647in}}%
\pgfpathlineto{\pgfqpoint{6.113131in}{6.037630in}}%
\pgfpathlineto{\pgfqpoint{6.292862in}{6.083787in}}%
\pgfpathlineto{\pgfqpoint{6.472593in}{6.138016in}}%
\pgfpathlineto{\pgfqpoint{6.652324in}{6.177137in}}%
\pgfpathlineto{\pgfqpoint{6.832055in}{6.222561in}}%
\pgfpathlineto{\pgfqpoint{7.011786in}{6.256372in}}%
\pgfpathlineto{\pgfqpoint{7.191517in}{6.291171in}}%
\pgfpathlineto{\pgfqpoint{7.371248in}{6.327728in}}%
\pgfpathlineto{\pgfqpoint{7.550979in}{6.351095in}}%
\pgfpathlineto{\pgfqpoint{7.730710in}{6.377424in}}%
\pgfpathlineto{\pgfqpoint{7.910441in}{6.405759in}}%
\pgfpathlineto{\pgfqpoint{8.090172in}{6.427561in}}%
\pgfpathlineto{\pgfqpoint{8.269903in}{6.450990in}}%
\pgfpathlineto{\pgfqpoint{8.449634in}{6.472084in}}%
\pgfpathlineto{\pgfqpoint{8.629365in}{6.491545in}}%
\pgfpathlineto{\pgfqpoint{8.809096in}{6.511776in}}%
\pgfpathlineto{\pgfqpoint{8.988827in}{6.530474in}}%
\pgfpathlineto{\pgfqpoint{9.168558in}{6.545656in}}%
\pgfpathlineto{\pgfqpoint{9.348289in}{6.564708in}}%
\pgfpathlineto{\pgfqpoint{9.528020in}{6.576997in}}%
\pgfpathlineto{\pgfqpoint{9.707751in}{6.591416in}}%
\pgfpathlineto{\pgfqpoint{9.887482in}{6.607548in}}%
\pgfusepath{stroke}%
\end{pgfscope}%
\begin{pgfscope}%
\pgfsetrectcap%
\pgfsetmiterjoin%
\pgfsetlinewidth{0.803000pt}%
\definecolor{currentstroke}{rgb}{0.000000,0.000000,0.000000}%
\pgfsetstrokecolor{currentstroke}%
\pgfsetdash{}{0pt}%
\pgfpathmoveto{\pgfqpoint{0.640323in}{5.147436in}}%
\pgfpathlineto{\pgfqpoint{0.640323in}{8.997436in}}%
\pgfusepath{stroke}%
\end{pgfscope}%
\begin{pgfscope}%
\pgfsetrectcap%
\pgfsetmiterjoin%
\pgfsetlinewidth{0.803000pt}%
\definecolor{currentstroke}{rgb}{0.000000,0.000000,0.000000}%
\pgfsetstrokecolor{currentstroke}%
\pgfsetdash{}{0pt}%
\pgfpathmoveto{\pgfqpoint{10.327822in}{5.147436in}}%
\pgfpathlineto{\pgfqpoint{10.327822in}{8.997436in}}%
\pgfusepath{stroke}%
\end{pgfscope}%
\begin{pgfscope}%
\pgfsetrectcap%
\pgfsetmiterjoin%
\pgfsetlinewidth{0.803000pt}%
\definecolor{currentstroke}{rgb}{0.000000,0.000000,0.000000}%
\pgfsetstrokecolor{currentstroke}%
\pgfsetdash{}{0pt}%
\pgfpathmoveto{\pgfqpoint{0.640322in}{5.147436in}}%
\pgfpathlineto{\pgfqpoint{10.327823in}{5.147436in}}%
\pgfusepath{stroke}%
\end{pgfscope}%
\begin{pgfscope}%
\pgfsetrectcap%
\pgfsetmiterjoin%
\pgfsetlinewidth{0.803000pt}%
\definecolor{currentstroke}{rgb}{0.000000,0.000000,0.000000}%
\pgfsetstrokecolor{currentstroke}%
\pgfsetdash{}{0pt}%
\pgfpathmoveto{\pgfqpoint{0.640322in}{8.997436in}}%
\pgfpathlineto{\pgfqpoint{10.327823in}{8.997436in}}%
\pgfusepath{stroke}%
\end{pgfscope}%
\begin{pgfscope}%
\definecolor{textcolor}{rgb}{0.000000,0.000000,0.000000}%
\pgfsetstrokecolor{textcolor}%
\pgfsetfillcolor{textcolor}%
\pgftext[x=5.484072in,y=9.080769in,,base]{\color{textcolor}\sffamily\fontsize{12.000000}{14.400000}\selectfont \(\displaystyle \overline{\langle I\rangle}\) over \(\displaystyle p_1\) for \(\displaystyle T=1000\) with \(\displaystyle p_2=0.6\), \(\displaystyle p_3=0.3\)}%
\end{pgfscope}%
\begin{pgfscope}%
\pgfsetbuttcap%
\pgfsetmiterjoin%
\definecolor{currentfill}{rgb}{1.000000,1.000000,1.000000}%
\pgfsetfillcolor{currentfill}%
\pgfsetfillopacity{0.800000}%
\pgfsetlinewidth{1.003750pt}%
\definecolor{currentstroke}{rgb}{0.800000,0.800000,0.800000}%
\pgfsetstrokecolor{currentstroke}%
\pgfsetstrokeopacity{0.800000}%
\pgfsetdash{}{0pt}%
\pgfpathmoveto{\pgfqpoint{0.737545in}{8.070896in}}%
\pgfpathlineto{\pgfqpoint{1.670029in}{8.070896in}}%
\pgfpathquadraticcurveto{\pgfqpoint{1.697806in}{8.070896in}}{\pgfqpoint{1.697806in}{8.098674in}}%
\pgfpathlineto{\pgfqpoint{1.697806in}{8.900214in}}%
\pgfpathquadraticcurveto{\pgfqpoint{1.697806in}{8.927991in}}{\pgfqpoint{1.670029in}{8.927991in}}%
\pgfpathlineto{\pgfqpoint{0.737545in}{8.927991in}}%
\pgfpathquadraticcurveto{\pgfqpoint{0.709767in}{8.927991in}}{\pgfqpoint{0.709767in}{8.900214in}}%
\pgfpathlineto{\pgfqpoint{0.709767in}{8.098674in}}%
\pgfpathquadraticcurveto{\pgfqpoint{0.709767in}{8.070896in}}{\pgfqpoint{0.737545in}{8.070896in}}%
\pgfpathlineto{\pgfqpoint{0.737545in}{8.070896in}}%
\pgfpathclose%
\pgfusepath{stroke,fill}%
\end{pgfscope}%
\begin{pgfscope}%
\pgfsetbuttcap%
\pgfsetroundjoin%
\definecolor{currentfill}{rgb}{0.000000,0.000000,1.000000}%
\pgfsetfillcolor{currentfill}%
\pgfsetfillopacity{0.500000}%
\pgfsetlinewidth{1.003750pt}%
\definecolor{currentstroke}{rgb}{0.000000,0.000000,1.000000}%
\pgfsetstrokecolor{currentstroke}%
\pgfsetstrokeopacity{0.500000}%
\pgfsetdash{}{0pt}%
\pgfsys@defobject{currentmarker}{\pgfqpoint{-0.021960in}{-0.021960in}}{\pgfqpoint{0.021960in}{0.021960in}}{%
\pgfpathmoveto{\pgfqpoint{0.000000in}{-0.021960in}}%
\pgfpathcurveto{\pgfqpoint{0.005824in}{-0.021960in}}{\pgfqpoint{0.011410in}{-0.019646in}}{\pgfqpoint{0.015528in}{-0.015528in}}%
\pgfpathcurveto{\pgfqpoint{0.019646in}{-0.011410in}}{\pgfqpoint{0.021960in}{-0.005824in}}{\pgfqpoint{0.021960in}{0.000000in}}%
\pgfpathcurveto{\pgfqpoint{0.021960in}{0.005824in}}{\pgfqpoint{0.019646in}{0.011410in}}{\pgfqpoint{0.015528in}{0.015528in}}%
\pgfpathcurveto{\pgfqpoint{0.011410in}{0.019646in}}{\pgfqpoint{0.005824in}{0.021960in}}{\pgfqpoint{0.000000in}{0.021960in}}%
\pgfpathcurveto{\pgfqpoint{-0.005824in}{0.021960in}}{\pgfqpoint{-0.011410in}{0.019646in}}{\pgfqpoint{-0.015528in}{0.015528in}}%
\pgfpathcurveto{\pgfqpoint{-0.019646in}{0.011410in}}{\pgfqpoint{-0.021960in}{0.005824in}}{\pgfqpoint{-0.021960in}{0.000000in}}%
\pgfpathcurveto{\pgfqpoint{-0.021960in}{-0.005824in}}{\pgfqpoint{-0.019646in}{-0.011410in}}{\pgfqpoint{-0.015528in}{-0.015528in}}%
\pgfpathcurveto{\pgfqpoint{-0.011410in}{-0.019646in}}{\pgfqpoint{-0.005824in}{-0.021960in}}{\pgfqpoint{0.000000in}{-0.021960in}}%
\pgfpathlineto{\pgfqpoint{0.000000in}{-0.021960in}}%
\pgfpathclose%
\pgfusepath{stroke,fill}%
}%
\begin{pgfscope}%
\pgfsys@transformshift{0.904211in}{8.803371in}%
\pgfsys@useobject{currentmarker}{}%
\end{pgfscope}%
\end{pgfscope}%
\begin{pgfscope}%
\definecolor{textcolor}{rgb}{0.000000,0.000000,0.000000}%
\pgfsetstrokecolor{textcolor}%
\pgfsetfillcolor{textcolor}%
\pgftext[x=1.154211in,y=8.766913in,left,base]{\color{textcolor}\sffamily\fontsize{10.000000}{12.000000}\selectfont \(\displaystyle L=16\)}%
\end{pgfscope}%
\begin{pgfscope}%
\pgfsetbuttcap%
\pgfsetroundjoin%
\definecolor{currentfill}{rgb}{0.980392,0.164706,0.333333}%
\pgfsetfillcolor{currentfill}%
\pgfsetfillopacity{0.500000}%
\pgfsetlinewidth{1.003750pt}%
\definecolor{currentstroke}{rgb}{0.980392,0.164706,0.333333}%
\pgfsetstrokecolor{currentstroke}%
\pgfsetstrokeopacity{0.500000}%
\pgfsetdash{}{0pt}%
\pgfsys@defobject{currentmarker}{\pgfqpoint{-0.021960in}{-0.021960in}}{\pgfqpoint{0.021960in}{0.021960in}}{%
\pgfpathmoveto{\pgfqpoint{0.000000in}{-0.021960in}}%
\pgfpathcurveto{\pgfqpoint{0.005824in}{-0.021960in}}{\pgfqpoint{0.011410in}{-0.019646in}}{\pgfqpoint{0.015528in}{-0.015528in}}%
\pgfpathcurveto{\pgfqpoint{0.019646in}{-0.011410in}}{\pgfqpoint{0.021960in}{-0.005824in}}{\pgfqpoint{0.021960in}{0.000000in}}%
\pgfpathcurveto{\pgfqpoint{0.021960in}{0.005824in}}{\pgfqpoint{0.019646in}{0.011410in}}{\pgfqpoint{0.015528in}{0.015528in}}%
\pgfpathcurveto{\pgfqpoint{0.011410in}{0.019646in}}{\pgfqpoint{0.005824in}{0.021960in}}{\pgfqpoint{0.000000in}{0.021960in}}%
\pgfpathcurveto{\pgfqpoint{-0.005824in}{0.021960in}}{\pgfqpoint{-0.011410in}{0.019646in}}{\pgfqpoint{-0.015528in}{0.015528in}}%
\pgfpathcurveto{\pgfqpoint{-0.019646in}{0.011410in}}{\pgfqpoint{-0.021960in}{0.005824in}}{\pgfqpoint{-0.021960in}{0.000000in}}%
\pgfpathcurveto{\pgfqpoint{-0.021960in}{-0.005824in}}{\pgfqpoint{-0.019646in}{-0.011410in}}{\pgfqpoint{-0.015528in}{-0.015528in}}%
\pgfpathcurveto{\pgfqpoint{-0.011410in}{-0.019646in}}{\pgfqpoint{-0.005824in}{-0.021960in}}{\pgfqpoint{0.000000in}{-0.021960in}}%
\pgfpathlineto{\pgfqpoint{0.000000in}{-0.021960in}}%
\pgfpathclose%
\pgfusepath{stroke,fill}%
}%
\begin{pgfscope}%
\pgfsys@transformshift{0.904211in}{8.599514in}%
\pgfsys@useobject{currentmarker}{}%
\end{pgfscope}%
\end{pgfscope}%
\begin{pgfscope}%
\definecolor{textcolor}{rgb}{0.000000,0.000000,0.000000}%
\pgfsetstrokecolor{textcolor}%
\pgfsetfillcolor{textcolor}%
\pgftext[x=1.154211in,y=8.563056in,left,base]{\color{textcolor}\sffamily\fontsize{10.000000}{12.000000}\selectfont \(\displaystyle L=32\)}%
\end{pgfscope}%
\begin{pgfscope}%
\pgfsetbuttcap%
\pgfsetroundjoin%
\definecolor{currentfill}{rgb}{0.239216,0.478431,0.992157}%
\pgfsetfillcolor{currentfill}%
\pgfsetfillopacity{0.500000}%
\pgfsetlinewidth{1.003750pt}%
\definecolor{currentstroke}{rgb}{0.239216,0.478431,0.992157}%
\pgfsetstrokecolor{currentstroke}%
\pgfsetstrokeopacity{0.500000}%
\pgfsetdash{}{0pt}%
\pgfsys@defobject{currentmarker}{\pgfqpoint{-0.021960in}{-0.021960in}}{\pgfqpoint{0.021960in}{0.021960in}}{%
\pgfpathmoveto{\pgfqpoint{0.000000in}{-0.021960in}}%
\pgfpathcurveto{\pgfqpoint{0.005824in}{-0.021960in}}{\pgfqpoint{0.011410in}{-0.019646in}}{\pgfqpoint{0.015528in}{-0.015528in}}%
\pgfpathcurveto{\pgfqpoint{0.019646in}{-0.011410in}}{\pgfqpoint{0.021960in}{-0.005824in}}{\pgfqpoint{0.021960in}{0.000000in}}%
\pgfpathcurveto{\pgfqpoint{0.021960in}{0.005824in}}{\pgfqpoint{0.019646in}{0.011410in}}{\pgfqpoint{0.015528in}{0.015528in}}%
\pgfpathcurveto{\pgfqpoint{0.011410in}{0.019646in}}{\pgfqpoint{0.005824in}{0.021960in}}{\pgfqpoint{0.000000in}{0.021960in}}%
\pgfpathcurveto{\pgfqpoint{-0.005824in}{0.021960in}}{\pgfqpoint{-0.011410in}{0.019646in}}{\pgfqpoint{-0.015528in}{0.015528in}}%
\pgfpathcurveto{\pgfqpoint{-0.019646in}{0.011410in}}{\pgfqpoint{-0.021960in}{0.005824in}}{\pgfqpoint{-0.021960in}{0.000000in}}%
\pgfpathcurveto{\pgfqpoint{-0.021960in}{-0.005824in}}{\pgfqpoint{-0.019646in}{-0.011410in}}{\pgfqpoint{-0.015528in}{-0.015528in}}%
\pgfpathcurveto{\pgfqpoint{-0.011410in}{-0.019646in}}{\pgfqpoint{-0.005824in}{-0.021960in}}{\pgfqpoint{0.000000in}{-0.021960in}}%
\pgfpathlineto{\pgfqpoint{0.000000in}{-0.021960in}}%
\pgfpathclose%
\pgfusepath{stroke,fill}%
}%
\begin{pgfscope}%
\pgfsys@transformshift{0.904211in}{8.395657in}%
\pgfsys@useobject{currentmarker}{}%
\end{pgfscope}%
\end{pgfscope}%
\begin{pgfscope}%
\definecolor{textcolor}{rgb}{0.000000,0.000000,0.000000}%
\pgfsetstrokecolor{textcolor}%
\pgfsetfillcolor{textcolor}%
\pgftext[x=1.154211in,y=8.359198in,left,base]{\color{textcolor}\sffamily\fontsize{10.000000}{12.000000}\selectfont \(\displaystyle L=64\)}%
\end{pgfscope}%
\begin{pgfscope}%
\pgfsetbuttcap%
\pgfsetroundjoin%
\definecolor{currentfill}{rgb}{0.000000,0.000000,0.000000}%
\pgfsetfillcolor{currentfill}%
\pgfsetfillopacity{0.500000}%
\pgfsetlinewidth{1.003750pt}%
\definecolor{currentstroke}{rgb}{0.000000,0.000000,0.000000}%
\pgfsetstrokecolor{currentstroke}%
\pgfsetstrokeopacity{0.500000}%
\pgfsetdash{}{0pt}%
\pgfsys@defobject{currentmarker}{\pgfqpoint{-0.021960in}{-0.021960in}}{\pgfqpoint{0.021960in}{0.021960in}}{%
\pgfpathmoveto{\pgfqpoint{0.000000in}{-0.021960in}}%
\pgfpathcurveto{\pgfqpoint{0.005824in}{-0.021960in}}{\pgfqpoint{0.011410in}{-0.019646in}}{\pgfqpoint{0.015528in}{-0.015528in}}%
\pgfpathcurveto{\pgfqpoint{0.019646in}{-0.011410in}}{\pgfqpoint{0.021960in}{-0.005824in}}{\pgfqpoint{0.021960in}{0.000000in}}%
\pgfpathcurveto{\pgfqpoint{0.021960in}{0.005824in}}{\pgfqpoint{0.019646in}{0.011410in}}{\pgfqpoint{0.015528in}{0.015528in}}%
\pgfpathcurveto{\pgfqpoint{0.011410in}{0.019646in}}{\pgfqpoint{0.005824in}{0.021960in}}{\pgfqpoint{0.000000in}{0.021960in}}%
\pgfpathcurveto{\pgfqpoint{-0.005824in}{0.021960in}}{\pgfqpoint{-0.011410in}{0.019646in}}{\pgfqpoint{-0.015528in}{0.015528in}}%
\pgfpathcurveto{\pgfqpoint{-0.019646in}{0.011410in}}{\pgfqpoint{-0.021960in}{0.005824in}}{\pgfqpoint{-0.021960in}{0.000000in}}%
\pgfpathcurveto{\pgfqpoint{-0.021960in}{-0.005824in}}{\pgfqpoint{-0.019646in}{-0.011410in}}{\pgfqpoint{-0.015528in}{-0.015528in}}%
\pgfpathcurveto{\pgfqpoint{-0.011410in}{-0.019646in}}{\pgfqpoint{-0.005824in}{-0.021960in}}{\pgfqpoint{0.000000in}{-0.021960in}}%
\pgfpathlineto{\pgfqpoint{0.000000in}{-0.021960in}}%
\pgfpathclose%
\pgfusepath{stroke,fill}%
}%
\begin{pgfscope}%
\pgfsys@transformshift{0.904211in}{8.191799in}%
\pgfsys@useobject{currentmarker}{}%
\end{pgfscope}%
\end{pgfscope}%
\begin{pgfscope}%
\definecolor{textcolor}{rgb}{0.000000,0.000000,0.000000}%
\pgfsetstrokecolor{textcolor}%
\pgfsetfillcolor{textcolor}%
\pgftext[x=1.154211in,y=8.155341in,left,base]{\color{textcolor}\sffamily\fontsize{10.000000}{12.000000}\selectfont \(\displaystyle L=128\)}%
\end{pgfscope}%
\begin{pgfscope}%
\pgfsetbuttcap%
\pgfsetmiterjoin%
\definecolor{currentfill}{rgb}{1.000000,1.000000,1.000000}%
\pgfsetfillcolor{currentfill}%
\pgfsetlinewidth{0.000000pt}%
\definecolor{currentstroke}{rgb}{0.000000,0.000000,0.000000}%
\pgfsetstrokecolor{currentstroke}%
\pgfsetstrokeopacity{0.000000}%
\pgfsetdash{}{0pt}%
\pgfpathmoveto{\pgfqpoint{0.640323in}{0.527436in}}%
\pgfpathlineto{\pgfqpoint{10.327822in}{0.527436in}}%
\pgfpathlineto{\pgfqpoint{10.327822in}{4.377436in}}%
\pgfpathlineto{\pgfqpoint{0.640323in}{4.377436in}}%
\pgfpathlineto{\pgfqpoint{0.640323in}{0.527436in}}%
\pgfpathclose%
\pgfusepath{fill}%
\end{pgfscope}%
\begin{pgfscope}%
\pgfpathrectangle{\pgfqpoint{0.640323in}{0.527436in}}{\pgfqpoint{9.687500in}{3.850000in}}%
\pgfusepath{clip}%
\pgfsetbuttcap%
\pgfsetroundjoin%
\definecolor{currentfill}{rgb}{0.000000,0.000000,1.000000}%
\pgfsetfillcolor{currentfill}%
\pgfsetfillopacity{0.500000}%
\pgfsetlinewidth{1.003750pt}%
\definecolor{currentstroke}{rgb}{0.000000,0.000000,1.000000}%
\pgfsetstrokecolor{currentstroke}%
\pgfsetstrokeopacity{0.500000}%
\pgfsetdash{{3.700000pt}{1.600000pt}}{0.000000pt}%
\pgfpathmoveto{\pgfqpoint{1.080663in}{0.637261in}}%
\pgfpathcurveto{\pgfqpoint{1.086487in}{0.637261in}}{\pgfqpoint{1.092074in}{0.639575in}}{\pgfqpoint{1.096192in}{0.643693in}}%
\pgfpathcurveto{\pgfqpoint{1.100310in}{0.647812in}}{\pgfqpoint{1.102624in}{0.653398in}}{\pgfqpoint{1.102624in}{0.659222in}}%
\pgfpathcurveto{\pgfqpoint{1.102624in}{0.665046in}}{\pgfqpoint{1.100310in}{0.670632in}}{\pgfqpoint{1.096192in}{0.674750in}}%
\pgfpathcurveto{\pgfqpoint{1.092074in}{0.678868in}}{\pgfqpoint{1.086487in}{0.681182in}}{\pgfqpoint{1.080663in}{0.681182in}}%
\pgfpathcurveto{\pgfqpoint{1.074839in}{0.681182in}}{\pgfqpoint{1.069253in}{0.678868in}}{\pgfqpoint{1.065135in}{0.674750in}}%
\pgfpathcurveto{\pgfqpoint{1.061017in}{0.670632in}}{\pgfqpoint{1.058703in}{0.665046in}}{\pgfqpoint{1.058703in}{0.659222in}}%
\pgfpathcurveto{\pgfqpoint{1.058703in}{0.653398in}}{\pgfqpoint{1.061017in}{0.647812in}}{\pgfqpoint{1.065135in}{0.643693in}}%
\pgfpathcurveto{\pgfqpoint{1.069253in}{0.639575in}}{\pgfqpoint{1.074839in}{0.637261in}}{\pgfqpoint{1.080663in}{0.637261in}}%
\pgfpathlineto{\pgfqpoint{1.080663in}{0.637261in}}%
\pgfpathclose%
\pgfusepath{stroke,fill}%
\end{pgfscope}%
\begin{pgfscope}%
\pgfpathrectangle{\pgfqpoint{0.640323in}{0.527436in}}{\pgfqpoint{9.687500in}{3.850000in}}%
\pgfusepath{clip}%
\pgfsetbuttcap%
\pgfsetroundjoin%
\definecolor{currentfill}{rgb}{0.000000,0.000000,1.000000}%
\pgfsetfillcolor{currentfill}%
\pgfsetfillopacity{0.500000}%
\pgfsetlinewidth{1.003750pt}%
\definecolor{currentstroke}{rgb}{0.000000,0.000000,1.000000}%
\pgfsetstrokecolor{currentstroke}%
\pgfsetstrokeopacity{0.500000}%
\pgfsetdash{{3.700000pt}{1.600000pt}}{0.000000pt}%
\pgfpathmoveto{\pgfqpoint{1.260394in}{0.639857in}}%
\pgfpathcurveto{\pgfqpoint{1.266218in}{0.639857in}}{\pgfqpoint{1.271805in}{0.642171in}}{\pgfqpoint{1.275923in}{0.646289in}}%
\pgfpathcurveto{\pgfqpoint{1.280041in}{0.650407in}}{\pgfqpoint{1.282355in}{0.655993in}}{\pgfqpoint{1.282355in}{0.661817in}}%
\pgfpathcurveto{\pgfqpoint{1.282355in}{0.667641in}}{\pgfqpoint{1.280041in}{0.673227in}}{\pgfqpoint{1.275923in}{0.677345in}}%
\pgfpathcurveto{\pgfqpoint{1.271805in}{0.681463in}}{\pgfqpoint{1.266218in}{0.683777in}}{\pgfqpoint{1.260394in}{0.683777in}}%
\pgfpathcurveto{\pgfqpoint{1.254570in}{0.683777in}}{\pgfqpoint{1.248984in}{0.681463in}}{\pgfqpoint{1.244866in}{0.677345in}}%
\pgfpathcurveto{\pgfqpoint{1.240748in}{0.673227in}}{\pgfqpoint{1.238434in}{0.667641in}}{\pgfqpoint{1.238434in}{0.661817in}}%
\pgfpathcurveto{\pgfqpoint{1.238434in}{0.655993in}}{\pgfqpoint{1.240748in}{0.650407in}}{\pgfqpoint{1.244866in}{0.646289in}}%
\pgfpathcurveto{\pgfqpoint{1.248984in}{0.642171in}}{\pgfqpoint{1.254570in}{0.639857in}}{\pgfqpoint{1.260394in}{0.639857in}}%
\pgfpathlineto{\pgfqpoint{1.260394in}{0.639857in}}%
\pgfpathclose%
\pgfusepath{stroke,fill}%
\end{pgfscope}%
\begin{pgfscope}%
\pgfpathrectangle{\pgfqpoint{0.640323in}{0.527436in}}{\pgfqpoint{9.687500in}{3.850000in}}%
\pgfusepath{clip}%
\pgfsetbuttcap%
\pgfsetroundjoin%
\definecolor{currentfill}{rgb}{0.000000,0.000000,1.000000}%
\pgfsetfillcolor{currentfill}%
\pgfsetfillopacity{0.500000}%
\pgfsetlinewidth{1.003750pt}%
\definecolor{currentstroke}{rgb}{0.000000,0.000000,1.000000}%
\pgfsetstrokecolor{currentstroke}%
\pgfsetstrokeopacity{0.500000}%
\pgfsetdash{{3.700000pt}{1.600000pt}}{0.000000pt}%
\pgfpathmoveto{\pgfqpoint{1.440125in}{0.641991in}}%
\pgfpathcurveto{\pgfqpoint{1.445949in}{0.641991in}}{\pgfqpoint{1.451535in}{0.644305in}}{\pgfqpoint{1.455654in}{0.648423in}}%
\pgfpathcurveto{\pgfqpoint{1.459772in}{0.652542in}}{\pgfqpoint{1.462086in}{0.658128in}}{\pgfqpoint{1.462086in}{0.663952in}}%
\pgfpathcurveto{\pgfqpoint{1.462086in}{0.669776in}}{\pgfqpoint{1.459772in}{0.675362in}}{\pgfqpoint{1.455654in}{0.679480in}}%
\pgfpathcurveto{\pgfqpoint{1.451535in}{0.683598in}}{\pgfqpoint{1.445949in}{0.685912in}}{\pgfqpoint{1.440125in}{0.685912in}}%
\pgfpathcurveto{\pgfqpoint{1.434301in}{0.685912in}}{\pgfqpoint{1.428715in}{0.683598in}}{\pgfqpoint{1.424597in}{0.679480in}}%
\pgfpathcurveto{\pgfqpoint{1.420479in}{0.675362in}}{\pgfqpoint{1.418165in}{0.669776in}}{\pgfqpoint{1.418165in}{0.663952in}}%
\pgfpathcurveto{\pgfqpoint{1.418165in}{0.658128in}}{\pgfqpoint{1.420479in}{0.652542in}}{\pgfqpoint{1.424597in}{0.648423in}}%
\pgfpathcurveto{\pgfqpoint{1.428715in}{0.644305in}}{\pgfqpoint{1.434301in}{0.641991in}}{\pgfqpoint{1.440125in}{0.641991in}}%
\pgfpathlineto{\pgfqpoint{1.440125in}{0.641991in}}%
\pgfpathclose%
\pgfusepath{stroke,fill}%
\end{pgfscope}%
\begin{pgfscope}%
\pgfpathrectangle{\pgfqpoint{0.640323in}{0.527436in}}{\pgfqpoint{9.687500in}{3.850000in}}%
\pgfusepath{clip}%
\pgfsetbuttcap%
\pgfsetroundjoin%
\definecolor{currentfill}{rgb}{0.000000,0.000000,1.000000}%
\pgfsetfillcolor{currentfill}%
\pgfsetfillopacity{0.500000}%
\pgfsetlinewidth{1.003750pt}%
\definecolor{currentstroke}{rgb}{0.000000,0.000000,1.000000}%
\pgfsetstrokecolor{currentstroke}%
\pgfsetstrokeopacity{0.500000}%
\pgfsetdash{{3.700000pt}{1.600000pt}}{0.000000pt}%
\pgfpathmoveto{\pgfqpoint{1.619856in}{0.641167in}}%
\pgfpathcurveto{\pgfqpoint{1.625680in}{0.641167in}}{\pgfqpoint{1.631266in}{0.643481in}}{\pgfqpoint{1.635385in}{0.647599in}}%
\pgfpathcurveto{\pgfqpoint{1.639503in}{0.651717in}}{\pgfqpoint{1.641817in}{0.657303in}}{\pgfqpoint{1.641817in}{0.663127in}}%
\pgfpathcurveto{\pgfqpoint{1.641817in}{0.668951in}}{\pgfqpoint{1.639503in}{0.674537in}}{\pgfqpoint{1.635385in}{0.678655in}}%
\pgfpathcurveto{\pgfqpoint{1.631266in}{0.682773in}}{\pgfqpoint{1.625680in}{0.685087in}}{\pgfqpoint{1.619856in}{0.685087in}}%
\pgfpathcurveto{\pgfqpoint{1.614032in}{0.685087in}}{\pgfqpoint{1.608446in}{0.682773in}}{\pgfqpoint{1.604328in}{0.678655in}}%
\pgfpathcurveto{\pgfqpoint{1.600210in}{0.674537in}}{\pgfqpoint{1.597896in}{0.668951in}}{\pgfqpoint{1.597896in}{0.663127in}}%
\pgfpathcurveto{\pgfqpoint{1.597896in}{0.657303in}}{\pgfqpoint{1.600210in}{0.651717in}}{\pgfqpoint{1.604328in}{0.647599in}}%
\pgfpathcurveto{\pgfqpoint{1.608446in}{0.643481in}}{\pgfqpoint{1.614032in}{0.641167in}}{\pgfqpoint{1.619856in}{0.641167in}}%
\pgfpathlineto{\pgfqpoint{1.619856in}{0.641167in}}%
\pgfpathclose%
\pgfusepath{stroke,fill}%
\end{pgfscope}%
\begin{pgfscope}%
\pgfpathrectangle{\pgfqpoint{0.640323in}{0.527436in}}{\pgfqpoint{9.687500in}{3.850000in}}%
\pgfusepath{clip}%
\pgfsetbuttcap%
\pgfsetroundjoin%
\definecolor{currentfill}{rgb}{0.000000,0.000000,1.000000}%
\pgfsetfillcolor{currentfill}%
\pgfsetfillopacity{0.500000}%
\pgfsetlinewidth{1.003750pt}%
\definecolor{currentstroke}{rgb}{0.000000,0.000000,1.000000}%
\pgfsetstrokecolor{currentstroke}%
\pgfsetstrokeopacity{0.500000}%
\pgfsetdash{{3.700000pt}{1.600000pt}}{0.000000pt}%
\pgfpathmoveto{\pgfqpoint{1.799587in}{0.640027in}}%
\pgfpathcurveto{\pgfqpoint{1.805411in}{0.640027in}}{\pgfqpoint{1.810997in}{0.642341in}}{\pgfqpoint{1.815116in}{0.646459in}}%
\pgfpathcurveto{\pgfqpoint{1.819234in}{0.650577in}}{\pgfqpoint{1.821548in}{0.656163in}}{\pgfqpoint{1.821548in}{0.661987in}}%
\pgfpathcurveto{\pgfqpoint{1.821548in}{0.667811in}}{\pgfqpoint{1.819234in}{0.673397in}}{\pgfqpoint{1.815116in}{0.677515in}}%
\pgfpathcurveto{\pgfqpoint{1.810997in}{0.681633in}}{\pgfqpoint{1.805411in}{0.683947in}}{\pgfqpoint{1.799587in}{0.683947in}}%
\pgfpathcurveto{\pgfqpoint{1.793763in}{0.683947in}}{\pgfqpoint{1.788177in}{0.681633in}}{\pgfqpoint{1.784059in}{0.677515in}}%
\pgfpathcurveto{\pgfqpoint{1.779941in}{0.673397in}}{\pgfqpoint{1.777627in}{0.667811in}}{\pgfqpoint{1.777627in}{0.661987in}}%
\pgfpathcurveto{\pgfqpoint{1.777627in}{0.656163in}}{\pgfqpoint{1.779941in}{0.650577in}}{\pgfqpoint{1.784059in}{0.646459in}}%
\pgfpathcurveto{\pgfqpoint{1.788177in}{0.642341in}}{\pgfqpoint{1.793763in}{0.640027in}}{\pgfqpoint{1.799587in}{0.640027in}}%
\pgfpathlineto{\pgfqpoint{1.799587in}{0.640027in}}%
\pgfpathclose%
\pgfusepath{stroke,fill}%
\end{pgfscope}%
\begin{pgfscope}%
\pgfpathrectangle{\pgfqpoint{0.640323in}{0.527436in}}{\pgfqpoint{9.687500in}{3.850000in}}%
\pgfusepath{clip}%
\pgfsetbuttcap%
\pgfsetroundjoin%
\definecolor{currentfill}{rgb}{0.000000,0.000000,1.000000}%
\pgfsetfillcolor{currentfill}%
\pgfsetfillopacity{0.500000}%
\pgfsetlinewidth{1.003750pt}%
\definecolor{currentstroke}{rgb}{0.000000,0.000000,1.000000}%
\pgfsetstrokecolor{currentstroke}%
\pgfsetstrokeopacity{0.500000}%
\pgfsetdash{{3.700000pt}{1.600000pt}}{0.000000pt}%
\pgfpathmoveto{\pgfqpoint{1.979318in}{0.648128in}}%
\pgfpathcurveto{\pgfqpoint{1.985142in}{0.648128in}}{\pgfqpoint{1.990728in}{0.650442in}}{\pgfqpoint{1.994847in}{0.654560in}}%
\pgfpathcurveto{\pgfqpoint{1.998965in}{0.658679in}}{\pgfqpoint{2.001279in}{0.664265in}}{\pgfqpoint{2.001279in}{0.670089in}}%
\pgfpathcurveto{\pgfqpoint{2.001279in}{0.675913in}}{\pgfqpoint{1.998965in}{0.681499in}}{\pgfqpoint{1.994847in}{0.685617in}}%
\pgfpathcurveto{\pgfqpoint{1.990728in}{0.689735in}}{\pgfqpoint{1.985142in}{0.692049in}}{\pgfqpoint{1.979318in}{0.692049in}}%
\pgfpathcurveto{\pgfqpoint{1.973494in}{0.692049in}}{\pgfqpoint{1.967908in}{0.689735in}}{\pgfqpoint{1.963790in}{0.685617in}}%
\pgfpathcurveto{\pgfqpoint{1.959672in}{0.681499in}}{\pgfqpoint{1.957358in}{0.675913in}}{\pgfqpoint{1.957358in}{0.670089in}}%
\pgfpathcurveto{\pgfqpoint{1.957358in}{0.664265in}}{\pgfqpoint{1.959672in}{0.658679in}}{\pgfqpoint{1.963790in}{0.654560in}}%
\pgfpathcurveto{\pgfqpoint{1.967908in}{0.650442in}}{\pgfqpoint{1.973494in}{0.648128in}}{\pgfqpoint{1.979318in}{0.648128in}}%
\pgfpathlineto{\pgfqpoint{1.979318in}{0.648128in}}%
\pgfpathclose%
\pgfusepath{stroke,fill}%
\end{pgfscope}%
\begin{pgfscope}%
\pgfpathrectangle{\pgfqpoint{0.640323in}{0.527436in}}{\pgfqpoint{9.687500in}{3.850000in}}%
\pgfusepath{clip}%
\pgfsetbuttcap%
\pgfsetroundjoin%
\definecolor{currentfill}{rgb}{0.000000,0.000000,1.000000}%
\pgfsetfillcolor{currentfill}%
\pgfsetfillopacity{0.500000}%
\pgfsetlinewidth{1.003750pt}%
\definecolor{currentstroke}{rgb}{0.000000,0.000000,1.000000}%
\pgfsetstrokecolor{currentstroke}%
\pgfsetstrokeopacity{0.500000}%
\pgfsetdash{{3.700000pt}{1.600000pt}}{0.000000pt}%
\pgfpathmoveto{\pgfqpoint{2.159049in}{0.661033in}}%
\pgfpathcurveto{\pgfqpoint{2.164873in}{0.661033in}}{\pgfqpoint{2.170459in}{0.663347in}}{\pgfqpoint{2.174578in}{0.667465in}}%
\pgfpathcurveto{\pgfqpoint{2.178696in}{0.671583in}}{\pgfqpoint{2.181010in}{0.677169in}}{\pgfqpoint{2.181010in}{0.682993in}}%
\pgfpathcurveto{\pgfqpoint{2.181010in}{0.688817in}}{\pgfqpoint{2.178696in}{0.694403in}}{\pgfqpoint{2.174578in}{0.698521in}}%
\pgfpathcurveto{\pgfqpoint{2.170459in}{0.702639in}}{\pgfqpoint{2.164873in}{0.704953in}}{\pgfqpoint{2.159049in}{0.704953in}}%
\pgfpathcurveto{\pgfqpoint{2.153225in}{0.704953in}}{\pgfqpoint{2.147639in}{0.702639in}}{\pgfqpoint{2.143521in}{0.698521in}}%
\pgfpathcurveto{\pgfqpoint{2.139403in}{0.694403in}}{\pgfqpoint{2.137089in}{0.688817in}}{\pgfqpoint{2.137089in}{0.682993in}}%
\pgfpathcurveto{\pgfqpoint{2.137089in}{0.677169in}}{\pgfqpoint{2.139403in}{0.671583in}}{\pgfqpoint{2.143521in}{0.667465in}}%
\pgfpathcurveto{\pgfqpoint{2.147639in}{0.663347in}}{\pgfqpoint{2.153225in}{0.661033in}}{\pgfqpoint{2.159049in}{0.661033in}}%
\pgfpathlineto{\pgfqpoint{2.159049in}{0.661033in}}%
\pgfpathclose%
\pgfusepath{stroke,fill}%
\end{pgfscope}%
\begin{pgfscope}%
\pgfpathrectangle{\pgfqpoint{0.640323in}{0.527436in}}{\pgfqpoint{9.687500in}{3.850000in}}%
\pgfusepath{clip}%
\pgfsetbuttcap%
\pgfsetroundjoin%
\definecolor{currentfill}{rgb}{0.000000,0.000000,1.000000}%
\pgfsetfillcolor{currentfill}%
\pgfsetfillopacity{0.500000}%
\pgfsetlinewidth{1.003750pt}%
\definecolor{currentstroke}{rgb}{0.000000,0.000000,1.000000}%
\pgfsetstrokecolor{currentstroke}%
\pgfsetstrokeopacity{0.500000}%
\pgfsetdash{{3.700000pt}{1.600000pt}}{0.000000pt}%
\pgfpathmoveto{\pgfqpoint{2.338780in}{0.663507in}}%
\pgfpathcurveto{\pgfqpoint{2.344604in}{0.663507in}}{\pgfqpoint{2.350190in}{0.665821in}}{\pgfqpoint{2.354309in}{0.669939in}}%
\pgfpathcurveto{\pgfqpoint{2.358427in}{0.674057in}}{\pgfqpoint{2.360741in}{0.679643in}}{\pgfqpoint{2.360741in}{0.685467in}}%
\pgfpathcurveto{\pgfqpoint{2.360741in}{0.691291in}}{\pgfqpoint{2.358427in}{0.696877in}}{\pgfqpoint{2.354309in}{0.700996in}}%
\pgfpathcurveto{\pgfqpoint{2.350190in}{0.705114in}}{\pgfqpoint{2.344604in}{0.707428in}}{\pgfqpoint{2.338780in}{0.707428in}}%
\pgfpathcurveto{\pgfqpoint{2.332956in}{0.707428in}}{\pgfqpoint{2.327370in}{0.705114in}}{\pgfqpoint{2.323252in}{0.700996in}}%
\pgfpathcurveto{\pgfqpoint{2.319134in}{0.696877in}}{\pgfqpoint{2.316820in}{0.691291in}}{\pgfqpoint{2.316820in}{0.685467in}}%
\pgfpathcurveto{\pgfqpoint{2.316820in}{0.679643in}}{\pgfqpoint{2.319134in}{0.674057in}}{\pgfqpoint{2.323252in}{0.669939in}}%
\pgfpathcurveto{\pgfqpoint{2.327370in}{0.665821in}}{\pgfqpoint{2.332956in}{0.663507in}}{\pgfqpoint{2.338780in}{0.663507in}}%
\pgfpathlineto{\pgfqpoint{2.338780in}{0.663507in}}%
\pgfpathclose%
\pgfusepath{stroke,fill}%
\end{pgfscope}%
\begin{pgfscope}%
\pgfpathrectangle{\pgfqpoint{0.640323in}{0.527436in}}{\pgfqpoint{9.687500in}{3.850000in}}%
\pgfusepath{clip}%
\pgfsetbuttcap%
\pgfsetroundjoin%
\definecolor{currentfill}{rgb}{0.000000,0.000000,1.000000}%
\pgfsetfillcolor{currentfill}%
\pgfsetfillopacity{0.500000}%
\pgfsetlinewidth{1.003750pt}%
\definecolor{currentstroke}{rgb}{0.000000,0.000000,1.000000}%
\pgfsetstrokecolor{currentstroke}%
\pgfsetstrokeopacity{0.500000}%
\pgfsetdash{{3.700000pt}{1.600000pt}}{0.000000pt}%
\pgfpathmoveto{\pgfqpoint{2.518511in}{0.666054in}}%
\pgfpathcurveto{\pgfqpoint{2.524335in}{0.666054in}}{\pgfqpoint{2.529921in}{0.668368in}}{\pgfqpoint{2.534040in}{0.672486in}}%
\pgfpathcurveto{\pgfqpoint{2.538158in}{0.676604in}}{\pgfqpoint{2.540472in}{0.682190in}}{\pgfqpoint{2.540472in}{0.688014in}}%
\pgfpathcurveto{\pgfqpoint{2.540472in}{0.693838in}}{\pgfqpoint{2.538158in}{0.699424in}}{\pgfqpoint{2.534040in}{0.703542in}}%
\pgfpathcurveto{\pgfqpoint{2.529921in}{0.707661in}}{\pgfqpoint{2.524335in}{0.709975in}}{\pgfqpoint{2.518511in}{0.709975in}}%
\pgfpathcurveto{\pgfqpoint{2.512687in}{0.709975in}}{\pgfqpoint{2.507101in}{0.707661in}}{\pgfqpoint{2.502983in}{0.703542in}}%
\pgfpathcurveto{\pgfqpoint{2.498865in}{0.699424in}}{\pgfqpoint{2.496551in}{0.693838in}}{\pgfqpoint{2.496551in}{0.688014in}}%
\pgfpathcurveto{\pgfqpoint{2.496551in}{0.682190in}}{\pgfqpoint{2.498865in}{0.676604in}}{\pgfqpoint{2.502983in}{0.672486in}}%
\pgfpathcurveto{\pgfqpoint{2.507101in}{0.668368in}}{\pgfqpoint{2.512687in}{0.666054in}}{\pgfqpoint{2.518511in}{0.666054in}}%
\pgfpathlineto{\pgfqpoint{2.518511in}{0.666054in}}%
\pgfpathclose%
\pgfusepath{stroke,fill}%
\end{pgfscope}%
\begin{pgfscope}%
\pgfpathrectangle{\pgfqpoint{0.640323in}{0.527436in}}{\pgfqpoint{9.687500in}{3.850000in}}%
\pgfusepath{clip}%
\pgfsetbuttcap%
\pgfsetroundjoin%
\definecolor{currentfill}{rgb}{0.000000,0.000000,1.000000}%
\pgfsetfillcolor{currentfill}%
\pgfsetfillopacity{0.500000}%
\pgfsetlinewidth{1.003750pt}%
\definecolor{currentstroke}{rgb}{0.000000,0.000000,1.000000}%
\pgfsetstrokecolor{currentstroke}%
\pgfsetstrokeopacity{0.500000}%
\pgfsetdash{{3.700000pt}{1.600000pt}}{0.000000pt}%
\pgfpathmoveto{\pgfqpoint{2.698242in}{0.674398in}}%
\pgfpathcurveto{\pgfqpoint{2.704066in}{0.674398in}}{\pgfqpoint{2.709652in}{0.676712in}}{\pgfqpoint{2.713771in}{0.680830in}}%
\pgfpathcurveto{\pgfqpoint{2.717889in}{0.684948in}}{\pgfqpoint{2.720203in}{0.690535in}}{\pgfqpoint{2.720203in}{0.696358in}}%
\pgfpathcurveto{\pgfqpoint{2.720203in}{0.702182in}}{\pgfqpoint{2.717889in}{0.707769in}}{\pgfqpoint{2.713771in}{0.711887in}}%
\pgfpathcurveto{\pgfqpoint{2.709652in}{0.716005in}}{\pgfqpoint{2.704066in}{0.718319in}}{\pgfqpoint{2.698242in}{0.718319in}}%
\pgfpathcurveto{\pgfqpoint{2.692418in}{0.718319in}}{\pgfqpoint{2.686832in}{0.716005in}}{\pgfqpoint{2.682714in}{0.711887in}}%
\pgfpathcurveto{\pgfqpoint{2.678596in}{0.707769in}}{\pgfqpoint{2.676282in}{0.702182in}}{\pgfqpoint{2.676282in}{0.696358in}}%
\pgfpathcurveto{\pgfqpoint{2.676282in}{0.690535in}}{\pgfqpoint{2.678596in}{0.684948in}}{\pgfqpoint{2.682714in}{0.680830in}}%
\pgfpathcurveto{\pgfqpoint{2.686832in}{0.676712in}}{\pgfqpoint{2.692418in}{0.674398in}}{\pgfqpoint{2.698242in}{0.674398in}}%
\pgfpathlineto{\pgfqpoint{2.698242in}{0.674398in}}%
\pgfpathclose%
\pgfusepath{stroke,fill}%
\end{pgfscope}%
\begin{pgfscope}%
\pgfpathrectangle{\pgfqpoint{0.640323in}{0.527436in}}{\pgfqpoint{9.687500in}{3.850000in}}%
\pgfusepath{clip}%
\pgfsetbuttcap%
\pgfsetroundjoin%
\definecolor{currentfill}{rgb}{0.000000,0.000000,1.000000}%
\pgfsetfillcolor{currentfill}%
\pgfsetfillopacity{0.500000}%
\pgfsetlinewidth{1.003750pt}%
\definecolor{currentstroke}{rgb}{0.000000,0.000000,1.000000}%
\pgfsetstrokecolor{currentstroke}%
\pgfsetstrokeopacity{0.500000}%
\pgfsetdash{{3.700000pt}{1.600000pt}}{0.000000pt}%
\pgfpathmoveto{\pgfqpoint{2.877973in}{1.612133in}}%
\pgfpathcurveto{\pgfqpoint{2.883797in}{1.612133in}}{\pgfqpoint{2.889383in}{1.614447in}}{\pgfqpoint{2.893501in}{1.618565in}}%
\pgfpathcurveto{\pgfqpoint{2.897620in}{1.622683in}}{\pgfqpoint{2.899934in}{1.628270in}}{\pgfqpoint{2.899934in}{1.634093in}}%
\pgfpathcurveto{\pgfqpoint{2.899934in}{1.639917in}}{\pgfqpoint{2.897620in}{1.645504in}}{\pgfqpoint{2.893501in}{1.649622in}}%
\pgfpathcurveto{\pgfqpoint{2.889383in}{1.653740in}}{\pgfqpoint{2.883797in}{1.656054in}}{\pgfqpoint{2.877973in}{1.656054in}}%
\pgfpathcurveto{\pgfqpoint{2.872149in}{1.656054in}}{\pgfqpoint{2.866563in}{1.653740in}}{\pgfqpoint{2.862445in}{1.649622in}}%
\pgfpathcurveto{\pgfqpoint{2.858327in}{1.645504in}}{\pgfqpoint{2.856013in}{1.639917in}}{\pgfqpoint{2.856013in}{1.634093in}}%
\pgfpathcurveto{\pgfqpoint{2.856013in}{1.628270in}}{\pgfqpoint{2.858327in}{1.622683in}}{\pgfqpoint{2.862445in}{1.618565in}}%
\pgfpathcurveto{\pgfqpoint{2.866563in}{1.614447in}}{\pgfqpoint{2.872149in}{1.612133in}}{\pgfqpoint{2.877973in}{1.612133in}}%
\pgfpathlineto{\pgfqpoint{2.877973in}{1.612133in}}%
\pgfpathclose%
\pgfusepath{stroke,fill}%
\end{pgfscope}%
\begin{pgfscope}%
\pgfpathrectangle{\pgfqpoint{0.640323in}{0.527436in}}{\pgfqpoint{9.687500in}{3.850000in}}%
\pgfusepath{clip}%
\pgfsetbuttcap%
\pgfsetroundjoin%
\definecolor{currentfill}{rgb}{0.000000,0.000000,1.000000}%
\pgfsetfillcolor{currentfill}%
\pgfsetfillopacity{0.500000}%
\pgfsetlinewidth{1.003750pt}%
\definecolor{currentstroke}{rgb}{0.000000,0.000000,1.000000}%
\pgfsetstrokecolor{currentstroke}%
\pgfsetstrokeopacity{0.500000}%
\pgfsetdash{{3.700000pt}{1.600000pt}}{0.000000pt}%
\pgfpathmoveto{\pgfqpoint{3.057704in}{1.862700in}}%
\pgfpathcurveto{\pgfqpoint{3.063528in}{1.862700in}}{\pgfqpoint{3.069114in}{1.865014in}}{\pgfqpoint{3.073232in}{1.869132in}}%
\pgfpathcurveto{\pgfqpoint{3.077351in}{1.873250in}}{\pgfqpoint{3.079664in}{1.878836in}}{\pgfqpoint{3.079664in}{1.884660in}}%
\pgfpathcurveto{\pgfqpoint{3.079664in}{1.890484in}}{\pgfqpoint{3.077351in}{1.896070in}}{\pgfqpoint{3.073232in}{1.900188in}}%
\pgfpathcurveto{\pgfqpoint{3.069114in}{1.904307in}}{\pgfqpoint{3.063528in}{1.906620in}}{\pgfqpoint{3.057704in}{1.906620in}}%
\pgfpathcurveto{\pgfqpoint{3.051880in}{1.906620in}}{\pgfqpoint{3.046294in}{1.904307in}}{\pgfqpoint{3.042176in}{1.900188in}}%
\pgfpathcurveto{\pgfqpoint{3.038058in}{1.896070in}}{\pgfqpoint{3.035744in}{1.890484in}}{\pgfqpoint{3.035744in}{1.884660in}}%
\pgfpathcurveto{\pgfqpoint{3.035744in}{1.878836in}}{\pgfqpoint{3.038058in}{1.873250in}}{\pgfqpoint{3.042176in}{1.869132in}}%
\pgfpathcurveto{\pgfqpoint{3.046294in}{1.865014in}}{\pgfqpoint{3.051880in}{1.862700in}}{\pgfqpoint{3.057704in}{1.862700in}}%
\pgfpathlineto{\pgfqpoint{3.057704in}{1.862700in}}%
\pgfpathclose%
\pgfusepath{stroke,fill}%
\end{pgfscope}%
\begin{pgfscope}%
\pgfpathrectangle{\pgfqpoint{0.640323in}{0.527436in}}{\pgfqpoint{9.687500in}{3.850000in}}%
\pgfusepath{clip}%
\pgfsetbuttcap%
\pgfsetroundjoin%
\definecolor{currentfill}{rgb}{0.000000,0.000000,1.000000}%
\pgfsetfillcolor{currentfill}%
\pgfsetfillopacity{0.500000}%
\pgfsetlinewidth{1.003750pt}%
\definecolor{currentstroke}{rgb}{0.000000,0.000000,1.000000}%
\pgfsetstrokecolor{currentstroke}%
\pgfsetstrokeopacity{0.500000}%
\pgfsetdash{{3.700000pt}{1.600000pt}}{0.000000pt}%
\pgfpathmoveto{\pgfqpoint{3.237435in}{2.097724in}}%
\pgfpathcurveto{\pgfqpoint{3.243259in}{2.097724in}}{\pgfqpoint{3.248845in}{2.100038in}}{\pgfqpoint{3.252963in}{2.104156in}}%
\pgfpathcurveto{\pgfqpoint{3.257082in}{2.108274in}}{\pgfqpoint{3.259395in}{2.113860in}}{\pgfqpoint{3.259395in}{2.119684in}}%
\pgfpathcurveto{\pgfqpoint{3.259395in}{2.125508in}}{\pgfqpoint{3.257082in}{2.131094in}}{\pgfqpoint{3.252963in}{2.135212in}}%
\pgfpathcurveto{\pgfqpoint{3.248845in}{2.139330in}}{\pgfqpoint{3.243259in}{2.141644in}}{\pgfqpoint{3.237435in}{2.141644in}}%
\pgfpathcurveto{\pgfqpoint{3.231611in}{2.141644in}}{\pgfqpoint{3.226025in}{2.139330in}}{\pgfqpoint{3.221907in}{2.135212in}}%
\pgfpathcurveto{\pgfqpoint{3.217789in}{2.131094in}}{\pgfqpoint{3.215475in}{2.125508in}}{\pgfqpoint{3.215475in}{2.119684in}}%
\pgfpathcurveto{\pgfqpoint{3.215475in}{2.113860in}}{\pgfqpoint{3.217789in}{2.108274in}}{\pgfqpoint{3.221907in}{2.104156in}}%
\pgfpathcurveto{\pgfqpoint{3.226025in}{2.100038in}}{\pgfqpoint{3.231611in}{2.097724in}}{\pgfqpoint{3.237435in}{2.097724in}}%
\pgfpathlineto{\pgfqpoint{3.237435in}{2.097724in}}%
\pgfpathclose%
\pgfusepath{stroke,fill}%
\end{pgfscope}%
\begin{pgfscope}%
\pgfpathrectangle{\pgfqpoint{0.640323in}{0.527436in}}{\pgfqpoint{9.687500in}{3.850000in}}%
\pgfusepath{clip}%
\pgfsetbuttcap%
\pgfsetroundjoin%
\definecolor{currentfill}{rgb}{0.000000,0.000000,1.000000}%
\pgfsetfillcolor{currentfill}%
\pgfsetfillopacity{0.500000}%
\pgfsetlinewidth{1.003750pt}%
\definecolor{currentstroke}{rgb}{0.000000,0.000000,1.000000}%
\pgfsetstrokecolor{currentstroke}%
\pgfsetstrokeopacity{0.500000}%
\pgfsetdash{{3.700000pt}{1.600000pt}}{0.000000pt}%
\pgfpathmoveto{\pgfqpoint{3.417166in}{2.309629in}}%
\pgfpathcurveto{\pgfqpoint{3.422990in}{2.309629in}}{\pgfqpoint{3.428576in}{2.311943in}}{\pgfqpoint{3.432694in}{2.316061in}}%
\pgfpathcurveto{\pgfqpoint{3.436813in}{2.320179in}}{\pgfqpoint{3.439126in}{2.325765in}}{\pgfqpoint{3.439126in}{2.331589in}}%
\pgfpathcurveto{\pgfqpoint{3.439126in}{2.337413in}}{\pgfqpoint{3.436813in}{2.342999in}}{\pgfqpoint{3.432694in}{2.347118in}}%
\pgfpathcurveto{\pgfqpoint{3.428576in}{2.351236in}}{\pgfqpoint{3.422990in}{2.353550in}}{\pgfqpoint{3.417166in}{2.353550in}}%
\pgfpathcurveto{\pgfqpoint{3.411342in}{2.353550in}}{\pgfqpoint{3.405756in}{2.351236in}}{\pgfqpoint{3.401638in}{2.347118in}}%
\pgfpathcurveto{\pgfqpoint{3.397520in}{2.342999in}}{\pgfqpoint{3.395206in}{2.337413in}}{\pgfqpoint{3.395206in}{2.331589in}}%
\pgfpathcurveto{\pgfqpoint{3.395206in}{2.325765in}}{\pgfqpoint{3.397520in}{2.320179in}}{\pgfqpoint{3.401638in}{2.316061in}}%
\pgfpathcurveto{\pgfqpoint{3.405756in}{2.311943in}}{\pgfqpoint{3.411342in}{2.309629in}}{\pgfqpoint{3.417166in}{2.309629in}}%
\pgfpathlineto{\pgfqpoint{3.417166in}{2.309629in}}%
\pgfpathclose%
\pgfusepath{stroke,fill}%
\end{pgfscope}%
\begin{pgfscope}%
\pgfpathrectangle{\pgfqpoint{0.640323in}{0.527436in}}{\pgfqpoint{9.687500in}{3.850000in}}%
\pgfusepath{clip}%
\pgfsetbuttcap%
\pgfsetroundjoin%
\definecolor{currentfill}{rgb}{0.000000,0.000000,1.000000}%
\pgfsetfillcolor{currentfill}%
\pgfsetfillopacity{0.500000}%
\pgfsetlinewidth{1.003750pt}%
\definecolor{currentstroke}{rgb}{0.000000,0.000000,1.000000}%
\pgfsetstrokecolor{currentstroke}%
\pgfsetstrokeopacity{0.500000}%
\pgfsetdash{{3.700000pt}{1.600000pt}}{0.000000pt}%
\pgfpathmoveto{\pgfqpoint{3.596897in}{2.484444in}}%
\pgfpathcurveto{\pgfqpoint{3.602721in}{2.484444in}}{\pgfqpoint{3.608307in}{2.486758in}}{\pgfqpoint{3.612425in}{2.490876in}}%
\pgfpathcurveto{\pgfqpoint{3.616544in}{2.494994in}}{\pgfqpoint{3.618857in}{2.500580in}}{\pgfqpoint{3.618857in}{2.506404in}}%
\pgfpathcurveto{\pgfqpoint{3.618857in}{2.512228in}}{\pgfqpoint{3.616544in}{2.517814in}}{\pgfqpoint{3.612425in}{2.521932in}}%
\pgfpathcurveto{\pgfqpoint{3.608307in}{2.526051in}}{\pgfqpoint{3.602721in}{2.528364in}}{\pgfqpoint{3.596897in}{2.528364in}}%
\pgfpathcurveto{\pgfqpoint{3.591073in}{2.528364in}}{\pgfqpoint{3.585487in}{2.526051in}}{\pgfqpoint{3.581369in}{2.521932in}}%
\pgfpathcurveto{\pgfqpoint{3.577251in}{2.517814in}}{\pgfqpoint{3.574937in}{2.512228in}}{\pgfqpoint{3.574937in}{2.506404in}}%
\pgfpathcurveto{\pgfqpoint{3.574937in}{2.500580in}}{\pgfqpoint{3.577251in}{2.494994in}}{\pgfqpoint{3.581369in}{2.490876in}}%
\pgfpathcurveto{\pgfqpoint{3.585487in}{2.486758in}}{\pgfqpoint{3.591073in}{2.484444in}}{\pgfqpoint{3.596897in}{2.484444in}}%
\pgfpathlineto{\pgfqpoint{3.596897in}{2.484444in}}%
\pgfpathclose%
\pgfusepath{stroke,fill}%
\end{pgfscope}%
\begin{pgfscope}%
\pgfpathrectangle{\pgfqpoint{0.640323in}{0.527436in}}{\pgfqpoint{9.687500in}{3.850000in}}%
\pgfusepath{clip}%
\pgfsetbuttcap%
\pgfsetroundjoin%
\definecolor{currentfill}{rgb}{0.000000,0.000000,1.000000}%
\pgfsetfillcolor{currentfill}%
\pgfsetfillopacity{0.500000}%
\pgfsetlinewidth{1.003750pt}%
\definecolor{currentstroke}{rgb}{0.000000,0.000000,1.000000}%
\pgfsetstrokecolor{currentstroke}%
\pgfsetstrokeopacity{0.500000}%
\pgfsetdash{{3.700000pt}{1.600000pt}}{0.000000pt}%
\pgfpathmoveto{\pgfqpoint{3.776628in}{2.584358in}}%
\pgfpathcurveto{\pgfqpoint{3.782452in}{2.584358in}}{\pgfqpoint{3.788038in}{2.586671in}}{\pgfqpoint{3.792156in}{2.590790in}}%
\pgfpathcurveto{\pgfqpoint{3.796275in}{2.594908in}}{\pgfqpoint{3.798588in}{2.600494in}}{\pgfqpoint{3.798588in}{2.606318in}}%
\pgfpathcurveto{\pgfqpoint{3.798588in}{2.612142in}}{\pgfqpoint{3.796275in}{2.617728in}}{\pgfqpoint{3.792156in}{2.621846in}}%
\pgfpathcurveto{\pgfqpoint{3.788038in}{2.625964in}}{\pgfqpoint{3.782452in}{2.628278in}}{\pgfqpoint{3.776628in}{2.628278in}}%
\pgfpathcurveto{\pgfqpoint{3.770804in}{2.628278in}}{\pgfqpoint{3.765218in}{2.625964in}}{\pgfqpoint{3.761100in}{2.621846in}}%
\pgfpathcurveto{\pgfqpoint{3.756982in}{2.617728in}}{\pgfqpoint{3.754668in}{2.612142in}}{\pgfqpoint{3.754668in}{2.606318in}}%
\pgfpathcurveto{\pgfqpoint{3.754668in}{2.600494in}}{\pgfqpoint{3.756982in}{2.594908in}}{\pgfqpoint{3.761100in}{2.590790in}}%
\pgfpathcurveto{\pgfqpoint{3.765218in}{2.586671in}}{\pgfqpoint{3.770804in}{2.584358in}}{\pgfqpoint{3.776628in}{2.584358in}}%
\pgfpathlineto{\pgfqpoint{3.776628in}{2.584358in}}%
\pgfpathclose%
\pgfusepath{stroke,fill}%
\end{pgfscope}%
\begin{pgfscope}%
\pgfpathrectangle{\pgfqpoint{0.640323in}{0.527436in}}{\pgfqpoint{9.687500in}{3.850000in}}%
\pgfusepath{clip}%
\pgfsetbuttcap%
\pgfsetroundjoin%
\definecolor{currentfill}{rgb}{0.000000,0.000000,1.000000}%
\pgfsetfillcolor{currentfill}%
\pgfsetfillopacity{0.500000}%
\pgfsetlinewidth{1.003750pt}%
\definecolor{currentstroke}{rgb}{0.000000,0.000000,1.000000}%
\pgfsetstrokecolor{currentstroke}%
\pgfsetstrokeopacity{0.500000}%
\pgfsetdash{{3.700000pt}{1.600000pt}}{0.000000pt}%
\pgfpathmoveto{\pgfqpoint{3.956359in}{2.767717in}}%
\pgfpathcurveto{\pgfqpoint{3.962183in}{2.767717in}}{\pgfqpoint{3.967769in}{2.770031in}}{\pgfqpoint{3.971887in}{2.774149in}}%
\pgfpathcurveto{\pgfqpoint{3.976006in}{2.778267in}}{\pgfqpoint{3.978319in}{2.783853in}}{\pgfqpoint{3.978319in}{2.789677in}}%
\pgfpathcurveto{\pgfqpoint{3.978319in}{2.795501in}}{\pgfqpoint{3.976006in}{2.801087in}}{\pgfqpoint{3.971887in}{2.805205in}}%
\pgfpathcurveto{\pgfqpoint{3.967769in}{2.809324in}}{\pgfqpoint{3.962183in}{2.811637in}}{\pgfqpoint{3.956359in}{2.811637in}}%
\pgfpathcurveto{\pgfqpoint{3.950535in}{2.811637in}}{\pgfqpoint{3.944949in}{2.809324in}}{\pgfqpoint{3.940831in}{2.805205in}}%
\pgfpathcurveto{\pgfqpoint{3.936713in}{2.801087in}}{\pgfqpoint{3.934399in}{2.795501in}}{\pgfqpoint{3.934399in}{2.789677in}}%
\pgfpathcurveto{\pgfqpoint{3.934399in}{2.783853in}}{\pgfqpoint{3.936713in}{2.778267in}}{\pgfqpoint{3.940831in}{2.774149in}}%
\pgfpathcurveto{\pgfqpoint{3.944949in}{2.770031in}}{\pgfqpoint{3.950535in}{2.767717in}}{\pgfqpoint{3.956359in}{2.767717in}}%
\pgfpathlineto{\pgfqpoint{3.956359in}{2.767717in}}%
\pgfpathclose%
\pgfusepath{stroke,fill}%
\end{pgfscope}%
\begin{pgfscope}%
\pgfpathrectangle{\pgfqpoint{0.640323in}{0.527436in}}{\pgfqpoint{9.687500in}{3.850000in}}%
\pgfusepath{clip}%
\pgfsetbuttcap%
\pgfsetroundjoin%
\definecolor{currentfill}{rgb}{0.000000,0.000000,1.000000}%
\pgfsetfillcolor{currentfill}%
\pgfsetfillopacity{0.500000}%
\pgfsetlinewidth{1.003750pt}%
\definecolor{currentstroke}{rgb}{0.000000,0.000000,1.000000}%
\pgfsetstrokecolor{currentstroke}%
\pgfsetstrokeopacity{0.500000}%
\pgfsetdash{{3.700000pt}{1.600000pt}}{0.000000pt}%
\pgfpathmoveto{\pgfqpoint{4.136090in}{2.807571in}}%
\pgfpathcurveto{\pgfqpoint{4.141914in}{2.807571in}}{\pgfqpoint{4.147500in}{2.809885in}}{\pgfqpoint{4.151618in}{2.814003in}}%
\pgfpathcurveto{\pgfqpoint{4.155737in}{2.818121in}}{\pgfqpoint{4.158050in}{2.823707in}}{\pgfqpoint{4.158050in}{2.829531in}}%
\pgfpathcurveto{\pgfqpoint{4.158050in}{2.835355in}}{\pgfqpoint{4.155737in}{2.840941in}}{\pgfqpoint{4.151618in}{2.845059in}}%
\pgfpathcurveto{\pgfqpoint{4.147500in}{2.849177in}}{\pgfqpoint{4.141914in}{2.851491in}}{\pgfqpoint{4.136090in}{2.851491in}}%
\pgfpathcurveto{\pgfqpoint{4.130266in}{2.851491in}}{\pgfqpoint{4.124680in}{2.849177in}}{\pgfqpoint{4.120562in}{2.845059in}}%
\pgfpathcurveto{\pgfqpoint{4.116444in}{2.840941in}}{\pgfqpoint{4.114130in}{2.835355in}}{\pgfqpoint{4.114130in}{2.829531in}}%
\pgfpathcurveto{\pgfqpoint{4.114130in}{2.823707in}}{\pgfqpoint{4.116444in}{2.818121in}}{\pgfqpoint{4.120562in}{2.814003in}}%
\pgfpathcurveto{\pgfqpoint{4.124680in}{2.809885in}}{\pgfqpoint{4.130266in}{2.807571in}}{\pgfqpoint{4.136090in}{2.807571in}}%
\pgfpathlineto{\pgfqpoint{4.136090in}{2.807571in}}%
\pgfpathclose%
\pgfusepath{stroke,fill}%
\end{pgfscope}%
\begin{pgfscope}%
\pgfpathrectangle{\pgfqpoint{0.640323in}{0.527436in}}{\pgfqpoint{9.687500in}{3.850000in}}%
\pgfusepath{clip}%
\pgfsetbuttcap%
\pgfsetroundjoin%
\definecolor{currentfill}{rgb}{0.000000,0.000000,1.000000}%
\pgfsetfillcolor{currentfill}%
\pgfsetfillopacity{0.500000}%
\pgfsetlinewidth{1.003750pt}%
\definecolor{currentstroke}{rgb}{0.000000,0.000000,1.000000}%
\pgfsetstrokecolor{currentstroke}%
\pgfsetstrokeopacity{0.500000}%
\pgfsetdash{{3.700000pt}{1.600000pt}}{0.000000pt}%
\pgfpathmoveto{\pgfqpoint{4.315821in}{2.951946in}}%
\pgfpathcurveto{\pgfqpoint{4.321645in}{2.951946in}}{\pgfqpoint{4.327231in}{2.954260in}}{\pgfqpoint{4.331349in}{2.958378in}}%
\pgfpathcurveto{\pgfqpoint{4.335467in}{2.962496in}}{\pgfqpoint{4.337781in}{2.968082in}}{\pgfqpoint{4.337781in}{2.973906in}}%
\pgfpathcurveto{\pgfqpoint{4.337781in}{2.979730in}}{\pgfqpoint{4.335467in}{2.985316in}}{\pgfqpoint{4.331349in}{2.989434in}}%
\pgfpathcurveto{\pgfqpoint{4.327231in}{2.993552in}}{\pgfqpoint{4.321645in}{2.995866in}}{\pgfqpoint{4.315821in}{2.995866in}}%
\pgfpathcurveto{\pgfqpoint{4.309997in}{2.995866in}}{\pgfqpoint{4.304411in}{2.993552in}}{\pgfqpoint{4.300293in}{2.989434in}}%
\pgfpathcurveto{\pgfqpoint{4.296175in}{2.985316in}}{\pgfqpoint{4.293861in}{2.979730in}}{\pgfqpoint{4.293861in}{2.973906in}}%
\pgfpathcurveto{\pgfqpoint{4.293861in}{2.968082in}}{\pgfqpoint{4.296175in}{2.962496in}}{\pgfqpoint{4.300293in}{2.958378in}}%
\pgfpathcurveto{\pgfqpoint{4.304411in}{2.954260in}}{\pgfqpoint{4.309997in}{2.951946in}}{\pgfqpoint{4.315821in}{2.951946in}}%
\pgfpathlineto{\pgfqpoint{4.315821in}{2.951946in}}%
\pgfpathclose%
\pgfusepath{stroke,fill}%
\end{pgfscope}%
\begin{pgfscope}%
\pgfpathrectangle{\pgfqpoint{0.640323in}{0.527436in}}{\pgfqpoint{9.687500in}{3.850000in}}%
\pgfusepath{clip}%
\pgfsetbuttcap%
\pgfsetroundjoin%
\definecolor{currentfill}{rgb}{0.000000,0.000000,1.000000}%
\pgfsetfillcolor{currentfill}%
\pgfsetfillopacity{0.500000}%
\pgfsetlinewidth{1.003750pt}%
\definecolor{currentstroke}{rgb}{0.000000,0.000000,1.000000}%
\pgfsetstrokecolor{currentstroke}%
\pgfsetstrokeopacity{0.500000}%
\pgfsetdash{{3.700000pt}{1.600000pt}}{0.000000pt}%
\pgfpathmoveto{\pgfqpoint{4.495552in}{3.010205in}}%
\pgfpathcurveto{\pgfqpoint{4.501376in}{3.010205in}}{\pgfqpoint{4.506962in}{3.012519in}}{\pgfqpoint{4.511080in}{3.016637in}}%
\pgfpathcurveto{\pgfqpoint{4.515198in}{3.020755in}}{\pgfqpoint{4.517512in}{3.026341in}}{\pgfqpoint{4.517512in}{3.032165in}}%
\pgfpathcurveto{\pgfqpoint{4.517512in}{3.037989in}}{\pgfqpoint{4.515198in}{3.043575in}}{\pgfqpoint{4.511080in}{3.047693in}}%
\pgfpathcurveto{\pgfqpoint{4.506962in}{3.051811in}}{\pgfqpoint{4.501376in}{3.054125in}}{\pgfqpoint{4.495552in}{3.054125in}}%
\pgfpathcurveto{\pgfqpoint{4.489728in}{3.054125in}}{\pgfqpoint{4.484142in}{3.051811in}}{\pgfqpoint{4.480024in}{3.047693in}}%
\pgfpathcurveto{\pgfqpoint{4.475906in}{3.043575in}}{\pgfqpoint{4.473592in}{3.037989in}}{\pgfqpoint{4.473592in}{3.032165in}}%
\pgfpathcurveto{\pgfqpoint{4.473592in}{3.026341in}}{\pgfqpoint{4.475906in}{3.020755in}}{\pgfqpoint{4.480024in}{3.016637in}}%
\pgfpathcurveto{\pgfqpoint{4.484142in}{3.012519in}}{\pgfqpoint{4.489728in}{3.010205in}}{\pgfqpoint{4.495552in}{3.010205in}}%
\pgfpathlineto{\pgfqpoint{4.495552in}{3.010205in}}%
\pgfpathclose%
\pgfusepath{stroke,fill}%
\end{pgfscope}%
\begin{pgfscope}%
\pgfpathrectangle{\pgfqpoint{0.640323in}{0.527436in}}{\pgfqpoint{9.687500in}{3.850000in}}%
\pgfusepath{clip}%
\pgfsetbuttcap%
\pgfsetroundjoin%
\definecolor{currentfill}{rgb}{0.000000,0.000000,1.000000}%
\pgfsetfillcolor{currentfill}%
\pgfsetfillopacity{0.500000}%
\pgfsetlinewidth{1.003750pt}%
\definecolor{currentstroke}{rgb}{0.000000,0.000000,1.000000}%
\pgfsetstrokecolor{currentstroke}%
\pgfsetstrokeopacity{0.500000}%
\pgfsetdash{{3.700000pt}{1.600000pt}}{0.000000pt}%
\pgfpathmoveto{\pgfqpoint{4.675283in}{3.101580in}}%
\pgfpathcurveto{\pgfqpoint{4.681107in}{3.101580in}}{\pgfqpoint{4.686693in}{3.103894in}}{\pgfqpoint{4.690811in}{3.108012in}}%
\pgfpathcurveto{\pgfqpoint{4.694929in}{3.112130in}}{\pgfqpoint{4.697243in}{3.117717in}}{\pgfqpoint{4.697243in}{3.123541in}}%
\pgfpathcurveto{\pgfqpoint{4.697243in}{3.129364in}}{\pgfqpoint{4.694929in}{3.134951in}}{\pgfqpoint{4.690811in}{3.139069in}}%
\pgfpathcurveto{\pgfqpoint{4.686693in}{3.143187in}}{\pgfqpoint{4.681107in}{3.145501in}}{\pgfqpoint{4.675283in}{3.145501in}}%
\pgfpathcurveto{\pgfqpoint{4.669459in}{3.145501in}}{\pgfqpoint{4.663873in}{3.143187in}}{\pgfqpoint{4.659755in}{3.139069in}}%
\pgfpathcurveto{\pgfqpoint{4.655637in}{3.134951in}}{\pgfqpoint{4.653323in}{3.129364in}}{\pgfqpoint{4.653323in}{3.123541in}}%
\pgfpathcurveto{\pgfqpoint{4.653323in}{3.117717in}}{\pgfqpoint{4.655637in}{3.112130in}}{\pgfqpoint{4.659755in}{3.108012in}}%
\pgfpathcurveto{\pgfqpoint{4.663873in}{3.103894in}}{\pgfqpoint{4.669459in}{3.101580in}}{\pgfqpoint{4.675283in}{3.101580in}}%
\pgfpathlineto{\pgfqpoint{4.675283in}{3.101580in}}%
\pgfpathclose%
\pgfusepath{stroke,fill}%
\end{pgfscope}%
\begin{pgfscope}%
\pgfpathrectangle{\pgfqpoint{0.640323in}{0.527436in}}{\pgfqpoint{9.687500in}{3.850000in}}%
\pgfusepath{clip}%
\pgfsetbuttcap%
\pgfsetroundjoin%
\definecolor{currentfill}{rgb}{0.000000,0.000000,1.000000}%
\pgfsetfillcolor{currentfill}%
\pgfsetfillopacity{0.500000}%
\pgfsetlinewidth{1.003750pt}%
\definecolor{currentstroke}{rgb}{0.000000,0.000000,1.000000}%
\pgfsetstrokecolor{currentstroke}%
\pgfsetstrokeopacity{0.500000}%
\pgfsetdash{{3.700000pt}{1.600000pt}}{0.000000pt}%
\pgfpathmoveto{\pgfqpoint{4.855014in}{3.177698in}}%
\pgfpathcurveto{\pgfqpoint{4.860838in}{3.177698in}}{\pgfqpoint{4.866424in}{3.180012in}}{\pgfqpoint{4.870542in}{3.184130in}}%
\pgfpathcurveto{\pgfqpoint{4.874660in}{3.188249in}}{\pgfqpoint{4.876974in}{3.193835in}}{\pgfqpoint{4.876974in}{3.199659in}}%
\pgfpathcurveto{\pgfqpoint{4.876974in}{3.205483in}}{\pgfqpoint{4.874660in}{3.211069in}}{\pgfqpoint{4.870542in}{3.215187in}}%
\pgfpathcurveto{\pgfqpoint{4.866424in}{3.219305in}}{\pgfqpoint{4.860838in}{3.221619in}}{\pgfqpoint{4.855014in}{3.221619in}}%
\pgfpathcurveto{\pgfqpoint{4.849190in}{3.221619in}}{\pgfqpoint{4.843604in}{3.219305in}}{\pgfqpoint{4.839486in}{3.215187in}}%
\pgfpathcurveto{\pgfqpoint{4.835368in}{3.211069in}}{\pgfqpoint{4.833054in}{3.205483in}}{\pgfqpoint{4.833054in}{3.199659in}}%
\pgfpathcurveto{\pgfqpoint{4.833054in}{3.193835in}}{\pgfqpoint{4.835368in}{3.188249in}}{\pgfqpoint{4.839486in}{3.184130in}}%
\pgfpathcurveto{\pgfqpoint{4.843604in}{3.180012in}}{\pgfqpoint{4.849190in}{3.177698in}}{\pgfqpoint{4.855014in}{3.177698in}}%
\pgfpathlineto{\pgfqpoint{4.855014in}{3.177698in}}%
\pgfpathclose%
\pgfusepath{stroke,fill}%
\end{pgfscope}%
\begin{pgfscope}%
\pgfpathrectangle{\pgfqpoint{0.640323in}{0.527436in}}{\pgfqpoint{9.687500in}{3.850000in}}%
\pgfusepath{clip}%
\pgfsetbuttcap%
\pgfsetroundjoin%
\definecolor{currentfill}{rgb}{0.000000,0.000000,1.000000}%
\pgfsetfillcolor{currentfill}%
\pgfsetfillopacity{0.500000}%
\pgfsetlinewidth{1.003750pt}%
\definecolor{currentstroke}{rgb}{0.000000,0.000000,1.000000}%
\pgfsetstrokecolor{currentstroke}%
\pgfsetstrokeopacity{0.500000}%
\pgfsetdash{{3.700000pt}{1.600000pt}}{0.000000pt}%
\pgfpathmoveto{\pgfqpoint{5.034745in}{3.210908in}}%
\pgfpathcurveto{\pgfqpoint{5.040569in}{3.210908in}}{\pgfqpoint{5.046155in}{3.213222in}}{\pgfqpoint{5.050273in}{3.217340in}}%
\pgfpathcurveto{\pgfqpoint{5.054391in}{3.221458in}}{\pgfqpoint{5.056705in}{3.227044in}}{\pgfqpoint{5.056705in}{3.232868in}}%
\pgfpathcurveto{\pgfqpoint{5.056705in}{3.238692in}}{\pgfqpoint{5.054391in}{3.244278in}}{\pgfqpoint{5.050273in}{3.248396in}}%
\pgfpathcurveto{\pgfqpoint{5.046155in}{3.252514in}}{\pgfqpoint{5.040569in}{3.254828in}}{\pgfqpoint{5.034745in}{3.254828in}}%
\pgfpathcurveto{\pgfqpoint{5.028921in}{3.254828in}}{\pgfqpoint{5.023335in}{3.252514in}}{\pgfqpoint{5.019217in}{3.248396in}}%
\pgfpathcurveto{\pgfqpoint{5.015099in}{3.244278in}}{\pgfqpoint{5.012785in}{3.238692in}}{\pgfqpoint{5.012785in}{3.232868in}}%
\pgfpathcurveto{\pgfqpoint{5.012785in}{3.227044in}}{\pgfqpoint{5.015099in}{3.221458in}}{\pgfqpoint{5.019217in}{3.217340in}}%
\pgfpathcurveto{\pgfqpoint{5.023335in}{3.213222in}}{\pgfqpoint{5.028921in}{3.210908in}}{\pgfqpoint{5.034745in}{3.210908in}}%
\pgfpathlineto{\pgfqpoint{5.034745in}{3.210908in}}%
\pgfpathclose%
\pgfusepath{stroke,fill}%
\end{pgfscope}%
\begin{pgfscope}%
\pgfpathrectangle{\pgfqpoint{0.640323in}{0.527436in}}{\pgfqpoint{9.687500in}{3.850000in}}%
\pgfusepath{clip}%
\pgfsetbuttcap%
\pgfsetroundjoin%
\definecolor{currentfill}{rgb}{0.000000,0.000000,1.000000}%
\pgfsetfillcolor{currentfill}%
\pgfsetfillopacity{0.500000}%
\pgfsetlinewidth{1.003750pt}%
\definecolor{currentstroke}{rgb}{0.000000,0.000000,1.000000}%
\pgfsetstrokecolor{currentstroke}%
\pgfsetstrokeopacity{0.500000}%
\pgfsetdash{{3.700000pt}{1.600000pt}}{0.000000pt}%
\pgfpathmoveto{\pgfqpoint{5.214476in}{3.277805in}}%
\pgfpathcurveto{\pgfqpoint{5.220300in}{3.277805in}}{\pgfqpoint{5.225886in}{3.280119in}}{\pgfqpoint{5.230004in}{3.284237in}}%
\pgfpathcurveto{\pgfqpoint{5.234122in}{3.288355in}}{\pgfqpoint{5.236436in}{3.293941in}}{\pgfqpoint{5.236436in}{3.299765in}}%
\pgfpathcurveto{\pgfqpoint{5.236436in}{3.305589in}}{\pgfqpoint{5.234122in}{3.311175in}}{\pgfqpoint{5.230004in}{3.315293in}}%
\pgfpathcurveto{\pgfqpoint{5.225886in}{3.319411in}}{\pgfqpoint{5.220300in}{3.321725in}}{\pgfqpoint{5.214476in}{3.321725in}}%
\pgfpathcurveto{\pgfqpoint{5.208652in}{3.321725in}}{\pgfqpoint{5.203066in}{3.319411in}}{\pgfqpoint{5.198948in}{3.315293in}}%
\pgfpathcurveto{\pgfqpoint{5.194830in}{3.311175in}}{\pgfqpoint{5.192516in}{3.305589in}}{\pgfqpoint{5.192516in}{3.299765in}}%
\pgfpathcurveto{\pgfqpoint{5.192516in}{3.293941in}}{\pgfqpoint{5.194830in}{3.288355in}}{\pgfqpoint{5.198948in}{3.284237in}}%
\pgfpathcurveto{\pgfqpoint{5.203066in}{3.280119in}}{\pgfqpoint{5.208652in}{3.277805in}}{\pgfqpoint{5.214476in}{3.277805in}}%
\pgfpathlineto{\pgfqpoint{5.214476in}{3.277805in}}%
\pgfpathclose%
\pgfusepath{stroke,fill}%
\end{pgfscope}%
\begin{pgfscope}%
\pgfpathrectangle{\pgfqpoint{0.640323in}{0.527436in}}{\pgfqpoint{9.687500in}{3.850000in}}%
\pgfusepath{clip}%
\pgfsetbuttcap%
\pgfsetroundjoin%
\definecolor{currentfill}{rgb}{0.000000,0.000000,1.000000}%
\pgfsetfillcolor{currentfill}%
\pgfsetfillopacity{0.500000}%
\pgfsetlinewidth{1.003750pt}%
\definecolor{currentstroke}{rgb}{0.000000,0.000000,1.000000}%
\pgfsetstrokecolor{currentstroke}%
\pgfsetstrokeopacity{0.500000}%
\pgfsetdash{{3.700000pt}{1.600000pt}}{0.000000pt}%
\pgfpathmoveto{\pgfqpoint{5.394207in}{3.337790in}}%
\pgfpathcurveto{\pgfqpoint{5.400031in}{3.337790in}}{\pgfqpoint{5.405617in}{3.340104in}}{\pgfqpoint{5.409735in}{3.344222in}}%
\pgfpathcurveto{\pgfqpoint{5.413853in}{3.348340in}}{\pgfqpoint{5.416167in}{3.353927in}}{\pgfqpoint{5.416167in}{3.359750in}}%
\pgfpathcurveto{\pgfqpoint{5.416167in}{3.365574in}}{\pgfqpoint{5.413853in}{3.371161in}}{\pgfqpoint{5.409735in}{3.375279in}}%
\pgfpathcurveto{\pgfqpoint{5.405617in}{3.379397in}}{\pgfqpoint{5.400031in}{3.381711in}}{\pgfqpoint{5.394207in}{3.381711in}}%
\pgfpathcurveto{\pgfqpoint{5.388383in}{3.381711in}}{\pgfqpoint{5.382797in}{3.379397in}}{\pgfqpoint{5.378679in}{3.375279in}}%
\pgfpathcurveto{\pgfqpoint{5.374561in}{3.371161in}}{\pgfqpoint{5.372247in}{3.365574in}}{\pgfqpoint{5.372247in}{3.359750in}}%
\pgfpathcurveto{\pgfqpoint{5.372247in}{3.353927in}}{\pgfqpoint{5.374561in}{3.348340in}}{\pgfqpoint{5.378679in}{3.344222in}}%
\pgfpathcurveto{\pgfqpoint{5.382797in}{3.340104in}}{\pgfqpoint{5.388383in}{3.337790in}}{\pgfqpoint{5.394207in}{3.337790in}}%
\pgfpathlineto{\pgfqpoint{5.394207in}{3.337790in}}%
\pgfpathclose%
\pgfusepath{stroke,fill}%
\end{pgfscope}%
\begin{pgfscope}%
\pgfpathrectangle{\pgfqpoint{0.640323in}{0.527436in}}{\pgfqpoint{9.687500in}{3.850000in}}%
\pgfusepath{clip}%
\pgfsetbuttcap%
\pgfsetroundjoin%
\definecolor{currentfill}{rgb}{0.000000,0.000000,1.000000}%
\pgfsetfillcolor{currentfill}%
\pgfsetfillopacity{0.500000}%
\pgfsetlinewidth{1.003750pt}%
\definecolor{currentstroke}{rgb}{0.000000,0.000000,1.000000}%
\pgfsetstrokecolor{currentstroke}%
\pgfsetstrokeopacity{0.500000}%
\pgfsetdash{{3.700000pt}{1.600000pt}}{0.000000pt}%
\pgfpathmoveto{\pgfqpoint{5.573938in}{3.384096in}}%
\pgfpathcurveto{\pgfqpoint{5.579762in}{3.384096in}}{\pgfqpoint{5.585348in}{3.386410in}}{\pgfqpoint{5.589466in}{3.390528in}}%
\pgfpathcurveto{\pgfqpoint{5.593584in}{3.394646in}}{\pgfqpoint{5.595898in}{3.400232in}}{\pgfqpoint{5.595898in}{3.406056in}}%
\pgfpathcurveto{\pgfqpoint{5.595898in}{3.411880in}}{\pgfqpoint{5.593584in}{3.417466in}}{\pgfqpoint{5.589466in}{3.421584in}}%
\pgfpathcurveto{\pgfqpoint{5.585348in}{3.425702in}}{\pgfqpoint{5.579762in}{3.428016in}}{\pgfqpoint{5.573938in}{3.428016in}}%
\pgfpathcurveto{\pgfqpoint{5.568114in}{3.428016in}}{\pgfqpoint{5.562528in}{3.425702in}}{\pgfqpoint{5.558410in}{3.421584in}}%
\pgfpathcurveto{\pgfqpoint{5.554292in}{3.417466in}}{\pgfqpoint{5.551978in}{3.411880in}}{\pgfqpoint{5.551978in}{3.406056in}}%
\pgfpathcurveto{\pgfqpoint{5.551978in}{3.400232in}}{\pgfqpoint{5.554292in}{3.394646in}}{\pgfqpoint{5.558410in}{3.390528in}}%
\pgfpathcurveto{\pgfqpoint{5.562528in}{3.386410in}}{\pgfqpoint{5.568114in}{3.384096in}}{\pgfqpoint{5.573938in}{3.384096in}}%
\pgfpathlineto{\pgfqpoint{5.573938in}{3.384096in}}%
\pgfpathclose%
\pgfusepath{stroke,fill}%
\end{pgfscope}%
\begin{pgfscope}%
\pgfpathrectangle{\pgfqpoint{0.640323in}{0.527436in}}{\pgfqpoint{9.687500in}{3.850000in}}%
\pgfusepath{clip}%
\pgfsetbuttcap%
\pgfsetroundjoin%
\definecolor{currentfill}{rgb}{0.000000,0.000000,1.000000}%
\pgfsetfillcolor{currentfill}%
\pgfsetfillopacity{0.500000}%
\pgfsetlinewidth{1.003750pt}%
\definecolor{currentstroke}{rgb}{0.000000,0.000000,1.000000}%
\pgfsetstrokecolor{currentstroke}%
\pgfsetstrokeopacity{0.500000}%
\pgfsetdash{{3.700000pt}{1.600000pt}}{0.000000pt}%
\pgfpathmoveto{\pgfqpoint{5.753669in}{3.404302in}}%
\pgfpathcurveto{\pgfqpoint{5.759493in}{3.404302in}}{\pgfqpoint{5.765079in}{3.406616in}}{\pgfqpoint{5.769197in}{3.410734in}}%
\pgfpathcurveto{\pgfqpoint{5.773315in}{3.414852in}}{\pgfqpoint{5.775629in}{3.420438in}}{\pgfqpoint{5.775629in}{3.426262in}}%
\pgfpathcurveto{\pgfqpoint{5.775629in}{3.432086in}}{\pgfqpoint{5.773315in}{3.437672in}}{\pgfqpoint{5.769197in}{3.441791in}}%
\pgfpathcurveto{\pgfqpoint{5.765079in}{3.445909in}}{\pgfqpoint{5.759493in}{3.448223in}}{\pgfqpoint{5.753669in}{3.448223in}}%
\pgfpathcurveto{\pgfqpoint{5.747845in}{3.448223in}}{\pgfqpoint{5.742259in}{3.445909in}}{\pgfqpoint{5.738141in}{3.441791in}}%
\pgfpathcurveto{\pgfqpoint{5.734023in}{3.437672in}}{\pgfqpoint{5.731709in}{3.432086in}}{\pgfqpoint{5.731709in}{3.426262in}}%
\pgfpathcurveto{\pgfqpoint{5.731709in}{3.420438in}}{\pgfqpoint{5.734023in}{3.414852in}}{\pgfqpoint{5.738141in}{3.410734in}}%
\pgfpathcurveto{\pgfqpoint{5.742259in}{3.406616in}}{\pgfqpoint{5.747845in}{3.404302in}}{\pgfqpoint{5.753669in}{3.404302in}}%
\pgfpathlineto{\pgfqpoint{5.753669in}{3.404302in}}%
\pgfpathclose%
\pgfusepath{stroke,fill}%
\end{pgfscope}%
\begin{pgfscope}%
\pgfpathrectangle{\pgfqpoint{0.640323in}{0.527436in}}{\pgfqpoint{9.687500in}{3.850000in}}%
\pgfusepath{clip}%
\pgfsetbuttcap%
\pgfsetroundjoin%
\definecolor{currentfill}{rgb}{0.000000,0.000000,1.000000}%
\pgfsetfillcolor{currentfill}%
\pgfsetfillopacity{0.500000}%
\pgfsetlinewidth{1.003750pt}%
\definecolor{currentstroke}{rgb}{0.000000,0.000000,1.000000}%
\pgfsetstrokecolor{currentstroke}%
\pgfsetstrokeopacity{0.500000}%
\pgfsetdash{{3.700000pt}{1.600000pt}}{0.000000pt}%
\pgfpathmoveto{\pgfqpoint{5.933400in}{3.482699in}}%
\pgfpathcurveto{\pgfqpoint{5.939224in}{3.482699in}}{\pgfqpoint{5.944810in}{3.485013in}}{\pgfqpoint{5.948928in}{3.489131in}}%
\pgfpathcurveto{\pgfqpoint{5.953046in}{3.493249in}}{\pgfqpoint{5.955360in}{3.498836in}}{\pgfqpoint{5.955360in}{3.504659in}}%
\pgfpathcurveto{\pgfqpoint{5.955360in}{3.510483in}}{\pgfqpoint{5.953046in}{3.516070in}}{\pgfqpoint{5.948928in}{3.520188in}}%
\pgfpathcurveto{\pgfqpoint{5.944810in}{3.524306in}}{\pgfqpoint{5.939224in}{3.526620in}}{\pgfqpoint{5.933400in}{3.526620in}}%
\pgfpathcurveto{\pgfqpoint{5.927576in}{3.526620in}}{\pgfqpoint{5.921990in}{3.524306in}}{\pgfqpoint{5.917872in}{3.520188in}}%
\pgfpathcurveto{\pgfqpoint{5.913754in}{3.516070in}}{\pgfqpoint{5.911440in}{3.510483in}}{\pgfqpoint{5.911440in}{3.504659in}}%
\pgfpathcurveto{\pgfqpoint{5.911440in}{3.498836in}}{\pgfqpoint{5.913754in}{3.493249in}}{\pgfqpoint{5.917872in}{3.489131in}}%
\pgfpathcurveto{\pgfqpoint{5.921990in}{3.485013in}}{\pgfqpoint{5.927576in}{3.482699in}}{\pgfqpoint{5.933400in}{3.482699in}}%
\pgfpathlineto{\pgfqpoint{5.933400in}{3.482699in}}%
\pgfpathclose%
\pgfusepath{stroke,fill}%
\end{pgfscope}%
\begin{pgfscope}%
\pgfpathrectangle{\pgfqpoint{0.640323in}{0.527436in}}{\pgfqpoint{9.687500in}{3.850000in}}%
\pgfusepath{clip}%
\pgfsetbuttcap%
\pgfsetroundjoin%
\definecolor{currentfill}{rgb}{0.000000,0.000000,1.000000}%
\pgfsetfillcolor{currentfill}%
\pgfsetfillopacity{0.500000}%
\pgfsetlinewidth{1.003750pt}%
\definecolor{currentstroke}{rgb}{0.000000,0.000000,1.000000}%
\pgfsetstrokecolor{currentstroke}%
\pgfsetstrokeopacity{0.500000}%
\pgfsetdash{{3.700000pt}{1.600000pt}}{0.000000pt}%
\pgfpathmoveto{\pgfqpoint{6.113131in}{3.532985in}}%
\pgfpathcurveto{\pgfqpoint{6.118955in}{3.532985in}}{\pgfqpoint{6.124541in}{3.535299in}}{\pgfqpoint{6.128659in}{3.539417in}}%
\pgfpathcurveto{\pgfqpoint{6.132777in}{3.543535in}}{\pgfqpoint{6.135091in}{3.549122in}}{\pgfqpoint{6.135091in}{3.554945in}}%
\pgfpathcurveto{\pgfqpoint{6.135091in}{3.560769in}}{\pgfqpoint{6.132777in}{3.566356in}}{\pgfqpoint{6.128659in}{3.570474in}}%
\pgfpathcurveto{\pgfqpoint{6.124541in}{3.574592in}}{\pgfqpoint{6.118955in}{3.576906in}}{\pgfqpoint{6.113131in}{3.576906in}}%
\pgfpathcurveto{\pgfqpoint{6.107307in}{3.576906in}}{\pgfqpoint{6.101721in}{3.574592in}}{\pgfqpoint{6.097603in}{3.570474in}}%
\pgfpathcurveto{\pgfqpoint{6.093485in}{3.566356in}}{\pgfqpoint{6.091171in}{3.560769in}}{\pgfqpoint{6.091171in}{3.554945in}}%
\pgfpathcurveto{\pgfqpoint{6.091171in}{3.549122in}}{\pgfqpoint{6.093485in}{3.543535in}}{\pgfqpoint{6.097603in}{3.539417in}}%
\pgfpathcurveto{\pgfqpoint{6.101721in}{3.535299in}}{\pgfqpoint{6.107307in}{3.532985in}}{\pgfqpoint{6.113131in}{3.532985in}}%
\pgfpathlineto{\pgfqpoint{6.113131in}{3.532985in}}%
\pgfpathclose%
\pgfusepath{stroke,fill}%
\end{pgfscope}%
\begin{pgfscope}%
\pgfpathrectangle{\pgfqpoint{0.640323in}{0.527436in}}{\pgfqpoint{9.687500in}{3.850000in}}%
\pgfusepath{clip}%
\pgfsetbuttcap%
\pgfsetroundjoin%
\definecolor{currentfill}{rgb}{0.000000,0.000000,1.000000}%
\pgfsetfillcolor{currentfill}%
\pgfsetfillopacity{0.500000}%
\pgfsetlinewidth{1.003750pt}%
\definecolor{currentstroke}{rgb}{0.000000,0.000000,1.000000}%
\pgfsetstrokecolor{currentstroke}%
\pgfsetstrokeopacity{0.500000}%
\pgfsetdash{{3.700000pt}{1.600000pt}}{0.000000pt}%
\pgfpathmoveto{\pgfqpoint{6.292862in}{3.552657in}}%
\pgfpathcurveto{\pgfqpoint{6.298686in}{3.552657in}}{\pgfqpoint{6.304272in}{3.554971in}}{\pgfqpoint{6.308390in}{3.559089in}}%
\pgfpathcurveto{\pgfqpoint{6.312508in}{3.563208in}}{\pgfqpoint{6.314822in}{3.568794in}}{\pgfqpoint{6.314822in}{3.574618in}}%
\pgfpathcurveto{\pgfqpoint{6.314822in}{3.580442in}}{\pgfqpoint{6.312508in}{3.586028in}}{\pgfqpoint{6.308390in}{3.590146in}}%
\pgfpathcurveto{\pgfqpoint{6.304272in}{3.594264in}}{\pgfqpoint{6.298686in}{3.596578in}}{\pgfqpoint{6.292862in}{3.596578in}}%
\pgfpathcurveto{\pgfqpoint{6.287038in}{3.596578in}}{\pgfqpoint{6.281452in}{3.594264in}}{\pgfqpoint{6.277334in}{3.590146in}}%
\pgfpathcurveto{\pgfqpoint{6.273216in}{3.586028in}}{\pgfqpoint{6.270902in}{3.580442in}}{\pgfqpoint{6.270902in}{3.574618in}}%
\pgfpathcurveto{\pgfqpoint{6.270902in}{3.568794in}}{\pgfqpoint{6.273216in}{3.563208in}}{\pgfqpoint{6.277334in}{3.559089in}}%
\pgfpathcurveto{\pgfqpoint{6.281452in}{3.554971in}}{\pgfqpoint{6.287038in}{3.552657in}}{\pgfqpoint{6.292862in}{3.552657in}}%
\pgfpathlineto{\pgfqpoint{6.292862in}{3.552657in}}%
\pgfpathclose%
\pgfusepath{stroke,fill}%
\end{pgfscope}%
\begin{pgfscope}%
\pgfpathrectangle{\pgfqpoint{0.640323in}{0.527436in}}{\pgfqpoint{9.687500in}{3.850000in}}%
\pgfusepath{clip}%
\pgfsetbuttcap%
\pgfsetroundjoin%
\definecolor{currentfill}{rgb}{0.000000,0.000000,1.000000}%
\pgfsetfillcolor{currentfill}%
\pgfsetfillopacity{0.500000}%
\pgfsetlinewidth{1.003750pt}%
\definecolor{currentstroke}{rgb}{0.000000,0.000000,1.000000}%
\pgfsetstrokecolor{currentstroke}%
\pgfsetstrokeopacity{0.500000}%
\pgfsetdash{{3.700000pt}{1.600000pt}}{0.000000pt}%
\pgfpathmoveto{\pgfqpoint{6.472593in}{3.580868in}}%
\pgfpathcurveto{\pgfqpoint{6.478417in}{3.580868in}}{\pgfqpoint{6.484003in}{3.583182in}}{\pgfqpoint{6.488121in}{3.587300in}}%
\pgfpathcurveto{\pgfqpoint{6.492239in}{3.591418in}}{\pgfqpoint{6.494553in}{3.597004in}}{\pgfqpoint{6.494553in}{3.602828in}}%
\pgfpathcurveto{\pgfqpoint{6.494553in}{3.608652in}}{\pgfqpoint{6.492239in}{3.614238in}}{\pgfqpoint{6.488121in}{3.618357in}}%
\pgfpathcurveto{\pgfqpoint{6.484003in}{3.622475in}}{\pgfqpoint{6.478417in}{3.624789in}}{\pgfqpoint{6.472593in}{3.624789in}}%
\pgfpathcurveto{\pgfqpoint{6.466769in}{3.624789in}}{\pgfqpoint{6.461183in}{3.622475in}}{\pgfqpoint{6.457065in}{3.618357in}}%
\pgfpathcurveto{\pgfqpoint{6.452947in}{3.614238in}}{\pgfqpoint{6.450633in}{3.608652in}}{\pgfqpoint{6.450633in}{3.602828in}}%
\pgfpathcurveto{\pgfqpoint{6.450633in}{3.597004in}}{\pgfqpoint{6.452947in}{3.591418in}}{\pgfqpoint{6.457065in}{3.587300in}}%
\pgfpathcurveto{\pgfqpoint{6.461183in}{3.583182in}}{\pgfqpoint{6.466769in}{3.580868in}}{\pgfqpoint{6.472593in}{3.580868in}}%
\pgfpathlineto{\pgfqpoint{6.472593in}{3.580868in}}%
\pgfpathclose%
\pgfusepath{stroke,fill}%
\end{pgfscope}%
\begin{pgfscope}%
\pgfpathrectangle{\pgfqpoint{0.640323in}{0.527436in}}{\pgfqpoint{9.687500in}{3.850000in}}%
\pgfusepath{clip}%
\pgfsetbuttcap%
\pgfsetroundjoin%
\definecolor{currentfill}{rgb}{0.000000,0.000000,1.000000}%
\pgfsetfillcolor{currentfill}%
\pgfsetfillopacity{0.500000}%
\pgfsetlinewidth{1.003750pt}%
\definecolor{currentstroke}{rgb}{0.000000,0.000000,1.000000}%
\pgfsetstrokecolor{currentstroke}%
\pgfsetstrokeopacity{0.500000}%
\pgfsetdash{{3.700000pt}{1.600000pt}}{0.000000pt}%
\pgfpathmoveto{\pgfqpoint{6.652324in}{3.631173in}}%
\pgfpathcurveto{\pgfqpoint{6.658148in}{3.631173in}}{\pgfqpoint{6.663734in}{3.633486in}}{\pgfqpoint{6.667852in}{3.637605in}}%
\pgfpathcurveto{\pgfqpoint{6.671970in}{3.641723in}}{\pgfqpoint{6.674284in}{3.647309in}}{\pgfqpoint{6.674284in}{3.653133in}}%
\pgfpathcurveto{\pgfqpoint{6.674284in}{3.658957in}}{\pgfqpoint{6.671970in}{3.664543in}}{\pgfqpoint{6.667852in}{3.668661in}}%
\pgfpathcurveto{\pgfqpoint{6.663734in}{3.672779in}}{\pgfqpoint{6.658148in}{3.675093in}}{\pgfqpoint{6.652324in}{3.675093in}}%
\pgfpathcurveto{\pgfqpoint{6.646500in}{3.675093in}}{\pgfqpoint{6.640914in}{3.672779in}}{\pgfqpoint{6.636796in}{3.668661in}}%
\pgfpathcurveto{\pgfqpoint{6.632678in}{3.664543in}}{\pgfqpoint{6.630364in}{3.658957in}}{\pgfqpoint{6.630364in}{3.653133in}}%
\pgfpathcurveto{\pgfqpoint{6.630364in}{3.647309in}}{\pgfqpoint{6.632678in}{3.641723in}}{\pgfqpoint{6.636796in}{3.637605in}}%
\pgfpathcurveto{\pgfqpoint{6.640914in}{3.633486in}}{\pgfqpoint{6.646500in}{3.631173in}}{\pgfqpoint{6.652324in}{3.631173in}}%
\pgfpathlineto{\pgfqpoint{6.652324in}{3.631173in}}%
\pgfpathclose%
\pgfusepath{stroke,fill}%
\end{pgfscope}%
\begin{pgfscope}%
\pgfpathrectangle{\pgfqpoint{0.640323in}{0.527436in}}{\pgfqpoint{9.687500in}{3.850000in}}%
\pgfusepath{clip}%
\pgfsetbuttcap%
\pgfsetroundjoin%
\definecolor{currentfill}{rgb}{0.000000,0.000000,1.000000}%
\pgfsetfillcolor{currentfill}%
\pgfsetfillopacity{0.500000}%
\pgfsetlinewidth{1.003750pt}%
\definecolor{currentstroke}{rgb}{0.000000,0.000000,1.000000}%
\pgfsetstrokecolor{currentstroke}%
\pgfsetstrokeopacity{0.500000}%
\pgfsetdash{{3.700000pt}{1.600000pt}}{0.000000pt}%
\pgfpathmoveto{\pgfqpoint{6.832055in}{3.665158in}}%
\pgfpathcurveto{\pgfqpoint{6.837879in}{3.665158in}}{\pgfqpoint{6.843465in}{3.667472in}}{\pgfqpoint{6.847583in}{3.671590in}}%
\pgfpathcurveto{\pgfqpoint{6.851701in}{3.675708in}}{\pgfqpoint{6.854015in}{3.681294in}}{\pgfqpoint{6.854015in}{3.687118in}}%
\pgfpathcurveto{\pgfqpoint{6.854015in}{3.692942in}}{\pgfqpoint{6.851701in}{3.698529in}}{\pgfqpoint{6.847583in}{3.702647in}}%
\pgfpathcurveto{\pgfqpoint{6.843465in}{3.706765in}}{\pgfqpoint{6.837879in}{3.709079in}}{\pgfqpoint{6.832055in}{3.709079in}}%
\pgfpathcurveto{\pgfqpoint{6.826231in}{3.709079in}}{\pgfqpoint{6.820645in}{3.706765in}}{\pgfqpoint{6.816527in}{3.702647in}}%
\pgfpathcurveto{\pgfqpoint{6.812408in}{3.698529in}}{\pgfqpoint{6.810095in}{3.692942in}}{\pgfqpoint{6.810095in}{3.687118in}}%
\pgfpathcurveto{\pgfqpoint{6.810095in}{3.681294in}}{\pgfqpoint{6.812408in}{3.675708in}}{\pgfqpoint{6.816527in}{3.671590in}}%
\pgfpathcurveto{\pgfqpoint{6.820645in}{3.667472in}}{\pgfqpoint{6.826231in}{3.665158in}}{\pgfqpoint{6.832055in}{3.665158in}}%
\pgfpathlineto{\pgfqpoint{6.832055in}{3.665158in}}%
\pgfpathclose%
\pgfusepath{stroke,fill}%
\end{pgfscope}%
\begin{pgfscope}%
\pgfpathrectangle{\pgfqpoint{0.640323in}{0.527436in}}{\pgfqpoint{9.687500in}{3.850000in}}%
\pgfusepath{clip}%
\pgfsetbuttcap%
\pgfsetroundjoin%
\definecolor{currentfill}{rgb}{0.000000,0.000000,1.000000}%
\pgfsetfillcolor{currentfill}%
\pgfsetfillopacity{0.500000}%
\pgfsetlinewidth{1.003750pt}%
\definecolor{currentstroke}{rgb}{0.000000,0.000000,1.000000}%
\pgfsetstrokecolor{currentstroke}%
\pgfsetstrokeopacity{0.500000}%
\pgfsetdash{{3.700000pt}{1.600000pt}}{0.000000pt}%
\pgfpathmoveto{\pgfqpoint{7.011786in}{3.672411in}}%
\pgfpathcurveto{\pgfqpoint{7.017610in}{3.672411in}}{\pgfqpoint{7.023196in}{3.674725in}}{\pgfqpoint{7.027314in}{3.678843in}}%
\pgfpathcurveto{\pgfqpoint{7.031432in}{3.682961in}}{\pgfqpoint{7.033746in}{3.688547in}}{\pgfqpoint{7.033746in}{3.694371in}}%
\pgfpathcurveto{\pgfqpoint{7.033746in}{3.700195in}}{\pgfqpoint{7.031432in}{3.705781in}}{\pgfqpoint{7.027314in}{3.709900in}}%
\pgfpathcurveto{\pgfqpoint{7.023196in}{3.714018in}}{\pgfqpoint{7.017610in}{3.716332in}}{\pgfqpoint{7.011786in}{3.716332in}}%
\pgfpathcurveto{\pgfqpoint{7.005962in}{3.716332in}}{\pgfqpoint{7.000376in}{3.714018in}}{\pgfqpoint{6.996258in}{3.709900in}}%
\pgfpathcurveto{\pgfqpoint{6.992139in}{3.705781in}}{\pgfqpoint{6.989826in}{3.700195in}}{\pgfqpoint{6.989826in}{3.694371in}}%
\pgfpathcurveto{\pgfqpoint{6.989826in}{3.688547in}}{\pgfqpoint{6.992139in}{3.682961in}}{\pgfqpoint{6.996258in}{3.678843in}}%
\pgfpathcurveto{\pgfqpoint{7.000376in}{3.674725in}}{\pgfqpoint{7.005962in}{3.672411in}}{\pgfqpoint{7.011786in}{3.672411in}}%
\pgfpathlineto{\pgfqpoint{7.011786in}{3.672411in}}%
\pgfpathclose%
\pgfusepath{stroke,fill}%
\end{pgfscope}%
\begin{pgfscope}%
\pgfpathrectangle{\pgfqpoint{0.640323in}{0.527436in}}{\pgfqpoint{9.687500in}{3.850000in}}%
\pgfusepath{clip}%
\pgfsetbuttcap%
\pgfsetroundjoin%
\definecolor{currentfill}{rgb}{0.000000,0.000000,1.000000}%
\pgfsetfillcolor{currentfill}%
\pgfsetfillopacity{0.500000}%
\pgfsetlinewidth{1.003750pt}%
\definecolor{currentstroke}{rgb}{0.000000,0.000000,1.000000}%
\pgfsetstrokecolor{currentstroke}%
\pgfsetstrokeopacity{0.500000}%
\pgfsetdash{{3.700000pt}{1.600000pt}}{0.000000pt}%
\pgfpathmoveto{\pgfqpoint{7.191517in}{3.714860in}}%
\pgfpathcurveto{\pgfqpoint{7.197341in}{3.714860in}}{\pgfqpoint{7.202927in}{3.717174in}}{\pgfqpoint{7.207045in}{3.721292in}}%
\pgfpathcurveto{\pgfqpoint{7.211163in}{3.725411in}}{\pgfqpoint{7.213477in}{3.730997in}}{\pgfqpoint{7.213477in}{3.736821in}}%
\pgfpathcurveto{\pgfqpoint{7.213477in}{3.742645in}}{\pgfqpoint{7.211163in}{3.748231in}}{\pgfqpoint{7.207045in}{3.752349in}}%
\pgfpathcurveto{\pgfqpoint{7.202927in}{3.756467in}}{\pgfqpoint{7.197341in}{3.758781in}}{\pgfqpoint{7.191517in}{3.758781in}}%
\pgfpathcurveto{\pgfqpoint{7.185693in}{3.758781in}}{\pgfqpoint{7.180107in}{3.756467in}}{\pgfqpoint{7.175989in}{3.752349in}}%
\pgfpathcurveto{\pgfqpoint{7.171870in}{3.748231in}}{\pgfqpoint{7.169557in}{3.742645in}}{\pgfqpoint{7.169557in}{3.736821in}}%
\pgfpathcurveto{\pgfqpoint{7.169557in}{3.730997in}}{\pgfqpoint{7.171870in}{3.725411in}}{\pgfqpoint{7.175989in}{3.721292in}}%
\pgfpathcurveto{\pgfqpoint{7.180107in}{3.717174in}}{\pgfqpoint{7.185693in}{3.714860in}}{\pgfqpoint{7.191517in}{3.714860in}}%
\pgfpathlineto{\pgfqpoint{7.191517in}{3.714860in}}%
\pgfpathclose%
\pgfusepath{stroke,fill}%
\end{pgfscope}%
\begin{pgfscope}%
\pgfpathrectangle{\pgfqpoint{0.640323in}{0.527436in}}{\pgfqpoint{9.687500in}{3.850000in}}%
\pgfusepath{clip}%
\pgfsetbuttcap%
\pgfsetroundjoin%
\definecolor{currentfill}{rgb}{0.000000,0.000000,1.000000}%
\pgfsetfillcolor{currentfill}%
\pgfsetfillopacity{0.500000}%
\pgfsetlinewidth{1.003750pt}%
\definecolor{currentstroke}{rgb}{0.000000,0.000000,1.000000}%
\pgfsetstrokecolor{currentstroke}%
\pgfsetstrokeopacity{0.500000}%
\pgfsetdash{{3.700000pt}{1.600000pt}}{0.000000pt}%
\pgfpathmoveto{\pgfqpoint{7.371248in}{3.734508in}}%
\pgfpathcurveto{\pgfqpoint{7.377072in}{3.734508in}}{\pgfqpoint{7.382658in}{3.736822in}}{\pgfqpoint{7.386776in}{3.740940in}}%
\pgfpathcurveto{\pgfqpoint{7.390894in}{3.745058in}}{\pgfqpoint{7.393208in}{3.750644in}}{\pgfqpoint{7.393208in}{3.756468in}}%
\pgfpathcurveto{\pgfqpoint{7.393208in}{3.762292in}}{\pgfqpoint{7.390894in}{3.767878in}}{\pgfqpoint{7.386776in}{3.771996in}}%
\pgfpathcurveto{\pgfqpoint{7.382658in}{3.776114in}}{\pgfqpoint{7.377072in}{3.778428in}}{\pgfqpoint{7.371248in}{3.778428in}}%
\pgfpathcurveto{\pgfqpoint{7.365424in}{3.778428in}}{\pgfqpoint{7.359838in}{3.776114in}}{\pgfqpoint{7.355720in}{3.771996in}}%
\pgfpathcurveto{\pgfqpoint{7.351601in}{3.767878in}}{\pgfqpoint{7.349288in}{3.762292in}}{\pgfqpoint{7.349288in}{3.756468in}}%
\pgfpathcurveto{\pgfqpoint{7.349288in}{3.750644in}}{\pgfqpoint{7.351601in}{3.745058in}}{\pgfqpoint{7.355720in}{3.740940in}}%
\pgfpathcurveto{\pgfqpoint{7.359838in}{3.736822in}}{\pgfqpoint{7.365424in}{3.734508in}}{\pgfqpoint{7.371248in}{3.734508in}}%
\pgfpathlineto{\pgfqpoint{7.371248in}{3.734508in}}%
\pgfpathclose%
\pgfusepath{stroke,fill}%
\end{pgfscope}%
\begin{pgfscope}%
\pgfpathrectangle{\pgfqpoint{0.640323in}{0.527436in}}{\pgfqpoint{9.687500in}{3.850000in}}%
\pgfusepath{clip}%
\pgfsetbuttcap%
\pgfsetroundjoin%
\definecolor{currentfill}{rgb}{0.000000,0.000000,1.000000}%
\pgfsetfillcolor{currentfill}%
\pgfsetfillopacity{0.500000}%
\pgfsetlinewidth{1.003750pt}%
\definecolor{currentstroke}{rgb}{0.000000,0.000000,1.000000}%
\pgfsetstrokecolor{currentstroke}%
\pgfsetstrokeopacity{0.500000}%
\pgfsetdash{{3.700000pt}{1.600000pt}}{0.000000pt}%
\pgfpathmoveto{\pgfqpoint{7.550979in}{3.766407in}}%
\pgfpathcurveto{\pgfqpoint{7.556803in}{3.766407in}}{\pgfqpoint{7.562389in}{3.768721in}}{\pgfqpoint{7.566507in}{3.772839in}}%
\pgfpathcurveto{\pgfqpoint{7.570625in}{3.776957in}}{\pgfqpoint{7.572939in}{3.782543in}}{\pgfqpoint{7.572939in}{3.788367in}}%
\pgfpathcurveto{\pgfqpoint{7.572939in}{3.794191in}}{\pgfqpoint{7.570625in}{3.799777in}}{\pgfqpoint{7.566507in}{3.803895in}}%
\pgfpathcurveto{\pgfqpoint{7.562389in}{3.808014in}}{\pgfqpoint{7.556803in}{3.810327in}}{\pgfqpoint{7.550979in}{3.810327in}}%
\pgfpathcurveto{\pgfqpoint{7.545155in}{3.810327in}}{\pgfqpoint{7.539569in}{3.808014in}}{\pgfqpoint{7.535451in}{3.803895in}}%
\pgfpathcurveto{\pgfqpoint{7.531332in}{3.799777in}}{\pgfqpoint{7.529019in}{3.794191in}}{\pgfqpoint{7.529019in}{3.788367in}}%
\pgfpathcurveto{\pgfqpoint{7.529019in}{3.782543in}}{\pgfqpoint{7.531332in}{3.776957in}}{\pgfqpoint{7.535451in}{3.772839in}}%
\pgfpathcurveto{\pgfqpoint{7.539569in}{3.768721in}}{\pgfqpoint{7.545155in}{3.766407in}}{\pgfqpoint{7.550979in}{3.766407in}}%
\pgfpathlineto{\pgfqpoint{7.550979in}{3.766407in}}%
\pgfpathclose%
\pgfusepath{stroke,fill}%
\end{pgfscope}%
\begin{pgfscope}%
\pgfpathrectangle{\pgfqpoint{0.640323in}{0.527436in}}{\pgfqpoint{9.687500in}{3.850000in}}%
\pgfusepath{clip}%
\pgfsetbuttcap%
\pgfsetroundjoin%
\definecolor{currentfill}{rgb}{0.000000,0.000000,1.000000}%
\pgfsetfillcolor{currentfill}%
\pgfsetfillopacity{0.500000}%
\pgfsetlinewidth{1.003750pt}%
\definecolor{currentstroke}{rgb}{0.000000,0.000000,1.000000}%
\pgfsetstrokecolor{currentstroke}%
\pgfsetstrokeopacity{0.500000}%
\pgfsetdash{{3.700000pt}{1.600000pt}}{0.000000pt}%
\pgfpathmoveto{\pgfqpoint{7.730710in}{3.786104in}}%
\pgfpathcurveto{\pgfqpoint{7.736534in}{3.786104in}}{\pgfqpoint{7.742120in}{3.788418in}}{\pgfqpoint{7.746238in}{3.792536in}}%
\pgfpathcurveto{\pgfqpoint{7.750356in}{3.796654in}}{\pgfqpoint{7.752670in}{3.802240in}}{\pgfqpoint{7.752670in}{3.808064in}}%
\pgfpathcurveto{\pgfqpoint{7.752670in}{3.813888in}}{\pgfqpoint{7.750356in}{3.819474in}}{\pgfqpoint{7.746238in}{3.823593in}}%
\pgfpathcurveto{\pgfqpoint{7.742120in}{3.827711in}}{\pgfqpoint{7.736534in}{3.830025in}}{\pgfqpoint{7.730710in}{3.830025in}}%
\pgfpathcurveto{\pgfqpoint{7.724886in}{3.830025in}}{\pgfqpoint{7.719300in}{3.827711in}}{\pgfqpoint{7.715182in}{3.823593in}}%
\pgfpathcurveto{\pgfqpoint{7.711063in}{3.819474in}}{\pgfqpoint{7.708750in}{3.813888in}}{\pgfqpoint{7.708750in}{3.808064in}}%
\pgfpathcurveto{\pgfqpoint{7.708750in}{3.802240in}}{\pgfqpoint{7.711063in}{3.796654in}}{\pgfqpoint{7.715182in}{3.792536in}}%
\pgfpathcurveto{\pgfqpoint{7.719300in}{3.788418in}}{\pgfqpoint{7.724886in}{3.786104in}}{\pgfqpoint{7.730710in}{3.786104in}}%
\pgfpathlineto{\pgfqpoint{7.730710in}{3.786104in}}%
\pgfpathclose%
\pgfusepath{stroke,fill}%
\end{pgfscope}%
\begin{pgfscope}%
\pgfpathrectangle{\pgfqpoint{0.640323in}{0.527436in}}{\pgfqpoint{9.687500in}{3.850000in}}%
\pgfusepath{clip}%
\pgfsetbuttcap%
\pgfsetroundjoin%
\definecolor{currentfill}{rgb}{0.000000,0.000000,1.000000}%
\pgfsetfillcolor{currentfill}%
\pgfsetfillopacity{0.500000}%
\pgfsetlinewidth{1.003750pt}%
\definecolor{currentstroke}{rgb}{0.000000,0.000000,1.000000}%
\pgfsetstrokecolor{currentstroke}%
\pgfsetstrokeopacity{0.500000}%
\pgfsetdash{{3.700000pt}{1.600000pt}}{0.000000pt}%
\pgfpathmoveto{\pgfqpoint{7.910441in}{3.816420in}}%
\pgfpathcurveto{\pgfqpoint{7.916265in}{3.816420in}}{\pgfqpoint{7.921851in}{3.818734in}}{\pgfqpoint{7.925969in}{3.822852in}}%
\pgfpathcurveto{\pgfqpoint{7.930087in}{3.826970in}}{\pgfqpoint{7.932401in}{3.832556in}}{\pgfqpoint{7.932401in}{3.838380in}}%
\pgfpathcurveto{\pgfqpoint{7.932401in}{3.844204in}}{\pgfqpoint{7.930087in}{3.849790in}}{\pgfqpoint{7.925969in}{3.853908in}}%
\pgfpathcurveto{\pgfqpoint{7.921851in}{3.858026in}}{\pgfqpoint{7.916265in}{3.860340in}}{\pgfqpoint{7.910441in}{3.860340in}}%
\pgfpathcurveto{\pgfqpoint{7.904617in}{3.860340in}}{\pgfqpoint{7.899031in}{3.858026in}}{\pgfqpoint{7.894913in}{3.853908in}}%
\pgfpathcurveto{\pgfqpoint{7.890794in}{3.849790in}}{\pgfqpoint{7.888481in}{3.844204in}}{\pgfqpoint{7.888481in}{3.838380in}}%
\pgfpathcurveto{\pgfqpoint{7.888481in}{3.832556in}}{\pgfqpoint{7.890794in}{3.826970in}}{\pgfqpoint{7.894913in}{3.822852in}}%
\pgfpathcurveto{\pgfqpoint{7.899031in}{3.818734in}}{\pgfqpoint{7.904617in}{3.816420in}}{\pgfqpoint{7.910441in}{3.816420in}}%
\pgfpathlineto{\pgfqpoint{7.910441in}{3.816420in}}%
\pgfpathclose%
\pgfusepath{stroke,fill}%
\end{pgfscope}%
\begin{pgfscope}%
\pgfpathrectangle{\pgfqpoint{0.640323in}{0.527436in}}{\pgfqpoint{9.687500in}{3.850000in}}%
\pgfusepath{clip}%
\pgfsetbuttcap%
\pgfsetroundjoin%
\definecolor{currentfill}{rgb}{0.000000,0.000000,1.000000}%
\pgfsetfillcolor{currentfill}%
\pgfsetfillopacity{0.500000}%
\pgfsetlinewidth{1.003750pt}%
\definecolor{currentstroke}{rgb}{0.000000,0.000000,1.000000}%
\pgfsetstrokecolor{currentstroke}%
\pgfsetstrokeopacity{0.500000}%
\pgfsetdash{{3.700000pt}{1.600000pt}}{0.000000pt}%
\pgfpathmoveto{\pgfqpoint{8.090172in}{3.833912in}}%
\pgfpathcurveto{\pgfqpoint{8.095996in}{3.833912in}}{\pgfqpoint{8.101582in}{3.836226in}}{\pgfqpoint{8.105700in}{3.840344in}}%
\pgfpathcurveto{\pgfqpoint{8.109818in}{3.844462in}}{\pgfqpoint{8.112132in}{3.850049in}}{\pgfqpoint{8.112132in}{3.855873in}}%
\pgfpathcurveto{\pgfqpoint{8.112132in}{3.861697in}}{\pgfqpoint{8.109818in}{3.867283in}}{\pgfqpoint{8.105700in}{3.871401in}}%
\pgfpathcurveto{\pgfqpoint{8.101582in}{3.875519in}}{\pgfqpoint{8.095996in}{3.877833in}}{\pgfqpoint{8.090172in}{3.877833in}}%
\pgfpathcurveto{\pgfqpoint{8.084348in}{3.877833in}}{\pgfqpoint{8.078762in}{3.875519in}}{\pgfqpoint{8.074644in}{3.871401in}}%
\pgfpathcurveto{\pgfqpoint{8.070525in}{3.867283in}}{\pgfqpoint{8.068211in}{3.861697in}}{\pgfqpoint{8.068211in}{3.855873in}}%
\pgfpathcurveto{\pgfqpoint{8.068211in}{3.850049in}}{\pgfqpoint{8.070525in}{3.844462in}}{\pgfqpoint{8.074644in}{3.840344in}}%
\pgfpathcurveto{\pgfqpoint{8.078762in}{3.836226in}}{\pgfqpoint{8.084348in}{3.833912in}}{\pgfqpoint{8.090172in}{3.833912in}}%
\pgfpathlineto{\pgfqpoint{8.090172in}{3.833912in}}%
\pgfpathclose%
\pgfusepath{stroke,fill}%
\end{pgfscope}%
\begin{pgfscope}%
\pgfpathrectangle{\pgfqpoint{0.640323in}{0.527436in}}{\pgfqpoint{9.687500in}{3.850000in}}%
\pgfusepath{clip}%
\pgfsetbuttcap%
\pgfsetroundjoin%
\definecolor{currentfill}{rgb}{0.000000,0.000000,1.000000}%
\pgfsetfillcolor{currentfill}%
\pgfsetfillopacity{0.500000}%
\pgfsetlinewidth{1.003750pt}%
\definecolor{currentstroke}{rgb}{0.000000,0.000000,1.000000}%
\pgfsetstrokecolor{currentstroke}%
\pgfsetstrokeopacity{0.500000}%
\pgfsetdash{{3.700000pt}{1.600000pt}}{0.000000pt}%
\pgfpathmoveto{\pgfqpoint{8.269903in}{3.837135in}}%
\pgfpathcurveto{\pgfqpoint{8.275727in}{3.837135in}}{\pgfqpoint{8.281313in}{3.839449in}}{\pgfqpoint{8.285431in}{3.843567in}}%
\pgfpathcurveto{\pgfqpoint{8.289549in}{3.847685in}}{\pgfqpoint{8.291863in}{3.853272in}}{\pgfqpoint{8.291863in}{3.859095in}}%
\pgfpathcurveto{\pgfqpoint{8.291863in}{3.864919in}}{\pgfqpoint{8.289549in}{3.870506in}}{\pgfqpoint{8.285431in}{3.874624in}}%
\pgfpathcurveto{\pgfqpoint{8.281313in}{3.878742in}}{\pgfqpoint{8.275727in}{3.881056in}}{\pgfqpoint{8.269903in}{3.881056in}}%
\pgfpathcurveto{\pgfqpoint{8.264079in}{3.881056in}}{\pgfqpoint{8.258493in}{3.878742in}}{\pgfqpoint{8.254374in}{3.874624in}}%
\pgfpathcurveto{\pgfqpoint{8.250256in}{3.870506in}}{\pgfqpoint{8.247942in}{3.864919in}}{\pgfqpoint{8.247942in}{3.859095in}}%
\pgfpathcurveto{\pgfqpoint{8.247942in}{3.853272in}}{\pgfqpoint{8.250256in}{3.847685in}}{\pgfqpoint{8.254374in}{3.843567in}}%
\pgfpathcurveto{\pgfqpoint{8.258493in}{3.839449in}}{\pgfqpoint{8.264079in}{3.837135in}}{\pgfqpoint{8.269903in}{3.837135in}}%
\pgfpathlineto{\pgfqpoint{8.269903in}{3.837135in}}%
\pgfpathclose%
\pgfusepath{stroke,fill}%
\end{pgfscope}%
\begin{pgfscope}%
\pgfpathrectangle{\pgfqpoint{0.640323in}{0.527436in}}{\pgfqpoint{9.687500in}{3.850000in}}%
\pgfusepath{clip}%
\pgfsetbuttcap%
\pgfsetroundjoin%
\definecolor{currentfill}{rgb}{0.000000,0.000000,1.000000}%
\pgfsetfillcolor{currentfill}%
\pgfsetfillopacity{0.500000}%
\pgfsetlinewidth{1.003750pt}%
\definecolor{currentstroke}{rgb}{0.000000,0.000000,1.000000}%
\pgfsetstrokecolor{currentstroke}%
\pgfsetstrokeopacity{0.500000}%
\pgfsetdash{{3.700000pt}{1.600000pt}}{0.000000pt}%
\pgfpathmoveto{\pgfqpoint{8.449634in}{3.871580in}}%
\pgfpathcurveto{\pgfqpoint{8.455458in}{3.871580in}}{\pgfqpoint{8.461044in}{3.873894in}}{\pgfqpoint{8.465162in}{3.878012in}}%
\pgfpathcurveto{\pgfqpoint{8.469280in}{3.882130in}}{\pgfqpoint{8.471594in}{3.887717in}}{\pgfqpoint{8.471594in}{3.893541in}}%
\pgfpathcurveto{\pgfqpoint{8.471594in}{3.899364in}}{\pgfqpoint{8.469280in}{3.904951in}}{\pgfqpoint{8.465162in}{3.909069in}}%
\pgfpathcurveto{\pgfqpoint{8.461044in}{3.913187in}}{\pgfqpoint{8.455458in}{3.915501in}}{\pgfqpoint{8.449634in}{3.915501in}}%
\pgfpathcurveto{\pgfqpoint{8.443810in}{3.915501in}}{\pgfqpoint{8.438224in}{3.913187in}}{\pgfqpoint{8.434105in}{3.909069in}}%
\pgfpathcurveto{\pgfqpoint{8.429987in}{3.904951in}}{\pgfqpoint{8.427673in}{3.899364in}}{\pgfqpoint{8.427673in}{3.893541in}}%
\pgfpathcurveto{\pgfqpoint{8.427673in}{3.887717in}}{\pgfqpoint{8.429987in}{3.882130in}}{\pgfqpoint{8.434105in}{3.878012in}}%
\pgfpathcurveto{\pgfqpoint{8.438224in}{3.873894in}}{\pgfqpoint{8.443810in}{3.871580in}}{\pgfqpoint{8.449634in}{3.871580in}}%
\pgfpathlineto{\pgfqpoint{8.449634in}{3.871580in}}%
\pgfpathclose%
\pgfusepath{stroke,fill}%
\end{pgfscope}%
\begin{pgfscope}%
\pgfpathrectangle{\pgfqpoint{0.640323in}{0.527436in}}{\pgfqpoint{9.687500in}{3.850000in}}%
\pgfusepath{clip}%
\pgfsetbuttcap%
\pgfsetroundjoin%
\definecolor{currentfill}{rgb}{0.000000,0.000000,1.000000}%
\pgfsetfillcolor{currentfill}%
\pgfsetfillopacity{0.500000}%
\pgfsetlinewidth{1.003750pt}%
\definecolor{currentstroke}{rgb}{0.000000,0.000000,1.000000}%
\pgfsetstrokecolor{currentstroke}%
\pgfsetstrokeopacity{0.500000}%
\pgfsetdash{{3.700000pt}{1.600000pt}}{0.000000pt}%
\pgfpathmoveto{\pgfqpoint{8.629365in}{3.888514in}}%
\pgfpathcurveto{\pgfqpoint{8.635189in}{3.888514in}}{\pgfqpoint{8.640775in}{3.890828in}}{\pgfqpoint{8.644893in}{3.894946in}}%
\pgfpathcurveto{\pgfqpoint{8.649011in}{3.899064in}}{\pgfqpoint{8.651325in}{3.904650in}}{\pgfqpoint{8.651325in}{3.910474in}}%
\pgfpathcurveto{\pgfqpoint{8.651325in}{3.916298in}}{\pgfqpoint{8.649011in}{3.921884in}}{\pgfqpoint{8.644893in}{3.926003in}}%
\pgfpathcurveto{\pgfqpoint{8.640775in}{3.930121in}}{\pgfqpoint{8.635189in}{3.932435in}}{\pgfqpoint{8.629365in}{3.932435in}}%
\pgfpathcurveto{\pgfqpoint{8.623541in}{3.932435in}}{\pgfqpoint{8.617955in}{3.930121in}}{\pgfqpoint{8.613836in}{3.926003in}}%
\pgfpathcurveto{\pgfqpoint{8.609718in}{3.921884in}}{\pgfqpoint{8.607404in}{3.916298in}}{\pgfqpoint{8.607404in}{3.910474in}}%
\pgfpathcurveto{\pgfqpoint{8.607404in}{3.904650in}}{\pgfqpoint{8.609718in}{3.899064in}}{\pgfqpoint{8.613836in}{3.894946in}}%
\pgfpathcurveto{\pgfqpoint{8.617955in}{3.890828in}}{\pgfqpoint{8.623541in}{3.888514in}}{\pgfqpoint{8.629365in}{3.888514in}}%
\pgfpathlineto{\pgfqpoint{8.629365in}{3.888514in}}%
\pgfpathclose%
\pgfusepath{stroke,fill}%
\end{pgfscope}%
\begin{pgfscope}%
\pgfpathrectangle{\pgfqpoint{0.640323in}{0.527436in}}{\pgfqpoint{9.687500in}{3.850000in}}%
\pgfusepath{clip}%
\pgfsetbuttcap%
\pgfsetroundjoin%
\definecolor{currentfill}{rgb}{0.000000,0.000000,1.000000}%
\pgfsetfillcolor{currentfill}%
\pgfsetfillopacity{0.500000}%
\pgfsetlinewidth{1.003750pt}%
\definecolor{currentstroke}{rgb}{0.000000,0.000000,1.000000}%
\pgfsetstrokecolor{currentstroke}%
\pgfsetstrokeopacity{0.500000}%
\pgfsetdash{{3.700000pt}{1.600000pt}}{0.000000pt}%
\pgfpathmoveto{\pgfqpoint{8.809096in}{3.899089in}}%
\pgfpathcurveto{\pgfqpoint{8.814920in}{3.899089in}}{\pgfqpoint{8.820506in}{3.901403in}}{\pgfqpoint{8.824624in}{3.905521in}}%
\pgfpathcurveto{\pgfqpoint{8.828742in}{3.909639in}}{\pgfqpoint{8.831056in}{3.915225in}}{\pgfqpoint{8.831056in}{3.921049in}}%
\pgfpathcurveto{\pgfqpoint{8.831056in}{3.926873in}}{\pgfqpoint{8.828742in}{3.932459in}}{\pgfqpoint{8.824624in}{3.936578in}}%
\pgfpathcurveto{\pgfqpoint{8.820506in}{3.940696in}}{\pgfqpoint{8.814920in}{3.943010in}}{\pgfqpoint{8.809096in}{3.943010in}}%
\pgfpathcurveto{\pgfqpoint{8.803272in}{3.943010in}}{\pgfqpoint{8.797686in}{3.940696in}}{\pgfqpoint{8.793567in}{3.936578in}}%
\pgfpathcurveto{\pgfqpoint{8.789449in}{3.932459in}}{\pgfqpoint{8.787135in}{3.926873in}}{\pgfqpoint{8.787135in}{3.921049in}}%
\pgfpathcurveto{\pgfqpoint{8.787135in}{3.915225in}}{\pgfqpoint{8.789449in}{3.909639in}}{\pgfqpoint{8.793567in}{3.905521in}}%
\pgfpathcurveto{\pgfqpoint{8.797686in}{3.901403in}}{\pgfqpoint{8.803272in}{3.899089in}}{\pgfqpoint{8.809096in}{3.899089in}}%
\pgfpathlineto{\pgfqpoint{8.809096in}{3.899089in}}%
\pgfpathclose%
\pgfusepath{stroke,fill}%
\end{pgfscope}%
\begin{pgfscope}%
\pgfpathrectangle{\pgfqpoint{0.640323in}{0.527436in}}{\pgfqpoint{9.687500in}{3.850000in}}%
\pgfusepath{clip}%
\pgfsetbuttcap%
\pgfsetroundjoin%
\definecolor{currentfill}{rgb}{0.000000,0.000000,1.000000}%
\pgfsetfillcolor{currentfill}%
\pgfsetfillopacity{0.500000}%
\pgfsetlinewidth{1.003750pt}%
\definecolor{currentstroke}{rgb}{0.000000,0.000000,1.000000}%
\pgfsetstrokecolor{currentstroke}%
\pgfsetstrokeopacity{0.500000}%
\pgfsetdash{{3.700000pt}{1.600000pt}}{0.000000pt}%
\pgfpathmoveto{\pgfqpoint{8.988827in}{3.928268in}}%
\pgfpathcurveto{\pgfqpoint{8.994651in}{3.928268in}}{\pgfqpoint{9.000237in}{3.930582in}}{\pgfqpoint{9.004355in}{3.934700in}}%
\pgfpathcurveto{\pgfqpoint{9.008473in}{3.938819in}}{\pgfqpoint{9.010787in}{3.944405in}}{\pgfqpoint{9.010787in}{3.950229in}}%
\pgfpathcurveto{\pgfqpoint{9.010787in}{3.956053in}}{\pgfqpoint{9.008473in}{3.961639in}}{\pgfqpoint{9.004355in}{3.965757in}}%
\pgfpathcurveto{\pgfqpoint{9.000237in}{3.969875in}}{\pgfqpoint{8.994651in}{3.972189in}}{\pgfqpoint{8.988827in}{3.972189in}}%
\pgfpathcurveto{\pgfqpoint{8.983003in}{3.972189in}}{\pgfqpoint{8.977417in}{3.969875in}}{\pgfqpoint{8.973298in}{3.965757in}}%
\pgfpathcurveto{\pgfqpoint{8.969180in}{3.961639in}}{\pgfqpoint{8.966866in}{3.956053in}}{\pgfqpoint{8.966866in}{3.950229in}}%
\pgfpathcurveto{\pgfqpoint{8.966866in}{3.944405in}}{\pgfqpoint{8.969180in}{3.938819in}}{\pgfqpoint{8.973298in}{3.934700in}}%
\pgfpathcurveto{\pgfqpoint{8.977417in}{3.930582in}}{\pgfqpoint{8.983003in}{3.928268in}}{\pgfqpoint{8.988827in}{3.928268in}}%
\pgfpathlineto{\pgfqpoint{8.988827in}{3.928268in}}%
\pgfpathclose%
\pgfusepath{stroke,fill}%
\end{pgfscope}%
\begin{pgfscope}%
\pgfpathrectangle{\pgfqpoint{0.640323in}{0.527436in}}{\pgfqpoint{9.687500in}{3.850000in}}%
\pgfusepath{clip}%
\pgfsetbuttcap%
\pgfsetroundjoin%
\definecolor{currentfill}{rgb}{0.000000,0.000000,1.000000}%
\pgfsetfillcolor{currentfill}%
\pgfsetfillopacity{0.500000}%
\pgfsetlinewidth{1.003750pt}%
\definecolor{currentstroke}{rgb}{0.000000,0.000000,1.000000}%
\pgfsetstrokecolor{currentstroke}%
\pgfsetstrokeopacity{0.500000}%
\pgfsetdash{{3.700000pt}{1.600000pt}}{0.000000pt}%
\pgfpathmoveto{\pgfqpoint{9.168558in}{3.937900in}}%
\pgfpathcurveto{\pgfqpoint{9.174382in}{3.937900in}}{\pgfqpoint{9.179968in}{3.940213in}}{\pgfqpoint{9.184086in}{3.944332in}}%
\pgfpathcurveto{\pgfqpoint{9.188204in}{3.948450in}}{\pgfqpoint{9.190518in}{3.954036in}}{\pgfqpoint{9.190518in}{3.959860in}}%
\pgfpathcurveto{\pgfqpoint{9.190518in}{3.965684in}}{\pgfqpoint{9.188204in}{3.971270in}}{\pgfqpoint{9.184086in}{3.975388in}}%
\pgfpathcurveto{\pgfqpoint{9.179968in}{3.979506in}}{\pgfqpoint{9.174382in}{3.981820in}}{\pgfqpoint{9.168558in}{3.981820in}}%
\pgfpathcurveto{\pgfqpoint{9.162734in}{3.981820in}}{\pgfqpoint{9.157148in}{3.979506in}}{\pgfqpoint{9.153029in}{3.975388in}}%
\pgfpathcurveto{\pgfqpoint{9.148911in}{3.971270in}}{\pgfqpoint{9.146597in}{3.965684in}}{\pgfqpoint{9.146597in}{3.959860in}}%
\pgfpathcurveto{\pgfqpoint{9.146597in}{3.954036in}}{\pgfqpoint{9.148911in}{3.948450in}}{\pgfqpoint{9.153029in}{3.944332in}}%
\pgfpathcurveto{\pgfqpoint{9.157148in}{3.940213in}}{\pgfqpoint{9.162734in}{3.937900in}}{\pgfqpoint{9.168558in}{3.937900in}}%
\pgfpathlineto{\pgfqpoint{9.168558in}{3.937900in}}%
\pgfpathclose%
\pgfusepath{stroke,fill}%
\end{pgfscope}%
\begin{pgfscope}%
\pgfpathrectangle{\pgfqpoint{0.640323in}{0.527436in}}{\pgfqpoint{9.687500in}{3.850000in}}%
\pgfusepath{clip}%
\pgfsetbuttcap%
\pgfsetroundjoin%
\definecolor{currentfill}{rgb}{0.000000,0.000000,1.000000}%
\pgfsetfillcolor{currentfill}%
\pgfsetfillopacity{0.500000}%
\pgfsetlinewidth{1.003750pt}%
\definecolor{currentstroke}{rgb}{0.000000,0.000000,1.000000}%
\pgfsetstrokecolor{currentstroke}%
\pgfsetstrokeopacity{0.500000}%
\pgfsetdash{{3.700000pt}{1.600000pt}}{0.000000pt}%
\pgfpathmoveto{\pgfqpoint{9.348289in}{3.950685in}}%
\pgfpathcurveto{\pgfqpoint{9.354113in}{3.950685in}}{\pgfqpoint{9.359699in}{3.952999in}}{\pgfqpoint{9.363817in}{3.957117in}}%
\pgfpathcurveto{\pgfqpoint{9.367935in}{3.961235in}}{\pgfqpoint{9.370249in}{3.966822in}}{\pgfqpoint{9.370249in}{3.972646in}}%
\pgfpathcurveto{\pgfqpoint{9.370249in}{3.978470in}}{\pgfqpoint{9.367935in}{3.984056in}}{\pgfqpoint{9.363817in}{3.988174in}}%
\pgfpathcurveto{\pgfqpoint{9.359699in}{3.992292in}}{\pgfqpoint{9.354113in}{3.994606in}}{\pgfqpoint{9.348289in}{3.994606in}}%
\pgfpathcurveto{\pgfqpoint{9.342465in}{3.994606in}}{\pgfqpoint{9.336879in}{3.992292in}}{\pgfqpoint{9.332760in}{3.988174in}}%
\pgfpathcurveto{\pgfqpoint{9.328642in}{3.984056in}}{\pgfqpoint{9.326328in}{3.978470in}}{\pgfqpoint{9.326328in}{3.972646in}}%
\pgfpathcurveto{\pgfqpoint{9.326328in}{3.966822in}}{\pgfqpoint{9.328642in}{3.961235in}}{\pgfqpoint{9.332760in}{3.957117in}}%
\pgfpathcurveto{\pgfqpoint{9.336879in}{3.952999in}}{\pgfqpoint{9.342465in}{3.950685in}}{\pgfqpoint{9.348289in}{3.950685in}}%
\pgfpathlineto{\pgfqpoint{9.348289in}{3.950685in}}%
\pgfpathclose%
\pgfusepath{stroke,fill}%
\end{pgfscope}%
\begin{pgfscope}%
\pgfpathrectangle{\pgfqpoint{0.640323in}{0.527436in}}{\pgfqpoint{9.687500in}{3.850000in}}%
\pgfusepath{clip}%
\pgfsetbuttcap%
\pgfsetroundjoin%
\definecolor{currentfill}{rgb}{0.000000,0.000000,1.000000}%
\pgfsetfillcolor{currentfill}%
\pgfsetfillopacity{0.500000}%
\pgfsetlinewidth{1.003750pt}%
\definecolor{currentstroke}{rgb}{0.000000,0.000000,1.000000}%
\pgfsetstrokecolor{currentstroke}%
\pgfsetstrokeopacity{0.500000}%
\pgfsetdash{{3.700000pt}{1.600000pt}}{0.000000pt}%
\pgfpathmoveto{\pgfqpoint{9.528020in}{3.971419in}}%
\pgfpathcurveto{\pgfqpoint{9.533844in}{3.971419in}}{\pgfqpoint{9.539430in}{3.973733in}}{\pgfqpoint{9.543548in}{3.977851in}}%
\pgfpathcurveto{\pgfqpoint{9.547666in}{3.981970in}}{\pgfqpoint{9.549980in}{3.987556in}}{\pgfqpoint{9.549980in}{3.993380in}}%
\pgfpathcurveto{\pgfqpoint{9.549980in}{3.999204in}}{\pgfqpoint{9.547666in}{4.004790in}}{\pgfqpoint{9.543548in}{4.008908in}}%
\pgfpathcurveto{\pgfqpoint{9.539430in}{4.013026in}}{\pgfqpoint{9.533844in}{4.015340in}}{\pgfqpoint{9.528020in}{4.015340in}}%
\pgfpathcurveto{\pgfqpoint{9.522196in}{4.015340in}}{\pgfqpoint{9.516610in}{4.013026in}}{\pgfqpoint{9.512491in}{4.008908in}}%
\pgfpathcurveto{\pgfqpoint{9.508373in}{4.004790in}}{\pgfqpoint{9.506059in}{3.999204in}}{\pgfqpoint{9.506059in}{3.993380in}}%
\pgfpathcurveto{\pgfqpoint{9.506059in}{3.987556in}}{\pgfqpoint{9.508373in}{3.981970in}}{\pgfqpoint{9.512491in}{3.977851in}}%
\pgfpathcurveto{\pgfqpoint{9.516610in}{3.973733in}}{\pgfqpoint{9.522196in}{3.971419in}}{\pgfqpoint{9.528020in}{3.971419in}}%
\pgfpathlineto{\pgfqpoint{9.528020in}{3.971419in}}%
\pgfpathclose%
\pgfusepath{stroke,fill}%
\end{pgfscope}%
\begin{pgfscope}%
\pgfpathrectangle{\pgfqpoint{0.640323in}{0.527436in}}{\pgfqpoint{9.687500in}{3.850000in}}%
\pgfusepath{clip}%
\pgfsetbuttcap%
\pgfsetroundjoin%
\definecolor{currentfill}{rgb}{0.000000,0.000000,1.000000}%
\pgfsetfillcolor{currentfill}%
\pgfsetfillopacity{0.500000}%
\pgfsetlinewidth{1.003750pt}%
\definecolor{currentstroke}{rgb}{0.000000,0.000000,1.000000}%
\pgfsetstrokecolor{currentstroke}%
\pgfsetstrokeopacity{0.500000}%
\pgfsetdash{{3.700000pt}{1.600000pt}}{0.000000pt}%
\pgfpathmoveto{\pgfqpoint{9.707751in}{4.004070in}}%
\pgfpathcurveto{\pgfqpoint{9.713575in}{4.004070in}}{\pgfqpoint{9.719161in}{4.006384in}}{\pgfqpoint{9.723279in}{4.010502in}}%
\pgfpathcurveto{\pgfqpoint{9.727397in}{4.014620in}}{\pgfqpoint{9.729711in}{4.020206in}}{\pgfqpoint{9.729711in}{4.026030in}}%
\pgfpathcurveto{\pgfqpoint{9.729711in}{4.031854in}}{\pgfqpoint{9.727397in}{4.037440in}}{\pgfqpoint{9.723279in}{4.041558in}}%
\pgfpathcurveto{\pgfqpoint{9.719161in}{4.045677in}}{\pgfqpoint{9.713575in}{4.047990in}}{\pgfqpoint{9.707751in}{4.047990in}}%
\pgfpathcurveto{\pgfqpoint{9.701927in}{4.047990in}}{\pgfqpoint{9.696340in}{4.045677in}}{\pgfqpoint{9.692222in}{4.041558in}}%
\pgfpathcurveto{\pgfqpoint{9.688104in}{4.037440in}}{\pgfqpoint{9.685790in}{4.031854in}}{\pgfqpoint{9.685790in}{4.026030in}}%
\pgfpathcurveto{\pgfqpoint{9.685790in}{4.020206in}}{\pgfqpoint{9.688104in}{4.014620in}}{\pgfqpoint{9.692222in}{4.010502in}}%
\pgfpathcurveto{\pgfqpoint{9.696340in}{4.006384in}}{\pgfqpoint{9.701927in}{4.004070in}}{\pgfqpoint{9.707751in}{4.004070in}}%
\pgfpathlineto{\pgfqpoint{9.707751in}{4.004070in}}%
\pgfpathclose%
\pgfusepath{stroke,fill}%
\end{pgfscope}%
\begin{pgfscope}%
\pgfpathrectangle{\pgfqpoint{0.640323in}{0.527436in}}{\pgfqpoint{9.687500in}{3.850000in}}%
\pgfusepath{clip}%
\pgfsetbuttcap%
\pgfsetroundjoin%
\definecolor{currentfill}{rgb}{0.000000,0.000000,1.000000}%
\pgfsetfillcolor{currentfill}%
\pgfsetfillopacity{0.500000}%
\pgfsetlinewidth{1.003750pt}%
\definecolor{currentstroke}{rgb}{0.000000,0.000000,1.000000}%
\pgfsetstrokecolor{currentstroke}%
\pgfsetstrokeopacity{0.500000}%
\pgfsetdash{{3.700000pt}{1.600000pt}}{0.000000pt}%
\pgfpathmoveto{\pgfqpoint{9.887482in}{4.000456in}}%
\pgfpathcurveto{\pgfqpoint{9.893306in}{4.000456in}}{\pgfqpoint{9.898892in}{4.002770in}}{\pgfqpoint{9.903010in}{4.006888in}}%
\pgfpathcurveto{\pgfqpoint{9.907128in}{4.011006in}}{\pgfqpoint{9.909442in}{4.016592in}}{\pgfqpoint{9.909442in}{4.022416in}}%
\pgfpathcurveto{\pgfqpoint{9.909442in}{4.028240in}}{\pgfqpoint{9.907128in}{4.033826in}}{\pgfqpoint{9.903010in}{4.037944in}}%
\pgfpathcurveto{\pgfqpoint{9.898892in}{4.042063in}}{\pgfqpoint{9.893306in}{4.044376in}}{\pgfqpoint{9.887482in}{4.044376in}}%
\pgfpathcurveto{\pgfqpoint{9.881658in}{4.044376in}}{\pgfqpoint{9.876071in}{4.042063in}}{\pgfqpoint{9.871953in}{4.037944in}}%
\pgfpathcurveto{\pgfqpoint{9.867835in}{4.033826in}}{\pgfqpoint{9.865521in}{4.028240in}}{\pgfqpoint{9.865521in}{4.022416in}}%
\pgfpathcurveto{\pgfqpoint{9.865521in}{4.016592in}}{\pgfqpoint{9.867835in}{4.011006in}}{\pgfqpoint{9.871953in}{4.006888in}}%
\pgfpathcurveto{\pgfqpoint{9.876071in}{4.002770in}}{\pgfqpoint{9.881658in}{4.000456in}}{\pgfqpoint{9.887482in}{4.000456in}}%
\pgfpathlineto{\pgfqpoint{9.887482in}{4.000456in}}%
\pgfpathclose%
\pgfusepath{stroke,fill}%
\end{pgfscope}%
\begin{pgfscope}%
\pgfpathrectangle{\pgfqpoint{0.640323in}{0.527436in}}{\pgfqpoint{9.687500in}{3.850000in}}%
\pgfusepath{clip}%
\pgfsetbuttcap%
\pgfsetroundjoin%
\definecolor{currentfill}{rgb}{0.980392,0.164706,0.333333}%
\pgfsetfillcolor{currentfill}%
\pgfsetfillopacity{0.500000}%
\pgfsetlinewidth{1.003750pt}%
\definecolor{currentstroke}{rgb}{0.980392,0.164706,0.333333}%
\pgfsetstrokecolor{currentstroke}%
\pgfsetstrokeopacity{0.500000}%
\pgfsetdash{{3.700000pt}{1.600000pt}}{0.000000pt}%
\pgfpathmoveto{\pgfqpoint{1.080663in}{0.637874in}}%
\pgfpathcurveto{\pgfqpoint{1.086487in}{0.637874in}}{\pgfqpoint{1.092074in}{0.640188in}}{\pgfqpoint{1.096192in}{0.644306in}}%
\pgfpathcurveto{\pgfqpoint{1.100310in}{0.648424in}}{\pgfqpoint{1.102624in}{0.654010in}}{\pgfqpoint{1.102624in}{0.659834in}}%
\pgfpathcurveto{\pgfqpoint{1.102624in}{0.665658in}}{\pgfqpoint{1.100310in}{0.671244in}}{\pgfqpoint{1.096192in}{0.675362in}}%
\pgfpathcurveto{\pgfqpoint{1.092074in}{0.679481in}}{\pgfqpoint{1.086487in}{0.681794in}}{\pgfqpoint{1.080663in}{0.681794in}}%
\pgfpathcurveto{\pgfqpoint{1.074839in}{0.681794in}}{\pgfqpoint{1.069253in}{0.679481in}}{\pgfqpoint{1.065135in}{0.675362in}}%
\pgfpathcurveto{\pgfqpoint{1.061017in}{0.671244in}}{\pgfqpoint{1.058703in}{0.665658in}}{\pgfqpoint{1.058703in}{0.659834in}}%
\pgfpathcurveto{\pgfqpoint{1.058703in}{0.654010in}}{\pgfqpoint{1.061017in}{0.648424in}}{\pgfqpoint{1.065135in}{0.644306in}}%
\pgfpathcurveto{\pgfqpoint{1.069253in}{0.640188in}}{\pgfqpoint{1.074839in}{0.637874in}}{\pgfqpoint{1.080663in}{0.637874in}}%
\pgfpathlineto{\pgfqpoint{1.080663in}{0.637874in}}%
\pgfpathclose%
\pgfusepath{stroke,fill}%
\end{pgfscope}%
\begin{pgfscope}%
\pgfpathrectangle{\pgfqpoint{0.640323in}{0.527436in}}{\pgfqpoint{9.687500in}{3.850000in}}%
\pgfusepath{clip}%
\pgfsetbuttcap%
\pgfsetroundjoin%
\definecolor{currentfill}{rgb}{0.980392,0.164706,0.333333}%
\pgfsetfillcolor{currentfill}%
\pgfsetfillopacity{0.500000}%
\pgfsetlinewidth{1.003750pt}%
\definecolor{currentstroke}{rgb}{0.980392,0.164706,0.333333}%
\pgfsetstrokecolor{currentstroke}%
\pgfsetstrokeopacity{0.500000}%
\pgfsetdash{{3.700000pt}{1.600000pt}}{0.000000pt}%
\pgfpathmoveto{\pgfqpoint{1.260394in}{0.638941in}}%
\pgfpathcurveto{\pgfqpoint{1.266218in}{0.638941in}}{\pgfqpoint{1.271805in}{0.641255in}}{\pgfqpoint{1.275923in}{0.645373in}}%
\pgfpathcurveto{\pgfqpoint{1.280041in}{0.649491in}}{\pgfqpoint{1.282355in}{0.655077in}}{\pgfqpoint{1.282355in}{0.660901in}}%
\pgfpathcurveto{\pgfqpoint{1.282355in}{0.666725in}}{\pgfqpoint{1.280041in}{0.672312in}}{\pgfqpoint{1.275923in}{0.676430in}}%
\pgfpathcurveto{\pgfqpoint{1.271805in}{0.680548in}}{\pgfqpoint{1.266218in}{0.682862in}}{\pgfqpoint{1.260394in}{0.682862in}}%
\pgfpathcurveto{\pgfqpoint{1.254570in}{0.682862in}}{\pgfqpoint{1.248984in}{0.680548in}}{\pgfqpoint{1.244866in}{0.676430in}}%
\pgfpathcurveto{\pgfqpoint{1.240748in}{0.672312in}}{\pgfqpoint{1.238434in}{0.666725in}}{\pgfqpoint{1.238434in}{0.660901in}}%
\pgfpathcurveto{\pgfqpoint{1.238434in}{0.655077in}}{\pgfqpoint{1.240748in}{0.649491in}}{\pgfqpoint{1.244866in}{0.645373in}}%
\pgfpathcurveto{\pgfqpoint{1.248984in}{0.641255in}}{\pgfqpoint{1.254570in}{0.638941in}}{\pgfqpoint{1.260394in}{0.638941in}}%
\pgfpathlineto{\pgfqpoint{1.260394in}{0.638941in}}%
\pgfpathclose%
\pgfusepath{stroke,fill}%
\end{pgfscope}%
\begin{pgfscope}%
\pgfpathrectangle{\pgfqpoint{0.640323in}{0.527436in}}{\pgfqpoint{9.687500in}{3.850000in}}%
\pgfusepath{clip}%
\pgfsetbuttcap%
\pgfsetroundjoin%
\definecolor{currentfill}{rgb}{0.980392,0.164706,0.333333}%
\pgfsetfillcolor{currentfill}%
\pgfsetfillopacity{0.500000}%
\pgfsetlinewidth{1.003750pt}%
\definecolor{currentstroke}{rgb}{0.980392,0.164706,0.333333}%
\pgfsetstrokecolor{currentstroke}%
\pgfsetstrokeopacity{0.500000}%
\pgfsetdash{{3.700000pt}{1.600000pt}}{0.000000pt}%
\pgfpathmoveto{\pgfqpoint{1.440125in}{0.639736in}}%
\pgfpathcurveto{\pgfqpoint{1.445949in}{0.639736in}}{\pgfqpoint{1.451535in}{0.642049in}}{\pgfqpoint{1.455654in}{0.646168in}}%
\pgfpathcurveto{\pgfqpoint{1.459772in}{0.650286in}}{\pgfqpoint{1.462086in}{0.655872in}}{\pgfqpoint{1.462086in}{0.661696in}}%
\pgfpathcurveto{\pgfqpoint{1.462086in}{0.667520in}}{\pgfqpoint{1.459772in}{0.673106in}}{\pgfqpoint{1.455654in}{0.677224in}}%
\pgfpathcurveto{\pgfqpoint{1.451535in}{0.681342in}}{\pgfqpoint{1.445949in}{0.683656in}}{\pgfqpoint{1.440125in}{0.683656in}}%
\pgfpathcurveto{\pgfqpoint{1.434301in}{0.683656in}}{\pgfqpoint{1.428715in}{0.681342in}}{\pgfqpoint{1.424597in}{0.677224in}}%
\pgfpathcurveto{\pgfqpoint{1.420479in}{0.673106in}}{\pgfqpoint{1.418165in}{0.667520in}}{\pgfqpoint{1.418165in}{0.661696in}}%
\pgfpathcurveto{\pgfqpoint{1.418165in}{0.655872in}}{\pgfqpoint{1.420479in}{0.650286in}}{\pgfqpoint{1.424597in}{0.646168in}}%
\pgfpathcurveto{\pgfqpoint{1.428715in}{0.642049in}}{\pgfqpoint{1.434301in}{0.639736in}}{\pgfqpoint{1.440125in}{0.639736in}}%
\pgfpathlineto{\pgfqpoint{1.440125in}{0.639736in}}%
\pgfpathclose%
\pgfusepath{stroke,fill}%
\end{pgfscope}%
\begin{pgfscope}%
\pgfpathrectangle{\pgfqpoint{0.640323in}{0.527436in}}{\pgfqpoint{9.687500in}{3.850000in}}%
\pgfusepath{clip}%
\pgfsetbuttcap%
\pgfsetroundjoin%
\definecolor{currentfill}{rgb}{0.980392,0.164706,0.333333}%
\pgfsetfillcolor{currentfill}%
\pgfsetfillopacity{0.500000}%
\pgfsetlinewidth{1.003750pt}%
\definecolor{currentstroke}{rgb}{0.980392,0.164706,0.333333}%
\pgfsetstrokecolor{currentstroke}%
\pgfsetstrokeopacity{0.500000}%
\pgfsetdash{{3.700000pt}{1.600000pt}}{0.000000pt}%
\pgfpathmoveto{\pgfqpoint{1.619856in}{0.642392in}}%
\pgfpathcurveto{\pgfqpoint{1.625680in}{0.642392in}}{\pgfqpoint{1.631266in}{0.644706in}}{\pgfqpoint{1.635385in}{0.648824in}}%
\pgfpathcurveto{\pgfqpoint{1.639503in}{0.652942in}}{\pgfqpoint{1.641817in}{0.658528in}}{\pgfqpoint{1.641817in}{0.664352in}}%
\pgfpathcurveto{\pgfqpoint{1.641817in}{0.670176in}}{\pgfqpoint{1.639503in}{0.675762in}}{\pgfqpoint{1.635385in}{0.679880in}}%
\pgfpathcurveto{\pgfqpoint{1.631266in}{0.683998in}}{\pgfqpoint{1.625680in}{0.686312in}}{\pgfqpoint{1.619856in}{0.686312in}}%
\pgfpathcurveto{\pgfqpoint{1.614032in}{0.686312in}}{\pgfqpoint{1.608446in}{0.683998in}}{\pgfqpoint{1.604328in}{0.679880in}}%
\pgfpathcurveto{\pgfqpoint{1.600210in}{0.675762in}}{\pgfqpoint{1.597896in}{0.670176in}}{\pgfqpoint{1.597896in}{0.664352in}}%
\pgfpathcurveto{\pgfqpoint{1.597896in}{0.658528in}}{\pgfqpoint{1.600210in}{0.652942in}}{\pgfqpoint{1.604328in}{0.648824in}}%
\pgfpathcurveto{\pgfqpoint{1.608446in}{0.644706in}}{\pgfqpoint{1.614032in}{0.642392in}}{\pgfqpoint{1.619856in}{0.642392in}}%
\pgfpathlineto{\pgfqpoint{1.619856in}{0.642392in}}%
\pgfpathclose%
\pgfusepath{stroke,fill}%
\end{pgfscope}%
\begin{pgfscope}%
\pgfpathrectangle{\pgfqpoint{0.640323in}{0.527436in}}{\pgfqpoint{9.687500in}{3.850000in}}%
\pgfusepath{clip}%
\pgfsetbuttcap%
\pgfsetroundjoin%
\definecolor{currentfill}{rgb}{0.980392,0.164706,0.333333}%
\pgfsetfillcolor{currentfill}%
\pgfsetfillopacity{0.500000}%
\pgfsetlinewidth{1.003750pt}%
\definecolor{currentstroke}{rgb}{0.980392,0.164706,0.333333}%
\pgfsetstrokecolor{currentstroke}%
\pgfsetstrokeopacity{0.500000}%
\pgfsetdash{{3.700000pt}{1.600000pt}}{0.000000pt}%
\pgfpathmoveto{\pgfqpoint{1.799587in}{0.643477in}}%
\pgfpathcurveto{\pgfqpoint{1.805411in}{0.643477in}}{\pgfqpoint{1.810997in}{0.645791in}}{\pgfqpoint{1.815116in}{0.649909in}}%
\pgfpathcurveto{\pgfqpoint{1.819234in}{0.654027in}}{\pgfqpoint{1.821548in}{0.659613in}}{\pgfqpoint{1.821548in}{0.665437in}}%
\pgfpathcurveto{\pgfqpoint{1.821548in}{0.671261in}}{\pgfqpoint{1.819234in}{0.676848in}}{\pgfqpoint{1.815116in}{0.680966in}}%
\pgfpathcurveto{\pgfqpoint{1.810997in}{0.685084in}}{\pgfqpoint{1.805411in}{0.687398in}}{\pgfqpoint{1.799587in}{0.687398in}}%
\pgfpathcurveto{\pgfqpoint{1.793763in}{0.687398in}}{\pgfqpoint{1.788177in}{0.685084in}}{\pgfqpoint{1.784059in}{0.680966in}}%
\pgfpathcurveto{\pgfqpoint{1.779941in}{0.676848in}}{\pgfqpoint{1.777627in}{0.671261in}}{\pgfqpoint{1.777627in}{0.665437in}}%
\pgfpathcurveto{\pgfqpoint{1.777627in}{0.659613in}}{\pgfqpoint{1.779941in}{0.654027in}}{\pgfqpoint{1.784059in}{0.649909in}}%
\pgfpathcurveto{\pgfqpoint{1.788177in}{0.645791in}}{\pgfqpoint{1.793763in}{0.643477in}}{\pgfqpoint{1.799587in}{0.643477in}}%
\pgfpathlineto{\pgfqpoint{1.799587in}{0.643477in}}%
\pgfpathclose%
\pgfusepath{stroke,fill}%
\end{pgfscope}%
\begin{pgfscope}%
\pgfpathrectangle{\pgfqpoint{0.640323in}{0.527436in}}{\pgfqpoint{9.687500in}{3.850000in}}%
\pgfusepath{clip}%
\pgfsetbuttcap%
\pgfsetroundjoin%
\definecolor{currentfill}{rgb}{0.980392,0.164706,0.333333}%
\pgfsetfillcolor{currentfill}%
\pgfsetfillopacity{0.500000}%
\pgfsetlinewidth{1.003750pt}%
\definecolor{currentstroke}{rgb}{0.980392,0.164706,0.333333}%
\pgfsetstrokecolor{currentstroke}%
\pgfsetstrokeopacity{0.500000}%
\pgfsetdash{{3.700000pt}{1.600000pt}}{0.000000pt}%
\pgfpathmoveto{\pgfqpoint{1.979318in}{0.652398in}}%
\pgfpathcurveto{\pgfqpoint{1.985142in}{0.652398in}}{\pgfqpoint{1.990728in}{0.654711in}}{\pgfqpoint{1.994847in}{0.658830in}}%
\pgfpathcurveto{\pgfqpoint{1.998965in}{0.662948in}}{\pgfqpoint{2.001279in}{0.668534in}}{\pgfqpoint{2.001279in}{0.674358in}}%
\pgfpathcurveto{\pgfqpoint{2.001279in}{0.680182in}}{\pgfqpoint{1.998965in}{0.685768in}}{\pgfqpoint{1.994847in}{0.689886in}}%
\pgfpathcurveto{\pgfqpoint{1.990728in}{0.694004in}}{\pgfqpoint{1.985142in}{0.696318in}}{\pgfqpoint{1.979318in}{0.696318in}}%
\pgfpathcurveto{\pgfqpoint{1.973494in}{0.696318in}}{\pgfqpoint{1.967908in}{0.694004in}}{\pgfqpoint{1.963790in}{0.689886in}}%
\pgfpathcurveto{\pgfqpoint{1.959672in}{0.685768in}}{\pgfqpoint{1.957358in}{0.680182in}}{\pgfqpoint{1.957358in}{0.674358in}}%
\pgfpathcurveto{\pgfqpoint{1.957358in}{0.668534in}}{\pgfqpoint{1.959672in}{0.662948in}}{\pgfqpoint{1.963790in}{0.658830in}}%
\pgfpathcurveto{\pgfqpoint{1.967908in}{0.654711in}}{\pgfqpoint{1.973494in}{0.652398in}}{\pgfqpoint{1.979318in}{0.652398in}}%
\pgfpathlineto{\pgfqpoint{1.979318in}{0.652398in}}%
\pgfpathclose%
\pgfusepath{stroke,fill}%
\end{pgfscope}%
\begin{pgfscope}%
\pgfpathrectangle{\pgfqpoint{0.640323in}{0.527436in}}{\pgfqpoint{9.687500in}{3.850000in}}%
\pgfusepath{clip}%
\pgfsetbuttcap%
\pgfsetroundjoin%
\definecolor{currentfill}{rgb}{0.980392,0.164706,0.333333}%
\pgfsetfillcolor{currentfill}%
\pgfsetfillopacity{0.500000}%
\pgfsetlinewidth{1.003750pt}%
\definecolor{currentstroke}{rgb}{0.980392,0.164706,0.333333}%
\pgfsetstrokecolor{currentstroke}%
\pgfsetstrokeopacity{0.500000}%
\pgfsetdash{{3.700000pt}{1.600000pt}}{0.000000pt}%
\pgfpathmoveto{\pgfqpoint{2.159049in}{0.655296in}}%
\pgfpathcurveto{\pgfqpoint{2.164873in}{0.655296in}}{\pgfqpoint{2.170459in}{0.657610in}}{\pgfqpoint{2.174578in}{0.661728in}}%
\pgfpathcurveto{\pgfqpoint{2.178696in}{0.665846in}}{\pgfqpoint{2.181010in}{0.671432in}}{\pgfqpoint{2.181010in}{0.677256in}}%
\pgfpathcurveto{\pgfqpoint{2.181010in}{0.683080in}}{\pgfqpoint{2.178696in}{0.688667in}}{\pgfqpoint{2.174578in}{0.692785in}}%
\pgfpathcurveto{\pgfqpoint{2.170459in}{0.696903in}}{\pgfqpoint{2.164873in}{0.699217in}}{\pgfqpoint{2.159049in}{0.699217in}}%
\pgfpathcurveto{\pgfqpoint{2.153225in}{0.699217in}}{\pgfqpoint{2.147639in}{0.696903in}}{\pgfqpoint{2.143521in}{0.692785in}}%
\pgfpathcurveto{\pgfqpoint{2.139403in}{0.688667in}}{\pgfqpoint{2.137089in}{0.683080in}}{\pgfqpoint{2.137089in}{0.677256in}}%
\pgfpathcurveto{\pgfqpoint{2.137089in}{0.671432in}}{\pgfqpoint{2.139403in}{0.665846in}}{\pgfqpoint{2.143521in}{0.661728in}}%
\pgfpathcurveto{\pgfqpoint{2.147639in}{0.657610in}}{\pgfqpoint{2.153225in}{0.655296in}}{\pgfqpoint{2.159049in}{0.655296in}}%
\pgfpathlineto{\pgfqpoint{2.159049in}{0.655296in}}%
\pgfpathclose%
\pgfusepath{stroke,fill}%
\end{pgfscope}%
\begin{pgfscope}%
\pgfpathrectangle{\pgfqpoint{0.640323in}{0.527436in}}{\pgfqpoint{9.687500in}{3.850000in}}%
\pgfusepath{clip}%
\pgfsetbuttcap%
\pgfsetroundjoin%
\definecolor{currentfill}{rgb}{0.980392,0.164706,0.333333}%
\pgfsetfillcolor{currentfill}%
\pgfsetfillopacity{0.500000}%
\pgfsetlinewidth{1.003750pt}%
\definecolor{currentstroke}{rgb}{0.980392,0.164706,0.333333}%
\pgfsetstrokecolor{currentstroke}%
\pgfsetstrokeopacity{0.500000}%
\pgfsetdash{{3.700000pt}{1.600000pt}}{0.000000pt}%
\pgfpathmoveto{\pgfqpoint{2.338780in}{0.693403in}}%
\pgfpathcurveto{\pgfqpoint{2.344604in}{0.693403in}}{\pgfqpoint{2.350190in}{0.695717in}}{\pgfqpoint{2.354309in}{0.699835in}}%
\pgfpathcurveto{\pgfqpoint{2.358427in}{0.703954in}}{\pgfqpoint{2.360741in}{0.709540in}}{\pgfqpoint{2.360741in}{0.715364in}}%
\pgfpathcurveto{\pgfqpoint{2.360741in}{0.721188in}}{\pgfqpoint{2.358427in}{0.726774in}}{\pgfqpoint{2.354309in}{0.730892in}}%
\pgfpathcurveto{\pgfqpoint{2.350190in}{0.735010in}}{\pgfqpoint{2.344604in}{0.737324in}}{\pgfqpoint{2.338780in}{0.737324in}}%
\pgfpathcurveto{\pgfqpoint{2.332956in}{0.737324in}}{\pgfqpoint{2.327370in}{0.735010in}}{\pgfqpoint{2.323252in}{0.730892in}}%
\pgfpathcurveto{\pgfqpoint{2.319134in}{0.726774in}}{\pgfqpoint{2.316820in}{0.721188in}}{\pgfqpoint{2.316820in}{0.715364in}}%
\pgfpathcurveto{\pgfqpoint{2.316820in}{0.709540in}}{\pgfqpoint{2.319134in}{0.703954in}}{\pgfqpoint{2.323252in}{0.699835in}}%
\pgfpathcurveto{\pgfqpoint{2.327370in}{0.695717in}}{\pgfqpoint{2.332956in}{0.693403in}}{\pgfqpoint{2.338780in}{0.693403in}}%
\pgfpathlineto{\pgfqpoint{2.338780in}{0.693403in}}%
\pgfpathclose%
\pgfusepath{stroke,fill}%
\end{pgfscope}%
\begin{pgfscope}%
\pgfpathrectangle{\pgfqpoint{0.640323in}{0.527436in}}{\pgfqpoint{9.687500in}{3.850000in}}%
\pgfusepath{clip}%
\pgfsetbuttcap%
\pgfsetroundjoin%
\definecolor{currentfill}{rgb}{0.980392,0.164706,0.333333}%
\pgfsetfillcolor{currentfill}%
\pgfsetfillopacity{0.500000}%
\pgfsetlinewidth{1.003750pt}%
\definecolor{currentstroke}{rgb}{0.980392,0.164706,0.333333}%
\pgfsetstrokecolor{currentstroke}%
\pgfsetstrokeopacity{0.500000}%
\pgfsetdash{{3.700000pt}{1.600000pt}}{0.000000pt}%
\pgfpathmoveto{\pgfqpoint{2.518511in}{0.731335in}}%
\pgfpathcurveto{\pgfqpoint{2.524335in}{0.731335in}}{\pgfqpoint{2.529921in}{0.733648in}}{\pgfqpoint{2.534040in}{0.737767in}}%
\pgfpathcurveto{\pgfqpoint{2.538158in}{0.741885in}}{\pgfqpoint{2.540472in}{0.747471in}}{\pgfqpoint{2.540472in}{0.753295in}}%
\pgfpathcurveto{\pgfqpoint{2.540472in}{0.759119in}}{\pgfqpoint{2.538158in}{0.764705in}}{\pgfqpoint{2.534040in}{0.768823in}}%
\pgfpathcurveto{\pgfqpoint{2.529921in}{0.772941in}}{\pgfqpoint{2.524335in}{0.775255in}}{\pgfqpoint{2.518511in}{0.775255in}}%
\pgfpathcurveto{\pgfqpoint{2.512687in}{0.775255in}}{\pgfqpoint{2.507101in}{0.772941in}}{\pgfqpoint{2.502983in}{0.768823in}}%
\pgfpathcurveto{\pgfqpoint{2.498865in}{0.764705in}}{\pgfqpoint{2.496551in}{0.759119in}}{\pgfqpoint{2.496551in}{0.753295in}}%
\pgfpathcurveto{\pgfqpoint{2.496551in}{0.747471in}}{\pgfqpoint{2.498865in}{0.741885in}}{\pgfqpoint{2.502983in}{0.737767in}}%
\pgfpathcurveto{\pgfqpoint{2.507101in}{0.733648in}}{\pgfqpoint{2.512687in}{0.731335in}}{\pgfqpoint{2.518511in}{0.731335in}}%
\pgfpathlineto{\pgfqpoint{2.518511in}{0.731335in}}%
\pgfpathclose%
\pgfusepath{stroke,fill}%
\end{pgfscope}%
\begin{pgfscope}%
\pgfpathrectangle{\pgfqpoint{0.640323in}{0.527436in}}{\pgfqpoint{9.687500in}{3.850000in}}%
\pgfusepath{clip}%
\pgfsetbuttcap%
\pgfsetroundjoin%
\definecolor{currentfill}{rgb}{0.980392,0.164706,0.333333}%
\pgfsetfillcolor{currentfill}%
\pgfsetfillopacity{0.500000}%
\pgfsetlinewidth{1.003750pt}%
\definecolor{currentstroke}{rgb}{0.980392,0.164706,0.333333}%
\pgfsetstrokecolor{currentstroke}%
\pgfsetstrokeopacity{0.500000}%
\pgfsetdash{{3.700000pt}{1.600000pt}}{0.000000pt}%
\pgfpathmoveto{\pgfqpoint{2.698242in}{1.175332in}}%
\pgfpathcurveto{\pgfqpoint{2.704066in}{1.175332in}}{\pgfqpoint{2.709652in}{1.177646in}}{\pgfqpoint{2.713771in}{1.181764in}}%
\pgfpathcurveto{\pgfqpoint{2.717889in}{1.185882in}}{\pgfqpoint{2.720203in}{1.191468in}}{\pgfqpoint{2.720203in}{1.197292in}}%
\pgfpathcurveto{\pgfqpoint{2.720203in}{1.203116in}}{\pgfqpoint{2.717889in}{1.208702in}}{\pgfqpoint{2.713771in}{1.212821in}}%
\pgfpathcurveto{\pgfqpoint{2.709652in}{1.216939in}}{\pgfqpoint{2.704066in}{1.219253in}}{\pgfqpoint{2.698242in}{1.219253in}}%
\pgfpathcurveto{\pgfqpoint{2.692418in}{1.219253in}}{\pgfqpoint{2.686832in}{1.216939in}}{\pgfqpoint{2.682714in}{1.212821in}}%
\pgfpathcurveto{\pgfqpoint{2.678596in}{1.208702in}}{\pgfqpoint{2.676282in}{1.203116in}}{\pgfqpoint{2.676282in}{1.197292in}}%
\pgfpathcurveto{\pgfqpoint{2.676282in}{1.191468in}}{\pgfqpoint{2.678596in}{1.185882in}}{\pgfqpoint{2.682714in}{1.181764in}}%
\pgfpathcurveto{\pgfqpoint{2.686832in}{1.177646in}}{\pgfqpoint{2.692418in}{1.175332in}}{\pgfqpoint{2.698242in}{1.175332in}}%
\pgfpathlineto{\pgfqpoint{2.698242in}{1.175332in}}%
\pgfpathclose%
\pgfusepath{stroke,fill}%
\end{pgfscope}%
\begin{pgfscope}%
\pgfpathrectangle{\pgfqpoint{0.640323in}{0.527436in}}{\pgfqpoint{9.687500in}{3.850000in}}%
\pgfusepath{clip}%
\pgfsetbuttcap%
\pgfsetroundjoin%
\definecolor{currentfill}{rgb}{0.980392,0.164706,0.333333}%
\pgfsetfillcolor{currentfill}%
\pgfsetfillopacity{0.500000}%
\pgfsetlinewidth{1.003750pt}%
\definecolor{currentstroke}{rgb}{0.980392,0.164706,0.333333}%
\pgfsetstrokecolor{currentstroke}%
\pgfsetstrokeopacity{0.500000}%
\pgfsetdash{{3.700000pt}{1.600000pt}}{0.000000pt}%
\pgfpathmoveto{\pgfqpoint{2.877973in}{1.680843in}}%
\pgfpathcurveto{\pgfqpoint{2.883797in}{1.680843in}}{\pgfqpoint{2.889383in}{1.683157in}}{\pgfqpoint{2.893501in}{1.687275in}}%
\pgfpathcurveto{\pgfqpoint{2.897620in}{1.691393in}}{\pgfqpoint{2.899934in}{1.696980in}}{\pgfqpoint{2.899934in}{1.702804in}}%
\pgfpathcurveto{\pgfqpoint{2.899934in}{1.708628in}}{\pgfqpoint{2.897620in}{1.714214in}}{\pgfqpoint{2.893501in}{1.718332in}}%
\pgfpathcurveto{\pgfqpoint{2.889383in}{1.722450in}}{\pgfqpoint{2.883797in}{1.724764in}}{\pgfqpoint{2.877973in}{1.724764in}}%
\pgfpathcurveto{\pgfqpoint{2.872149in}{1.724764in}}{\pgfqpoint{2.866563in}{1.722450in}}{\pgfqpoint{2.862445in}{1.718332in}}%
\pgfpathcurveto{\pgfqpoint{2.858327in}{1.714214in}}{\pgfqpoint{2.856013in}{1.708628in}}{\pgfqpoint{2.856013in}{1.702804in}}%
\pgfpathcurveto{\pgfqpoint{2.856013in}{1.696980in}}{\pgfqpoint{2.858327in}{1.691393in}}{\pgfqpoint{2.862445in}{1.687275in}}%
\pgfpathcurveto{\pgfqpoint{2.866563in}{1.683157in}}{\pgfqpoint{2.872149in}{1.680843in}}{\pgfqpoint{2.877973in}{1.680843in}}%
\pgfpathlineto{\pgfqpoint{2.877973in}{1.680843in}}%
\pgfpathclose%
\pgfusepath{stroke,fill}%
\end{pgfscope}%
\begin{pgfscope}%
\pgfpathrectangle{\pgfqpoint{0.640323in}{0.527436in}}{\pgfqpoint{9.687500in}{3.850000in}}%
\pgfusepath{clip}%
\pgfsetbuttcap%
\pgfsetroundjoin%
\definecolor{currentfill}{rgb}{0.980392,0.164706,0.333333}%
\pgfsetfillcolor{currentfill}%
\pgfsetfillopacity{0.500000}%
\pgfsetlinewidth{1.003750pt}%
\definecolor{currentstroke}{rgb}{0.980392,0.164706,0.333333}%
\pgfsetstrokecolor{currentstroke}%
\pgfsetstrokeopacity{0.500000}%
\pgfsetdash{{3.700000pt}{1.600000pt}}{0.000000pt}%
\pgfpathmoveto{\pgfqpoint{3.057704in}{2.052437in}}%
\pgfpathcurveto{\pgfqpoint{3.063528in}{2.052437in}}{\pgfqpoint{3.069114in}{2.054750in}}{\pgfqpoint{3.073232in}{2.058869in}}%
\pgfpathcurveto{\pgfqpoint{3.077351in}{2.062987in}}{\pgfqpoint{3.079664in}{2.068573in}}{\pgfqpoint{3.079664in}{2.074397in}}%
\pgfpathcurveto{\pgfqpoint{3.079664in}{2.080221in}}{\pgfqpoint{3.077351in}{2.085807in}}{\pgfqpoint{3.073232in}{2.089925in}}%
\pgfpathcurveto{\pgfqpoint{3.069114in}{2.094043in}}{\pgfqpoint{3.063528in}{2.096357in}}{\pgfqpoint{3.057704in}{2.096357in}}%
\pgfpathcurveto{\pgfqpoint{3.051880in}{2.096357in}}{\pgfqpoint{3.046294in}{2.094043in}}{\pgfqpoint{3.042176in}{2.089925in}}%
\pgfpathcurveto{\pgfqpoint{3.038058in}{2.085807in}}{\pgfqpoint{3.035744in}{2.080221in}}{\pgfqpoint{3.035744in}{2.074397in}}%
\pgfpathcurveto{\pgfqpoint{3.035744in}{2.068573in}}{\pgfqpoint{3.038058in}{2.062987in}}{\pgfqpoint{3.042176in}{2.058869in}}%
\pgfpathcurveto{\pgfqpoint{3.046294in}{2.054750in}}{\pgfqpoint{3.051880in}{2.052437in}}{\pgfqpoint{3.057704in}{2.052437in}}%
\pgfpathlineto{\pgfqpoint{3.057704in}{2.052437in}}%
\pgfpathclose%
\pgfusepath{stroke,fill}%
\end{pgfscope}%
\begin{pgfscope}%
\pgfpathrectangle{\pgfqpoint{0.640323in}{0.527436in}}{\pgfqpoint{9.687500in}{3.850000in}}%
\pgfusepath{clip}%
\pgfsetbuttcap%
\pgfsetroundjoin%
\definecolor{currentfill}{rgb}{0.980392,0.164706,0.333333}%
\pgfsetfillcolor{currentfill}%
\pgfsetfillopacity{0.500000}%
\pgfsetlinewidth{1.003750pt}%
\definecolor{currentstroke}{rgb}{0.980392,0.164706,0.333333}%
\pgfsetstrokecolor{currentstroke}%
\pgfsetstrokeopacity{0.500000}%
\pgfsetdash{{3.700000pt}{1.600000pt}}{0.000000pt}%
\pgfpathmoveto{\pgfqpoint{3.237435in}{2.232499in}}%
\pgfpathcurveto{\pgfqpoint{3.243259in}{2.232499in}}{\pgfqpoint{3.248845in}{2.234813in}}{\pgfqpoint{3.252963in}{2.238931in}}%
\pgfpathcurveto{\pgfqpoint{3.257082in}{2.243049in}}{\pgfqpoint{3.259395in}{2.248635in}}{\pgfqpoint{3.259395in}{2.254459in}}%
\pgfpathcurveto{\pgfqpoint{3.259395in}{2.260283in}}{\pgfqpoint{3.257082in}{2.265869in}}{\pgfqpoint{3.252963in}{2.269987in}}%
\pgfpathcurveto{\pgfqpoint{3.248845in}{2.274105in}}{\pgfqpoint{3.243259in}{2.276419in}}{\pgfqpoint{3.237435in}{2.276419in}}%
\pgfpathcurveto{\pgfqpoint{3.231611in}{2.276419in}}{\pgfqpoint{3.226025in}{2.274105in}}{\pgfqpoint{3.221907in}{2.269987in}}%
\pgfpathcurveto{\pgfqpoint{3.217789in}{2.265869in}}{\pgfqpoint{3.215475in}{2.260283in}}{\pgfqpoint{3.215475in}{2.254459in}}%
\pgfpathcurveto{\pgfqpoint{3.215475in}{2.248635in}}{\pgfqpoint{3.217789in}{2.243049in}}{\pgfqpoint{3.221907in}{2.238931in}}%
\pgfpathcurveto{\pgfqpoint{3.226025in}{2.234813in}}{\pgfqpoint{3.231611in}{2.232499in}}{\pgfqpoint{3.237435in}{2.232499in}}%
\pgfpathlineto{\pgfqpoint{3.237435in}{2.232499in}}%
\pgfpathclose%
\pgfusepath{stroke,fill}%
\end{pgfscope}%
\begin{pgfscope}%
\pgfpathrectangle{\pgfqpoint{0.640323in}{0.527436in}}{\pgfqpoint{9.687500in}{3.850000in}}%
\pgfusepath{clip}%
\pgfsetbuttcap%
\pgfsetroundjoin%
\definecolor{currentfill}{rgb}{0.980392,0.164706,0.333333}%
\pgfsetfillcolor{currentfill}%
\pgfsetfillopacity{0.500000}%
\pgfsetlinewidth{1.003750pt}%
\definecolor{currentstroke}{rgb}{0.980392,0.164706,0.333333}%
\pgfsetstrokecolor{currentstroke}%
\pgfsetstrokeopacity{0.500000}%
\pgfsetdash{{3.700000pt}{1.600000pt}}{0.000000pt}%
\pgfpathmoveto{\pgfqpoint{3.417166in}{2.409456in}}%
\pgfpathcurveto{\pgfqpoint{3.422990in}{2.409456in}}{\pgfqpoint{3.428576in}{2.411770in}}{\pgfqpoint{3.432694in}{2.415888in}}%
\pgfpathcurveto{\pgfqpoint{3.436813in}{2.420006in}}{\pgfqpoint{3.439126in}{2.425592in}}{\pgfqpoint{3.439126in}{2.431416in}}%
\pgfpathcurveto{\pgfqpoint{3.439126in}{2.437240in}}{\pgfqpoint{3.436813in}{2.442826in}}{\pgfqpoint{3.432694in}{2.446944in}}%
\pgfpathcurveto{\pgfqpoint{3.428576in}{2.451062in}}{\pgfqpoint{3.422990in}{2.453376in}}{\pgfqpoint{3.417166in}{2.453376in}}%
\pgfpathcurveto{\pgfqpoint{3.411342in}{2.453376in}}{\pgfqpoint{3.405756in}{2.451062in}}{\pgfqpoint{3.401638in}{2.446944in}}%
\pgfpathcurveto{\pgfqpoint{3.397520in}{2.442826in}}{\pgfqpoint{3.395206in}{2.437240in}}{\pgfqpoint{3.395206in}{2.431416in}}%
\pgfpathcurveto{\pgfqpoint{3.395206in}{2.425592in}}{\pgfqpoint{3.397520in}{2.420006in}}{\pgfqpoint{3.401638in}{2.415888in}}%
\pgfpathcurveto{\pgfqpoint{3.405756in}{2.411770in}}{\pgfqpoint{3.411342in}{2.409456in}}{\pgfqpoint{3.417166in}{2.409456in}}%
\pgfpathlineto{\pgfqpoint{3.417166in}{2.409456in}}%
\pgfpathclose%
\pgfusepath{stroke,fill}%
\end{pgfscope}%
\begin{pgfscope}%
\pgfpathrectangle{\pgfqpoint{0.640323in}{0.527436in}}{\pgfqpoint{9.687500in}{3.850000in}}%
\pgfusepath{clip}%
\pgfsetbuttcap%
\pgfsetroundjoin%
\definecolor{currentfill}{rgb}{0.980392,0.164706,0.333333}%
\pgfsetfillcolor{currentfill}%
\pgfsetfillopacity{0.500000}%
\pgfsetlinewidth{1.003750pt}%
\definecolor{currentstroke}{rgb}{0.980392,0.164706,0.333333}%
\pgfsetstrokecolor{currentstroke}%
\pgfsetstrokeopacity{0.500000}%
\pgfsetdash{{3.700000pt}{1.600000pt}}{0.000000pt}%
\pgfpathmoveto{\pgfqpoint{3.596897in}{2.547013in}}%
\pgfpathcurveto{\pgfqpoint{3.602721in}{2.547013in}}{\pgfqpoint{3.608307in}{2.549326in}}{\pgfqpoint{3.612425in}{2.553445in}}%
\pgfpathcurveto{\pgfqpoint{3.616544in}{2.557563in}}{\pgfqpoint{3.618857in}{2.563149in}}{\pgfqpoint{3.618857in}{2.568973in}}%
\pgfpathcurveto{\pgfqpoint{3.618857in}{2.574797in}}{\pgfqpoint{3.616544in}{2.580383in}}{\pgfqpoint{3.612425in}{2.584501in}}%
\pgfpathcurveto{\pgfqpoint{3.608307in}{2.588619in}}{\pgfqpoint{3.602721in}{2.590933in}}{\pgfqpoint{3.596897in}{2.590933in}}%
\pgfpathcurveto{\pgfqpoint{3.591073in}{2.590933in}}{\pgfqpoint{3.585487in}{2.588619in}}{\pgfqpoint{3.581369in}{2.584501in}}%
\pgfpathcurveto{\pgfqpoint{3.577251in}{2.580383in}}{\pgfqpoint{3.574937in}{2.574797in}}{\pgfqpoint{3.574937in}{2.568973in}}%
\pgfpathcurveto{\pgfqpoint{3.574937in}{2.563149in}}{\pgfqpoint{3.577251in}{2.557563in}}{\pgfqpoint{3.581369in}{2.553445in}}%
\pgfpathcurveto{\pgfqpoint{3.585487in}{2.549326in}}{\pgfqpoint{3.591073in}{2.547013in}}{\pgfqpoint{3.596897in}{2.547013in}}%
\pgfpathlineto{\pgfqpoint{3.596897in}{2.547013in}}%
\pgfpathclose%
\pgfusepath{stroke,fill}%
\end{pgfscope}%
\begin{pgfscope}%
\pgfpathrectangle{\pgfqpoint{0.640323in}{0.527436in}}{\pgfqpoint{9.687500in}{3.850000in}}%
\pgfusepath{clip}%
\pgfsetbuttcap%
\pgfsetroundjoin%
\definecolor{currentfill}{rgb}{0.980392,0.164706,0.333333}%
\pgfsetfillcolor{currentfill}%
\pgfsetfillopacity{0.500000}%
\pgfsetlinewidth{1.003750pt}%
\definecolor{currentstroke}{rgb}{0.980392,0.164706,0.333333}%
\pgfsetstrokecolor{currentstroke}%
\pgfsetstrokeopacity{0.500000}%
\pgfsetdash{{3.700000pt}{1.600000pt}}{0.000000pt}%
\pgfpathmoveto{\pgfqpoint{3.776628in}{2.697784in}}%
\pgfpathcurveto{\pgfqpoint{3.782452in}{2.697784in}}{\pgfqpoint{3.788038in}{2.700097in}}{\pgfqpoint{3.792156in}{2.704216in}}%
\pgfpathcurveto{\pgfqpoint{3.796275in}{2.708334in}}{\pgfqpoint{3.798588in}{2.713920in}}{\pgfqpoint{3.798588in}{2.719744in}}%
\pgfpathcurveto{\pgfqpoint{3.798588in}{2.725568in}}{\pgfqpoint{3.796275in}{2.731154in}}{\pgfqpoint{3.792156in}{2.735272in}}%
\pgfpathcurveto{\pgfqpoint{3.788038in}{2.739390in}}{\pgfqpoint{3.782452in}{2.741704in}}{\pgfqpoint{3.776628in}{2.741704in}}%
\pgfpathcurveto{\pgfqpoint{3.770804in}{2.741704in}}{\pgfqpoint{3.765218in}{2.739390in}}{\pgfqpoint{3.761100in}{2.735272in}}%
\pgfpathcurveto{\pgfqpoint{3.756982in}{2.731154in}}{\pgfqpoint{3.754668in}{2.725568in}}{\pgfqpoint{3.754668in}{2.719744in}}%
\pgfpathcurveto{\pgfqpoint{3.754668in}{2.713920in}}{\pgfqpoint{3.756982in}{2.708334in}}{\pgfqpoint{3.761100in}{2.704216in}}%
\pgfpathcurveto{\pgfqpoint{3.765218in}{2.700097in}}{\pgfqpoint{3.770804in}{2.697784in}}{\pgfqpoint{3.776628in}{2.697784in}}%
\pgfpathlineto{\pgfqpoint{3.776628in}{2.697784in}}%
\pgfpathclose%
\pgfusepath{stroke,fill}%
\end{pgfscope}%
\begin{pgfscope}%
\pgfpathrectangle{\pgfqpoint{0.640323in}{0.527436in}}{\pgfqpoint{9.687500in}{3.850000in}}%
\pgfusepath{clip}%
\pgfsetbuttcap%
\pgfsetroundjoin%
\definecolor{currentfill}{rgb}{0.980392,0.164706,0.333333}%
\pgfsetfillcolor{currentfill}%
\pgfsetfillopacity{0.500000}%
\pgfsetlinewidth{1.003750pt}%
\definecolor{currentstroke}{rgb}{0.980392,0.164706,0.333333}%
\pgfsetstrokecolor{currentstroke}%
\pgfsetstrokeopacity{0.500000}%
\pgfsetdash{{3.700000pt}{1.600000pt}}{0.000000pt}%
\pgfpathmoveto{\pgfqpoint{3.956359in}{2.820605in}}%
\pgfpathcurveto{\pgfqpoint{3.962183in}{2.820605in}}{\pgfqpoint{3.967769in}{2.822919in}}{\pgfqpoint{3.971887in}{2.827037in}}%
\pgfpathcurveto{\pgfqpoint{3.976006in}{2.831155in}}{\pgfqpoint{3.978319in}{2.836741in}}{\pgfqpoint{3.978319in}{2.842565in}}%
\pgfpathcurveto{\pgfqpoint{3.978319in}{2.848389in}}{\pgfqpoint{3.976006in}{2.853975in}}{\pgfqpoint{3.971887in}{2.858093in}}%
\pgfpathcurveto{\pgfqpoint{3.967769in}{2.862211in}}{\pgfqpoint{3.962183in}{2.864525in}}{\pgfqpoint{3.956359in}{2.864525in}}%
\pgfpathcurveto{\pgfqpoint{3.950535in}{2.864525in}}{\pgfqpoint{3.944949in}{2.862211in}}{\pgfqpoint{3.940831in}{2.858093in}}%
\pgfpathcurveto{\pgfqpoint{3.936713in}{2.853975in}}{\pgfqpoint{3.934399in}{2.848389in}}{\pgfqpoint{3.934399in}{2.842565in}}%
\pgfpathcurveto{\pgfqpoint{3.934399in}{2.836741in}}{\pgfqpoint{3.936713in}{2.831155in}}{\pgfqpoint{3.940831in}{2.827037in}}%
\pgfpathcurveto{\pgfqpoint{3.944949in}{2.822919in}}{\pgfqpoint{3.950535in}{2.820605in}}{\pgfqpoint{3.956359in}{2.820605in}}%
\pgfpathlineto{\pgfqpoint{3.956359in}{2.820605in}}%
\pgfpathclose%
\pgfusepath{stroke,fill}%
\end{pgfscope}%
\begin{pgfscope}%
\pgfpathrectangle{\pgfqpoint{0.640323in}{0.527436in}}{\pgfqpoint{9.687500in}{3.850000in}}%
\pgfusepath{clip}%
\pgfsetbuttcap%
\pgfsetroundjoin%
\definecolor{currentfill}{rgb}{0.980392,0.164706,0.333333}%
\pgfsetfillcolor{currentfill}%
\pgfsetfillopacity{0.500000}%
\pgfsetlinewidth{1.003750pt}%
\definecolor{currentstroke}{rgb}{0.980392,0.164706,0.333333}%
\pgfsetstrokecolor{currentstroke}%
\pgfsetstrokeopacity{0.500000}%
\pgfsetdash{{3.700000pt}{1.600000pt}}{0.000000pt}%
\pgfpathmoveto{\pgfqpoint{4.136090in}{2.903094in}}%
\pgfpathcurveto{\pgfqpoint{4.141914in}{2.903094in}}{\pgfqpoint{4.147500in}{2.905408in}}{\pgfqpoint{4.151618in}{2.909526in}}%
\pgfpathcurveto{\pgfqpoint{4.155737in}{2.913644in}}{\pgfqpoint{4.158050in}{2.919230in}}{\pgfqpoint{4.158050in}{2.925054in}}%
\pgfpathcurveto{\pgfqpoint{4.158050in}{2.930878in}}{\pgfqpoint{4.155737in}{2.936464in}}{\pgfqpoint{4.151618in}{2.940583in}}%
\pgfpathcurveto{\pgfqpoint{4.147500in}{2.944701in}}{\pgfqpoint{4.141914in}{2.947015in}}{\pgfqpoint{4.136090in}{2.947015in}}%
\pgfpathcurveto{\pgfqpoint{4.130266in}{2.947015in}}{\pgfqpoint{4.124680in}{2.944701in}}{\pgfqpoint{4.120562in}{2.940583in}}%
\pgfpathcurveto{\pgfqpoint{4.116444in}{2.936464in}}{\pgfqpoint{4.114130in}{2.930878in}}{\pgfqpoint{4.114130in}{2.925054in}}%
\pgfpathcurveto{\pgfqpoint{4.114130in}{2.919230in}}{\pgfqpoint{4.116444in}{2.913644in}}{\pgfqpoint{4.120562in}{2.909526in}}%
\pgfpathcurveto{\pgfqpoint{4.124680in}{2.905408in}}{\pgfqpoint{4.130266in}{2.903094in}}{\pgfqpoint{4.136090in}{2.903094in}}%
\pgfpathlineto{\pgfqpoint{4.136090in}{2.903094in}}%
\pgfpathclose%
\pgfusepath{stroke,fill}%
\end{pgfscope}%
\begin{pgfscope}%
\pgfpathrectangle{\pgfqpoint{0.640323in}{0.527436in}}{\pgfqpoint{9.687500in}{3.850000in}}%
\pgfusepath{clip}%
\pgfsetbuttcap%
\pgfsetroundjoin%
\definecolor{currentfill}{rgb}{0.980392,0.164706,0.333333}%
\pgfsetfillcolor{currentfill}%
\pgfsetfillopacity{0.500000}%
\pgfsetlinewidth{1.003750pt}%
\definecolor{currentstroke}{rgb}{0.980392,0.164706,0.333333}%
\pgfsetstrokecolor{currentstroke}%
\pgfsetstrokeopacity{0.500000}%
\pgfsetdash{{3.700000pt}{1.600000pt}}{0.000000pt}%
\pgfpathmoveto{\pgfqpoint{4.315821in}{3.004840in}}%
\pgfpathcurveto{\pgfqpoint{4.321645in}{3.004840in}}{\pgfqpoint{4.327231in}{3.007154in}}{\pgfqpoint{4.331349in}{3.011272in}}%
\pgfpathcurveto{\pgfqpoint{4.335467in}{3.015390in}}{\pgfqpoint{4.337781in}{3.020976in}}{\pgfqpoint{4.337781in}{3.026800in}}%
\pgfpathcurveto{\pgfqpoint{4.337781in}{3.032624in}}{\pgfqpoint{4.335467in}{3.038210in}}{\pgfqpoint{4.331349in}{3.042328in}}%
\pgfpathcurveto{\pgfqpoint{4.327231in}{3.046446in}}{\pgfqpoint{4.321645in}{3.048760in}}{\pgfqpoint{4.315821in}{3.048760in}}%
\pgfpathcurveto{\pgfqpoint{4.309997in}{3.048760in}}{\pgfqpoint{4.304411in}{3.046446in}}{\pgfqpoint{4.300293in}{3.042328in}}%
\pgfpathcurveto{\pgfqpoint{4.296175in}{3.038210in}}{\pgfqpoint{4.293861in}{3.032624in}}{\pgfqpoint{4.293861in}{3.026800in}}%
\pgfpathcurveto{\pgfqpoint{4.293861in}{3.020976in}}{\pgfqpoint{4.296175in}{3.015390in}}{\pgfqpoint{4.300293in}{3.011272in}}%
\pgfpathcurveto{\pgfqpoint{4.304411in}{3.007154in}}{\pgfqpoint{4.309997in}{3.004840in}}{\pgfqpoint{4.315821in}{3.004840in}}%
\pgfpathlineto{\pgfqpoint{4.315821in}{3.004840in}}%
\pgfpathclose%
\pgfusepath{stroke,fill}%
\end{pgfscope}%
\begin{pgfscope}%
\pgfpathrectangle{\pgfqpoint{0.640323in}{0.527436in}}{\pgfqpoint{9.687500in}{3.850000in}}%
\pgfusepath{clip}%
\pgfsetbuttcap%
\pgfsetroundjoin%
\definecolor{currentfill}{rgb}{0.980392,0.164706,0.333333}%
\pgfsetfillcolor{currentfill}%
\pgfsetfillopacity{0.500000}%
\pgfsetlinewidth{1.003750pt}%
\definecolor{currentstroke}{rgb}{0.980392,0.164706,0.333333}%
\pgfsetstrokecolor{currentstroke}%
\pgfsetstrokeopacity{0.500000}%
\pgfsetdash{{3.700000pt}{1.600000pt}}{0.000000pt}%
\pgfpathmoveto{\pgfqpoint{4.495552in}{3.082386in}}%
\pgfpathcurveto{\pgfqpoint{4.501376in}{3.082386in}}{\pgfqpoint{4.506962in}{3.084700in}}{\pgfqpoint{4.511080in}{3.088818in}}%
\pgfpathcurveto{\pgfqpoint{4.515198in}{3.092936in}}{\pgfqpoint{4.517512in}{3.098522in}}{\pgfqpoint{4.517512in}{3.104346in}}%
\pgfpathcurveto{\pgfqpoint{4.517512in}{3.110170in}}{\pgfqpoint{4.515198in}{3.115757in}}{\pgfqpoint{4.511080in}{3.119875in}}%
\pgfpathcurveto{\pgfqpoint{4.506962in}{3.123993in}}{\pgfqpoint{4.501376in}{3.126307in}}{\pgfqpoint{4.495552in}{3.126307in}}%
\pgfpathcurveto{\pgfqpoint{4.489728in}{3.126307in}}{\pgfqpoint{4.484142in}{3.123993in}}{\pgfqpoint{4.480024in}{3.119875in}}%
\pgfpathcurveto{\pgfqpoint{4.475906in}{3.115757in}}{\pgfqpoint{4.473592in}{3.110170in}}{\pgfqpoint{4.473592in}{3.104346in}}%
\pgfpathcurveto{\pgfqpoint{4.473592in}{3.098522in}}{\pgfqpoint{4.475906in}{3.092936in}}{\pgfqpoint{4.480024in}{3.088818in}}%
\pgfpathcurveto{\pgfqpoint{4.484142in}{3.084700in}}{\pgfqpoint{4.489728in}{3.082386in}}{\pgfqpoint{4.495552in}{3.082386in}}%
\pgfpathlineto{\pgfqpoint{4.495552in}{3.082386in}}%
\pgfpathclose%
\pgfusepath{stroke,fill}%
\end{pgfscope}%
\begin{pgfscope}%
\pgfpathrectangle{\pgfqpoint{0.640323in}{0.527436in}}{\pgfqpoint{9.687500in}{3.850000in}}%
\pgfusepath{clip}%
\pgfsetbuttcap%
\pgfsetroundjoin%
\definecolor{currentfill}{rgb}{0.980392,0.164706,0.333333}%
\pgfsetfillcolor{currentfill}%
\pgfsetfillopacity{0.500000}%
\pgfsetlinewidth{1.003750pt}%
\definecolor{currentstroke}{rgb}{0.980392,0.164706,0.333333}%
\pgfsetstrokecolor{currentstroke}%
\pgfsetstrokeopacity{0.500000}%
\pgfsetdash{{3.700000pt}{1.600000pt}}{0.000000pt}%
\pgfpathmoveto{\pgfqpoint{4.675283in}{3.168042in}}%
\pgfpathcurveto{\pgfqpoint{4.681107in}{3.168042in}}{\pgfqpoint{4.686693in}{3.170356in}}{\pgfqpoint{4.690811in}{3.174474in}}%
\pgfpathcurveto{\pgfqpoint{4.694929in}{3.178593in}}{\pgfqpoint{4.697243in}{3.184179in}}{\pgfqpoint{4.697243in}{3.190003in}}%
\pgfpathcurveto{\pgfqpoint{4.697243in}{3.195827in}}{\pgfqpoint{4.694929in}{3.201413in}}{\pgfqpoint{4.690811in}{3.205531in}}%
\pgfpathcurveto{\pgfqpoint{4.686693in}{3.209649in}}{\pgfqpoint{4.681107in}{3.211963in}}{\pgfqpoint{4.675283in}{3.211963in}}%
\pgfpathcurveto{\pgfqpoint{4.669459in}{3.211963in}}{\pgfqpoint{4.663873in}{3.209649in}}{\pgfqpoint{4.659755in}{3.205531in}}%
\pgfpathcurveto{\pgfqpoint{4.655637in}{3.201413in}}{\pgfqpoint{4.653323in}{3.195827in}}{\pgfqpoint{4.653323in}{3.190003in}}%
\pgfpathcurveto{\pgfqpoint{4.653323in}{3.184179in}}{\pgfqpoint{4.655637in}{3.178593in}}{\pgfqpoint{4.659755in}{3.174474in}}%
\pgfpathcurveto{\pgfqpoint{4.663873in}{3.170356in}}{\pgfqpoint{4.669459in}{3.168042in}}{\pgfqpoint{4.675283in}{3.168042in}}%
\pgfpathlineto{\pgfqpoint{4.675283in}{3.168042in}}%
\pgfpathclose%
\pgfusepath{stroke,fill}%
\end{pgfscope}%
\begin{pgfscope}%
\pgfpathrectangle{\pgfqpoint{0.640323in}{0.527436in}}{\pgfqpoint{9.687500in}{3.850000in}}%
\pgfusepath{clip}%
\pgfsetbuttcap%
\pgfsetroundjoin%
\definecolor{currentfill}{rgb}{0.980392,0.164706,0.333333}%
\pgfsetfillcolor{currentfill}%
\pgfsetfillopacity{0.500000}%
\pgfsetlinewidth{1.003750pt}%
\definecolor{currentstroke}{rgb}{0.980392,0.164706,0.333333}%
\pgfsetstrokecolor{currentstroke}%
\pgfsetstrokeopacity{0.500000}%
\pgfsetdash{{3.700000pt}{1.600000pt}}{0.000000pt}%
\pgfpathmoveto{\pgfqpoint{4.855014in}{3.205450in}}%
\pgfpathcurveto{\pgfqpoint{4.860838in}{3.205450in}}{\pgfqpoint{4.866424in}{3.207763in}}{\pgfqpoint{4.870542in}{3.211882in}}%
\pgfpathcurveto{\pgfqpoint{4.874660in}{3.216000in}}{\pgfqpoint{4.876974in}{3.221586in}}{\pgfqpoint{4.876974in}{3.227410in}}%
\pgfpathcurveto{\pgfqpoint{4.876974in}{3.233234in}}{\pgfqpoint{4.874660in}{3.238820in}}{\pgfqpoint{4.870542in}{3.242938in}}%
\pgfpathcurveto{\pgfqpoint{4.866424in}{3.247056in}}{\pgfqpoint{4.860838in}{3.249370in}}{\pgfqpoint{4.855014in}{3.249370in}}%
\pgfpathcurveto{\pgfqpoint{4.849190in}{3.249370in}}{\pgfqpoint{4.843604in}{3.247056in}}{\pgfqpoint{4.839486in}{3.242938in}}%
\pgfpathcurveto{\pgfqpoint{4.835368in}{3.238820in}}{\pgfqpoint{4.833054in}{3.233234in}}{\pgfqpoint{4.833054in}{3.227410in}}%
\pgfpathcurveto{\pgfqpoint{4.833054in}{3.221586in}}{\pgfqpoint{4.835368in}{3.216000in}}{\pgfqpoint{4.839486in}{3.211882in}}%
\pgfpathcurveto{\pgfqpoint{4.843604in}{3.207763in}}{\pgfqpoint{4.849190in}{3.205450in}}{\pgfqpoint{4.855014in}{3.205450in}}%
\pgfpathlineto{\pgfqpoint{4.855014in}{3.205450in}}%
\pgfpathclose%
\pgfusepath{stroke,fill}%
\end{pgfscope}%
\begin{pgfscope}%
\pgfpathrectangle{\pgfqpoint{0.640323in}{0.527436in}}{\pgfqpoint{9.687500in}{3.850000in}}%
\pgfusepath{clip}%
\pgfsetbuttcap%
\pgfsetroundjoin%
\definecolor{currentfill}{rgb}{0.980392,0.164706,0.333333}%
\pgfsetfillcolor{currentfill}%
\pgfsetfillopacity{0.500000}%
\pgfsetlinewidth{1.003750pt}%
\definecolor{currentstroke}{rgb}{0.980392,0.164706,0.333333}%
\pgfsetstrokecolor{currentstroke}%
\pgfsetstrokeopacity{0.500000}%
\pgfsetdash{{3.700000pt}{1.600000pt}}{0.000000pt}%
\pgfpathmoveto{\pgfqpoint{5.034745in}{3.260107in}}%
\pgfpathcurveto{\pgfqpoint{5.040569in}{3.260107in}}{\pgfqpoint{5.046155in}{3.262421in}}{\pgfqpoint{5.050273in}{3.266539in}}%
\pgfpathcurveto{\pgfqpoint{5.054391in}{3.270657in}}{\pgfqpoint{5.056705in}{3.276243in}}{\pgfqpoint{5.056705in}{3.282067in}}%
\pgfpathcurveto{\pgfqpoint{5.056705in}{3.287891in}}{\pgfqpoint{5.054391in}{3.293477in}}{\pgfqpoint{5.050273in}{3.297596in}}%
\pgfpathcurveto{\pgfqpoint{5.046155in}{3.301714in}}{\pgfqpoint{5.040569in}{3.304028in}}{\pgfqpoint{5.034745in}{3.304028in}}%
\pgfpathcurveto{\pgfqpoint{5.028921in}{3.304028in}}{\pgfqpoint{5.023335in}{3.301714in}}{\pgfqpoint{5.019217in}{3.297596in}}%
\pgfpathcurveto{\pgfqpoint{5.015099in}{3.293477in}}{\pgfqpoint{5.012785in}{3.287891in}}{\pgfqpoint{5.012785in}{3.282067in}}%
\pgfpathcurveto{\pgfqpoint{5.012785in}{3.276243in}}{\pgfqpoint{5.015099in}{3.270657in}}{\pgfqpoint{5.019217in}{3.266539in}}%
\pgfpathcurveto{\pgfqpoint{5.023335in}{3.262421in}}{\pgfqpoint{5.028921in}{3.260107in}}{\pgfqpoint{5.034745in}{3.260107in}}%
\pgfpathlineto{\pgfqpoint{5.034745in}{3.260107in}}%
\pgfpathclose%
\pgfusepath{stroke,fill}%
\end{pgfscope}%
\begin{pgfscope}%
\pgfpathrectangle{\pgfqpoint{0.640323in}{0.527436in}}{\pgfqpoint{9.687500in}{3.850000in}}%
\pgfusepath{clip}%
\pgfsetbuttcap%
\pgfsetroundjoin%
\definecolor{currentfill}{rgb}{0.980392,0.164706,0.333333}%
\pgfsetfillcolor{currentfill}%
\pgfsetfillopacity{0.500000}%
\pgfsetlinewidth{1.003750pt}%
\definecolor{currentstroke}{rgb}{0.980392,0.164706,0.333333}%
\pgfsetstrokecolor{currentstroke}%
\pgfsetstrokeopacity{0.500000}%
\pgfsetdash{{3.700000pt}{1.600000pt}}{0.000000pt}%
\pgfpathmoveto{\pgfqpoint{5.214476in}{3.330842in}}%
\pgfpathcurveto{\pgfqpoint{5.220300in}{3.330842in}}{\pgfqpoint{5.225886in}{3.333155in}}{\pgfqpoint{5.230004in}{3.337274in}}%
\pgfpathcurveto{\pgfqpoint{5.234122in}{3.341392in}}{\pgfqpoint{5.236436in}{3.346978in}}{\pgfqpoint{5.236436in}{3.352802in}}%
\pgfpathcurveto{\pgfqpoint{5.236436in}{3.358626in}}{\pgfqpoint{5.234122in}{3.364212in}}{\pgfqpoint{5.230004in}{3.368330in}}%
\pgfpathcurveto{\pgfqpoint{5.225886in}{3.372448in}}{\pgfqpoint{5.220300in}{3.374762in}}{\pgfqpoint{5.214476in}{3.374762in}}%
\pgfpathcurveto{\pgfqpoint{5.208652in}{3.374762in}}{\pgfqpoint{5.203066in}{3.372448in}}{\pgfqpoint{5.198948in}{3.368330in}}%
\pgfpathcurveto{\pgfqpoint{5.194830in}{3.364212in}}{\pgfqpoint{5.192516in}{3.358626in}}{\pgfqpoint{5.192516in}{3.352802in}}%
\pgfpathcurveto{\pgfqpoint{5.192516in}{3.346978in}}{\pgfqpoint{5.194830in}{3.341392in}}{\pgfqpoint{5.198948in}{3.337274in}}%
\pgfpathcurveto{\pgfqpoint{5.203066in}{3.333155in}}{\pgfqpoint{5.208652in}{3.330842in}}{\pgfqpoint{5.214476in}{3.330842in}}%
\pgfpathlineto{\pgfqpoint{5.214476in}{3.330842in}}%
\pgfpathclose%
\pgfusepath{stroke,fill}%
\end{pgfscope}%
\begin{pgfscope}%
\pgfpathrectangle{\pgfqpoint{0.640323in}{0.527436in}}{\pgfqpoint{9.687500in}{3.850000in}}%
\pgfusepath{clip}%
\pgfsetbuttcap%
\pgfsetroundjoin%
\definecolor{currentfill}{rgb}{0.980392,0.164706,0.333333}%
\pgfsetfillcolor{currentfill}%
\pgfsetfillopacity{0.500000}%
\pgfsetlinewidth{1.003750pt}%
\definecolor{currentstroke}{rgb}{0.980392,0.164706,0.333333}%
\pgfsetstrokecolor{currentstroke}%
\pgfsetstrokeopacity{0.500000}%
\pgfsetdash{{3.700000pt}{1.600000pt}}{0.000000pt}%
\pgfpathmoveto{\pgfqpoint{5.394207in}{3.374899in}}%
\pgfpathcurveto{\pgfqpoint{5.400031in}{3.374899in}}{\pgfqpoint{5.405617in}{3.377213in}}{\pgfqpoint{5.409735in}{3.381331in}}%
\pgfpathcurveto{\pgfqpoint{5.413853in}{3.385449in}}{\pgfqpoint{5.416167in}{3.391036in}}{\pgfqpoint{5.416167in}{3.396859in}}%
\pgfpathcurveto{\pgfqpoint{5.416167in}{3.402683in}}{\pgfqpoint{5.413853in}{3.408270in}}{\pgfqpoint{5.409735in}{3.412388in}}%
\pgfpathcurveto{\pgfqpoint{5.405617in}{3.416506in}}{\pgfqpoint{5.400031in}{3.418820in}}{\pgfqpoint{5.394207in}{3.418820in}}%
\pgfpathcurveto{\pgfqpoint{5.388383in}{3.418820in}}{\pgfqpoint{5.382797in}{3.416506in}}{\pgfqpoint{5.378679in}{3.412388in}}%
\pgfpathcurveto{\pgfqpoint{5.374561in}{3.408270in}}{\pgfqpoint{5.372247in}{3.402683in}}{\pgfqpoint{5.372247in}{3.396859in}}%
\pgfpathcurveto{\pgfqpoint{5.372247in}{3.391036in}}{\pgfqpoint{5.374561in}{3.385449in}}{\pgfqpoint{5.378679in}{3.381331in}}%
\pgfpathcurveto{\pgfqpoint{5.382797in}{3.377213in}}{\pgfqpoint{5.388383in}{3.374899in}}{\pgfqpoint{5.394207in}{3.374899in}}%
\pgfpathlineto{\pgfqpoint{5.394207in}{3.374899in}}%
\pgfpathclose%
\pgfusepath{stroke,fill}%
\end{pgfscope}%
\begin{pgfscope}%
\pgfpathrectangle{\pgfqpoint{0.640323in}{0.527436in}}{\pgfqpoint{9.687500in}{3.850000in}}%
\pgfusepath{clip}%
\pgfsetbuttcap%
\pgfsetroundjoin%
\definecolor{currentfill}{rgb}{0.980392,0.164706,0.333333}%
\pgfsetfillcolor{currentfill}%
\pgfsetfillopacity{0.500000}%
\pgfsetlinewidth{1.003750pt}%
\definecolor{currentstroke}{rgb}{0.980392,0.164706,0.333333}%
\pgfsetstrokecolor{currentstroke}%
\pgfsetstrokeopacity{0.500000}%
\pgfsetdash{{3.700000pt}{1.600000pt}}{0.000000pt}%
\pgfpathmoveto{\pgfqpoint{5.573938in}{3.416479in}}%
\pgfpathcurveto{\pgfqpoint{5.579762in}{3.416479in}}{\pgfqpoint{5.585348in}{3.418793in}}{\pgfqpoint{5.589466in}{3.422911in}}%
\pgfpathcurveto{\pgfqpoint{5.593584in}{3.427029in}}{\pgfqpoint{5.595898in}{3.432616in}}{\pgfqpoint{5.595898in}{3.438439in}}%
\pgfpathcurveto{\pgfqpoint{5.595898in}{3.444263in}}{\pgfqpoint{5.593584in}{3.449850in}}{\pgfqpoint{5.589466in}{3.453968in}}%
\pgfpathcurveto{\pgfqpoint{5.585348in}{3.458086in}}{\pgfqpoint{5.579762in}{3.460400in}}{\pgfqpoint{5.573938in}{3.460400in}}%
\pgfpathcurveto{\pgfqpoint{5.568114in}{3.460400in}}{\pgfqpoint{5.562528in}{3.458086in}}{\pgfqpoint{5.558410in}{3.453968in}}%
\pgfpathcurveto{\pgfqpoint{5.554292in}{3.449850in}}{\pgfqpoint{5.551978in}{3.444263in}}{\pgfqpoint{5.551978in}{3.438439in}}%
\pgfpathcurveto{\pgfqpoint{5.551978in}{3.432616in}}{\pgfqpoint{5.554292in}{3.427029in}}{\pgfqpoint{5.558410in}{3.422911in}}%
\pgfpathcurveto{\pgfqpoint{5.562528in}{3.418793in}}{\pgfqpoint{5.568114in}{3.416479in}}{\pgfqpoint{5.573938in}{3.416479in}}%
\pgfpathlineto{\pgfqpoint{5.573938in}{3.416479in}}%
\pgfpathclose%
\pgfusepath{stroke,fill}%
\end{pgfscope}%
\begin{pgfscope}%
\pgfpathrectangle{\pgfqpoint{0.640323in}{0.527436in}}{\pgfqpoint{9.687500in}{3.850000in}}%
\pgfusepath{clip}%
\pgfsetbuttcap%
\pgfsetroundjoin%
\definecolor{currentfill}{rgb}{0.980392,0.164706,0.333333}%
\pgfsetfillcolor{currentfill}%
\pgfsetfillopacity{0.500000}%
\pgfsetlinewidth{1.003750pt}%
\definecolor{currentstroke}{rgb}{0.980392,0.164706,0.333333}%
\pgfsetstrokecolor{currentstroke}%
\pgfsetstrokeopacity{0.500000}%
\pgfsetdash{{3.700000pt}{1.600000pt}}{0.000000pt}%
\pgfpathmoveto{\pgfqpoint{5.753669in}{3.461934in}}%
\pgfpathcurveto{\pgfqpoint{5.759493in}{3.461934in}}{\pgfqpoint{5.765079in}{3.464248in}}{\pgfqpoint{5.769197in}{3.468366in}}%
\pgfpathcurveto{\pgfqpoint{5.773315in}{3.472484in}}{\pgfqpoint{5.775629in}{3.478070in}}{\pgfqpoint{5.775629in}{3.483894in}}%
\pgfpathcurveto{\pgfqpoint{5.775629in}{3.489718in}}{\pgfqpoint{5.773315in}{3.495304in}}{\pgfqpoint{5.769197in}{3.499423in}}%
\pgfpathcurveto{\pgfqpoint{5.765079in}{3.503541in}}{\pgfqpoint{5.759493in}{3.505855in}}{\pgfqpoint{5.753669in}{3.505855in}}%
\pgfpathcurveto{\pgfqpoint{5.747845in}{3.505855in}}{\pgfqpoint{5.742259in}{3.503541in}}{\pgfqpoint{5.738141in}{3.499423in}}%
\pgfpathcurveto{\pgfqpoint{5.734023in}{3.495304in}}{\pgfqpoint{5.731709in}{3.489718in}}{\pgfqpoint{5.731709in}{3.483894in}}%
\pgfpathcurveto{\pgfqpoint{5.731709in}{3.478070in}}{\pgfqpoint{5.734023in}{3.472484in}}{\pgfqpoint{5.738141in}{3.468366in}}%
\pgfpathcurveto{\pgfqpoint{5.742259in}{3.464248in}}{\pgfqpoint{5.747845in}{3.461934in}}{\pgfqpoint{5.753669in}{3.461934in}}%
\pgfpathlineto{\pgfqpoint{5.753669in}{3.461934in}}%
\pgfpathclose%
\pgfusepath{stroke,fill}%
\end{pgfscope}%
\begin{pgfscope}%
\pgfpathrectangle{\pgfqpoint{0.640323in}{0.527436in}}{\pgfqpoint{9.687500in}{3.850000in}}%
\pgfusepath{clip}%
\pgfsetbuttcap%
\pgfsetroundjoin%
\definecolor{currentfill}{rgb}{0.980392,0.164706,0.333333}%
\pgfsetfillcolor{currentfill}%
\pgfsetfillopacity{0.500000}%
\pgfsetlinewidth{1.003750pt}%
\definecolor{currentstroke}{rgb}{0.980392,0.164706,0.333333}%
\pgfsetstrokecolor{currentstroke}%
\pgfsetstrokeopacity{0.500000}%
\pgfsetdash{{3.700000pt}{1.600000pt}}{0.000000pt}%
\pgfpathmoveto{\pgfqpoint{5.933400in}{3.502781in}}%
\pgfpathcurveto{\pgfqpoint{5.939224in}{3.502781in}}{\pgfqpoint{5.944810in}{3.505095in}}{\pgfqpoint{5.948928in}{3.509213in}}%
\pgfpathcurveto{\pgfqpoint{5.953046in}{3.513331in}}{\pgfqpoint{5.955360in}{3.518918in}}{\pgfqpoint{5.955360in}{3.524742in}}%
\pgfpathcurveto{\pgfqpoint{5.955360in}{3.530565in}}{\pgfqpoint{5.953046in}{3.536152in}}{\pgfqpoint{5.948928in}{3.540270in}}%
\pgfpathcurveto{\pgfqpoint{5.944810in}{3.544388in}}{\pgfqpoint{5.939224in}{3.546702in}}{\pgfqpoint{5.933400in}{3.546702in}}%
\pgfpathcurveto{\pgfqpoint{5.927576in}{3.546702in}}{\pgfqpoint{5.921990in}{3.544388in}}{\pgfqpoint{5.917872in}{3.540270in}}%
\pgfpathcurveto{\pgfqpoint{5.913754in}{3.536152in}}{\pgfqpoint{5.911440in}{3.530565in}}{\pgfqpoint{5.911440in}{3.524742in}}%
\pgfpathcurveto{\pgfqpoint{5.911440in}{3.518918in}}{\pgfqpoint{5.913754in}{3.513331in}}{\pgfqpoint{5.917872in}{3.509213in}}%
\pgfpathcurveto{\pgfqpoint{5.921990in}{3.505095in}}{\pgfqpoint{5.927576in}{3.502781in}}{\pgfqpoint{5.933400in}{3.502781in}}%
\pgfpathlineto{\pgfqpoint{5.933400in}{3.502781in}}%
\pgfpathclose%
\pgfusepath{stroke,fill}%
\end{pgfscope}%
\begin{pgfscope}%
\pgfpathrectangle{\pgfqpoint{0.640323in}{0.527436in}}{\pgfqpoint{9.687500in}{3.850000in}}%
\pgfusepath{clip}%
\pgfsetbuttcap%
\pgfsetroundjoin%
\definecolor{currentfill}{rgb}{0.980392,0.164706,0.333333}%
\pgfsetfillcolor{currentfill}%
\pgfsetfillopacity{0.500000}%
\pgfsetlinewidth{1.003750pt}%
\definecolor{currentstroke}{rgb}{0.980392,0.164706,0.333333}%
\pgfsetstrokecolor{currentstroke}%
\pgfsetstrokeopacity{0.500000}%
\pgfsetdash{{3.700000pt}{1.600000pt}}{0.000000pt}%
\pgfpathmoveto{\pgfqpoint{6.113131in}{3.538251in}}%
\pgfpathcurveto{\pgfqpoint{6.118955in}{3.538251in}}{\pgfqpoint{6.124541in}{3.540565in}}{\pgfqpoint{6.128659in}{3.544683in}}%
\pgfpathcurveto{\pgfqpoint{6.132777in}{3.548801in}}{\pgfqpoint{6.135091in}{3.554387in}}{\pgfqpoint{6.135091in}{3.560211in}}%
\pgfpathcurveto{\pgfqpoint{6.135091in}{3.566035in}}{\pgfqpoint{6.132777in}{3.571621in}}{\pgfqpoint{6.128659in}{3.575739in}}%
\pgfpathcurveto{\pgfqpoint{6.124541in}{3.579858in}}{\pgfqpoint{6.118955in}{3.582171in}}{\pgfqpoint{6.113131in}{3.582171in}}%
\pgfpathcurveto{\pgfqpoint{6.107307in}{3.582171in}}{\pgfqpoint{6.101721in}{3.579858in}}{\pgfqpoint{6.097603in}{3.575739in}}%
\pgfpathcurveto{\pgfqpoint{6.093485in}{3.571621in}}{\pgfqpoint{6.091171in}{3.566035in}}{\pgfqpoint{6.091171in}{3.560211in}}%
\pgfpathcurveto{\pgfqpoint{6.091171in}{3.554387in}}{\pgfqpoint{6.093485in}{3.548801in}}{\pgfqpoint{6.097603in}{3.544683in}}%
\pgfpathcurveto{\pgfqpoint{6.101721in}{3.540565in}}{\pgfqpoint{6.107307in}{3.538251in}}{\pgfqpoint{6.113131in}{3.538251in}}%
\pgfpathlineto{\pgfqpoint{6.113131in}{3.538251in}}%
\pgfpathclose%
\pgfusepath{stroke,fill}%
\end{pgfscope}%
\begin{pgfscope}%
\pgfpathrectangle{\pgfqpoint{0.640323in}{0.527436in}}{\pgfqpoint{9.687500in}{3.850000in}}%
\pgfusepath{clip}%
\pgfsetbuttcap%
\pgfsetroundjoin%
\definecolor{currentfill}{rgb}{0.980392,0.164706,0.333333}%
\pgfsetfillcolor{currentfill}%
\pgfsetfillopacity{0.500000}%
\pgfsetlinewidth{1.003750pt}%
\definecolor{currentstroke}{rgb}{0.980392,0.164706,0.333333}%
\pgfsetstrokecolor{currentstroke}%
\pgfsetstrokeopacity{0.500000}%
\pgfsetdash{{3.700000pt}{1.600000pt}}{0.000000pt}%
\pgfpathmoveto{\pgfqpoint{6.292862in}{3.581210in}}%
\pgfpathcurveto{\pgfqpoint{6.298686in}{3.581210in}}{\pgfqpoint{6.304272in}{3.583523in}}{\pgfqpoint{6.308390in}{3.587642in}}%
\pgfpathcurveto{\pgfqpoint{6.312508in}{3.591760in}}{\pgfqpoint{6.314822in}{3.597346in}}{\pgfqpoint{6.314822in}{3.603170in}}%
\pgfpathcurveto{\pgfqpoint{6.314822in}{3.608994in}}{\pgfqpoint{6.312508in}{3.614580in}}{\pgfqpoint{6.308390in}{3.618698in}}%
\pgfpathcurveto{\pgfqpoint{6.304272in}{3.622816in}}{\pgfqpoint{6.298686in}{3.625130in}}{\pgfqpoint{6.292862in}{3.625130in}}%
\pgfpathcurveto{\pgfqpoint{6.287038in}{3.625130in}}{\pgfqpoint{6.281452in}{3.622816in}}{\pgfqpoint{6.277334in}{3.618698in}}%
\pgfpathcurveto{\pgfqpoint{6.273216in}{3.614580in}}{\pgfqpoint{6.270902in}{3.608994in}}{\pgfqpoint{6.270902in}{3.603170in}}%
\pgfpathcurveto{\pgfqpoint{6.270902in}{3.597346in}}{\pgfqpoint{6.273216in}{3.591760in}}{\pgfqpoint{6.277334in}{3.587642in}}%
\pgfpathcurveto{\pgfqpoint{6.281452in}{3.583523in}}{\pgfqpoint{6.287038in}{3.581210in}}{\pgfqpoint{6.292862in}{3.581210in}}%
\pgfpathlineto{\pgfqpoint{6.292862in}{3.581210in}}%
\pgfpathclose%
\pgfusepath{stroke,fill}%
\end{pgfscope}%
\begin{pgfscope}%
\pgfpathrectangle{\pgfqpoint{0.640323in}{0.527436in}}{\pgfqpoint{9.687500in}{3.850000in}}%
\pgfusepath{clip}%
\pgfsetbuttcap%
\pgfsetroundjoin%
\definecolor{currentfill}{rgb}{0.980392,0.164706,0.333333}%
\pgfsetfillcolor{currentfill}%
\pgfsetfillopacity{0.500000}%
\pgfsetlinewidth{1.003750pt}%
\definecolor{currentstroke}{rgb}{0.980392,0.164706,0.333333}%
\pgfsetstrokecolor{currentstroke}%
\pgfsetstrokeopacity{0.500000}%
\pgfsetdash{{3.700000pt}{1.600000pt}}{0.000000pt}%
\pgfpathmoveto{\pgfqpoint{6.472593in}{3.617486in}}%
\pgfpathcurveto{\pgfqpoint{6.478417in}{3.617486in}}{\pgfqpoint{6.484003in}{3.619800in}}{\pgfqpoint{6.488121in}{3.623918in}}%
\pgfpathcurveto{\pgfqpoint{6.492239in}{3.628037in}}{\pgfqpoint{6.494553in}{3.633623in}}{\pgfqpoint{6.494553in}{3.639447in}}%
\pgfpathcurveto{\pgfqpoint{6.494553in}{3.645271in}}{\pgfqpoint{6.492239in}{3.650857in}}{\pgfqpoint{6.488121in}{3.654975in}}%
\pgfpathcurveto{\pgfqpoint{6.484003in}{3.659093in}}{\pgfqpoint{6.478417in}{3.661407in}}{\pgfqpoint{6.472593in}{3.661407in}}%
\pgfpathcurveto{\pgfqpoint{6.466769in}{3.661407in}}{\pgfqpoint{6.461183in}{3.659093in}}{\pgfqpoint{6.457065in}{3.654975in}}%
\pgfpathcurveto{\pgfqpoint{6.452947in}{3.650857in}}{\pgfqpoint{6.450633in}{3.645271in}}{\pgfqpoint{6.450633in}{3.639447in}}%
\pgfpathcurveto{\pgfqpoint{6.450633in}{3.633623in}}{\pgfqpoint{6.452947in}{3.628037in}}{\pgfqpoint{6.457065in}{3.623918in}}%
\pgfpathcurveto{\pgfqpoint{6.461183in}{3.619800in}}{\pgfqpoint{6.466769in}{3.617486in}}{\pgfqpoint{6.472593in}{3.617486in}}%
\pgfpathlineto{\pgfqpoint{6.472593in}{3.617486in}}%
\pgfpathclose%
\pgfusepath{stroke,fill}%
\end{pgfscope}%
\begin{pgfscope}%
\pgfpathrectangle{\pgfqpoint{0.640323in}{0.527436in}}{\pgfqpoint{9.687500in}{3.850000in}}%
\pgfusepath{clip}%
\pgfsetbuttcap%
\pgfsetroundjoin%
\definecolor{currentfill}{rgb}{0.980392,0.164706,0.333333}%
\pgfsetfillcolor{currentfill}%
\pgfsetfillopacity{0.500000}%
\pgfsetlinewidth{1.003750pt}%
\definecolor{currentstroke}{rgb}{0.980392,0.164706,0.333333}%
\pgfsetstrokecolor{currentstroke}%
\pgfsetstrokeopacity{0.500000}%
\pgfsetdash{{3.700000pt}{1.600000pt}}{0.000000pt}%
\pgfpathmoveto{\pgfqpoint{6.652324in}{3.634544in}}%
\pgfpathcurveto{\pgfqpoint{6.658148in}{3.634544in}}{\pgfqpoint{6.663734in}{3.636858in}}{\pgfqpoint{6.667852in}{3.640976in}}%
\pgfpathcurveto{\pgfqpoint{6.671970in}{3.645095in}}{\pgfqpoint{6.674284in}{3.650681in}}{\pgfqpoint{6.674284in}{3.656505in}}%
\pgfpathcurveto{\pgfqpoint{6.674284in}{3.662329in}}{\pgfqpoint{6.671970in}{3.667915in}}{\pgfqpoint{6.667852in}{3.672033in}}%
\pgfpathcurveto{\pgfqpoint{6.663734in}{3.676151in}}{\pgfqpoint{6.658148in}{3.678465in}}{\pgfqpoint{6.652324in}{3.678465in}}%
\pgfpathcurveto{\pgfqpoint{6.646500in}{3.678465in}}{\pgfqpoint{6.640914in}{3.676151in}}{\pgfqpoint{6.636796in}{3.672033in}}%
\pgfpathcurveto{\pgfqpoint{6.632678in}{3.667915in}}{\pgfqpoint{6.630364in}{3.662329in}}{\pgfqpoint{6.630364in}{3.656505in}}%
\pgfpathcurveto{\pgfqpoint{6.630364in}{3.650681in}}{\pgfqpoint{6.632678in}{3.645095in}}{\pgfqpoint{6.636796in}{3.640976in}}%
\pgfpathcurveto{\pgfqpoint{6.640914in}{3.636858in}}{\pgfqpoint{6.646500in}{3.634544in}}{\pgfqpoint{6.652324in}{3.634544in}}%
\pgfpathlineto{\pgfqpoint{6.652324in}{3.634544in}}%
\pgfpathclose%
\pgfusepath{stroke,fill}%
\end{pgfscope}%
\begin{pgfscope}%
\pgfpathrectangle{\pgfqpoint{0.640323in}{0.527436in}}{\pgfqpoint{9.687500in}{3.850000in}}%
\pgfusepath{clip}%
\pgfsetbuttcap%
\pgfsetroundjoin%
\definecolor{currentfill}{rgb}{0.980392,0.164706,0.333333}%
\pgfsetfillcolor{currentfill}%
\pgfsetfillopacity{0.500000}%
\pgfsetlinewidth{1.003750pt}%
\definecolor{currentstroke}{rgb}{0.980392,0.164706,0.333333}%
\pgfsetstrokecolor{currentstroke}%
\pgfsetstrokeopacity{0.500000}%
\pgfsetdash{{3.700000pt}{1.600000pt}}{0.000000pt}%
\pgfpathmoveto{\pgfqpoint{6.832055in}{3.665661in}}%
\pgfpathcurveto{\pgfqpoint{6.837879in}{3.665661in}}{\pgfqpoint{6.843465in}{3.667975in}}{\pgfqpoint{6.847583in}{3.672093in}}%
\pgfpathcurveto{\pgfqpoint{6.851701in}{3.676211in}}{\pgfqpoint{6.854015in}{3.681797in}}{\pgfqpoint{6.854015in}{3.687621in}}%
\pgfpathcurveto{\pgfqpoint{6.854015in}{3.693445in}}{\pgfqpoint{6.851701in}{3.699032in}}{\pgfqpoint{6.847583in}{3.703150in}}%
\pgfpathcurveto{\pgfqpoint{6.843465in}{3.707268in}}{\pgfqpoint{6.837879in}{3.709582in}}{\pgfqpoint{6.832055in}{3.709582in}}%
\pgfpathcurveto{\pgfqpoint{6.826231in}{3.709582in}}{\pgfqpoint{6.820645in}{3.707268in}}{\pgfqpoint{6.816527in}{3.703150in}}%
\pgfpathcurveto{\pgfqpoint{6.812408in}{3.699032in}}{\pgfqpoint{6.810095in}{3.693445in}}{\pgfqpoint{6.810095in}{3.687621in}}%
\pgfpathcurveto{\pgfqpoint{6.810095in}{3.681797in}}{\pgfqpoint{6.812408in}{3.676211in}}{\pgfqpoint{6.816527in}{3.672093in}}%
\pgfpathcurveto{\pgfqpoint{6.820645in}{3.667975in}}{\pgfqpoint{6.826231in}{3.665661in}}{\pgfqpoint{6.832055in}{3.665661in}}%
\pgfpathlineto{\pgfqpoint{6.832055in}{3.665661in}}%
\pgfpathclose%
\pgfusepath{stroke,fill}%
\end{pgfscope}%
\begin{pgfscope}%
\pgfpathrectangle{\pgfqpoint{0.640323in}{0.527436in}}{\pgfqpoint{9.687500in}{3.850000in}}%
\pgfusepath{clip}%
\pgfsetbuttcap%
\pgfsetroundjoin%
\definecolor{currentfill}{rgb}{0.980392,0.164706,0.333333}%
\pgfsetfillcolor{currentfill}%
\pgfsetfillopacity{0.500000}%
\pgfsetlinewidth{1.003750pt}%
\definecolor{currentstroke}{rgb}{0.980392,0.164706,0.333333}%
\pgfsetstrokecolor{currentstroke}%
\pgfsetstrokeopacity{0.500000}%
\pgfsetdash{{3.700000pt}{1.600000pt}}{0.000000pt}%
\pgfpathmoveto{\pgfqpoint{7.011786in}{3.694393in}}%
\pgfpathcurveto{\pgfqpoint{7.017610in}{3.694393in}}{\pgfqpoint{7.023196in}{3.696707in}}{\pgfqpoint{7.027314in}{3.700825in}}%
\pgfpathcurveto{\pgfqpoint{7.031432in}{3.704943in}}{\pgfqpoint{7.033746in}{3.710530in}}{\pgfqpoint{7.033746in}{3.716354in}}%
\pgfpathcurveto{\pgfqpoint{7.033746in}{3.722178in}}{\pgfqpoint{7.031432in}{3.727764in}}{\pgfqpoint{7.027314in}{3.731882in}}%
\pgfpathcurveto{\pgfqpoint{7.023196in}{3.736000in}}{\pgfqpoint{7.017610in}{3.738314in}}{\pgfqpoint{7.011786in}{3.738314in}}%
\pgfpathcurveto{\pgfqpoint{7.005962in}{3.738314in}}{\pgfqpoint{7.000376in}{3.736000in}}{\pgfqpoint{6.996258in}{3.731882in}}%
\pgfpathcurveto{\pgfqpoint{6.992139in}{3.727764in}}{\pgfqpoint{6.989826in}{3.722178in}}{\pgfqpoint{6.989826in}{3.716354in}}%
\pgfpathcurveto{\pgfqpoint{6.989826in}{3.710530in}}{\pgfqpoint{6.992139in}{3.704943in}}{\pgfqpoint{6.996258in}{3.700825in}}%
\pgfpathcurveto{\pgfqpoint{7.000376in}{3.696707in}}{\pgfqpoint{7.005962in}{3.694393in}}{\pgfqpoint{7.011786in}{3.694393in}}%
\pgfpathlineto{\pgfqpoint{7.011786in}{3.694393in}}%
\pgfpathclose%
\pgfusepath{stroke,fill}%
\end{pgfscope}%
\begin{pgfscope}%
\pgfpathrectangle{\pgfqpoint{0.640323in}{0.527436in}}{\pgfqpoint{9.687500in}{3.850000in}}%
\pgfusepath{clip}%
\pgfsetbuttcap%
\pgfsetroundjoin%
\definecolor{currentfill}{rgb}{0.980392,0.164706,0.333333}%
\pgfsetfillcolor{currentfill}%
\pgfsetfillopacity{0.500000}%
\pgfsetlinewidth{1.003750pt}%
\definecolor{currentstroke}{rgb}{0.980392,0.164706,0.333333}%
\pgfsetstrokecolor{currentstroke}%
\pgfsetstrokeopacity{0.500000}%
\pgfsetdash{{3.700000pt}{1.600000pt}}{0.000000pt}%
\pgfpathmoveto{\pgfqpoint{7.191517in}{3.737855in}}%
\pgfpathcurveto{\pgfqpoint{7.197341in}{3.737855in}}{\pgfqpoint{7.202927in}{3.740169in}}{\pgfqpoint{7.207045in}{3.744287in}}%
\pgfpathcurveto{\pgfqpoint{7.211163in}{3.748405in}}{\pgfqpoint{7.213477in}{3.753991in}}{\pgfqpoint{7.213477in}{3.759815in}}%
\pgfpathcurveto{\pgfqpoint{7.213477in}{3.765639in}}{\pgfqpoint{7.211163in}{3.771225in}}{\pgfqpoint{7.207045in}{3.775343in}}%
\pgfpathcurveto{\pgfqpoint{7.202927in}{3.779461in}}{\pgfqpoint{7.197341in}{3.781775in}}{\pgfqpoint{7.191517in}{3.781775in}}%
\pgfpathcurveto{\pgfqpoint{7.185693in}{3.781775in}}{\pgfqpoint{7.180107in}{3.779461in}}{\pgfqpoint{7.175989in}{3.775343in}}%
\pgfpathcurveto{\pgfqpoint{7.171870in}{3.771225in}}{\pgfqpoint{7.169557in}{3.765639in}}{\pgfqpoint{7.169557in}{3.759815in}}%
\pgfpathcurveto{\pgfqpoint{7.169557in}{3.753991in}}{\pgfqpoint{7.171870in}{3.748405in}}{\pgfqpoint{7.175989in}{3.744287in}}%
\pgfpathcurveto{\pgfqpoint{7.180107in}{3.740169in}}{\pgfqpoint{7.185693in}{3.737855in}}{\pgfqpoint{7.191517in}{3.737855in}}%
\pgfpathlineto{\pgfqpoint{7.191517in}{3.737855in}}%
\pgfpathclose%
\pgfusepath{stroke,fill}%
\end{pgfscope}%
\begin{pgfscope}%
\pgfpathrectangle{\pgfqpoint{0.640323in}{0.527436in}}{\pgfqpoint{9.687500in}{3.850000in}}%
\pgfusepath{clip}%
\pgfsetbuttcap%
\pgfsetroundjoin%
\definecolor{currentfill}{rgb}{0.980392,0.164706,0.333333}%
\pgfsetfillcolor{currentfill}%
\pgfsetfillopacity{0.500000}%
\pgfsetlinewidth{1.003750pt}%
\definecolor{currentstroke}{rgb}{0.980392,0.164706,0.333333}%
\pgfsetstrokecolor{currentstroke}%
\pgfsetstrokeopacity{0.500000}%
\pgfsetdash{{3.700000pt}{1.600000pt}}{0.000000pt}%
\pgfpathmoveto{\pgfqpoint{7.371248in}{3.762495in}}%
\pgfpathcurveto{\pgfqpoint{7.377072in}{3.762495in}}{\pgfqpoint{7.382658in}{3.764809in}}{\pgfqpoint{7.386776in}{3.768927in}}%
\pgfpathcurveto{\pgfqpoint{7.390894in}{3.773045in}}{\pgfqpoint{7.393208in}{3.778631in}}{\pgfqpoint{7.393208in}{3.784455in}}%
\pgfpathcurveto{\pgfqpoint{7.393208in}{3.790279in}}{\pgfqpoint{7.390894in}{3.795865in}}{\pgfqpoint{7.386776in}{3.799983in}}%
\pgfpathcurveto{\pgfqpoint{7.382658in}{3.804101in}}{\pgfqpoint{7.377072in}{3.806415in}}{\pgfqpoint{7.371248in}{3.806415in}}%
\pgfpathcurveto{\pgfqpoint{7.365424in}{3.806415in}}{\pgfqpoint{7.359838in}{3.804101in}}{\pgfqpoint{7.355720in}{3.799983in}}%
\pgfpathcurveto{\pgfqpoint{7.351601in}{3.795865in}}{\pgfqpoint{7.349288in}{3.790279in}}{\pgfqpoint{7.349288in}{3.784455in}}%
\pgfpathcurveto{\pgfqpoint{7.349288in}{3.778631in}}{\pgfqpoint{7.351601in}{3.773045in}}{\pgfqpoint{7.355720in}{3.768927in}}%
\pgfpathcurveto{\pgfqpoint{7.359838in}{3.764809in}}{\pgfqpoint{7.365424in}{3.762495in}}{\pgfqpoint{7.371248in}{3.762495in}}%
\pgfpathlineto{\pgfqpoint{7.371248in}{3.762495in}}%
\pgfpathclose%
\pgfusepath{stroke,fill}%
\end{pgfscope}%
\begin{pgfscope}%
\pgfpathrectangle{\pgfqpoint{0.640323in}{0.527436in}}{\pgfqpoint{9.687500in}{3.850000in}}%
\pgfusepath{clip}%
\pgfsetbuttcap%
\pgfsetroundjoin%
\definecolor{currentfill}{rgb}{0.980392,0.164706,0.333333}%
\pgfsetfillcolor{currentfill}%
\pgfsetfillopacity{0.500000}%
\pgfsetlinewidth{1.003750pt}%
\definecolor{currentstroke}{rgb}{0.980392,0.164706,0.333333}%
\pgfsetstrokecolor{currentstroke}%
\pgfsetstrokeopacity{0.500000}%
\pgfsetdash{{3.700000pt}{1.600000pt}}{0.000000pt}%
\pgfpathmoveto{\pgfqpoint{7.550979in}{3.776578in}}%
\pgfpathcurveto{\pgfqpoint{7.556803in}{3.776578in}}{\pgfqpoint{7.562389in}{3.778892in}}{\pgfqpoint{7.566507in}{3.783010in}}%
\pgfpathcurveto{\pgfqpoint{7.570625in}{3.787129in}}{\pgfqpoint{7.572939in}{3.792715in}}{\pgfqpoint{7.572939in}{3.798539in}}%
\pgfpathcurveto{\pgfqpoint{7.572939in}{3.804363in}}{\pgfqpoint{7.570625in}{3.809949in}}{\pgfqpoint{7.566507in}{3.814067in}}%
\pgfpathcurveto{\pgfqpoint{7.562389in}{3.818185in}}{\pgfqpoint{7.556803in}{3.820499in}}{\pgfqpoint{7.550979in}{3.820499in}}%
\pgfpathcurveto{\pgfqpoint{7.545155in}{3.820499in}}{\pgfqpoint{7.539569in}{3.818185in}}{\pgfqpoint{7.535451in}{3.814067in}}%
\pgfpathcurveto{\pgfqpoint{7.531332in}{3.809949in}}{\pgfqpoint{7.529019in}{3.804363in}}{\pgfqpoint{7.529019in}{3.798539in}}%
\pgfpathcurveto{\pgfqpoint{7.529019in}{3.792715in}}{\pgfqpoint{7.531332in}{3.787129in}}{\pgfqpoint{7.535451in}{3.783010in}}%
\pgfpathcurveto{\pgfqpoint{7.539569in}{3.778892in}}{\pgfqpoint{7.545155in}{3.776578in}}{\pgfqpoint{7.550979in}{3.776578in}}%
\pgfpathlineto{\pgfqpoint{7.550979in}{3.776578in}}%
\pgfpathclose%
\pgfusepath{stroke,fill}%
\end{pgfscope}%
\begin{pgfscope}%
\pgfpathrectangle{\pgfqpoint{0.640323in}{0.527436in}}{\pgfqpoint{9.687500in}{3.850000in}}%
\pgfusepath{clip}%
\pgfsetbuttcap%
\pgfsetroundjoin%
\definecolor{currentfill}{rgb}{0.980392,0.164706,0.333333}%
\pgfsetfillcolor{currentfill}%
\pgfsetfillopacity{0.500000}%
\pgfsetlinewidth{1.003750pt}%
\definecolor{currentstroke}{rgb}{0.980392,0.164706,0.333333}%
\pgfsetstrokecolor{currentstroke}%
\pgfsetstrokeopacity{0.500000}%
\pgfsetdash{{3.700000pt}{1.600000pt}}{0.000000pt}%
\pgfpathmoveto{\pgfqpoint{7.730710in}{3.800380in}}%
\pgfpathcurveto{\pgfqpoint{7.736534in}{3.800380in}}{\pgfqpoint{7.742120in}{3.802694in}}{\pgfqpoint{7.746238in}{3.806812in}}%
\pgfpathcurveto{\pgfqpoint{7.750356in}{3.810930in}}{\pgfqpoint{7.752670in}{3.816516in}}{\pgfqpoint{7.752670in}{3.822340in}}%
\pgfpathcurveto{\pgfqpoint{7.752670in}{3.828164in}}{\pgfqpoint{7.750356in}{3.833750in}}{\pgfqpoint{7.746238in}{3.837869in}}%
\pgfpathcurveto{\pgfqpoint{7.742120in}{3.841987in}}{\pgfqpoint{7.736534in}{3.844301in}}{\pgfqpoint{7.730710in}{3.844301in}}%
\pgfpathcurveto{\pgfqpoint{7.724886in}{3.844301in}}{\pgfqpoint{7.719300in}{3.841987in}}{\pgfqpoint{7.715182in}{3.837869in}}%
\pgfpathcurveto{\pgfqpoint{7.711063in}{3.833750in}}{\pgfqpoint{7.708750in}{3.828164in}}{\pgfqpoint{7.708750in}{3.822340in}}%
\pgfpathcurveto{\pgfqpoint{7.708750in}{3.816516in}}{\pgfqpoint{7.711063in}{3.810930in}}{\pgfqpoint{7.715182in}{3.806812in}}%
\pgfpathcurveto{\pgfqpoint{7.719300in}{3.802694in}}{\pgfqpoint{7.724886in}{3.800380in}}{\pgfqpoint{7.730710in}{3.800380in}}%
\pgfpathlineto{\pgfqpoint{7.730710in}{3.800380in}}%
\pgfpathclose%
\pgfusepath{stroke,fill}%
\end{pgfscope}%
\begin{pgfscope}%
\pgfpathrectangle{\pgfqpoint{0.640323in}{0.527436in}}{\pgfqpoint{9.687500in}{3.850000in}}%
\pgfusepath{clip}%
\pgfsetbuttcap%
\pgfsetroundjoin%
\definecolor{currentfill}{rgb}{0.980392,0.164706,0.333333}%
\pgfsetfillcolor{currentfill}%
\pgfsetfillopacity{0.500000}%
\pgfsetlinewidth{1.003750pt}%
\definecolor{currentstroke}{rgb}{0.980392,0.164706,0.333333}%
\pgfsetstrokecolor{currentstroke}%
\pgfsetstrokeopacity{0.500000}%
\pgfsetdash{{3.700000pt}{1.600000pt}}{0.000000pt}%
\pgfpathmoveto{\pgfqpoint{7.910441in}{3.827131in}}%
\pgfpathcurveto{\pgfqpoint{7.916265in}{3.827131in}}{\pgfqpoint{7.921851in}{3.829445in}}{\pgfqpoint{7.925969in}{3.833563in}}%
\pgfpathcurveto{\pgfqpoint{7.930087in}{3.837682in}}{\pgfqpoint{7.932401in}{3.843268in}}{\pgfqpoint{7.932401in}{3.849092in}}%
\pgfpathcurveto{\pgfqpoint{7.932401in}{3.854916in}}{\pgfqpoint{7.930087in}{3.860502in}}{\pgfqpoint{7.925969in}{3.864620in}}%
\pgfpathcurveto{\pgfqpoint{7.921851in}{3.868738in}}{\pgfqpoint{7.916265in}{3.871052in}}{\pgfqpoint{7.910441in}{3.871052in}}%
\pgfpathcurveto{\pgfqpoint{7.904617in}{3.871052in}}{\pgfqpoint{7.899031in}{3.868738in}}{\pgfqpoint{7.894913in}{3.864620in}}%
\pgfpathcurveto{\pgfqpoint{7.890794in}{3.860502in}}{\pgfqpoint{7.888481in}{3.854916in}}{\pgfqpoint{7.888481in}{3.849092in}}%
\pgfpathcurveto{\pgfqpoint{7.888481in}{3.843268in}}{\pgfqpoint{7.890794in}{3.837682in}}{\pgfqpoint{7.894913in}{3.833563in}}%
\pgfpathcurveto{\pgfqpoint{7.899031in}{3.829445in}}{\pgfqpoint{7.904617in}{3.827131in}}{\pgfqpoint{7.910441in}{3.827131in}}%
\pgfpathlineto{\pgfqpoint{7.910441in}{3.827131in}}%
\pgfpathclose%
\pgfusepath{stroke,fill}%
\end{pgfscope}%
\begin{pgfscope}%
\pgfpathrectangle{\pgfqpoint{0.640323in}{0.527436in}}{\pgfqpoint{9.687500in}{3.850000in}}%
\pgfusepath{clip}%
\pgfsetbuttcap%
\pgfsetroundjoin%
\definecolor{currentfill}{rgb}{0.980392,0.164706,0.333333}%
\pgfsetfillcolor{currentfill}%
\pgfsetfillopacity{0.500000}%
\pgfsetlinewidth{1.003750pt}%
\definecolor{currentstroke}{rgb}{0.980392,0.164706,0.333333}%
\pgfsetstrokecolor{currentstroke}%
\pgfsetstrokeopacity{0.500000}%
\pgfsetdash{{3.700000pt}{1.600000pt}}{0.000000pt}%
\pgfpathmoveto{\pgfqpoint{8.090172in}{3.845506in}}%
\pgfpathcurveto{\pgfqpoint{8.095996in}{3.845506in}}{\pgfqpoint{8.101582in}{3.847820in}}{\pgfqpoint{8.105700in}{3.851938in}}%
\pgfpathcurveto{\pgfqpoint{8.109818in}{3.856056in}}{\pgfqpoint{8.112132in}{3.861642in}}{\pgfqpoint{8.112132in}{3.867466in}}%
\pgfpathcurveto{\pgfqpoint{8.112132in}{3.873290in}}{\pgfqpoint{8.109818in}{3.878876in}}{\pgfqpoint{8.105700in}{3.882994in}}%
\pgfpathcurveto{\pgfqpoint{8.101582in}{3.887112in}}{\pgfqpoint{8.095996in}{3.889426in}}{\pgfqpoint{8.090172in}{3.889426in}}%
\pgfpathcurveto{\pgfqpoint{8.084348in}{3.889426in}}{\pgfqpoint{8.078762in}{3.887112in}}{\pgfqpoint{8.074644in}{3.882994in}}%
\pgfpathcurveto{\pgfqpoint{8.070525in}{3.878876in}}{\pgfqpoint{8.068211in}{3.873290in}}{\pgfqpoint{8.068211in}{3.867466in}}%
\pgfpathcurveto{\pgfqpoint{8.068211in}{3.861642in}}{\pgfqpoint{8.070525in}{3.856056in}}{\pgfqpoint{8.074644in}{3.851938in}}%
\pgfpathcurveto{\pgfqpoint{8.078762in}{3.847820in}}{\pgfqpoint{8.084348in}{3.845506in}}{\pgfqpoint{8.090172in}{3.845506in}}%
\pgfpathlineto{\pgfqpoint{8.090172in}{3.845506in}}%
\pgfpathclose%
\pgfusepath{stroke,fill}%
\end{pgfscope}%
\begin{pgfscope}%
\pgfpathrectangle{\pgfqpoint{0.640323in}{0.527436in}}{\pgfqpoint{9.687500in}{3.850000in}}%
\pgfusepath{clip}%
\pgfsetbuttcap%
\pgfsetroundjoin%
\definecolor{currentfill}{rgb}{0.980392,0.164706,0.333333}%
\pgfsetfillcolor{currentfill}%
\pgfsetfillopacity{0.500000}%
\pgfsetlinewidth{1.003750pt}%
\definecolor{currentstroke}{rgb}{0.980392,0.164706,0.333333}%
\pgfsetstrokecolor{currentstroke}%
\pgfsetstrokeopacity{0.500000}%
\pgfsetdash{{3.700000pt}{1.600000pt}}{0.000000pt}%
\pgfpathmoveto{\pgfqpoint{8.269903in}{3.866377in}}%
\pgfpathcurveto{\pgfqpoint{8.275727in}{3.866377in}}{\pgfqpoint{8.281313in}{3.868690in}}{\pgfqpoint{8.285431in}{3.872809in}}%
\pgfpathcurveto{\pgfqpoint{8.289549in}{3.876927in}}{\pgfqpoint{8.291863in}{3.882513in}}{\pgfqpoint{8.291863in}{3.888337in}}%
\pgfpathcurveto{\pgfqpoint{8.291863in}{3.894161in}}{\pgfqpoint{8.289549in}{3.899747in}}{\pgfqpoint{8.285431in}{3.903865in}}%
\pgfpathcurveto{\pgfqpoint{8.281313in}{3.907983in}}{\pgfqpoint{8.275727in}{3.910297in}}{\pgfqpoint{8.269903in}{3.910297in}}%
\pgfpathcurveto{\pgfqpoint{8.264079in}{3.910297in}}{\pgfqpoint{8.258493in}{3.907983in}}{\pgfqpoint{8.254374in}{3.903865in}}%
\pgfpathcurveto{\pgfqpoint{8.250256in}{3.899747in}}{\pgfqpoint{8.247942in}{3.894161in}}{\pgfqpoint{8.247942in}{3.888337in}}%
\pgfpathcurveto{\pgfqpoint{8.247942in}{3.882513in}}{\pgfqpoint{8.250256in}{3.876927in}}{\pgfqpoint{8.254374in}{3.872809in}}%
\pgfpathcurveto{\pgfqpoint{8.258493in}{3.868690in}}{\pgfqpoint{8.264079in}{3.866377in}}{\pgfqpoint{8.269903in}{3.866377in}}%
\pgfpathlineto{\pgfqpoint{8.269903in}{3.866377in}}%
\pgfpathclose%
\pgfusepath{stroke,fill}%
\end{pgfscope}%
\begin{pgfscope}%
\pgfpathrectangle{\pgfqpoint{0.640323in}{0.527436in}}{\pgfqpoint{9.687500in}{3.850000in}}%
\pgfusepath{clip}%
\pgfsetbuttcap%
\pgfsetroundjoin%
\definecolor{currentfill}{rgb}{0.980392,0.164706,0.333333}%
\pgfsetfillcolor{currentfill}%
\pgfsetfillopacity{0.500000}%
\pgfsetlinewidth{1.003750pt}%
\definecolor{currentstroke}{rgb}{0.980392,0.164706,0.333333}%
\pgfsetstrokecolor{currentstroke}%
\pgfsetstrokeopacity{0.500000}%
\pgfsetdash{{3.700000pt}{1.600000pt}}{0.000000pt}%
\pgfpathmoveto{\pgfqpoint{8.449634in}{3.884776in}}%
\pgfpathcurveto{\pgfqpoint{8.455458in}{3.884776in}}{\pgfqpoint{8.461044in}{3.887090in}}{\pgfqpoint{8.465162in}{3.891208in}}%
\pgfpathcurveto{\pgfqpoint{8.469280in}{3.895326in}}{\pgfqpoint{8.471594in}{3.900912in}}{\pgfqpoint{8.471594in}{3.906736in}}%
\pgfpathcurveto{\pgfqpoint{8.471594in}{3.912560in}}{\pgfqpoint{8.469280in}{3.918146in}}{\pgfqpoint{8.465162in}{3.922264in}}%
\pgfpathcurveto{\pgfqpoint{8.461044in}{3.926382in}}{\pgfqpoint{8.455458in}{3.928696in}}{\pgfqpoint{8.449634in}{3.928696in}}%
\pgfpathcurveto{\pgfqpoint{8.443810in}{3.928696in}}{\pgfqpoint{8.438224in}{3.926382in}}{\pgfqpoint{8.434105in}{3.922264in}}%
\pgfpathcurveto{\pgfqpoint{8.429987in}{3.918146in}}{\pgfqpoint{8.427673in}{3.912560in}}{\pgfqpoint{8.427673in}{3.906736in}}%
\pgfpathcurveto{\pgfqpoint{8.427673in}{3.900912in}}{\pgfqpoint{8.429987in}{3.895326in}}{\pgfqpoint{8.434105in}{3.891208in}}%
\pgfpathcurveto{\pgfqpoint{8.438224in}{3.887090in}}{\pgfqpoint{8.443810in}{3.884776in}}{\pgfqpoint{8.449634in}{3.884776in}}%
\pgfpathlineto{\pgfqpoint{8.449634in}{3.884776in}}%
\pgfpathclose%
\pgfusepath{stroke,fill}%
\end{pgfscope}%
\begin{pgfscope}%
\pgfpathrectangle{\pgfqpoint{0.640323in}{0.527436in}}{\pgfqpoint{9.687500in}{3.850000in}}%
\pgfusepath{clip}%
\pgfsetbuttcap%
\pgfsetroundjoin%
\definecolor{currentfill}{rgb}{0.980392,0.164706,0.333333}%
\pgfsetfillcolor{currentfill}%
\pgfsetfillopacity{0.500000}%
\pgfsetlinewidth{1.003750pt}%
\definecolor{currentstroke}{rgb}{0.980392,0.164706,0.333333}%
\pgfsetstrokecolor{currentstroke}%
\pgfsetstrokeopacity{0.500000}%
\pgfsetdash{{3.700000pt}{1.600000pt}}{0.000000pt}%
\pgfpathmoveto{\pgfqpoint{8.629365in}{3.903784in}}%
\pgfpathcurveto{\pgfqpoint{8.635189in}{3.903784in}}{\pgfqpoint{8.640775in}{3.906098in}}{\pgfqpoint{8.644893in}{3.910216in}}%
\pgfpathcurveto{\pgfqpoint{8.649011in}{3.914334in}}{\pgfqpoint{8.651325in}{3.919920in}}{\pgfqpoint{8.651325in}{3.925744in}}%
\pgfpathcurveto{\pgfqpoint{8.651325in}{3.931568in}}{\pgfqpoint{8.649011in}{3.937154in}}{\pgfqpoint{8.644893in}{3.941272in}}%
\pgfpathcurveto{\pgfqpoint{8.640775in}{3.945390in}}{\pgfqpoint{8.635189in}{3.947704in}}{\pgfqpoint{8.629365in}{3.947704in}}%
\pgfpathcurveto{\pgfqpoint{8.623541in}{3.947704in}}{\pgfqpoint{8.617955in}{3.945390in}}{\pgfqpoint{8.613836in}{3.941272in}}%
\pgfpathcurveto{\pgfqpoint{8.609718in}{3.937154in}}{\pgfqpoint{8.607404in}{3.931568in}}{\pgfqpoint{8.607404in}{3.925744in}}%
\pgfpathcurveto{\pgfqpoint{8.607404in}{3.919920in}}{\pgfqpoint{8.609718in}{3.914334in}}{\pgfqpoint{8.613836in}{3.910216in}}%
\pgfpathcurveto{\pgfqpoint{8.617955in}{3.906098in}}{\pgfqpoint{8.623541in}{3.903784in}}{\pgfqpoint{8.629365in}{3.903784in}}%
\pgfpathlineto{\pgfqpoint{8.629365in}{3.903784in}}%
\pgfpathclose%
\pgfusepath{stroke,fill}%
\end{pgfscope}%
\begin{pgfscope}%
\pgfpathrectangle{\pgfqpoint{0.640323in}{0.527436in}}{\pgfqpoint{9.687500in}{3.850000in}}%
\pgfusepath{clip}%
\pgfsetbuttcap%
\pgfsetroundjoin%
\definecolor{currentfill}{rgb}{0.980392,0.164706,0.333333}%
\pgfsetfillcolor{currentfill}%
\pgfsetfillopacity{0.500000}%
\pgfsetlinewidth{1.003750pt}%
\definecolor{currentstroke}{rgb}{0.980392,0.164706,0.333333}%
\pgfsetstrokecolor{currentstroke}%
\pgfsetstrokeopacity{0.500000}%
\pgfsetdash{{3.700000pt}{1.600000pt}}{0.000000pt}%
\pgfpathmoveto{\pgfqpoint{8.809096in}{3.919271in}}%
\pgfpathcurveto{\pgfqpoint{8.814920in}{3.919271in}}{\pgfqpoint{8.820506in}{3.921584in}}{\pgfqpoint{8.824624in}{3.925703in}}%
\pgfpathcurveto{\pgfqpoint{8.828742in}{3.929821in}}{\pgfqpoint{8.831056in}{3.935407in}}{\pgfqpoint{8.831056in}{3.941231in}}%
\pgfpathcurveto{\pgfqpoint{8.831056in}{3.947055in}}{\pgfqpoint{8.828742in}{3.952641in}}{\pgfqpoint{8.824624in}{3.956759in}}%
\pgfpathcurveto{\pgfqpoint{8.820506in}{3.960877in}}{\pgfqpoint{8.814920in}{3.963191in}}{\pgfqpoint{8.809096in}{3.963191in}}%
\pgfpathcurveto{\pgfqpoint{8.803272in}{3.963191in}}{\pgfqpoint{8.797686in}{3.960877in}}{\pgfqpoint{8.793567in}{3.956759in}}%
\pgfpathcurveto{\pgfqpoint{8.789449in}{3.952641in}}{\pgfqpoint{8.787135in}{3.947055in}}{\pgfqpoint{8.787135in}{3.941231in}}%
\pgfpathcurveto{\pgfqpoint{8.787135in}{3.935407in}}{\pgfqpoint{8.789449in}{3.929821in}}{\pgfqpoint{8.793567in}{3.925703in}}%
\pgfpathcurveto{\pgfqpoint{8.797686in}{3.921584in}}{\pgfqpoint{8.803272in}{3.919271in}}{\pgfqpoint{8.809096in}{3.919271in}}%
\pgfpathlineto{\pgfqpoint{8.809096in}{3.919271in}}%
\pgfpathclose%
\pgfusepath{stroke,fill}%
\end{pgfscope}%
\begin{pgfscope}%
\pgfpathrectangle{\pgfqpoint{0.640323in}{0.527436in}}{\pgfqpoint{9.687500in}{3.850000in}}%
\pgfusepath{clip}%
\pgfsetbuttcap%
\pgfsetroundjoin%
\definecolor{currentfill}{rgb}{0.980392,0.164706,0.333333}%
\pgfsetfillcolor{currentfill}%
\pgfsetfillopacity{0.500000}%
\pgfsetlinewidth{1.003750pt}%
\definecolor{currentstroke}{rgb}{0.980392,0.164706,0.333333}%
\pgfsetstrokecolor{currentstroke}%
\pgfsetstrokeopacity{0.500000}%
\pgfsetdash{{3.700000pt}{1.600000pt}}{0.000000pt}%
\pgfpathmoveto{\pgfqpoint{8.988827in}{3.934807in}}%
\pgfpathcurveto{\pgfqpoint{8.994651in}{3.934807in}}{\pgfqpoint{9.000237in}{3.937121in}}{\pgfqpoint{9.004355in}{3.941239in}}%
\pgfpathcurveto{\pgfqpoint{9.008473in}{3.945357in}}{\pgfqpoint{9.010787in}{3.950944in}}{\pgfqpoint{9.010787in}{3.956767in}}%
\pgfpathcurveto{\pgfqpoint{9.010787in}{3.962591in}}{\pgfqpoint{9.008473in}{3.968178in}}{\pgfqpoint{9.004355in}{3.972296in}}%
\pgfpathcurveto{\pgfqpoint{9.000237in}{3.976414in}}{\pgfqpoint{8.994651in}{3.978728in}}{\pgfqpoint{8.988827in}{3.978728in}}%
\pgfpathcurveto{\pgfqpoint{8.983003in}{3.978728in}}{\pgfqpoint{8.977417in}{3.976414in}}{\pgfqpoint{8.973298in}{3.972296in}}%
\pgfpathcurveto{\pgfqpoint{8.969180in}{3.968178in}}{\pgfqpoint{8.966866in}{3.962591in}}{\pgfqpoint{8.966866in}{3.956767in}}%
\pgfpathcurveto{\pgfqpoint{8.966866in}{3.950944in}}{\pgfqpoint{8.969180in}{3.945357in}}{\pgfqpoint{8.973298in}{3.941239in}}%
\pgfpathcurveto{\pgfqpoint{8.977417in}{3.937121in}}{\pgfqpoint{8.983003in}{3.934807in}}{\pgfqpoint{8.988827in}{3.934807in}}%
\pgfpathlineto{\pgfqpoint{8.988827in}{3.934807in}}%
\pgfpathclose%
\pgfusepath{stroke,fill}%
\end{pgfscope}%
\begin{pgfscope}%
\pgfpathrectangle{\pgfqpoint{0.640323in}{0.527436in}}{\pgfqpoint{9.687500in}{3.850000in}}%
\pgfusepath{clip}%
\pgfsetbuttcap%
\pgfsetroundjoin%
\definecolor{currentfill}{rgb}{0.980392,0.164706,0.333333}%
\pgfsetfillcolor{currentfill}%
\pgfsetfillopacity{0.500000}%
\pgfsetlinewidth{1.003750pt}%
\definecolor{currentstroke}{rgb}{0.980392,0.164706,0.333333}%
\pgfsetstrokecolor{currentstroke}%
\pgfsetstrokeopacity{0.500000}%
\pgfsetdash{{3.700000pt}{1.600000pt}}{0.000000pt}%
\pgfpathmoveto{\pgfqpoint{9.168558in}{3.957882in}}%
\pgfpathcurveto{\pgfqpoint{9.174382in}{3.957882in}}{\pgfqpoint{9.179968in}{3.960196in}}{\pgfqpoint{9.184086in}{3.964314in}}%
\pgfpathcurveto{\pgfqpoint{9.188204in}{3.968432in}}{\pgfqpoint{9.190518in}{3.974019in}}{\pgfqpoint{9.190518in}{3.979843in}}%
\pgfpathcurveto{\pgfqpoint{9.190518in}{3.985667in}}{\pgfqpoint{9.188204in}{3.991253in}}{\pgfqpoint{9.184086in}{3.995371in}}%
\pgfpathcurveto{\pgfqpoint{9.179968in}{3.999489in}}{\pgfqpoint{9.174382in}{4.001803in}}{\pgfqpoint{9.168558in}{4.001803in}}%
\pgfpathcurveto{\pgfqpoint{9.162734in}{4.001803in}}{\pgfqpoint{9.157148in}{3.999489in}}{\pgfqpoint{9.153029in}{3.995371in}}%
\pgfpathcurveto{\pgfqpoint{9.148911in}{3.991253in}}{\pgfqpoint{9.146597in}{3.985667in}}{\pgfqpoint{9.146597in}{3.979843in}}%
\pgfpathcurveto{\pgfqpoint{9.146597in}{3.974019in}}{\pgfqpoint{9.148911in}{3.968432in}}{\pgfqpoint{9.153029in}{3.964314in}}%
\pgfpathcurveto{\pgfqpoint{9.157148in}{3.960196in}}{\pgfqpoint{9.162734in}{3.957882in}}{\pgfqpoint{9.168558in}{3.957882in}}%
\pgfpathlineto{\pgfqpoint{9.168558in}{3.957882in}}%
\pgfpathclose%
\pgfusepath{stroke,fill}%
\end{pgfscope}%
\begin{pgfscope}%
\pgfpathrectangle{\pgfqpoint{0.640323in}{0.527436in}}{\pgfqpoint{9.687500in}{3.850000in}}%
\pgfusepath{clip}%
\pgfsetbuttcap%
\pgfsetroundjoin%
\definecolor{currentfill}{rgb}{0.980392,0.164706,0.333333}%
\pgfsetfillcolor{currentfill}%
\pgfsetfillopacity{0.500000}%
\pgfsetlinewidth{1.003750pt}%
\definecolor{currentstroke}{rgb}{0.980392,0.164706,0.333333}%
\pgfsetstrokecolor{currentstroke}%
\pgfsetstrokeopacity{0.500000}%
\pgfsetdash{{3.700000pt}{1.600000pt}}{0.000000pt}%
\pgfpathmoveto{\pgfqpoint{9.348289in}{3.961931in}}%
\pgfpathcurveto{\pgfqpoint{9.354113in}{3.961931in}}{\pgfqpoint{9.359699in}{3.964245in}}{\pgfqpoint{9.363817in}{3.968363in}}%
\pgfpathcurveto{\pgfqpoint{9.367935in}{3.972481in}}{\pgfqpoint{9.370249in}{3.978067in}}{\pgfqpoint{9.370249in}{3.983891in}}%
\pgfpathcurveto{\pgfqpoint{9.370249in}{3.989715in}}{\pgfqpoint{9.367935in}{3.995301in}}{\pgfqpoint{9.363817in}{3.999420in}}%
\pgfpathcurveto{\pgfqpoint{9.359699in}{4.003538in}}{\pgfqpoint{9.354113in}{4.005852in}}{\pgfqpoint{9.348289in}{4.005852in}}%
\pgfpathcurveto{\pgfqpoint{9.342465in}{4.005852in}}{\pgfqpoint{9.336879in}{4.003538in}}{\pgfqpoint{9.332760in}{3.999420in}}%
\pgfpathcurveto{\pgfqpoint{9.328642in}{3.995301in}}{\pgfqpoint{9.326328in}{3.989715in}}{\pgfqpoint{9.326328in}{3.983891in}}%
\pgfpathcurveto{\pgfqpoint{9.326328in}{3.978067in}}{\pgfqpoint{9.328642in}{3.972481in}}{\pgfqpoint{9.332760in}{3.968363in}}%
\pgfpathcurveto{\pgfqpoint{9.336879in}{3.964245in}}{\pgfqpoint{9.342465in}{3.961931in}}{\pgfqpoint{9.348289in}{3.961931in}}%
\pgfpathlineto{\pgfqpoint{9.348289in}{3.961931in}}%
\pgfpathclose%
\pgfusepath{stroke,fill}%
\end{pgfscope}%
\begin{pgfscope}%
\pgfpathrectangle{\pgfqpoint{0.640323in}{0.527436in}}{\pgfqpoint{9.687500in}{3.850000in}}%
\pgfusepath{clip}%
\pgfsetbuttcap%
\pgfsetroundjoin%
\definecolor{currentfill}{rgb}{0.980392,0.164706,0.333333}%
\pgfsetfillcolor{currentfill}%
\pgfsetfillopacity{0.500000}%
\pgfsetlinewidth{1.003750pt}%
\definecolor{currentstroke}{rgb}{0.980392,0.164706,0.333333}%
\pgfsetstrokecolor{currentstroke}%
\pgfsetstrokeopacity{0.500000}%
\pgfsetdash{{3.700000pt}{1.600000pt}}{0.000000pt}%
\pgfpathmoveto{\pgfqpoint{9.528020in}{3.987409in}}%
\pgfpathcurveto{\pgfqpoint{9.533844in}{3.987409in}}{\pgfqpoint{9.539430in}{3.989723in}}{\pgfqpoint{9.543548in}{3.993841in}}%
\pgfpathcurveto{\pgfqpoint{9.547666in}{3.997960in}}{\pgfqpoint{9.549980in}{4.003546in}}{\pgfqpoint{9.549980in}{4.009370in}}%
\pgfpathcurveto{\pgfqpoint{9.549980in}{4.015194in}}{\pgfqpoint{9.547666in}{4.020780in}}{\pgfqpoint{9.543548in}{4.024898in}}%
\pgfpathcurveto{\pgfqpoint{9.539430in}{4.029016in}}{\pgfqpoint{9.533844in}{4.031330in}}{\pgfqpoint{9.528020in}{4.031330in}}%
\pgfpathcurveto{\pgfqpoint{9.522196in}{4.031330in}}{\pgfqpoint{9.516610in}{4.029016in}}{\pgfqpoint{9.512491in}{4.024898in}}%
\pgfpathcurveto{\pgfqpoint{9.508373in}{4.020780in}}{\pgfqpoint{9.506059in}{4.015194in}}{\pgfqpoint{9.506059in}{4.009370in}}%
\pgfpathcurveto{\pgfqpoint{9.506059in}{4.003546in}}{\pgfqpoint{9.508373in}{3.997960in}}{\pgfqpoint{9.512491in}{3.993841in}}%
\pgfpathcurveto{\pgfqpoint{9.516610in}{3.989723in}}{\pgfqpoint{9.522196in}{3.987409in}}{\pgfqpoint{9.528020in}{3.987409in}}%
\pgfpathlineto{\pgfqpoint{9.528020in}{3.987409in}}%
\pgfpathclose%
\pgfusepath{stroke,fill}%
\end{pgfscope}%
\begin{pgfscope}%
\pgfpathrectangle{\pgfqpoint{0.640323in}{0.527436in}}{\pgfqpoint{9.687500in}{3.850000in}}%
\pgfusepath{clip}%
\pgfsetbuttcap%
\pgfsetroundjoin%
\definecolor{currentfill}{rgb}{0.980392,0.164706,0.333333}%
\pgfsetfillcolor{currentfill}%
\pgfsetfillopacity{0.500000}%
\pgfsetlinewidth{1.003750pt}%
\definecolor{currentstroke}{rgb}{0.980392,0.164706,0.333333}%
\pgfsetstrokecolor{currentstroke}%
\pgfsetstrokeopacity{0.500000}%
\pgfsetdash{{3.700000pt}{1.600000pt}}{0.000000pt}%
\pgfpathmoveto{\pgfqpoint{9.707751in}{3.994886in}}%
\pgfpathcurveto{\pgfqpoint{9.713575in}{3.994886in}}{\pgfqpoint{9.719161in}{3.997200in}}{\pgfqpoint{9.723279in}{4.001318in}}%
\pgfpathcurveto{\pgfqpoint{9.727397in}{4.005436in}}{\pgfqpoint{9.729711in}{4.011022in}}{\pgfqpoint{9.729711in}{4.016846in}}%
\pgfpathcurveto{\pgfqpoint{9.729711in}{4.022670in}}{\pgfqpoint{9.727397in}{4.028256in}}{\pgfqpoint{9.723279in}{4.032374in}}%
\pgfpathcurveto{\pgfqpoint{9.719161in}{4.036492in}}{\pgfqpoint{9.713575in}{4.038806in}}{\pgfqpoint{9.707751in}{4.038806in}}%
\pgfpathcurveto{\pgfqpoint{9.701927in}{4.038806in}}{\pgfqpoint{9.696340in}{4.036492in}}{\pgfqpoint{9.692222in}{4.032374in}}%
\pgfpathcurveto{\pgfqpoint{9.688104in}{4.028256in}}{\pgfqpoint{9.685790in}{4.022670in}}{\pgfqpoint{9.685790in}{4.016846in}}%
\pgfpathcurveto{\pgfqpoint{9.685790in}{4.011022in}}{\pgfqpoint{9.688104in}{4.005436in}}{\pgfqpoint{9.692222in}{4.001318in}}%
\pgfpathcurveto{\pgfqpoint{9.696340in}{3.997200in}}{\pgfqpoint{9.701927in}{3.994886in}}{\pgfqpoint{9.707751in}{3.994886in}}%
\pgfpathlineto{\pgfqpoint{9.707751in}{3.994886in}}%
\pgfpathclose%
\pgfusepath{stroke,fill}%
\end{pgfscope}%
\begin{pgfscope}%
\pgfpathrectangle{\pgfqpoint{0.640323in}{0.527436in}}{\pgfqpoint{9.687500in}{3.850000in}}%
\pgfusepath{clip}%
\pgfsetbuttcap%
\pgfsetroundjoin%
\definecolor{currentfill}{rgb}{0.980392,0.164706,0.333333}%
\pgfsetfillcolor{currentfill}%
\pgfsetfillopacity{0.500000}%
\pgfsetlinewidth{1.003750pt}%
\definecolor{currentstroke}{rgb}{0.980392,0.164706,0.333333}%
\pgfsetstrokecolor{currentstroke}%
\pgfsetstrokeopacity{0.500000}%
\pgfsetdash{{3.700000pt}{1.600000pt}}{0.000000pt}%
\pgfpathmoveto{\pgfqpoint{9.887482in}{4.015614in}}%
\pgfpathcurveto{\pgfqpoint{9.893306in}{4.015614in}}{\pgfqpoint{9.898892in}{4.017928in}}{\pgfqpoint{9.903010in}{4.022046in}}%
\pgfpathcurveto{\pgfqpoint{9.907128in}{4.026164in}}{\pgfqpoint{9.909442in}{4.031750in}}{\pgfqpoint{9.909442in}{4.037574in}}%
\pgfpathcurveto{\pgfqpoint{9.909442in}{4.043398in}}{\pgfqpoint{9.907128in}{4.048984in}}{\pgfqpoint{9.903010in}{4.053102in}}%
\pgfpathcurveto{\pgfqpoint{9.898892in}{4.057220in}}{\pgfqpoint{9.893306in}{4.059534in}}{\pgfqpoint{9.887482in}{4.059534in}}%
\pgfpathcurveto{\pgfqpoint{9.881658in}{4.059534in}}{\pgfqpoint{9.876071in}{4.057220in}}{\pgfqpoint{9.871953in}{4.053102in}}%
\pgfpathcurveto{\pgfqpoint{9.867835in}{4.048984in}}{\pgfqpoint{9.865521in}{4.043398in}}{\pgfqpoint{9.865521in}{4.037574in}}%
\pgfpathcurveto{\pgfqpoint{9.865521in}{4.031750in}}{\pgfqpoint{9.867835in}{4.026164in}}{\pgfqpoint{9.871953in}{4.022046in}}%
\pgfpathcurveto{\pgfqpoint{9.876071in}{4.017928in}}{\pgfqpoint{9.881658in}{4.015614in}}{\pgfqpoint{9.887482in}{4.015614in}}%
\pgfpathlineto{\pgfqpoint{9.887482in}{4.015614in}}%
\pgfpathclose%
\pgfusepath{stroke,fill}%
\end{pgfscope}%
\begin{pgfscope}%
\pgfpathrectangle{\pgfqpoint{0.640323in}{0.527436in}}{\pgfqpoint{9.687500in}{3.850000in}}%
\pgfusepath{clip}%
\pgfsetbuttcap%
\pgfsetroundjoin%
\definecolor{currentfill}{rgb}{0.239216,0.478431,0.992157}%
\pgfsetfillcolor{currentfill}%
\pgfsetfillopacity{0.500000}%
\pgfsetlinewidth{1.003750pt}%
\definecolor{currentstroke}{rgb}{0.239216,0.478431,0.992157}%
\pgfsetstrokecolor{currentstroke}%
\pgfsetstrokeopacity{0.500000}%
\pgfsetdash{{3.700000pt}{1.600000pt}}{0.000000pt}%
\pgfpathmoveto{\pgfqpoint{1.080663in}{0.637815in}}%
\pgfpathcurveto{\pgfqpoint{1.086487in}{0.637815in}}{\pgfqpoint{1.092074in}{0.640129in}}{\pgfqpoint{1.096192in}{0.644247in}}%
\pgfpathcurveto{\pgfqpoint{1.100310in}{0.648365in}}{\pgfqpoint{1.102624in}{0.653951in}}{\pgfqpoint{1.102624in}{0.659775in}}%
\pgfpathcurveto{\pgfqpoint{1.102624in}{0.665599in}}{\pgfqpoint{1.100310in}{0.671185in}}{\pgfqpoint{1.096192in}{0.675303in}}%
\pgfpathcurveto{\pgfqpoint{1.092074in}{0.679421in}}{\pgfqpoint{1.086487in}{0.681735in}}{\pgfqpoint{1.080663in}{0.681735in}}%
\pgfpathcurveto{\pgfqpoint{1.074839in}{0.681735in}}{\pgfqpoint{1.069253in}{0.679421in}}{\pgfqpoint{1.065135in}{0.675303in}}%
\pgfpathcurveto{\pgfqpoint{1.061017in}{0.671185in}}{\pgfqpoint{1.058703in}{0.665599in}}{\pgfqpoint{1.058703in}{0.659775in}}%
\pgfpathcurveto{\pgfqpoint{1.058703in}{0.653951in}}{\pgfqpoint{1.061017in}{0.648365in}}{\pgfqpoint{1.065135in}{0.644247in}}%
\pgfpathcurveto{\pgfqpoint{1.069253in}{0.640129in}}{\pgfqpoint{1.074839in}{0.637815in}}{\pgfqpoint{1.080663in}{0.637815in}}%
\pgfpathlineto{\pgfqpoint{1.080663in}{0.637815in}}%
\pgfpathclose%
\pgfusepath{stroke,fill}%
\end{pgfscope}%
\begin{pgfscope}%
\pgfpathrectangle{\pgfqpoint{0.640323in}{0.527436in}}{\pgfqpoint{9.687500in}{3.850000in}}%
\pgfusepath{clip}%
\pgfsetbuttcap%
\pgfsetroundjoin%
\definecolor{currentfill}{rgb}{0.239216,0.478431,0.992157}%
\pgfsetfillcolor{currentfill}%
\pgfsetfillopacity{0.500000}%
\pgfsetlinewidth{1.003750pt}%
\definecolor{currentstroke}{rgb}{0.239216,0.478431,0.992157}%
\pgfsetstrokecolor{currentstroke}%
\pgfsetstrokeopacity{0.500000}%
\pgfsetdash{{3.700000pt}{1.600000pt}}{0.000000pt}%
\pgfpathmoveto{\pgfqpoint{1.260394in}{0.638123in}}%
\pgfpathcurveto{\pgfqpoint{1.266218in}{0.638123in}}{\pgfqpoint{1.271805in}{0.640436in}}{\pgfqpoint{1.275923in}{0.644555in}}%
\pgfpathcurveto{\pgfqpoint{1.280041in}{0.648673in}}{\pgfqpoint{1.282355in}{0.654259in}}{\pgfqpoint{1.282355in}{0.660083in}}%
\pgfpathcurveto{\pgfqpoint{1.282355in}{0.665907in}}{\pgfqpoint{1.280041in}{0.671493in}}{\pgfqpoint{1.275923in}{0.675611in}}%
\pgfpathcurveto{\pgfqpoint{1.271805in}{0.679729in}}{\pgfqpoint{1.266218in}{0.682043in}}{\pgfqpoint{1.260394in}{0.682043in}}%
\pgfpathcurveto{\pgfqpoint{1.254570in}{0.682043in}}{\pgfqpoint{1.248984in}{0.679729in}}{\pgfqpoint{1.244866in}{0.675611in}}%
\pgfpathcurveto{\pgfqpoint{1.240748in}{0.671493in}}{\pgfqpoint{1.238434in}{0.665907in}}{\pgfqpoint{1.238434in}{0.660083in}}%
\pgfpathcurveto{\pgfqpoint{1.238434in}{0.654259in}}{\pgfqpoint{1.240748in}{0.648673in}}{\pgfqpoint{1.244866in}{0.644555in}}%
\pgfpathcurveto{\pgfqpoint{1.248984in}{0.640436in}}{\pgfqpoint{1.254570in}{0.638123in}}{\pgfqpoint{1.260394in}{0.638123in}}%
\pgfpathlineto{\pgfqpoint{1.260394in}{0.638123in}}%
\pgfpathclose%
\pgfusepath{stroke,fill}%
\end{pgfscope}%
\begin{pgfscope}%
\pgfpathrectangle{\pgfqpoint{0.640323in}{0.527436in}}{\pgfqpoint{9.687500in}{3.850000in}}%
\pgfusepath{clip}%
\pgfsetbuttcap%
\pgfsetroundjoin%
\definecolor{currentfill}{rgb}{0.239216,0.478431,0.992157}%
\pgfsetfillcolor{currentfill}%
\pgfsetfillopacity{0.500000}%
\pgfsetlinewidth{1.003750pt}%
\definecolor{currentstroke}{rgb}{0.239216,0.478431,0.992157}%
\pgfsetstrokecolor{currentstroke}%
\pgfsetstrokeopacity{0.500000}%
\pgfsetdash{{3.700000pt}{1.600000pt}}{0.000000pt}%
\pgfpathmoveto{\pgfqpoint{1.440125in}{0.640518in}}%
\pgfpathcurveto{\pgfqpoint{1.445949in}{0.640518in}}{\pgfqpoint{1.451535in}{0.642832in}}{\pgfqpoint{1.455654in}{0.646950in}}%
\pgfpathcurveto{\pgfqpoint{1.459772in}{0.651068in}}{\pgfqpoint{1.462086in}{0.656654in}}{\pgfqpoint{1.462086in}{0.662478in}}%
\pgfpathcurveto{\pgfqpoint{1.462086in}{0.668302in}}{\pgfqpoint{1.459772in}{0.673888in}}{\pgfqpoint{1.455654in}{0.678006in}}%
\pgfpathcurveto{\pgfqpoint{1.451535in}{0.682125in}}{\pgfqpoint{1.445949in}{0.684438in}}{\pgfqpoint{1.440125in}{0.684438in}}%
\pgfpathcurveto{\pgfqpoint{1.434301in}{0.684438in}}{\pgfqpoint{1.428715in}{0.682125in}}{\pgfqpoint{1.424597in}{0.678006in}}%
\pgfpathcurveto{\pgfqpoint{1.420479in}{0.673888in}}{\pgfqpoint{1.418165in}{0.668302in}}{\pgfqpoint{1.418165in}{0.662478in}}%
\pgfpathcurveto{\pgfqpoint{1.418165in}{0.656654in}}{\pgfqpoint{1.420479in}{0.651068in}}{\pgfqpoint{1.424597in}{0.646950in}}%
\pgfpathcurveto{\pgfqpoint{1.428715in}{0.642832in}}{\pgfqpoint{1.434301in}{0.640518in}}{\pgfqpoint{1.440125in}{0.640518in}}%
\pgfpathlineto{\pgfqpoint{1.440125in}{0.640518in}}%
\pgfpathclose%
\pgfusepath{stroke,fill}%
\end{pgfscope}%
\begin{pgfscope}%
\pgfpathrectangle{\pgfqpoint{0.640323in}{0.527436in}}{\pgfqpoint{9.687500in}{3.850000in}}%
\pgfusepath{clip}%
\pgfsetbuttcap%
\pgfsetroundjoin%
\definecolor{currentfill}{rgb}{0.239216,0.478431,0.992157}%
\pgfsetfillcolor{currentfill}%
\pgfsetfillopacity{0.500000}%
\pgfsetlinewidth{1.003750pt}%
\definecolor{currentstroke}{rgb}{0.239216,0.478431,0.992157}%
\pgfsetstrokecolor{currentstroke}%
\pgfsetstrokeopacity{0.500000}%
\pgfsetdash{{3.700000pt}{1.600000pt}}{0.000000pt}%
\pgfpathmoveto{\pgfqpoint{1.619856in}{0.642237in}}%
\pgfpathcurveto{\pgfqpoint{1.625680in}{0.642237in}}{\pgfqpoint{1.631266in}{0.644551in}}{\pgfqpoint{1.635385in}{0.648669in}}%
\pgfpathcurveto{\pgfqpoint{1.639503in}{0.652787in}}{\pgfqpoint{1.641817in}{0.658373in}}{\pgfqpoint{1.641817in}{0.664197in}}%
\pgfpathcurveto{\pgfqpoint{1.641817in}{0.670021in}}{\pgfqpoint{1.639503in}{0.675607in}}{\pgfqpoint{1.635385in}{0.679726in}}%
\pgfpathcurveto{\pgfqpoint{1.631266in}{0.683844in}}{\pgfqpoint{1.625680in}{0.686158in}}{\pgfqpoint{1.619856in}{0.686158in}}%
\pgfpathcurveto{\pgfqpoint{1.614032in}{0.686158in}}{\pgfqpoint{1.608446in}{0.683844in}}{\pgfqpoint{1.604328in}{0.679726in}}%
\pgfpathcurveto{\pgfqpoint{1.600210in}{0.675607in}}{\pgfqpoint{1.597896in}{0.670021in}}{\pgfqpoint{1.597896in}{0.664197in}}%
\pgfpathcurveto{\pgfqpoint{1.597896in}{0.658373in}}{\pgfqpoint{1.600210in}{0.652787in}}{\pgfqpoint{1.604328in}{0.648669in}}%
\pgfpathcurveto{\pgfqpoint{1.608446in}{0.644551in}}{\pgfqpoint{1.614032in}{0.642237in}}{\pgfqpoint{1.619856in}{0.642237in}}%
\pgfpathlineto{\pgfqpoint{1.619856in}{0.642237in}}%
\pgfpathclose%
\pgfusepath{stroke,fill}%
\end{pgfscope}%
\begin{pgfscope}%
\pgfpathrectangle{\pgfqpoint{0.640323in}{0.527436in}}{\pgfqpoint{9.687500in}{3.850000in}}%
\pgfusepath{clip}%
\pgfsetbuttcap%
\pgfsetroundjoin%
\definecolor{currentfill}{rgb}{0.239216,0.478431,0.992157}%
\pgfsetfillcolor{currentfill}%
\pgfsetfillopacity{0.500000}%
\pgfsetlinewidth{1.003750pt}%
\definecolor{currentstroke}{rgb}{0.239216,0.478431,0.992157}%
\pgfsetstrokecolor{currentstroke}%
\pgfsetstrokeopacity{0.500000}%
\pgfsetdash{{3.700000pt}{1.600000pt}}{0.000000pt}%
\pgfpathmoveto{\pgfqpoint{1.799587in}{0.644875in}}%
\pgfpathcurveto{\pgfqpoint{1.805411in}{0.644875in}}{\pgfqpoint{1.810997in}{0.647189in}}{\pgfqpoint{1.815116in}{0.651307in}}%
\pgfpathcurveto{\pgfqpoint{1.819234in}{0.655425in}}{\pgfqpoint{1.821548in}{0.661011in}}{\pgfqpoint{1.821548in}{0.666835in}}%
\pgfpathcurveto{\pgfqpoint{1.821548in}{0.672659in}}{\pgfqpoint{1.819234in}{0.678245in}}{\pgfqpoint{1.815116in}{0.682363in}}%
\pgfpathcurveto{\pgfqpoint{1.810997in}{0.686482in}}{\pgfqpoint{1.805411in}{0.688795in}}{\pgfqpoint{1.799587in}{0.688795in}}%
\pgfpathcurveto{\pgfqpoint{1.793763in}{0.688795in}}{\pgfqpoint{1.788177in}{0.686482in}}{\pgfqpoint{1.784059in}{0.682363in}}%
\pgfpathcurveto{\pgfqpoint{1.779941in}{0.678245in}}{\pgfqpoint{1.777627in}{0.672659in}}{\pgfqpoint{1.777627in}{0.666835in}}%
\pgfpathcurveto{\pgfqpoint{1.777627in}{0.661011in}}{\pgfqpoint{1.779941in}{0.655425in}}{\pgfqpoint{1.784059in}{0.651307in}}%
\pgfpathcurveto{\pgfqpoint{1.788177in}{0.647189in}}{\pgfqpoint{1.793763in}{0.644875in}}{\pgfqpoint{1.799587in}{0.644875in}}%
\pgfpathlineto{\pgfqpoint{1.799587in}{0.644875in}}%
\pgfpathclose%
\pgfusepath{stroke,fill}%
\end{pgfscope}%
\begin{pgfscope}%
\pgfpathrectangle{\pgfqpoint{0.640323in}{0.527436in}}{\pgfqpoint{9.687500in}{3.850000in}}%
\pgfusepath{clip}%
\pgfsetbuttcap%
\pgfsetroundjoin%
\definecolor{currentfill}{rgb}{0.239216,0.478431,0.992157}%
\pgfsetfillcolor{currentfill}%
\pgfsetfillopacity{0.500000}%
\pgfsetlinewidth{1.003750pt}%
\definecolor{currentstroke}{rgb}{0.239216,0.478431,0.992157}%
\pgfsetstrokecolor{currentstroke}%
\pgfsetstrokeopacity{0.500000}%
\pgfsetdash{{3.700000pt}{1.600000pt}}{0.000000pt}%
\pgfpathmoveto{\pgfqpoint{1.979318in}{0.651476in}}%
\pgfpathcurveto{\pgfqpoint{1.985142in}{0.651476in}}{\pgfqpoint{1.990728in}{0.653790in}}{\pgfqpoint{1.994847in}{0.657908in}}%
\pgfpathcurveto{\pgfqpoint{1.998965in}{0.662026in}}{\pgfqpoint{2.001279in}{0.667612in}}{\pgfqpoint{2.001279in}{0.673436in}}%
\pgfpathcurveto{\pgfqpoint{2.001279in}{0.679260in}}{\pgfqpoint{1.998965in}{0.684846in}}{\pgfqpoint{1.994847in}{0.688964in}}%
\pgfpathcurveto{\pgfqpoint{1.990728in}{0.693082in}}{\pgfqpoint{1.985142in}{0.695396in}}{\pgfqpoint{1.979318in}{0.695396in}}%
\pgfpathcurveto{\pgfqpoint{1.973494in}{0.695396in}}{\pgfqpoint{1.967908in}{0.693082in}}{\pgfqpoint{1.963790in}{0.688964in}}%
\pgfpathcurveto{\pgfqpoint{1.959672in}{0.684846in}}{\pgfqpoint{1.957358in}{0.679260in}}{\pgfqpoint{1.957358in}{0.673436in}}%
\pgfpathcurveto{\pgfqpoint{1.957358in}{0.667612in}}{\pgfqpoint{1.959672in}{0.662026in}}{\pgfqpoint{1.963790in}{0.657908in}}%
\pgfpathcurveto{\pgfqpoint{1.967908in}{0.653790in}}{\pgfqpoint{1.973494in}{0.651476in}}{\pgfqpoint{1.979318in}{0.651476in}}%
\pgfpathlineto{\pgfqpoint{1.979318in}{0.651476in}}%
\pgfpathclose%
\pgfusepath{stroke,fill}%
\end{pgfscope}%
\begin{pgfscope}%
\pgfpathrectangle{\pgfqpoint{0.640323in}{0.527436in}}{\pgfqpoint{9.687500in}{3.850000in}}%
\pgfusepath{clip}%
\pgfsetbuttcap%
\pgfsetroundjoin%
\definecolor{currentfill}{rgb}{0.239216,0.478431,0.992157}%
\pgfsetfillcolor{currentfill}%
\pgfsetfillopacity{0.500000}%
\pgfsetlinewidth{1.003750pt}%
\definecolor{currentstroke}{rgb}{0.239216,0.478431,0.992157}%
\pgfsetstrokecolor{currentstroke}%
\pgfsetstrokeopacity{0.500000}%
\pgfsetdash{{3.700000pt}{1.600000pt}}{0.000000pt}%
\pgfpathmoveto{\pgfqpoint{2.159049in}{0.662170in}}%
\pgfpathcurveto{\pgfqpoint{2.164873in}{0.662170in}}{\pgfqpoint{2.170459in}{0.664484in}}{\pgfqpoint{2.174578in}{0.668602in}}%
\pgfpathcurveto{\pgfqpoint{2.178696in}{0.672720in}}{\pgfqpoint{2.181010in}{0.678306in}}{\pgfqpoint{2.181010in}{0.684130in}}%
\pgfpathcurveto{\pgfqpoint{2.181010in}{0.689954in}}{\pgfqpoint{2.178696in}{0.695540in}}{\pgfqpoint{2.174578in}{0.699658in}}%
\pgfpathcurveto{\pgfqpoint{2.170459in}{0.703777in}}{\pgfqpoint{2.164873in}{0.706090in}}{\pgfqpoint{2.159049in}{0.706090in}}%
\pgfpathcurveto{\pgfqpoint{2.153225in}{0.706090in}}{\pgfqpoint{2.147639in}{0.703777in}}{\pgfqpoint{2.143521in}{0.699658in}}%
\pgfpathcurveto{\pgfqpoint{2.139403in}{0.695540in}}{\pgfqpoint{2.137089in}{0.689954in}}{\pgfqpoint{2.137089in}{0.684130in}}%
\pgfpathcurveto{\pgfqpoint{2.137089in}{0.678306in}}{\pgfqpoint{2.139403in}{0.672720in}}{\pgfqpoint{2.143521in}{0.668602in}}%
\pgfpathcurveto{\pgfqpoint{2.147639in}{0.664484in}}{\pgfqpoint{2.153225in}{0.662170in}}{\pgfqpoint{2.159049in}{0.662170in}}%
\pgfpathlineto{\pgfqpoint{2.159049in}{0.662170in}}%
\pgfpathclose%
\pgfusepath{stroke,fill}%
\end{pgfscope}%
\begin{pgfscope}%
\pgfpathrectangle{\pgfqpoint{0.640323in}{0.527436in}}{\pgfqpoint{9.687500in}{3.850000in}}%
\pgfusepath{clip}%
\pgfsetbuttcap%
\pgfsetroundjoin%
\definecolor{currentfill}{rgb}{0.239216,0.478431,0.992157}%
\pgfsetfillcolor{currentfill}%
\pgfsetfillopacity{0.500000}%
\pgfsetlinewidth{1.003750pt}%
\definecolor{currentstroke}{rgb}{0.239216,0.478431,0.992157}%
\pgfsetstrokecolor{currentstroke}%
\pgfsetstrokeopacity{0.500000}%
\pgfsetdash{{3.700000pt}{1.600000pt}}{0.000000pt}%
\pgfpathmoveto{\pgfqpoint{2.338780in}{0.677615in}}%
\pgfpathcurveto{\pgfqpoint{2.344604in}{0.677615in}}{\pgfqpoint{2.350190in}{0.679929in}}{\pgfqpoint{2.354309in}{0.684047in}}%
\pgfpathcurveto{\pgfqpoint{2.358427in}{0.688165in}}{\pgfqpoint{2.360741in}{0.693752in}}{\pgfqpoint{2.360741in}{0.699575in}}%
\pgfpathcurveto{\pgfqpoint{2.360741in}{0.705399in}}{\pgfqpoint{2.358427in}{0.710986in}}{\pgfqpoint{2.354309in}{0.715104in}}%
\pgfpathcurveto{\pgfqpoint{2.350190in}{0.719222in}}{\pgfqpoint{2.344604in}{0.721536in}}{\pgfqpoint{2.338780in}{0.721536in}}%
\pgfpathcurveto{\pgfqpoint{2.332956in}{0.721536in}}{\pgfqpoint{2.327370in}{0.719222in}}{\pgfqpoint{2.323252in}{0.715104in}}%
\pgfpathcurveto{\pgfqpoint{2.319134in}{0.710986in}}{\pgfqpoint{2.316820in}{0.705399in}}{\pgfqpoint{2.316820in}{0.699575in}}%
\pgfpathcurveto{\pgfqpoint{2.316820in}{0.693752in}}{\pgfqpoint{2.319134in}{0.688165in}}{\pgfqpoint{2.323252in}{0.684047in}}%
\pgfpathcurveto{\pgfqpoint{2.327370in}{0.679929in}}{\pgfqpoint{2.332956in}{0.677615in}}{\pgfqpoint{2.338780in}{0.677615in}}%
\pgfpathlineto{\pgfqpoint{2.338780in}{0.677615in}}%
\pgfpathclose%
\pgfusepath{stroke,fill}%
\end{pgfscope}%
\begin{pgfscope}%
\pgfpathrectangle{\pgfqpoint{0.640323in}{0.527436in}}{\pgfqpoint{9.687500in}{3.850000in}}%
\pgfusepath{clip}%
\pgfsetbuttcap%
\pgfsetroundjoin%
\definecolor{currentfill}{rgb}{0.239216,0.478431,0.992157}%
\pgfsetfillcolor{currentfill}%
\pgfsetfillopacity{0.500000}%
\pgfsetlinewidth{1.003750pt}%
\definecolor{currentstroke}{rgb}{0.239216,0.478431,0.992157}%
\pgfsetstrokecolor{currentstroke}%
\pgfsetstrokeopacity{0.500000}%
\pgfsetdash{{3.700000pt}{1.600000pt}}{0.000000pt}%
\pgfpathmoveto{\pgfqpoint{2.518511in}{0.795312in}}%
\pgfpathcurveto{\pgfqpoint{2.524335in}{0.795312in}}{\pgfqpoint{2.529921in}{0.797626in}}{\pgfqpoint{2.534040in}{0.801744in}}%
\pgfpathcurveto{\pgfqpoint{2.538158in}{0.805862in}}{\pgfqpoint{2.540472in}{0.811449in}}{\pgfqpoint{2.540472in}{0.817273in}}%
\pgfpathcurveto{\pgfqpoint{2.540472in}{0.823096in}}{\pgfqpoint{2.538158in}{0.828683in}}{\pgfqpoint{2.534040in}{0.832801in}}%
\pgfpathcurveto{\pgfqpoint{2.529921in}{0.836919in}}{\pgfqpoint{2.524335in}{0.839233in}}{\pgfqpoint{2.518511in}{0.839233in}}%
\pgfpathcurveto{\pgfqpoint{2.512687in}{0.839233in}}{\pgfqpoint{2.507101in}{0.836919in}}{\pgfqpoint{2.502983in}{0.832801in}}%
\pgfpathcurveto{\pgfqpoint{2.498865in}{0.828683in}}{\pgfqpoint{2.496551in}{0.823096in}}{\pgfqpoint{2.496551in}{0.817273in}}%
\pgfpathcurveto{\pgfqpoint{2.496551in}{0.811449in}}{\pgfqpoint{2.498865in}{0.805862in}}{\pgfqpoint{2.502983in}{0.801744in}}%
\pgfpathcurveto{\pgfqpoint{2.507101in}{0.797626in}}{\pgfqpoint{2.512687in}{0.795312in}}{\pgfqpoint{2.518511in}{0.795312in}}%
\pgfpathlineto{\pgfqpoint{2.518511in}{0.795312in}}%
\pgfpathclose%
\pgfusepath{stroke,fill}%
\end{pgfscope}%
\begin{pgfscope}%
\pgfpathrectangle{\pgfqpoint{0.640323in}{0.527436in}}{\pgfqpoint{9.687500in}{3.850000in}}%
\pgfusepath{clip}%
\pgfsetbuttcap%
\pgfsetroundjoin%
\definecolor{currentfill}{rgb}{0.239216,0.478431,0.992157}%
\pgfsetfillcolor{currentfill}%
\pgfsetfillopacity{0.500000}%
\pgfsetlinewidth{1.003750pt}%
\definecolor{currentstroke}{rgb}{0.239216,0.478431,0.992157}%
\pgfsetstrokecolor{currentstroke}%
\pgfsetstrokeopacity{0.500000}%
\pgfsetdash{{3.700000pt}{1.600000pt}}{0.000000pt}%
\pgfpathmoveto{\pgfqpoint{2.698242in}{1.383015in}}%
\pgfpathcurveto{\pgfqpoint{2.704066in}{1.383015in}}{\pgfqpoint{2.709652in}{1.385329in}}{\pgfqpoint{2.713771in}{1.389447in}}%
\pgfpathcurveto{\pgfqpoint{2.717889in}{1.393565in}}{\pgfqpoint{2.720203in}{1.399151in}}{\pgfqpoint{2.720203in}{1.404975in}}%
\pgfpathcurveto{\pgfqpoint{2.720203in}{1.410799in}}{\pgfqpoint{2.717889in}{1.416385in}}{\pgfqpoint{2.713771in}{1.420503in}}%
\pgfpathcurveto{\pgfqpoint{2.709652in}{1.424621in}}{\pgfqpoint{2.704066in}{1.426935in}}{\pgfqpoint{2.698242in}{1.426935in}}%
\pgfpathcurveto{\pgfqpoint{2.692418in}{1.426935in}}{\pgfqpoint{2.686832in}{1.424621in}}{\pgfqpoint{2.682714in}{1.420503in}}%
\pgfpathcurveto{\pgfqpoint{2.678596in}{1.416385in}}{\pgfqpoint{2.676282in}{1.410799in}}{\pgfqpoint{2.676282in}{1.404975in}}%
\pgfpathcurveto{\pgfqpoint{2.676282in}{1.399151in}}{\pgfqpoint{2.678596in}{1.393565in}}{\pgfqpoint{2.682714in}{1.389447in}}%
\pgfpathcurveto{\pgfqpoint{2.686832in}{1.385329in}}{\pgfqpoint{2.692418in}{1.383015in}}{\pgfqpoint{2.698242in}{1.383015in}}%
\pgfpathlineto{\pgfqpoint{2.698242in}{1.383015in}}%
\pgfpathclose%
\pgfusepath{stroke,fill}%
\end{pgfscope}%
\begin{pgfscope}%
\pgfpathrectangle{\pgfqpoint{0.640323in}{0.527436in}}{\pgfqpoint{9.687500in}{3.850000in}}%
\pgfusepath{clip}%
\pgfsetbuttcap%
\pgfsetroundjoin%
\definecolor{currentfill}{rgb}{0.239216,0.478431,0.992157}%
\pgfsetfillcolor{currentfill}%
\pgfsetfillopacity{0.500000}%
\pgfsetlinewidth{1.003750pt}%
\definecolor{currentstroke}{rgb}{0.239216,0.478431,0.992157}%
\pgfsetstrokecolor{currentstroke}%
\pgfsetstrokeopacity{0.500000}%
\pgfsetdash{{3.700000pt}{1.600000pt}}{0.000000pt}%
\pgfpathmoveto{\pgfqpoint{2.877973in}{1.817829in}}%
\pgfpathcurveto{\pgfqpoint{2.883797in}{1.817829in}}{\pgfqpoint{2.889383in}{1.820143in}}{\pgfqpoint{2.893501in}{1.824261in}}%
\pgfpathcurveto{\pgfqpoint{2.897620in}{1.828379in}}{\pgfqpoint{2.899934in}{1.833965in}}{\pgfqpoint{2.899934in}{1.839789in}}%
\pgfpathcurveto{\pgfqpoint{2.899934in}{1.845613in}}{\pgfqpoint{2.897620in}{1.851199in}}{\pgfqpoint{2.893501in}{1.855317in}}%
\pgfpathcurveto{\pgfqpoint{2.889383in}{1.859435in}}{\pgfqpoint{2.883797in}{1.861749in}}{\pgfqpoint{2.877973in}{1.861749in}}%
\pgfpathcurveto{\pgfqpoint{2.872149in}{1.861749in}}{\pgfqpoint{2.866563in}{1.859435in}}{\pgfqpoint{2.862445in}{1.855317in}}%
\pgfpathcurveto{\pgfqpoint{2.858327in}{1.851199in}}{\pgfqpoint{2.856013in}{1.845613in}}{\pgfqpoint{2.856013in}{1.839789in}}%
\pgfpathcurveto{\pgfqpoint{2.856013in}{1.833965in}}{\pgfqpoint{2.858327in}{1.828379in}}{\pgfqpoint{2.862445in}{1.824261in}}%
\pgfpathcurveto{\pgfqpoint{2.866563in}{1.820143in}}{\pgfqpoint{2.872149in}{1.817829in}}{\pgfqpoint{2.877973in}{1.817829in}}%
\pgfpathlineto{\pgfqpoint{2.877973in}{1.817829in}}%
\pgfpathclose%
\pgfusepath{stroke,fill}%
\end{pgfscope}%
\begin{pgfscope}%
\pgfpathrectangle{\pgfqpoint{0.640323in}{0.527436in}}{\pgfqpoint{9.687500in}{3.850000in}}%
\pgfusepath{clip}%
\pgfsetbuttcap%
\pgfsetroundjoin%
\definecolor{currentfill}{rgb}{0.239216,0.478431,0.992157}%
\pgfsetfillcolor{currentfill}%
\pgfsetfillopacity{0.500000}%
\pgfsetlinewidth{1.003750pt}%
\definecolor{currentstroke}{rgb}{0.239216,0.478431,0.992157}%
\pgfsetstrokecolor{currentstroke}%
\pgfsetstrokeopacity{0.500000}%
\pgfsetdash{{3.700000pt}{1.600000pt}}{0.000000pt}%
\pgfpathmoveto{\pgfqpoint{3.057704in}{2.062204in}}%
\pgfpathcurveto{\pgfqpoint{3.063528in}{2.062204in}}{\pgfqpoint{3.069114in}{2.064518in}}{\pgfqpoint{3.073232in}{2.068636in}}%
\pgfpathcurveto{\pgfqpoint{3.077351in}{2.072755in}}{\pgfqpoint{3.079664in}{2.078341in}}{\pgfqpoint{3.079664in}{2.084165in}}%
\pgfpathcurveto{\pgfqpoint{3.079664in}{2.089989in}}{\pgfqpoint{3.077351in}{2.095575in}}{\pgfqpoint{3.073232in}{2.099693in}}%
\pgfpathcurveto{\pgfqpoint{3.069114in}{2.103811in}}{\pgfqpoint{3.063528in}{2.106125in}}{\pgfqpoint{3.057704in}{2.106125in}}%
\pgfpathcurveto{\pgfqpoint{3.051880in}{2.106125in}}{\pgfqpoint{3.046294in}{2.103811in}}{\pgfqpoint{3.042176in}{2.099693in}}%
\pgfpathcurveto{\pgfqpoint{3.038058in}{2.095575in}}{\pgfqpoint{3.035744in}{2.089989in}}{\pgfqpoint{3.035744in}{2.084165in}}%
\pgfpathcurveto{\pgfqpoint{3.035744in}{2.078341in}}{\pgfqpoint{3.038058in}{2.072755in}}{\pgfqpoint{3.042176in}{2.068636in}}%
\pgfpathcurveto{\pgfqpoint{3.046294in}{2.064518in}}{\pgfqpoint{3.051880in}{2.062204in}}{\pgfqpoint{3.057704in}{2.062204in}}%
\pgfpathlineto{\pgfqpoint{3.057704in}{2.062204in}}%
\pgfpathclose%
\pgfusepath{stroke,fill}%
\end{pgfscope}%
\begin{pgfscope}%
\pgfpathrectangle{\pgfqpoint{0.640323in}{0.527436in}}{\pgfqpoint{9.687500in}{3.850000in}}%
\pgfusepath{clip}%
\pgfsetbuttcap%
\pgfsetroundjoin%
\definecolor{currentfill}{rgb}{0.239216,0.478431,0.992157}%
\pgfsetfillcolor{currentfill}%
\pgfsetfillopacity{0.500000}%
\pgfsetlinewidth{1.003750pt}%
\definecolor{currentstroke}{rgb}{0.239216,0.478431,0.992157}%
\pgfsetstrokecolor{currentstroke}%
\pgfsetstrokeopacity{0.500000}%
\pgfsetdash{{3.700000pt}{1.600000pt}}{0.000000pt}%
\pgfpathmoveto{\pgfqpoint{3.237435in}{2.282573in}}%
\pgfpathcurveto{\pgfqpoint{3.243259in}{2.282573in}}{\pgfqpoint{3.248845in}{2.284887in}}{\pgfqpoint{3.252963in}{2.289005in}}%
\pgfpathcurveto{\pgfqpoint{3.257082in}{2.293124in}}{\pgfqpoint{3.259395in}{2.298710in}}{\pgfqpoint{3.259395in}{2.304534in}}%
\pgfpathcurveto{\pgfqpoint{3.259395in}{2.310358in}}{\pgfqpoint{3.257082in}{2.315944in}}{\pgfqpoint{3.252963in}{2.320062in}}%
\pgfpathcurveto{\pgfqpoint{3.248845in}{2.324180in}}{\pgfqpoint{3.243259in}{2.326494in}}{\pgfqpoint{3.237435in}{2.326494in}}%
\pgfpathcurveto{\pgfqpoint{3.231611in}{2.326494in}}{\pgfqpoint{3.226025in}{2.324180in}}{\pgfqpoint{3.221907in}{2.320062in}}%
\pgfpathcurveto{\pgfqpoint{3.217789in}{2.315944in}}{\pgfqpoint{3.215475in}{2.310358in}}{\pgfqpoint{3.215475in}{2.304534in}}%
\pgfpathcurveto{\pgfqpoint{3.215475in}{2.298710in}}{\pgfqpoint{3.217789in}{2.293124in}}{\pgfqpoint{3.221907in}{2.289005in}}%
\pgfpathcurveto{\pgfqpoint{3.226025in}{2.284887in}}{\pgfqpoint{3.231611in}{2.282573in}}{\pgfqpoint{3.237435in}{2.282573in}}%
\pgfpathlineto{\pgfqpoint{3.237435in}{2.282573in}}%
\pgfpathclose%
\pgfusepath{stroke,fill}%
\end{pgfscope}%
\begin{pgfscope}%
\pgfpathrectangle{\pgfqpoint{0.640323in}{0.527436in}}{\pgfqpoint{9.687500in}{3.850000in}}%
\pgfusepath{clip}%
\pgfsetbuttcap%
\pgfsetroundjoin%
\definecolor{currentfill}{rgb}{0.239216,0.478431,0.992157}%
\pgfsetfillcolor{currentfill}%
\pgfsetfillopacity{0.500000}%
\pgfsetlinewidth{1.003750pt}%
\definecolor{currentstroke}{rgb}{0.239216,0.478431,0.992157}%
\pgfsetstrokecolor{currentstroke}%
\pgfsetstrokeopacity{0.500000}%
\pgfsetdash{{3.700000pt}{1.600000pt}}{0.000000pt}%
\pgfpathmoveto{\pgfqpoint{3.417166in}{2.451719in}}%
\pgfpathcurveto{\pgfqpoint{3.422990in}{2.451719in}}{\pgfqpoint{3.428576in}{2.454033in}}{\pgfqpoint{3.432694in}{2.458151in}}%
\pgfpathcurveto{\pgfqpoint{3.436813in}{2.462269in}}{\pgfqpoint{3.439126in}{2.467855in}}{\pgfqpoint{3.439126in}{2.473679in}}%
\pgfpathcurveto{\pgfqpoint{3.439126in}{2.479503in}}{\pgfqpoint{3.436813in}{2.485089in}}{\pgfqpoint{3.432694in}{2.489207in}}%
\pgfpathcurveto{\pgfqpoint{3.428576in}{2.493326in}}{\pgfqpoint{3.422990in}{2.495639in}}{\pgfqpoint{3.417166in}{2.495639in}}%
\pgfpathcurveto{\pgfqpoint{3.411342in}{2.495639in}}{\pgfqpoint{3.405756in}{2.493326in}}{\pgfqpoint{3.401638in}{2.489207in}}%
\pgfpathcurveto{\pgfqpoint{3.397520in}{2.485089in}}{\pgfqpoint{3.395206in}{2.479503in}}{\pgfqpoint{3.395206in}{2.473679in}}%
\pgfpathcurveto{\pgfqpoint{3.395206in}{2.467855in}}{\pgfqpoint{3.397520in}{2.462269in}}{\pgfqpoint{3.401638in}{2.458151in}}%
\pgfpathcurveto{\pgfqpoint{3.405756in}{2.454033in}}{\pgfqpoint{3.411342in}{2.451719in}}{\pgfqpoint{3.417166in}{2.451719in}}%
\pgfpathlineto{\pgfqpoint{3.417166in}{2.451719in}}%
\pgfpathclose%
\pgfusepath{stroke,fill}%
\end{pgfscope}%
\begin{pgfscope}%
\pgfpathrectangle{\pgfqpoint{0.640323in}{0.527436in}}{\pgfqpoint{9.687500in}{3.850000in}}%
\pgfusepath{clip}%
\pgfsetbuttcap%
\pgfsetroundjoin%
\definecolor{currentfill}{rgb}{0.239216,0.478431,0.992157}%
\pgfsetfillcolor{currentfill}%
\pgfsetfillopacity{0.500000}%
\pgfsetlinewidth{1.003750pt}%
\definecolor{currentstroke}{rgb}{0.239216,0.478431,0.992157}%
\pgfsetstrokecolor{currentstroke}%
\pgfsetstrokeopacity{0.500000}%
\pgfsetdash{{3.700000pt}{1.600000pt}}{0.000000pt}%
\pgfpathmoveto{\pgfqpoint{3.596897in}{2.594715in}}%
\pgfpathcurveto{\pgfqpoint{3.602721in}{2.594715in}}{\pgfqpoint{3.608307in}{2.597029in}}{\pgfqpoint{3.612425in}{2.601147in}}%
\pgfpathcurveto{\pgfqpoint{3.616544in}{2.605265in}}{\pgfqpoint{3.618857in}{2.610852in}}{\pgfqpoint{3.618857in}{2.616676in}}%
\pgfpathcurveto{\pgfqpoint{3.618857in}{2.622500in}}{\pgfqpoint{3.616544in}{2.628086in}}{\pgfqpoint{3.612425in}{2.632204in}}%
\pgfpathcurveto{\pgfqpoint{3.608307in}{2.636322in}}{\pgfqpoint{3.602721in}{2.638636in}}{\pgfqpoint{3.596897in}{2.638636in}}%
\pgfpathcurveto{\pgfqpoint{3.591073in}{2.638636in}}{\pgfqpoint{3.585487in}{2.636322in}}{\pgfqpoint{3.581369in}{2.632204in}}%
\pgfpathcurveto{\pgfqpoint{3.577251in}{2.628086in}}{\pgfqpoint{3.574937in}{2.622500in}}{\pgfqpoint{3.574937in}{2.616676in}}%
\pgfpathcurveto{\pgfqpoint{3.574937in}{2.610852in}}{\pgfqpoint{3.577251in}{2.605265in}}{\pgfqpoint{3.581369in}{2.601147in}}%
\pgfpathcurveto{\pgfqpoint{3.585487in}{2.597029in}}{\pgfqpoint{3.591073in}{2.594715in}}{\pgfqpoint{3.596897in}{2.594715in}}%
\pgfpathlineto{\pgfqpoint{3.596897in}{2.594715in}}%
\pgfpathclose%
\pgfusepath{stroke,fill}%
\end{pgfscope}%
\begin{pgfscope}%
\pgfpathrectangle{\pgfqpoint{0.640323in}{0.527436in}}{\pgfqpoint{9.687500in}{3.850000in}}%
\pgfusepath{clip}%
\pgfsetbuttcap%
\pgfsetroundjoin%
\definecolor{currentfill}{rgb}{0.239216,0.478431,0.992157}%
\pgfsetfillcolor{currentfill}%
\pgfsetfillopacity{0.500000}%
\pgfsetlinewidth{1.003750pt}%
\definecolor{currentstroke}{rgb}{0.239216,0.478431,0.992157}%
\pgfsetstrokecolor{currentstroke}%
\pgfsetstrokeopacity{0.500000}%
\pgfsetdash{{3.700000pt}{1.600000pt}}{0.000000pt}%
\pgfpathmoveto{\pgfqpoint{3.776628in}{2.731452in}}%
\pgfpathcurveto{\pgfqpoint{3.782452in}{2.731452in}}{\pgfqpoint{3.788038in}{2.733766in}}{\pgfqpoint{3.792156in}{2.737884in}}%
\pgfpathcurveto{\pgfqpoint{3.796275in}{2.742003in}}{\pgfqpoint{3.798588in}{2.747589in}}{\pgfqpoint{3.798588in}{2.753413in}}%
\pgfpathcurveto{\pgfqpoint{3.798588in}{2.759237in}}{\pgfqpoint{3.796275in}{2.764823in}}{\pgfqpoint{3.792156in}{2.768941in}}%
\pgfpathcurveto{\pgfqpoint{3.788038in}{2.773059in}}{\pgfqpoint{3.782452in}{2.775373in}}{\pgfqpoint{3.776628in}{2.775373in}}%
\pgfpathcurveto{\pgfqpoint{3.770804in}{2.775373in}}{\pgfqpoint{3.765218in}{2.773059in}}{\pgfqpoint{3.761100in}{2.768941in}}%
\pgfpathcurveto{\pgfqpoint{3.756982in}{2.764823in}}{\pgfqpoint{3.754668in}{2.759237in}}{\pgfqpoint{3.754668in}{2.753413in}}%
\pgfpathcurveto{\pgfqpoint{3.754668in}{2.747589in}}{\pgfqpoint{3.756982in}{2.742003in}}{\pgfqpoint{3.761100in}{2.737884in}}%
\pgfpathcurveto{\pgfqpoint{3.765218in}{2.733766in}}{\pgfqpoint{3.770804in}{2.731452in}}{\pgfqpoint{3.776628in}{2.731452in}}%
\pgfpathlineto{\pgfqpoint{3.776628in}{2.731452in}}%
\pgfpathclose%
\pgfusepath{stroke,fill}%
\end{pgfscope}%
\begin{pgfscope}%
\pgfpathrectangle{\pgfqpoint{0.640323in}{0.527436in}}{\pgfqpoint{9.687500in}{3.850000in}}%
\pgfusepath{clip}%
\pgfsetbuttcap%
\pgfsetroundjoin%
\definecolor{currentfill}{rgb}{0.239216,0.478431,0.992157}%
\pgfsetfillcolor{currentfill}%
\pgfsetfillopacity{0.500000}%
\pgfsetlinewidth{1.003750pt}%
\definecolor{currentstroke}{rgb}{0.239216,0.478431,0.992157}%
\pgfsetstrokecolor{currentstroke}%
\pgfsetstrokeopacity{0.500000}%
\pgfsetdash{{3.700000pt}{1.600000pt}}{0.000000pt}%
\pgfpathmoveto{\pgfqpoint{3.956359in}{2.844940in}}%
\pgfpathcurveto{\pgfqpoint{3.962183in}{2.844940in}}{\pgfqpoint{3.967769in}{2.847254in}}{\pgfqpoint{3.971887in}{2.851373in}}%
\pgfpathcurveto{\pgfqpoint{3.976006in}{2.855491in}}{\pgfqpoint{3.978319in}{2.861077in}}{\pgfqpoint{3.978319in}{2.866901in}}%
\pgfpathcurveto{\pgfqpoint{3.978319in}{2.872725in}}{\pgfqpoint{3.976006in}{2.878311in}}{\pgfqpoint{3.971887in}{2.882429in}}%
\pgfpathcurveto{\pgfqpoint{3.967769in}{2.886547in}}{\pgfqpoint{3.962183in}{2.888861in}}{\pgfqpoint{3.956359in}{2.888861in}}%
\pgfpathcurveto{\pgfqpoint{3.950535in}{2.888861in}}{\pgfqpoint{3.944949in}{2.886547in}}{\pgfqpoint{3.940831in}{2.882429in}}%
\pgfpathcurveto{\pgfqpoint{3.936713in}{2.878311in}}{\pgfqpoint{3.934399in}{2.872725in}}{\pgfqpoint{3.934399in}{2.866901in}}%
\pgfpathcurveto{\pgfqpoint{3.934399in}{2.861077in}}{\pgfqpoint{3.936713in}{2.855491in}}{\pgfqpoint{3.940831in}{2.851373in}}%
\pgfpathcurveto{\pgfqpoint{3.944949in}{2.847254in}}{\pgfqpoint{3.950535in}{2.844940in}}{\pgfqpoint{3.956359in}{2.844940in}}%
\pgfpathlineto{\pgfqpoint{3.956359in}{2.844940in}}%
\pgfpathclose%
\pgfusepath{stroke,fill}%
\end{pgfscope}%
\begin{pgfscope}%
\pgfpathrectangle{\pgfqpoint{0.640323in}{0.527436in}}{\pgfqpoint{9.687500in}{3.850000in}}%
\pgfusepath{clip}%
\pgfsetbuttcap%
\pgfsetroundjoin%
\definecolor{currentfill}{rgb}{0.239216,0.478431,0.992157}%
\pgfsetfillcolor{currentfill}%
\pgfsetfillopacity{0.500000}%
\pgfsetlinewidth{1.003750pt}%
\definecolor{currentstroke}{rgb}{0.239216,0.478431,0.992157}%
\pgfsetstrokecolor{currentstroke}%
\pgfsetstrokeopacity{0.500000}%
\pgfsetdash{{3.700000pt}{1.600000pt}}{0.000000pt}%
\pgfpathmoveto{\pgfqpoint{4.136090in}{2.936043in}}%
\pgfpathcurveto{\pgfqpoint{4.141914in}{2.936043in}}{\pgfqpoint{4.147500in}{2.938357in}}{\pgfqpoint{4.151618in}{2.942475in}}%
\pgfpathcurveto{\pgfqpoint{4.155737in}{2.946593in}}{\pgfqpoint{4.158050in}{2.952179in}}{\pgfqpoint{4.158050in}{2.958003in}}%
\pgfpathcurveto{\pgfqpoint{4.158050in}{2.963827in}}{\pgfqpoint{4.155737in}{2.969413in}}{\pgfqpoint{4.151618in}{2.973531in}}%
\pgfpathcurveto{\pgfqpoint{4.147500in}{2.977649in}}{\pgfqpoint{4.141914in}{2.979963in}}{\pgfqpoint{4.136090in}{2.979963in}}%
\pgfpathcurveto{\pgfqpoint{4.130266in}{2.979963in}}{\pgfqpoint{4.124680in}{2.977649in}}{\pgfqpoint{4.120562in}{2.973531in}}%
\pgfpathcurveto{\pgfqpoint{4.116444in}{2.969413in}}{\pgfqpoint{4.114130in}{2.963827in}}{\pgfqpoint{4.114130in}{2.958003in}}%
\pgfpathcurveto{\pgfqpoint{4.114130in}{2.952179in}}{\pgfqpoint{4.116444in}{2.946593in}}{\pgfqpoint{4.120562in}{2.942475in}}%
\pgfpathcurveto{\pgfqpoint{4.124680in}{2.938357in}}{\pgfqpoint{4.130266in}{2.936043in}}{\pgfqpoint{4.136090in}{2.936043in}}%
\pgfpathlineto{\pgfqpoint{4.136090in}{2.936043in}}%
\pgfpathclose%
\pgfusepath{stroke,fill}%
\end{pgfscope}%
\begin{pgfscope}%
\pgfpathrectangle{\pgfqpoint{0.640323in}{0.527436in}}{\pgfqpoint{9.687500in}{3.850000in}}%
\pgfusepath{clip}%
\pgfsetbuttcap%
\pgfsetroundjoin%
\definecolor{currentfill}{rgb}{0.239216,0.478431,0.992157}%
\pgfsetfillcolor{currentfill}%
\pgfsetfillopacity{0.500000}%
\pgfsetlinewidth{1.003750pt}%
\definecolor{currentstroke}{rgb}{0.239216,0.478431,0.992157}%
\pgfsetstrokecolor{currentstroke}%
\pgfsetstrokeopacity{0.500000}%
\pgfsetdash{{3.700000pt}{1.600000pt}}{0.000000pt}%
\pgfpathmoveto{\pgfqpoint{4.315821in}{3.029796in}}%
\pgfpathcurveto{\pgfqpoint{4.321645in}{3.029796in}}{\pgfqpoint{4.327231in}{3.032110in}}{\pgfqpoint{4.331349in}{3.036228in}}%
\pgfpathcurveto{\pgfqpoint{4.335467in}{3.040347in}}{\pgfqpoint{4.337781in}{3.045933in}}{\pgfqpoint{4.337781in}{3.051757in}}%
\pgfpathcurveto{\pgfqpoint{4.337781in}{3.057581in}}{\pgfqpoint{4.335467in}{3.063167in}}{\pgfqpoint{4.331349in}{3.067285in}}%
\pgfpathcurveto{\pgfqpoint{4.327231in}{3.071403in}}{\pgfqpoint{4.321645in}{3.073717in}}{\pgfqpoint{4.315821in}{3.073717in}}%
\pgfpathcurveto{\pgfqpoint{4.309997in}{3.073717in}}{\pgfqpoint{4.304411in}{3.071403in}}{\pgfqpoint{4.300293in}{3.067285in}}%
\pgfpathcurveto{\pgfqpoint{4.296175in}{3.063167in}}{\pgfqpoint{4.293861in}{3.057581in}}{\pgfqpoint{4.293861in}{3.051757in}}%
\pgfpathcurveto{\pgfqpoint{4.293861in}{3.045933in}}{\pgfqpoint{4.296175in}{3.040347in}}{\pgfqpoint{4.300293in}{3.036228in}}%
\pgfpathcurveto{\pgfqpoint{4.304411in}{3.032110in}}{\pgfqpoint{4.309997in}{3.029796in}}{\pgfqpoint{4.315821in}{3.029796in}}%
\pgfpathlineto{\pgfqpoint{4.315821in}{3.029796in}}%
\pgfpathclose%
\pgfusepath{stroke,fill}%
\end{pgfscope}%
\begin{pgfscope}%
\pgfpathrectangle{\pgfqpoint{0.640323in}{0.527436in}}{\pgfqpoint{9.687500in}{3.850000in}}%
\pgfusepath{clip}%
\pgfsetbuttcap%
\pgfsetroundjoin%
\definecolor{currentfill}{rgb}{0.239216,0.478431,0.992157}%
\pgfsetfillcolor{currentfill}%
\pgfsetfillopacity{0.500000}%
\pgfsetlinewidth{1.003750pt}%
\definecolor{currentstroke}{rgb}{0.239216,0.478431,0.992157}%
\pgfsetstrokecolor{currentstroke}%
\pgfsetstrokeopacity{0.500000}%
\pgfsetdash{{3.700000pt}{1.600000pt}}{0.000000pt}%
\pgfpathmoveto{\pgfqpoint{4.495552in}{3.100301in}}%
\pgfpathcurveto{\pgfqpoint{4.501376in}{3.100301in}}{\pgfqpoint{4.506962in}{3.102615in}}{\pgfqpoint{4.511080in}{3.106733in}}%
\pgfpathcurveto{\pgfqpoint{4.515198in}{3.110851in}}{\pgfqpoint{4.517512in}{3.116437in}}{\pgfqpoint{4.517512in}{3.122261in}}%
\pgfpathcurveto{\pgfqpoint{4.517512in}{3.128085in}}{\pgfqpoint{4.515198in}{3.133671in}}{\pgfqpoint{4.511080in}{3.137790in}}%
\pgfpathcurveto{\pgfqpoint{4.506962in}{3.141908in}}{\pgfqpoint{4.501376in}{3.144222in}}{\pgfqpoint{4.495552in}{3.144222in}}%
\pgfpathcurveto{\pgfqpoint{4.489728in}{3.144222in}}{\pgfqpoint{4.484142in}{3.141908in}}{\pgfqpoint{4.480024in}{3.137790in}}%
\pgfpathcurveto{\pgfqpoint{4.475906in}{3.133671in}}{\pgfqpoint{4.473592in}{3.128085in}}{\pgfqpoint{4.473592in}{3.122261in}}%
\pgfpathcurveto{\pgfqpoint{4.473592in}{3.116437in}}{\pgfqpoint{4.475906in}{3.110851in}}{\pgfqpoint{4.480024in}{3.106733in}}%
\pgfpathcurveto{\pgfqpoint{4.484142in}{3.102615in}}{\pgfqpoint{4.489728in}{3.100301in}}{\pgfqpoint{4.495552in}{3.100301in}}%
\pgfpathlineto{\pgfqpoint{4.495552in}{3.100301in}}%
\pgfpathclose%
\pgfusepath{stroke,fill}%
\end{pgfscope}%
\begin{pgfscope}%
\pgfpathrectangle{\pgfqpoint{0.640323in}{0.527436in}}{\pgfqpoint{9.687500in}{3.850000in}}%
\pgfusepath{clip}%
\pgfsetbuttcap%
\pgfsetroundjoin%
\definecolor{currentfill}{rgb}{0.239216,0.478431,0.992157}%
\pgfsetfillcolor{currentfill}%
\pgfsetfillopacity{0.500000}%
\pgfsetlinewidth{1.003750pt}%
\definecolor{currentstroke}{rgb}{0.239216,0.478431,0.992157}%
\pgfsetstrokecolor{currentstroke}%
\pgfsetstrokeopacity{0.500000}%
\pgfsetdash{{3.700000pt}{1.600000pt}}{0.000000pt}%
\pgfpathmoveto{\pgfqpoint{4.675283in}{3.165366in}}%
\pgfpathcurveto{\pgfqpoint{4.681107in}{3.165366in}}{\pgfqpoint{4.686693in}{3.167680in}}{\pgfqpoint{4.690811in}{3.171798in}}%
\pgfpathcurveto{\pgfqpoint{4.694929in}{3.175916in}}{\pgfqpoint{4.697243in}{3.181502in}}{\pgfqpoint{4.697243in}{3.187326in}}%
\pgfpathcurveto{\pgfqpoint{4.697243in}{3.193150in}}{\pgfqpoint{4.694929in}{3.198736in}}{\pgfqpoint{4.690811in}{3.202855in}}%
\pgfpathcurveto{\pgfqpoint{4.686693in}{3.206973in}}{\pgfqpoint{4.681107in}{3.209287in}}{\pgfqpoint{4.675283in}{3.209287in}}%
\pgfpathcurveto{\pgfqpoint{4.669459in}{3.209287in}}{\pgfqpoint{4.663873in}{3.206973in}}{\pgfqpoint{4.659755in}{3.202855in}}%
\pgfpathcurveto{\pgfqpoint{4.655637in}{3.198736in}}{\pgfqpoint{4.653323in}{3.193150in}}{\pgfqpoint{4.653323in}{3.187326in}}%
\pgfpathcurveto{\pgfqpoint{4.653323in}{3.181502in}}{\pgfqpoint{4.655637in}{3.175916in}}{\pgfqpoint{4.659755in}{3.171798in}}%
\pgfpathcurveto{\pgfqpoint{4.663873in}{3.167680in}}{\pgfqpoint{4.669459in}{3.165366in}}{\pgfqpoint{4.675283in}{3.165366in}}%
\pgfpathlineto{\pgfqpoint{4.675283in}{3.165366in}}%
\pgfpathclose%
\pgfusepath{stroke,fill}%
\end{pgfscope}%
\begin{pgfscope}%
\pgfpathrectangle{\pgfqpoint{0.640323in}{0.527436in}}{\pgfqpoint{9.687500in}{3.850000in}}%
\pgfusepath{clip}%
\pgfsetbuttcap%
\pgfsetroundjoin%
\definecolor{currentfill}{rgb}{0.239216,0.478431,0.992157}%
\pgfsetfillcolor{currentfill}%
\pgfsetfillopacity{0.500000}%
\pgfsetlinewidth{1.003750pt}%
\definecolor{currentstroke}{rgb}{0.239216,0.478431,0.992157}%
\pgfsetstrokecolor{currentstroke}%
\pgfsetstrokeopacity{0.500000}%
\pgfsetdash{{3.700000pt}{1.600000pt}}{0.000000pt}%
\pgfpathmoveto{\pgfqpoint{4.855014in}{3.224333in}}%
\pgfpathcurveto{\pgfqpoint{4.860838in}{3.224333in}}{\pgfqpoint{4.866424in}{3.226647in}}{\pgfqpoint{4.870542in}{3.230765in}}%
\pgfpathcurveto{\pgfqpoint{4.874660in}{3.234883in}}{\pgfqpoint{4.876974in}{3.240469in}}{\pgfqpoint{4.876974in}{3.246293in}}%
\pgfpathcurveto{\pgfqpoint{4.876974in}{3.252117in}}{\pgfqpoint{4.874660in}{3.257704in}}{\pgfqpoint{4.870542in}{3.261822in}}%
\pgfpathcurveto{\pgfqpoint{4.866424in}{3.265940in}}{\pgfqpoint{4.860838in}{3.268254in}}{\pgfqpoint{4.855014in}{3.268254in}}%
\pgfpathcurveto{\pgfqpoint{4.849190in}{3.268254in}}{\pgfqpoint{4.843604in}{3.265940in}}{\pgfqpoint{4.839486in}{3.261822in}}%
\pgfpathcurveto{\pgfqpoint{4.835368in}{3.257704in}}{\pgfqpoint{4.833054in}{3.252117in}}{\pgfqpoint{4.833054in}{3.246293in}}%
\pgfpathcurveto{\pgfqpoint{4.833054in}{3.240469in}}{\pgfqpoint{4.835368in}{3.234883in}}{\pgfqpoint{4.839486in}{3.230765in}}%
\pgfpathcurveto{\pgfqpoint{4.843604in}{3.226647in}}{\pgfqpoint{4.849190in}{3.224333in}}{\pgfqpoint{4.855014in}{3.224333in}}%
\pgfpathlineto{\pgfqpoint{4.855014in}{3.224333in}}%
\pgfpathclose%
\pgfusepath{stroke,fill}%
\end{pgfscope}%
\begin{pgfscope}%
\pgfpathrectangle{\pgfqpoint{0.640323in}{0.527436in}}{\pgfqpoint{9.687500in}{3.850000in}}%
\pgfusepath{clip}%
\pgfsetbuttcap%
\pgfsetroundjoin%
\definecolor{currentfill}{rgb}{0.239216,0.478431,0.992157}%
\pgfsetfillcolor{currentfill}%
\pgfsetfillopacity{0.500000}%
\pgfsetlinewidth{1.003750pt}%
\definecolor{currentstroke}{rgb}{0.239216,0.478431,0.992157}%
\pgfsetstrokecolor{currentstroke}%
\pgfsetstrokeopacity{0.500000}%
\pgfsetdash{{3.700000pt}{1.600000pt}}{0.000000pt}%
\pgfpathmoveto{\pgfqpoint{5.034745in}{3.294000in}}%
\pgfpathcurveto{\pgfqpoint{5.040569in}{3.294000in}}{\pgfqpoint{5.046155in}{3.296313in}}{\pgfqpoint{5.050273in}{3.300432in}}%
\pgfpathcurveto{\pgfqpoint{5.054391in}{3.304550in}}{\pgfqpoint{5.056705in}{3.310136in}}{\pgfqpoint{5.056705in}{3.315960in}}%
\pgfpathcurveto{\pgfqpoint{5.056705in}{3.321784in}}{\pgfqpoint{5.054391in}{3.327370in}}{\pgfqpoint{5.050273in}{3.331488in}}%
\pgfpathcurveto{\pgfqpoint{5.046155in}{3.335606in}}{\pgfqpoint{5.040569in}{3.337920in}}{\pgfqpoint{5.034745in}{3.337920in}}%
\pgfpathcurveto{\pgfqpoint{5.028921in}{3.337920in}}{\pgfqpoint{5.023335in}{3.335606in}}{\pgfqpoint{5.019217in}{3.331488in}}%
\pgfpathcurveto{\pgfqpoint{5.015099in}{3.327370in}}{\pgfqpoint{5.012785in}{3.321784in}}{\pgfqpoint{5.012785in}{3.315960in}}%
\pgfpathcurveto{\pgfqpoint{5.012785in}{3.310136in}}{\pgfqpoint{5.015099in}{3.304550in}}{\pgfqpoint{5.019217in}{3.300432in}}%
\pgfpathcurveto{\pgfqpoint{5.023335in}{3.296313in}}{\pgfqpoint{5.028921in}{3.294000in}}{\pgfqpoint{5.034745in}{3.294000in}}%
\pgfpathlineto{\pgfqpoint{5.034745in}{3.294000in}}%
\pgfpathclose%
\pgfusepath{stroke,fill}%
\end{pgfscope}%
\begin{pgfscope}%
\pgfpathrectangle{\pgfqpoint{0.640323in}{0.527436in}}{\pgfqpoint{9.687500in}{3.850000in}}%
\pgfusepath{clip}%
\pgfsetbuttcap%
\pgfsetroundjoin%
\definecolor{currentfill}{rgb}{0.239216,0.478431,0.992157}%
\pgfsetfillcolor{currentfill}%
\pgfsetfillopacity{0.500000}%
\pgfsetlinewidth{1.003750pt}%
\definecolor{currentstroke}{rgb}{0.239216,0.478431,0.992157}%
\pgfsetstrokecolor{currentstroke}%
\pgfsetstrokeopacity{0.500000}%
\pgfsetdash{{3.700000pt}{1.600000pt}}{0.000000pt}%
\pgfpathmoveto{\pgfqpoint{5.214476in}{3.349377in}}%
\pgfpathcurveto{\pgfqpoint{5.220300in}{3.349377in}}{\pgfqpoint{5.225886in}{3.351691in}}{\pgfqpoint{5.230004in}{3.355809in}}%
\pgfpathcurveto{\pgfqpoint{5.234122in}{3.359928in}}{\pgfqpoint{5.236436in}{3.365514in}}{\pgfqpoint{5.236436in}{3.371338in}}%
\pgfpathcurveto{\pgfqpoint{5.236436in}{3.377162in}}{\pgfqpoint{5.234122in}{3.382748in}}{\pgfqpoint{5.230004in}{3.386866in}}%
\pgfpathcurveto{\pgfqpoint{5.225886in}{3.390984in}}{\pgfqpoint{5.220300in}{3.393298in}}{\pgfqpoint{5.214476in}{3.393298in}}%
\pgfpathcurveto{\pgfqpoint{5.208652in}{3.393298in}}{\pgfqpoint{5.203066in}{3.390984in}}{\pgfqpoint{5.198948in}{3.386866in}}%
\pgfpathcurveto{\pgfqpoint{5.194830in}{3.382748in}}{\pgfqpoint{5.192516in}{3.377162in}}{\pgfqpoint{5.192516in}{3.371338in}}%
\pgfpathcurveto{\pgfqpoint{5.192516in}{3.365514in}}{\pgfqpoint{5.194830in}{3.359928in}}{\pgfqpoint{5.198948in}{3.355809in}}%
\pgfpathcurveto{\pgfqpoint{5.203066in}{3.351691in}}{\pgfqpoint{5.208652in}{3.349377in}}{\pgfqpoint{5.214476in}{3.349377in}}%
\pgfpathlineto{\pgfqpoint{5.214476in}{3.349377in}}%
\pgfpathclose%
\pgfusepath{stroke,fill}%
\end{pgfscope}%
\begin{pgfscope}%
\pgfpathrectangle{\pgfqpoint{0.640323in}{0.527436in}}{\pgfqpoint{9.687500in}{3.850000in}}%
\pgfusepath{clip}%
\pgfsetbuttcap%
\pgfsetroundjoin%
\definecolor{currentfill}{rgb}{0.239216,0.478431,0.992157}%
\pgfsetfillcolor{currentfill}%
\pgfsetfillopacity{0.500000}%
\pgfsetlinewidth{1.003750pt}%
\definecolor{currentstroke}{rgb}{0.239216,0.478431,0.992157}%
\pgfsetstrokecolor{currentstroke}%
\pgfsetstrokeopacity{0.500000}%
\pgfsetdash{{3.700000pt}{1.600000pt}}{0.000000pt}%
\pgfpathmoveto{\pgfqpoint{5.394207in}{3.389585in}}%
\pgfpathcurveto{\pgfqpoint{5.400031in}{3.389585in}}{\pgfqpoint{5.405617in}{3.391899in}}{\pgfqpoint{5.409735in}{3.396017in}}%
\pgfpathcurveto{\pgfqpoint{5.413853in}{3.400135in}}{\pgfqpoint{5.416167in}{3.405721in}}{\pgfqpoint{5.416167in}{3.411545in}}%
\pgfpathcurveto{\pgfqpoint{5.416167in}{3.417369in}}{\pgfqpoint{5.413853in}{3.422955in}}{\pgfqpoint{5.409735in}{3.427074in}}%
\pgfpathcurveto{\pgfqpoint{5.405617in}{3.431192in}}{\pgfqpoint{5.400031in}{3.433506in}}{\pgfqpoint{5.394207in}{3.433506in}}%
\pgfpathcurveto{\pgfqpoint{5.388383in}{3.433506in}}{\pgfqpoint{5.382797in}{3.431192in}}{\pgfqpoint{5.378679in}{3.427074in}}%
\pgfpathcurveto{\pgfqpoint{5.374561in}{3.422955in}}{\pgfqpoint{5.372247in}{3.417369in}}{\pgfqpoint{5.372247in}{3.411545in}}%
\pgfpathcurveto{\pgfqpoint{5.372247in}{3.405721in}}{\pgfqpoint{5.374561in}{3.400135in}}{\pgfqpoint{5.378679in}{3.396017in}}%
\pgfpathcurveto{\pgfqpoint{5.382797in}{3.391899in}}{\pgfqpoint{5.388383in}{3.389585in}}{\pgfqpoint{5.394207in}{3.389585in}}%
\pgfpathlineto{\pgfqpoint{5.394207in}{3.389585in}}%
\pgfpathclose%
\pgfusepath{stroke,fill}%
\end{pgfscope}%
\begin{pgfscope}%
\pgfpathrectangle{\pgfqpoint{0.640323in}{0.527436in}}{\pgfqpoint{9.687500in}{3.850000in}}%
\pgfusepath{clip}%
\pgfsetbuttcap%
\pgfsetroundjoin%
\definecolor{currentfill}{rgb}{0.239216,0.478431,0.992157}%
\pgfsetfillcolor{currentfill}%
\pgfsetfillopacity{0.500000}%
\pgfsetlinewidth{1.003750pt}%
\definecolor{currentstroke}{rgb}{0.239216,0.478431,0.992157}%
\pgfsetstrokecolor{currentstroke}%
\pgfsetstrokeopacity{0.500000}%
\pgfsetdash{{3.700000pt}{1.600000pt}}{0.000000pt}%
\pgfpathmoveto{\pgfqpoint{5.573938in}{3.442610in}}%
\pgfpathcurveto{\pgfqpoint{5.579762in}{3.442610in}}{\pgfqpoint{5.585348in}{3.444923in}}{\pgfqpoint{5.589466in}{3.449042in}}%
\pgfpathcurveto{\pgfqpoint{5.593584in}{3.453160in}}{\pgfqpoint{5.595898in}{3.458746in}}{\pgfqpoint{5.595898in}{3.464570in}}%
\pgfpathcurveto{\pgfqpoint{5.595898in}{3.470394in}}{\pgfqpoint{5.593584in}{3.475980in}}{\pgfqpoint{5.589466in}{3.480098in}}%
\pgfpathcurveto{\pgfqpoint{5.585348in}{3.484216in}}{\pgfqpoint{5.579762in}{3.486530in}}{\pgfqpoint{5.573938in}{3.486530in}}%
\pgfpathcurveto{\pgfqpoint{5.568114in}{3.486530in}}{\pgfqpoint{5.562528in}{3.484216in}}{\pgfqpoint{5.558410in}{3.480098in}}%
\pgfpathcurveto{\pgfqpoint{5.554292in}{3.475980in}}{\pgfqpoint{5.551978in}{3.470394in}}{\pgfqpoint{5.551978in}{3.464570in}}%
\pgfpathcurveto{\pgfqpoint{5.551978in}{3.458746in}}{\pgfqpoint{5.554292in}{3.453160in}}{\pgfqpoint{5.558410in}{3.449042in}}%
\pgfpathcurveto{\pgfqpoint{5.562528in}{3.444923in}}{\pgfqpoint{5.568114in}{3.442610in}}{\pgfqpoint{5.573938in}{3.442610in}}%
\pgfpathlineto{\pgfqpoint{5.573938in}{3.442610in}}%
\pgfpathclose%
\pgfusepath{stroke,fill}%
\end{pgfscope}%
\begin{pgfscope}%
\pgfpathrectangle{\pgfqpoint{0.640323in}{0.527436in}}{\pgfqpoint{9.687500in}{3.850000in}}%
\pgfusepath{clip}%
\pgfsetbuttcap%
\pgfsetroundjoin%
\definecolor{currentfill}{rgb}{0.239216,0.478431,0.992157}%
\pgfsetfillcolor{currentfill}%
\pgfsetfillopacity{0.500000}%
\pgfsetlinewidth{1.003750pt}%
\definecolor{currentstroke}{rgb}{0.239216,0.478431,0.992157}%
\pgfsetstrokecolor{currentstroke}%
\pgfsetstrokeopacity{0.500000}%
\pgfsetdash{{3.700000pt}{1.600000pt}}{0.000000pt}%
\pgfpathmoveto{\pgfqpoint{5.753669in}{3.477849in}}%
\pgfpathcurveto{\pgfqpoint{5.759493in}{3.477849in}}{\pgfqpoint{5.765079in}{3.480163in}}{\pgfqpoint{5.769197in}{3.484281in}}%
\pgfpathcurveto{\pgfqpoint{5.773315in}{3.488400in}}{\pgfqpoint{5.775629in}{3.493986in}}{\pgfqpoint{5.775629in}{3.499810in}}%
\pgfpathcurveto{\pgfqpoint{5.775629in}{3.505634in}}{\pgfqpoint{5.773315in}{3.511220in}}{\pgfqpoint{5.769197in}{3.515338in}}%
\pgfpathcurveto{\pgfqpoint{5.765079in}{3.519456in}}{\pgfqpoint{5.759493in}{3.521770in}}{\pgfqpoint{5.753669in}{3.521770in}}%
\pgfpathcurveto{\pgfqpoint{5.747845in}{3.521770in}}{\pgfqpoint{5.742259in}{3.519456in}}{\pgfqpoint{5.738141in}{3.515338in}}%
\pgfpathcurveto{\pgfqpoint{5.734023in}{3.511220in}}{\pgfqpoint{5.731709in}{3.505634in}}{\pgfqpoint{5.731709in}{3.499810in}}%
\pgfpathcurveto{\pgfqpoint{5.731709in}{3.493986in}}{\pgfqpoint{5.734023in}{3.488400in}}{\pgfqpoint{5.738141in}{3.484281in}}%
\pgfpathcurveto{\pgfqpoint{5.742259in}{3.480163in}}{\pgfqpoint{5.747845in}{3.477849in}}{\pgfqpoint{5.753669in}{3.477849in}}%
\pgfpathlineto{\pgfqpoint{5.753669in}{3.477849in}}%
\pgfpathclose%
\pgfusepath{stroke,fill}%
\end{pgfscope}%
\begin{pgfscope}%
\pgfpathrectangle{\pgfqpoint{0.640323in}{0.527436in}}{\pgfqpoint{9.687500in}{3.850000in}}%
\pgfusepath{clip}%
\pgfsetbuttcap%
\pgfsetroundjoin%
\definecolor{currentfill}{rgb}{0.239216,0.478431,0.992157}%
\pgfsetfillcolor{currentfill}%
\pgfsetfillopacity{0.500000}%
\pgfsetlinewidth{1.003750pt}%
\definecolor{currentstroke}{rgb}{0.239216,0.478431,0.992157}%
\pgfsetstrokecolor{currentstroke}%
\pgfsetstrokeopacity{0.500000}%
\pgfsetdash{{3.700000pt}{1.600000pt}}{0.000000pt}%
\pgfpathmoveto{\pgfqpoint{5.933400in}{3.521758in}}%
\pgfpathcurveto{\pgfqpoint{5.939224in}{3.521758in}}{\pgfqpoint{5.944810in}{3.524072in}}{\pgfqpoint{5.948928in}{3.528190in}}%
\pgfpathcurveto{\pgfqpoint{5.953046in}{3.532308in}}{\pgfqpoint{5.955360in}{3.537894in}}{\pgfqpoint{5.955360in}{3.543718in}}%
\pgfpathcurveto{\pgfqpoint{5.955360in}{3.549542in}}{\pgfqpoint{5.953046in}{3.555128in}}{\pgfqpoint{5.948928in}{3.559247in}}%
\pgfpathcurveto{\pgfqpoint{5.944810in}{3.563365in}}{\pgfqpoint{5.939224in}{3.565679in}}{\pgfqpoint{5.933400in}{3.565679in}}%
\pgfpathcurveto{\pgfqpoint{5.927576in}{3.565679in}}{\pgfqpoint{5.921990in}{3.563365in}}{\pgfqpoint{5.917872in}{3.559247in}}%
\pgfpathcurveto{\pgfqpoint{5.913754in}{3.555128in}}{\pgfqpoint{5.911440in}{3.549542in}}{\pgfqpoint{5.911440in}{3.543718in}}%
\pgfpathcurveto{\pgfqpoint{5.911440in}{3.537894in}}{\pgfqpoint{5.913754in}{3.532308in}}{\pgfqpoint{5.917872in}{3.528190in}}%
\pgfpathcurveto{\pgfqpoint{5.921990in}{3.524072in}}{\pgfqpoint{5.927576in}{3.521758in}}{\pgfqpoint{5.933400in}{3.521758in}}%
\pgfpathlineto{\pgfqpoint{5.933400in}{3.521758in}}%
\pgfpathclose%
\pgfusepath{stroke,fill}%
\end{pgfscope}%
\begin{pgfscope}%
\pgfpathrectangle{\pgfqpoint{0.640323in}{0.527436in}}{\pgfqpoint{9.687500in}{3.850000in}}%
\pgfusepath{clip}%
\pgfsetbuttcap%
\pgfsetroundjoin%
\definecolor{currentfill}{rgb}{0.239216,0.478431,0.992157}%
\pgfsetfillcolor{currentfill}%
\pgfsetfillopacity{0.500000}%
\pgfsetlinewidth{1.003750pt}%
\definecolor{currentstroke}{rgb}{0.239216,0.478431,0.992157}%
\pgfsetstrokecolor{currentstroke}%
\pgfsetstrokeopacity{0.500000}%
\pgfsetdash{{3.700000pt}{1.600000pt}}{0.000000pt}%
\pgfpathmoveto{\pgfqpoint{6.113131in}{3.551341in}}%
\pgfpathcurveto{\pgfqpoint{6.118955in}{3.551341in}}{\pgfqpoint{6.124541in}{3.553655in}}{\pgfqpoint{6.128659in}{3.557773in}}%
\pgfpathcurveto{\pgfqpoint{6.132777in}{3.561891in}}{\pgfqpoint{6.135091in}{3.567477in}}{\pgfqpoint{6.135091in}{3.573301in}}%
\pgfpathcurveto{\pgfqpoint{6.135091in}{3.579125in}}{\pgfqpoint{6.132777in}{3.584711in}}{\pgfqpoint{6.128659in}{3.588829in}}%
\pgfpathcurveto{\pgfqpoint{6.124541in}{3.592948in}}{\pgfqpoint{6.118955in}{3.595261in}}{\pgfqpoint{6.113131in}{3.595261in}}%
\pgfpathcurveto{\pgfqpoint{6.107307in}{3.595261in}}{\pgfqpoint{6.101721in}{3.592948in}}{\pgfqpoint{6.097603in}{3.588829in}}%
\pgfpathcurveto{\pgfqpoint{6.093485in}{3.584711in}}{\pgfqpoint{6.091171in}{3.579125in}}{\pgfqpoint{6.091171in}{3.573301in}}%
\pgfpathcurveto{\pgfqpoint{6.091171in}{3.567477in}}{\pgfqpoint{6.093485in}{3.561891in}}{\pgfqpoint{6.097603in}{3.557773in}}%
\pgfpathcurveto{\pgfqpoint{6.101721in}{3.553655in}}{\pgfqpoint{6.107307in}{3.551341in}}{\pgfqpoint{6.113131in}{3.551341in}}%
\pgfpathlineto{\pgfqpoint{6.113131in}{3.551341in}}%
\pgfpathclose%
\pgfusepath{stroke,fill}%
\end{pgfscope}%
\begin{pgfscope}%
\pgfpathrectangle{\pgfqpoint{0.640323in}{0.527436in}}{\pgfqpoint{9.687500in}{3.850000in}}%
\pgfusepath{clip}%
\pgfsetbuttcap%
\pgfsetroundjoin%
\definecolor{currentfill}{rgb}{0.239216,0.478431,0.992157}%
\pgfsetfillcolor{currentfill}%
\pgfsetfillopacity{0.500000}%
\pgfsetlinewidth{1.003750pt}%
\definecolor{currentstroke}{rgb}{0.239216,0.478431,0.992157}%
\pgfsetstrokecolor{currentstroke}%
\pgfsetstrokeopacity{0.500000}%
\pgfsetdash{{3.700000pt}{1.600000pt}}{0.000000pt}%
\pgfpathmoveto{\pgfqpoint{6.292862in}{3.598944in}}%
\pgfpathcurveto{\pgfqpoint{6.298686in}{3.598944in}}{\pgfqpoint{6.304272in}{3.601258in}}{\pgfqpoint{6.308390in}{3.605376in}}%
\pgfpathcurveto{\pgfqpoint{6.312508in}{3.609495in}}{\pgfqpoint{6.314822in}{3.615081in}}{\pgfqpoint{6.314822in}{3.620905in}}%
\pgfpathcurveto{\pgfqpoint{6.314822in}{3.626729in}}{\pgfqpoint{6.312508in}{3.632315in}}{\pgfqpoint{6.308390in}{3.636433in}}%
\pgfpathcurveto{\pgfqpoint{6.304272in}{3.640551in}}{\pgfqpoint{6.298686in}{3.642865in}}{\pgfqpoint{6.292862in}{3.642865in}}%
\pgfpathcurveto{\pgfqpoint{6.287038in}{3.642865in}}{\pgfqpoint{6.281452in}{3.640551in}}{\pgfqpoint{6.277334in}{3.636433in}}%
\pgfpathcurveto{\pgfqpoint{6.273216in}{3.632315in}}{\pgfqpoint{6.270902in}{3.626729in}}{\pgfqpoint{6.270902in}{3.620905in}}%
\pgfpathcurveto{\pgfqpoint{6.270902in}{3.615081in}}{\pgfqpoint{6.273216in}{3.609495in}}{\pgfqpoint{6.277334in}{3.605376in}}%
\pgfpathcurveto{\pgfqpoint{6.281452in}{3.601258in}}{\pgfqpoint{6.287038in}{3.598944in}}{\pgfqpoint{6.292862in}{3.598944in}}%
\pgfpathlineto{\pgfqpoint{6.292862in}{3.598944in}}%
\pgfpathclose%
\pgfusepath{stroke,fill}%
\end{pgfscope}%
\begin{pgfscope}%
\pgfpathrectangle{\pgfqpoint{0.640323in}{0.527436in}}{\pgfqpoint{9.687500in}{3.850000in}}%
\pgfusepath{clip}%
\pgfsetbuttcap%
\pgfsetroundjoin%
\definecolor{currentfill}{rgb}{0.239216,0.478431,0.992157}%
\pgfsetfillcolor{currentfill}%
\pgfsetfillopacity{0.500000}%
\pgfsetlinewidth{1.003750pt}%
\definecolor{currentstroke}{rgb}{0.239216,0.478431,0.992157}%
\pgfsetstrokecolor{currentstroke}%
\pgfsetstrokeopacity{0.500000}%
\pgfsetdash{{3.700000pt}{1.600000pt}}{0.000000pt}%
\pgfpathmoveto{\pgfqpoint{6.472593in}{3.627881in}}%
\pgfpathcurveto{\pgfqpoint{6.478417in}{3.627881in}}{\pgfqpoint{6.484003in}{3.630195in}}{\pgfqpoint{6.488121in}{3.634313in}}%
\pgfpathcurveto{\pgfqpoint{6.492239in}{3.638432in}}{\pgfqpoint{6.494553in}{3.644018in}}{\pgfqpoint{6.494553in}{3.649842in}}%
\pgfpathcurveto{\pgfqpoint{6.494553in}{3.655666in}}{\pgfqpoint{6.492239in}{3.661252in}}{\pgfqpoint{6.488121in}{3.665370in}}%
\pgfpathcurveto{\pgfqpoint{6.484003in}{3.669488in}}{\pgfqpoint{6.478417in}{3.671802in}}{\pgfqpoint{6.472593in}{3.671802in}}%
\pgfpathcurveto{\pgfqpoint{6.466769in}{3.671802in}}{\pgfqpoint{6.461183in}{3.669488in}}{\pgfqpoint{6.457065in}{3.665370in}}%
\pgfpathcurveto{\pgfqpoint{6.452947in}{3.661252in}}{\pgfqpoint{6.450633in}{3.655666in}}{\pgfqpoint{6.450633in}{3.649842in}}%
\pgfpathcurveto{\pgfqpoint{6.450633in}{3.644018in}}{\pgfqpoint{6.452947in}{3.638432in}}{\pgfqpoint{6.457065in}{3.634313in}}%
\pgfpathcurveto{\pgfqpoint{6.461183in}{3.630195in}}{\pgfqpoint{6.466769in}{3.627881in}}{\pgfqpoint{6.472593in}{3.627881in}}%
\pgfpathlineto{\pgfqpoint{6.472593in}{3.627881in}}%
\pgfpathclose%
\pgfusepath{stroke,fill}%
\end{pgfscope}%
\begin{pgfscope}%
\pgfpathrectangle{\pgfqpoint{0.640323in}{0.527436in}}{\pgfqpoint{9.687500in}{3.850000in}}%
\pgfusepath{clip}%
\pgfsetbuttcap%
\pgfsetroundjoin%
\definecolor{currentfill}{rgb}{0.239216,0.478431,0.992157}%
\pgfsetfillcolor{currentfill}%
\pgfsetfillopacity{0.500000}%
\pgfsetlinewidth{1.003750pt}%
\definecolor{currentstroke}{rgb}{0.239216,0.478431,0.992157}%
\pgfsetstrokecolor{currentstroke}%
\pgfsetstrokeopacity{0.500000}%
\pgfsetdash{{3.700000pt}{1.600000pt}}{0.000000pt}%
\pgfpathmoveto{\pgfqpoint{6.652324in}{3.649758in}}%
\pgfpathcurveto{\pgfqpoint{6.658148in}{3.649758in}}{\pgfqpoint{6.663734in}{3.652072in}}{\pgfqpoint{6.667852in}{3.656190in}}%
\pgfpathcurveto{\pgfqpoint{6.671970in}{3.660308in}}{\pgfqpoint{6.674284in}{3.665894in}}{\pgfqpoint{6.674284in}{3.671718in}}%
\pgfpathcurveto{\pgfqpoint{6.674284in}{3.677542in}}{\pgfqpoint{6.671970in}{3.683129in}}{\pgfqpoint{6.667852in}{3.687247in}}%
\pgfpathcurveto{\pgfqpoint{6.663734in}{3.691365in}}{\pgfqpoint{6.658148in}{3.693679in}}{\pgfqpoint{6.652324in}{3.693679in}}%
\pgfpathcurveto{\pgfqpoint{6.646500in}{3.693679in}}{\pgfqpoint{6.640914in}{3.691365in}}{\pgfqpoint{6.636796in}{3.687247in}}%
\pgfpathcurveto{\pgfqpoint{6.632678in}{3.683129in}}{\pgfqpoint{6.630364in}{3.677542in}}{\pgfqpoint{6.630364in}{3.671718in}}%
\pgfpathcurveto{\pgfqpoint{6.630364in}{3.665894in}}{\pgfqpoint{6.632678in}{3.660308in}}{\pgfqpoint{6.636796in}{3.656190in}}%
\pgfpathcurveto{\pgfqpoint{6.640914in}{3.652072in}}{\pgfqpoint{6.646500in}{3.649758in}}{\pgfqpoint{6.652324in}{3.649758in}}%
\pgfpathlineto{\pgfqpoint{6.652324in}{3.649758in}}%
\pgfpathclose%
\pgfusepath{stroke,fill}%
\end{pgfscope}%
\begin{pgfscope}%
\pgfpathrectangle{\pgfqpoint{0.640323in}{0.527436in}}{\pgfqpoint{9.687500in}{3.850000in}}%
\pgfusepath{clip}%
\pgfsetbuttcap%
\pgfsetroundjoin%
\definecolor{currentfill}{rgb}{0.239216,0.478431,0.992157}%
\pgfsetfillcolor{currentfill}%
\pgfsetfillopacity{0.500000}%
\pgfsetlinewidth{1.003750pt}%
\definecolor{currentstroke}{rgb}{0.239216,0.478431,0.992157}%
\pgfsetstrokecolor{currentstroke}%
\pgfsetstrokeopacity{0.500000}%
\pgfsetdash{{3.700000pt}{1.600000pt}}{0.000000pt}%
\pgfpathmoveto{\pgfqpoint{6.832055in}{3.689544in}}%
\pgfpathcurveto{\pgfqpoint{6.837879in}{3.689544in}}{\pgfqpoint{6.843465in}{3.691857in}}{\pgfqpoint{6.847583in}{3.695976in}}%
\pgfpathcurveto{\pgfqpoint{6.851701in}{3.700094in}}{\pgfqpoint{6.854015in}{3.705680in}}{\pgfqpoint{6.854015in}{3.711504in}}%
\pgfpathcurveto{\pgfqpoint{6.854015in}{3.717328in}}{\pgfqpoint{6.851701in}{3.722914in}}{\pgfqpoint{6.847583in}{3.727032in}}%
\pgfpathcurveto{\pgfqpoint{6.843465in}{3.731150in}}{\pgfqpoint{6.837879in}{3.733464in}}{\pgfqpoint{6.832055in}{3.733464in}}%
\pgfpathcurveto{\pgfqpoint{6.826231in}{3.733464in}}{\pgfqpoint{6.820645in}{3.731150in}}{\pgfqpoint{6.816527in}{3.727032in}}%
\pgfpathcurveto{\pgfqpoint{6.812408in}{3.722914in}}{\pgfqpoint{6.810095in}{3.717328in}}{\pgfqpoint{6.810095in}{3.711504in}}%
\pgfpathcurveto{\pgfqpoint{6.810095in}{3.705680in}}{\pgfqpoint{6.812408in}{3.700094in}}{\pgfqpoint{6.816527in}{3.695976in}}%
\pgfpathcurveto{\pgfqpoint{6.820645in}{3.691857in}}{\pgfqpoint{6.826231in}{3.689544in}}{\pgfqpoint{6.832055in}{3.689544in}}%
\pgfpathlineto{\pgfqpoint{6.832055in}{3.689544in}}%
\pgfpathclose%
\pgfusepath{stroke,fill}%
\end{pgfscope}%
\begin{pgfscope}%
\pgfpathrectangle{\pgfqpoint{0.640323in}{0.527436in}}{\pgfqpoint{9.687500in}{3.850000in}}%
\pgfusepath{clip}%
\pgfsetbuttcap%
\pgfsetroundjoin%
\definecolor{currentfill}{rgb}{0.239216,0.478431,0.992157}%
\pgfsetfillcolor{currentfill}%
\pgfsetfillopacity{0.500000}%
\pgfsetlinewidth{1.003750pt}%
\definecolor{currentstroke}{rgb}{0.239216,0.478431,0.992157}%
\pgfsetstrokecolor{currentstroke}%
\pgfsetstrokeopacity{0.500000}%
\pgfsetdash{{3.700000pt}{1.600000pt}}{0.000000pt}%
\pgfpathmoveto{\pgfqpoint{7.011786in}{3.711458in}}%
\pgfpathcurveto{\pgfqpoint{7.017610in}{3.711458in}}{\pgfqpoint{7.023196in}{3.713771in}}{\pgfqpoint{7.027314in}{3.717890in}}%
\pgfpathcurveto{\pgfqpoint{7.031432in}{3.722008in}}{\pgfqpoint{7.033746in}{3.727594in}}{\pgfqpoint{7.033746in}{3.733418in}}%
\pgfpathcurveto{\pgfqpoint{7.033746in}{3.739242in}}{\pgfqpoint{7.031432in}{3.744828in}}{\pgfqpoint{7.027314in}{3.748946in}}%
\pgfpathcurveto{\pgfqpoint{7.023196in}{3.753064in}}{\pgfqpoint{7.017610in}{3.755378in}}{\pgfqpoint{7.011786in}{3.755378in}}%
\pgfpathcurveto{\pgfqpoint{7.005962in}{3.755378in}}{\pgfqpoint{7.000376in}{3.753064in}}{\pgfqpoint{6.996258in}{3.748946in}}%
\pgfpathcurveto{\pgfqpoint{6.992139in}{3.744828in}}{\pgfqpoint{6.989826in}{3.739242in}}{\pgfqpoint{6.989826in}{3.733418in}}%
\pgfpathcurveto{\pgfqpoint{6.989826in}{3.727594in}}{\pgfqpoint{6.992139in}{3.722008in}}{\pgfqpoint{6.996258in}{3.717890in}}%
\pgfpathcurveto{\pgfqpoint{7.000376in}{3.713771in}}{\pgfqpoint{7.005962in}{3.711458in}}{\pgfqpoint{7.011786in}{3.711458in}}%
\pgfpathlineto{\pgfqpoint{7.011786in}{3.711458in}}%
\pgfpathclose%
\pgfusepath{stroke,fill}%
\end{pgfscope}%
\begin{pgfscope}%
\pgfpathrectangle{\pgfqpoint{0.640323in}{0.527436in}}{\pgfqpoint{9.687500in}{3.850000in}}%
\pgfusepath{clip}%
\pgfsetbuttcap%
\pgfsetroundjoin%
\definecolor{currentfill}{rgb}{0.239216,0.478431,0.992157}%
\pgfsetfillcolor{currentfill}%
\pgfsetfillopacity{0.500000}%
\pgfsetlinewidth{1.003750pt}%
\definecolor{currentstroke}{rgb}{0.239216,0.478431,0.992157}%
\pgfsetstrokecolor{currentstroke}%
\pgfsetstrokeopacity{0.500000}%
\pgfsetdash{{3.700000pt}{1.600000pt}}{0.000000pt}%
\pgfpathmoveto{\pgfqpoint{7.191517in}{3.742431in}}%
\pgfpathcurveto{\pgfqpoint{7.197341in}{3.742431in}}{\pgfqpoint{7.202927in}{3.744745in}}{\pgfqpoint{7.207045in}{3.748863in}}%
\pgfpathcurveto{\pgfqpoint{7.211163in}{3.752982in}}{\pgfqpoint{7.213477in}{3.758568in}}{\pgfqpoint{7.213477in}{3.764392in}}%
\pgfpathcurveto{\pgfqpoint{7.213477in}{3.770216in}}{\pgfqpoint{7.211163in}{3.775802in}}{\pgfqpoint{7.207045in}{3.779920in}}%
\pgfpathcurveto{\pgfqpoint{7.202927in}{3.784038in}}{\pgfqpoint{7.197341in}{3.786352in}}{\pgfqpoint{7.191517in}{3.786352in}}%
\pgfpathcurveto{\pgfqpoint{7.185693in}{3.786352in}}{\pgfqpoint{7.180107in}{3.784038in}}{\pgfqpoint{7.175989in}{3.779920in}}%
\pgfpathcurveto{\pgfqpoint{7.171870in}{3.775802in}}{\pgfqpoint{7.169557in}{3.770216in}}{\pgfqpoint{7.169557in}{3.764392in}}%
\pgfpathcurveto{\pgfqpoint{7.169557in}{3.758568in}}{\pgfqpoint{7.171870in}{3.752982in}}{\pgfqpoint{7.175989in}{3.748863in}}%
\pgfpathcurveto{\pgfqpoint{7.180107in}{3.744745in}}{\pgfqpoint{7.185693in}{3.742431in}}{\pgfqpoint{7.191517in}{3.742431in}}%
\pgfpathlineto{\pgfqpoint{7.191517in}{3.742431in}}%
\pgfpathclose%
\pgfusepath{stroke,fill}%
\end{pgfscope}%
\begin{pgfscope}%
\pgfpathrectangle{\pgfqpoint{0.640323in}{0.527436in}}{\pgfqpoint{9.687500in}{3.850000in}}%
\pgfusepath{clip}%
\pgfsetbuttcap%
\pgfsetroundjoin%
\definecolor{currentfill}{rgb}{0.239216,0.478431,0.992157}%
\pgfsetfillcolor{currentfill}%
\pgfsetfillopacity{0.500000}%
\pgfsetlinewidth{1.003750pt}%
\definecolor{currentstroke}{rgb}{0.239216,0.478431,0.992157}%
\pgfsetstrokecolor{currentstroke}%
\pgfsetstrokeopacity{0.500000}%
\pgfsetdash{{3.700000pt}{1.600000pt}}{0.000000pt}%
\pgfpathmoveto{\pgfqpoint{7.371248in}{3.767891in}}%
\pgfpathcurveto{\pgfqpoint{7.377072in}{3.767891in}}{\pgfqpoint{7.382658in}{3.770205in}}{\pgfqpoint{7.386776in}{3.774323in}}%
\pgfpathcurveto{\pgfqpoint{7.390894in}{3.778441in}}{\pgfqpoint{7.393208in}{3.784027in}}{\pgfqpoint{7.393208in}{3.789851in}}%
\pgfpathcurveto{\pgfqpoint{7.393208in}{3.795675in}}{\pgfqpoint{7.390894in}{3.801261in}}{\pgfqpoint{7.386776in}{3.805380in}}%
\pgfpathcurveto{\pgfqpoint{7.382658in}{3.809498in}}{\pgfqpoint{7.377072in}{3.811812in}}{\pgfqpoint{7.371248in}{3.811812in}}%
\pgfpathcurveto{\pgfqpoint{7.365424in}{3.811812in}}{\pgfqpoint{7.359838in}{3.809498in}}{\pgfqpoint{7.355720in}{3.805380in}}%
\pgfpathcurveto{\pgfqpoint{7.351601in}{3.801261in}}{\pgfqpoint{7.349288in}{3.795675in}}{\pgfqpoint{7.349288in}{3.789851in}}%
\pgfpathcurveto{\pgfqpoint{7.349288in}{3.784027in}}{\pgfqpoint{7.351601in}{3.778441in}}{\pgfqpoint{7.355720in}{3.774323in}}%
\pgfpathcurveto{\pgfqpoint{7.359838in}{3.770205in}}{\pgfqpoint{7.365424in}{3.767891in}}{\pgfqpoint{7.371248in}{3.767891in}}%
\pgfpathlineto{\pgfqpoint{7.371248in}{3.767891in}}%
\pgfpathclose%
\pgfusepath{stroke,fill}%
\end{pgfscope}%
\begin{pgfscope}%
\pgfpathrectangle{\pgfqpoint{0.640323in}{0.527436in}}{\pgfqpoint{9.687500in}{3.850000in}}%
\pgfusepath{clip}%
\pgfsetbuttcap%
\pgfsetroundjoin%
\definecolor{currentfill}{rgb}{0.239216,0.478431,0.992157}%
\pgfsetfillcolor{currentfill}%
\pgfsetfillopacity{0.500000}%
\pgfsetlinewidth{1.003750pt}%
\definecolor{currentstroke}{rgb}{0.239216,0.478431,0.992157}%
\pgfsetstrokecolor{currentstroke}%
\pgfsetstrokeopacity{0.500000}%
\pgfsetdash{{3.700000pt}{1.600000pt}}{0.000000pt}%
\pgfpathmoveto{\pgfqpoint{7.550979in}{3.788010in}}%
\pgfpathcurveto{\pgfqpoint{7.556803in}{3.788010in}}{\pgfqpoint{7.562389in}{3.790324in}}{\pgfqpoint{7.566507in}{3.794442in}}%
\pgfpathcurveto{\pgfqpoint{7.570625in}{3.798561in}}{\pgfqpoint{7.572939in}{3.804147in}}{\pgfqpoint{7.572939in}{3.809971in}}%
\pgfpathcurveto{\pgfqpoint{7.572939in}{3.815795in}}{\pgfqpoint{7.570625in}{3.821381in}}{\pgfqpoint{7.566507in}{3.825499in}}%
\pgfpathcurveto{\pgfqpoint{7.562389in}{3.829617in}}{\pgfqpoint{7.556803in}{3.831931in}}{\pgfqpoint{7.550979in}{3.831931in}}%
\pgfpathcurveto{\pgfqpoint{7.545155in}{3.831931in}}{\pgfqpoint{7.539569in}{3.829617in}}{\pgfqpoint{7.535451in}{3.825499in}}%
\pgfpathcurveto{\pgfqpoint{7.531332in}{3.821381in}}{\pgfqpoint{7.529019in}{3.815795in}}{\pgfqpoint{7.529019in}{3.809971in}}%
\pgfpathcurveto{\pgfqpoint{7.529019in}{3.804147in}}{\pgfqpoint{7.531332in}{3.798561in}}{\pgfqpoint{7.535451in}{3.794442in}}%
\pgfpathcurveto{\pgfqpoint{7.539569in}{3.790324in}}{\pgfqpoint{7.545155in}{3.788010in}}{\pgfqpoint{7.550979in}{3.788010in}}%
\pgfpathlineto{\pgfqpoint{7.550979in}{3.788010in}}%
\pgfpathclose%
\pgfusepath{stroke,fill}%
\end{pgfscope}%
\begin{pgfscope}%
\pgfpathrectangle{\pgfqpoint{0.640323in}{0.527436in}}{\pgfqpoint{9.687500in}{3.850000in}}%
\pgfusepath{clip}%
\pgfsetbuttcap%
\pgfsetroundjoin%
\definecolor{currentfill}{rgb}{0.239216,0.478431,0.992157}%
\pgfsetfillcolor{currentfill}%
\pgfsetfillopacity{0.500000}%
\pgfsetlinewidth{1.003750pt}%
\definecolor{currentstroke}{rgb}{0.239216,0.478431,0.992157}%
\pgfsetstrokecolor{currentstroke}%
\pgfsetstrokeopacity{0.500000}%
\pgfsetdash{{3.700000pt}{1.600000pt}}{0.000000pt}%
\pgfpathmoveto{\pgfqpoint{7.730710in}{3.813458in}}%
\pgfpathcurveto{\pgfqpoint{7.736534in}{3.813458in}}{\pgfqpoint{7.742120in}{3.815772in}}{\pgfqpoint{7.746238in}{3.819890in}}%
\pgfpathcurveto{\pgfqpoint{7.750356in}{3.824008in}}{\pgfqpoint{7.752670in}{3.829594in}}{\pgfqpoint{7.752670in}{3.835418in}}%
\pgfpathcurveto{\pgfqpoint{7.752670in}{3.841242in}}{\pgfqpoint{7.750356in}{3.846828in}}{\pgfqpoint{7.746238in}{3.850946in}}%
\pgfpathcurveto{\pgfqpoint{7.742120in}{3.855064in}}{\pgfqpoint{7.736534in}{3.857378in}}{\pgfqpoint{7.730710in}{3.857378in}}%
\pgfpathcurveto{\pgfqpoint{7.724886in}{3.857378in}}{\pgfqpoint{7.719300in}{3.855064in}}{\pgfqpoint{7.715182in}{3.850946in}}%
\pgfpathcurveto{\pgfqpoint{7.711063in}{3.846828in}}{\pgfqpoint{7.708750in}{3.841242in}}{\pgfqpoint{7.708750in}{3.835418in}}%
\pgfpathcurveto{\pgfqpoint{7.708750in}{3.829594in}}{\pgfqpoint{7.711063in}{3.824008in}}{\pgfqpoint{7.715182in}{3.819890in}}%
\pgfpathcurveto{\pgfqpoint{7.719300in}{3.815772in}}{\pgfqpoint{7.724886in}{3.813458in}}{\pgfqpoint{7.730710in}{3.813458in}}%
\pgfpathlineto{\pgfqpoint{7.730710in}{3.813458in}}%
\pgfpathclose%
\pgfusepath{stroke,fill}%
\end{pgfscope}%
\begin{pgfscope}%
\pgfpathrectangle{\pgfqpoint{0.640323in}{0.527436in}}{\pgfqpoint{9.687500in}{3.850000in}}%
\pgfusepath{clip}%
\pgfsetbuttcap%
\pgfsetroundjoin%
\definecolor{currentfill}{rgb}{0.239216,0.478431,0.992157}%
\pgfsetfillcolor{currentfill}%
\pgfsetfillopacity{0.500000}%
\pgfsetlinewidth{1.003750pt}%
\definecolor{currentstroke}{rgb}{0.239216,0.478431,0.992157}%
\pgfsetstrokecolor{currentstroke}%
\pgfsetstrokeopacity{0.500000}%
\pgfsetdash{{3.700000pt}{1.600000pt}}{0.000000pt}%
\pgfpathmoveto{\pgfqpoint{7.910441in}{3.828479in}}%
\pgfpathcurveto{\pgfqpoint{7.916265in}{3.828479in}}{\pgfqpoint{7.921851in}{3.830793in}}{\pgfqpoint{7.925969in}{3.834911in}}%
\pgfpathcurveto{\pgfqpoint{7.930087in}{3.839029in}}{\pgfqpoint{7.932401in}{3.844615in}}{\pgfqpoint{7.932401in}{3.850439in}}%
\pgfpathcurveto{\pgfqpoint{7.932401in}{3.856263in}}{\pgfqpoint{7.930087in}{3.861849in}}{\pgfqpoint{7.925969in}{3.865967in}}%
\pgfpathcurveto{\pgfqpoint{7.921851in}{3.870086in}}{\pgfqpoint{7.916265in}{3.872399in}}{\pgfqpoint{7.910441in}{3.872399in}}%
\pgfpathcurveto{\pgfqpoint{7.904617in}{3.872399in}}{\pgfqpoint{7.899031in}{3.870086in}}{\pgfqpoint{7.894913in}{3.865967in}}%
\pgfpathcurveto{\pgfqpoint{7.890794in}{3.861849in}}{\pgfqpoint{7.888481in}{3.856263in}}{\pgfqpoint{7.888481in}{3.850439in}}%
\pgfpathcurveto{\pgfqpoint{7.888481in}{3.844615in}}{\pgfqpoint{7.890794in}{3.839029in}}{\pgfqpoint{7.894913in}{3.834911in}}%
\pgfpathcurveto{\pgfqpoint{7.899031in}{3.830793in}}{\pgfqpoint{7.904617in}{3.828479in}}{\pgfqpoint{7.910441in}{3.828479in}}%
\pgfpathlineto{\pgfqpoint{7.910441in}{3.828479in}}%
\pgfpathclose%
\pgfusepath{stroke,fill}%
\end{pgfscope}%
\begin{pgfscope}%
\pgfpathrectangle{\pgfqpoint{0.640323in}{0.527436in}}{\pgfqpoint{9.687500in}{3.850000in}}%
\pgfusepath{clip}%
\pgfsetbuttcap%
\pgfsetroundjoin%
\definecolor{currentfill}{rgb}{0.239216,0.478431,0.992157}%
\pgfsetfillcolor{currentfill}%
\pgfsetfillopacity{0.500000}%
\pgfsetlinewidth{1.003750pt}%
\definecolor{currentstroke}{rgb}{0.239216,0.478431,0.992157}%
\pgfsetstrokecolor{currentstroke}%
\pgfsetstrokeopacity{0.500000}%
\pgfsetdash{{3.700000pt}{1.600000pt}}{0.000000pt}%
\pgfpathmoveto{\pgfqpoint{8.090172in}{3.858683in}}%
\pgfpathcurveto{\pgfqpoint{8.095996in}{3.858683in}}{\pgfqpoint{8.101582in}{3.860997in}}{\pgfqpoint{8.105700in}{3.865115in}}%
\pgfpathcurveto{\pgfqpoint{8.109818in}{3.869233in}}{\pgfqpoint{8.112132in}{3.874819in}}{\pgfqpoint{8.112132in}{3.880643in}}%
\pgfpathcurveto{\pgfqpoint{8.112132in}{3.886467in}}{\pgfqpoint{8.109818in}{3.892053in}}{\pgfqpoint{8.105700in}{3.896171in}}%
\pgfpathcurveto{\pgfqpoint{8.101582in}{3.900289in}}{\pgfqpoint{8.095996in}{3.902603in}}{\pgfqpoint{8.090172in}{3.902603in}}%
\pgfpathcurveto{\pgfqpoint{8.084348in}{3.902603in}}{\pgfqpoint{8.078762in}{3.900289in}}{\pgfqpoint{8.074644in}{3.896171in}}%
\pgfpathcurveto{\pgfqpoint{8.070525in}{3.892053in}}{\pgfqpoint{8.068211in}{3.886467in}}{\pgfqpoint{8.068211in}{3.880643in}}%
\pgfpathcurveto{\pgfqpoint{8.068211in}{3.874819in}}{\pgfqpoint{8.070525in}{3.869233in}}{\pgfqpoint{8.074644in}{3.865115in}}%
\pgfpathcurveto{\pgfqpoint{8.078762in}{3.860997in}}{\pgfqpoint{8.084348in}{3.858683in}}{\pgfqpoint{8.090172in}{3.858683in}}%
\pgfpathlineto{\pgfqpoint{8.090172in}{3.858683in}}%
\pgfpathclose%
\pgfusepath{stroke,fill}%
\end{pgfscope}%
\begin{pgfscope}%
\pgfpathrectangle{\pgfqpoint{0.640323in}{0.527436in}}{\pgfqpoint{9.687500in}{3.850000in}}%
\pgfusepath{clip}%
\pgfsetbuttcap%
\pgfsetroundjoin%
\definecolor{currentfill}{rgb}{0.239216,0.478431,0.992157}%
\pgfsetfillcolor{currentfill}%
\pgfsetfillopacity{0.500000}%
\pgfsetlinewidth{1.003750pt}%
\definecolor{currentstroke}{rgb}{0.239216,0.478431,0.992157}%
\pgfsetstrokecolor{currentstroke}%
\pgfsetstrokeopacity{0.500000}%
\pgfsetdash{{3.700000pt}{1.600000pt}}{0.000000pt}%
\pgfpathmoveto{\pgfqpoint{8.269903in}{3.873903in}}%
\pgfpathcurveto{\pgfqpoint{8.275727in}{3.873903in}}{\pgfqpoint{8.281313in}{3.876217in}}{\pgfqpoint{8.285431in}{3.880335in}}%
\pgfpathcurveto{\pgfqpoint{8.289549in}{3.884453in}}{\pgfqpoint{8.291863in}{3.890039in}}{\pgfqpoint{8.291863in}{3.895863in}}%
\pgfpathcurveto{\pgfqpoint{8.291863in}{3.901687in}}{\pgfqpoint{8.289549in}{3.907273in}}{\pgfqpoint{8.285431in}{3.911391in}}%
\pgfpathcurveto{\pgfqpoint{8.281313in}{3.915509in}}{\pgfqpoint{8.275727in}{3.917823in}}{\pgfqpoint{8.269903in}{3.917823in}}%
\pgfpathcurveto{\pgfqpoint{8.264079in}{3.917823in}}{\pgfqpoint{8.258493in}{3.915509in}}{\pgfqpoint{8.254374in}{3.911391in}}%
\pgfpathcurveto{\pgfqpoint{8.250256in}{3.907273in}}{\pgfqpoint{8.247942in}{3.901687in}}{\pgfqpoint{8.247942in}{3.895863in}}%
\pgfpathcurveto{\pgfqpoint{8.247942in}{3.890039in}}{\pgfqpoint{8.250256in}{3.884453in}}{\pgfqpoint{8.254374in}{3.880335in}}%
\pgfpathcurveto{\pgfqpoint{8.258493in}{3.876217in}}{\pgfqpoint{8.264079in}{3.873903in}}{\pgfqpoint{8.269903in}{3.873903in}}%
\pgfpathlineto{\pgfqpoint{8.269903in}{3.873903in}}%
\pgfpathclose%
\pgfusepath{stroke,fill}%
\end{pgfscope}%
\begin{pgfscope}%
\pgfpathrectangle{\pgfqpoint{0.640323in}{0.527436in}}{\pgfqpoint{9.687500in}{3.850000in}}%
\pgfusepath{clip}%
\pgfsetbuttcap%
\pgfsetroundjoin%
\definecolor{currentfill}{rgb}{0.239216,0.478431,0.992157}%
\pgfsetfillcolor{currentfill}%
\pgfsetfillopacity{0.500000}%
\pgfsetlinewidth{1.003750pt}%
\definecolor{currentstroke}{rgb}{0.239216,0.478431,0.992157}%
\pgfsetstrokecolor{currentstroke}%
\pgfsetstrokeopacity{0.500000}%
\pgfsetdash{{3.700000pt}{1.600000pt}}{0.000000pt}%
\pgfpathmoveto{\pgfqpoint{8.449634in}{3.892724in}}%
\pgfpathcurveto{\pgfqpoint{8.455458in}{3.892724in}}{\pgfqpoint{8.461044in}{3.895038in}}{\pgfqpoint{8.465162in}{3.899156in}}%
\pgfpathcurveto{\pgfqpoint{8.469280in}{3.903274in}}{\pgfqpoint{8.471594in}{3.908861in}}{\pgfqpoint{8.471594in}{3.914684in}}%
\pgfpathcurveto{\pgfqpoint{8.471594in}{3.920508in}}{\pgfqpoint{8.469280in}{3.926095in}}{\pgfqpoint{8.465162in}{3.930213in}}%
\pgfpathcurveto{\pgfqpoint{8.461044in}{3.934331in}}{\pgfqpoint{8.455458in}{3.936645in}}{\pgfqpoint{8.449634in}{3.936645in}}%
\pgfpathcurveto{\pgfqpoint{8.443810in}{3.936645in}}{\pgfqpoint{8.438224in}{3.934331in}}{\pgfqpoint{8.434105in}{3.930213in}}%
\pgfpathcurveto{\pgfqpoint{8.429987in}{3.926095in}}{\pgfqpoint{8.427673in}{3.920508in}}{\pgfqpoint{8.427673in}{3.914684in}}%
\pgfpathcurveto{\pgfqpoint{8.427673in}{3.908861in}}{\pgfqpoint{8.429987in}{3.903274in}}{\pgfqpoint{8.434105in}{3.899156in}}%
\pgfpathcurveto{\pgfqpoint{8.438224in}{3.895038in}}{\pgfqpoint{8.443810in}{3.892724in}}{\pgfqpoint{8.449634in}{3.892724in}}%
\pgfpathlineto{\pgfqpoint{8.449634in}{3.892724in}}%
\pgfpathclose%
\pgfusepath{stroke,fill}%
\end{pgfscope}%
\begin{pgfscope}%
\pgfpathrectangle{\pgfqpoint{0.640323in}{0.527436in}}{\pgfqpoint{9.687500in}{3.850000in}}%
\pgfusepath{clip}%
\pgfsetbuttcap%
\pgfsetroundjoin%
\definecolor{currentfill}{rgb}{0.239216,0.478431,0.992157}%
\pgfsetfillcolor{currentfill}%
\pgfsetfillopacity{0.500000}%
\pgfsetlinewidth{1.003750pt}%
\definecolor{currentstroke}{rgb}{0.239216,0.478431,0.992157}%
\pgfsetstrokecolor{currentstroke}%
\pgfsetstrokeopacity{0.500000}%
\pgfsetdash{{3.700000pt}{1.600000pt}}{0.000000pt}%
\pgfpathmoveto{\pgfqpoint{8.629365in}{3.907534in}}%
\pgfpathcurveto{\pgfqpoint{8.635189in}{3.907534in}}{\pgfqpoint{8.640775in}{3.909848in}}{\pgfqpoint{8.644893in}{3.913966in}}%
\pgfpathcurveto{\pgfqpoint{8.649011in}{3.918084in}}{\pgfqpoint{8.651325in}{3.923671in}}{\pgfqpoint{8.651325in}{3.929495in}}%
\pgfpathcurveto{\pgfqpoint{8.651325in}{3.935318in}}{\pgfqpoint{8.649011in}{3.940905in}}{\pgfqpoint{8.644893in}{3.945023in}}%
\pgfpathcurveto{\pgfqpoint{8.640775in}{3.949141in}}{\pgfqpoint{8.635189in}{3.951455in}}{\pgfqpoint{8.629365in}{3.951455in}}%
\pgfpathcurveto{\pgfqpoint{8.623541in}{3.951455in}}{\pgfqpoint{8.617955in}{3.949141in}}{\pgfqpoint{8.613836in}{3.945023in}}%
\pgfpathcurveto{\pgfqpoint{8.609718in}{3.940905in}}{\pgfqpoint{8.607404in}{3.935318in}}{\pgfqpoint{8.607404in}{3.929495in}}%
\pgfpathcurveto{\pgfqpoint{8.607404in}{3.923671in}}{\pgfqpoint{8.609718in}{3.918084in}}{\pgfqpoint{8.613836in}{3.913966in}}%
\pgfpathcurveto{\pgfqpoint{8.617955in}{3.909848in}}{\pgfqpoint{8.623541in}{3.907534in}}{\pgfqpoint{8.629365in}{3.907534in}}%
\pgfpathlineto{\pgfqpoint{8.629365in}{3.907534in}}%
\pgfpathclose%
\pgfusepath{stroke,fill}%
\end{pgfscope}%
\begin{pgfscope}%
\pgfpathrectangle{\pgfqpoint{0.640323in}{0.527436in}}{\pgfqpoint{9.687500in}{3.850000in}}%
\pgfusepath{clip}%
\pgfsetbuttcap%
\pgfsetroundjoin%
\definecolor{currentfill}{rgb}{0.239216,0.478431,0.992157}%
\pgfsetfillcolor{currentfill}%
\pgfsetfillopacity{0.500000}%
\pgfsetlinewidth{1.003750pt}%
\definecolor{currentstroke}{rgb}{0.239216,0.478431,0.992157}%
\pgfsetstrokecolor{currentstroke}%
\pgfsetstrokeopacity{0.500000}%
\pgfsetdash{{3.700000pt}{1.600000pt}}{0.000000pt}%
\pgfpathmoveto{\pgfqpoint{8.809096in}{3.926269in}}%
\pgfpathcurveto{\pgfqpoint{8.814920in}{3.926269in}}{\pgfqpoint{8.820506in}{3.928583in}}{\pgfqpoint{8.824624in}{3.932701in}}%
\pgfpathcurveto{\pgfqpoint{8.828742in}{3.936819in}}{\pgfqpoint{8.831056in}{3.942405in}}{\pgfqpoint{8.831056in}{3.948229in}}%
\pgfpathcurveto{\pgfqpoint{8.831056in}{3.954053in}}{\pgfqpoint{8.828742in}{3.959639in}}{\pgfqpoint{8.824624in}{3.963757in}}%
\pgfpathcurveto{\pgfqpoint{8.820506in}{3.967876in}}{\pgfqpoint{8.814920in}{3.970189in}}{\pgfqpoint{8.809096in}{3.970189in}}%
\pgfpathcurveto{\pgfqpoint{8.803272in}{3.970189in}}{\pgfqpoint{8.797686in}{3.967876in}}{\pgfqpoint{8.793567in}{3.963757in}}%
\pgfpathcurveto{\pgfqpoint{8.789449in}{3.959639in}}{\pgfqpoint{8.787135in}{3.954053in}}{\pgfqpoint{8.787135in}{3.948229in}}%
\pgfpathcurveto{\pgfqpoint{8.787135in}{3.942405in}}{\pgfqpoint{8.789449in}{3.936819in}}{\pgfqpoint{8.793567in}{3.932701in}}%
\pgfpathcurveto{\pgfqpoint{8.797686in}{3.928583in}}{\pgfqpoint{8.803272in}{3.926269in}}{\pgfqpoint{8.809096in}{3.926269in}}%
\pgfpathlineto{\pgfqpoint{8.809096in}{3.926269in}}%
\pgfpathclose%
\pgfusepath{stroke,fill}%
\end{pgfscope}%
\begin{pgfscope}%
\pgfpathrectangle{\pgfqpoint{0.640323in}{0.527436in}}{\pgfqpoint{9.687500in}{3.850000in}}%
\pgfusepath{clip}%
\pgfsetbuttcap%
\pgfsetroundjoin%
\definecolor{currentfill}{rgb}{0.239216,0.478431,0.992157}%
\pgfsetfillcolor{currentfill}%
\pgfsetfillopacity{0.500000}%
\pgfsetlinewidth{1.003750pt}%
\definecolor{currentstroke}{rgb}{0.239216,0.478431,0.992157}%
\pgfsetstrokecolor{currentstroke}%
\pgfsetstrokeopacity{0.500000}%
\pgfsetdash{{3.700000pt}{1.600000pt}}{0.000000pt}%
\pgfpathmoveto{\pgfqpoint{8.988827in}{3.945997in}}%
\pgfpathcurveto{\pgfqpoint{8.994651in}{3.945997in}}{\pgfqpoint{9.000237in}{3.948311in}}{\pgfqpoint{9.004355in}{3.952429in}}%
\pgfpathcurveto{\pgfqpoint{9.008473in}{3.956547in}}{\pgfqpoint{9.010787in}{3.962133in}}{\pgfqpoint{9.010787in}{3.967957in}}%
\pgfpathcurveto{\pgfqpoint{9.010787in}{3.973781in}}{\pgfqpoint{9.008473in}{3.979367in}}{\pgfqpoint{9.004355in}{3.983486in}}%
\pgfpathcurveto{\pgfqpoint{9.000237in}{3.987604in}}{\pgfqpoint{8.994651in}{3.989918in}}{\pgfqpoint{8.988827in}{3.989918in}}%
\pgfpathcurveto{\pgfqpoint{8.983003in}{3.989918in}}{\pgfqpoint{8.977417in}{3.987604in}}{\pgfqpoint{8.973298in}{3.983486in}}%
\pgfpathcurveto{\pgfqpoint{8.969180in}{3.979367in}}{\pgfqpoint{8.966866in}{3.973781in}}{\pgfqpoint{8.966866in}{3.967957in}}%
\pgfpathcurveto{\pgfqpoint{8.966866in}{3.962133in}}{\pgfqpoint{8.969180in}{3.956547in}}{\pgfqpoint{8.973298in}{3.952429in}}%
\pgfpathcurveto{\pgfqpoint{8.977417in}{3.948311in}}{\pgfqpoint{8.983003in}{3.945997in}}{\pgfqpoint{8.988827in}{3.945997in}}%
\pgfpathlineto{\pgfqpoint{8.988827in}{3.945997in}}%
\pgfpathclose%
\pgfusepath{stroke,fill}%
\end{pgfscope}%
\begin{pgfscope}%
\pgfpathrectangle{\pgfqpoint{0.640323in}{0.527436in}}{\pgfqpoint{9.687500in}{3.850000in}}%
\pgfusepath{clip}%
\pgfsetbuttcap%
\pgfsetroundjoin%
\definecolor{currentfill}{rgb}{0.239216,0.478431,0.992157}%
\pgfsetfillcolor{currentfill}%
\pgfsetfillopacity{0.500000}%
\pgfsetlinewidth{1.003750pt}%
\definecolor{currentstroke}{rgb}{0.239216,0.478431,0.992157}%
\pgfsetstrokecolor{currentstroke}%
\pgfsetstrokeopacity{0.500000}%
\pgfsetdash{{3.700000pt}{1.600000pt}}{0.000000pt}%
\pgfpathmoveto{\pgfqpoint{9.168558in}{3.962353in}}%
\pgfpathcurveto{\pgfqpoint{9.174382in}{3.962353in}}{\pgfqpoint{9.179968in}{3.964667in}}{\pgfqpoint{9.184086in}{3.968785in}}%
\pgfpathcurveto{\pgfqpoint{9.188204in}{3.972903in}}{\pgfqpoint{9.190518in}{3.978490in}}{\pgfqpoint{9.190518in}{3.984314in}}%
\pgfpathcurveto{\pgfqpoint{9.190518in}{3.990138in}}{\pgfqpoint{9.188204in}{3.995724in}}{\pgfqpoint{9.184086in}{3.999842in}}%
\pgfpathcurveto{\pgfqpoint{9.179968in}{4.003960in}}{\pgfqpoint{9.174382in}{4.006274in}}{\pgfqpoint{9.168558in}{4.006274in}}%
\pgfpathcurveto{\pgfqpoint{9.162734in}{4.006274in}}{\pgfqpoint{9.157148in}{4.003960in}}{\pgfqpoint{9.153029in}{3.999842in}}%
\pgfpathcurveto{\pgfqpoint{9.148911in}{3.995724in}}{\pgfqpoint{9.146597in}{3.990138in}}{\pgfqpoint{9.146597in}{3.984314in}}%
\pgfpathcurveto{\pgfqpoint{9.146597in}{3.978490in}}{\pgfqpoint{9.148911in}{3.972903in}}{\pgfqpoint{9.153029in}{3.968785in}}%
\pgfpathcurveto{\pgfqpoint{9.157148in}{3.964667in}}{\pgfqpoint{9.162734in}{3.962353in}}{\pgfqpoint{9.168558in}{3.962353in}}%
\pgfpathlineto{\pgfqpoint{9.168558in}{3.962353in}}%
\pgfpathclose%
\pgfusepath{stroke,fill}%
\end{pgfscope}%
\begin{pgfscope}%
\pgfpathrectangle{\pgfqpoint{0.640323in}{0.527436in}}{\pgfqpoint{9.687500in}{3.850000in}}%
\pgfusepath{clip}%
\pgfsetbuttcap%
\pgfsetroundjoin%
\definecolor{currentfill}{rgb}{0.239216,0.478431,0.992157}%
\pgfsetfillcolor{currentfill}%
\pgfsetfillopacity{0.500000}%
\pgfsetlinewidth{1.003750pt}%
\definecolor{currentstroke}{rgb}{0.239216,0.478431,0.992157}%
\pgfsetstrokecolor{currentstroke}%
\pgfsetstrokeopacity{0.500000}%
\pgfsetdash{{3.700000pt}{1.600000pt}}{0.000000pt}%
\pgfpathmoveto{\pgfqpoint{9.348289in}{3.980392in}}%
\pgfpathcurveto{\pgfqpoint{9.354113in}{3.980392in}}{\pgfqpoint{9.359699in}{3.982706in}}{\pgfqpoint{9.363817in}{3.986824in}}%
\pgfpathcurveto{\pgfqpoint{9.367935in}{3.990943in}}{\pgfqpoint{9.370249in}{3.996529in}}{\pgfqpoint{9.370249in}{4.002353in}}%
\pgfpathcurveto{\pgfqpoint{9.370249in}{4.008177in}}{\pgfqpoint{9.367935in}{4.013763in}}{\pgfqpoint{9.363817in}{4.017881in}}%
\pgfpathcurveto{\pgfqpoint{9.359699in}{4.021999in}}{\pgfqpoint{9.354113in}{4.024313in}}{\pgfqpoint{9.348289in}{4.024313in}}%
\pgfpathcurveto{\pgfqpoint{9.342465in}{4.024313in}}{\pgfqpoint{9.336879in}{4.021999in}}{\pgfqpoint{9.332760in}{4.017881in}}%
\pgfpathcurveto{\pgfqpoint{9.328642in}{4.013763in}}{\pgfqpoint{9.326328in}{4.008177in}}{\pgfqpoint{9.326328in}{4.002353in}}%
\pgfpathcurveto{\pgfqpoint{9.326328in}{3.996529in}}{\pgfqpoint{9.328642in}{3.990943in}}{\pgfqpoint{9.332760in}{3.986824in}}%
\pgfpathcurveto{\pgfqpoint{9.336879in}{3.982706in}}{\pgfqpoint{9.342465in}{3.980392in}}{\pgfqpoint{9.348289in}{3.980392in}}%
\pgfpathlineto{\pgfqpoint{9.348289in}{3.980392in}}%
\pgfpathclose%
\pgfusepath{stroke,fill}%
\end{pgfscope}%
\begin{pgfscope}%
\pgfpathrectangle{\pgfqpoint{0.640323in}{0.527436in}}{\pgfqpoint{9.687500in}{3.850000in}}%
\pgfusepath{clip}%
\pgfsetbuttcap%
\pgfsetroundjoin%
\definecolor{currentfill}{rgb}{0.239216,0.478431,0.992157}%
\pgfsetfillcolor{currentfill}%
\pgfsetfillopacity{0.500000}%
\pgfsetlinewidth{1.003750pt}%
\definecolor{currentstroke}{rgb}{0.239216,0.478431,0.992157}%
\pgfsetstrokecolor{currentstroke}%
\pgfsetstrokeopacity{0.500000}%
\pgfsetdash{{3.700000pt}{1.600000pt}}{0.000000pt}%
\pgfpathmoveto{\pgfqpoint{9.528020in}{3.990011in}}%
\pgfpathcurveto{\pgfqpoint{9.533844in}{3.990011in}}{\pgfqpoint{9.539430in}{3.992325in}}{\pgfqpoint{9.543548in}{3.996443in}}%
\pgfpathcurveto{\pgfqpoint{9.547666in}{4.000561in}}{\pgfqpoint{9.549980in}{4.006148in}}{\pgfqpoint{9.549980in}{4.011971in}}%
\pgfpathcurveto{\pgfqpoint{9.549980in}{4.017795in}}{\pgfqpoint{9.547666in}{4.023382in}}{\pgfqpoint{9.543548in}{4.027500in}}%
\pgfpathcurveto{\pgfqpoint{9.539430in}{4.031618in}}{\pgfqpoint{9.533844in}{4.033932in}}{\pgfqpoint{9.528020in}{4.033932in}}%
\pgfpathcurveto{\pgfqpoint{9.522196in}{4.033932in}}{\pgfqpoint{9.516610in}{4.031618in}}{\pgfqpoint{9.512491in}{4.027500in}}%
\pgfpathcurveto{\pgfqpoint{9.508373in}{4.023382in}}{\pgfqpoint{9.506059in}{4.017795in}}{\pgfqpoint{9.506059in}{4.011971in}}%
\pgfpathcurveto{\pgfqpoint{9.506059in}{4.006148in}}{\pgfqpoint{9.508373in}{4.000561in}}{\pgfqpoint{9.512491in}{3.996443in}}%
\pgfpathcurveto{\pgfqpoint{9.516610in}{3.992325in}}{\pgfqpoint{9.522196in}{3.990011in}}{\pgfqpoint{9.528020in}{3.990011in}}%
\pgfpathlineto{\pgfqpoint{9.528020in}{3.990011in}}%
\pgfpathclose%
\pgfusepath{stroke,fill}%
\end{pgfscope}%
\begin{pgfscope}%
\pgfpathrectangle{\pgfqpoint{0.640323in}{0.527436in}}{\pgfqpoint{9.687500in}{3.850000in}}%
\pgfusepath{clip}%
\pgfsetbuttcap%
\pgfsetroundjoin%
\definecolor{currentfill}{rgb}{0.239216,0.478431,0.992157}%
\pgfsetfillcolor{currentfill}%
\pgfsetfillopacity{0.500000}%
\pgfsetlinewidth{1.003750pt}%
\definecolor{currentstroke}{rgb}{0.239216,0.478431,0.992157}%
\pgfsetstrokecolor{currentstroke}%
\pgfsetstrokeopacity{0.500000}%
\pgfsetdash{{3.700000pt}{1.600000pt}}{0.000000pt}%
\pgfpathmoveto{\pgfqpoint{9.707751in}{3.999512in}}%
\pgfpathcurveto{\pgfqpoint{9.713575in}{3.999512in}}{\pgfqpoint{9.719161in}{4.001826in}}{\pgfqpoint{9.723279in}{4.005944in}}%
\pgfpathcurveto{\pgfqpoint{9.727397in}{4.010062in}}{\pgfqpoint{9.729711in}{4.015648in}}{\pgfqpoint{9.729711in}{4.021472in}}%
\pgfpathcurveto{\pgfqpoint{9.729711in}{4.027296in}}{\pgfqpoint{9.727397in}{4.032882in}}{\pgfqpoint{9.723279in}{4.037001in}}%
\pgfpathcurveto{\pgfqpoint{9.719161in}{4.041119in}}{\pgfqpoint{9.713575in}{4.043433in}}{\pgfqpoint{9.707751in}{4.043433in}}%
\pgfpathcurveto{\pgfqpoint{9.701927in}{4.043433in}}{\pgfqpoint{9.696340in}{4.041119in}}{\pgfqpoint{9.692222in}{4.037001in}}%
\pgfpathcurveto{\pgfqpoint{9.688104in}{4.032882in}}{\pgfqpoint{9.685790in}{4.027296in}}{\pgfqpoint{9.685790in}{4.021472in}}%
\pgfpathcurveto{\pgfqpoint{9.685790in}{4.015648in}}{\pgfqpoint{9.688104in}{4.010062in}}{\pgfqpoint{9.692222in}{4.005944in}}%
\pgfpathcurveto{\pgfqpoint{9.696340in}{4.001826in}}{\pgfqpoint{9.701927in}{3.999512in}}{\pgfqpoint{9.707751in}{3.999512in}}%
\pgfpathlineto{\pgfqpoint{9.707751in}{3.999512in}}%
\pgfpathclose%
\pgfusepath{stroke,fill}%
\end{pgfscope}%
\begin{pgfscope}%
\pgfpathrectangle{\pgfqpoint{0.640323in}{0.527436in}}{\pgfqpoint{9.687500in}{3.850000in}}%
\pgfusepath{clip}%
\pgfsetbuttcap%
\pgfsetroundjoin%
\definecolor{currentfill}{rgb}{0.239216,0.478431,0.992157}%
\pgfsetfillcolor{currentfill}%
\pgfsetfillopacity{0.500000}%
\pgfsetlinewidth{1.003750pt}%
\definecolor{currentstroke}{rgb}{0.239216,0.478431,0.992157}%
\pgfsetstrokecolor{currentstroke}%
\pgfsetstrokeopacity{0.500000}%
\pgfsetdash{{3.700000pt}{1.600000pt}}{0.000000pt}%
\pgfpathmoveto{\pgfqpoint{9.887482in}{4.021302in}}%
\pgfpathcurveto{\pgfqpoint{9.893306in}{4.021302in}}{\pgfqpoint{9.898892in}{4.023616in}}{\pgfqpoint{9.903010in}{4.027734in}}%
\pgfpathcurveto{\pgfqpoint{9.907128in}{4.031852in}}{\pgfqpoint{9.909442in}{4.037438in}}{\pgfqpoint{9.909442in}{4.043262in}}%
\pgfpathcurveto{\pgfqpoint{9.909442in}{4.049086in}}{\pgfqpoint{9.907128in}{4.054672in}}{\pgfqpoint{9.903010in}{4.058790in}}%
\pgfpathcurveto{\pgfqpoint{9.898892in}{4.062908in}}{\pgfqpoint{9.893306in}{4.065222in}}{\pgfqpoint{9.887482in}{4.065222in}}%
\pgfpathcurveto{\pgfqpoint{9.881658in}{4.065222in}}{\pgfqpoint{9.876071in}{4.062908in}}{\pgfqpoint{9.871953in}{4.058790in}}%
\pgfpathcurveto{\pgfqpoint{9.867835in}{4.054672in}}{\pgfqpoint{9.865521in}{4.049086in}}{\pgfqpoint{9.865521in}{4.043262in}}%
\pgfpathcurveto{\pgfqpoint{9.865521in}{4.037438in}}{\pgfqpoint{9.867835in}{4.031852in}}{\pgfqpoint{9.871953in}{4.027734in}}%
\pgfpathcurveto{\pgfqpoint{9.876071in}{4.023616in}}{\pgfqpoint{9.881658in}{4.021302in}}{\pgfqpoint{9.887482in}{4.021302in}}%
\pgfpathlineto{\pgfqpoint{9.887482in}{4.021302in}}%
\pgfpathclose%
\pgfusepath{stroke,fill}%
\end{pgfscope}%
\begin{pgfscope}%
\pgfpathrectangle{\pgfqpoint{0.640323in}{0.527436in}}{\pgfqpoint{9.687500in}{3.850000in}}%
\pgfusepath{clip}%
\pgfsetbuttcap%
\pgfsetroundjoin%
\definecolor{currentfill}{rgb}{0.000000,0.000000,0.000000}%
\pgfsetfillcolor{currentfill}%
\pgfsetfillopacity{0.500000}%
\pgfsetlinewidth{1.003750pt}%
\definecolor{currentstroke}{rgb}{0.000000,0.000000,0.000000}%
\pgfsetstrokecolor{currentstroke}%
\pgfsetstrokeopacity{0.500000}%
\pgfsetdash{{3.700000pt}{1.600000pt}}{0.000000pt}%
\pgfpathmoveto{\pgfqpoint{1.080663in}{0.637576in}}%
\pgfpathcurveto{\pgfqpoint{1.086487in}{0.637576in}}{\pgfqpoint{1.092074in}{0.639890in}}{\pgfqpoint{1.096192in}{0.644008in}}%
\pgfpathcurveto{\pgfqpoint{1.100310in}{0.648126in}}{\pgfqpoint{1.102624in}{0.653712in}}{\pgfqpoint{1.102624in}{0.659536in}}%
\pgfpathcurveto{\pgfqpoint{1.102624in}{0.665360in}}{\pgfqpoint{1.100310in}{0.670946in}}{\pgfqpoint{1.096192in}{0.675065in}}%
\pgfpathcurveto{\pgfqpoint{1.092074in}{0.679183in}}{\pgfqpoint{1.086487in}{0.681497in}}{\pgfqpoint{1.080663in}{0.681497in}}%
\pgfpathcurveto{\pgfqpoint{1.074839in}{0.681497in}}{\pgfqpoint{1.069253in}{0.679183in}}{\pgfqpoint{1.065135in}{0.675065in}}%
\pgfpathcurveto{\pgfqpoint{1.061017in}{0.670946in}}{\pgfqpoint{1.058703in}{0.665360in}}{\pgfqpoint{1.058703in}{0.659536in}}%
\pgfpathcurveto{\pgfqpoint{1.058703in}{0.653712in}}{\pgfqpoint{1.061017in}{0.648126in}}{\pgfqpoint{1.065135in}{0.644008in}}%
\pgfpathcurveto{\pgfqpoint{1.069253in}{0.639890in}}{\pgfqpoint{1.074839in}{0.637576in}}{\pgfqpoint{1.080663in}{0.637576in}}%
\pgfpathlineto{\pgfqpoint{1.080663in}{0.637576in}}%
\pgfpathclose%
\pgfusepath{stroke,fill}%
\end{pgfscope}%
\begin{pgfscope}%
\pgfpathrectangle{\pgfqpoint{0.640323in}{0.527436in}}{\pgfqpoint{9.687500in}{3.850000in}}%
\pgfusepath{clip}%
\pgfsetbuttcap%
\pgfsetroundjoin%
\definecolor{currentfill}{rgb}{0.000000,0.000000,0.000000}%
\pgfsetfillcolor{currentfill}%
\pgfsetfillopacity{0.500000}%
\pgfsetlinewidth{1.003750pt}%
\definecolor{currentstroke}{rgb}{0.000000,0.000000,0.000000}%
\pgfsetstrokecolor{currentstroke}%
\pgfsetstrokeopacity{0.500000}%
\pgfsetdash{{3.700000pt}{1.600000pt}}{0.000000pt}%
\pgfpathmoveto{\pgfqpoint{1.260394in}{0.638566in}}%
\pgfpathcurveto{\pgfqpoint{1.266218in}{0.638566in}}{\pgfqpoint{1.271805in}{0.640880in}}{\pgfqpoint{1.275923in}{0.644998in}}%
\pgfpathcurveto{\pgfqpoint{1.280041in}{0.649116in}}{\pgfqpoint{1.282355in}{0.654702in}}{\pgfqpoint{1.282355in}{0.660526in}}%
\pgfpathcurveto{\pgfqpoint{1.282355in}{0.666350in}}{\pgfqpoint{1.280041in}{0.671936in}}{\pgfqpoint{1.275923in}{0.676054in}}%
\pgfpathcurveto{\pgfqpoint{1.271805in}{0.680173in}}{\pgfqpoint{1.266218in}{0.682486in}}{\pgfqpoint{1.260394in}{0.682486in}}%
\pgfpathcurveto{\pgfqpoint{1.254570in}{0.682486in}}{\pgfqpoint{1.248984in}{0.680173in}}{\pgfqpoint{1.244866in}{0.676054in}}%
\pgfpathcurveto{\pgfqpoint{1.240748in}{0.671936in}}{\pgfqpoint{1.238434in}{0.666350in}}{\pgfqpoint{1.238434in}{0.660526in}}%
\pgfpathcurveto{\pgfqpoint{1.238434in}{0.654702in}}{\pgfqpoint{1.240748in}{0.649116in}}{\pgfqpoint{1.244866in}{0.644998in}}%
\pgfpathcurveto{\pgfqpoint{1.248984in}{0.640880in}}{\pgfqpoint{1.254570in}{0.638566in}}{\pgfqpoint{1.260394in}{0.638566in}}%
\pgfpathlineto{\pgfqpoint{1.260394in}{0.638566in}}%
\pgfpathclose%
\pgfusepath{stroke,fill}%
\end{pgfscope}%
\begin{pgfscope}%
\pgfpathrectangle{\pgfqpoint{0.640323in}{0.527436in}}{\pgfqpoint{9.687500in}{3.850000in}}%
\pgfusepath{clip}%
\pgfsetbuttcap%
\pgfsetroundjoin%
\definecolor{currentfill}{rgb}{0.000000,0.000000,0.000000}%
\pgfsetfillcolor{currentfill}%
\pgfsetfillopacity{0.500000}%
\pgfsetlinewidth{1.003750pt}%
\definecolor{currentstroke}{rgb}{0.000000,0.000000,0.000000}%
\pgfsetstrokecolor{currentstroke}%
\pgfsetstrokeopacity{0.500000}%
\pgfsetdash{{3.700000pt}{1.600000pt}}{0.000000pt}%
\pgfpathmoveto{\pgfqpoint{1.440125in}{0.640284in}}%
\pgfpathcurveto{\pgfqpoint{1.445949in}{0.640284in}}{\pgfqpoint{1.451535in}{0.642598in}}{\pgfqpoint{1.455654in}{0.646716in}}%
\pgfpathcurveto{\pgfqpoint{1.459772in}{0.650834in}}{\pgfqpoint{1.462086in}{0.656420in}}{\pgfqpoint{1.462086in}{0.662244in}}%
\pgfpathcurveto{\pgfqpoint{1.462086in}{0.668068in}}{\pgfqpoint{1.459772in}{0.673654in}}{\pgfqpoint{1.455654in}{0.677773in}}%
\pgfpathcurveto{\pgfqpoint{1.451535in}{0.681891in}}{\pgfqpoint{1.445949in}{0.684205in}}{\pgfqpoint{1.440125in}{0.684205in}}%
\pgfpathcurveto{\pgfqpoint{1.434301in}{0.684205in}}{\pgfqpoint{1.428715in}{0.681891in}}{\pgfqpoint{1.424597in}{0.677773in}}%
\pgfpathcurveto{\pgfqpoint{1.420479in}{0.673654in}}{\pgfqpoint{1.418165in}{0.668068in}}{\pgfqpoint{1.418165in}{0.662244in}}%
\pgfpathcurveto{\pgfqpoint{1.418165in}{0.656420in}}{\pgfqpoint{1.420479in}{0.650834in}}{\pgfqpoint{1.424597in}{0.646716in}}%
\pgfpathcurveto{\pgfqpoint{1.428715in}{0.642598in}}{\pgfqpoint{1.434301in}{0.640284in}}{\pgfqpoint{1.440125in}{0.640284in}}%
\pgfpathlineto{\pgfqpoint{1.440125in}{0.640284in}}%
\pgfpathclose%
\pgfusepath{stroke,fill}%
\end{pgfscope}%
\begin{pgfscope}%
\pgfpathrectangle{\pgfqpoint{0.640323in}{0.527436in}}{\pgfqpoint{9.687500in}{3.850000in}}%
\pgfusepath{clip}%
\pgfsetbuttcap%
\pgfsetroundjoin%
\definecolor{currentfill}{rgb}{0.000000,0.000000,0.000000}%
\pgfsetfillcolor{currentfill}%
\pgfsetfillopacity{0.500000}%
\pgfsetlinewidth{1.003750pt}%
\definecolor{currentstroke}{rgb}{0.000000,0.000000,0.000000}%
\pgfsetstrokecolor{currentstroke}%
\pgfsetstrokeopacity{0.500000}%
\pgfsetdash{{3.700000pt}{1.600000pt}}{0.000000pt}%
\pgfpathmoveto{\pgfqpoint{1.619856in}{0.642325in}}%
\pgfpathcurveto{\pgfqpoint{1.625680in}{0.642325in}}{\pgfqpoint{1.631266in}{0.644638in}}{\pgfqpoint{1.635385in}{0.648757in}}%
\pgfpathcurveto{\pgfqpoint{1.639503in}{0.652875in}}{\pgfqpoint{1.641817in}{0.658461in}}{\pgfqpoint{1.641817in}{0.664285in}}%
\pgfpathcurveto{\pgfqpoint{1.641817in}{0.670109in}}{\pgfqpoint{1.639503in}{0.675695in}}{\pgfqpoint{1.635385in}{0.679813in}}%
\pgfpathcurveto{\pgfqpoint{1.631266in}{0.683931in}}{\pgfqpoint{1.625680in}{0.686245in}}{\pgfqpoint{1.619856in}{0.686245in}}%
\pgfpathcurveto{\pgfqpoint{1.614032in}{0.686245in}}{\pgfqpoint{1.608446in}{0.683931in}}{\pgfqpoint{1.604328in}{0.679813in}}%
\pgfpathcurveto{\pgfqpoint{1.600210in}{0.675695in}}{\pgfqpoint{1.597896in}{0.670109in}}{\pgfqpoint{1.597896in}{0.664285in}}%
\pgfpathcurveto{\pgfqpoint{1.597896in}{0.658461in}}{\pgfqpoint{1.600210in}{0.652875in}}{\pgfqpoint{1.604328in}{0.648757in}}%
\pgfpathcurveto{\pgfqpoint{1.608446in}{0.644638in}}{\pgfqpoint{1.614032in}{0.642325in}}{\pgfqpoint{1.619856in}{0.642325in}}%
\pgfpathlineto{\pgfqpoint{1.619856in}{0.642325in}}%
\pgfpathclose%
\pgfusepath{stroke,fill}%
\end{pgfscope}%
\begin{pgfscope}%
\pgfpathrectangle{\pgfqpoint{0.640323in}{0.527436in}}{\pgfqpoint{9.687500in}{3.850000in}}%
\pgfusepath{clip}%
\pgfsetbuttcap%
\pgfsetroundjoin%
\definecolor{currentfill}{rgb}{0.000000,0.000000,0.000000}%
\pgfsetfillcolor{currentfill}%
\pgfsetfillopacity{0.500000}%
\pgfsetlinewidth{1.003750pt}%
\definecolor{currentstroke}{rgb}{0.000000,0.000000,0.000000}%
\pgfsetstrokecolor{currentstroke}%
\pgfsetstrokeopacity{0.500000}%
\pgfsetdash{{3.700000pt}{1.600000pt}}{0.000000pt}%
\pgfpathmoveto{\pgfqpoint{1.799587in}{0.644563in}}%
\pgfpathcurveto{\pgfqpoint{1.805411in}{0.644563in}}{\pgfqpoint{1.810997in}{0.646877in}}{\pgfqpoint{1.815116in}{0.650995in}}%
\pgfpathcurveto{\pgfqpoint{1.819234in}{0.655113in}}{\pgfqpoint{1.821548in}{0.660699in}}{\pgfqpoint{1.821548in}{0.666523in}}%
\pgfpathcurveto{\pgfqpoint{1.821548in}{0.672347in}}{\pgfqpoint{1.819234in}{0.677933in}}{\pgfqpoint{1.815116in}{0.682051in}}%
\pgfpathcurveto{\pgfqpoint{1.810997in}{0.686169in}}{\pgfqpoint{1.805411in}{0.688483in}}{\pgfqpoint{1.799587in}{0.688483in}}%
\pgfpathcurveto{\pgfqpoint{1.793763in}{0.688483in}}{\pgfqpoint{1.788177in}{0.686169in}}{\pgfqpoint{1.784059in}{0.682051in}}%
\pgfpathcurveto{\pgfqpoint{1.779941in}{0.677933in}}{\pgfqpoint{1.777627in}{0.672347in}}{\pgfqpoint{1.777627in}{0.666523in}}%
\pgfpathcurveto{\pgfqpoint{1.777627in}{0.660699in}}{\pgfqpoint{1.779941in}{0.655113in}}{\pgfqpoint{1.784059in}{0.650995in}}%
\pgfpathcurveto{\pgfqpoint{1.788177in}{0.646877in}}{\pgfqpoint{1.793763in}{0.644563in}}{\pgfqpoint{1.799587in}{0.644563in}}%
\pgfpathlineto{\pgfqpoint{1.799587in}{0.644563in}}%
\pgfpathclose%
\pgfusepath{stroke,fill}%
\end{pgfscope}%
\begin{pgfscope}%
\pgfpathrectangle{\pgfqpoint{0.640323in}{0.527436in}}{\pgfqpoint{9.687500in}{3.850000in}}%
\pgfusepath{clip}%
\pgfsetbuttcap%
\pgfsetroundjoin%
\definecolor{currentfill}{rgb}{0.000000,0.000000,0.000000}%
\pgfsetfillcolor{currentfill}%
\pgfsetfillopacity{0.500000}%
\pgfsetlinewidth{1.003750pt}%
\definecolor{currentstroke}{rgb}{0.000000,0.000000,0.000000}%
\pgfsetstrokecolor{currentstroke}%
\pgfsetstrokeopacity{0.500000}%
\pgfsetdash{{3.700000pt}{1.600000pt}}{0.000000pt}%
\pgfpathmoveto{\pgfqpoint{1.979318in}{0.649628in}}%
\pgfpathcurveto{\pgfqpoint{1.985142in}{0.649628in}}{\pgfqpoint{1.990728in}{0.651942in}}{\pgfqpoint{1.994847in}{0.656060in}}%
\pgfpathcurveto{\pgfqpoint{1.998965in}{0.660179in}}{\pgfqpoint{2.001279in}{0.665765in}}{\pgfqpoint{2.001279in}{0.671589in}}%
\pgfpathcurveto{\pgfqpoint{2.001279in}{0.677413in}}{\pgfqpoint{1.998965in}{0.682999in}}{\pgfqpoint{1.994847in}{0.687117in}}%
\pgfpathcurveto{\pgfqpoint{1.990728in}{0.691235in}}{\pgfqpoint{1.985142in}{0.693549in}}{\pgfqpoint{1.979318in}{0.693549in}}%
\pgfpathcurveto{\pgfqpoint{1.973494in}{0.693549in}}{\pgfqpoint{1.967908in}{0.691235in}}{\pgfqpoint{1.963790in}{0.687117in}}%
\pgfpathcurveto{\pgfqpoint{1.959672in}{0.682999in}}{\pgfqpoint{1.957358in}{0.677413in}}{\pgfqpoint{1.957358in}{0.671589in}}%
\pgfpathcurveto{\pgfqpoint{1.957358in}{0.665765in}}{\pgfqpoint{1.959672in}{0.660179in}}{\pgfqpoint{1.963790in}{0.656060in}}%
\pgfpathcurveto{\pgfqpoint{1.967908in}{0.651942in}}{\pgfqpoint{1.973494in}{0.649628in}}{\pgfqpoint{1.979318in}{0.649628in}}%
\pgfpathlineto{\pgfqpoint{1.979318in}{0.649628in}}%
\pgfpathclose%
\pgfusepath{stroke,fill}%
\end{pgfscope}%
\begin{pgfscope}%
\pgfpathrectangle{\pgfqpoint{0.640323in}{0.527436in}}{\pgfqpoint{9.687500in}{3.850000in}}%
\pgfusepath{clip}%
\pgfsetbuttcap%
\pgfsetroundjoin%
\definecolor{currentfill}{rgb}{0.000000,0.000000,0.000000}%
\pgfsetfillcolor{currentfill}%
\pgfsetfillopacity{0.500000}%
\pgfsetlinewidth{1.003750pt}%
\definecolor{currentstroke}{rgb}{0.000000,0.000000,0.000000}%
\pgfsetstrokecolor{currentstroke}%
\pgfsetstrokeopacity{0.500000}%
\pgfsetdash{{3.700000pt}{1.600000pt}}{0.000000pt}%
\pgfpathmoveto{\pgfqpoint{2.159049in}{0.658756in}}%
\pgfpathcurveto{\pgfqpoint{2.164873in}{0.658756in}}{\pgfqpoint{2.170459in}{0.661070in}}{\pgfqpoint{2.174578in}{0.665188in}}%
\pgfpathcurveto{\pgfqpoint{2.178696in}{0.669306in}}{\pgfqpoint{2.181010in}{0.674892in}}{\pgfqpoint{2.181010in}{0.680716in}}%
\pgfpathcurveto{\pgfqpoint{2.181010in}{0.686540in}}{\pgfqpoint{2.178696in}{0.692126in}}{\pgfqpoint{2.174578in}{0.696244in}}%
\pgfpathcurveto{\pgfqpoint{2.170459in}{0.700362in}}{\pgfqpoint{2.164873in}{0.702676in}}{\pgfqpoint{2.159049in}{0.702676in}}%
\pgfpathcurveto{\pgfqpoint{2.153225in}{0.702676in}}{\pgfqpoint{2.147639in}{0.700362in}}{\pgfqpoint{2.143521in}{0.696244in}}%
\pgfpathcurveto{\pgfqpoint{2.139403in}{0.692126in}}{\pgfqpoint{2.137089in}{0.686540in}}{\pgfqpoint{2.137089in}{0.680716in}}%
\pgfpathcurveto{\pgfqpoint{2.137089in}{0.674892in}}{\pgfqpoint{2.139403in}{0.669306in}}{\pgfqpoint{2.143521in}{0.665188in}}%
\pgfpathcurveto{\pgfqpoint{2.147639in}{0.661070in}}{\pgfqpoint{2.153225in}{0.658756in}}{\pgfqpoint{2.159049in}{0.658756in}}%
\pgfpathlineto{\pgfqpoint{2.159049in}{0.658756in}}%
\pgfpathclose%
\pgfusepath{stroke,fill}%
\end{pgfscope}%
\begin{pgfscope}%
\pgfpathrectangle{\pgfqpoint{0.640323in}{0.527436in}}{\pgfqpoint{9.687500in}{3.850000in}}%
\pgfusepath{clip}%
\pgfsetbuttcap%
\pgfsetroundjoin%
\definecolor{currentfill}{rgb}{0.000000,0.000000,0.000000}%
\pgfsetfillcolor{currentfill}%
\pgfsetfillopacity{0.500000}%
\pgfsetlinewidth{1.003750pt}%
\definecolor{currentstroke}{rgb}{0.000000,0.000000,0.000000}%
\pgfsetstrokecolor{currentstroke}%
\pgfsetstrokeopacity{0.500000}%
\pgfsetdash{{3.700000pt}{1.600000pt}}{0.000000pt}%
\pgfpathmoveto{\pgfqpoint{2.338780in}{0.679264in}}%
\pgfpathcurveto{\pgfqpoint{2.344604in}{0.679264in}}{\pgfqpoint{2.350190in}{0.681577in}}{\pgfqpoint{2.354309in}{0.685696in}}%
\pgfpathcurveto{\pgfqpoint{2.358427in}{0.689814in}}{\pgfqpoint{2.360741in}{0.695400in}}{\pgfqpoint{2.360741in}{0.701224in}}%
\pgfpathcurveto{\pgfqpoint{2.360741in}{0.707048in}}{\pgfqpoint{2.358427in}{0.712634in}}{\pgfqpoint{2.354309in}{0.716752in}}%
\pgfpathcurveto{\pgfqpoint{2.350190in}{0.720870in}}{\pgfqpoint{2.344604in}{0.723184in}}{\pgfqpoint{2.338780in}{0.723184in}}%
\pgfpathcurveto{\pgfqpoint{2.332956in}{0.723184in}}{\pgfqpoint{2.327370in}{0.720870in}}{\pgfqpoint{2.323252in}{0.716752in}}%
\pgfpathcurveto{\pgfqpoint{2.319134in}{0.712634in}}{\pgfqpoint{2.316820in}{0.707048in}}{\pgfqpoint{2.316820in}{0.701224in}}%
\pgfpathcurveto{\pgfqpoint{2.316820in}{0.695400in}}{\pgfqpoint{2.319134in}{0.689814in}}{\pgfqpoint{2.323252in}{0.685696in}}%
\pgfpathcurveto{\pgfqpoint{2.327370in}{0.681577in}}{\pgfqpoint{2.332956in}{0.679264in}}{\pgfqpoint{2.338780in}{0.679264in}}%
\pgfpathlineto{\pgfqpoint{2.338780in}{0.679264in}}%
\pgfpathclose%
\pgfusepath{stroke,fill}%
\end{pgfscope}%
\begin{pgfscope}%
\pgfpathrectangle{\pgfqpoint{0.640323in}{0.527436in}}{\pgfqpoint{9.687500in}{3.850000in}}%
\pgfusepath{clip}%
\pgfsetbuttcap%
\pgfsetroundjoin%
\definecolor{currentfill}{rgb}{0.000000,0.000000,0.000000}%
\pgfsetfillcolor{currentfill}%
\pgfsetfillopacity{0.500000}%
\pgfsetlinewidth{1.003750pt}%
\definecolor{currentstroke}{rgb}{0.000000,0.000000,0.000000}%
\pgfsetstrokecolor{currentstroke}%
\pgfsetstrokeopacity{0.500000}%
\pgfsetdash{{3.700000pt}{1.600000pt}}{0.000000pt}%
\pgfpathmoveto{\pgfqpoint{2.518511in}{0.806554in}}%
\pgfpathcurveto{\pgfqpoint{2.524335in}{0.806554in}}{\pgfqpoint{2.529921in}{0.808868in}}{\pgfqpoint{2.534040in}{0.812986in}}%
\pgfpathcurveto{\pgfqpoint{2.538158in}{0.817104in}}{\pgfqpoint{2.540472in}{0.822691in}}{\pgfqpoint{2.540472in}{0.828515in}}%
\pgfpathcurveto{\pgfqpoint{2.540472in}{0.834338in}}{\pgfqpoint{2.538158in}{0.839925in}}{\pgfqpoint{2.534040in}{0.844043in}}%
\pgfpathcurveto{\pgfqpoint{2.529921in}{0.848161in}}{\pgfqpoint{2.524335in}{0.850475in}}{\pgfqpoint{2.518511in}{0.850475in}}%
\pgfpathcurveto{\pgfqpoint{2.512687in}{0.850475in}}{\pgfqpoint{2.507101in}{0.848161in}}{\pgfqpoint{2.502983in}{0.844043in}}%
\pgfpathcurveto{\pgfqpoint{2.498865in}{0.839925in}}{\pgfqpoint{2.496551in}{0.834338in}}{\pgfqpoint{2.496551in}{0.828515in}}%
\pgfpathcurveto{\pgfqpoint{2.496551in}{0.822691in}}{\pgfqpoint{2.498865in}{0.817104in}}{\pgfqpoint{2.502983in}{0.812986in}}%
\pgfpathcurveto{\pgfqpoint{2.507101in}{0.808868in}}{\pgfqpoint{2.512687in}{0.806554in}}{\pgfqpoint{2.518511in}{0.806554in}}%
\pgfpathlineto{\pgfqpoint{2.518511in}{0.806554in}}%
\pgfpathclose%
\pgfusepath{stroke,fill}%
\end{pgfscope}%
\begin{pgfscope}%
\pgfpathrectangle{\pgfqpoint{0.640323in}{0.527436in}}{\pgfqpoint{9.687500in}{3.850000in}}%
\pgfusepath{clip}%
\pgfsetbuttcap%
\pgfsetroundjoin%
\definecolor{currentfill}{rgb}{0.000000,0.000000,0.000000}%
\pgfsetfillcolor{currentfill}%
\pgfsetfillopacity{0.500000}%
\pgfsetlinewidth{1.003750pt}%
\definecolor{currentstroke}{rgb}{0.000000,0.000000,0.000000}%
\pgfsetstrokecolor{currentstroke}%
\pgfsetstrokeopacity{0.500000}%
\pgfsetdash{{3.700000pt}{1.600000pt}}{0.000000pt}%
\pgfpathmoveto{\pgfqpoint{2.698242in}{1.424061in}}%
\pgfpathcurveto{\pgfqpoint{2.704066in}{1.424061in}}{\pgfqpoint{2.709652in}{1.426375in}}{\pgfqpoint{2.713771in}{1.430493in}}%
\pgfpathcurveto{\pgfqpoint{2.717889in}{1.434611in}}{\pgfqpoint{2.720203in}{1.440197in}}{\pgfqpoint{2.720203in}{1.446021in}}%
\pgfpathcurveto{\pgfqpoint{2.720203in}{1.451845in}}{\pgfqpoint{2.717889in}{1.457431in}}{\pgfqpoint{2.713771in}{1.461549in}}%
\pgfpathcurveto{\pgfqpoint{2.709652in}{1.465667in}}{\pgfqpoint{2.704066in}{1.467981in}}{\pgfqpoint{2.698242in}{1.467981in}}%
\pgfpathcurveto{\pgfqpoint{2.692418in}{1.467981in}}{\pgfqpoint{2.686832in}{1.465667in}}{\pgfqpoint{2.682714in}{1.461549in}}%
\pgfpathcurveto{\pgfqpoint{2.678596in}{1.457431in}}{\pgfqpoint{2.676282in}{1.451845in}}{\pgfqpoint{2.676282in}{1.446021in}}%
\pgfpathcurveto{\pgfqpoint{2.676282in}{1.440197in}}{\pgfqpoint{2.678596in}{1.434611in}}{\pgfqpoint{2.682714in}{1.430493in}}%
\pgfpathcurveto{\pgfqpoint{2.686832in}{1.426375in}}{\pgfqpoint{2.692418in}{1.424061in}}{\pgfqpoint{2.698242in}{1.424061in}}%
\pgfpathlineto{\pgfqpoint{2.698242in}{1.424061in}}%
\pgfpathclose%
\pgfusepath{stroke,fill}%
\end{pgfscope}%
\begin{pgfscope}%
\pgfpathrectangle{\pgfqpoint{0.640323in}{0.527436in}}{\pgfqpoint{9.687500in}{3.850000in}}%
\pgfusepath{clip}%
\pgfsetbuttcap%
\pgfsetroundjoin%
\definecolor{currentfill}{rgb}{0.000000,0.000000,0.000000}%
\pgfsetfillcolor{currentfill}%
\pgfsetfillopacity{0.500000}%
\pgfsetlinewidth{1.003750pt}%
\definecolor{currentstroke}{rgb}{0.000000,0.000000,0.000000}%
\pgfsetstrokecolor{currentstroke}%
\pgfsetstrokeopacity{0.500000}%
\pgfsetdash{{3.700000pt}{1.600000pt}}{0.000000pt}%
\pgfpathmoveto{\pgfqpoint{2.877973in}{1.818922in}}%
\pgfpathcurveto{\pgfqpoint{2.883797in}{1.818922in}}{\pgfqpoint{2.889383in}{1.821236in}}{\pgfqpoint{2.893501in}{1.825354in}}%
\pgfpathcurveto{\pgfqpoint{2.897620in}{1.829472in}}{\pgfqpoint{2.899934in}{1.835058in}}{\pgfqpoint{2.899934in}{1.840882in}}%
\pgfpathcurveto{\pgfqpoint{2.899934in}{1.846706in}}{\pgfqpoint{2.897620in}{1.852292in}}{\pgfqpoint{2.893501in}{1.856410in}}%
\pgfpathcurveto{\pgfqpoint{2.889383in}{1.860528in}}{\pgfqpoint{2.883797in}{1.862842in}}{\pgfqpoint{2.877973in}{1.862842in}}%
\pgfpathcurveto{\pgfqpoint{2.872149in}{1.862842in}}{\pgfqpoint{2.866563in}{1.860528in}}{\pgfqpoint{2.862445in}{1.856410in}}%
\pgfpathcurveto{\pgfqpoint{2.858327in}{1.852292in}}{\pgfqpoint{2.856013in}{1.846706in}}{\pgfqpoint{2.856013in}{1.840882in}}%
\pgfpathcurveto{\pgfqpoint{2.856013in}{1.835058in}}{\pgfqpoint{2.858327in}{1.829472in}}{\pgfqpoint{2.862445in}{1.825354in}}%
\pgfpathcurveto{\pgfqpoint{2.866563in}{1.821236in}}{\pgfqpoint{2.872149in}{1.818922in}}{\pgfqpoint{2.877973in}{1.818922in}}%
\pgfpathlineto{\pgfqpoint{2.877973in}{1.818922in}}%
\pgfpathclose%
\pgfusepath{stroke,fill}%
\end{pgfscope}%
\begin{pgfscope}%
\pgfpathrectangle{\pgfqpoint{0.640323in}{0.527436in}}{\pgfqpoint{9.687500in}{3.850000in}}%
\pgfusepath{clip}%
\pgfsetbuttcap%
\pgfsetroundjoin%
\definecolor{currentfill}{rgb}{0.000000,0.000000,0.000000}%
\pgfsetfillcolor{currentfill}%
\pgfsetfillopacity{0.500000}%
\pgfsetlinewidth{1.003750pt}%
\definecolor{currentstroke}{rgb}{0.000000,0.000000,0.000000}%
\pgfsetstrokecolor{currentstroke}%
\pgfsetstrokeopacity{0.500000}%
\pgfsetdash{{3.700000pt}{1.600000pt}}{0.000000pt}%
\pgfpathmoveto{\pgfqpoint{3.057704in}{2.105666in}}%
\pgfpathcurveto{\pgfqpoint{3.063528in}{2.105666in}}{\pgfqpoint{3.069114in}{2.107980in}}{\pgfqpoint{3.073232in}{2.112098in}}%
\pgfpathcurveto{\pgfqpoint{3.077351in}{2.116216in}}{\pgfqpoint{3.079664in}{2.121802in}}{\pgfqpoint{3.079664in}{2.127626in}}%
\pgfpathcurveto{\pgfqpoint{3.079664in}{2.133450in}}{\pgfqpoint{3.077351in}{2.139036in}}{\pgfqpoint{3.073232in}{2.143154in}}%
\pgfpathcurveto{\pgfqpoint{3.069114in}{2.147273in}}{\pgfqpoint{3.063528in}{2.149586in}}{\pgfqpoint{3.057704in}{2.149586in}}%
\pgfpathcurveto{\pgfqpoint{3.051880in}{2.149586in}}{\pgfqpoint{3.046294in}{2.147273in}}{\pgfqpoint{3.042176in}{2.143154in}}%
\pgfpathcurveto{\pgfqpoint{3.038058in}{2.139036in}}{\pgfqpoint{3.035744in}{2.133450in}}{\pgfqpoint{3.035744in}{2.127626in}}%
\pgfpathcurveto{\pgfqpoint{3.035744in}{2.121802in}}{\pgfqpoint{3.038058in}{2.116216in}}{\pgfqpoint{3.042176in}{2.112098in}}%
\pgfpathcurveto{\pgfqpoint{3.046294in}{2.107980in}}{\pgfqpoint{3.051880in}{2.105666in}}{\pgfqpoint{3.057704in}{2.105666in}}%
\pgfpathlineto{\pgfqpoint{3.057704in}{2.105666in}}%
\pgfpathclose%
\pgfusepath{stroke,fill}%
\end{pgfscope}%
\begin{pgfscope}%
\pgfpathrectangle{\pgfqpoint{0.640323in}{0.527436in}}{\pgfqpoint{9.687500in}{3.850000in}}%
\pgfusepath{clip}%
\pgfsetbuttcap%
\pgfsetroundjoin%
\definecolor{currentfill}{rgb}{0.000000,0.000000,0.000000}%
\pgfsetfillcolor{currentfill}%
\pgfsetfillopacity{0.500000}%
\pgfsetlinewidth{1.003750pt}%
\definecolor{currentstroke}{rgb}{0.000000,0.000000,0.000000}%
\pgfsetstrokecolor{currentstroke}%
\pgfsetstrokeopacity{0.500000}%
\pgfsetdash{{3.700000pt}{1.600000pt}}{0.000000pt}%
\pgfpathmoveto{\pgfqpoint{3.237435in}{2.312846in}}%
\pgfpathcurveto{\pgfqpoint{3.243259in}{2.312846in}}{\pgfqpoint{3.248845in}{2.315160in}}{\pgfqpoint{3.252963in}{2.319278in}}%
\pgfpathcurveto{\pgfqpoint{3.257082in}{2.323396in}}{\pgfqpoint{3.259395in}{2.328982in}}{\pgfqpoint{3.259395in}{2.334806in}}%
\pgfpathcurveto{\pgfqpoint{3.259395in}{2.340630in}}{\pgfqpoint{3.257082in}{2.346216in}}{\pgfqpoint{3.252963in}{2.350334in}}%
\pgfpathcurveto{\pgfqpoint{3.248845in}{2.354452in}}{\pgfqpoint{3.243259in}{2.356766in}}{\pgfqpoint{3.237435in}{2.356766in}}%
\pgfpathcurveto{\pgfqpoint{3.231611in}{2.356766in}}{\pgfqpoint{3.226025in}{2.354452in}}{\pgfqpoint{3.221907in}{2.350334in}}%
\pgfpathcurveto{\pgfqpoint{3.217789in}{2.346216in}}{\pgfqpoint{3.215475in}{2.340630in}}{\pgfqpoint{3.215475in}{2.334806in}}%
\pgfpathcurveto{\pgfqpoint{3.215475in}{2.328982in}}{\pgfqpoint{3.217789in}{2.323396in}}{\pgfqpoint{3.221907in}{2.319278in}}%
\pgfpathcurveto{\pgfqpoint{3.226025in}{2.315160in}}{\pgfqpoint{3.231611in}{2.312846in}}{\pgfqpoint{3.237435in}{2.312846in}}%
\pgfpathlineto{\pgfqpoint{3.237435in}{2.312846in}}%
\pgfpathclose%
\pgfusepath{stroke,fill}%
\end{pgfscope}%
\begin{pgfscope}%
\pgfpathrectangle{\pgfqpoint{0.640323in}{0.527436in}}{\pgfqpoint{9.687500in}{3.850000in}}%
\pgfusepath{clip}%
\pgfsetbuttcap%
\pgfsetroundjoin%
\definecolor{currentfill}{rgb}{0.000000,0.000000,0.000000}%
\pgfsetfillcolor{currentfill}%
\pgfsetfillopacity{0.500000}%
\pgfsetlinewidth{1.003750pt}%
\definecolor{currentstroke}{rgb}{0.000000,0.000000,0.000000}%
\pgfsetstrokecolor{currentstroke}%
\pgfsetstrokeopacity{0.500000}%
\pgfsetdash{{3.700000pt}{1.600000pt}}{0.000000pt}%
\pgfpathmoveto{\pgfqpoint{3.417166in}{2.481823in}}%
\pgfpathcurveto{\pgfqpoint{3.422990in}{2.481823in}}{\pgfqpoint{3.428576in}{2.484137in}}{\pgfqpoint{3.432694in}{2.488255in}}%
\pgfpathcurveto{\pgfqpoint{3.436813in}{2.492374in}}{\pgfqpoint{3.439126in}{2.497960in}}{\pgfqpoint{3.439126in}{2.503784in}}%
\pgfpathcurveto{\pgfqpoint{3.439126in}{2.509608in}}{\pgfqpoint{3.436813in}{2.515194in}}{\pgfqpoint{3.432694in}{2.519312in}}%
\pgfpathcurveto{\pgfqpoint{3.428576in}{2.523430in}}{\pgfqpoint{3.422990in}{2.525744in}}{\pgfqpoint{3.417166in}{2.525744in}}%
\pgfpathcurveto{\pgfqpoint{3.411342in}{2.525744in}}{\pgfqpoint{3.405756in}{2.523430in}}{\pgfqpoint{3.401638in}{2.519312in}}%
\pgfpathcurveto{\pgfqpoint{3.397520in}{2.515194in}}{\pgfqpoint{3.395206in}{2.509608in}}{\pgfqpoint{3.395206in}{2.503784in}}%
\pgfpathcurveto{\pgfqpoint{3.395206in}{2.497960in}}{\pgfqpoint{3.397520in}{2.492374in}}{\pgfqpoint{3.401638in}{2.488255in}}%
\pgfpathcurveto{\pgfqpoint{3.405756in}{2.484137in}}{\pgfqpoint{3.411342in}{2.481823in}}{\pgfqpoint{3.417166in}{2.481823in}}%
\pgfpathlineto{\pgfqpoint{3.417166in}{2.481823in}}%
\pgfpathclose%
\pgfusepath{stroke,fill}%
\end{pgfscope}%
\begin{pgfscope}%
\pgfpathrectangle{\pgfqpoint{0.640323in}{0.527436in}}{\pgfqpoint{9.687500in}{3.850000in}}%
\pgfusepath{clip}%
\pgfsetbuttcap%
\pgfsetroundjoin%
\definecolor{currentfill}{rgb}{0.000000,0.000000,0.000000}%
\pgfsetfillcolor{currentfill}%
\pgfsetfillopacity{0.500000}%
\pgfsetlinewidth{1.003750pt}%
\definecolor{currentstroke}{rgb}{0.000000,0.000000,0.000000}%
\pgfsetstrokecolor{currentstroke}%
\pgfsetstrokeopacity{0.500000}%
\pgfsetdash{{3.700000pt}{1.600000pt}}{0.000000pt}%
\pgfpathmoveto{\pgfqpoint{3.596897in}{2.629663in}}%
\pgfpathcurveto{\pgfqpoint{3.602721in}{2.629663in}}{\pgfqpoint{3.608307in}{2.631977in}}{\pgfqpoint{3.612425in}{2.636095in}}%
\pgfpathcurveto{\pgfqpoint{3.616544in}{2.640214in}}{\pgfqpoint{3.618857in}{2.645800in}}{\pgfqpoint{3.618857in}{2.651624in}}%
\pgfpathcurveto{\pgfqpoint{3.618857in}{2.657448in}}{\pgfqpoint{3.616544in}{2.663034in}}{\pgfqpoint{3.612425in}{2.667152in}}%
\pgfpathcurveto{\pgfqpoint{3.608307in}{2.671270in}}{\pgfqpoint{3.602721in}{2.673584in}}{\pgfqpoint{3.596897in}{2.673584in}}%
\pgfpathcurveto{\pgfqpoint{3.591073in}{2.673584in}}{\pgfqpoint{3.585487in}{2.671270in}}{\pgfqpoint{3.581369in}{2.667152in}}%
\pgfpathcurveto{\pgfqpoint{3.577251in}{2.663034in}}{\pgfqpoint{3.574937in}{2.657448in}}{\pgfqpoint{3.574937in}{2.651624in}}%
\pgfpathcurveto{\pgfqpoint{3.574937in}{2.645800in}}{\pgfqpoint{3.577251in}{2.640214in}}{\pgfqpoint{3.581369in}{2.636095in}}%
\pgfpathcurveto{\pgfqpoint{3.585487in}{2.631977in}}{\pgfqpoint{3.591073in}{2.629663in}}{\pgfqpoint{3.596897in}{2.629663in}}%
\pgfpathlineto{\pgfqpoint{3.596897in}{2.629663in}}%
\pgfpathclose%
\pgfusepath{stroke,fill}%
\end{pgfscope}%
\begin{pgfscope}%
\pgfpathrectangle{\pgfqpoint{0.640323in}{0.527436in}}{\pgfqpoint{9.687500in}{3.850000in}}%
\pgfusepath{clip}%
\pgfsetbuttcap%
\pgfsetroundjoin%
\definecolor{currentfill}{rgb}{0.000000,0.000000,0.000000}%
\pgfsetfillcolor{currentfill}%
\pgfsetfillopacity{0.500000}%
\pgfsetlinewidth{1.003750pt}%
\definecolor{currentstroke}{rgb}{0.000000,0.000000,0.000000}%
\pgfsetstrokecolor{currentstroke}%
\pgfsetstrokeopacity{0.500000}%
\pgfsetdash{{3.700000pt}{1.600000pt}}{0.000000pt}%
\pgfpathmoveto{\pgfqpoint{3.776628in}{2.750510in}}%
\pgfpathcurveto{\pgfqpoint{3.782452in}{2.750510in}}{\pgfqpoint{3.788038in}{2.752824in}}{\pgfqpoint{3.792156in}{2.756942in}}%
\pgfpathcurveto{\pgfqpoint{3.796275in}{2.761060in}}{\pgfqpoint{3.798588in}{2.766646in}}{\pgfqpoint{3.798588in}{2.772470in}}%
\pgfpathcurveto{\pgfqpoint{3.798588in}{2.778294in}}{\pgfqpoint{3.796275in}{2.783880in}}{\pgfqpoint{3.792156in}{2.787998in}}%
\pgfpathcurveto{\pgfqpoint{3.788038in}{2.792117in}}{\pgfqpoint{3.782452in}{2.794430in}}{\pgfqpoint{3.776628in}{2.794430in}}%
\pgfpathcurveto{\pgfqpoint{3.770804in}{2.794430in}}{\pgfqpoint{3.765218in}{2.792117in}}{\pgfqpoint{3.761100in}{2.787998in}}%
\pgfpathcurveto{\pgfqpoint{3.756982in}{2.783880in}}{\pgfqpoint{3.754668in}{2.778294in}}{\pgfqpoint{3.754668in}{2.772470in}}%
\pgfpathcurveto{\pgfqpoint{3.754668in}{2.766646in}}{\pgfqpoint{3.756982in}{2.761060in}}{\pgfqpoint{3.761100in}{2.756942in}}%
\pgfpathcurveto{\pgfqpoint{3.765218in}{2.752824in}}{\pgfqpoint{3.770804in}{2.750510in}}{\pgfqpoint{3.776628in}{2.750510in}}%
\pgfpathlineto{\pgfqpoint{3.776628in}{2.750510in}}%
\pgfpathclose%
\pgfusepath{stroke,fill}%
\end{pgfscope}%
\begin{pgfscope}%
\pgfpathrectangle{\pgfqpoint{0.640323in}{0.527436in}}{\pgfqpoint{9.687500in}{3.850000in}}%
\pgfusepath{clip}%
\pgfsetbuttcap%
\pgfsetroundjoin%
\definecolor{currentfill}{rgb}{0.000000,0.000000,0.000000}%
\pgfsetfillcolor{currentfill}%
\pgfsetfillopacity{0.500000}%
\pgfsetlinewidth{1.003750pt}%
\definecolor{currentstroke}{rgb}{0.000000,0.000000,0.000000}%
\pgfsetstrokecolor{currentstroke}%
\pgfsetstrokeopacity{0.500000}%
\pgfsetdash{{3.700000pt}{1.600000pt}}{0.000000pt}%
\pgfpathmoveto{\pgfqpoint{3.956359in}{2.860483in}}%
\pgfpathcurveto{\pgfqpoint{3.962183in}{2.860483in}}{\pgfqpoint{3.967769in}{2.862797in}}{\pgfqpoint{3.971887in}{2.866915in}}%
\pgfpathcurveto{\pgfqpoint{3.976006in}{2.871033in}}{\pgfqpoint{3.978319in}{2.876620in}}{\pgfqpoint{3.978319in}{2.882444in}}%
\pgfpathcurveto{\pgfqpoint{3.978319in}{2.888268in}}{\pgfqpoint{3.976006in}{2.893854in}}{\pgfqpoint{3.971887in}{2.897972in}}%
\pgfpathcurveto{\pgfqpoint{3.967769in}{2.902090in}}{\pgfqpoint{3.962183in}{2.904404in}}{\pgfqpoint{3.956359in}{2.904404in}}%
\pgfpathcurveto{\pgfqpoint{3.950535in}{2.904404in}}{\pgfqpoint{3.944949in}{2.902090in}}{\pgfqpoint{3.940831in}{2.897972in}}%
\pgfpathcurveto{\pgfqpoint{3.936713in}{2.893854in}}{\pgfqpoint{3.934399in}{2.888268in}}{\pgfqpoint{3.934399in}{2.882444in}}%
\pgfpathcurveto{\pgfqpoint{3.934399in}{2.876620in}}{\pgfqpoint{3.936713in}{2.871033in}}{\pgfqpoint{3.940831in}{2.866915in}}%
\pgfpathcurveto{\pgfqpoint{3.944949in}{2.862797in}}{\pgfqpoint{3.950535in}{2.860483in}}{\pgfqpoint{3.956359in}{2.860483in}}%
\pgfpathlineto{\pgfqpoint{3.956359in}{2.860483in}}%
\pgfpathclose%
\pgfusepath{stroke,fill}%
\end{pgfscope}%
\begin{pgfscope}%
\pgfpathrectangle{\pgfqpoint{0.640323in}{0.527436in}}{\pgfqpoint{9.687500in}{3.850000in}}%
\pgfusepath{clip}%
\pgfsetbuttcap%
\pgfsetroundjoin%
\definecolor{currentfill}{rgb}{0.000000,0.000000,0.000000}%
\pgfsetfillcolor{currentfill}%
\pgfsetfillopacity{0.500000}%
\pgfsetlinewidth{1.003750pt}%
\definecolor{currentstroke}{rgb}{0.000000,0.000000,0.000000}%
\pgfsetstrokecolor{currentstroke}%
\pgfsetstrokeopacity{0.500000}%
\pgfsetdash{{3.700000pt}{1.600000pt}}{0.000000pt}%
\pgfpathmoveto{\pgfqpoint{4.136090in}{2.951499in}}%
\pgfpathcurveto{\pgfqpoint{4.141914in}{2.951499in}}{\pgfqpoint{4.147500in}{2.953812in}}{\pgfqpoint{4.151618in}{2.957931in}}%
\pgfpathcurveto{\pgfqpoint{4.155737in}{2.962049in}}{\pgfqpoint{4.158050in}{2.967635in}}{\pgfqpoint{4.158050in}{2.973459in}}%
\pgfpathcurveto{\pgfqpoint{4.158050in}{2.979283in}}{\pgfqpoint{4.155737in}{2.984869in}}{\pgfqpoint{4.151618in}{2.988987in}}%
\pgfpathcurveto{\pgfqpoint{4.147500in}{2.993105in}}{\pgfqpoint{4.141914in}{2.995419in}}{\pgfqpoint{4.136090in}{2.995419in}}%
\pgfpathcurveto{\pgfqpoint{4.130266in}{2.995419in}}{\pgfqpoint{4.124680in}{2.993105in}}{\pgfqpoint{4.120562in}{2.988987in}}%
\pgfpathcurveto{\pgfqpoint{4.116444in}{2.984869in}}{\pgfqpoint{4.114130in}{2.979283in}}{\pgfqpoint{4.114130in}{2.973459in}}%
\pgfpathcurveto{\pgfqpoint{4.114130in}{2.967635in}}{\pgfqpoint{4.116444in}{2.962049in}}{\pgfqpoint{4.120562in}{2.957931in}}%
\pgfpathcurveto{\pgfqpoint{4.124680in}{2.953812in}}{\pgfqpoint{4.130266in}{2.951499in}}{\pgfqpoint{4.136090in}{2.951499in}}%
\pgfpathlineto{\pgfqpoint{4.136090in}{2.951499in}}%
\pgfpathclose%
\pgfusepath{stroke,fill}%
\end{pgfscope}%
\begin{pgfscope}%
\pgfpathrectangle{\pgfqpoint{0.640323in}{0.527436in}}{\pgfqpoint{9.687500in}{3.850000in}}%
\pgfusepath{clip}%
\pgfsetbuttcap%
\pgfsetroundjoin%
\definecolor{currentfill}{rgb}{0.000000,0.000000,0.000000}%
\pgfsetfillcolor{currentfill}%
\pgfsetfillopacity{0.500000}%
\pgfsetlinewidth{1.003750pt}%
\definecolor{currentstroke}{rgb}{0.000000,0.000000,0.000000}%
\pgfsetstrokecolor{currentstroke}%
\pgfsetstrokeopacity{0.500000}%
\pgfsetdash{{3.700000pt}{1.600000pt}}{0.000000pt}%
\pgfpathmoveto{\pgfqpoint{4.315821in}{3.036372in}}%
\pgfpathcurveto{\pgfqpoint{4.321645in}{3.036372in}}{\pgfqpoint{4.327231in}{3.038686in}}{\pgfqpoint{4.331349in}{3.042804in}}%
\pgfpathcurveto{\pgfqpoint{4.335467in}{3.046923in}}{\pgfqpoint{4.337781in}{3.052509in}}{\pgfqpoint{4.337781in}{3.058333in}}%
\pgfpathcurveto{\pgfqpoint{4.337781in}{3.064157in}}{\pgfqpoint{4.335467in}{3.069743in}}{\pgfqpoint{4.331349in}{3.073861in}}%
\pgfpathcurveto{\pgfqpoint{4.327231in}{3.077979in}}{\pgfqpoint{4.321645in}{3.080293in}}{\pgfqpoint{4.315821in}{3.080293in}}%
\pgfpathcurveto{\pgfqpoint{4.309997in}{3.080293in}}{\pgfqpoint{4.304411in}{3.077979in}}{\pgfqpoint{4.300293in}{3.073861in}}%
\pgfpathcurveto{\pgfqpoint{4.296175in}{3.069743in}}{\pgfqpoint{4.293861in}{3.064157in}}{\pgfqpoint{4.293861in}{3.058333in}}%
\pgfpathcurveto{\pgfqpoint{4.293861in}{3.052509in}}{\pgfqpoint{4.296175in}{3.046923in}}{\pgfqpoint{4.300293in}{3.042804in}}%
\pgfpathcurveto{\pgfqpoint{4.304411in}{3.038686in}}{\pgfqpoint{4.309997in}{3.036372in}}{\pgfqpoint{4.315821in}{3.036372in}}%
\pgfpathlineto{\pgfqpoint{4.315821in}{3.036372in}}%
\pgfpathclose%
\pgfusepath{stroke,fill}%
\end{pgfscope}%
\begin{pgfscope}%
\pgfpathrectangle{\pgfqpoint{0.640323in}{0.527436in}}{\pgfqpoint{9.687500in}{3.850000in}}%
\pgfusepath{clip}%
\pgfsetbuttcap%
\pgfsetroundjoin%
\definecolor{currentfill}{rgb}{0.000000,0.000000,0.000000}%
\pgfsetfillcolor{currentfill}%
\pgfsetfillopacity{0.500000}%
\pgfsetlinewidth{1.003750pt}%
\definecolor{currentstroke}{rgb}{0.000000,0.000000,0.000000}%
\pgfsetstrokecolor{currentstroke}%
\pgfsetstrokeopacity{0.500000}%
\pgfsetdash{{3.700000pt}{1.600000pt}}{0.000000pt}%
\pgfpathmoveto{\pgfqpoint{4.495552in}{3.110274in}}%
\pgfpathcurveto{\pgfqpoint{4.501376in}{3.110274in}}{\pgfqpoint{4.506962in}{3.112588in}}{\pgfqpoint{4.511080in}{3.116706in}}%
\pgfpathcurveto{\pgfqpoint{4.515198in}{3.120824in}}{\pgfqpoint{4.517512in}{3.126410in}}{\pgfqpoint{4.517512in}{3.132234in}}%
\pgfpathcurveto{\pgfqpoint{4.517512in}{3.138058in}}{\pgfqpoint{4.515198in}{3.143644in}}{\pgfqpoint{4.511080in}{3.147762in}}%
\pgfpathcurveto{\pgfqpoint{4.506962in}{3.151880in}}{\pgfqpoint{4.501376in}{3.154194in}}{\pgfqpoint{4.495552in}{3.154194in}}%
\pgfpathcurveto{\pgfqpoint{4.489728in}{3.154194in}}{\pgfqpoint{4.484142in}{3.151880in}}{\pgfqpoint{4.480024in}{3.147762in}}%
\pgfpathcurveto{\pgfqpoint{4.475906in}{3.143644in}}{\pgfqpoint{4.473592in}{3.138058in}}{\pgfqpoint{4.473592in}{3.132234in}}%
\pgfpathcurveto{\pgfqpoint{4.473592in}{3.126410in}}{\pgfqpoint{4.475906in}{3.120824in}}{\pgfqpoint{4.480024in}{3.116706in}}%
\pgfpathcurveto{\pgfqpoint{4.484142in}{3.112588in}}{\pgfqpoint{4.489728in}{3.110274in}}{\pgfqpoint{4.495552in}{3.110274in}}%
\pgfpathlineto{\pgfqpoint{4.495552in}{3.110274in}}%
\pgfpathclose%
\pgfusepath{stroke,fill}%
\end{pgfscope}%
\begin{pgfscope}%
\pgfpathrectangle{\pgfqpoint{0.640323in}{0.527436in}}{\pgfqpoint{9.687500in}{3.850000in}}%
\pgfusepath{clip}%
\pgfsetbuttcap%
\pgfsetroundjoin%
\definecolor{currentfill}{rgb}{0.000000,0.000000,0.000000}%
\pgfsetfillcolor{currentfill}%
\pgfsetfillopacity{0.500000}%
\pgfsetlinewidth{1.003750pt}%
\definecolor{currentstroke}{rgb}{0.000000,0.000000,0.000000}%
\pgfsetstrokecolor{currentstroke}%
\pgfsetstrokeopacity{0.500000}%
\pgfsetdash{{3.700000pt}{1.600000pt}}{0.000000pt}%
\pgfpathmoveto{\pgfqpoint{4.675283in}{3.181493in}}%
\pgfpathcurveto{\pgfqpoint{4.681107in}{3.181493in}}{\pgfqpoint{4.686693in}{3.183806in}}{\pgfqpoint{4.690811in}{3.187925in}}%
\pgfpathcurveto{\pgfqpoint{4.694929in}{3.192043in}}{\pgfqpoint{4.697243in}{3.197629in}}{\pgfqpoint{4.697243in}{3.203453in}}%
\pgfpathcurveto{\pgfqpoint{4.697243in}{3.209277in}}{\pgfqpoint{4.694929in}{3.214863in}}{\pgfqpoint{4.690811in}{3.218981in}}%
\pgfpathcurveto{\pgfqpoint{4.686693in}{3.223099in}}{\pgfqpoint{4.681107in}{3.225413in}}{\pgfqpoint{4.675283in}{3.225413in}}%
\pgfpathcurveto{\pgfqpoint{4.669459in}{3.225413in}}{\pgfqpoint{4.663873in}{3.223099in}}{\pgfqpoint{4.659755in}{3.218981in}}%
\pgfpathcurveto{\pgfqpoint{4.655637in}{3.214863in}}{\pgfqpoint{4.653323in}{3.209277in}}{\pgfqpoint{4.653323in}{3.203453in}}%
\pgfpathcurveto{\pgfqpoint{4.653323in}{3.197629in}}{\pgfqpoint{4.655637in}{3.192043in}}{\pgfqpoint{4.659755in}{3.187925in}}%
\pgfpathcurveto{\pgfqpoint{4.663873in}{3.183806in}}{\pgfqpoint{4.669459in}{3.181493in}}{\pgfqpoint{4.675283in}{3.181493in}}%
\pgfpathlineto{\pgfqpoint{4.675283in}{3.181493in}}%
\pgfpathclose%
\pgfusepath{stroke,fill}%
\end{pgfscope}%
\begin{pgfscope}%
\pgfpathrectangle{\pgfqpoint{0.640323in}{0.527436in}}{\pgfqpoint{9.687500in}{3.850000in}}%
\pgfusepath{clip}%
\pgfsetbuttcap%
\pgfsetroundjoin%
\definecolor{currentfill}{rgb}{0.000000,0.000000,0.000000}%
\pgfsetfillcolor{currentfill}%
\pgfsetfillopacity{0.500000}%
\pgfsetlinewidth{1.003750pt}%
\definecolor{currentstroke}{rgb}{0.000000,0.000000,0.000000}%
\pgfsetstrokecolor{currentstroke}%
\pgfsetstrokeopacity{0.500000}%
\pgfsetdash{{3.700000pt}{1.600000pt}}{0.000000pt}%
\pgfpathmoveto{\pgfqpoint{4.855014in}{3.241199in}}%
\pgfpathcurveto{\pgfqpoint{4.860838in}{3.241199in}}{\pgfqpoint{4.866424in}{3.243513in}}{\pgfqpoint{4.870542in}{3.247631in}}%
\pgfpathcurveto{\pgfqpoint{4.874660in}{3.251749in}}{\pgfqpoint{4.876974in}{3.257335in}}{\pgfqpoint{4.876974in}{3.263159in}}%
\pgfpathcurveto{\pgfqpoint{4.876974in}{3.268983in}}{\pgfqpoint{4.874660in}{3.274569in}}{\pgfqpoint{4.870542in}{3.278687in}}%
\pgfpathcurveto{\pgfqpoint{4.866424in}{3.282805in}}{\pgfqpoint{4.860838in}{3.285119in}}{\pgfqpoint{4.855014in}{3.285119in}}%
\pgfpathcurveto{\pgfqpoint{4.849190in}{3.285119in}}{\pgfqpoint{4.843604in}{3.282805in}}{\pgfqpoint{4.839486in}{3.278687in}}%
\pgfpathcurveto{\pgfqpoint{4.835368in}{3.274569in}}{\pgfqpoint{4.833054in}{3.268983in}}{\pgfqpoint{4.833054in}{3.263159in}}%
\pgfpathcurveto{\pgfqpoint{4.833054in}{3.257335in}}{\pgfqpoint{4.835368in}{3.251749in}}{\pgfqpoint{4.839486in}{3.247631in}}%
\pgfpathcurveto{\pgfqpoint{4.843604in}{3.243513in}}{\pgfqpoint{4.849190in}{3.241199in}}{\pgfqpoint{4.855014in}{3.241199in}}%
\pgfpathlineto{\pgfqpoint{4.855014in}{3.241199in}}%
\pgfpathclose%
\pgfusepath{stroke,fill}%
\end{pgfscope}%
\begin{pgfscope}%
\pgfpathrectangle{\pgfqpoint{0.640323in}{0.527436in}}{\pgfqpoint{9.687500in}{3.850000in}}%
\pgfusepath{clip}%
\pgfsetbuttcap%
\pgfsetroundjoin%
\definecolor{currentfill}{rgb}{0.000000,0.000000,0.000000}%
\pgfsetfillcolor{currentfill}%
\pgfsetfillopacity{0.500000}%
\pgfsetlinewidth{1.003750pt}%
\definecolor{currentstroke}{rgb}{0.000000,0.000000,0.000000}%
\pgfsetstrokecolor{currentstroke}%
\pgfsetstrokeopacity{0.500000}%
\pgfsetdash{{3.700000pt}{1.600000pt}}{0.000000pt}%
\pgfpathmoveto{\pgfqpoint{5.034745in}{3.293813in}}%
\pgfpathcurveto{\pgfqpoint{5.040569in}{3.293813in}}{\pgfqpoint{5.046155in}{3.296127in}}{\pgfqpoint{5.050273in}{3.300245in}}%
\pgfpathcurveto{\pgfqpoint{5.054391in}{3.304363in}}{\pgfqpoint{5.056705in}{3.309950in}}{\pgfqpoint{5.056705in}{3.315773in}}%
\pgfpathcurveto{\pgfqpoint{5.056705in}{3.321597in}}{\pgfqpoint{5.054391in}{3.327184in}}{\pgfqpoint{5.050273in}{3.331302in}}%
\pgfpathcurveto{\pgfqpoint{5.046155in}{3.335420in}}{\pgfqpoint{5.040569in}{3.337734in}}{\pgfqpoint{5.034745in}{3.337734in}}%
\pgfpathcurveto{\pgfqpoint{5.028921in}{3.337734in}}{\pgfqpoint{5.023335in}{3.335420in}}{\pgfqpoint{5.019217in}{3.331302in}}%
\pgfpathcurveto{\pgfqpoint{5.015099in}{3.327184in}}{\pgfqpoint{5.012785in}{3.321597in}}{\pgfqpoint{5.012785in}{3.315773in}}%
\pgfpathcurveto{\pgfqpoint{5.012785in}{3.309950in}}{\pgfqpoint{5.015099in}{3.304363in}}{\pgfqpoint{5.019217in}{3.300245in}}%
\pgfpathcurveto{\pgfqpoint{5.023335in}{3.296127in}}{\pgfqpoint{5.028921in}{3.293813in}}{\pgfqpoint{5.034745in}{3.293813in}}%
\pgfpathlineto{\pgfqpoint{5.034745in}{3.293813in}}%
\pgfpathclose%
\pgfusepath{stroke,fill}%
\end{pgfscope}%
\begin{pgfscope}%
\pgfpathrectangle{\pgfqpoint{0.640323in}{0.527436in}}{\pgfqpoint{9.687500in}{3.850000in}}%
\pgfusepath{clip}%
\pgfsetbuttcap%
\pgfsetroundjoin%
\definecolor{currentfill}{rgb}{0.000000,0.000000,0.000000}%
\pgfsetfillcolor{currentfill}%
\pgfsetfillopacity{0.500000}%
\pgfsetlinewidth{1.003750pt}%
\definecolor{currentstroke}{rgb}{0.000000,0.000000,0.000000}%
\pgfsetstrokecolor{currentstroke}%
\pgfsetstrokeopacity{0.500000}%
\pgfsetdash{{3.700000pt}{1.600000pt}}{0.000000pt}%
\pgfpathmoveto{\pgfqpoint{5.214476in}{3.351414in}}%
\pgfpathcurveto{\pgfqpoint{5.220300in}{3.351414in}}{\pgfqpoint{5.225886in}{3.353728in}}{\pgfqpoint{5.230004in}{3.357846in}}%
\pgfpathcurveto{\pgfqpoint{5.234122in}{3.361964in}}{\pgfqpoint{5.236436in}{3.367551in}}{\pgfqpoint{5.236436in}{3.373374in}}%
\pgfpathcurveto{\pgfqpoint{5.236436in}{3.379198in}}{\pgfqpoint{5.234122in}{3.384785in}}{\pgfqpoint{5.230004in}{3.388903in}}%
\pgfpathcurveto{\pgfqpoint{5.225886in}{3.393021in}}{\pgfqpoint{5.220300in}{3.395335in}}{\pgfqpoint{5.214476in}{3.395335in}}%
\pgfpathcurveto{\pgfqpoint{5.208652in}{3.395335in}}{\pgfqpoint{5.203066in}{3.393021in}}{\pgfqpoint{5.198948in}{3.388903in}}%
\pgfpathcurveto{\pgfqpoint{5.194830in}{3.384785in}}{\pgfqpoint{5.192516in}{3.379198in}}{\pgfqpoint{5.192516in}{3.373374in}}%
\pgfpathcurveto{\pgfqpoint{5.192516in}{3.367551in}}{\pgfqpoint{5.194830in}{3.361964in}}{\pgfqpoint{5.198948in}{3.357846in}}%
\pgfpathcurveto{\pgfqpoint{5.203066in}{3.353728in}}{\pgfqpoint{5.208652in}{3.351414in}}{\pgfqpoint{5.214476in}{3.351414in}}%
\pgfpathlineto{\pgfqpoint{5.214476in}{3.351414in}}%
\pgfpathclose%
\pgfusepath{stroke,fill}%
\end{pgfscope}%
\begin{pgfscope}%
\pgfpathrectangle{\pgfqpoint{0.640323in}{0.527436in}}{\pgfqpoint{9.687500in}{3.850000in}}%
\pgfusepath{clip}%
\pgfsetbuttcap%
\pgfsetroundjoin%
\definecolor{currentfill}{rgb}{0.000000,0.000000,0.000000}%
\pgfsetfillcolor{currentfill}%
\pgfsetfillopacity{0.500000}%
\pgfsetlinewidth{1.003750pt}%
\definecolor{currentstroke}{rgb}{0.000000,0.000000,0.000000}%
\pgfsetstrokecolor{currentstroke}%
\pgfsetstrokeopacity{0.500000}%
\pgfsetdash{{3.700000pt}{1.600000pt}}{0.000000pt}%
\pgfpathmoveto{\pgfqpoint{5.394207in}{3.399415in}}%
\pgfpathcurveto{\pgfqpoint{5.400031in}{3.399415in}}{\pgfqpoint{5.405617in}{3.401729in}}{\pgfqpoint{5.409735in}{3.405847in}}%
\pgfpathcurveto{\pgfqpoint{5.413853in}{3.409965in}}{\pgfqpoint{5.416167in}{3.415551in}}{\pgfqpoint{5.416167in}{3.421375in}}%
\pgfpathcurveto{\pgfqpoint{5.416167in}{3.427199in}}{\pgfqpoint{5.413853in}{3.432785in}}{\pgfqpoint{5.409735in}{3.436904in}}%
\pgfpathcurveto{\pgfqpoint{5.405617in}{3.441022in}}{\pgfqpoint{5.400031in}{3.443336in}}{\pgfqpoint{5.394207in}{3.443336in}}%
\pgfpathcurveto{\pgfqpoint{5.388383in}{3.443336in}}{\pgfqpoint{5.382797in}{3.441022in}}{\pgfqpoint{5.378679in}{3.436904in}}%
\pgfpathcurveto{\pgfqpoint{5.374561in}{3.432785in}}{\pgfqpoint{5.372247in}{3.427199in}}{\pgfqpoint{5.372247in}{3.421375in}}%
\pgfpathcurveto{\pgfqpoint{5.372247in}{3.415551in}}{\pgfqpoint{5.374561in}{3.409965in}}{\pgfqpoint{5.378679in}{3.405847in}}%
\pgfpathcurveto{\pgfqpoint{5.382797in}{3.401729in}}{\pgfqpoint{5.388383in}{3.399415in}}{\pgfqpoint{5.394207in}{3.399415in}}%
\pgfpathlineto{\pgfqpoint{5.394207in}{3.399415in}}%
\pgfpathclose%
\pgfusepath{stroke,fill}%
\end{pgfscope}%
\begin{pgfscope}%
\pgfpathrectangle{\pgfqpoint{0.640323in}{0.527436in}}{\pgfqpoint{9.687500in}{3.850000in}}%
\pgfusepath{clip}%
\pgfsetbuttcap%
\pgfsetroundjoin%
\definecolor{currentfill}{rgb}{0.000000,0.000000,0.000000}%
\pgfsetfillcolor{currentfill}%
\pgfsetfillopacity{0.500000}%
\pgfsetlinewidth{1.003750pt}%
\definecolor{currentstroke}{rgb}{0.000000,0.000000,0.000000}%
\pgfsetstrokecolor{currentstroke}%
\pgfsetstrokeopacity{0.500000}%
\pgfsetdash{{3.700000pt}{1.600000pt}}{0.000000pt}%
\pgfpathmoveto{\pgfqpoint{5.573938in}{3.443740in}}%
\pgfpathcurveto{\pgfqpoint{5.579762in}{3.443740in}}{\pgfqpoint{5.585348in}{3.446054in}}{\pgfqpoint{5.589466in}{3.450172in}}%
\pgfpathcurveto{\pgfqpoint{5.593584in}{3.454290in}}{\pgfqpoint{5.595898in}{3.459876in}}{\pgfqpoint{5.595898in}{3.465700in}}%
\pgfpathcurveto{\pgfqpoint{5.595898in}{3.471524in}}{\pgfqpoint{5.593584in}{3.477110in}}{\pgfqpoint{5.589466in}{3.481228in}}%
\pgfpathcurveto{\pgfqpoint{5.585348in}{3.485346in}}{\pgfqpoint{5.579762in}{3.487660in}}{\pgfqpoint{5.573938in}{3.487660in}}%
\pgfpathcurveto{\pgfqpoint{5.568114in}{3.487660in}}{\pgfqpoint{5.562528in}{3.485346in}}{\pgfqpoint{5.558410in}{3.481228in}}%
\pgfpathcurveto{\pgfqpoint{5.554292in}{3.477110in}}{\pgfqpoint{5.551978in}{3.471524in}}{\pgfqpoint{5.551978in}{3.465700in}}%
\pgfpathcurveto{\pgfqpoint{5.551978in}{3.459876in}}{\pgfqpoint{5.554292in}{3.454290in}}{\pgfqpoint{5.558410in}{3.450172in}}%
\pgfpathcurveto{\pgfqpoint{5.562528in}{3.446054in}}{\pgfqpoint{5.568114in}{3.443740in}}{\pgfqpoint{5.573938in}{3.443740in}}%
\pgfpathlineto{\pgfqpoint{5.573938in}{3.443740in}}%
\pgfpathclose%
\pgfusepath{stroke,fill}%
\end{pgfscope}%
\begin{pgfscope}%
\pgfpathrectangle{\pgfqpoint{0.640323in}{0.527436in}}{\pgfqpoint{9.687500in}{3.850000in}}%
\pgfusepath{clip}%
\pgfsetbuttcap%
\pgfsetroundjoin%
\definecolor{currentfill}{rgb}{0.000000,0.000000,0.000000}%
\pgfsetfillcolor{currentfill}%
\pgfsetfillopacity{0.500000}%
\pgfsetlinewidth{1.003750pt}%
\definecolor{currentstroke}{rgb}{0.000000,0.000000,0.000000}%
\pgfsetstrokecolor{currentstroke}%
\pgfsetstrokeopacity{0.500000}%
\pgfsetdash{{3.700000pt}{1.600000pt}}{0.000000pt}%
\pgfpathmoveto{\pgfqpoint{5.753669in}{3.487071in}}%
\pgfpathcurveto{\pgfqpoint{5.759493in}{3.487071in}}{\pgfqpoint{5.765079in}{3.489385in}}{\pgfqpoint{5.769197in}{3.493503in}}%
\pgfpathcurveto{\pgfqpoint{5.773315in}{3.497621in}}{\pgfqpoint{5.775629in}{3.503207in}}{\pgfqpoint{5.775629in}{3.509031in}}%
\pgfpathcurveto{\pgfqpoint{5.775629in}{3.514855in}}{\pgfqpoint{5.773315in}{3.520441in}}{\pgfqpoint{5.769197in}{3.524559in}}%
\pgfpathcurveto{\pgfqpoint{5.765079in}{3.528677in}}{\pgfqpoint{5.759493in}{3.530991in}}{\pgfqpoint{5.753669in}{3.530991in}}%
\pgfpathcurveto{\pgfqpoint{5.747845in}{3.530991in}}{\pgfqpoint{5.742259in}{3.528677in}}{\pgfqpoint{5.738141in}{3.524559in}}%
\pgfpathcurveto{\pgfqpoint{5.734023in}{3.520441in}}{\pgfqpoint{5.731709in}{3.514855in}}{\pgfqpoint{5.731709in}{3.509031in}}%
\pgfpathcurveto{\pgfqpoint{5.731709in}{3.503207in}}{\pgfqpoint{5.734023in}{3.497621in}}{\pgfqpoint{5.738141in}{3.493503in}}%
\pgfpathcurveto{\pgfqpoint{5.742259in}{3.489385in}}{\pgfqpoint{5.747845in}{3.487071in}}{\pgfqpoint{5.753669in}{3.487071in}}%
\pgfpathlineto{\pgfqpoint{5.753669in}{3.487071in}}%
\pgfpathclose%
\pgfusepath{stroke,fill}%
\end{pgfscope}%
\begin{pgfscope}%
\pgfpathrectangle{\pgfqpoint{0.640323in}{0.527436in}}{\pgfqpoint{9.687500in}{3.850000in}}%
\pgfusepath{clip}%
\pgfsetbuttcap%
\pgfsetroundjoin%
\definecolor{currentfill}{rgb}{0.000000,0.000000,0.000000}%
\pgfsetfillcolor{currentfill}%
\pgfsetfillopacity{0.500000}%
\pgfsetlinewidth{1.003750pt}%
\definecolor{currentstroke}{rgb}{0.000000,0.000000,0.000000}%
\pgfsetstrokecolor{currentstroke}%
\pgfsetstrokeopacity{0.500000}%
\pgfsetdash{{3.700000pt}{1.600000pt}}{0.000000pt}%
\pgfpathmoveto{\pgfqpoint{5.933400in}{3.527396in}}%
\pgfpathcurveto{\pgfqpoint{5.939224in}{3.527396in}}{\pgfqpoint{5.944810in}{3.529710in}}{\pgfqpoint{5.948928in}{3.533828in}}%
\pgfpathcurveto{\pgfqpoint{5.953046in}{3.537947in}}{\pgfqpoint{5.955360in}{3.543533in}}{\pgfqpoint{5.955360in}{3.549357in}}%
\pgfpathcurveto{\pgfqpoint{5.955360in}{3.555181in}}{\pgfqpoint{5.953046in}{3.560767in}}{\pgfqpoint{5.948928in}{3.564885in}}%
\pgfpathcurveto{\pgfqpoint{5.944810in}{3.569003in}}{\pgfqpoint{5.939224in}{3.571317in}}{\pgfqpoint{5.933400in}{3.571317in}}%
\pgfpathcurveto{\pgfqpoint{5.927576in}{3.571317in}}{\pgfqpoint{5.921990in}{3.569003in}}{\pgfqpoint{5.917872in}{3.564885in}}%
\pgfpathcurveto{\pgfqpoint{5.913754in}{3.560767in}}{\pgfqpoint{5.911440in}{3.555181in}}{\pgfqpoint{5.911440in}{3.549357in}}%
\pgfpathcurveto{\pgfqpoint{5.911440in}{3.543533in}}{\pgfqpoint{5.913754in}{3.537947in}}{\pgfqpoint{5.917872in}{3.533828in}}%
\pgfpathcurveto{\pgfqpoint{5.921990in}{3.529710in}}{\pgfqpoint{5.927576in}{3.527396in}}{\pgfqpoint{5.933400in}{3.527396in}}%
\pgfpathlineto{\pgfqpoint{5.933400in}{3.527396in}}%
\pgfpathclose%
\pgfusepath{stroke,fill}%
\end{pgfscope}%
\begin{pgfscope}%
\pgfpathrectangle{\pgfqpoint{0.640323in}{0.527436in}}{\pgfqpoint{9.687500in}{3.850000in}}%
\pgfusepath{clip}%
\pgfsetbuttcap%
\pgfsetroundjoin%
\definecolor{currentfill}{rgb}{0.000000,0.000000,0.000000}%
\pgfsetfillcolor{currentfill}%
\pgfsetfillopacity{0.500000}%
\pgfsetlinewidth{1.003750pt}%
\definecolor{currentstroke}{rgb}{0.000000,0.000000,0.000000}%
\pgfsetstrokecolor{currentstroke}%
\pgfsetstrokeopacity{0.500000}%
\pgfsetdash{{3.700000pt}{1.600000pt}}{0.000000pt}%
\pgfpathmoveto{\pgfqpoint{6.113131in}{3.564270in}}%
\pgfpathcurveto{\pgfqpoint{6.118955in}{3.564270in}}{\pgfqpoint{6.124541in}{3.566583in}}{\pgfqpoint{6.128659in}{3.570702in}}%
\pgfpathcurveto{\pgfqpoint{6.132777in}{3.574820in}}{\pgfqpoint{6.135091in}{3.580406in}}{\pgfqpoint{6.135091in}{3.586230in}}%
\pgfpathcurveto{\pgfqpoint{6.135091in}{3.592054in}}{\pgfqpoint{6.132777in}{3.597640in}}{\pgfqpoint{6.128659in}{3.601758in}}%
\pgfpathcurveto{\pgfqpoint{6.124541in}{3.605876in}}{\pgfqpoint{6.118955in}{3.608190in}}{\pgfqpoint{6.113131in}{3.608190in}}%
\pgfpathcurveto{\pgfqpoint{6.107307in}{3.608190in}}{\pgfqpoint{6.101721in}{3.605876in}}{\pgfqpoint{6.097603in}{3.601758in}}%
\pgfpathcurveto{\pgfqpoint{6.093485in}{3.597640in}}{\pgfqpoint{6.091171in}{3.592054in}}{\pgfqpoint{6.091171in}{3.586230in}}%
\pgfpathcurveto{\pgfqpoint{6.091171in}{3.580406in}}{\pgfqpoint{6.093485in}{3.574820in}}{\pgfqpoint{6.097603in}{3.570702in}}%
\pgfpathcurveto{\pgfqpoint{6.101721in}{3.566583in}}{\pgfqpoint{6.107307in}{3.564270in}}{\pgfqpoint{6.113131in}{3.564270in}}%
\pgfpathlineto{\pgfqpoint{6.113131in}{3.564270in}}%
\pgfpathclose%
\pgfusepath{stroke,fill}%
\end{pgfscope}%
\begin{pgfscope}%
\pgfpathrectangle{\pgfqpoint{0.640323in}{0.527436in}}{\pgfqpoint{9.687500in}{3.850000in}}%
\pgfusepath{clip}%
\pgfsetbuttcap%
\pgfsetroundjoin%
\definecolor{currentfill}{rgb}{0.000000,0.000000,0.000000}%
\pgfsetfillcolor{currentfill}%
\pgfsetfillopacity{0.500000}%
\pgfsetlinewidth{1.003750pt}%
\definecolor{currentstroke}{rgb}{0.000000,0.000000,0.000000}%
\pgfsetstrokecolor{currentstroke}%
\pgfsetstrokeopacity{0.500000}%
\pgfsetdash{{3.700000pt}{1.600000pt}}{0.000000pt}%
\pgfpathmoveto{\pgfqpoint{6.292862in}{3.599509in}}%
\pgfpathcurveto{\pgfqpoint{6.298686in}{3.599509in}}{\pgfqpoint{6.304272in}{3.601823in}}{\pgfqpoint{6.308390in}{3.605941in}}%
\pgfpathcurveto{\pgfqpoint{6.312508in}{3.610060in}}{\pgfqpoint{6.314822in}{3.615646in}}{\pgfqpoint{6.314822in}{3.621470in}}%
\pgfpathcurveto{\pgfqpoint{6.314822in}{3.627294in}}{\pgfqpoint{6.312508in}{3.632880in}}{\pgfqpoint{6.308390in}{3.636998in}}%
\pgfpathcurveto{\pgfqpoint{6.304272in}{3.641116in}}{\pgfqpoint{6.298686in}{3.643430in}}{\pgfqpoint{6.292862in}{3.643430in}}%
\pgfpathcurveto{\pgfqpoint{6.287038in}{3.643430in}}{\pgfqpoint{6.281452in}{3.641116in}}{\pgfqpoint{6.277334in}{3.636998in}}%
\pgfpathcurveto{\pgfqpoint{6.273216in}{3.632880in}}{\pgfqpoint{6.270902in}{3.627294in}}{\pgfqpoint{6.270902in}{3.621470in}}%
\pgfpathcurveto{\pgfqpoint{6.270902in}{3.615646in}}{\pgfqpoint{6.273216in}{3.610060in}}{\pgfqpoint{6.277334in}{3.605941in}}%
\pgfpathcurveto{\pgfqpoint{6.281452in}{3.601823in}}{\pgfqpoint{6.287038in}{3.599509in}}{\pgfqpoint{6.292862in}{3.599509in}}%
\pgfpathlineto{\pgfqpoint{6.292862in}{3.599509in}}%
\pgfpathclose%
\pgfusepath{stroke,fill}%
\end{pgfscope}%
\begin{pgfscope}%
\pgfpathrectangle{\pgfqpoint{0.640323in}{0.527436in}}{\pgfqpoint{9.687500in}{3.850000in}}%
\pgfusepath{clip}%
\pgfsetbuttcap%
\pgfsetroundjoin%
\definecolor{currentfill}{rgb}{0.000000,0.000000,0.000000}%
\pgfsetfillcolor{currentfill}%
\pgfsetfillopacity{0.500000}%
\pgfsetlinewidth{1.003750pt}%
\definecolor{currentstroke}{rgb}{0.000000,0.000000,0.000000}%
\pgfsetstrokecolor{currentstroke}%
\pgfsetstrokeopacity{0.500000}%
\pgfsetdash{{3.700000pt}{1.600000pt}}{0.000000pt}%
\pgfpathmoveto{\pgfqpoint{6.472593in}{3.632222in}}%
\pgfpathcurveto{\pgfqpoint{6.478417in}{3.632222in}}{\pgfqpoint{6.484003in}{3.634536in}}{\pgfqpoint{6.488121in}{3.638654in}}%
\pgfpathcurveto{\pgfqpoint{6.492239in}{3.642772in}}{\pgfqpoint{6.494553in}{3.648358in}}{\pgfqpoint{6.494553in}{3.654182in}}%
\pgfpathcurveto{\pgfqpoint{6.494553in}{3.660006in}}{\pgfqpoint{6.492239in}{3.665592in}}{\pgfqpoint{6.488121in}{3.669711in}}%
\pgfpathcurveto{\pgfqpoint{6.484003in}{3.673829in}}{\pgfqpoint{6.478417in}{3.676143in}}{\pgfqpoint{6.472593in}{3.676143in}}%
\pgfpathcurveto{\pgfqpoint{6.466769in}{3.676143in}}{\pgfqpoint{6.461183in}{3.673829in}}{\pgfqpoint{6.457065in}{3.669711in}}%
\pgfpathcurveto{\pgfqpoint{6.452947in}{3.665592in}}{\pgfqpoint{6.450633in}{3.660006in}}{\pgfqpoint{6.450633in}{3.654182in}}%
\pgfpathcurveto{\pgfqpoint{6.450633in}{3.648358in}}{\pgfqpoint{6.452947in}{3.642772in}}{\pgfqpoint{6.457065in}{3.638654in}}%
\pgfpathcurveto{\pgfqpoint{6.461183in}{3.634536in}}{\pgfqpoint{6.466769in}{3.632222in}}{\pgfqpoint{6.472593in}{3.632222in}}%
\pgfpathlineto{\pgfqpoint{6.472593in}{3.632222in}}%
\pgfpathclose%
\pgfusepath{stroke,fill}%
\end{pgfscope}%
\begin{pgfscope}%
\pgfpathrectangle{\pgfqpoint{0.640323in}{0.527436in}}{\pgfqpoint{9.687500in}{3.850000in}}%
\pgfusepath{clip}%
\pgfsetbuttcap%
\pgfsetroundjoin%
\definecolor{currentfill}{rgb}{0.000000,0.000000,0.000000}%
\pgfsetfillcolor{currentfill}%
\pgfsetfillopacity{0.500000}%
\pgfsetlinewidth{1.003750pt}%
\definecolor{currentstroke}{rgb}{0.000000,0.000000,0.000000}%
\pgfsetstrokecolor{currentstroke}%
\pgfsetstrokeopacity{0.500000}%
\pgfsetdash{{3.700000pt}{1.600000pt}}{0.000000pt}%
\pgfpathmoveto{\pgfqpoint{6.652324in}{3.659402in}}%
\pgfpathcurveto{\pgfqpoint{6.658148in}{3.659402in}}{\pgfqpoint{6.663734in}{3.661716in}}{\pgfqpoint{6.667852in}{3.665834in}}%
\pgfpathcurveto{\pgfqpoint{6.671970in}{3.669952in}}{\pgfqpoint{6.674284in}{3.675538in}}{\pgfqpoint{6.674284in}{3.681362in}}%
\pgfpathcurveto{\pgfqpoint{6.674284in}{3.687186in}}{\pgfqpoint{6.671970in}{3.692772in}}{\pgfqpoint{6.667852in}{3.696890in}}%
\pgfpathcurveto{\pgfqpoint{6.663734in}{3.701008in}}{\pgfqpoint{6.658148in}{3.703322in}}{\pgfqpoint{6.652324in}{3.703322in}}%
\pgfpathcurveto{\pgfqpoint{6.646500in}{3.703322in}}{\pgfqpoint{6.640914in}{3.701008in}}{\pgfqpoint{6.636796in}{3.696890in}}%
\pgfpathcurveto{\pgfqpoint{6.632678in}{3.692772in}}{\pgfqpoint{6.630364in}{3.687186in}}{\pgfqpoint{6.630364in}{3.681362in}}%
\pgfpathcurveto{\pgfqpoint{6.630364in}{3.675538in}}{\pgfqpoint{6.632678in}{3.669952in}}{\pgfqpoint{6.636796in}{3.665834in}}%
\pgfpathcurveto{\pgfqpoint{6.640914in}{3.661716in}}{\pgfqpoint{6.646500in}{3.659402in}}{\pgfqpoint{6.652324in}{3.659402in}}%
\pgfpathlineto{\pgfqpoint{6.652324in}{3.659402in}}%
\pgfpathclose%
\pgfusepath{stroke,fill}%
\end{pgfscope}%
\begin{pgfscope}%
\pgfpathrectangle{\pgfqpoint{0.640323in}{0.527436in}}{\pgfqpoint{9.687500in}{3.850000in}}%
\pgfusepath{clip}%
\pgfsetbuttcap%
\pgfsetroundjoin%
\definecolor{currentfill}{rgb}{0.000000,0.000000,0.000000}%
\pgfsetfillcolor{currentfill}%
\pgfsetfillopacity{0.500000}%
\pgfsetlinewidth{1.003750pt}%
\definecolor{currentstroke}{rgb}{0.000000,0.000000,0.000000}%
\pgfsetstrokecolor{currentstroke}%
\pgfsetstrokeopacity{0.500000}%
\pgfsetdash{{3.700000pt}{1.600000pt}}{0.000000pt}%
\pgfpathmoveto{\pgfqpoint{6.832055in}{3.690599in}}%
\pgfpathcurveto{\pgfqpoint{6.837879in}{3.690599in}}{\pgfqpoint{6.843465in}{3.692913in}}{\pgfqpoint{6.847583in}{3.697031in}}%
\pgfpathcurveto{\pgfqpoint{6.851701in}{3.701149in}}{\pgfqpoint{6.854015in}{3.706736in}}{\pgfqpoint{6.854015in}{3.712559in}}%
\pgfpathcurveto{\pgfqpoint{6.854015in}{3.718383in}}{\pgfqpoint{6.851701in}{3.723970in}}{\pgfqpoint{6.847583in}{3.728088in}}%
\pgfpathcurveto{\pgfqpoint{6.843465in}{3.732206in}}{\pgfqpoint{6.837879in}{3.734520in}}{\pgfqpoint{6.832055in}{3.734520in}}%
\pgfpathcurveto{\pgfqpoint{6.826231in}{3.734520in}}{\pgfqpoint{6.820645in}{3.732206in}}{\pgfqpoint{6.816527in}{3.728088in}}%
\pgfpathcurveto{\pgfqpoint{6.812408in}{3.723970in}}{\pgfqpoint{6.810095in}{3.718383in}}{\pgfqpoint{6.810095in}{3.712559in}}%
\pgfpathcurveto{\pgfqpoint{6.810095in}{3.706736in}}{\pgfqpoint{6.812408in}{3.701149in}}{\pgfqpoint{6.816527in}{3.697031in}}%
\pgfpathcurveto{\pgfqpoint{6.820645in}{3.692913in}}{\pgfqpoint{6.826231in}{3.690599in}}{\pgfqpoint{6.832055in}{3.690599in}}%
\pgfpathlineto{\pgfqpoint{6.832055in}{3.690599in}}%
\pgfpathclose%
\pgfusepath{stroke,fill}%
\end{pgfscope}%
\begin{pgfscope}%
\pgfpathrectangle{\pgfqpoint{0.640323in}{0.527436in}}{\pgfqpoint{9.687500in}{3.850000in}}%
\pgfusepath{clip}%
\pgfsetbuttcap%
\pgfsetroundjoin%
\definecolor{currentfill}{rgb}{0.000000,0.000000,0.000000}%
\pgfsetfillcolor{currentfill}%
\pgfsetfillopacity{0.500000}%
\pgfsetlinewidth{1.003750pt}%
\definecolor{currentstroke}{rgb}{0.000000,0.000000,0.000000}%
\pgfsetstrokecolor{currentstroke}%
\pgfsetstrokeopacity{0.500000}%
\pgfsetdash{{3.700000pt}{1.600000pt}}{0.000000pt}%
\pgfpathmoveto{\pgfqpoint{7.011786in}{3.717108in}}%
\pgfpathcurveto{\pgfqpoint{7.017610in}{3.717108in}}{\pgfqpoint{7.023196in}{3.719422in}}{\pgfqpoint{7.027314in}{3.723540in}}%
\pgfpathcurveto{\pgfqpoint{7.031432in}{3.727658in}}{\pgfqpoint{7.033746in}{3.733245in}}{\pgfqpoint{7.033746in}{3.739069in}}%
\pgfpathcurveto{\pgfqpoint{7.033746in}{3.744893in}}{\pgfqpoint{7.031432in}{3.750479in}}{\pgfqpoint{7.027314in}{3.754597in}}%
\pgfpathcurveto{\pgfqpoint{7.023196in}{3.758715in}}{\pgfqpoint{7.017610in}{3.761029in}}{\pgfqpoint{7.011786in}{3.761029in}}%
\pgfpathcurveto{\pgfqpoint{7.005962in}{3.761029in}}{\pgfqpoint{7.000376in}{3.758715in}}{\pgfqpoint{6.996258in}{3.754597in}}%
\pgfpathcurveto{\pgfqpoint{6.992139in}{3.750479in}}{\pgfqpoint{6.989826in}{3.744893in}}{\pgfqpoint{6.989826in}{3.739069in}}%
\pgfpathcurveto{\pgfqpoint{6.989826in}{3.733245in}}{\pgfqpoint{6.992139in}{3.727658in}}{\pgfqpoint{6.996258in}{3.723540in}}%
\pgfpathcurveto{\pgfqpoint{7.000376in}{3.719422in}}{\pgfqpoint{7.005962in}{3.717108in}}{\pgfqpoint{7.011786in}{3.717108in}}%
\pgfpathlineto{\pgfqpoint{7.011786in}{3.717108in}}%
\pgfpathclose%
\pgfusepath{stroke,fill}%
\end{pgfscope}%
\begin{pgfscope}%
\pgfpathrectangle{\pgfqpoint{0.640323in}{0.527436in}}{\pgfqpoint{9.687500in}{3.850000in}}%
\pgfusepath{clip}%
\pgfsetbuttcap%
\pgfsetroundjoin%
\definecolor{currentfill}{rgb}{0.000000,0.000000,0.000000}%
\pgfsetfillcolor{currentfill}%
\pgfsetfillopacity{0.500000}%
\pgfsetlinewidth{1.003750pt}%
\definecolor{currentstroke}{rgb}{0.000000,0.000000,0.000000}%
\pgfsetstrokecolor{currentstroke}%
\pgfsetstrokeopacity{0.500000}%
\pgfsetdash{{3.700000pt}{1.600000pt}}{0.000000pt}%
\pgfpathmoveto{\pgfqpoint{7.191517in}{3.744878in}}%
\pgfpathcurveto{\pgfqpoint{7.197341in}{3.744878in}}{\pgfqpoint{7.202927in}{3.747192in}}{\pgfqpoint{7.207045in}{3.751310in}}%
\pgfpathcurveto{\pgfqpoint{7.211163in}{3.755428in}}{\pgfqpoint{7.213477in}{3.761014in}}{\pgfqpoint{7.213477in}{3.766838in}}%
\pgfpathcurveto{\pgfqpoint{7.213477in}{3.772662in}}{\pgfqpoint{7.211163in}{3.778248in}}{\pgfqpoint{7.207045in}{3.782367in}}%
\pgfpathcurveto{\pgfqpoint{7.202927in}{3.786485in}}{\pgfqpoint{7.197341in}{3.788799in}}{\pgfqpoint{7.191517in}{3.788799in}}%
\pgfpathcurveto{\pgfqpoint{7.185693in}{3.788799in}}{\pgfqpoint{7.180107in}{3.786485in}}{\pgfqpoint{7.175989in}{3.782367in}}%
\pgfpathcurveto{\pgfqpoint{7.171870in}{3.778248in}}{\pgfqpoint{7.169557in}{3.772662in}}{\pgfqpoint{7.169557in}{3.766838in}}%
\pgfpathcurveto{\pgfqpoint{7.169557in}{3.761014in}}{\pgfqpoint{7.171870in}{3.755428in}}{\pgfqpoint{7.175989in}{3.751310in}}%
\pgfpathcurveto{\pgfqpoint{7.180107in}{3.747192in}}{\pgfqpoint{7.185693in}{3.744878in}}{\pgfqpoint{7.191517in}{3.744878in}}%
\pgfpathlineto{\pgfqpoint{7.191517in}{3.744878in}}%
\pgfpathclose%
\pgfusepath{stroke,fill}%
\end{pgfscope}%
\begin{pgfscope}%
\pgfpathrectangle{\pgfqpoint{0.640323in}{0.527436in}}{\pgfqpoint{9.687500in}{3.850000in}}%
\pgfusepath{clip}%
\pgfsetbuttcap%
\pgfsetroundjoin%
\definecolor{currentfill}{rgb}{0.000000,0.000000,0.000000}%
\pgfsetfillcolor{currentfill}%
\pgfsetfillopacity{0.500000}%
\pgfsetlinewidth{1.003750pt}%
\definecolor{currentstroke}{rgb}{0.000000,0.000000,0.000000}%
\pgfsetstrokecolor{currentstroke}%
\pgfsetstrokeopacity{0.500000}%
\pgfsetdash{{3.700000pt}{1.600000pt}}{0.000000pt}%
\pgfpathmoveto{\pgfqpoint{7.371248in}{3.771859in}}%
\pgfpathcurveto{\pgfqpoint{7.377072in}{3.771859in}}{\pgfqpoint{7.382658in}{3.774173in}}{\pgfqpoint{7.386776in}{3.778291in}}%
\pgfpathcurveto{\pgfqpoint{7.390894in}{3.782409in}}{\pgfqpoint{7.393208in}{3.787995in}}{\pgfqpoint{7.393208in}{3.793819in}}%
\pgfpathcurveto{\pgfqpoint{7.393208in}{3.799643in}}{\pgfqpoint{7.390894in}{3.805229in}}{\pgfqpoint{7.386776in}{3.809348in}}%
\pgfpathcurveto{\pgfqpoint{7.382658in}{3.813466in}}{\pgfqpoint{7.377072in}{3.815780in}}{\pgfqpoint{7.371248in}{3.815780in}}%
\pgfpathcurveto{\pgfqpoint{7.365424in}{3.815780in}}{\pgfqpoint{7.359838in}{3.813466in}}{\pgfqpoint{7.355720in}{3.809348in}}%
\pgfpathcurveto{\pgfqpoint{7.351601in}{3.805229in}}{\pgfqpoint{7.349288in}{3.799643in}}{\pgfqpoint{7.349288in}{3.793819in}}%
\pgfpathcurveto{\pgfqpoint{7.349288in}{3.787995in}}{\pgfqpoint{7.351601in}{3.782409in}}{\pgfqpoint{7.355720in}{3.778291in}}%
\pgfpathcurveto{\pgfqpoint{7.359838in}{3.774173in}}{\pgfqpoint{7.365424in}{3.771859in}}{\pgfqpoint{7.371248in}{3.771859in}}%
\pgfpathlineto{\pgfqpoint{7.371248in}{3.771859in}}%
\pgfpathclose%
\pgfusepath{stroke,fill}%
\end{pgfscope}%
\begin{pgfscope}%
\pgfpathrectangle{\pgfqpoint{0.640323in}{0.527436in}}{\pgfqpoint{9.687500in}{3.850000in}}%
\pgfusepath{clip}%
\pgfsetbuttcap%
\pgfsetroundjoin%
\definecolor{currentfill}{rgb}{0.000000,0.000000,0.000000}%
\pgfsetfillcolor{currentfill}%
\pgfsetfillopacity{0.500000}%
\pgfsetlinewidth{1.003750pt}%
\definecolor{currentstroke}{rgb}{0.000000,0.000000,0.000000}%
\pgfsetstrokecolor{currentstroke}%
\pgfsetstrokeopacity{0.500000}%
\pgfsetdash{{3.700000pt}{1.600000pt}}{0.000000pt}%
\pgfpathmoveto{\pgfqpoint{7.550979in}{3.794021in}}%
\pgfpathcurveto{\pgfqpoint{7.556803in}{3.794021in}}{\pgfqpoint{7.562389in}{3.796335in}}{\pgfqpoint{7.566507in}{3.800453in}}%
\pgfpathcurveto{\pgfqpoint{7.570625in}{3.804572in}}{\pgfqpoint{7.572939in}{3.810158in}}{\pgfqpoint{7.572939in}{3.815982in}}%
\pgfpathcurveto{\pgfqpoint{7.572939in}{3.821806in}}{\pgfqpoint{7.570625in}{3.827392in}}{\pgfqpoint{7.566507in}{3.831510in}}%
\pgfpathcurveto{\pgfqpoint{7.562389in}{3.835628in}}{\pgfqpoint{7.556803in}{3.837942in}}{\pgfqpoint{7.550979in}{3.837942in}}%
\pgfpathcurveto{\pgfqpoint{7.545155in}{3.837942in}}{\pgfqpoint{7.539569in}{3.835628in}}{\pgfqpoint{7.535451in}{3.831510in}}%
\pgfpathcurveto{\pgfqpoint{7.531332in}{3.827392in}}{\pgfqpoint{7.529019in}{3.821806in}}{\pgfqpoint{7.529019in}{3.815982in}}%
\pgfpathcurveto{\pgfqpoint{7.529019in}{3.810158in}}{\pgfqpoint{7.531332in}{3.804572in}}{\pgfqpoint{7.535451in}{3.800453in}}%
\pgfpathcurveto{\pgfqpoint{7.539569in}{3.796335in}}{\pgfqpoint{7.545155in}{3.794021in}}{\pgfqpoint{7.550979in}{3.794021in}}%
\pgfpathlineto{\pgfqpoint{7.550979in}{3.794021in}}%
\pgfpathclose%
\pgfusepath{stroke,fill}%
\end{pgfscope}%
\begin{pgfscope}%
\pgfpathrectangle{\pgfqpoint{0.640323in}{0.527436in}}{\pgfqpoint{9.687500in}{3.850000in}}%
\pgfusepath{clip}%
\pgfsetbuttcap%
\pgfsetroundjoin%
\definecolor{currentfill}{rgb}{0.000000,0.000000,0.000000}%
\pgfsetfillcolor{currentfill}%
\pgfsetfillopacity{0.500000}%
\pgfsetlinewidth{1.003750pt}%
\definecolor{currentstroke}{rgb}{0.000000,0.000000,0.000000}%
\pgfsetstrokecolor{currentstroke}%
\pgfsetstrokeopacity{0.500000}%
\pgfsetdash{{3.700000pt}{1.600000pt}}{0.000000pt}%
\pgfpathmoveto{\pgfqpoint{7.730710in}{3.814700in}}%
\pgfpathcurveto{\pgfqpoint{7.736534in}{3.814700in}}{\pgfqpoint{7.742120in}{3.817013in}}{\pgfqpoint{7.746238in}{3.821132in}}%
\pgfpathcurveto{\pgfqpoint{7.750356in}{3.825250in}}{\pgfqpoint{7.752670in}{3.830836in}}{\pgfqpoint{7.752670in}{3.836660in}}%
\pgfpathcurveto{\pgfqpoint{7.752670in}{3.842484in}}{\pgfqpoint{7.750356in}{3.848070in}}{\pgfqpoint{7.746238in}{3.852188in}}%
\pgfpathcurveto{\pgfqpoint{7.742120in}{3.856306in}}{\pgfqpoint{7.736534in}{3.858620in}}{\pgfqpoint{7.730710in}{3.858620in}}%
\pgfpathcurveto{\pgfqpoint{7.724886in}{3.858620in}}{\pgfqpoint{7.719300in}{3.856306in}}{\pgfqpoint{7.715182in}{3.852188in}}%
\pgfpathcurveto{\pgfqpoint{7.711063in}{3.848070in}}{\pgfqpoint{7.708750in}{3.842484in}}{\pgfqpoint{7.708750in}{3.836660in}}%
\pgfpathcurveto{\pgfqpoint{7.708750in}{3.830836in}}{\pgfqpoint{7.711063in}{3.825250in}}{\pgfqpoint{7.715182in}{3.821132in}}%
\pgfpathcurveto{\pgfqpoint{7.719300in}{3.817013in}}{\pgfqpoint{7.724886in}{3.814700in}}{\pgfqpoint{7.730710in}{3.814700in}}%
\pgfpathlineto{\pgfqpoint{7.730710in}{3.814700in}}%
\pgfpathclose%
\pgfusepath{stroke,fill}%
\end{pgfscope}%
\begin{pgfscope}%
\pgfpathrectangle{\pgfqpoint{0.640323in}{0.527436in}}{\pgfqpoint{9.687500in}{3.850000in}}%
\pgfusepath{clip}%
\pgfsetbuttcap%
\pgfsetroundjoin%
\definecolor{currentfill}{rgb}{0.000000,0.000000,0.000000}%
\pgfsetfillcolor{currentfill}%
\pgfsetfillopacity{0.500000}%
\pgfsetlinewidth{1.003750pt}%
\definecolor{currentstroke}{rgb}{0.000000,0.000000,0.000000}%
\pgfsetstrokecolor{currentstroke}%
\pgfsetstrokeopacity{0.500000}%
\pgfsetdash{{3.700000pt}{1.600000pt}}{0.000000pt}%
\pgfpathmoveto{\pgfqpoint{7.910441in}{3.838452in}}%
\pgfpathcurveto{\pgfqpoint{7.916265in}{3.838452in}}{\pgfqpoint{7.921851in}{3.840765in}}{\pgfqpoint{7.925969in}{3.844884in}}%
\pgfpathcurveto{\pgfqpoint{7.930087in}{3.849002in}}{\pgfqpoint{7.932401in}{3.854588in}}{\pgfqpoint{7.932401in}{3.860412in}}%
\pgfpathcurveto{\pgfqpoint{7.932401in}{3.866236in}}{\pgfqpoint{7.930087in}{3.871822in}}{\pgfqpoint{7.925969in}{3.875940in}}%
\pgfpathcurveto{\pgfqpoint{7.921851in}{3.880058in}}{\pgfqpoint{7.916265in}{3.882372in}}{\pgfqpoint{7.910441in}{3.882372in}}%
\pgfpathcurveto{\pgfqpoint{7.904617in}{3.882372in}}{\pgfqpoint{7.899031in}{3.880058in}}{\pgfqpoint{7.894913in}{3.875940in}}%
\pgfpathcurveto{\pgfqpoint{7.890794in}{3.871822in}}{\pgfqpoint{7.888481in}{3.866236in}}{\pgfqpoint{7.888481in}{3.860412in}}%
\pgfpathcurveto{\pgfqpoint{7.888481in}{3.854588in}}{\pgfqpoint{7.890794in}{3.849002in}}{\pgfqpoint{7.894913in}{3.844884in}}%
\pgfpathcurveto{\pgfqpoint{7.899031in}{3.840765in}}{\pgfqpoint{7.904617in}{3.838452in}}{\pgfqpoint{7.910441in}{3.838452in}}%
\pgfpathlineto{\pgfqpoint{7.910441in}{3.838452in}}%
\pgfpathclose%
\pgfusepath{stroke,fill}%
\end{pgfscope}%
\begin{pgfscope}%
\pgfpathrectangle{\pgfqpoint{0.640323in}{0.527436in}}{\pgfqpoint{9.687500in}{3.850000in}}%
\pgfusepath{clip}%
\pgfsetbuttcap%
\pgfsetroundjoin%
\definecolor{currentfill}{rgb}{0.000000,0.000000,0.000000}%
\pgfsetfillcolor{currentfill}%
\pgfsetfillopacity{0.500000}%
\pgfsetlinewidth{1.003750pt}%
\definecolor{currentstroke}{rgb}{0.000000,0.000000,0.000000}%
\pgfsetstrokecolor{currentstroke}%
\pgfsetstrokeopacity{0.500000}%
\pgfsetdash{{3.700000pt}{1.600000pt}}{0.000000pt}%
\pgfpathmoveto{\pgfqpoint{8.090172in}{3.856385in}}%
\pgfpathcurveto{\pgfqpoint{8.095996in}{3.856385in}}{\pgfqpoint{8.101582in}{3.858699in}}{\pgfqpoint{8.105700in}{3.862817in}}%
\pgfpathcurveto{\pgfqpoint{8.109818in}{3.866935in}}{\pgfqpoint{8.112132in}{3.872522in}}{\pgfqpoint{8.112132in}{3.878345in}}%
\pgfpathcurveto{\pgfqpoint{8.112132in}{3.884169in}}{\pgfqpoint{8.109818in}{3.889756in}}{\pgfqpoint{8.105700in}{3.893874in}}%
\pgfpathcurveto{\pgfqpoint{8.101582in}{3.897992in}}{\pgfqpoint{8.095996in}{3.900306in}}{\pgfqpoint{8.090172in}{3.900306in}}%
\pgfpathcurveto{\pgfqpoint{8.084348in}{3.900306in}}{\pgfqpoint{8.078762in}{3.897992in}}{\pgfqpoint{8.074644in}{3.893874in}}%
\pgfpathcurveto{\pgfqpoint{8.070525in}{3.889756in}}{\pgfqpoint{8.068211in}{3.884169in}}{\pgfqpoint{8.068211in}{3.878345in}}%
\pgfpathcurveto{\pgfqpoint{8.068211in}{3.872522in}}{\pgfqpoint{8.070525in}{3.866935in}}{\pgfqpoint{8.074644in}{3.862817in}}%
\pgfpathcurveto{\pgfqpoint{8.078762in}{3.858699in}}{\pgfqpoint{8.084348in}{3.856385in}}{\pgfqpoint{8.090172in}{3.856385in}}%
\pgfpathlineto{\pgfqpoint{8.090172in}{3.856385in}}%
\pgfpathclose%
\pgfusepath{stroke,fill}%
\end{pgfscope}%
\begin{pgfscope}%
\pgfpathrectangle{\pgfqpoint{0.640323in}{0.527436in}}{\pgfqpoint{9.687500in}{3.850000in}}%
\pgfusepath{clip}%
\pgfsetbuttcap%
\pgfsetroundjoin%
\definecolor{currentfill}{rgb}{0.000000,0.000000,0.000000}%
\pgfsetfillcolor{currentfill}%
\pgfsetfillopacity{0.500000}%
\pgfsetlinewidth{1.003750pt}%
\definecolor{currentstroke}{rgb}{0.000000,0.000000,0.000000}%
\pgfsetstrokecolor{currentstroke}%
\pgfsetstrokeopacity{0.500000}%
\pgfsetdash{{3.700000pt}{1.600000pt}}{0.000000pt}%
\pgfpathmoveto{\pgfqpoint{8.269903in}{3.876467in}}%
\pgfpathcurveto{\pgfqpoint{8.275727in}{3.876467in}}{\pgfqpoint{8.281313in}{3.878781in}}{\pgfqpoint{8.285431in}{3.882899in}}%
\pgfpathcurveto{\pgfqpoint{8.289549in}{3.887017in}}{\pgfqpoint{8.291863in}{3.892604in}}{\pgfqpoint{8.291863in}{3.898428in}}%
\pgfpathcurveto{\pgfqpoint{8.291863in}{3.904251in}}{\pgfqpoint{8.289549in}{3.909838in}}{\pgfqpoint{8.285431in}{3.913956in}}%
\pgfpathcurveto{\pgfqpoint{8.281313in}{3.918074in}}{\pgfqpoint{8.275727in}{3.920388in}}{\pgfqpoint{8.269903in}{3.920388in}}%
\pgfpathcurveto{\pgfqpoint{8.264079in}{3.920388in}}{\pgfqpoint{8.258493in}{3.918074in}}{\pgfqpoint{8.254374in}{3.913956in}}%
\pgfpathcurveto{\pgfqpoint{8.250256in}{3.909838in}}{\pgfqpoint{8.247942in}{3.904251in}}{\pgfqpoint{8.247942in}{3.898428in}}%
\pgfpathcurveto{\pgfqpoint{8.247942in}{3.892604in}}{\pgfqpoint{8.250256in}{3.887017in}}{\pgfqpoint{8.254374in}{3.882899in}}%
\pgfpathcurveto{\pgfqpoint{8.258493in}{3.878781in}}{\pgfqpoint{8.264079in}{3.876467in}}{\pgfqpoint{8.269903in}{3.876467in}}%
\pgfpathlineto{\pgfqpoint{8.269903in}{3.876467in}}%
\pgfpathclose%
\pgfusepath{stroke,fill}%
\end{pgfscope}%
\begin{pgfscope}%
\pgfpathrectangle{\pgfqpoint{0.640323in}{0.527436in}}{\pgfqpoint{9.687500in}{3.850000in}}%
\pgfusepath{clip}%
\pgfsetbuttcap%
\pgfsetroundjoin%
\definecolor{currentfill}{rgb}{0.000000,0.000000,0.000000}%
\pgfsetfillcolor{currentfill}%
\pgfsetfillopacity{0.500000}%
\pgfsetlinewidth{1.003750pt}%
\definecolor{currentstroke}{rgb}{0.000000,0.000000,0.000000}%
\pgfsetstrokecolor{currentstroke}%
\pgfsetstrokeopacity{0.500000}%
\pgfsetdash{{3.700000pt}{1.600000pt}}{0.000000pt}%
\pgfpathmoveto{\pgfqpoint{8.449634in}{3.895624in}}%
\pgfpathcurveto{\pgfqpoint{8.455458in}{3.895624in}}{\pgfqpoint{8.461044in}{3.897938in}}{\pgfqpoint{8.465162in}{3.902056in}}%
\pgfpathcurveto{\pgfqpoint{8.469280in}{3.906174in}}{\pgfqpoint{8.471594in}{3.911760in}}{\pgfqpoint{8.471594in}{3.917584in}}%
\pgfpathcurveto{\pgfqpoint{8.471594in}{3.923408in}}{\pgfqpoint{8.469280in}{3.928994in}}{\pgfqpoint{8.465162in}{3.933113in}}%
\pgfpathcurveto{\pgfqpoint{8.461044in}{3.937231in}}{\pgfqpoint{8.455458in}{3.939545in}}{\pgfqpoint{8.449634in}{3.939545in}}%
\pgfpathcurveto{\pgfqpoint{8.443810in}{3.939545in}}{\pgfqpoint{8.438224in}{3.937231in}}{\pgfqpoint{8.434105in}{3.933113in}}%
\pgfpathcurveto{\pgfqpoint{8.429987in}{3.928994in}}{\pgfqpoint{8.427673in}{3.923408in}}{\pgfqpoint{8.427673in}{3.917584in}}%
\pgfpathcurveto{\pgfqpoint{8.427673in}{3.911760in}}{\pgfqpoint{8.429987in}{3.906174in}}{\pgfqpoint{8.434105in}{3.902056in}}%
\pgfpathcurveto{\pgfqpoint{8.438224in}{3.897938in}}{\pgfqpoint{8.443810in}{3.895624in}}{\pgfqpoint{8.449634in}{3.895624in}}%
\pgfpathlineto{\pgfqpoint{8.449634in}{3.895624in}}%
\pgfpathclose%
\pgfusepath{stroke,fill}%
\end{pgfscope}%
\begin{pgfscope}%
\pgfpathrectangle{\pgfqpoint{0.640323in}{0.527436in}}{\pgfqpoint{9.687500in}{3.850000in}}%
\pgfusepath{clip}%
\pgfsetbuttcap%
\pgfsetroundjoin%
\definecolor{currentfill}{rgb}{0.000000,0.000000,0.000000}%
\pgfsetfillcolor{currentfill}%
\pgfsetfillopacity{0.500000}%
\pgfsetlinewidth{1.003750pt}%
\definecolor{currentstroke}{rgb}{0.000000,0.000000,0.000000}%
\pgfsetstrokecolor{currentstroke}%
\pgfsetstrokeopacity{0.500000}%
\pgfsetdash{{3.700000pt}{1.600000pt}}{0.000000pt}%
\pgfpathmoveto{\pgfqpoint{8.629365in}{3.912875in}}%
\pgfpathcurveto{\pgfqpoint{8.635189in}{3.912875in}}{\pgfqpoint{8.640775in}{3.915188in}}{\pgfqpoint{8.644893in}{3.919307in}}%
\pgfpathcurveto{\pgfqpoint{8.649011in}{3.923425in}}{\pgfqpoint{8.651325in}{3.929011in}}{\pgfqpoint{8.651325in}{3.934835in}}%
\pgfpathcurveto{\pgfqpoint{8.651325in}{3.940659in}}{\pgfqpoint{8.649011in}{3.946245in}}{\pgfqpoint{8.644893in}{3.950363in}}%
\pgfpathcurveto{\pgfqpoint{8.640775in}{3.954481in}}{\pgfqpoint{8.635189in}{3.956795in}}{\pgfqpoint{8.629365in}{3.956795in}}%
\pgfpathcurveto{\pgfqpoint{8.623541in}{3.956795in}}{\pgfqpoint{8.617955in}{3.954481in}}{\pgfqpoint{8.613836in}{3.950363in}}%
\pgfpathcurveto{\pgfqpoint{8.609718in}{3.946245in}}{\pgfqpoint{8.607404in}{3.940659in}}{\pgfqpoint{8.607404in}{3.934835in}}%
\pgfpathcurveto{\pgfqpoint{8.607404in}{3.929011in}}{\pgfqpoint{8.609718in}{3.923425in}}{\pgfqpoint{8.613836in}{3.919307in}}%
\pgfpathcurveto{\pgfqpoint{8.617955in}{3.915188in}}{\pgfqpoint{8.623541in}{3.912875in}}{\pgfqpoint{8.629365in}{3.912875in}}%
\pgfpathlineto{\pgfqpoint{8.629365in}{3.912875in}}%
\pgfpathclose%
\pgfusepath{stroke,fill}%
\end{pgfscope}%
\begin{pgfscope}%
\pgfpathrectangle{\pgfqpoint{0.640323in}{0.527436in}}{\pgfqpoint{9.687500in}{3.850000in}}%
\pgfusepath{clip}%
\pgfsetbuttcap%
\pgfsetroundjoin%
\definecolor{currentfill}{rgb}{0.000000,0.000000,0.000000}%
\pgfsetfillcolor{currentfill}%
\pgfsetfillopacity{0.500000}%
\pgfsetlinewidth{1.003750pt}%
\definecolor{currentstroke}{rgb}{0.000000,0.000000,0.000000}%
\pgfsetstrokecolor{currentstroke}%
\pgfsetstrokeopacity{0.500000}%
\pgfsetdash{{3.700000pt}{1.600000pt}}{0.000000pt}%
\pgfpathmoveto{\pgfqpoint{8.809096in}{3.930566in}}%
\pgfpathcurveto{\pgfqpoint{8.814920in}{3.930566in}}{\pgfqpoint{8.820506in}{3.932880in}}{\pgfqpoint{8.824624in}{3.936998in}}%
\pgfpathcurveto{\pgfqpoint{8.828742in}{3.941116in}}{\pgfqpoint{8.831056in}{3.946702in}}{\pgfqpoint{8.831056in}{3.952526in}}%
\pgfpathcurveto{\pgfqpoint{8.831056in}{3.958350in}}{\pgfqpoint{8.828742in}{3.963936in}}{\pgfqpoint{8.824624in}{3.968054in}}%
\pgfpathcurveto{\pgfqpoint{8.820506in}{3.972173in}}{\pgfqpoint{8.814920in}{3.974486in}}{\pgfqpoint{8.809096in}{3.974486in}}%
\pgfpathcurveto{\pgfqpoint{8.803272in}{3.974486in}}{\pgfqpoint{8.797686in}{3.972173in}}{\pgfqpoint{8.793567in}{3.968054in}}%
\pgfpathcurveto{\pgfqpoint{8.789449in}{3.963936in}}{\pgfqpoint{8.787135in}{3.958350in}}{\pgfqpoint{8.787135in}{3.952526in}}%
\pgfpathcurveto{\pgfqpoint{8.787135in}{3.946702in}}{\pgfqpoint{8.789449in}{3.941116in}}{\pgfqpoint{8.793567in}{3.936998in}}%
\pgfpathcurveto{\pgfqpoint{8.797686in}{3.932880in}}{\pgfqpoint{8.803272in}{3.930566in}}{\pgfqpoint{8.809096in}{3.930566in}}%
\pgfpathlineto{\pgfqpoint{8.809096in}{3.930566in}}%
\pgfpathclose%
\pgfusepath{stroke,fill}%
\end{pgfscope}%
\begin{pgfscope}%
\pgfpathrectangle{\pgfqpoint{0.640323in}{0.527436in}}{\pgfqpoint{9.687500in}{3.850000in}}%
\pgfusepath{clip}%
\pgfsetbuttcap%
\pgfsetroundjoin%
\definecolor{currentfill}{rgb}{0.000000,0.000000,0.000000}%
\pgfsetfillcolor{currentfill}%
\pgfsetfillopacity{0.500000}%
\pgfsetlinewidth{1.003750pt}%
\definecolor{currentstroke}{rgb}{0.000000,0.000000,0.000000}%
\pgfsetstrokecolor{currentstroke}%
\pgfsetstrokeopacity{0.500000}%
\pgfsetdash{{3.700000pt}{1.600000pt}}{0.000000pt}%
\pgfpathmoveto{\pgfqpoint{8.988827in}{3.945469in}}%
\pgfpathcurveto{\pgfqpoint{8.994651in}{3.945469in}}{\pgfqpoint{9.000237in}{3.947783in}}{\pgfqpoint{9.004355in}{3.951901in}}%
\pgfpathcurveto{\pgfqpoint{9.008473in}{3.956019in}}{\pgfqpoint{9.010787in}{3.961606in}}{\pgfqpoint{9.010787in}{3.967429in}}%
\pgfpathcurveto{\pgfqpoint{9.010787in}{3.973253in}}{\pgfqpoint{9.008473in}{3.978840in}}{\pgfqpoint{9.004355in}{3.982958in}}%
\pgfpathcurveto{\pgfqpoint{9.000237in}{3.987076in}}{\pgfqpoint{8.994651in}{3.989390in}}{\pgfqpoint{8.988827in}{3.989390in}}%
\pgfpathcurveto{\pgfqpoint{8.983003in}{3.989390in}}{\pgfqpoint{8.977417in}{3.987076in}}{\pgfqpoint{8.973298in}{3.982958in}}%
\pgfpathcurveto{\pgfqpoint{8.969180in}{3.978840in}}{\pgfqpoint{8.966866in}{3.973253in}}{\pgfqpoint{8.966866in}{3.967429in}}%
\pgfpathcurveto{\pgfqpoint{8.966866in}{3.961606in}}{\pgfqpoint{8.969180in}{3.956019in}}{\pgfqpoint{8.973298in}{3.951901in}}%
\pgfpathcurveto{\pgfqpoint{8.977417in}{3.947783in}}{\pgfqpoint{8.983003in}{3.945469in}}{\pgfqpoint{8.988827in}{3.945469in}}%
\pgfpathlineto{\pgfqpoint{8.988827in}{3.945469in}}%
\pgfpathclose%
\pgfusepath{stroke,fill}%
\end{pgfscope}%
\begin{pgfscope}%
\pgfpathrectangle{\pgfqpoint{0.640323in}{0.527436in}}{\pgfqpoint{9.687500in}{3.850000in}}%
\pgfusepath{clip}%
\pgfsetbuttcap%
\pgfsetroundjoin%
\definecolor{currentfill}{rgb}{0.000000,0.000000,0.000000}%
\pgfsetfillcolor{currentfill}%
\pgfsetfillopacity{0.500000}%
\pgfsetlinewidth{1.003750pt}%
\definecolor{currentstroke}{rgb}{0.000000,0.000000,0.000000}%
\pgfsetstrokecolor{currentstroke}%
\pgfsetstrokeopacity{0.500000}%
\pgfsetdash{{3.700000pt}{1.600000pt}}{0.000000pt}%
\pgfpathmoveto{\pgfqpoint{9.168558in}{3.960950in}}%
\pgfpathcurveto{\pgfqpoint{9.174382in}{3.960950in}}{\pgfqpoint{9.179968in}{3.963264in}}{\pgfqpoint{9.184086in}{3.967382in}}%
\pgfpathcurveto{\pgfqpoint{9.188204in}{3.971500in}}{\pgfqpoint{9.190518in}{3.977086in}}{\pgfqpoint{9.190518in}{3.982910in}}%
\pgfpathcurveto{\pgfqpoint{9.190518in}{3.988734in}}{\pgfqpoint{9.188204in}{3.994320in}}{\pgfqpoint{9.184086in}{3.998438in}}%
\pgfpathcurveto{\pgfqpoint{9.179968in}{4.002557in}}{\pgfqpoint{9.174382in}{4.004870in}}{\pgfqpoint{9.168558in}{4.004870in}}%
\pgfpathcurveto{\pgfqpoint{9.162734in}{4.004870in}}{\pgfqpoint{9.157148in}{4.002557in}}{\pgfqpoint{9.153029in}{3.998438in}}%
\pgfpathcurveto{\pgfqpoint{9.148911in}{3.994320in}}{\pgfqpoint{9.146597in}{3.988734in}}{\pgfqpoint{9.146597in}{3.982910in}}%
\pgfpathcurveto{\pgfqpoint{9.146597in}{3.977086in}}{\pgfqpoint{9.148911in}{3.971500in}}{\pgfqpoint{9.153029in}{3.967382in}}%
\pgfpathcurveto{\pgfqpoint{9.157148in}{3.963264in}}{\pgfqpoint{9.162734in}{3.960950in}}{\pgfqpoint{9.168558in}{3.960950in}}%
\pgfpathlineto{\pgfqpoint{9.168558in}{3.960950in}}%
\pgfpathclose%
\pgfusepath{stroke,fill}%
\end{pgfscope}%
\begin{pgfscope}%
\pgfpathrectangle{\pgfqpoint{0.640323in}{0.527436in}}{\pgfqpoint{9.687500in}{3.850000in}}%
\pgfusepath{clip}%
\pgfsetbuttcap%
\pgfsetroundjoin%
\definecolor{currentfill}{rgb}{0.000000,0.000000,0.000000}%
\pgfsetfillcolor{currentfill}%
\pgfsetfillopacity{0.500000}%
\pgfsetlinewidth{1.003750pt}%
\definecolor{currentstroke}{rgb}{0.000000,0.000000,0.000000}%
\pgfsetstrokecolor{currentstroke}%
\pgfsetstrokeopacity{0.500000}%
\pgfsetdash{{3.700000pt}{1.600000pt}}{0.000000pt}%
\pgfpathmoveto{\pgfqpoint{9.348289in}{3.979418in}}%
\pgfpathcurveto{\pgfqpoint{9.354113in}{3.979418in}}{\pgfqpoint{9.359699in}{3.981731in}}{\pgfqpoint{9.363817in}{3.985850in}}%
\pgfpathcurveto{\pgfqpoint{9.367935in}{3.989968in}}{\pgfqpoint{9.370249in}{3.995554in}}{\pgfqpoint{9.370249in}{4.001378in}}%
\pgfpathcurveto{\pgfqpoint{9.370249in}{4.007202in}}{\pgfqpoint{9.367935in}{4.012788in}}{\pgfqpoint{9.363817in}{4.016906in}}%
\pgfpathcurveto{\pgfqpoint{9.359699in}{4.021024in}}{\pgfqpoint{9.354113in}{4.023338in}}{\pgfqpoint{9.348289in}{4.023338in}}%
\pgfpathcurveto{\pgfqpoint{9.342465in}{4.023338in}}{\pgfqpoint{9.336879in}{4.021024in}}{\pgfqpoint{9.332760in}{4.016906in}}%
\pgfpathcurveto{\pgfqpoint{9.328642in}{4.012788in}}{\pgfqpoint{9.326328in}{4.007202in}}{\pgfqpoint{9.326328in}{4.001378in}}%
\pgfpathcurveto{\pgfqpoint{9.326328in}{3.995554in}}{\pgfqpoint{9.328642in}{3.989968in}}{\pgfqpoint{9.332760in}{3.985850in}}%
\pgfpathcurveto{\pgfqpoint{9.336879in}{3.981731in}}{\pgfqpoint{9.342465in}{3.979418in}}{\pgfqpoint{9.348289in}{3.979418in}}%
\pgfpathlineto{\pgfqpoint{9.348289in}{3.979418in}}%
\pgfpathclose%
\pgfusepath{stroke,fill}%
\end{pgfscope}%
\begin{pgfscope}%
\pgfpathrectangle{\pgfqpoint{0.640323in}{0.527436in}}{\pgfqpoint{9.687500in}{3.850000in}}%
\pgfusepath{clip}%
\pgfsetbuttcap%
\pgfsetroundjoin%
\definecolor{currentfill}{rgb}{0.000000,0.000000,0.000000}%
\pgfsetfillcolor{currentfill}%
\pgfsetfillopacity{0.500000}%
\pgfsetlinewidth{1.003750pt}%
\definecolor{currentstroke}{rgb}{0.000000,0.000000,0.000000}%
\pgfsetstrokecolor{currentstroke}%
\pgfsetstrokeopacity{0.500000}%
\pgfsetdash{{3.700000pt}{1.600000pt}}{0.000000pt}%
\pgfpathmoveto{\pgfqpoint{9.528020in}{3.992340in}}%
\pgfpathcurveto{\pgfqpoint{9.533844in}{3.992340in}}{\pgfqpoint{9.539430in}{3.994654in}}{\pgfqpoint{9.543548in}{3.998772in}}%
\pgfpathcurveto{\pgfqpoint{9.547666in}{4.002890in}}{\pgfqpoint{9.549980in}{4.008476in}}{\pgfqpoint{9.549980in}{4.014300in}}%
\pgfpathcurveto{\pgfqpoint{9.549980in}{4.020124in}}{\pgfqpoint{9.547666in}{4.025710in}}{\pgfqpoint{9.543548in}{4.029828in}}%
\pgfpathcurveto{\pgfqpoint{9.539430in}{4.033946in}}{\pgfqpoint{9.533844in}{4.036260in}}{\pgfqpoint{9.528020in}{4.036260in}}%
\pgfpathcurveto{\pgfqpoint{9.522196in}{4.036260in}}{\pgfqpoint{9.516610in}{4.033946in}}{\pgfqpoint{9.512491in}{4.029828in}}%
\pgfpathcurveto{\pgfqpoint{9.508373in}{4.025710in}}{\pgfqpoint{9.506059in}{4.020124in}}{\pgfqpoint{9.506059in}{4.014300in}}%
\pgfpathcurveto{\pgfqpoint{9.506059in}{4.008476in}}{\pgfqpoint{9.508373in}{4.002890in}}{\pgfqpoint{9.512491in}{3.998772in}}%
\pgfpathcurveto{\pgfqpoint{9.516610in}{3.994654in}}{\pgfqpoint{9.522196in}{3.992340in}}{\pgfqpoint{9.528020in}{3.992340in}}%
\pgfpathlineto{\pgfqpoint{9.528020in}{3.992340in}}%
\pgfpathclose%
\pgfusepath{stroke,fill}%
\end{pgfscope}%
\begin{pgfscope}%
\pgfpathrectangle{\pgfqpoint{0.640323in}{0.527436in}}{\pgfqpoint{9.687500in}{3.850000in}}%
\pgfusepath{clip}%
\pgfsetbuttcap%
\pgfsetroundjoin%
\definecolor{currentfill}{rgb}{0.000000,0.000000,0.000000}%
\pgfsetfillcolor{currentfill}%
\pgfsetfillopacity{0.500000}%
\pgfsetlinewidth{1.003750pt}%
\definecolor{currentstroke}{rgb}{0.000000,0.000000,0.000000}%
\pgfsetstrokecolor{currentstroke}%
\pgfsetstrokeopacity{0.500000}%
\pgfsetdash{{3.700000pt}{1.600000pt}}{0.000000pt}%
\pgfpathmoveto{\pgfqpoint{9.707751in}{4.006908in}}%
\pgfpathcurveto{\pgfqpoint{9.713575in}{4.006908in}}{\pgfqpoint{9.719161in}{4.009222in}}{\pgfqpoint{9.723279in}{4.013340in}}%
\pgfpathcurveto{\pgfqpoint{9.727397in}{4.017458in}}{\pgfqpoint{9.729711in}{4.023044in}}{\pgfqpoint{9.729711in}{4.028868in}}%
\pgfpathcurveto{\pgfqpoint{9.729711in}{4.034692in}}{\pgfqpoint{9.727397in}{4.040278in}}{\pgfqpoint{9.723279in}{4.044396in}}%
\pgfpathcurveto{\pgfqpoint{9.719161in}{4.048514in}}{\pgfqpoint{9.713575in}{4.050828in}}{\pgfqpoint{9.707751in}{4.050828in}}%
\pgfpathcurveto{\pgfqpoint{9.701927in}{4.050828in}}{\pgfqpoint{9.696340in}{4.048514in}}{\pgfqpoint{9.692222in}{4.044396in}}%
\pgfpathcurveto{\pgfqpoint{9.688104in}{4.040278in}}{\pgfqpoint{9.685790in}{4.034692in}}{\pgfqpoint{9.685790in}{4.028868in}}%
\pgfpathcurveto{\pgfqpoint{9.685790in}{4.023044in}}{\pgfqpoint{9.688104in}{4.017458in}}{\pgfqpoint{9.692222in}{4.013340in}}%
\pgfpathcurveto{\pgfqpoint{9.696340in}{4.009222in}}{\pgfqpoint{9.701927in}{4.006908in}}{\pgfqpoint{9.707751in}{4.006908in}}%
\pgfpathlineto{\pgfqpoint{9.707751in}{4.006908in}}%
\pgfpathclose%
\pgfusepath{stroke,fill}%
\end{pgfscope}%
\begin{pgfscope}%
\pgfpathrectangle{\pgfqpoint{0.640323in}{0.527436in}}{\pgfqpoint{9.687500in}{3.850000in}}%
\pgfusepath{clip}%
\pgfsetbuttcap%
\pgfsetroundjoin%
\definecolor{currentfill}{rgb}{0.000000,0.000000,0.000000}%
\pgfsetfillcolor{currentfill}%
\pgfsetfillopacity{0.500000}%
\pgfsetlinewidth{1.003750pt}%
\definecolor{currentstroke}{rgb}{0.000000,0.000000,0.000000}%
\pgfsetstrokecolor{currentstroke}%
\pgfsetstrokeopacity{0.500000}%
\pgfsetdash{{3.700000pt}{1.600000pt}}{0.000000pt}%
\pgfpathmoveto{\pgfqpoint{9.887482in}{4.020463in}}%
\pgfpathcurveto{\pgfqpoint{9.893306in}{4.020463in}}{\pgfqpoint{9.898892in}{4.022777in}}{\pgfqpoint{9.903010in}{4.026895in}}%
\pgfpathcurveto{\pgfqpoint{9.907128in}{4.031014in}}{\pgfqpoint{9.909442in}{4.036600in}}{\pgfqpoint{9.909442in}{4.042424in}}%
\pgfpathcurveto{\pgfqpoint{9.909442in}{4.048248in}}{\pgfqpoint{9.907128in}{4.053834in}}{\pgfqpoint{9.903010in}{4.057952in}}%
\pgfpathcurveto{\pgfqpoint{9.898892in}{4.062070in}}{\pgfqpoint{9.893306in}{4.064384in}}{\pgfqpoint{9.887482in}{4.064384in}}%
\pgfpathcurveto{\pgfqpoint{9.881658in}{4.064384in}}{\pgfqpoint{9.876071in}{4.062070in}}{\pgfqpoint{9.871953in}{4.057952in}}%
\pgfpathcurveto{\pgfqpoint{9.867835in}{4.053834in}}{\pgfqpoint{9.865521in}{4.048248in}}{\pgfqpoint{9.865521in}{4.042424in}}%
\pgfpathcurveto{\pgfqpoint{9.865521in}{4.036600in}}{\pgfqpoint{9.867835in}{4.031014in}}{\pgfqpoint{9.871953in}{4.026895in}}%
\pgfpathcurveto{\pgfqpoint{9.876071in}{4.022777in}}{\pgfqpoint{9.881658in}{4.020463in}}{\pgfqpoint{9.887482in}{4.020463in}}%
\pgfpathlineto{\pgfqpoint{9.887482in}{4.020463in}}%
\pgfpathclose%
\pgfusepath{stroke,fill}%
\end{pgfscope}%
\begin{pgfscope}%
\pgfpathrectangle{\pgfqpoint{0.640323in}{0.527436in}}{\pgfqpoint{9.687500in}{3.850000in}}%
\pgfusepath{clip}%
\pgfsetrectcap%
\pgfsetroundjoin%
\pgfsetlinewidth{0.803000pt}%
\definecolor{currentstroke}{rgb}{0.690196,0.690196,0.690196}%
\pgfsetstrokecolor{currentstroke}%
\pgfsetdash{}{0pt}%
\pgfpathmoveto{\pgfqpoint{1.080663in}{0.527436in}}%
\pgfpathlineto{\pgfqpoint{1.080663in}{4.377436in}}%
\pgfusepath{stroke}%
\end{pgfscope}%
\begin{pgfscope}%
\pgfsetbuttcap%
\pgfsetroundjoin%
\definecolor{currentfill}{rgb}{0.000000,0.000000,0.000000}%
\pgfsetfillcolor{currentfill}%
\pgfsetlinewidth{0.803000pt}%
\definecolor{currentstroke}{rgb}{0.000000,0.000000,0.000000}%
\pgfsetstrokecolor{currentstroke}%
\pgfsetdash{}{0pt}%
\pgfsys@defobject{currentmarker}{\pgfqpoint{0.000000in}{-0.048611in}}{\pgfqpoint{0.000000in}{0.000000in}}{%
\pgfpathmoveto{\pgfqpoint{0.000000in}{0.000000in}}%
\pgfpathlineto{\pgfqpoint{0.000000in}{-0.048611in}}%
\pgfusepath{stroke,fill}%
}%
\begin{pgfscope}%
\pgfsys@transformshift{1.080663in}{0.527436in}%
\pgfsys@useobject{currentmarker}{}%
\end{pgfscope}%
\end{pgfscope}%
\begin{pgfscope}%
\definecolor{textcolor}{rgb}{0.000000,0.000000,0.000000}%
\pgfsetstrokecolor{textcolor}%
\pgfsetfillcolor{textcolor}%
\pgftext[x=1.080663in,y=0.430214in,,top]{\color{textcolor}\sffamily\fontsize{10.000000}{12.000000}\selectfont 0.0}%
\end{pgfscope}%
\begin{pgfscope}%
\pgfpathrectangle{\pgfqpoint{0.640323in}{0.527436in}}{\pgfqpoint{9.687500in}{3.850000in}}%
\pgfusepath{clip}%
\pgfsetrectcap%
\pgfsetroundjoin%
\pgfsetlinewidth{0.803000pt}%
\definecolor{currentstroke}{rgb}{0.690196,0.690196,0.690196}%
\pgfsetstrokecolor{currentstroke}%
\pgfsetdash{}{0pt}%
\pgfpathmoveto{\pgfqpoint{2.877973in}{0.527436in}}%
\pgfpathlineto{\pgfqpoint{2.877973in}{4.377436in}}%
\pgfusepath{stroke}%
\end{pgfscope}%
\begin{pgfscope}%
\pgfsetbuttcap%
\pgfsetroundjoin%
\definecolor{currentfill}{rgb}{0.000000,0.000000,0.000000}%
\pgfsetfillcolor{currentfill}%
\pgfsetlinewidth{0.803000pt}%
\definecolor{currentstroke}{rgb}{0.000000,0.000000,0.000000}%
\pgfsetstrokecolor{currentstroke}%
\pgfsetdash{}{0pt}%
\pgfsys@defobject{currentmarker}{\pgfqpoint{0.000000in}{-0.048611in}}{\pgfqpoint{0.000000in}{0.000000in}}{%
\pgfpathmoveto{\pgfqpoint{0.000000in}{0.000000in}}%
\pgfpathlineto{\pgfqpoint{0.000000in}{-0.048611in}}%
\pgfusepath{stroke,fill}%
}%
\begin{pgfscope}%
\pgfsys@transformshift{2.877973in}{0.527436in}%
\pgfsys@useobject{currentmarker}{}%
\end{pgfscope}%
\end{pgfscope}%
\begin{pgfscope}%
\definecolor{textcolor}{rgb}{0.000000,0.000000,0.000000}%
\pgfsetstrokecolor{textcolor}%
\pgfsetfillcolor{textcolor}%
\pgftext[x=2.877973in,y=0.430214in,,top]{\color{textcolor}\sffamily\fontsize{10.000000}{12.000000}\selectfont 0.2}%
\end{pgfscope}%
\begin{pgfscope}%
\pgfpathrectangle{\pgfqpoint{0.640323in}{0.527436in}}{\pgfqpoint{9.687500in}{3.850000in}}%
\pgfusepath{clip}%
\pgfsetrectcap%
\pgfsetroundjoin%
\pgfsetlinewidth{0.803000pt}%
\definecolor{currentstroke}{rgb}{0.690196,0.690196,0.690196}%
\pgfsetstrokecolor{currentstroke}%
\pgfsetdash{}{0pt}%
\pgfpathmoveto{\pgfqpoint{4.675283in}{0.527436in}}%
\pgfpathlineto{\pgfqpoint{4.675283in}{4.377436in}}%
\pgfusepath{stroke}%
\end{pgfscope}%
\begin{pgfscope}%
\pgfsetbuttcap%
\pgfsetroundjoin%
\definecolor{currentfill}{rgb}{0.000000,0.000000,0.000000}%
\pgfsetfillcolor{currentfill}%
\pgfsetlinewidth{0.803000pt}%
\definecolor{currentstroke}{rgb}{0.000000,0.000000,0.000000}%
\pgfsetstrokecolor{currentstroke}%
\pgfsetdash{}{0pt}%
\pgfsys@defobject{currentmarker}{\pgfqpoint{0.000000in}{-0.048611in}}{\pgfqpoint{0.000000in}{0.000000in}}{%
\pgfpathmoveto{\pgfqpoint{0.000000in}{0.000000in}}%
\pgfpathlineto{\pgfqpoint{0.000000in}{-0.048611in}}%
\pgfusepath{stroke,fill}%
}%
\begin{pgfscope}%
\pgfsys@transformshift{4.675283in}{0.527436in}%
\pgfsys@useobject{currentmarker}{}%
\end{pgfscope}%
\end{pgfscope}%
\begin{pgfscope}%
\definecolor{textcolor}{rgb}{0.000000,0.000000,0.000000}%
\pgfsetstrokecolor{textcolor}%
\pgfsetfillcolor{textcolor}%
\pgftext[x=4.675283in,y=0.430214in,,top]{\color{textcolor}\sffamily\fontsize{10.000000}{12.000000}\selectfont 0.4}%
\end{pgfscope}%
\begin{pgfscope}%
\pgfpathrectangle{\pgfqpoint{0.640323in}{0.527436in}}{\pgfqpoint{9.687500in}{3.850000in}}%
\pgfusepath{clip}%
\pgfsetrectcap%
\pgfsetroundjoin%
\pgfsetlinewidth{0.803000pt}%
\definecolor{currentstroke}{rgb}{0.690196,0.690196,0.690196}%
\pgfsetstrokecolor{currentstroke}%
\pgfsetdash{}{0pt}%
\pgfpathmoveto{\pgfqpoint{6.472593in}{0.527436in}}%
\pgfpathlineto{\pgfqpoint{6.472593in}{4.377436in}}%
\pgfusepath{stroke}%
\end{pgfscope}%
\begin{pgfscope}%
\pgfsetbuttcap%
\pgfsetroundjoin%
\definecolor{currentfill}{rgb}{0.000000,0.000000,0.000000}%
\pgfsetfillcolor{currentfill}%
\pgfsetlinewidth{0.803000pt}%
\definecolor{currentstroke}{rgb}{0.000000,0.000000,0.000000}%
\pgfsetstrokecolor{currentstroke}%
\pgfsetdash{}{0pt}%
\pgfsys@defobject{currentmarker}{\pgfqpoint{0.000000in}{-0.048611in}}{\pgfqpoint{0.000000in}{0.000000in}}{%
\pgfpathmoveto{\pgfqpoint{0.000000in}{0.000000in}}%
\pgfpathlineto{\pgfqpoint{0.000000in}{-0.048611in}}%
\pgfusepath{stroke,fill}%
}%
\begin{pgfscope}%
\pgfsys@transformshift{6.472593in}{0.527436in}%
\pgfsys@useobject{currentmarker}{}%
\end{pgfscope}%
\end{pgfscope}%
\begin{pgfscope}%
\definecolor{textcolor}{rgb}{0.000000,0.000000,0.000000}%
\pgfsetstrokecolor{textcolor}%
\pgfsetfillcolor{textcolor}%
\pgftext[x=6.472593in,y=0.430214in,,top]{\color{textcolor}\sffamily\fontsize{10.000000}{12.000000}\selectfont 0.6}%
\end{pgfscope}%
\begin{pgfscope}%
\pgfpathrectangle{\pgfqpoint{0.640323in}{0.527436in}}{\pgfqpoint{9.687500in}{3.850000in}}%
\pgfusepath{clip}%
\pgfsetrectcap%
\pgfsetroundjoin%
\pgfsetlinewidth{0.803000pt}%
\definecolor{currentstroke}{rgb}{0.690196,0.690196,0.690196}%
\pgfsetstrokecolor{currentstroke}%
\pgfsetdash{}{0pt}%
\pgfpathmoveto{\pgfqpoint{8.269903in}{0.527436in}}%
\pgfpathlineto{\pgfqpoint{8.269903in}{4.377436in}}%
\pgfusepath{stroke}%
\end{pgfscope}%
\begin{pgfscope}%
\pgfsetbuttcap%
\pgfsetroundjoin%
\definecolor{currentfill}{rgb}{0.000000,0.000000,0.000000}%
\pgfsetfillcolor{currentfill}%
\pgfsetlinewidth{0.803000pt}%
\definecolor{currentstroke}{rgb}{0.000000,0.000000,0.000000}%
\pgfsetstrokecolor{currentstroke}%
\pgfsetdash{}{0pt}%
\pgfsys@defobject{currentmarker}{\pgfqpoint{0.000000in}{-0.048611in}}{\pgfqpoint{0.000000in}{0.000000in}}{%
\pgfpathmoveto{\pgfqpoint{0.000000in}{0.000000in}}%
\pgfpathlineto{\pgfqpoint{0.000000in}{-0.048611in}}%
\pgfusepath{stroke,fill}%
}%
\begin{pgfscope}%
\pgfsys@transformshift{8.269903in}{0.527436in}%
\pgfsys@useobject{currentmarker}{}%
\end{pgfscope}%
\end{pgfscope}%
\begin{pgfscope}%
\definecolor{textcolor}{rgb}{0.000000,0.000000,0.000000}%
\pgfsetstrokecolor{textcolor}%
\pgfsetfillcolor{textcolor}%
\pgftext[x=8.269903in,y=0.430214in,,top]{\color{textcolor}\sffamily\fontsize{10.000000}{12.000000}\selectfont 0.8}%
\end{pgfscope}%
\begin{pgfscope}%
\pgfpathrectangle{\pgfqpoint{0.640323in}{0.527436in}}{\pgfqpoint{9.687500in}{3.850000in}}%
\pgfusepath{clip}%
\pgfsetrectcap%
\pgfsetroundjoin%
\pgfsetlinewidth{0.803000pt}%
\definecolor{currentstroke}{rgb}{0.690196,0.690196,0.690196}%
\pgfsetstrokecolor{currentstroke}%
\pgfsetdash{}{0pt}%
\pgfpathmoveto{\pgfqpoint{10.067213in}{0.527436in}}%
\pgfpathlineto{\pgfqpoint{10.067213in}{4.377436in}}%
\pgfusepath{stroke}%
\end{pgfscope}%
\begin{pgfscope}%
\pgfsetbuttcap%
\pgfsetroundjoin%
\definecolor{currentfill}{rgb}{0.000000,0.000000,0.000000}%
\pgfsetfillcolor{currentfill}%
\pgfsetlinewidth{0.803000pt}%
\definecolor{currentstroke}{rgb}{0.000000,0.000000,0.000000}%
\pgfsetstrokecolor{currentstroke}%
\pgfsetdash{}{0pt}%
\pgfsys@defobject{currentmarker}{\pgfqpoint{0.000000in}{-0.048611in}}{\pgfqpoint{0.000000in}{0.000000in}}{%
\pgfpathmoveto{\pgfqpoint{0.000000in}{0.000000in}}%
\pgfpathlineto{\pgfqpoint{0.000000in}{-0.048611in}}%
\pgfusepath{stroke,fill}%
}%
\begin{pgfscope}%
\pgfsys@transformshift{10.067213in}{0.527436in}%
\pgfsys@useobject{currentmarker}{}%
\end{pgfscope}%
\end{pgfscope}%
\begin{pgfscope}%
\definecolor{textcolor}{rgb}{0.000000,0.000000,0.000000}%
\pgfsetstrokecolor{textcolor}%
\pgfsetfillcolor{textcolor}%
\pgftext[x=10.067213in,y=0.430214in,,top]{\color{textcolor}\sffamily\fontsize{10.000000}{12.000000}\selectfont 1.0}%
\end{pgfscope}%
\begin{pgfscope}%
\pgfpathrectangle{\pgfqpoint{0.640323in}{0.527436in}}{\pgfqpoint{9.687500in}{3.850000in}}%
\pgfusepath{clip}%
\pgfsetrectcap%
\pgfsetroundjoin%
\pgfsetlinewidth{0.803000pt}%
\definecolor{currentstroke}{rgb}{0.600000,0.600000,0.600000}%
\pgfsetstrokecolor{currentstroke}%
\pgfsetstrokeopacity{0.200000}%
\pgfsetdash{}{0pt}%
\pgfpathmoveto{\pgfqpoint{1.529991in}{0.527436in}}%
\pgfpathlineto{\pgfqpoint{1.529991in}{4.377436in}}%
\pgfusepath{stroke}%
\end{pgfscope}%
\begin{pgfscope}%
\pgfsetbuttcap%
\pgfsetroundjoin%
\definecolor{currentfill}{rgb}{0.000000,0.000000,0.000000}%
\pgfsetfillcolor{currentfill}%
\pgfsetlinewidth{0.602250pt}%
\definecolor{currentstroke}{rgb}{0.000000,0.000000,0.000000}%
\pgfsetstrokecolor{currentstroke}%
\pgfsetdash{}{0pt}%
\pgfsys@defobject{currentmarker}{\pgfqpoint{0.000000in}{-0.027778in}}{\pgfqpoint{0.000000in}{0.000000in}}{%
\pgfpathmoveto{\pgfqpoint{0.000000in}{0.000000in}}%
\pgfpathlineto{\pgfqpoint{0.000000in}{-0.027778in}}%
\pgfusepath{stroke,fill}%
}%
\begin{pgfscope}%
\pgfsys@transformshift{1.529991in}{0.527436in}%
\pgfsys@useobject{currentmarker}{}%
\end{pgfscope}%
\end{pgfscope}%
\begin{pgfscope}%
\pgfpathrectangle{\pgfqpoint{0.640323in}{0.527436in}}{\pgfqpoint{9.687500in}{3.850000in}}%
\pgfusepath{clip}%
\pgfsetrectcap%
\pgfsetroundjoin%
\pgfsetlinewidth{0.803000pt}%
\definecolor{currentstroke}{rgb}{0.600000,0.600000,0.600000}%
\pgfsetstrokecolor{currentstroke}%
\pgfsetstrokeopacity{0.200000}%
\pgfsetdash{}{0pt}%
\pgfpathmoveto{\pgfqpoint{1.979318in}{0.527436in}}%
\pgfpathlineto{\pgfqpoint{1.979318in}{4.377436in}}%
\pgfusepath{stroke}%
\end{pgfscope}%
\begin{pgfscope}%
\pgfsetbuttcap%
\pgfsetroundjoin%
\definecolor{currentfill}{rgb}{0.000000,0.000000,0.000000}%
\pgfsetfillcolor{currentfill}%
\pgfsetlinewidth{0.602250pt}%
\definecolor{currentstroke}{rgb}{0.000000,0.000000,0.000000}%
\pgfsetstrokecolor{currentstroke}%
\pgfsetdash{}{0pt}%
\pgfsys@defobject{currentmarker}{\pgfqpoint{0.000000in}{-0.027778in}}{\pgfqpoint{0.000000in}{0.000000in}}{%
\pgfpathmoveto{\pgfqpoint{0.000000in}{0.000000in}}%
\pgfpathlineto{\pgfqpoint{0.000000in}{-0.027778in}}%
\pgfusepath{stroke,fill}%
}%
\begin{pgfscope}%
\pgfsys@transformshift{1.979318in}{0.527436in}%
\pgfsys@useobject{currentmarker}{}%
\end{pgfscope}%
\end{pgfscope}%
\begin{pgfscope}%
\pgfpathrectangle{\pgfqpoint{0.640323in}{0.527436in}}{\pgfqpoint{9.687500in}{3.850000in}}%
\pgfusepath{clip}%
\pgfsetrectcap%
\pgfsetroundjoin%
\pgfsetlinewidth{0.803000pt}%
\definecolor{currentstroke}{rgb}{0.600000,0.600000,0.600000}%
\pgfsetstrokecolor{currentstroke}%
\pgfsetstrokeopacity{0.200000}%
\pgfsetdash{}{0pt}%
\pgfpathmoveto{\pgfqpoint{2.428646in}{0.527436in}}%
\pgfpathlineto{\pgfqpoint{2.428646in}{4.377436in}}%
\pgfusepath{stroke}%
\end{pgfscope}%
\begin{pgfscope}%
\pgfsetbuttcap%
\pgfsetroundjoin%
\definecolor{currentfill}{rgb}{0.000000,0.000000,0.000000}%
\pgfsetfillcolor{currentfill}%
\pgfsetlinewidth{0.602250pt}%
\definecolor{currentstroke}{rgb}{0.000000,0.000000,0.000000}%
\pgfsetstrokecolor{currentstroke}%
\pgfsetdash{}{0pt}%
\pgfsys@defobject{currentmarker}{\pgfqpoint{0.000000in}{-0.027778in}}{\pgfqpoint{0.000000in}{0.000000in}}{%
\pgfpathmoveto{\pgfqpoint{0.000000in}{0.000000in}}%
\pgfpathlineto{\pgfqpoint{0.000000in}{-0.027778in}}%
\pgfusepath{stroke,fill}%
}%
\begin{pgfscope}%
\pgfsys@transformshift{2.428646in}{0.527436in}%
\pgfsys@useobject{currentmarker}{}%
\end{pgfscope}%
\end{pgfscope}%
\begin{pgfscope}%
\pgfpathrectangle{\pgfqpoint{0.640323in}{0.527436in}}{\pgfqpoint{9.687500in}{3.850000in}}%
\pgfusepath{clip}%
\pgfsetrectcap%
\pgfsetroundjoin%
\pgfsetlinewidth{0.803000pt}%
\definecolor{currentstroke}{rgb}{0.600000,0.600000,0.600000}%
\pgfsetstrokecolor{currentstroke}%
\pgfsetstrokeopacity{0.200000}%
\pgfsetdash{}{0pt}%
\pgfpathmoveto{\pgfqpoint{3.327301in}{0.527436in}}%
\pgfpathlineto{\pgfqpoint{3.327301in}{4.377436in}}%
\pgfusepath{stroke}%
\end{pgfscope}%
\begin{pgfscope}%
\pgfsetbuttcap%
\pgfsetroundjoin%
\definecolor{currentfill}{rgb}{0.000000,0.000000,0.000000}%
\pgfsetfillcolor{currentfill}%
\pgfsetlinewidth{0.602250pt}%
\definecolor{currentstroke}{rgb}{0.000000,0.000000,0.000000}%
\pgfsetstrokecolor{currentstroke}%
\pgfsetdash{}{0pt}%
\pgfsys@defobject{currentmarker}{\pgfqpoint{0.000000in}{-0.027778in}}{\pgfqpoint{0.000000in}{0.000000in}}{%
\pgfpathmoveto{\pgfqpoint{0.000000in}{0.000000in}}%
\pgfpathlineto{\pgfqpoint{0.000000in}{-0.027778in}}%
\pgfusepath{stroke,fill}%
}%
\begin{pgfscope}%
\pgfsys@transformshift{3.327301in}{0.527436in}%
\pgfsys@useobject{currentmarker}{}%
\end{pgfscope}%
\end{pgfscope}%
\begin{pgfscope}%
\pgfpathrectangle{\pgfqpoint{0.640323in}{0.527436in}}{\pgfqpoint{9.687500in}{3.850000in}}%
\pgfusepath{clip}%
\pgfsetrectcap%
\pgfsetroundjoin%
\pgfsetlinewidth{0.803000pt}%
\definecolor{currentstroke}{rgb}{0.600000,0.600000,0.600000}%
\pgfsetstrokecolor{currentstroke}%
\pgfsetstrokeopacity{0.200000}%
\pgfsetdash{}{0pt}%
\pgfpathmoveto{\pgfqpoint{3.776628in}{0.527436in}}%
\pgfpathlineto{\pgfqpoint{3.776628in}{4.377436in}}%
\pgfusepath{stroke}%
\end{pgfscope}%
\begin{pgfscope}%
\pgfsetbuttcap%
\pgfsetroundjoin%
\definecolor{currentfill}{rgb}{0.000000,0.000000,0.000000}%
\pgfsetfillcolor{currentfill}%
\pgfsetlinewidth{0.602250pt}%
\definecolor{currentstroke}{rgb}{0.000000,0.000000,0.000000}%
\pgfsetstrokecolor{currentstroke}%
\pgfsetdash{}{0pt}%
\pgfsys@defobject{currentmarker}{\pgfqpoint{0.000000in}{-0.027778in}}{\pgfqpoint{0.000000in}{0.000000in}}{%
\pgfpathmoveto{\pgfqpoint{0.000000in}{0.000000in}}%
\pgfpathlineto{\pgfqpoint{0.000000in}{-0.027778in}}%
\pgfusepath{stroke,fill}%
}%
\begin{pgfscope}%
\pgfsys@transformshift{3.776628in}{0.527436in}%
\pgfsys@useobject{currentmarker}{}%
\end{pgfscope}%
\end{pgfscope}%
\begin{pgfscope}%
\pgfpathrectangle{\pgfqpoint{0.640323in}{0.527436in}}{\pgfqpoint{9.687500in}{3.850000in}}%
\pgfusepath{clip}%
\pgfsetrectcap%
\pgfsetroundjoin%
\pgfsetlinewidth{0.803000pt}%
\definecolor{currentstroke}{rgb}{0.600000,0.600000,0.600000}%
\pgfsetstrokecolor{currentstroke}%
\pgfsetstrokeopacity{0.200000}%
\pgfsetdash{}{0pt}%
\pgfpathmoveto{\pgfqpoint{4.225956in}{0.527436in}}%
\pgfpathlineto{\pgfqpoint{4.225956in}{4.377436in}}%
\pgfusepath{stroke}%
\end{pgfscope}%
\begin{pgfscope}%
\pgfsetbuttcap%
\pgfsetroundjoin%
\definecolor{currentfill}{rgb}{0.000000,0.000000,0.000000}%
\pgfsetfillcolor{currentfill}%
\pgfsetlinewidth{0.602250pt}%
\definecolor{currentstroke}{rgb}{0.000000,0.000000,0.000000}%
\pgfsetstrokecolor{currentstroke}%
\pgfsetdash{}{0pt}%
\pgfsys@defobject{currentmarker}{\pgfqpoint{0.000000in}{-0.027778in}}{\pgfqpoint{0.000000in}{0.000000in}}{%
\pgfpathmoveto{\pgfqpoint{0.000000in}{0.000000in}}%
\pgfpathlineto{\pgfqpoint{0.000000in}{-0.027778in}}%
\pgfusepath{stroke,fill}%
}%
\begin{pgfscope}%
\pgfsys@transformshift{4.225956in}{0.527436in}%
\pgfsys@useobject{currentmarker}{}%
\end{pgfscope}%
\end{pgfscope}%
\begin{pgfscope}%
\pgfpathrectangle{\pgfqpoint{0.640323in}{0.527436in}}{\pgfqpoint{9.687500in}{3.850000in}}%
\pgfusepath{clip}%
\pgfsetrectcap%
\pgfsetroundjoin%
\pgfsetlinewidth{0.803000pt}%
\definecolor{currentstroke}{rgb}{0.600000,0.600000,0.600000}%
\pgfsetstrokecolor{currentstroke}%
\pgfsetstrokeopacity{0.200000}%
\pgfsetdash{}{0pt}%
\pgfpathmoveto{\pgfqpoint{5.124611in}{0.527436in}}%
\pgfpathlineto{\pgfqpoint{5.124611in}{4.377436in}}%
\pgfusepath{stroke}%
\end{pgfscope}%
\begin{pgfscope}%
\pgfsetbuttcap%
\pgfsetroundjoin%
\definecolor{currentfill}{rgb}{0.000000,0.000000,0.000000}%
\pgfsetfillcolor{currentfill}%
\pgfsetlinewidth{0.602250pt}%
\definecolor{currentstroke}{rgb}{0.000000,0.000000,0.000000}%
\pgfsetstrokecolor{currentstroke}%
\pgfsetdash{}{0pt}%
\pgfsys@defobject{currentmarker}{\pgfqpoint{0.000000in}{-0.027778in}}{\pgfqpoint{0.000000in}{0.000000in}}{%
\pgfpathmoveto{\pgfqpoint{0.000000in}{0.000000in}}%
\pgfpathlineto{\pgfqpoint{0.000000in}{-0.027778in}}%
\pgfusepath{stroke,fill}%
}%
\begin{pgfscope}%
\pgfsys@transformshift{5.124611in}{0.527436in}%
\pgfsys@useobject{currentmarker}{}%
\end{pgfscope}%
\end{pgfscope}%
\begin{pgfscope}%
\pgfpathrectangle{\pgfqpoint{0.640323in}{0.527436in}}{\pgfqpoint{9.687500in}{3.850000in}}%
\pgfusepath{clip}%
\pgfsetrectcap%
\pgfsetroundjoin%
\pgfsetlinewidth{0.803000pt}%
\definecolor{currentstroke}{rgb}{0.600000,0.600000,0.600000}%
\pgfsetstrokecolor{currentstroke}%
\pgfsetstrokeopacity{0.200000}%
\pgfsetdash{}{0pt}%
\pgfpathmoveto{\pgfqpoint{5.573938in}{0.527436in}}%
\pgfpathlineto{\pgfqpoint{5.573938in}{4.377436in}}%
\pgfusepath{stroke}%
\end{pgfscope}%
\begin{pgfscope}%
\pgfsetbuttcap%
\pgfsetroundjoin%
\definecolor{currentfill}{rgb}{0.000000,0.000000,0.000000}%
\pgfsetfillcolor{currentfill}%
\pgfsetlinewidth{0.602250pt}%
\definecolor{currentstroke}{rgb}{0.000000,0.000000,0.000000}%
\pgfsetstrokecolor{currentstroke}%
\pgfsetdash{}{0pt}%
\pgfsys@defobject{currentmarker}{\pgfqpoint{0.000000in}{-0.027778in}}{\pgfqpoint{0.000000in}{0.000000in}}{%
\pgfpathmoveto{\pgfqpoint{0.000000in}{0.000000in}}%
\pgfpathlineto{\pgfqpoint{0.000000in}{-0.027778in}}%
\pgfusepath{stroke,fill}%
}%
\begin{pgfscope}%
\pgfsys@transformshift{5.573938in}{0.527436in}%
\pgfsys@useobject{currentmarker}{}%
\end{pgfscope}%
\end{pgfscope}%
\begin{pgfscope}%
\pgfpathrectangle{\pgfqpoint{0.640323in}{0.527436in}}{\pgfqpoint{9.687500in}{3.850000in}}%
\pgfusepath{clip}%
\pgfsetrectcap%
\pgfsetroundjoin%
\pgfsetlinewidth{0.803000pt}%
\definecolor{currentstroke}{rgb}{0.600000,0.600000,0.600000}%
\pgfsetstrokecolor{currentstroke}%
\pgfsetstrokeopacity{0.200000}%
\pgfsetdash{}{0pt}%
\pgfpathmoveto{\pgfqpoint{6.023265in}{0.527436in}}%
\pgfpathlineto{\pgfqpoint{6.023265in}{4.377436in}}%
\pgfusepath{stroke}%
\end{pgfscope}%
\begin{pgfscope}%
\pgfsetbuttcap%
\pgfsetroundjoin%
\definecolor{currentfill}{rgb}{0.000000,0.000000,0.000000}%
\pgfsetfillcolor{currentfill}%
\pgfsetlinewidth{0.602250pt}%
\definecolor{currentstroke}{rgb}{0.000000,0.000000,0.000000}%
\pgfsetstrokecolor{currentstroke}%
\pgfsetdash{}{0pt}%
\pgfsys@defobject{currentmarker}{\pgfqpoint{0.000000in}{-0.027778in}}{\pgfqpoint{0.000000in}{0.000000in}}{%
\pgfpathmoveto{\pgfqpoint{0.000000in}{0.000000in}}%
\pgfpathlineto{\pgfqpoint{0.000000in}{-0.027778in}}%
\pgfusepath{stroke,fill}%
}%
\begin{pgfscope}%
\pgfsys@transformshift{6.023265in}{0.527436in}%
\pgfsys@useobject{currentmarker}{}%
\end{pgfscope}%
\end{pgfscope}%
\begin{pgfscope}%
\pgfpathrectangle{\pgfqpoint{0.640323in}{0.527436in}}{\pgfqpoint{9.687500in}{3.850000in}}%
\pgfusepath{clip}%
\pgfsetrectcap%
\pgfsetroundjoin%
\pgfsetlinewidth{0.803000pt}%
\definecolor{currentstroke}{rgb}{0.600000,0.600000,0.600000}%
\pgfsetstrokecolor{currentstroke}%
\pgfsetstrokeopacity{0.200000}%
\pgfsetdash{}{0pt}%
\pgfpathmoveto{\pgfqpoint{6.921920in}{0.527436in}}%
\pgfpathlineto{\pgfqpoint{6.921920in}{4.377436in}}%
\pgfusepath{stroke}%
\end{pgfscope}%
\begin{pgfscope}%
\pgfsetbuttcap%
\pgfsetroundjoin%
\definecolor{currentfill}{rgb}{0.000000,0.000000,0.000000}%
\pgfsetfillcolor{currentfill}%
\pgfsetlinewidth{0.602250pt}%
\definecolor{currentstroke}{rgb}{0.000000,0.000000,0.000000}%
\pgfsetstrokecolor{currentstroke}%
\pgfsetdash{}{0pt}%
\pgfsys@defobject{currentmarker}{\pgfqpoint{0.000000in}{-0.027778in}}{\pgfqpoint{0.000000in}{0.000000in}}{%
\pgfpathmoveto{\pgfqpoint{0.000000in}{0.000000in}}%
\pgfpathlineto{\pgfqpoint{0.000000in}{-0.027778in}}%
\pgfusepath{stroke,fill}%
}%
\begin{pgfscope}%
\pgfsys@transformshift{6.921920in}{0.527436in}%
\pgfsys@useobject{currentmarker}{}%
\end{pgfscope}%
\end{pgfscope}%
\begin{pgfscope}%
\pgfpathrectangle{\pgfqpoint{0.640323in}{0.527436in}}{\pgfqpoint{9.687500in}{3.850000in}}%
\pgfusepath{clip}%
\pgfsetrectcap%
\pgfsetroundjoin%
\pgfsetlinewidth{0.803000pt}%
\definecolor{currentstroke}{rgb}{0.600000,0.600000,0.600000}%
\pgfsetstrokecolor{currentstroke}%
\pgfsetstrokeopacity{0.200000}%
\pgfsetdash{}{0pt}%
\pgfpathmoveto{\pgfqpoint{7.371248in}{0.527436in}}%
\pgfpathlineto{\pgfqpoint{7.371248in}{4.377436in}}%
\pgfusepath{stroke}%
\end{pgfscope}%
\begin{pgfscope}%
\pgfsetbuttcap%
\pgfsetroundjoin%
\definecolor{currentfill}{rgb}{0.000000,0.000000,0.000000}%
\pgfsetfillcolor{currentfill}%
\pgfsetlinewidth{0.602250pt}%
\definecolor{currentstroke}{rgb}{0.000000,0.000000,0.000000}%
\pgfsetstrokecolor{currentstroke}%
\pgfsetdash{}{0pt}%
\pgfsys@defobject{currentmarker}{\pgfqpoint{0.000000in}{-0.027778in}}{\pgfqpoint{0.000000in}{0.000000in}}{%
\pgfpathmoveto{\pgfqpoint{0.000000in}{0.000000in}}%
\pgfpathlineto{\pgfqpoint{0.000000in}{-0.027778in}}%
\pgfusepath{stroke,fill}%
}%
\begin{pgfscope}%
\pgfsys@transformshift{7.371248in}{0.527436in}%
\pgfsys@useobject{currentmarker}{}%
\end{pgfscope}%
\end{pgfscope}%
\begin{pgfscope}%
\pgfpathrectangle{\pgfqpoint{0.640323in}{0.527436in}}{\pgfqpoint{9.687500in}{3.850000in}}%
\pgfusepath{clip}%
\pgfsetrectcap%
\pgfsetroundjoin%
\pgfsetlinewidth{0.803000pt}%
\definecolor{currentstroke}{rgb}{0.600000,0.600000,0.600000}%
\pgfsetstrokecolor{currentstroke}%
\pgfsetstrokeopacity{0.200000}%
\pgfsetdash{}{0pt}%
\pgfpathmoveto{\pgfqpoint{7.820575in}{0.527436in}}%
\pgfpathlineto{\pgfqpoint{7.820575in}{4.377436in}}%
\pgfusepath{stroke}%
\end{pgfscope}%
\begin{pgfscope}%
\pgfsetbuttcap%
\pgfsetroundjoin%
\definecolor{currentfill}{rgb}{0.000000,0.000000,0.000000}%
\pgfsetfillcolor{currentfill}%
\pgfsetlinewidth{0.602250pt}%
\definecolor{currentstroke}{rgb}{0.000000,0.000000,0.000000}%
\pgfsetstrokecolor{currentstroke}%
\pgfsetdash{}{0pt}%
\pgfsys@defobject{currentmarker}{\pgfqpoint{0.000000in}{-0.027778in}}{\pgfqpoint{0.000000in}{0.000000in}}{%
\pgfpathmoveto{\pgfqpoint{0.000000in}{0.000000in}}%
\pgfpathlineto{\pgfqpoint{0.000000in}{-0.027778in}}%
\pgfusepath{stroke,fill}%
}%
\begin{pgfscope}%
\pgfsys@transformshift{7.820575in}{0.527436in}%
\pgfsys@useobject{currentmarker}{}%
\end{pgfscope}%
\end{pgfscope}%
\begin{pgfscope}%
\pgfpathrectangle{\pgfqpoint{0.640323in}{0.527436in}}{\pgfqpoint{9.687500in}{3.850000in}}%
\pgfusepath{clip}%
\pgfsetrectcap%
\pgfsetroundjoin%
\pgfsetlinewidth{0.803000pt}%
\definecolor{currentstroke}{rgb}{0.600000,0.600000,0.600000}%
\pgfsetstrokecolor{currentstroke}%
\pgfsetstrokeopacity{0.200000}%
\pgfsetdash{}{0pt}%
\pgfpathmoveto{\pgfqpoint{8.719230in}{0.527436in}}%
\pgfpathlineto{\pgfqpoint{8.719230in}{4.377436in}}%
\pgfusepath{stroke}%
\end{pgfscope}%
\begin{pgfscope}%
\pgfsetbuttcap%
\pgfsetroundjoin%
\definecolor{currentfill}{rgb}{0.000000,0.000000,0.000000}%
\pgfsetfillcolor{currentfill}%
\pgfsetlinewidth{0.602250pt}%
\definecolor{currentstroke}{rgb}{0.000000,0.000000,0.000000}%
\pgfsetstrokecolor{currentstroke}%
\pgfsetdash{}{0pt}%
\pgfsys@defobject{currentmarker}{\pgfqpoint{0.000000in}{-0.027778in}}{\pgfqpoint{0.000000in}{0.000000in}}{%
\pgfpathmoveto{\pgfqpoint{0.000000in}{0.000000in}}%
\pgfpathlineto{\pgfqpoint{0.000000in}{-0.027778in}}%
\pgfusepath{stroke,fill}%
}%
\begin{pgfscope}%
\pgfsys@transformshift{8.719230in}{0.527436in}%
\pgfsys@useobject{currentmarker}{}%
\end{pgfscope}%
\end{pgfscope}%
\begin{pgfscope}%
\pgfpathrectangle{\pgfqpoint{0.640323in}{0.527436in}}{\pgfqpoint{9.687500in}{3.850000in}}%
\pgfusepath{clip}%
\pgfsetrectcap%
\pgfsetroundjoin%
\pgfsetlinewidth{0.803000pt}%
\definecolor{currentstroke}{rgb}{0.600000,0.600000,0.600000}%
\pgfsetstrokecolor{currentstroke}%
\pgfsetstrokeopacity{0.200000}%
\pgfsetdash{}{0pt}%
\pgfpathmoveto{\pgfqpoint{9.168558in}{0.527436in}}%
\pgfpathlineto{\pgfqpoint{9.168558in}{4.377436in}}%
\pgfusepath{stroke}%
\end{pgfscope}%
\begin{pgfscope}%
\pgfsetbuttcap%
\pgfsetroundjoin%
\definecolor{currentfill}{rgb}{0.000000,0.000000,0.000000}%
\pgfsetfillcolor{currentfill}%
\pgfsetlinewidth{0.602250pt}%
\definecolor{currentstroke}{rgb}{0.000000,0.000000,0.000000}%
\pgfsetstrokecolor{currentstroke}%
\pgfsetdash{}{0pt}%
\pgfsys@defobject{currentmarker}{\pgfqpoint{0.000000in}{-0.027778in}}{\pgfqpoint{0.000000in}{0.000000in}}{%
\pgfpathmoveto{\pgfqpoint{0.000000in}{0.000000in}}%
\pgfpathlineto{\pgfqpoint{0.000000in}{-0.027778in}}%
\pgfusepath{stroke,fill}%
}%
\begin{pgfscope}%
\pgfsys@transformshift{9.168558in}{0.527436in}%
\pgfsys@useobject{currentmarker}{}%
\end{pgfscope}%
\end{pgfscope}%
\begin{pgfscope}%
\pgfpathrectangle{\pgfqpoint{0.640323in}{0.527436in}}{\pgfqpoint{9.687500in}{3.850000in}}%
\pgfusepath{clip}%
\pgfsetrectcap%
\pgfsetroundjoin%
\pgfsetlinewidth{0.803000pt}%
\definecolor{currentstroke}{rgb}{0.600000,0.600000,0.600000}%
\pgfsetstrokecolor{currentstroke}%
\pgfsetstrokeopacity{0.200000}%
\pgfsetdash{}{0pt}%
\pgfpathmoveto{\pgfqpoint{9.617885in}{0.527436in}}%
\pgfpathlineto{\pgfqpoint{9.617885in}{4.377436in}}%
\pgfusepath{stroke}%
\end{pgfscope}%
\begin{pgfscope}%
\pgfsetbuttcap%
\pgfsetroundjoin%
\definecolor{currentfill}{rgb}{0.000000,0.000000,0.000000}%
\pgfsetfillcolor{currentfill}%
\pgfsetlinewidth{0.602250pt}%
\definecolor{currentstroke}{rgb}{0.000000,0.000000,0.000000}%
\pgfsetstrokecolor{currentstroke}%
\pgfsetdash{}{0pt}%
\pgfsys@defobject{currentmarker}{\pgfqpoint{0.000000in}{-0.027778in}}{\pgfqpoint{0.000000in}{0.000000in}}{%
\pgfpathmoveto{\pgfqpoint{0.000000in}{0.000000in}}%
\pgfpathlineto{\pgfqpoint{0.000000in}{-0.027778in}}%
\pgfusepath{stroke,fill}%
}%
\begin{pgfscope}%
\pgfsys@transformshift{9.617885in}{0.527436in}%
\pgfsys@useobject{currentmarker}{}%
\end{pgfscope}%
\end{pgfscope}%
\begin{pgfscope}%
\definecolor{textcolor}{rgb}{0.000000,0.000000,0.000000}%
\pgfsetstrokecolor{textcolor}%
\pgfsetfillcolor{textcolor}%
\pgftext[x=5.484072in,y=0.240245in,,top]{\color{textcolor}\sffamily\fontsize{10.000000}{12.000000}\selectfont turnover probability \(\displaystyle p_1\,(S\rightarrow I\,)\)}%
\end{pgfscope}%
\begin{pgfscope}%
\pgfpathrectangle{\pgfqpoint{0.640323in}{0.527436in}}{\pgfqpoint{9.687500in}{3.850000in}}%
\pgfusepath{clip}%
\pgfsetrectcap%
\pgfsetroundjoin%
\pgfsetlinewidth{0.803000pt}%
\definecolor{currentstroke}{rgb}{0.690196,0.690196,0.690196}%
\pgfsetstrokecolor{currentstroke}%
\pgfsetdash{}{0pt}%
\pgfpathmoveto{\pgfqpoint{0.640323in}{0.651629in}}%
\pgfpathlineto{\pgfqpoint{10.327822in}{0.651629in}}%
\pgfusepath{stroke}%
\end{pgfscope}%
\begin{pgfscope}%
\pgfsetbuttcap%
\pgfsetroundjoin%
\definecolor{currentfill}{rgb}{0.000000,0.000000,0.000000}%
\pgfsetfillcolor{currentfill}%
\pgfsetlinewidth{0.803000pt}%
\definecolor{currentstroke}{rgb}{0.000000,0.000000,0.000000}%
\pgfsetstrokecolor{currentstroke}%
\pgfsetdash{}{0pt}%
\pgfsys@defobject{currentmarker}{\pgfqpoint{-0.048611in}{0.000000in}}{\pgfqpoint{-0.000000in}{0.000000in}}{%
\pgfpathmoveto{\pgfqpoint{-0.000000in}{0.000000in}}%
\pgfpathlineto{\pgfqpoint{-0.048611in}{0.000000in}}%
\pgfusepath{stroke,fill}%
}%
\begin{pgfscope}%
\pgfsys@transformshift{0.640323in}{0.651629in}%
\pgfsys@useobject{currentmarker}{}%
\end{pgfscope}%
\end{pgfscope}%
\begin{pgfscope}%
\definecolor{textcolor}{rgb}{0.000000,0.000000,0.000000}%
\pgfsetstrokecolor{textcolor}%
\pgfsetfillcolor{textcolor}%
\pgftext[x=0.322221in, y=0.598868in, left, base]{\color{textcolor}\sffamily\fontsize{10.000000}{12.000000}\selectfont 0.0}%
\end{pgfscope}%
\begin{pgfscope}%
\pgfpathrectangle{\pgfqpoint{0.640323in}{0.527436in}}{\pgfqpoint{9.687500in}{3.850000in}}%
\pgfusepath{clip}%
\pgfsetrectcap%
\pgfsetroundjoin%
\pgfsetlinewidth{0.803000pt}%
\definecolor{currentstroke}{rgb}{0.690196,0.690196,0.690196}%
\pgfsetstrokecolor{currentstroke}%
\pgfsetdash{}{0pt}%
\pgfpathmoveto{\pgfqpoint{0.640323in}{1.272597in}}%
\pgfpathlineto{\pgfqpoint{10.327822in}{1.272597in}}%
\pgfusepath{stroke}%
\end{pgfscope}%
\begin{pgfscope}%
\pgfsetbuttcap%
\pgfsetroundjoin%
\definecolor{currentfill}{rgb}{0.000000,0.000000,0.000000}%
\pgfsetfillcolor{currentfill}%
\pgfsetlinewidth{0.803000pt}%
\definecolor{currentstroke}{rgb}{0.000000,0.000000,0.000000}%
\pgfsetstrokecolor{currentstroke}%
\pgfsetdash{}{0pt}%
\pgfsys@defobject{currentmarker}{\pgfqpoint{-0.048611in}{0.000000in}}{\pgfqpoint{-0.000000in}{0.000000in}}{%
\pgfpathmoveto{\pgfqpoint{-0.000000in}{0.000000in}}%
\pgfpathlineto{\pgfqpoint{-0.048611in}{0.000000in}}%
\pgfusepath{stroke,fill}%
}%
\begin{pgfscope}%
\pgfsys@transformshift{0.640323in}{1.272597in}%
\pgfsys@useobject{currentmarker}{}%
\end{pgfscope}%
\end{pgfscope}%
\begin{pgfscope}%
\definecolor{textcolor}{rgb}{0.000000,0.000000,0.000000}%
\pgfsetstrokecolor{textcolor}%
\pgfsetfillcolor{textcolor}%
\pgftext[x=0.322221in, y=1.219836in, left, base]{\color{textcolor}\sffamily\fontsize{10.000000}{12.000000}\selectfont 0.1}%
\end{pgfscope}%
\begin{pgfscope}%
\pgfpathrectangle{\pgfqpoint{0.640323in}{0.527436in}}{\pgfqpoint{9.687500in}{3.850000in}}%
\pgfusepath{clip}%
\pgfsetrectcap%
\pgfsetroundjoin%
\pgfsetlinewidth{0.803000pt}%
\definecolor{currentstroke}{rgb}{0.690196,0.690196,0.690196}%
\pgfsetstrokecolor{currentstroke}%
\pgfsetdash{}{0pt}%
\pgfpathmoveto{\pgfqpoint{0.640323in}{1.893565in}}%
\pgfpathlineto{\pgfqpoint{10.327822in}{1.893565in}}%
\pgfusepath{stroke}%
\end{pgfscope}%
\begin{pgfscope}%
\pgfsetbuttcap%
\pgfsetroundjoin%
\definecolor{currentfill}{rgb}{0.000000,0.000000,0.000000}%
\pgfsetfillcolor{currentfill}%
\pgfsetlinewidth{0.803000pt}%
\definecolor{currentstroke}{rgb}{0.000000,0.000000,0.000000}%
\pgfsetstrokecolor{currentstroke}%
\pgfsetdash{}{0pt}%
\pgfsys@defobject{currentmarker}{\pgfqpoint{-0.048611in}{0.000000in}}{\pgfqpoint{-0.000000in}{0.000000in}}{%
\pgfpathmoveto{\pgfqpoint{-0.000000in}{0.000000in}}%
\pgfpathlineto{\pgfqpoint{-0.048611in}{0.000000in}}%
\pgfusepath{stroke,fill}%
}%
\begin{pgfscope}%
\pgfsys@transformshift{0.640323in}{1.893565in}%
\pgfsys@useobject{currentmarker}{}%
\end{pgfscope}%
\end{pgfscope}%
\begin{pgfscope}%
\definecolor{textcolor}{rgb}{0.000000,0.000000,0.000000}%
\pgfsetstrokecolor{textcolor}%
\pgfsetfillcolor{textcolor}%
\pgftext[x=0.322221in, y=1.840803in, left, base]{\color{textcolor}\sffamily\fontsize{10.000000}{12.000000}\selectfont 0.2}%
\end{pgfscope}%
\begin{pgfscope}%
\pgfpathrectangle{\pgfqpoint{0.640323in}{0.527436in}}{\pgfqpoint{9.687500in}{3.850000in}}%
\pgfusepath{clip}%
\pgfsetrectcap%
\pgfsetroundjoin%
\pgfsetlinewidth{0.803000pt}%
\definecolor{currentstroke}{rgb}{0.690196,0.690196,0.690196}%
\pgfsetstrokecolor{currentstroke}%
\pgfsetdash{}{0pt}%
\pgfpathmoveto{\pgfqpoint{0.640323in}{2.514533in}}%
\pgfpathlineto{\pgfqpoint{10.327822in}{2.514533in}}%
\pgfusepath{stroke}%
\end{pgfscope}%
\begin{pgfscope}%
\pgfsetbuttcap%
\pgfsetroundjoin%
\definecolor{currentfill}{rgb}{0.000000,0.000000,0.000000}%
\pgfsetfillcolor{currentfill}%
\pgfsetlinewidth{0.803000pt}%
\definecolor{currentstroke}{rgb}{0.000000,0.000000,0.000000}%
\pgfsetstrokecolor{currentstroke}%
\pgfsetdash{}{0pt}%
\pgfsys@defobject{currentmarker}{\pgfqpoint{-0.048611in}{0.000000in}}{\pgfqpoint{-0.000000in}{0.000000in}}{%
\pgfpathmoveto{\pgfqpoint{-0.000000in}{0.000000in}}%
\pgfpathlineto{\pgfqpoint{-0.048611in}{0.000000in}}%
\pgfusepath{stroke,fill}%
}%
\begin{pgfscope}%
\pgfsys@transformshift{0.640323in}{2.514533in}%
\pgfsys@useobject{currentmarker}{}%
\end{pgfscope}%
\end{pgfscope}%
\begin{pgfscope}%
\definecolor{textcolor}{rgb}{0.000000,0.000000,0.000000}%
\pgfsetstrokecolor{textcolor}%
\pgfsetfillcolor{textcolor}%
\pgftext[x=0.322221in, y=2.461771in, left, base]{\color{textcolor}\sffamily\fontsize{10.000000}{12.000000}\selectfont 0.3}%
\end{pgfscope}%
\begin{pgfscope}%
\pgfpathrectangle{\pgfqpoint{0.640323in}{0.527436in}}{\pgfqpoint{9.687500in}{3.850000in}}%
\pgfusepath{clip}%
\pgfsetrectcap%
\pgfsetroundjoin%
\pgfsetlinewidth{0.803000pt}%
\definecolor{currentstroke}{rgb}{0.690196,0.690196,0.690196}%
\pgfsetstrokecolor{currentstroke}%
\pgfsetdash{}{0pt}%
\pgfpathmoveto{\pgfqpoint{0.640323in}{3.135500in}}%
\pgfpathlineto{\pgfqpoint{10.327822in}{3.135500in}}%
\pgfusepath{stroke}%
\end{pgfscope}%
\begin{pgfscope}%
\pgfsetbuttcap%
\pgfsetroundjoin%
\definecolor{currentfill}{rgb}{0.000000,0.000000,0.000000}%
\pgfsetfillcolor{currentfill}%
\pgfsetlinewidth{0.803000pt}%
\definecolor{currentstroke}{rgb}{0.000000,0.000000,0.000000}%
\pgfsetstrokecolor{currentstroke}%
\pgfsetdash{}{0pt}%
\pgfsys@defobject{currentmarker}{\pgfqpoint{-0.048611in}{0.000000in}}{\pgfqpoint{-0.000000in}{0.000000in}}{%
\pgfpathmoveto{\pgfqpoint{-0.000000in}{0.000000in}}%
\pgfpathlineto{\pgfqpoint{-0.048611in}{0.000000in}}%
\pgfusepath{stroke,fill}%
}%
\begin{pgfscope}%
\pgfsys@transformshift{0.640323in}{3.135500in}%
\pgfsys@useobject{currentmarker}{}%
\end{pgfscope}%
\end{pgfscope}%
\begin{pgfscope}%
\definecolor{textcolor}{rgb}{0.000000,0.000000,0.000000}%
\pgfsetstrokecolor{textcolor}%
\pgfsetfillcolor{textcolor}%
\pgftext[x=0.322221in, y=3.082739in, left, base]{\color{textcolor}\sffamily\fontsize{10.000000}{12.000000}\selectfont 0.4}%
\end{pgfscope}%
\begin{pgfscope}%
\pgfpathrectangle{\pgfqpoint{0.640323in}{0.527436in}}{\pgfqpoint{9.687500in}{3.850000in}}%
\pgfusepath{clip}%
\pgfsetrectcap%
\pgfsetroundjoin%
\pgfsetlinewidth{0.803000pt}%
\definecolor{currentstroke}{rgb}{0.690196,0.690196,0.690196}%
\pgfsetstrokecolor{currentstroke}%
\pgfsetdash{}{0pt}%
\pgfpathmoveto{\pgfqpoint{0.640323in}{3.756468in}}%
\pgfpathlineto{\pgfqpoint{10.327822in}{3.756468in}}%
\pgfusepath{stroke}%
\end{pgfscope}%
\begin{pgfscope}%
\pgfsetbuttcap%
\pgfsetroundjoin%
\definecolor{currentfill}{rgb}{0.000000,0.000000,0.000000}%
\pgfsetfillcolor{currentfill}%
\pgfsetlinewidth{0.803000pt}%
\definecolor{currentstroke}{rgb}{0.000000,0.000000,0.000000}%
\pgfsetstrokecolor{currentstroke}%
\pgfsetdash{}{0pt}%
\pgfsys@defobject{currentmarker}{\pgfqpoint{-0.048611in}{0.000000in}}{\pgfqpoint{-0.000000in}{0.000000in}}{%
\pgfpathmoveto{\pgfqpoint{-0.000000in}{0.000000in}}%
\pgfpathlineto{\pgfqpoint{-0.048611in}{0.000000in}}%
\pgfusepath{stroke,fill}%
}%
\begin{pgfscope}%
\pgfsys@transformshift{0.640323in}{3.756468in}%
\pgfsys@useobject{currentmarker}{}%
\end{pgfscope}%
\end{pgfscope}%
\begin{pgfscope}%
\definecolor{textcolor}{rgb}{0.000000,0.000000,0.000000}%
\pgfsetstrokecolor{textcolor}%
\pgfsetfillcolor{textcolor}%
\pgftext[x=0.322221in, y=3.703707in, left, base]{\color{textcolor}\sffamily\fontsize{10.000000}{12.000000}\selectfont 0.5}%
\end{pgfscope}%
\begin{pgfscope}%
\pgfpathrectangle{\pgfqpoint{0.640323in}{0.527436in}}{\pgfqpoint{9.687500in}{3.850000in}}%
\pgfusepath{clip}%
\pgfsetrectcap%
\pgfsetroundjoin%
\pgfsetlinewidth{0.803000pt}%
\definecolor{currentstroke}{rgb}{0.690196,0.690196,0.690196}%
\pgfsetstrokecolor{currentstroke}%
\pgfsetdash{}{0pt}%
\pgfpathmoveto{\pgfqpoint{0.640323in}{4.377436in}}%
\pgfpathlineto{\pgfqpoint{10.327822in}{4.377436in}}%
\pgfusepath{stroke}%
\end{pgfscope}%
\begin{pgfscope}%
\pgfsetbuttcap%
\pgfsetroundjoin%
\definecolor{currentfill}{rgb}{0.000000,0.000000,0.000000}%
\pgfsetfillcolor{currentfill}%
\pgfsetlinewidth{0.803000pt}%
\definecolor{currentstroke}{rgb}{0.000000,0.000000,0.000000}%
\pgfsetstrokecolor{currentstroke}%
\pgfsetdash{}{0pt}%
\pgfsys@defobject{currentmarker}{\pgfqpoint{-0.048611in}{0.000000in}}{\pgfqpoint{-0.000000in}{0.000000in}}{%
\pgfpathmoveto{\pgfqpoint{-0.000000in}{0.000000in}}%
\pgfpathlineto{\pgfqpoint{-0.048611in}{0.000000in}}%
\pgfusepath{stroke,fill}%
}%
\begin{pgfscope}%
\pgfsys@transformshift{0.640323in}{4.377436in}%
\pgfsys@useobject{currentmarker}{}%
\end{pgfscope}%
\end{pgfscope}%
\begin{pgfscope}%
\definecolor{textcolor}{rgb}{0.000000,0.000000,0.000000}%
\pgfsetstrokecolor{textcolor}%
\pgfsetfillcolor{textcolor}%
\pgftext[x=0.322221in, y=4.324674in, left, base]{\color{textcolor}\sffamily\fontsize{10.000000}{12.000000}\selectfont 0.6}%
\end{pgfscope}%
\begin{pgfscope}%
\pgfpathrectangle{\pgfqpoint{0.640323in}{0.527436in}}{\pgfqpoint{9.687500in}{3.850000in}}%
\pgfusepath{clip}%
\pgfsetrectcap%
\pgfsetroundjoin%
\pgfsetlinewidth{0.803000pt}%
\definecolor{currentstroke}{rgb}{0.600000,0.600000,0.600000}%
\pgfsetstrokecolor{currentstroke}%
\pgfsetstrokeopacity{0.200000}%
\pgfsetdash{}{0pt}%
\pgfpathmoveto{\pgfqpoint{0.640323in}{0.775823in}}%
\pgfpathlineto{\pgfqpoint{10.327822in}{0.775823in}}%
\pgfusepath{stroke}%
\end{pgfscope}%
\begin{pgfscope}%
\pgfsetbuttcap%
\pgfsetroundjoin%
\definecolor{currentfill}{rgb}{0.000000,0.000000,0.000000}%
\pgfsetfillcolor{currentfill}%
\pgfsetlinewidth{0.602250pt}%
\definecolor{currentstroke}{rgb}{0.000000,0.000000,0.000000}%
\pgfsetstrokecolor{currentstroke}%
\pgfsetdash{}{0pt}%
\pgfsys@defobject{currentmarker}{\pgfqpoint{-0.027778in}{0.000000in}}{\pgfqpoint{-0.000000in}{0.000000in}}{%
\pgfpathmoveto{\pgfqpoint{-0.000000in}{0.000000in}}%
\pgfpathlineto{\pgfqpoint{-0.027778in}{0.000000in}}%
\pgfusepath{stroke,fill}%
}%
\begin{pgfscope}%
\pgfsys@transformshift{0.640323in}{0.775823in}%
\pgfsys@useobject{currentmarker}{}%
\end{pgfscope}%
\end{pgfscope}%
\begin{pgfscope}%
\pgfpathrectangle{\pgfqpoint{0.640323in}{0.527436in}}{\pgfqpoint{9.687500in}{3.850000in}}%
\pgfusepath{clip}%
\pgfsetrectcap%
\pgfsetroundjoin%
\pgfsetlinewidth{0.803000pt}%
\definecolor{currentstroke}{rgb}{0.600000,0.600000,0.600000}%
\pgfsetstrokecolor{currentstroke}%
\pgfsetstrokeopacity{0.200000}%
\pgfsetdash{}{0pt}%
\pgfpathmoveto{\pgfqpoint{0.640323in}{0.900016in}}%
\pgfpathlineto{\pgfqpoint{10.327822in}{0.900016in}}%
\pgfusepath{stroke}%
\end{pgfscope}%
\begin{pgfscope}%
\pgfsetbuttcap%
\pgfsetroundjoin%
\definecolor{currentfill}{rgb}{0.000000,0.000000,0.000000}%
\pgfsetfillcolor{currentfill}%
\pgfsetlinewidth{0.602250pt}%
\definecolor{currentstroke}{rgb}{0.000000,0.000000,0.000000}%
\pgfsetstrokecolor{currentstroke}%
\pgfsetdash{}{0pt}%
\pgfsys@defobject{currentmarker}{\pgfqpoint{-0.027778in}{0.000000in}}{\pgfqpoint{-0.000000in}{0.000000in}}{%
\pgfpathmoveto{\pgfqpoint{-0.000000in}{0.000000in}}%
\pgfpathlineto{\pgfqpoint{-0.027778in}{0.000000in}}%
\pgfusepath{stroke,fill}%
}%
\begin{pgfscope}%
\pgfsys@transformshift{0.640323in}{0.900016in}%
\pgfsys@useobject{currentmarker}{}%
\end{pgfscope}%
\end{pgfscope}%
\begin{pgfscope}%
\pgfpathrectangle{\pgfqpoint{0.640323in}{0.527436in}}{\pgfqpoint{9.687500in}{3.850000in}}%
\pgfusepath{clip}%
\pgfsetrectcap%
\pgfsetroundjoin%
\pgfsetlinewidth{0.803000pt}%
\definecolor{currentstroke}{rgb}{0.600000,0.600000,0.600000}%
\pgfsetstrokecolor{currentstroke}%
\pgfsetstrokeopacity{0.200000}%
\pgfsetdash{}{0pt}%
\pgfpathmoveto{\pgfqpoint{0.640323in}{1.024210in}}%
\pgfpathlineto{\pgfqpoint{10.327822in}{1.024210in}}%
\pgfusepath{stroke}%
\end{pgfscope}%
\begin{pgfscope}%
\pgfsetbuttcap%
\pgfsetroundjoin%
\definecolor{currentfill}{rgb}{0.000000,0.000000,0.000000}%
\pgfsetfillcolor{currentfill}%
\pgfsetlinewidth{0.602250pt}%
\definecolor{currentstroke}{rgb}{0.000000,0.000000,0.000000}%
\pgfsetstrokecolor{currentstroke}%
\pgfsetdash{}{0pt}%
\pgfsys@defobject{currentmarker}{\pgfqpoint{-0.027778in}{0.000000in}}{\pgfqpoint{-0.000000in}{0.000000in}}{%
\pgfpathmoveto{\pgfqpoint{-0.000000in}{0.000000in}}%
\pgfpathlineto{\pgfqpoint{-0.027778in}{0.000000in}}%
\pgfusepath{stroke,fill}%
}%
\begin{pgfscope}%
\pgfsys@transformshift{0.640323in}{1.024210in}%
\pgfsys@useobject{currentmarker}{}%
\end{pgfscope}%
\end{pgfscope}%
\begin{pgfscope}%
\pgfpathrectangle{\pgfqpoint{0.640323in}{0.527436in}}{\pgfqpoint{9.687500in}{3.850000in}}%
\pgfusepath{clip}%
\pgfsetrectcap%
\pgfsetroundjoin%
\pgfsetlinewidth{0.803000pt}%
\definecolor{currentstroke}{rgb}{0.600000,0.600000,0.600000}%
\pgfsetstrokecolor{currentstroke}%
\pgfsetstrokeopacity{0.200000}%
\pgfsetdash{}{0pt}%
\pgfpathmoveto{\pgfqpoint{0.640323in}{1.148404in}}%
\pgfpathlineto{\pgfqpoint{10.327822in}{1.148404in}}%
\pgfusepath{stroke}%
\end{pgfscope}%
\begin{pgfscope}%
\pgfsetbuttcap%
\pgfsetroundjoin%
\definecolor{currentfill}{rgb}{0.000000,0.000000,0.000000}%
\pgfsetfillcolor{currentfill}%
\pgfsetlinewidth{0.602250pt}%
\definecolor{currentstroke}{rgb}{0.000000,0.000000,0.000000}%
\pgfsetstrokecolor{currentstroke}%
\pgfsetdash{}{0pt}%
\pgfsys@defobject{currentmarker}{\pgfqpoint{-0.027778in}{0.000000in}}{\pgfqpoint{-0.000000in}{0.000000in}}{%
\pgfpathmoveto{\pgfqpoint{-0.000000in}{0.000000in}}%
\pgfpathlineto{\pgfqpoint{-0.027778in}{0.000000in}}%
\pgfusepath{stroke,fill}%
}%
\begin{pgfscope}%
\pgfsys@transformshift{0.640323in}{1.148404in}%
\pgfsys@useobject{currentmarker}{}%
\end{pgfscope}%
\end{pgfscope}%
\begin{pgfscope}%
\pgfpathrectangle{\pgfqpoint{0.640323in}{0.527436in}}{\pgfqpoint{9.687500in}{3.850000in}}%
\pgfusepath{clip}%
\pgfsetrectcap%
\pgfsetroundjoin%
\pgfsetlinewidth{0.803000pt}%
\definecolor{currentstroke}{rgb}{0.600000,0.600000,0.600000}%
\pgfsetstrokecolor{currentstroke}%
\pgfsetstrokeopacity{0.200000}%
\pgfsetdash{}{0pt}%
\pgfpathmoveto{\pgfqpoint{0.640323in}{1.396791in}}%
\pgfpathlineto{\pgfqpoint{10.327822in}{1.396791in}}%
\pgfusepath{stroke}%
\end{pgfscope}%
\begin{pgfscope}%
\pgfsetbuttcap%
\pgfsetroundjoin%
\definecolor{currentfill}{rgb}{0.000000,0.000000,0.000000}%
\pgfsetfillcolor{currentfill}%
\pgfsetlinewidth{0.602250pt}%
\definecolor{currentstroke}{rgb}{0.000000,0.000000,0.000000}%
\pgfsetstrokecolor{currentstroke}%
\pgfsetdash{}{0pt}%
\pgfsys@defobject{currentmarker}{\pgfqpoint{-0.027778in}{0.000000in}}{\pgfqpoint{-0.000000in}{0.000000in}}{%
\pgfpathmoveto{\pgfqpoint{-0.000000in}{0.000000in}}%
\pgfpathlineto{\pgfqpoint{-0.027778in}{0.000000in}}%
\pgfusepath{stroke,fill}%
}%
\begin{pgfscope}%
\pgfsys@transformshift{0.640323in}{1.396791in}%
\pgfsys@useobject{currentmarker}{}%
\end{pgfscope}%
\end{pgfscope}%
\begin{pgfscope}%
\pgfpathrectangle{\pgfqpoint{0.640323in}{0.527436in}}{\pgfqpoint{9.687500in}{3.850000in}}%
\pgfusepath{clip}%
\pgfsetrectcap%
\pgfsetroundjoin%
\pgfsetlinewidth{0.803000pt}%
\definecolor{currentstroke}{rgb}{0.600000,0.600000,0.600000}%
\pgfsetstrokecolor{currentstroke}%
\pgfsetstrokeopacity{0.200000}%
\pgfsetdash{}{0pt}%
\pgfpathmoveto{\pgfqpoint{0.640323in}{1.520984in}}%
\pgfpathlineto{\pgfqpoint{10.327822in}{1.520984in}}%
\pgfusepath{stroke}%
\end{pgfscope}%
\begin{pgfscope}%
\pgfsetbuttcap%
\pgfsetroundjoin%
\definecolor{currentfill}{rgb}{0.000000,0.000000,0.000000}%
\pgfsetfillcolor{currentfill}%
\pgfsetlinewidth{0.602250pt}%
\definecolor{currentstroke}{rgb}{0.000000,0.000000,0.000000}%
\pgfsetstrokecolor{currentstroke}%
\pgfsetdash{}{0pt}%
\pgfsys@defobject{currentmarker}{\pgfqpoint{-0.027778in}{0.000000in}}{\pgfqpoint{-0.000000in}{0.000000in}}{%
\pgfpathmoveto{\pgfqpoint{-0.000000in}{0.000000in}}%
\pgfpathlineto{\pgfqpoint{-0.027778in}{0.000000in}}%
\pgfusepath{stroke,fill}%
}%
\begin{pgfscope}%
\pgfsys@transformshift{0.640323in}{1.520984in}%
\pgfsys@useobject{currentmarker}{}%
\end{pgfscope}%
\end{pgfscope}%
\begin{pgfscope}%
\pgfpathrectangle{\pgfqpoint{0.640323in}{0.527436in}}{\pgfqpoint{9.687500in}{3.850000in}}%
\pgfusepath{clip}%
\pgfsetrectcap%
\pgfsetroundjoin%
\pgfsetlinewidth{0.803000pt}%
\definecolor{currentstroke}{rgb}{0.600000,0.600000,0.600000}%
\pgfsetstrokecolor{currentstroke}%
\pgfsetstrokeopacity{0.200000}%
\pgfsetdash{}{0pt}%
\pgfpathmoveto{\pgfqpoint{0.640323in}{1.645178in}}%
\pgfpathlineto{\pgfqpoint{10.327822in}{1.645178in}}%
\pgfusepath{stroke}%
\end{pgfscope}%
\begin{pgfscope}%
\pgfsetbuttcap%
\pgfsetroundjoin%
\definecolor{currentfill}{rgb}{0.000000,0.000000,0.000000}%
\pgfsetfillcolor{currentfill}%
\pgfsetlinewidth{0.602250pt}%
\definecolor{currentstroke}{rgb}{0.000000,0.000000,0.000000}%
\pgfsetstrokecolor{currentstroke}%
\pgfsetdash{}{0pt}%
\pgfsys@defobject{currentmarker}{\pgfqpoint{-0.027778in}{0.000000in}}{\pgfqpoint{-0.000000in}{0.000000in}}{%
\pgfpathmoveto{\pgfqpoint{-0.000000in}{0.000000in}}%
\pgfpathlineto{\pgfqpoint{-0.027778in}{0.000000in}}%
\pgfusepath{stroke,fill}%
}%
\begin{pgfscope}%
\pgfsys@transformshift{0.640323in}{1.645178in}%
\pgfsys@useobject{currentmarker}{}%
\end{pgfscope}%
\end{pgfscope}%
\begin{pgfscope}%
\pgfpathrectangle{\pgfqpoint{0.640323in}{0.527436in}}{\pgfqpoint{9.687500in}{3.850000in}}%
\pgfusepath{clip}%
\pgfsetrectcap%
\pgfsetroundjoin%
\pgfsetlinewidth{0.803000pt}%
\definecolor{currentstroke}{rgb}{0.600000,0.600000,0.600000}%
\pgfsetstrokecolor{currentstroke}%
\pgfsetstrokeopacity{0.200000}%
\pgfsetdash{}{0pt}%
\pgfpathmoveto{\pgfqpoint{0.640323in}{1.769371in}}%
\pgfpathlineto{\pgfqpoint{10.327822in}{1.769371in}}%
\pgfusepath{stroke}%
\end{pgfscope}%
\begin{pgfscope}%
\pgfsetbuttcap%
\pgfsetroundjoin%
\definecolor{currentfill}{rgb}{0.000000,0.000000,0.000000}%
\pgfsetfillcolor{currentfill}%
\pgfsetlinewidth{0.602250pt}%
\definecolor{currentstroke}{rgb}{0.000000,0.000000,0.000000}%
\pgfsetstrokecolor{currentstroke}%
\pgfsetdash{}{0pt}%
\pgfsys@defobject{currentmarker}{\pgfqpoint{-0.027778in}{0.000000in}}{\pgfqpoint{-0.000000in}{0.000000in}}{%
\pgfpathmoveto{\pgfqpoint{-0.000000in}{0.000000in}}%
\pgfpathlineto{\pgfqpoint{-0.027778in}{0.000000in}}%
\pgfusepath{stroke,fill}%
}%
\begin{pgfscope}%
\pgfsys@transformshift{0.640323in}{1.769371in}%
\pgfsys@useobject{currentmarker}{}%
\end{pgfscope}%
\end{pgfscope}%
\begin{pgfscope}%
\pgfpathrectangle{\pgfqpoint{0.640323in}{0.527436in}}{\pgfqpoint{9.687500in}{3.850000in}}%
\pgfusepath{clip}%
\pgfsetrectcap%
\pgfsetroundjoin%
\pgfsetlinewidth{0.803000pt}%
\definecolor{currentstroke}{rgb}{0.600000,0.600000,0.600000}%
\pgfsetstrokecolor{currentstroke}%
\pgfsetstrokeopacity{0.200000}%
\pgfsetdash{}{0pt}%
\pgfpathmoveto{\pgfqpoint{0.640323in}{2.017758in}}%
\pgfpathlineto{\pgfqpoint{10.327822in}{2.017758in}}%
\pgfusepath{stroke}%
\end{pgfscope}%
\begin{pgfscope}%
\pgfsetbuttcap%
\pgfsetroundjoin%
\definecolor{currentfill}{rgb}{0.000000,0.000000,0.000000}%
\pgfsetfillcolor{currentfill}%
\pgfsetlinewidth{0.602250pt}%
\definecolor{currentstroke}{rgb}{0.000000,0.000000,0.000000}%
\pgfsetstrokecolor{currentstroke}%
\pgfsetdash{}{0pt}%
\pgfsys@defobject{currentmarker}{\pgfqpoint{-0.027778in}{0.000000in}}{\pgfqpoint{-0.000000in}{0.000000in}}{%
\pgfpathmoveto{\pgfqpoint{-0.000000in}{0.000000in}}%
\pgfpathlineto{\pgfqpoint{-0.027778in}{0.000000in}}%
\pgfusepath{stroke,fill}%
}%
\begin{pgfscope}%
\pgfsys@transformshift{0.640323in}{2.017758in}%
\pgfsys@useobject{currentmarker}{}%
\end{pgfscope}%
\end{pgfscope}%
\begin{pgfscope}%
\pgfpathrectangle{\pgfqpoint{0.640323in}{0.527436in}}{\pgfqpoint{9.687500in}{3.850000in}}%
\pgfusepath{clip}%
\pgfsetrectcap%
\pgfsetroundjoin%
\pgfsetlinewidth{0.803000pt}%
\definecolor{currentstroke}{rgb}{0.600000,0.600000,0.600000}%
\pgfsetstrokecolor{currentstroke}%
\pgfsetstrokeopacity{0.200000}%
\pgfsetdash{}{0pt}%
\pgfpathmoveto{\pgfqpoint{0.640323in}{2.141952in}}%
\pgfpathlineto{\pgfqpoint{10.327822in}{2.141952in}}%
\pgfusepath{stroke}%
\end{pgfscope}%
\begin{pgfscope}%
\pgfsetbuttcap%
\pgfsetroundjoin%
\definecolor{currentfill}{rgb}{0.000000,0.000000,0.000000}%
\pgfsetfillcolor{currentfill}%
\pgfsetlinewidth{0.602250pt}%
\definecolor{currentstroke}{rgb}{0.000000,0.000000,0.000000}%
\pgfsetstrokecolor{currentstroke}%
\pgfsetdash{}{0pt}%
\pgfsys@defobject{currentmarker}{\pgfqpoint{-0.027778in}{0.000000in}}{\pgfqpoint{-0.000000in}{0.000000in}}{%
\pgfpathmoveto{\pgfqpoint{-0.000000in}{0.000000in}}%
\pgfpathlineto{\pgfqpoint{-0.027778in}{0.000000in}}%
\pgfusepath{stroke,fill}%
}%
\begin{pgfscope}%
\pgfsys@transformshift{0.640323in}{2.141952in}%
\pgfsys@useobject{currentmarker}{}%
\end{pgfscope}%
\end{pgfscope}%
\begin{pgfscope}%
\pgfpathrectangle{\pgfqpoint{0.640323in}{0.527436in}}{\pgfqpoint{9.687500in}{3.850000in}}%
\pgfusepath{clip}%
\pgfsetrectcap%
\pgfsetroundjoin%
\pgfsetlinewidth{0.803000pt}%
\definecolor{currentstroke}{rgb}{0.600000,0.600000,0.600000}%
\pgfsetstrokecolor{currentstroke}%
\pgfsetstrokeopacity{0.200000}%
\pgfsetdash{}{0pt}%
\pgfpathmoveto{\pgfqpoint{0.640323in}{2.266146in}}%
\pgfpathlineto{\pgfqpoint{10.327822in}{2.266146in}}%
\pgfusepath{stroke}%
\end{pgfscope}%
\begin{pgfscope}%
\pgfsetbuttcap%
\pgfsetroundjoin%
\definecolor{currentfill}{rgb}{0.000000,0.000000,0.000000}%
\pgfsetfillcolor{currentfill}%
\pgfsetlinewidth{0.602250pt}%
\definecolor{currentstroke}{rgb}{0.000000,0.000000,0.000000}%
\pgfsetstrokecolor{currentstroke}%
\pgfsetdash{}{0pt}%
\pgfsys@defobject{currentmarker}{\pgfqpoint{-0.027778in}{0.000000in}}{\pgfqpoint{-0.000000in}{0.000000in}}{%
\pgfpathmoveto{\pgfqpoint{-0.000000in}{0.000000in}}%
\pgfpathlineto{\pgfqpoint{-0.027778in}{0.000000in}}%
\pgfusepath{stroke,fill}%
}%
\begin{pgfscope}%
\pgfsys@transformshift{0.640323in}{2.266146in}%
\pgfsys@useobject{currentmarker}{}%
\end{pgfscope}%
\end{pgfscope}%
\begin{pgfscope}%
\pgfpathrectangle{\pgfqpoint{0.640323in}{0.527436in}}{\pgfqpoint{9.687500in}{3.850000in}}%
\pgfusepath{clip}%
\pgfsetrectcap%
\pgfsetroundjoin%
\pgfsetlinewidth{0.803000pt}%
\definecolor{currentstroke}{rgb}{0.600000,0.600000,0.600000}%
\pgfsetstrokecolor{currentstroke}%
\pgfsetstrokeopacity{0.200000}%
\pgfsetdash{}{0pt}%
\pgfpathmoveto{\pgfqpoint{0.640323in}{2.390339in}}%
\pgfpathlineto{\pgfqpoint{10.327822in}{2.390339in}}%
\pgfusepath{stroke}%
\end{pgfscope}%
\begin{pgfscope}%
\pgfsetbuttcap%
\pgfsetroundjoin%
\definecolor{currentfill}{rgb}{0.000000,0.000000,0.000000}%
\pgfsetfillcolor{currentfill}%
\pgfsetlinewidth{0.602250pt}%
\definecolor{currentstroke}{rgb}{0.000000,0.000000,0.000000}%
\pgfsetstrokecolor{currentstroke}%
\pgfsetdash{}{0pt}%
\pgfsys@defobject{currentmarker}{\pgfqpoint{-0.027778in}{0.000000in}}{\pgfqpoint{-0.000000in}{0.000000in}}{%
\pgfpathmoveto{\pgfqpoint{-0.000000in}{0.000000in}}%
\pgfpathlineto{\pgfqpoint{-0.027778in}{0.000000in}}%
\pgfusepath{stroke,fill}%
}%
\begin{pgfscope}%
\pgfsys@transformshift{0.640323in}{2.390339in}%
\pgfsys@useobject{currentmarker}{}%
\end{pgfscope}%
\end{pgfscope}%
\begin{pgfscope}%
\pgfpathrectangle{\pgfqpoint{0.640323in}{0.527436in}}{\pgfqpoint{9.687500in}{3.850000in}}%
\pgfusepath{clip}%
\pgfsetrectcap%
\pgfsetroundjoin%
\pgfsetlinewidth{0.803000pt}%
\definecolor{currentstroke}{rgb}{0.600000,0.600000,0.600000}%
\pgfsetstrokecolor{currentstroke}%
\pgfsetstrokeopacity{0.200000}%
\pgfsetdash{}{0pt}%
\pgfpathmoveto{\pgfqpoint{0.640323in}{2.638726in}}%
\pgfpathlineto{\pgfqpoint{10.327822in}{2.638726in}}%
\pgfusepath{stroke}%
\end{pgfscope}%
\begin{pgfscope}%
\pgfsetbuttcap%
\pgfsetroundjoin%
\definecolor{currentfill}{rgb}{0.000000,0.000000,0.000000}%
\pgfsetfillcolor{currentfill}%
\pgfsetlinewidth{0.602250pt}%
\definecolor{currentstroke}{rgb}{0.000000,0.000000,0.000000}%
\pgfsetstrokecolor{currentstroke}%
\pgfsetdash{}{0pt}%
\pgfsys@defobject{currentmarker}{\pgfqpoint{-0.027778in}{0.000000in}}{\pgfqpoint{-0.000000in}{0.000000in}}{%
\pgfpathmoveto{\pgfqpoint{-0.000000in}{0.000000in}}%
\pgfpathlineto{\pgfqpoint{-0.027778in}{0.000000in}}%
\pgfusepath{stroke,fill}%
}%
\begin{pgfscope}%
\pgfsys@transformshift{0.640323in}{2.638726in}%
\pgfsys@useobject{currentmarker}{}%
\end{pgfscope}%
\end{pgfscope}%
\begin{pgfscope}%
\pgfpathrectangle{\pgfqpoint{0.640323in}{0.527436in}}{\pgfqpoint{9.687500in}{3.850000in}}%
\pgfusepath{clip}%
\pgfsetrectcap%
\pgfsetroundjoin%
\pgfsetlinewidth{0.803000pt}%
\definecolor{currentstroke}{rgb}{0.600000,0.600000,0.600000}%
\pgfsetstrokecolor{currentstroke}%
\pgfsetstrokeopacity{0.200000}%
\pgfsetdash{}{0pt}%
\pgfpathmoveto{\pgfqpoint{0.640323in}{2.762920in}}%
\pgfpathlineto{\pgfqpoint{10.327822in}{2.762920in}}%
\pgfusepath{stroke}%
\end{pgfscope}%
\begin{pgfscope}%
\pgfsetbuttcap%
\pgfsetroundjoin%
\definecolor{currentfill}{rgb}{0.000000,0.000000,0.000000}%
\pgfsetfillcolor{currentfill}%
\pgfsetlinewidth{0.602250pt}%
\definecolor{currentstroke}{rgb}{0.000000,0.000000,0.000000}%
\pgfsetstrokecolor{currentstroke}%
\pgfsetdash{}{0pt}%
\pgfsys@defobject{currentmarker}{\pgfqpoint{-0.027778in}{0.000000in}}{\pgfqpoint{-0.000000in}{0.000000in}}{%
\pgfpathmoveto{\pgfqpoint{-0.000000in}{0.000000in}}%
\pgfpathlineto{\pgfqpoint{-0.027778in}{0.000000in}}%
\pgfusepath{stroke,fill}%
}%
\begin{pgfscope}%
\pgfsys@transformshift{0.640323in}{2.762920in}%
\pgfsys@useobject{currentmarker}{}%
\end{pgfscope}%
\end{pgfscope}%
\begin{pgfscope}%
\pgfpathrectangle{\pgfqpoint{0.640323in}{0.527436in}}{\pgfqpoint{9.687500in}{3.850000in}}%
\pgfusepath{clip}%
\pgfsetrectcap%
\pgfsetroundjoin%
\pgfsetlinewidth{0.803000pt}%
\definecolor{currentstroke}{rgb}{0.600000,0.600000,0.600000}%
\pgfsetstrokecolor{currentstroke}%
\pgfsetstrokeopacity{0.200000}%
\pgfsetdash{}{0pt}%
\pgfpathmoveto{\pgfqpoint{0.640323in}{2.887113in}}%
\pgfpathlineto{\pgfqpoint{10.327822in}{2.887113in}}%
\pgfusepath{stroke}%
\end{pgfscope}%
\begin{pgfscope}%
\pgfsetbuttcap%
\pgfsetroundjoin%
\definecolor{currentfill}{rgb}{0.000000,0.000000,0.000000}%
\pgfsetfillcolor{currentfill}%
\pgfsetlinewidth{0.602250pt}%
\definecolor{currentstroke}{rgb}{0.000000,0.000000,0.000000}%
\pgfsetstrokecolor{currentstroke}%
\pgfsetdash{}{0pt}%
\pgfsys@defobject{currentmarker}{\pgfqpoint{-0.027778in}{0.000000in}}{\pgfqpoint{-0.000000in}{0.000000in}}{%
\pgfpathmoveto{\pgfqpoint{-0.000000in}{0.000000in}}%
\pgfpathlineto{\pgfqpoint{-0.027778in}{0.000000in}}%
\pgfusepath{stroke,fill}%
}%
\begin{pgfscope}%
\pgfsys@transformshift{0.640323in}{2.887113in}%
\pgfsys@useobject{currentmarker}{}%
\end{pgfscope}%
\end{pgfscope}%
\begin{pgfscope}%
\pgfpathrectangle{\pgfqpoint{0.640323in}{0.527436in}}{\pgfqpoint{9.687500in}{3.850000in}}%
\pgfusepath{clip}%
\pgfsetrectcap%
\pgfsetroundjoin%
\pgfsetlinewidth{0.803000pt}%
\definecolor{currentstroke}{rgb}{0.600000,0.600000,0.600000}%
\pgfsetstrokecolor{currentstroke}%
\pgfsetstrokeopacity{0.200000}%
\pgfsetdash{}{0pt}%
\pgfpathmoveto{\pgfqpoint{0.640323in}{3.011307in}}%
\pgfpathlineto{\pgfqpoint{10.327822in}{3.011307in}}%
\pgfusepath{stroke}%
\end{pgfscope}%
\begin{pgfscope}%
\pgfsetbuttcap%
\pgfsetroundjoin%
\definecolor{currentfill}{rgb}{0.000000,0.000000,0.000000}%
\pgfsetfillcolor{currentfill}%
\pgfsetlinewidth{0.602250pt}%
\definecolor{currentstroke}{rgb}{0.000000,0.000000,0.000000}%
\pgfsetstrokecolor{currentstroke}%
\pgfsetdash{}{0pt}%
\pgfsys@defobject{currentmarker}{\pgfqpoint{-0.027778in}{0.000000in}}{\pgfqpoint{-0.000000in}{0.000000in}}{%
\pgfpathmoveto{\pgfqpoint{-0.000000in}{0.000000in}}%
\pgfpathlineto{\pgfqpoint{-0.027778in}{0.000000in}}%
\pgfusepath{stroke,fill}%
}%
\begin{pgfscope}%
\pgfsys@transformshift{0.640323in}{3.011307in}%
\pgfsys@useobject{currentmarker}{}%
\end{pgfscope}%
\end{pgfscope}%
\begin{pgfscope}%
\pgfpathrectangle{\pgfqpoint{0.640323in}{0.527436in}}{\pgfqpoint{9.687500in}{3.850000in}}%
\pgfusepath{clip}%
\pgfsetrectcap%
\pgfsetroundjoin%
\pgfsetlinewidth{0.803000pt}%
\definecolor{currentstroke}{rgb}{0.600000,0.600000,0.600000}%
\pgfsetstrokecolor{currentstroke}%
\pgfsetstrokeopacity{0.200000}%
\pgfsetdash{}{0pt}%
\pgfpathmoveto{\pgfqpoint{0.640323in}{3.259694in}}%
\pgfpathlineto{\pgfqpoint{10.327822in}{3.259694in}}%
\pgfusepath{stroke}%
\end{pgfscope}%
\begin{pgfscope}%
\pgfsetbuttcap%
\pgfsetroundjoin%
\definecolor{currentfill}{rgb}{0.000000,0.000000,0.000000}%
\pgfsetfillcolor{currentfill}%
\pgfsetlinewidth{0.602250pt}%
\definecolor{currentstroke}{rgb}{0.000000,0.000000,0.000000}%
\pgfsetstrokecolor{currentstroke}%
\pgfsetdash{}{0pt}%
\pgfsys@defobject{currentmarker}{\pgfqpoint{-0.027778in}{0.000000in}}{\pgfqpoint{-0.000000in}{0.000000in}}{%
\pgfpathmoveto{\pgfqpoint{-0.000000in}{0.000000in}}%
\pgfpathlineto{\pgfqpoint{-0.027778in}{0.000000in}}%
\pgfusepath{stroke,fill}%
}%
\begin{pgfscope}%
\pgfsys@transformshift{0.640323in}{3.259694in}%
\pgfsys@useobject{currentmarker}{}%
\end{pgfscope}%
\end{pgfscope}%
\begin{pgfscope}%
\pgfpathrectangle{\pgfqpoint{0.640323in}{0.527436in}}{\pgfqpoint{9.687500in}{3.850000in}}%
\pgfusepath{clip}%
\pgfsetrectcap%
\pgfsetroundjoin%
\pgfsetlinewidth{0.803000pt}%
\definecolor{currentstroke}{rgb}{0.600000,0.600000,0.600000}%
\pgfsetstrokecolor{currentstroke}%
\pgfsetstrokeopacity{0.200000}%
\pgfsetdash{}{0pt}%
\pgfpathmoveto{\pgfqpoint{0.640323in}{3.383887in}}%
\pgfpathlineto{\pgfqpoint{10.327822in}{3.383887in}}%
\pgfusepath{stroke}%
\end{pgfscope}%
\begin{pgfscope}%
\pgfsetbuttcap%
\pgfsetroundjoin%
\definecolor{currentfill}{rgb}{0.000000,0.000000,0.000000}%
\pgfsetfillcolor{currentfill}%
\pgfsetlinewidth{0.602250pt}%
\definecolor{currentstroke}{rgb}{0.000000,0.000000,0.000000}%
\pgfsetstrokecolor{currentstroke}%
\pgfsetdash{}{0pt}%
\pgfsys@defobject{currentmarker}{\pgfqpoint{-0.027778in}{0.000000in}}{\pgfqpoint{-0.000000in}{0.000000in}}{%
\pgfpathmoveto{\pgfqpoint{-0.000000in}{0.000000in}}%
\pgfpathlineto{\pgfqpoint{-0.027778in}{0.000000in}}%
\pgfusepath{stroke,fill}%
}%
\begin{pgfscope}%
\pgfsys@transformshift{0.640323in}{3.383887in}%
\pgfsys@useobject{currentmarker}{}%
\end{pgfscope}%
\end{pgfscope}%
\begin{pgfscope}%
\pgfpathrectangle{\pgfqpoint{0.640323in}{0.527436in}}{\pgfqpoint{9.687500in}{3.850000in}}%
\pgfusepath{clip}%
\pgfsetrectcap%
\pgfsetroundjoin%
\pgfsetlinewidth{0.803000pt}%
\definecolor{currentstroke}{rgb}{0.600000,0.600000,0.600000}%
\pgfsetstrokecolor{currentstroke}%
\pgfsetstrokeopacity{0.200000}%
\pgfsetdash{}{0pt}%
\pgfpathmoveto{\pgfqpoint{0.640323in}{3.508081in}}%
\pgfpathlineto{\pgfqpoint{10.327822in}{3.508081in}}%
\pgfusepath{stroke}%
\end{pgfscope}%
\begin{pgfscope}%
\pgfsetbuttcap%
\pgfsetroundjoin%
\definecolor{currentfill}{rgb}{0.000000,0.000000,0.000000}%
\pgfsetfillcolor{currentfill}%
\pgfsetlinewidth{0.602250pt}%
\definecolor{currentstroke}{rgb}{0.000000,0.000000,0.000000}%
\pgfsetstrokecolor{currentstroke}%
\pgfsetdash{}{0pt}%
\pgfsys@defobject{currentmarker}{\pgfqpoint{-0.027778in}{0.000000in}}{\pgfqpoint{-0.000000in}{0.000000in}}{%
\pgfpathmoveto{\pgfqpoint{-0.000000in}{0.000000in}}%
\pgfpathlineto{\pgfqpoint{-0.027778in}{0.000000in}}%
\pgfusepath{stroke,fill}%
}%
\begin{pgfscope}%
\pgfsys@transformshift{0.640323in}{3.508081in}%
\pgfsys@useobject{currentmarker}{}%
\end{pgfscope}%
\end{pgfscope}%
\begin{pgfscope}%
\pgfpathrectangle{\pgfqpoint{0.640323in}{0.527436in}}{\pgfqpoint{9.687500in}{3.850000in}}%
\pgfusepath{clip}%
\pgfsetrectcap%
\pgfsetroundjoin%
\pgfsetlinewidth{0.803000pt}%
\definecolor{currentstroke}{rgb}{0.600000,0.600000,0.600000}%
\pgfsetstrokecolor{currentstroke}%
\pgfsetstrokeopacity{0.200000}%
\pgfsetdash{}{0pt}%
\pgfpathmoveto{\pgfqpoint{0.640323in}{3.632275in}}%
\pgfpathlineto{\pgfqpoint{10.327822in}{3.632275in}}%
\pgfusepath{stroke}%
\end{pgfscope}%
\begin{pgfscope}%
\pgfsetbuttcap%
\pgfsetroundjoin%
\definecolor{currentfill}{rgb}{0.000000,0.000000,0.000000}%
\pgfsetfillcolor{currentfill}%
\pgfsetlinewidth{0.602250pt}%
\definecolor{currentstroke}{rgb}{0.000000,0.000000,0.000000}%
\pgfsetstrokecolor{currentstroke}%
\pgfsetdash{}{0pt}%
\pgfsys@defobject{currentmarker}{\pgfqpoint{-0.027778in}{0.000000in}}{\pgfqpoint{-0.000000in}{0.000000in}}{%
\pgfpathmoveto{\pgfqpoint{-0.000000in}{0.000000in}}%
\pgfpathlineto{\pgfqpoint{-0.027778in}{0.000000in}}%
\pgfusepath{stroke,fill}%
}%
\begin{pgfscope}%
\pgfsys@transformshift{0.640323in}{3.632275in}%
\pgfsys@useobject{currentmarker}{}%
\end{pgfscope}%
\end{pgfscope}%
\begin{pgfscope}%
\pgfpathrectangle{\pgfqpoint{0.640323in}{0.527436in}}{\pgfqpoint{9.687500in}{3.850000in}}%
\pgfusepath{clip}%
\pgfsetrectcap%
\pgfsetroundjoin%
\pgfsetlinewidth{0.803000pt}%
\definecolor{currentstroke}{rgb}{0.600000,0.600000,0.600000}%
\pgfsetstrokecolor{currentstroke}%
\pgfsetstrokeopacity{0.200000}%
\pgfsetdash{}{0pt}%
\pgfpathmoveto{\pgfqpoint{0.640323in}{3.880662in}}%
\pgfpathlineto{\pgfqpoint{10.327822in}{3.880662in}}%
\pgfusepath{stroke}%
\end{pgfscope}%
\begin{pgfscope}%
\pgfsetbuttcap%
\pgfsetroundjoin%
\definecolor{currentfill}{rgb}{0.000000,0.000000,0.000000}%
\pgfsetfillcolor{currentfill}%
\pgfsetlinewidth{0.602250pt}%
\definecolor{currentstroke}{rgb}{0.000000,0.000000,0.000000}%
\pgfsetstrokecolor{currentstroke}%
\pgfsetdash{}{0pt}%
\pgfsys@defobject{currentmarker}{\pgfqpoint{-0.027778in}{0.000000in}}{\pgfqpoint{-0.000000in}{0.000000in}}{%
\pgfpathmoveto{\pgfqpoint{-0.000000in}{0.000000in}}%
\pgfpathlineto{\pgfqpoint{-0.027778in}{0.000000in}}%
\pgfusepath{stroke,fill}%
}%
\begin{pgfscope}%
\pgfsys@transformshift{0.640323in}{3.880662in}%
\pgfsys@useobject{currentmarker}{}%
\end{pgfscope}%
\end{pgfscope}%
\begin{pgfscope}%
\pgfpathrectangle{\pgfqpoint{0.640323in}{0.527436in}}{\pgfqpoint{9.687500in}{3.850000in}}%
\pgfusepath{clip}%
\pgfsetrectcap%
\pgfsetroundjoin%
\pgfsetlinewidth{0.803000pt}%
\definecolor{currentstroke}{rgb}{0.600000,0.600000,0.600000}%
\pgfsetstrokecolor{currentstroke}%
\pgfsetstrokeopacity{0.200000}%
\pgfsetdash{}{0pt}%
\pgfpathmoveto{\pgfqpoint{0.640323in}{4.004855in}}%
\pgfpathlineto{\pgfqpoint{10.327822in}{4.004855in}}%
\pgfusepath{stroke}%
\end{pgfscope}%
\begin{pgfscope}%
\pgfsetbuttcap%
\pgfsetroundjoin%
\definecolor{currentfill}{rgb}{0.000000,0.000000,0.000000}%
\pgfsetfillcolor{currentfill}%
\pgfsetlinewidth{0.602250pt}%
\definecolor{currentstroke}{rgb}{0.000000,0.000000,0.000000}%
\pgfsetstrokecolor{currentstroke}%
\pgfsetdash{}{0pt}%
\pgfsys@defobject{currentmarker}{\pgfqpoint{-0.027778in}{0.000000in}}{\pgfqpoint{-0.000000in}{0.000000in}}{%
\pgfpathmoveto{\pgfqpoint{-0.000000in}{0.000000in}}%
\pgfpathlineto{\pgfqpoint{-0.027778in}{0.000000in}}%
\pgfusepath{stroke,fill}%
}%
\begin{pgfscope}%
\pgfsys@transformshift{0.640323in}{4.004855in}%
\pgfsys@useobject{currentmarker}{}%
\end{pgfscope}%
\end{pgfscope}%
\begin{pgfscope}%
\pgfpathrectangle{\pgfqpoint{0.640323in}{0.527436in}}{\pgfqpoint{9.687500in}{3.850000in}}%
\pgfusepath{clip}%
\pgfsetrectcap%
\pgfsetroundjoin%
\pgfsetlinewidth{0.803000pt}%
\definecolor{currentstroke}{rgb}{0.600000,0.600000,0.600000}%
\pgfsetstrokecolor{currentstroke}%
\pgfsetstrokeopacity{0.200000}%
\pgfsetdash{}{0pt}%
\pgfpathmoveto{\pgfqpoint{0.640323in}{4.129049in}}%
\pgfpathlineto{\pgfqpoint{10.327822in}{4.129049in}}%
\pgfusepath{stroke}%
\end{pgfscope}%
\begin{pgfscope}%
\pgfsetbuttcap%
\pgfsetroundjoin%
\definecolor{currentfill}{rgb}{0.000000,0.000000,0.000000}%
\pgfsetfillcolor{currentfill}%
\pgfsetlinewidth{0.602250pt}%
\definecolor{currentstroke}{rgb}{0.000000,0.000000,0.000000}%
\pgfsetstrokecolor{currentstroke}%
\pgfsetdash{}{0pt}%
\pgfsys@defobject{currentmarker}{\pgfqpoint{-0.027778in}{0.000000in}}{\pgfqpoint{-0.000000in}{0.000000in}}{%
\pgfpathmoveto{\pgfqpoint{-0.000000in}{0.000000in}}%
\pgfpathlineto{\pgfqpoint{-0.027778in}{0.000000in}}%
\pgfusepath{stroke,fill}%
}%
\begin{pgfscope}%
\pgfsys@transformshift{0.640323in}{4.129049in}%
\pgfsys@useobject{currentmarker}{}%
\end{pgfscope}%
\end{pgfscope}%
\begin{pgfscope}%
\pgfpathrectangle{\pgfqpoint{0.640323in}{0.527436in}}{\pgfqpoint{9.687500in}{3.850000in}}%
\pgfusepath{clip}%
\pgfsetrectcap%
\pgfsetroundjoin%
\pgfsetlinewidth{0.803000pt}%
\definecolor{currentstroke}{rgb}{0.600000,0.600000,0.600000}%
\pgfsetstrokecolor{currentstroke}%
\pgfsetstrokeopacity{0.200000}%
\pgfsetdash{}{0pt}%
\pgfpathmoveto{\pgfqpoint{0.640323in}{4.253242in}}%
\pgfpathlineto{\pgfqpoint{10.327822in}{4.253242in}}%
\pgfusepath{stroke}%
\end{pgfscope}%
\begin{pgfscope}%
\pgfsetbuttcap%
\pgfsetroundjoin%
\definecolor{currentfill}{rgb}{0.000000,0.000000,0.000000}%
\pgfsetfillcolor{currentfill}%
\pgfsetlinewidth{0.602250pt}%
\definecolor{currentstroke}{rgb}{0.000000,0.000000,0.000000}%
\pgfsetstrokecolor{currentstroke}%
\pgfsetdash{}{0pt}%
\pgfsys@defobject{currentmarker}{\pgfqpoint{-0.027778in}{0.000000in}}{\pgfqpoint{-0.000000in}{0.000000in}}{%
\pgfpathmoveto{\pgfqpoint{-0.000000in}{0.000000in}}%
\pgfpathlineto{\pgfqpoint{-0.027778in}{0.000000in}}%
\pgfusepath{stroke,fill}%
}%
\begin{pgfscope}%
\pgfsys@transformshift{0.640323in}{4.253242in}%
\pgfsys@useobject{currentmarker}{}%
\end{pgfscope}%
\end{pgfscope}%
\begin{pgfscope}%
\definecolor{textcolor}{rgb}{0.000000,0.000000,0.000000}%
\pgfsetstrokecolor{textcolor}%
\pgfsetfillcolor{textcolor}%
\pgftext[x=0.266665in,y=2.452436in,,bottom,rotate=90.000000]{\color{textcolor}\sffamily\fontsize{10.000000}{12.000000}\selectfont avg. infection rate \(\displaystyle \overline{\langle I\rangle}\)}%
\end{pgfscope}%
\begin{pgfscope}%
\pgfpathrectangle{\pgfqpoint{0.640323in}{0.527436in}}{\pgfqpoint{9.687500in}{3.850000in}}%
\pgfusepath{clip}%
\pgfsetbuttcap%
\pgfsetroundjoin%
\pgfsetlinewidth{1.003750pt}%
\definecolor{currentstroke}{rgb}{0.000000,0.000000,1.000000}%
\pgfsetstrokecolor{currentstroke}%
\pgfsetstrokeopacity{0.500000}%
\pgfsetdash{{3.700000pt}{1.600000pt}}{0.000000pt}%
\pgfpathmoveto{\pgfqpoint{1.080663in}{0.659222in}}%
\pgfpathlineto{\pgfqpoint{1.260394in}{0.661817in}}%
\pgfpathlineto{\pgfqpoint{1.440125in}{0.663952in}}%
\pgfpathlineto{\pgfqpoint{1.619856in}{0.663127in}}%
\pgfpathlineto{\pgfqpoint{1.799587in}{0.661987in}}%
\pgfpathlineto{\pgfqpoint{1.979318in}{0.670089in}}%
\pgfpathlineto{\pgfqpoint{2.159049in}{0.682993in}}%
\pgfpathlineto{\pgfqpoint{2.338780in}{0.685467in}}%
\pgfpathlineto{\pgfqpoint{2.518511in}{0.688014in}}%
\pgfpathlineto{\pgfqpoint{2.698242in}{0.696358in}}%
\pgfpathlineto{\pgfqpoint{2.877973in}{1.634093in}}%
\pgfpathlineto{\pgfqpoint{3.057704in}{1.884660in}}%
\pgfpathlineto{\pgfqpoint{3.237435in}{2.119684in}}%
\pgfpathlineto{\pgfqpoint{3.417166in}{2.331589in}}%
\pgfpathlineto{\pgfqpoint{3.596897in}{2.506404in}}%
\pgfpathlineto{\pgfqpoint{3.776628in}{2.606318in}}%
\pgfpathlineto{\pgfqpoint{3.956359in}{2.789677in}}%
\pgfpathlineto{\pgfqpoint{4.136090in}{2.829531in}}%
\pgfpathlineto{\pgfqpoint{4.315821in}{2.973906in}}%
\pgfpathlineto{\pgfqpoint{4.495552in}{3.032165in}}%
\pgfpathlineto{\pgfqpoint{4.675283in}{3.123541in}}%
\pgfpathlineto{\pgfqpoint{4.855014in}{3.199659in}}%
\pgfpathlineto{\pgfqpoint{5.034745in}{3.232868in}}%
\pgfpathlineto{\pgfqpoint{5.214476in}{3.299765in}}%
\pgfpathlineto{\pgfqpoint{5.394207in}{3.359750in}}%
\pgfpathlineto{\pgfqpoint{5.573938in}{3.406056in}}%
\pgfpathlineto{\pgfqpoint{5.753669in}{3.426262in}}%
\pgfpathlineto{\pgfqpoint{5.933400in}{3.504659in}}%
\pgfpathlineto{\pgfqpoint{6.113131in}{3.554945in}}%
\pgfpathlineto{\pgfqpoint{6.292862in}{3.574618in}}%
\pgfpathlineto{\pgfqpoint{6.472593in}{3.602828in}}%
\pgfpathlineto{\pgfqpoint{6.652324in}{3.653133in}}%
\pgfpathlineto{\pgfqpoint{6.832055in}{3.687118in}}%
\pgfpathlineto{\pgfqpoint{7.011786in}{3.694371in}}%
\pgfpathlineto{\pgfqpoint{7.191517in}{3.736821in}}%
\pgfpathlineto{\pgfqpoint{7.371248in}{3.756468in}}%
\pgfpathlineto{\pgfqpoint{7.550979in}{3.788367in}}%
\pgfpathlineto{\pgfqpoint{7.730710in}{3.808064in}}%
\pgfpathlineto{\pgfqpoint{7.910441in}{3.838380in}}%
\pgfpathlineto{\pgfqpoint{8.090172in}{3.855873in}}%
\pgfpathlineto{\pgfqpoint{8.269903in}{3.859095in}}%
\pgfpathlineto{\pgfqpoint{8.449634in}{3.893541in}}%
\pgfpathlineto{\pgfqpoint{8.629365in}{3.910474in}}%
\pgfpathlineto{\pgfqpoint{8.809096in}{3.921049in}}%
\pgfpathlineto{\pgfqpoint{8.988827in}{3.950229in}}%
\pgfpathlineto{\pgfqpoint{9.168558in}{3.959860in}}%
\pgfpathlineto{\pgfqpoint{9.348289in}{3.972646in}}%
\pgfpathlineto{\pgfqpoint{9.528020in}{3.993380in}}%
\pgfpathlineto{\pgfqpoint{9.707751in}{4.026030in}}%
\pgfpathlineto{\pgfqpoint{9.887482in}{4.022416in}}%
\pgfusepath{stroke}%
\end{pgfscope}%
\begin{pgfscope}%
\pgfpathrectangle{\pgfqpoint{0.640323in}{0.527436in}}{\pgfqpoint{9.687500in}{3.850000in}}%
\pgfusepath{clip}%
\pgfsetbuttcap%
\pgfsetroundjoin%
\pgfsetlinewidth{1.003750pt}%
\definecolor{currentstroke}{rgb}{0.980392,0.164706,0.333333}%
\pgfsetstrokecolor{currentstroke}%
\pgfsetstrokeopacity{0.500000}%
\pgfsetdash{{3.700000pt}{1.600000pt}}{0.000000pt}%
\pgfpathmoveto{\pgfqpoint{1.080663in}{0.659834in}}%
\pgfpathlineto{\pgfqpoint{1.260394in}{0.660901in}}%
\pgfpathlineto{\pgfqpoint{1.440125in}{0.661696in}}%
\pgfpathlineto{\pgfqpoint{1.619856in}{0.664352in}}%
\pgfpathlineto{\pgfqpoint{1.799587in}{0.665437in}}%
\pgfpathlineto{\pgfqpoint{1.979318in}{0.674358in}}%
\pgfpathlineto{\pgfqpoint{2.159049in}{0.677256in}}%
\pgfpathlineto{\pgfqpoint{2.338780in}{0.715364in}}%
\pgfpathlineto{\pgfqpoint{2.518511in}{0.753295in}}%
\pgfpathlineto{\pgfqpoint{2.698242in}{1.197292in}}%
\pgfpathlineto{\pgfqpoint{2.877973in}{1.702804in}}%
\pgfpathlineto{\pgfqpoint{3.057704in}{2.074397in}}%
\pgfpathlineto{\pgfqpoint{3.237435in}{2.254459in}}%
\pgfpathlineto{\pgfqpoint{3.417166in}{2.431416in}}%
\pgfpathlineto{\pgfqpoint{3.596897in}{2.568973in}}%
\pgfpathlineto{\pgfqpoint{3.776628in}{2.719744in}}%
\pgfpathlineto{\pgfqpoint{3.956359in}{2.842565in}}%
\pgfpathlineto{\pgfqpoint{4.136090in}{2.925054in}}%
\pgfpathlineto{\pgfqpoint{4.315821in}{3.026800in}}%
\pgfpathlineto{\pgfqpoint{4.495552in}{3.104346in}}%
\pgfpathlineto{\pgfqpoint{4.675283in}{3.190003in}}%
\pgfpathlineto{\pgfqpoint{4.855014in}{3.227410in}}%
\pgfpathlineto{\pgfqpoint{5.034745in}{3.282067in}}%
\pgfpathlineto{\pgfqpoint{5.214476in}{3.352802in}}%
\pgfpathlineto{\pgfqpoint{5.394207in}{3.396859in}}%
\pgfpathlineto{\pgfqpoint{5.573938in}{3.438439in}}%
\pgfpathlineto{\pgfqpoint{5.753669in}{3.483894in}}%
\pgfpathlineto{\pgfqpoint{5.933400in}{3.524742in}}%
\pgfpathlineto{\pgfqpoint{6.113131in}{3.560211in}}%
\pgfpathlineto{\pgfqpoint{6.292862in}{3.603170in}}%
\pgfpathlineto{\pgfqpoint{6.472593in}{3.639447in}}%
\pgfpathlineto{\pgfqpoint{6.652324in}{3.656505in}}%
\pgfpathlineto{\pgfqpoint{6.832055in}{3.687621in}}%
\pgfpathlineto{\pgfqpoint{7.011786in}{3.716354in}}%
\pgfpathlineto{\pgfqpoint{7.191517in}{3.759815in}}%
\pgfpathlineto{\pgfqpoint{7.371248in}{3.784455in}}%
\pgfpathlineto{\pgfqpoint{7.550979in}{3.798539in}}%
\pgfpathlineto{\pgfqpoint{7.730710in}{3.822340in}}%
\pgfpathlineto{\pgfqpoint{7.910441in}{3.849092in}}%
\pgfpathlineto{\pgfqpoint{8.090172in}{3.867466in}}%
\pgfpathlineto{\pgfqpoint{8.269903in}{3.888337in}}%
\pgfpathlineto{\pgfqpoint{8.449634in}{3.906736in}}%
\pgfpathlineto{\pgfqpoint{8.629365in}{3.925744in}}%
\pgfpathlineto{\pgfqpoint{8.809096in}{3.941231in}}%
\pgfpathlineto{\pgfqpoint{8.988827in}{3.956767in}}%
\pgfpathlineto{\pgfqpoint{9.168558in}{3.979843in}}%
\pgfpathlineto{\pgfqpoint{9.348289in}{3.983891in}}%
\pgfpathlineto{\pgfqpoint{9.528020in}{4.009370in}}%
\pgfpathlineto{\pgfqpoint{9.707751in}{4.016846in}}%
\pgfpathlineto{\pgfqpoint{9.887482in}{4.037574in}}%
\pgfusepath{stroke}%
\end{pgfscope}%
\begin{pgfscope}%
\pgfpathrectangle{\pgfqpoint{0.640323in}{0.527436in}}{\pgfqpoint{9.687500in}{3.850000in}}%
\pgfusepath{clip}%
\pgfsetbuttcap%
\pgfsetroundjoin%
\pgfsetlinewidth{1.003750pt}%
\definecolor{currentstroke}{rgb}{0.239216,0.478431,0.992157}%
\pgfsetstrokecolor{currentstroke}%
\pgfsetstrokeopacity{0.500000}%
\pgfsetdash{{3.700000pt}{1.600000pt}}{0.000000pt}%
\pgfpathmoveto{\pgfqpoint{1.080663in}{0.659775in}}%
\pgfpathlineto{\pgfqpoint{1.260394in}{0.660083in}}%
\pgfpathlineto{\pgfqpoint{1.440125in}{0.662478in}}%
\pgfpathlineto{\pgfqpoint{1.619856in}{0.664197in}}%
\pgfpathlineto{\pgfqpoint{1.799587in}{0.666835in}}%
\pgfpathlineto{\pgfqpoint{1.979318in}{0.673436in}}%
\pgfpathlineto{\pgfqpoint{2.159049in}{0.684130in}}%
\pgfpathlineto{\pgfqpoint{2.338780in}{0.699575in}}%
\pgfpathlineto{\pgfqpoint{2.518511in}{0.817273in}}%
\pgfpathlineto{\pgfqpoint{2.698242in}{1.404975in}}%
\pgfpathlineto{\pgfqpoint{2.877973in}{1.839789in}}%
\pgfpathlineto{\pgfqpoint{3.057704in}{2.084165in}}%
\pgfpathlineto{\pgfqpoint{3.237435in}{2.304534in}}%
\pgfpathlineto{\pgfqpoint{3.417166in}{2.473679in}}%
\pgfpathlineto{\pgfqpoint{3.596897in}{2.616676in}}%
\pgfpathlineto{\pgfqpoint{3.776628in}{2.753413in}}%
\pgfpathlineto{\pgfqpoint{3.956359in}{2.866901in}}%
\pgfpathlineto{\pgfqpoint{4.136090in}{2.958003in}}%
\pgfpathlineto{\pgfqpoint{4.315821in}{3.051757in}}%
\pgfpathlineto{\pgfqpoint{4.495552in}{3.122261in}}%
\pgfpathlineto{\pgfqpoint{4.675283in}{3.187326in}}%
\pgfpathlineto{\pgfqpoint{4.855014in}{3.246293in}}%
\pgfpathlineto{\pgfqpoint{5.034745in}{3.315960in}}%
\pgfpathlineto{\pgfqpoint{5.214476in}{3.371338in}}%
\pgfpathlineto{\pgfqpoint{5.394207in}{3.411545in}}%
\pgfpathlineto{\pgfqpoint{5.573938in}{3.464570in}}%
\pgfpathlineto{\pgfqpoint{5.753669in}{3.499810in}}%
\pgfpathlineto{\pgfqpoint{5.933400in}{3.543718in}}%
\pgfpathlineto{\pgfqpoint{6.113131in}{3.573301in}}%
\pgfpathlineto{\pgfqpoint{6.292862in}{3.620905in}}%
\pgfpathlineto{\pgfqpoint{6.472593in}{3.649842in}}%
\pgfpathlineto{\pgfqpoint{6.652324in}{3.671718in}}%
\pgfpathlineto{\pgfqpoint{6.832055in}{3.711504in}}%
\pgfpathlineto{\pgfqpoint{7.011786in}{3.733418in}}%
\pgfpathlineto{\pgfqpoint{7.191517in}{3.764392in}}%
\pgfpathlineto{\pgfqpoint{7.371248in}{3.789851in}}%
\pgfpathlineto{\pgfqpoint{7.550979in}{3.809971in}}%
\pgfpathlineto{\pgfqpoint{7.730710in}{3.835418in}}%
\pgfpathlineto{\pgfqpoint{7.910441in}{3.850439in}}%
\pgfpathlineto{\pgfqpoint{8.090172in}{3.880643in}}%
\pgfpathlineto{\pgfqpoint{8.269903in}{3.895863in}}%
\pgfpathlineto{\pgfqpoint{8.449634in}{3.914684in}}%
\pgfpathlineto{\pgfqpoint{8.629365in}{3.929495in}}%
\pgfpathlineto{\pgfqpoint{8.809096in}{3.948229in}}%
\pgfpathlineto{\pgfqpoint{8.988827in}{3.967957in}}%
\pgfpathlineto{\pgfqpoint{9.168558in}{3.984314in}}%
\pgfpathlineto{\pgfqpoint{9.348289in}{4.002353in}}%
\pgfpathlineto{\pgfqpoint{9.528020in}{4.011971in}}%
\pgfpathlineto{\pgfqpoint{9.707751in}{4.021472in}}%
\pgfpathlineto{\pgfqpoint{9.887482in}{4.043262in}}%
\pgfusepath{stroke}%
\end{pgfscope}%
\begin{pgfscope}%
\pgfpathrectangle{\pgfqpoint{0.640323in}{0.527436in}}{\pgfqpoint{9.687500in}{3.850000in}}%
\pgfusepath{clip}%
\pgfsetbuttcap%
\pgfsetroundjoin%
\pgfsetlinewidth{1.003750pt}%
\definecolor{currentstroke}{rgb}{0.000000,0.000000,0.000000}%
\pgfsetstrokecolor{currentstroke}%
\pgfsetstrokeopacity{0.500000}%
\pgfsetdash{{3.700000pt}{1.600000pt}}{0.000000pt}%
\pgfpathmoveto{\pgfqpoint{1.080663in}{0.659536in}}%
\pgfpathlineto{\pgfqpoint{1.260394in}{0.660526in}}%
\pgfpathlineto{\pgfqpoint{1.440125in}{0.662244in}}%
\pgfpathlineto{\pgfqpoint{1.619856in}{0.664285in}}%
\pgfpathlineto{\pgfqpoint{1.799587in}{0.666523in}}%
\pgfpathlineto{\pgfqpoint{1.979318in}{0.671589in}}%
\pgfpathlineto{\pgfqpoint{2.159049in}{0.680716in}}%
\pgfpathlineto{\pgfqpoint{2.338780in}{0.701224in}}%
\pgfpathlineto{\pgfqpoint{2.518511in}{0.828515in}}%
\pgfpathlineto{\pgfqpoint{2.698242in}{1.446021in}}%
\pgfpathlineto{\pgfqpoint{2.877973in}{1.840882in}}%
\pgfpathlineto{\pgfqpoint{3.057704in}{2.127626in}}%
\pgfpathlineto{\pgfqpoint{3.237435in}{2.334806in}}%
\pgfpathlineto{\pgfqpoint{3.417166in}{2.503784in}}%
\pgfpathlineto{\pgfqpoint{3.596897in}{2.651624in}}%
\pgfpathlineto{\pgfqpoint{3.776628in}{2.772470in}}%
\pgfpathlineto{\pgfqpoint{3.956359in}{2.882444in}}%
\pgfpathlineto{\pgfqpoint{4.136090in}{2.973459in}}%
\pgfpathlineto{\pgfqpoint{4.315821in}{3.058333in}}%
\pgfpathlineto{\pgfqpoint{4.495552in}{3.132234in}}%
\pgfpathlineto{\pgfqpoint{4.675283in}{3.203453in}}%
\pgfpathlineto{\pgfqpoint{4.855014in}{3.263159in}}%
\pgfpathlineto{\pgfqpoint{5.034745in}{3.315773in}}%
\pgfpathlineto{\pgfqpoint{5.214476in}{3.373374in}}%
\pgfpathlineto{\pgfqpoint{5.394207in}{3.421375in}}%
\pgfpathlineto{\pgfqpoint{5.573938in}{3.465700in}}%
\pgfpathlineto{\pgfqpoint{5.753669in}{3.509031in}}%
\pgfpathlineto{\pgfqpoint{5.933400in}{3.549357in}}%
\pgfpathlineto{\pgfqpoint{6.113131in}{3.586230in}}%
\pgfpathlineto{\pgfqpoint{6.292862in}{3.621470in}}%
\pgfpathlineto{\pgfqpoint{6.472593in}{3.654182in}}%
\pgfpathlineto{\pgfqpoint{6.652324in}{3.681362in}}%
\pgfpathlineto{\pgfqpoint{6.832055in}{3.712559in}}%
\pgfpathlineto{\pgfqpoint{7.011786in}{3.739069in}}%
\pgfpathlineto{\pgfqpoint{7.191517in}{3.766838in}}%
\pgfpathlineto{\pgfqpoint{7.371248in}{3.793819in}}%
\pgfpathlineto{\pgfqpoint{7.550979in}{3.815982in}}%
\pgfpathlineto{\pgfqpoint{7.730710in}{3.836660in}}%
\pgfpathlineto{\pgfqpoint{7.910441in}{3.860412in}}%
\pgfpathlineto{\pgfqpoint{8.090172in}{3.878345in}}%
\pgfpathlineto{\pgfqpoint{8.269903in}{3.898428in}}%
\pgfpathlineto{\pgfqpoint{8.449634in}{3.917584in}}%
\pgfpathlineto{\pgfqpoint{8.629365in}{3.934835in}}%
\pgfpathlineto{\pgfqpoint{8.809096in}{3.952526in}}%
\pgfpathlineto{\pgfqpoint{8.988827in}{3.967429in}}%
\pgfpathlineto{\pgfqpoint{9.168558in}{3.982910in}}%
\pgfpathlineto{\pgfqpoint{9.348289in}{4.001378in}}%
\pgfpathlineto{\pgfqpoint{9.528020in}{4.014300in}}%
\pgfpathlineto{\pgfqpoint{9.707751in}{4.028868in}}%
\pgfpathlineto{\pgfqpoint{9.887482in}{4.042424in}}%
\pgfusepath{stroke}%
\end{pgfscope}%
\begin{pgfscope}%
\pgfsetrectcap%
\pgfsetmiterjoin%
\pgfsetlinewidth{0.803000pt}%
\definecolor{currentstroke}{rgb}{0.000000,0.000000,0.000000}%
\pgfsetstrokecolor{currentstroke}%
\pgfsetdash{}{0pt}%
\pgfpathmoveto{\pgfqpoint{0.640323in}{0.527436in}}%
\pgfpathlineto{\pgfqpoint{0.640323in}{4.377436in}}%
\pgfusepath{stroke}%
\end{pgfscope}%
\begin{pgfscope}%
\pgfsetrectcap%
\pgfsetmiterjoin%
\pgfsetlinewidth{0.803000pt}%
\definecolor{currentstroke}{rgb}{0.000000,0.000000,0.000000}%
\pgfsetstrokecolor{currentstroke}%
\pgfsetdash{}{0pt}%
\pgfpathmoveto{\pgfqpoint{10.327822in}{0.527436in}}%
\pgfpathlineto{\pgfqpoint{10.327822in}{4.377436in}}%
\pgfusepath{stroke}%
\end{pgfscope}%
\begin{pgfscope}%
\pgfsetrectcap%
\pgfsetmiterjoin%
\pgfsetlinewidth{0.803000pt}%
\definecolor{currentstroke}{rgb}{0.000000,0.000000,0.000000}%
\pgfsetstrokecolor{currentstroke}%
\pgfsetdash{}{0pt}%
\pgfpathmoveto{\pgfqpoint{0.640322in}{0.527436in}}%
\pgfpathlineto{\pgfqpoint{10.327823in}{0.527436in}}%
\pgfusepath{stroke}%
\end{pgfscope}%
\begin{pgfscope}%
\pgfsetrectcap%
\pgfsetmiterjoin%
\pgfsetlinewidth{0.803000pt}%
\definecolor{currentstroke}{rgb}{0.000000,0.000000,0.000000}%
\pgfsetstrokecolor{currentstroke}%
\pgfsetdash{}{0pt}%
\pgfpathmoveto{\pgfqpoint{0.640322in}{4.377436in}}%
\pgfpathlineto{\pgfqpoint{10.327823in}{4.377436in}}%
\pgfusepath{stroke}%
\end{pgfscope}%
\begin{pgfscope}%
\definecolor{textcolor}{rgb}{0.000000,0.000000,0.000000}%
\pgfsetstrokecolor{textcolor}%
\pgfsetfillcolor{textcolor}%
\pgftext[x=5.484072in,y=4.460769in,,base]{\color{textcolor}\sffamily\fontsize{12.000000}{14.400000}\selectfont \(\displaystyle \overline{\langle I\rangle}\) over \(\displaystyle p_1\) for \(\displaystyle T=1000\) with \(\displaystyle p_2=0.3\), \(\displaystyle p_3=0.6\)}%
\end{pgfscope}%
\begin{pgfscope}%
\pgfsetbuttcap%
\pgfsetmiterjoin%
\definecolor{currentfill}{rgb}{1.000000,1.000000,1.000000}%
\pgfsetfillcolor{currentfill}%
\pgfsetfillopacity{0.800000}%
\pgfsetlinewidth{1.003750pt}%
\definecolor{currentstroke}{rgb}{0.800000,0.800000,0.800000}%
\pgfsetstrokecolor{currentstroke}%
\pgfsetstrokeopacity{0.800000}%
\pgfsetdash{}{0pt}%
\pgfpathmoveto{\pgfqpoint{0.737545in}{3.450896in}}%
\pgfpathlineto{\pgfqpoint{1.670029in}{3.450896in}}%
\pgfpathquadraticcurveto{\pgfqpoint{1.697806in}{3.450896in}}{\pgfqpoint{1.697806in}{3.478674in}}%
\pgfpathlineto{\pgfqpoint{1.697806in}{4.280214in}}%
\pgfpathquadraticcurveto{\pgfqpoint{1.697806in}{4.307991in}}{\pgfqpoint{1.670029in}{4.307991in}}%
\pgfpathlineto{\pgfqpoint{0.737545in}{4.307991in}}%
\pgfpathquadraticcurveto{\pgfqpoint{0.709767in}{4.307991in}}{\pgfqpoint{0.709767in}{4.280214in}}%
\pgfpathlineto{\pgfqpoint{0.709767in}{3.478674in}}%
\pgfpathquadraticcurveto{\pgfqpoint{0.709767in}{3.450896in}}{\pgfqpoint{0.737545in}{3.450896in}}%
\pgfpathlineto{\pgfqpoint{0.737545in}{3.450896in}}%
\pgfpathclose%
\pgfusepath{stroke,fill}%
\end{pgfscope}%
\begin{pgfscope}%
\pgfsetbuttcap%
\pgfsetroundjoin%
\definecolor{currentfill}{rgb}{0.000000,0.000000,1.000000}%
\pgfsetfillcolor{currentfill}%
\pgfsetfillopacity{0.500000}%
\pgfsetlinewidth{1.003750pt}%
\definecolor{currentstroke}{rgb}{0.000000,0.000000,1.000000}%
\pgfsetstrokecolor{currentstroke}%
\pgfsetstrokeopacity{0.500000}%
\pgfsetdash{{3.700000pt}{1.600000pt}}{0.000000pt}%
\pgfpathmoveto{\pgfqpoint{0.904211in}{4.161411in}}%
\pgfpathcurveto{\pgfqpoint{0.910035in}{4.161411in}}{\pgfqpoint{0.915621in}{4.163725in}}{\pgfqpoint{0.919740in}{4.167843in}}%
\pgfpathcurveto{\pgfqpoint{0.923858in}{4.171961in}}{\pgfqpoint{0.926172in}{4.177547in}}{\pgfqpoint{0.926172in}{4.183371in}}%
\pgfpathcurveto{\pgfqpoint{0.926172in}{4.189195in}}{\pgfqpoint{0.923858in}{4.194781in}}{\pgfqpoint{0.919740in}{4.198899in}}%
\pgfpathcurveto{\pgfqpoint{0.915621in}{4.203018in}}{\pgfqpoint{0.910035in}{4.205331in}}{\pgfqpoint{0.904211in}{4.205331in}}%
\pgfpathcurveto{\pgfqpoint{0.898387in}{4.205331in}}{\pgfqpoint{0.892801in}{4.203018in}}{\pgfqpoint{0.888683in}{4.198899in}}%
\pgfpathcurveto{\pgfqpoint{0.884565in}{4.194781in}}{\pgfqpoint{0.882251in}{4.189195in}}{\pgfqpoint{0.882251in}{4.183371in}}%
\pgfpathcurveto{\pgfqpoint{0.882251in}{4.177547in}}{\pgfqpoint{0.884565in}{4.171961in}}{\pgfqpoint{0.888683in}{4.167843in}}%
\pgfpathcurveto{\pgfqpoint{0.892801in}{4.163725in}}{\pgfqpoint{0.898387in}{4.161411in}}{\pgfqpoint{0.904211in}{4.161411in}}%
\pgfpathlineto{\pgfqpoint{0.904211in}{4.161411in}}%
\pgfpathclose%
\pgfusepath{stroke,fill}%
\end{pgfscope}%
\begin{pgfscope}%
\definecolor{textcolor}{rgb}{0.000000,0.000000,0.000000}%
\pgfsetstrokecolor{textcolor}%
\pgfsetfillcolor{textcolor}%
\pgftext[x=1.154211in,y=4.146913in,left,base]{\color{textcolor}\sffamily\fontsize{10.000000}{12.000000}\selectfont \(\displaystyle L=16\)}%
\end{pgfscope}%
\begin{pgfscope}%
\pgfsetbuttcap%
\pgfsetroundjoin%
\definecolor{currentfill}{rgb}{0.980392,0.164706,0.333333}%
\pgfsetfillcolor{currentfill}%
\pgfsetfillopacity{0.500000}%
\pgfsetlinewidth{1.003750pt}%
\definecolor{currentstroke}{rgb}{0.980392,0.164706,0.333333}%
\pgfsetstrokecolor{currentstroke}%
\pgfsetstrokeopacity{0.500000}%
\pgfsetdash{{3.700000pt}{1.600000pt}}{0.000000pt}%
\pgfpathmoveto{\pgfqpoint{0.904211in}{3.957554in}}%
\pgfpathcurveto{\pgfqpoint{0.910035in}{3.957554in}}{\pgfqpoint{0.915621in}{3.959867in}}{\pgfqpoint{0.919740in}{3.963986in}}%
\pgfpathcurveto{\pgfqpoint{0.923858in}{3.968104in}}{\pgfqpoint{0.926172in}{3.973690in}}{\pgfqpoint{0.926172in}{3.979514in}}%
\pgfpathcurveto{\pgfqpoint{0.926172in}{3.985338in}}{\pgfqpoint{0.923858in}{3.990924in}}{\pgfqpoint{0.919740in}{3.995042in}}%
\pgfpathcurveto{\pgfqpoint{0.915621in}{3.999160in}}{\pgfqpoint{0.910035in}{4.001474in}}{\pgfqpoint{0.904211in}{4.001474in}}%
\pgfpathcurveto{\pgfqpoint{0.898387in}{4.001474in}}{\pgfqpoint{0.892801in}{3.999160in}}{\pgfqpoint{0.888683in}{3.995042in}}%
\pgfpathcurveto{\pgfqpoint{0.884565in}{3.990924in}}{\pgfqpoint{0.882251in}{3.985338in}}{\pgfqpoint{0.882251in}{3.979514in}}%
\pgfpathcurveto{\pgfqpoint{0.882251in}{3.973690in}}{\pgfqpoint{0.884565in}{3.968104in}}{\pgfqpoint{0.888683in}{3.963986in}}%
\pgfpathcurveto{\pgfqpoint{0.892801in}{3.959867in}}{\pgfqpoint{0.898387in}{3.957554in}}{\pgfqpoint{0.904211in}{3.957554in}}%
\pgfpathlineto{\pgfqpoint{0.904211in}{3.957554in}}%
\pgfpathclose%
\pgfusepath{stroke,fill}%
\end{pgfscope}%
\begin{pgfscope}%
\definecolor{textcolor}{rgb}{0.000000,0.000000,0.000000}%
\pgfsetstrokecolor{textcolor}%
\pgfsetfillcolor{textcolor}%
\pgftext[x=1.154211in,y=3.943056in,left,base]{\color{textcolor}\sffamily\fontsize{10.000000}{12.000000}\selectfont \(\displaystyle L=32\)}%
\end{pgfscope}%
\begin{pgfscope}%
\pgfsetbuttcap%
\pgfsetroundjoin%
\definecolor{currentfill}{rgb}{0.239216,0.478431,0.992157}%
\pgfsetfillcolor{currentfill}%
\pgfsetfillopacity{0.500000}%
\pgfsetlinewidth{1.003750pt}%
\definecolor{currentstroke}{rgb}{0.239216,0.478431,0.992157}%
\pgfsetstrokecolor{currentstroke}%
\pgfsetstrokeopacity{0.500000}%
\pgfsetdash{{3.700000pt}{1.600000pt}}{0.000000pt}%
\pgfpathmoveto{\pgfqpoint{0.904211in}{3.753696in}}%
\pgfpathcurveto{\pgfqpoint{0.910035in}{3.753696in}}{\pgfqpoint{0.915621in}{3.756010in}}{\pgfqpoint{0.919740in}{3.760128in}}%
\pgfpathcurveto{\pgfqpoint{0.923858in}{3.764247in}}{\pgfqpoint{0.926172in}{3.769833in}}{\pgfqpoint{0.926172in}{3.775657in}}%
\pgfpathcurveto{\pgfqpoint{0.926172in}{3.781481in}}{\pgfqpoint{0.923858in}{3.787067in}}{\pgfqpoint{0.919740in}{3.791185in}}%
\pgfpathcurveto{\pgfqpoint{0.915621in}{3.795303in}}{\pgfqpoint{0.910035in}{3.797617in}}{\pgfqpoint{0.904211in}{3.797617in}}%
\pgfpathcurveto{\pgfqpoint{0.898387in}{3.797617in}}{\pgfqpoint{0.892801in}{3.795303in}}{\pgfqpoint{0.888683in}{3.791185in}}%
\pgfpathcurveto{\pgfqpoint{0.884565in}{3.787067in}}{\pgfqpoint{0.882251in}{3.781481in}}{\pgfqpoint{0.882251in}{3.775657in}}%
\pgfpathcurveto{\pgfqpoint{0.882251in}{3.769833in}}{\pgfqpoint{0.884565in}{3.764247in}}{\pgfqpoint{0.888683in}{3.760128in}}%
\pgfpathcurveto{\pgfqpoint{0.892801in}{3.756010in}}{\pgfqpoint{0.898387in}{3.753696in}}{\pgfqpoint{0.904211in}{3.753696in}}%
\pgfpathlineto{\pgfqpoint{0.904211in}{3.753696in}}%
\pgfpathclose%
\pgfusepath{stroke,fill}%
\end{pgfscope}%
\begin{pgfscope}%
\definecolor{textcolor}{rgb}{0.000000,0.000000,0.000000}%
\pgfsetstrokecolor{textcolor}%
\pgfsetfillcolor{textcolor}%
\pgftext[x=1.154211in,y=3.739198in,left,base]{\color{textcolor}\sffamily\fontsize{10.000000}{12.000000}\selectfont \(\displaystyle L=64\)}%
\end{pgfscope}%
\begin{pgfscope}%
\pgfsetbuttcap%
\pgfsetroundjoin%
\definecolor{currentfill}{rgb}{0.000000,0.000000,0.000000}%
\pgfsetfillcolor{currentfill}%
\pgfsetfillopacity{0.500000}%
\pgfsetlinewidth{1.003750pt}%
\definecolor{currentstroke}{rgb}{0.000000,0.000000,0.000000}%
\pgfsetstrokecolor{currentstroke}%
\pgfsetstrokeopacity{0.500000}%
\pgfsetdash{{3.700000pt}{1.600000pt}}{0.000000pt}%
\pgfpathmoveto{\pgfqpoint{0.904211in}{3.549839in}}%
\pgfpathcurveto{\pgfqpoint{0.910035in}{3.549839in}}{\pgfqpoint{0.915621in}{3.552153in}}{\pgfqpoint{0.919740in}{3.556271in}}%
\pgfpathcurveto{\pgfqpoint{0.923858in}{3.560389in}}{\pgfqpoint{0.926172in}{3.565976in}}{\pgfqpoint{0.926172in}{3.571799in}}%
\pgfpathcurveto{\pgfqpoint{0.926172in}{3.577623in}}{\pgfqpoint{0.923858in}{3.583210in}}{\pgfqpoint{0.919740in}{3.587328in}}%
\pgfpathcurveto{\pgfqpoint{0.915621in}{3.591446in}}{\pgfqpoint{0.910035in}{3.593760in}}{\pgfqpoint{0.904211in}{3.593760in}}%
\pgfpathcurveto{\pgfqpoint{0.898387in}{3.593760in}}{\pgfqpoint{0.892801in}{3.591446in}}{\pgfqpoint{0.888683in}{3.587328in}}%
\pgfpathcurveto{\pgfqpoint{0.884565in}{3.583210in}}{\pgfqpoint{0.882251in}{3.577623in}}{\pgfqpoint{0.882251in}{3.571799in}}%
\pgfpathcurveto{\pgfqpoint{0.882251in}{3.565976in}}{\pgfqpoint{0.884565in}{3.560389in}}{\pgfqpoint{0.888683in}{3.556271in}}%
\pgfpathcurveto{\pgfqpoint{0.892801in}{3.552153in}}{\pgfqpoint{0.898387in}{3.549839in}}{\pgfqpoint{0.904211in}{3.549839in}}%
\pgfpathlineto{\pgfqpoint{0.904211in}{3.549839in}}%
\pgfpathclose%
\pgfusepath{stroke,fill}%
\end{pgfscope}%
\begin{pgfscope}%
\definecolor{textcolor}{rgb}{0.000000,0.000000,0.000000}%
\pgfsetstrokecolor{textcolor}%
\pgfsetfillcolor{textcolor}%
\pgftext[x=1.154211in,y=3.535341in,left,base]{\color{textcolor}\sffamily\fontsize{10.000000}{12.000000}\selectfont \(\displaystyle L=128\)}%
\end{pgfscope}%
\end{pgfpicture}%
\makeatother%
\endgroup%
}
    \caption{Graphic}\label{fig:Res_Dis_Avg_Inf_over_p1}
\end{figure}


\begin{figure}[ht]
    \centering
    \resizebox{\textwidth}{!}{\input{images/soi_average_infected_over_p1_L96.pgf}}
    \caption{Graphic}\label{fig:Res_Dis_Avg_Inf_over_p1_L96}
\end{figure}

\subsection{Vaccinated People without Participation in the Spread}

\begin{figure}[ht]
    \centering
    \resizebox{\textwidth}{!}{%% Creator: Matplotlib, PGF backend
%%
%% To include the figure in your LaTeX document, write
%%   \input{<filename>.pgf}
%%
%% Make sure the required packages are loaded in your preamble
%%   \usepackage{pgf}
%%
%% Also ensure that all the required font packages are loaded; for instance,
%% the lmodern package is sometimes necessary when using math font.
%%   \usepackage{lmodern}
%%
%% Figures using additional raster images can only be included by \input if
%% they are in the same directory as the main LaTeX file. For loading figures
%% from other directories you can use the `import` package
%%   \usepackage{import}
%%
%% and then include the figures with
%%   \import{<path to file>}{<filename>.pgf}
%%
%% Matplotlib used the following preamble
%%   
%%   \usepackage{fontspec}
%%   \setmainfont{DejaVuSerif.ttf}[Path=\detokenize{/home/carlo/.local/lib/python3.10/site-packages/matplotlib/mpl-data/fonts/ttf/}]
%%   \setsansfont{DejaVuSans.ttf}[Path=\detokenize{/home/carlo/.local/lib/python3.10/site-packages/matplotlib/mpl-data/fonts/ttf/}]
%%   \setmonofont{DejaVuSansMono.ttf}[Path=\detokenize{/home/carlo/.local/lib/python3.10/site-packages/matplotlib/mpl-data/fonts/ttf/}]
%%   \makeatletter\@ifpackageloaded{underscore}{}{\usepackage[strings]{underscore}}\makeatother
%%
\begingroup%
\makeatletter%
\begin{pgfpicture}%
\pgfpathrectangle{\pgfpointorigin}{\pgfqpoint{10.516188in}{5.092713in}}%
\pgfusepath{use as bounding box, clip}%
\begin{pgfscope}%
\pgfsetbuttcap%
\pgfsetmiterjoin%
\definecolor{currentfill}{rgb}{1.000000,1.000000,1.000000}%
\pgfsetfillcolor{currentfill}%
\pgfsetlinewidth{0.000000pt}%
\definecolor{currentstroke}{rgb}{1.000000,1.000000,1.000000}%
\pgfsetstrokecolor{currentstroke}%
\pgfsetdash{}{0pt}%
\pgfpathmoveto{\pgfqpoint{0.000000in}{0.000000in}}%
\pgfpathlineto{\pgfqpoint{10.516188in}{0.000000in}}%
\pgfpathlineto{\pgfqpoint{10.516188in}{5.092713in}}%
\pgfpathlineto{\pgfqpoint{0.000000in}{5.092713in}}%
\pgfpathlineto{\pgfqpoint{0.000000in}{0.000000in}}%
\pgfpathclose%
\pgfusepath{fill}%
\end{pgfscope}%
\begin{pgfscope}%
\pgfsetbuttcap%
\pgfsetmiterjoin%
\definecolor{currentfill}{rgb}{1.000000,1.000000,1.000000}%
\pgfsetfillcolor{currentfill}%
\pgfsetlinewidth{0.000000pt}%
\definecolor{currentstroke}{rgb}{0.000000,0.000000,0.000000}%
\pgfsetstrokecolor{currentstroke}%
\pgfsetstrokeopacity{0.000000}%
\pgfsetdash{}{0pt}%
\pgfpathmoveto{\pgfqpoint{0.728688in}{0.521603in}}%
\pgfpathlineto{\pgfqpoint{10.416188in}{0.521603in}}%
\pgfpathlineto{\pgfqpoint{10.416188in}{4.756603in}}%
\pgfpathlineto{\pgfqpoint{0.728688in}{4.756603in}}%
\pgfpathlineto{\pgfqpoint{0.728688in}{0.521603in}}%
\pgfpathclose%
\pgfusepath{fill}%
\end{pgfscope}%
\begin{pgfscope}%
\pgfpathrectangle{\pgfqpoint{0.728688in}{0.521603in}}{\pgfqpoint{9.687500in}{4.235000in}}%
\pgfusepath{clip}%
\pgfsetbuttcap%
\pgfsetroundjoin%
\definecolor{currentfill}{rgb}{0.000000,0.000000,1.000000}%
\pgfsetfillcolor{currentfill}%
\pgfsetfillopacity{0.500000}%
\pgfsetlinewidth{1.003750pt}%
\definecolor{currentstroke}{rgb}{0.000000,0.000000,1.000000}%
\pgfsetstrokecolor{currentstroke}%
\pgfsetstrokeopacity{0.500000}%
\pgfsetdash{}{0pt}%
\pgfsys@defobject{currentmarker}{\pgfqpoint{-0.021960in}{-0.021960in}}{\pgfqpoint{0.021960in}{0.021960in}}{%
\pgfpathmoveto{\pgfqpoint{0.000000in}{-0.021960in}}%
\pgfpathcurveto{\pgfqpoint{0.005824in}{-0.021960in}}{\pgfqpoint{0.011410in}{-0.019646in}}{\pgfqpoint{0.015528in}{-0.015528in}}%
\pgfpathcurveto{\pgfqpoint{0.019646in}{-0.011410in}}{\pgfqpoint{0.021960in}{-0.005824in}}{\pgfqpoint{0.021960in}{0.000000in}}%
\pgfpathcurveto{\pgfqpoint{0.021960in}{0.005824in}}{\pgfqpoint{0.019646in}{0.011410in}}{\pgfqpoint{0.015528in}{0.015528in}}%
\pgfpathcurveto{\pgfqpoint{0.011410in}{0.019646in}}{\pgfqpoint{0.005824in}{0.021960in}}{\pgfqpoint{0.000000in}{0.021960in}}%
\pgfpathcurveto{\pgfqpoint{-0.005824in}{0.021960in}}{\pgfqpoint{-0.011410in}{0.019646in}}{\pgfqpoint{-0.015528in}{0.015528in}}%
\pgfpathcurveto{\pgfqpoint{-0.019646in}{0.011410in}}{\pgfqpoint{-0.021960in}{0.005824in}}{\pgfqpoint{-0.021960in}{0.000000in}}%
\pgfpathcurveto{\pgfqpoint{-0.021960in}{-0.005824in}}{\pgfqpoint{-0.019646in}{-0.011410in}}{\pgfqpoint{-0.015528in}{-0.015528in}}%
\pgfpathcurveto{\pgfqpoint{-0.011410in}{-0.019646in}}{\pgfqpoint{-0.005824in}{-0.021960in}}{\pgfqpoint{0.000000in}{-0.021960in}}%
\pgfpathlineto{\pgfqpoint{0.000000in}{-0.021960in}}%
\pgfpathclose%
\pgfusepath{stroke,fill}%
}%
\begin{pgfscope}%
\pgfsys@transformshift{1.169029in}{4.046571in}%
\pgfsys@useobject{currentmarker}{}%
\end{pgfscope}%
\begin{pgfscope}%
\pgfsys@transformshift{1.348760in}{4.207524in}%
\pgfsys@useobject{currentmarker}{}%
\end{pgfscope}%
\begin{pgfscope}%
\pgfsys@transformshift{1.528491in}{3.612032in}%
\pgfsys@useobject{currentmarker}{}%
\end{pgfscope}%
\begin{pgfscope}%
\pgfsys@transformshift{1.708222in}{3.675538in}%
\pgfsys@useobject{currentmarker}{}%
\end{pgfscope}%
\begin{pgfscope}%
\pgfsys@transformshift{1.887953in}{3.140982in}%
\pgfsys@useobject{currentmarker}{}%
\end{pgfscope}%
\begin{pgfscope}%
\pgfsys@transformshift{2.067684in}{3.043000in}%
\pgfsys@useobject{currentmarker}{}%
\end{pgfscope}%
\begin{pgfscope}%
\pgfsys@transformshift{2.247415in}{0.862755in}%
\pgfsys@useobject{currentmarker}{}%
\end{pgfscope}%
\begin{pgfscope}%
\pgfsys@transformshift{2.427146in}{1.019771in}%
\pgfsys@useobject{currentmarker}{}%
\end{pgfscope}%
\begin{pgfscope}%
\pgfsys@transformshift{2.606877in}{0.809341in}%
\pgfsys@useobject{currentmarker}{}%
\end{pgfscope}%
\begin{pgfscope}%
\pgfsys@transformshift{2.786608in}{1.230678in}%
\pgfsys@useobject{currentmarker}{}%
\end{pgfscope}%
\begin{pgfscope}%
\pgfsys@transformshift{2.966339in}{0.982488in}%
\pgfsys@useobject{currentmarker}{}%
\end{pgfscope}%
\begin{pgfscope}%
\pgfsys@transformshift{3.146070in}{0.878766in}%
\pgfsys@useobject{currentmarker}{}%
\end{pgfscope}%
\begin{pgfscope}%
\pgfsys@transformshift{3.325801in}{0.780542in}%
\pgfsys@useobject{currentmarker}{}%
\end{pgfscope}%
\begin{pgfscope}%
\pgfsys@transformshift{3.505532in}{0.772835in}%
\pgfsys@useobject{currentmarker}{}%
\end{pgfscope}%
\begin{pgfscope}%
\pgfsys@transformshift{3.685263in}{0.745829in}%
\pgfsys@useobject{currentmarker}{}%
\end{pgfscope}%
\begin{pgfscope}%
\pgfsys@transformshift{3.864994in}{0.778630in}%
\pgfsys@useobject{currentmarker}{}%
\end{pgfscope}%
\begin{pgfscope}%
\pgfsys@transformshift{4.044725in}{0.752521in}%
\pgfsys@useobject{currentmarker}{}%
\end{pgfscope}%
\begin{pgfscope}%
\pgfsys@transformshift{4.224456in}{0.752342in}%
\pgfsys@useobject{currentmarker}{}%
\end{pgfscope}%
\begin{pgfscope}%
\pgfsys@transformshift{4.404187in}{0.755628in}%
\pgfsys@useobject{currentmarker}{}%
\end{pgfscope}%
\begin{pgfscope}%
\pgfsys@transformshift{4.583918in}{0.731131in}%
\pgfsys@useobject{currentmarker}{}%
\end{pgfscope}%
\begin{pgfscope}%
\pgfsys@transformshift{4.763648in}{0.732625in}%
\pgfsys@useobject{currentmarker}{}%
\end{pgfscope}%
\begin{pgfscope}%
\pgfsys@transformshift{4.943379in}{0.731609in}%
\pgfsys@useobject{currentmarker}{}%
\end{pgfscope}%
\begin{pgfscope}%
\pgfsys@transformshift{5.123110in}{0.734955in}%
\pgfsys@useobject{currentmarker}{}%
\end{pgfscope}%
\begin{pgfscope}%
\pgfsys@transformshift{5.302841in}{0.739974in}%
\pgfsys@useobject{currentmarker}{}%
\end{pgfscope}%
\begin{pgfscope}%
\pgfsys@transformshift{5.482572in}{0.741348in}%
\pgfsys@useobject{currentmarker}{}%
\end{pgfscope}%
\begin{pgfscope}%
\pgfsys@transformshift{5.662303in}{0.727128in}%
\pgfsys@useobject{currentmarker}{}%
\end{pgfscope}%
\begin{pgfscope}%
\pgfsys@transformshift{5.842034in}{0.722886in}%
\pgfsys@useobject{currentmarker}{}%
\end{pgfscope}%
\begin{pgfscope}%
\pgfsys@transformshift{6.021765in}{0.720078in}%
\pgfsys@useobject{currentmarker}{}%
\end{pgfscope}%
\begin{pgfscope}%
\pgfsys@transformshift{6.201496in}{0.726053in}%
\pgfsys@useobject{currentmarker}{}%
\end{pgfscope}%
\begin{pgfscope}%
\pgfsys@transformshift{6.381227in}{0.719301in}%
\pgfsys@useobject{currentmarker}{}%
\end{pgfscope}%
\begin{pgfscope}%
\pgfsys@transformshift{6.560958in}{0.718883in}%
\pgfsys@useobject{currentmarker}{}%
\end{pgfscope}%
\begin{pgfscope}%
\pgfsys@transformshift{6.740689in}{0.728502in}%
\pgfsys@useobject{currentmarker}{}%
\end{pgfscope}%
\begin{pgfscope}%
\pgfsys@transformshift{6.920420in}{0.721811in}%
\pgfsys@useobject{currentmarker}{}%
\end{pgfscope}%
\begin{pgfscope}%
\pgfsys@transformshift{7.100151in}{0.719720in}%
\pgfsys@useobject{currentmarker}{}%
\end{pgfscope}%
\begin{pgfscope}%
\pgfsys@transformshift{7.279882in}{0.719540in}%
\pgfsys@useobject{currentmarker}{}%
\end{pgfscope}%
\begin{pgfscope}%
\pgfsys@transformshift{7.459613in}{0.722647in}%
\pgfsys@useobject{currentmarker}{}%
\end{pgfscope}%
\begin{pgfscope}%
\pgfsys@transformshift{7.639344in}{0.718166in}%
\pgfsys@useobject{currentmarker}{}%
\end{pgfscope}%
\begin{pgfscope}%
\pgfsys@transformshift{7.819075in}{0.719182in}%
\pgfsys@useobject{currentmarker}{}%
\end{pgfscope}%
\begin{pgfscope}%
\pgfsys@transformshift{7.998806in}{0.715717in}%
\pgfsys@useobject{currentmarker}{}%
\end{pgfscope}%
\begin{pgfscope}%
\pgfsys@transformshift{8.178537in}{0.718106in}%
\pgfsys@useobject{currentmarker}{}%
\end{pgfscope}%
\begin{pgfscope}%
\pgfsys@transformshift{8.358268in}{0.718106in}%
\pgfsys@useobject{currentmarker}{}%
\end{pgfscope}%
\begin{pgfscope}%
\pgfsys@transformshift{8.537999in}{0.715955in}%
\pgfsys@useobject{currentmarker}{}%
\end{pgfscope}%
\begin{pgfscope}%
\pgfsys@transformshift{8.717730in}{0.715298in}%
\pgfsys@useobject{currentmarker}{}%
\end{pgfscope}%
\begin{pgfscope}%
\pgfsys@transformshift{8.897461in}{0.715657in}%
\pgfsys@useobject{currentmarker}{}%
\end{pgfscope}%
\begin{pgfscope}%
\pgfsys@transformshift{9.077192in}{0.717808in}%
\pgfsys@useobject{currentmarker}{}%
\end{pgfscope}%
\begin{pgfscope}%
\pgfsys@transformshift{9.256923in}{0.715478in}%
\pgfsys@useobject{currentmarker}{}%
\end{pgfscope}%
\begin{pgfscope}%
\pgfsys@transformshift{9.436654in}{0.714103in}%
\pgfsys@useobject{currentmarker}{}%
\end{pgfscope}%
\begin{pgfscope}%
\pgfsys@transformshift{9.616385in}{0.716135in}%
\pgfsys@useobject{currentmarker}{}%
\end{pgfscope}%
\begin{pgfscope}%
\pgfsys@transformshift{9.796116in}{0.714283in}%
\pgfsys@useobject{currentmarker}{}%
\end{pgfscope}%
\begin{pgfscope}%
\pgfsys@transformshift{9.975847in}{0.714342in}%
\pgfsys@useobject{currentmarker}{}%
\end{pgfscope}%
\end{pgfscope}%
\begin{pgfscope}%
\pgfpathrectangle{\pgfqpoint{0.728688in}{0.521603in}}{\pgfqpoint{9.687500in}{4.235000in}}%
\pgfusepath{clip}%
\pgfsetbuttcap%
\pgfsetroundjoin%
\definecolor{currentfill}{rgb}{0.980392,0.164706,0.333333}%
\pgfsetfillcolor{currentfill}%
\pgfsetfillopacity{0.500000}%
\pgfsetlinewidth{1.003750pt}%
\definecolor{currentstroke}{rgb}{0.980392,0.164706,0.333333}%
\pgfsetstrokecolor{currentstroke}%
\pgfsetstrokeopacity{0.500000}%
\pgfsetdash{}{0pt}%
\pgfsys@defobject{currentmarker}{\pgfqpoint{-0.021960in}{-0.021960in}}{\pgfqpoint{0.021960in}{0.021960in}}{%
\pgfpathmoveto{\pgfqpoint{0.000000in}{-0.021960in}}%
\pgfpathcurveto{\pgfqpoint{0.005824in}{-0.021960in}}{\pgfqpoint{0.011410in}{-0.019646in}}{\pgfqpoint{0.015528in}{-0.015528in}}%
\pgfpathcurveto{\pgfqpoint{0.019646in}{-0.011410in}}{\pgfqpoint{0.021960in}{-0.005824in}}{\pgfqpoint{0.021960in}{0.000000in}}%
\pgfpathcurveto{\pgfqpoint{0.021960in}{0.005824in}}{\pgfqpoint{0.019646in}{0.011410in}}{\pgfqpoint{0.015528in}{0.015528in}}%
\pgfpathcurveto{\pgfqpoint{0.011410in}{0.019646in}}{\pgfqpoint{0.005824in}{0.021960in}}{\pgfqpoint{0.000000in}{0.021960in}}%
\pgfpathcurveto{\pgfqpoint{-0.005824in}{0.021960in}}{\pgfqpoint{-0.011410in}{0.019646in}}{\pgfqpoint{-0.015528in}{0.015528in}}%
\pgfpathcurveto{\pgfqpoint{-0.019646in}{0.011410in}}{\pgfqpoint{-0.021960in}{0.005824in}}{\pgfqpoint{-0.021960in}{0.000000in}}%
\pgfpathcurveto{\pgfqpoint{-0.021960in}{-0.005824in}}{\pgfqpoint{-0.019646in}{-0.011410in}}{\pgfqpoint{-0.015528in}{-0.015528in}}%
\pgfpathcurveto{\pgfqpoint{-0.011410in}{-0.019646in}}{\pgfqpoint{-0.005824in}{-0.021960in}}{\pgfqpoint{0.000000in}{-0.021960in}}%
\pgfpathlineto{\pgfqpoint{0.000000in}{-0.021960in}}%
\pgfpathclose%
\pgfusepath{stroke,fill}%
}%
\begin{pgfscope}%
\pgfsys@transformshift{1.169029in}{4.391817in}%
\pgfsys@useobject{currentmarker}{}%
\end{pgfscope}%
\begin{pgfscope}%
\pgfsys@transformshift{1.348760in}{4.322162in}%
\pgfsys@useobject{currentmarker}{}%
\end{pgfscope}%
\begin{pgfscope}%
\pgfsys@transformshift{1.528491in}{4.026152in}%
\pgfsys@useobject{currentmarker}{}%
\end{pgfscope}%
\begin{pgfscope}%
\pgfsys@transformshift{1.708222in}{3.852214in}%
\pgfsys@useobject{currentmarker}{}%
\end{pgfscope}%
\begin{pgfscope}%
\pgfsys@transformshift{1.887953in}{3.563958in}%
\pgfsys@useobject{currentmarker}{}%
\end{pgfscope}%
\begin{pgfscope}%
\pgfsys@transformshift{2.067684in}{3.194653in}%
\pgfsys@useobject{currentmarker}{}%
\end{pgfscope}%
\begin{pgfscope}%
\pgfsys@transformshift{2.247415in}{2.992969in}%
\pgfsys@useobject{currentmarker}{}%
\end{pgfscope}%
\begin{pgfscope}%
\pgfsys@transformshift{2.427146in}{2.672533in}%
\pgfsys@useobject{currentmarker}{}%
\end{pgfscope}%
\begin{pgfscope}%
\pgfsys@transformshift{2.606877in}{2.406884in}%
\pgfsys@useobject{currentmarker}{}%
\end{pgfscope}%
\begin{pgfscope}%
\pgfsys@transformshift{2.786608in}{2.273233in}%
\pgfsys@useobject{currentmarker}{}%
\end{pgfscope}%
\begin{pgfscope}%
\pgfsys@transformshift{2.966339in}{1.935697in}%
\pgfsys@useobject{currentmarker}{}%
\end{pgfscope}%
\begin{pgfscope}%
\pgfsys@transformshift{3.146070in}{1.472835in}%
\pgfsys@useobject{currentmarker}{}%
\end{pgfscope}%
\begin{pgfscope}%
\pgfsys@transformshift{3.325801in}{0.914585in}%
\pgfsys@useobject{currentmarker}{}%
\end{pgfscope}%
\begin{pgfscope}%
\pgfsys@transformshift{3.505532in}{0.808922in}%
\pgfsys@useobject{currentmarker}{}%
\end{pgfscope}%
\begin{pgfscope}%
\pgfsys@transformshift{3.685263in}{0.784725in}%
\pgfsys@useobject{currentmarker}{}%
\end{pgfscope}%
\begin{pgfscope}%
\pgfsys@transformshift{3.864994in}{0.777092in}%
\pgfsys@useobject{currentmarker}{}%
\end{pgfscope}%
\begin{pgfscope}%
\pgfsys@transformshift{4.044725in}{0.782215in}%
\pgfsys@useobject{currentmarker}{}%
\end{pgfscope}%
\begin{pgfscope}%
\pgfsys@transformshift{4.224456in}{0.762887in}%
\pgfsys@useobject{currentmarker}{}%
\end{pgfscope}%
\begin{pgfscope}%
\pgfsys@transformshift{4.404187in}{0.735836in}%
\pgfsys@useobject{currentmarker}{}%
\end{pgfscope}%
\begin{pgfscope}%
\pgfsys@transformshift{4.583918in}{0.743051in}%
\pgfsys@useobject{currentmarker}{}%
\end{pgfscope}%
\begin{pgfscope}%
\pgfsys@transformshift{4.763648in}{0.742334in}%
\pgfsys@useobject{currentmarker}{}%
\end{pgfscope}%
\begin{pgfscope}%
\pgfsys@transformshift{4.943379in}{0.731923in}%
\pgfsys@useobject{currentmarker}{}%
\end{pgfscope}%
\begin{pgfscope}%
\pgfsys@transformshift{5.123110in}{0.737465in}%
\pgfsys@useobject{currentmarker}{}%
\end{pgfscope}%
\begin{pgfscope}%
\pgfsys@transformshift{5.302841in}{0.734134in}%
\pgfsys@useobject{currentmarker}{}%
\end{pgfscope}%
\begin{pgfscope}%
\pgfsys@transformshift{5.482572in}{0.732909in}%
\pgfsys@useobject{currentmarker}{}%
\end{pgfscope}%
\begin{pgfscope}%
\pgfsys@transformshift{5.662303in}{0.729443in}%
\pgfsys@useobject{currentmarker}{}%
\end{pgfscope}%
\begin{pgfscope}%
\pgfsys@transformshift{5.842034in}{0.725948in}%
\pgfsys@useobject{currentmarker}{}%
\end{pgfscope}%
\begin{pgfscope}%
\pgfsys@transformshift{6.021765in}{0.726098in}%
\pgfsys@useobject{currentmarker}{}%
\end{pgfscope}%
\begin{pgfscope}%
\pgfsys@transformshift{6.201496in}{0.724440in}%
\pgfsys@useobject{currentmarker}{}%
\end{pgfscope}%
\begin{pgfscope}%
\pgfsys@transformshift{6.381227in}{0.726830in}%
\pgfsys@useobject{currentmarker}{}%
\end{pgfscope}%
\begin{pgfscope}%
\pgfsys@transformshift{6.560958in}{0.723230in}%
\pgfsys@useobject{currentmarker}{}%
\end{pgfscope}%
\begin{pgfscope}%
\pgfsys@transformshift{6.740689in}{0.724664in}%
\pgfsys@useobject{currentmarker}{}%
\end{pgfscope}%
\begin{pgfscope}%
\pgfsys@transformshift{6.920420in}{0.720870in}%
\pgfsys@useobject{currentmarker}{}%
\end{pgfscope}%
\begin{pgfscope}%
\pgfsys@transformshift{7.100151in}{0.720586in}%
\pgfsys@useobject{currentmarker}{}%
\end{pgfscope}%
\begin{pgfscope}%
\pgfsys@transformshift{7.279882in}{0.720750in}%
\pgfsys@useobject{currentmarker}{}%
\end{pgfscope}%
\begin{pgfscope}%
\pgfsys@transformshift{7.459613in}{0.721139in}%
\pgfsys@useobject{currentmarker}{}%
\end{pgfscope}%
\begin{pgfscope}%
\pgfsys@transformshift{7.639344in}{0.719003in}%
\pgfsys@useobject{currentmarker}{}%
\end{pgfscope}%
\begin{pgfscope}%
\pgfsys@transformshift{7.819075in}{0.719122in}%
\pgfsys@useobject{currentmarker}{}%
\end{pgfscope}%
\begin{pgfscope}%
\pgfsys@transformshift{7.998806in}{0.717852in}%
\pgfsys@useobject{currentmarker}{}%
\end{pgfscope}%
\begin{pgfscope}%
\pgfsys@transformshift{8.178537in}{0.719167in}%
\pgfsys@useobject{currentmarker}{}%
\end{pgfscope}%
\begin{pgfscope}%
\pgfsys@transformshift{8.358268in}{0.717494in}%
\pgfsys@useobject{currentmarker}{}%
\end{pgfscope}%
\begin{pgfscope}%
\pgfsys@transformshift{8.537999in}{0.716643in}%
\pgfsys@useobject{currentmarker}{}%
\end{pgfscope}%
\begin{pgfscope}%
\pgfsys@transformshift{8.717730in}{0.716404in}%
\pgfsys@useobject{currentmarker}{}%
\end{pgfscope}%
\begin{pgfscope}%
\pgfsys@transformshift{8.897461in}{0.716090in}%
\pgfsys@useobject{currentmarker}{}%
\end{pgfscope}%
\begin{pgfscope}%
\pgfsys@transformshift{9.077192in}{0.716344in}%
\pgfsys@useobject{currentmarker}{}%
\end{pgfscope}%
\begin{pgfscope}%
\pgfsys@transformshift{9.256923in}{0.715089in}%
\pgfsys@useobject{currentmarker}{}%
\end{pgfscope}%
\begin{pgfscope}%
\pgfsys@transformshift{9.436654in}{0.715283in}%
\pgfsys@useobject{currentmarker}{}%
\end{pgfscope}%
\begin{pgfscope}%
\pgfsys@transformshift{9.616385in}{0.714835in}%
\pgfsys@useobject{currentmarker}{}%
\end{pgfscope}%
\begin{pgfscope}%
\pgfsys@transformshift{9.796116in}{0.714566in}%
\pgfsys@useobject{currentmarker}{}%
\end{pgfscope}%
\begin{pgfscope}%
\pgfsys@transformshift{9.975847in}{0.714327in}%
\pgfsys@useobject{currentmarker}{}%
\end{pgfscope}%
\end{pgfscope}%
\begin{pgfscope}%
\pgfpathrectangle{\pgfqpoint{0.728688in}{0.521603in}}{\pgfqpoint{9.687500in}{4.235000in}}%
\pgfusepath{clip}%
\pgfsetbuttcap%
\pgfsetroundjoin%
\definecolor{currentfill}{rgb}{0.239216,0.478431,0.992157}%
\pgfsetfillcolor{currentfill}%
\pgfsetfillopacity{0.500000}%
\pgfsetlinewidth{1.003750pt}%
\definecolor{currentstroke}{rgb}{0.239216,0.478431,0.992157}%
\pgfsetstrokecolor{currentstroke}%
\pgfsetstrokeopacity{0.500000}%
\pgfsetdash{}{0pt}%
\pgfsys@defobject{currentmarker}{\pgfqpoint{-0.021960in}{-0.021960in}}{\pgfqpoint{0.021960in}{0.021960in}}{%
\pgfpathmoveto{\pgfqpoint{0.000000in}{-0.021960in}}%
\pgfpathcurveto{\pgfqpoint{0.005824in}{-0.021960in}}{\pgfqpoint{0.011410in}{-0.019646in}}{\pgfqpoint{0.015528in}{-0.015528in}}%
\pgfpathcurveto{\pgfqpoint{0.019646in}{-0.011410in}}{\pgfqpoint{0.021960in}{-0.005824in}}{\pgfqpoint{0.021960in}{0.000000in}}%
\pgfpathcurveto{\pgfqpoint{0.021960in}{0.005824in}}{\pgfqpoint{0.019646in}{0.011410in}}{\pgfqpoint{0.015528in}{0.015528in}}%
\pgfpathcurveto{\pgfqpoint{0.011410in}{0.019646in}}{\pgfqpoint{0.005824in}{0.021960in}}{\pgfqpoint{0.000000in}{0.021960in}}%
\pgfpathcurveto{\pgfqpoint{-0.005824in}{0.021960in}}{\pgfqpoint{-0.011410in}{0.019646in}}{\pgfqpoint{-0.015528in}{0.015528in}}%
\pgfpathcurveto{\pgfqpoint{-0.019646in}{0.011410in}}{\pgfqpoint{-0.021960in}{0.005824in}}{\pgfqpoint{-0.021960in}{0.000000in}}%
\pgfpathcurveto{\pgfqpoint{-0.021960in}{-0.005824in}}{\pgfqpoint{-0.019646in}{-0.011410in}}{\pgfqpoint{-0.015528in}{-0.015528in}}%
\pgfpathcurveto{\pgfqpoint{-0.011410in}{-0.019646in}}{\pgfqpoint{-0.005824in}{-0.021960in}}{\pgfqpoint{0.000000in}{-0.021960in}}%
\pgfpathlineto{\pgfqpoint{0.000000in}{-0.021960in}}%
\pgfpathclose%
\pgfusepath{stroke,fill}%
}%
\begin{pgfscope}%
\pgfsys@transformshift{1.169029in}{4.508214in}%
\pgfsys@useobject{currentmarker}{}%
\end{pgfscope}%
\begin{pgfscope}%
\pgfsys@transformshift{1.348760in}{4.350030in}%
\pgfsys@useobject{currentmarker}{}%
\end{pgfscope}%
\begin{pgfscope}%
\pgfsys@transformshift{1.528491in}{4.111408in}%
\pgfsys@useobject{currentmarker}{}%
\end{pgfscope}%
\begin{pgfscope}%
\pgfsys@transformshift{1.708222in}{3.898727in}%
\pgfsys@useobject{currentmarker}{}%
\end{pgfscope}%
\begin{pgfscope}%
\pgfsys@transformshift{1.887953in}{3.717998in}%
\pgfsys@useobject{currentmarker}{}%
\end{pgfscope}%
\begin{pgfscope}%
\pgfsys@transformshift{2.067684in}{3.402746in}%
\pgfsys@useobject{currentmarker}{}%
\end{pgfscope}%
\begin{pgfscope}%
\pgfsys@transformshift{2.247415in}{3.179113in}%
\pgfsys@useobject{currentmarker}{}%
\end{pgfscope}%
\begin{pgfscope}%
\pgfsys@transformshift{2.427146in}{2.843213in}%
\pgfsys@useobject{currentmarker}{}%
\end{pgfscope}%
\begin{pgfscope}%
\pgfsys@transformshift{2.606877in}{2.673695in}%
\pgfsys@useobject{currentmarker}{}%
\end{pgfscope}%
\begin{pgfscope}%
\pgfsys@transformshift{2.786608in}{2.320542in}%
\pgfsys@useobject{currentmarker}{}%
\end{pgfscope}%
\begin{pgfscope}%
\pgfsys@transformshift{2.966339in}{1.944972in}%
\pgfsys@useobject{currentmarker}{}%
\end{pgfscope}%
\begin{pgfscope}%
\pgfsys@transformshift{3.146070in}{1.702614in}%
\pgfsys@useobject{currentmarker}{}%
\end{pgfscope}%
\begin{pgfscope}%
\pgfsys@transformshift{3.325801in}{1.103289in}%
\pgfsys@useobject{currentmarker}{}%
\end{pgfscope}%
\begin{pgfscope}%
\pgfsys@transformshift{3.505532in}{1.007478in}%
\pgfsys@useobject{currentmarker}{}%
\end{pgfscope}%
\begin{pgfscope}%
\pgfsys@transformshift{3.685263in}{0.803504in}%
\pgfsys@useobject{currentmarker}{}%
\end{pgfscope}%
\begin{pgfscope}%
\pgfsys@transformshift{3.864994in}{0.846261in}%
\pgfsys@useobject{currentmarker}{}%
\end{pgfscope}%
\begin{pgfscope}%
\pgfsys@transformshift{4.044725in}{0.763596in}%
\pgfsys@useobject{currentmarker}{}%
\end{pgfscope}%
\begin{pgfscope}%
\pgfsys@transformshift{4.224456in}{0.764336in}%
\pgfsys@useobject{currentmarker}{}%
\end{pgfscope}%
\begin{pgfscope}%
\pgfsys@transformshift{4.404187in}{0.759059in}%
\pgfsys@useobject{currentmarker}{}%
\end{pgfscope}%
\begin{pgfscope}%
\pgfsys@transformshift{4.583918in}{0.750605in}%
\pgfsys@useobject{currentmarker}{}%
\end{pgfscope}%
\begin{pgfscope}%
\pgfsys@transformshift{4.763648in}{0.748484in}%
\pgfsys@useobject{currentmarker}{}%
\end{pgfscope}%
\begin{pgfscope}%
\pgfsys@transformshift{4.943379in}{0.741053in}%
\pgfsys@useobject{currentmarker}{}%
\end{pgfscope}%
\begin{pgfscope}%
\pgfsys@transformshift{5.123110in}{0.734007in}%
\pgfsys@useobject{currentmarker}{}%
\end{pgfscope}%
\begin{pgfscope}%
\pgfsys@transformshift{5.302841in}{0.734862in}%
\pgfsys@useobject{currentmarker}{}%
\end{pgfscope}%
\begin{pgfscope}%
\pgfsys@transformshift{5.482572in}{0.729369in}%
\pgfsys@useobject{currentmarker}{}%
\end{pgfscope}%
\begin{pgfscope}%
\pgfsys@transformshift{5.662303in}{0.731867in}%
\pgfsys@useobject{currentmarker}{}%
\end{pgfscope}%
\begin{pgfscope}%
\pgfsys@transformshift{5.842034in}{0.729111in}%
\pgfsys@useobject{currentmarker}{}%
\end{pgfscope}%
\begin{pgfscope}%
\pgfsys@transformshift{6.021765in}{0.727804in}%
\pgfsys@useobject{currentmarker}{}%
\end{pgfscope}%
\begin{pgfscope}%
\pgfsys@transformshift{6.201496in}{0.726949in}%
\pgfsys@useobject{currentmarker}{}%
\end{pgfscope}%
\begin{pgfscope}%
\pgfsys@transformshift{6.381227in}{0.725582in}%
\pgfsys@useobject{currentmarker}{}%
\end{pgfscope}%
\begin{pgfscope}%
\pgfsys@transformshift{6.560958in}{0.724895in}%
\pgfsys@useobject{currentmarker}{}%
\end{pgfscope}%
\begin{pgfscope}%
\pgfsys@transformshift{6.740689in}{0.721590in}%
\pgfsys@useobject{currentmarker}{}%
\end{pgfscope}%
\begin{pgfscope}%
\pgfsys@transformshift{6.920420in}{0.722128in}%
\pgfsys@useobject{currentmarker}{}%
\end{pgfscope}%
\begin{pgfscope}%
\pgfsys@transformshift{7.100151in}{0.722307in}%
\pgfsys@useobject{currentmarker}{}%
\end{pgfscope}%
\begin{pgfscope}%
\pgfsys@transformshift{7.279882in}{0.720369in}%
\pgfsys@useobject{currentmarker}{}%
\end{pgfscope}%
\begin{pgfscope}%
\pgfsys@transformshift{7.459613in}{0.719708in}%
\pgfsys@useobject{currentmarker}{}%
\end{pgfscope}%
\begin{pgfscope}%
\pgfsys@transformshift{7.639344in}{0.719215in}%
\pgfsys@useobject{currentmarker}{}%
\end{pgfscope}%
\begin{pgfscope}%
\pgfsys@transformshift{7.819075in}{0.718719in}%
\pgfsys@useobject{currentmarker}{}%
\end{pgfscope}%
\begin{pgfscope}%
\pgfsys@transformshift{7.998806in}{0.718327in}%
\pgfsys@useobject{currentmarker}{}%
\end{pgfscope}%
\begin{pgfscope}%
\pgfsys@transformshift{8.178537in}{0.717572in}%
\pgfsys@useobject{currentmarker}{}%
\end{pgfscope}%
\begin{pgfscope}%
\pgfsys@transformshift{8.358268in}{0.717240in}%
\pgfsys@useobject{currentmarker}{}%
\end{pgfscope}%
\begin{pgfscope}%
\pgfsys@transformshift{8.537999in}{0.717341in}%
\pgfsys@useobject{currentmarker}{}%
\end{pgfscope}%
\begin{pgfscope}%
\pgfsys@transformshift{8.717730in}{0.716687in}%
\pgfsys@useobject{currentmarker}{}%
\end{pgfscope}%
\begin{pgfscope}%
\pgfsys@transformshift{8.897461in}{0.716187in}%
\pgfsys@useobject{currentmarker}{}%
\end{pgfscope}%
\begin{pgfscope}%
\pgfsys@transformshift{9.077192in}{0.715814in}%
\pgfsys@useobject{currentmarker}{}%
\end{pgfscope}%
\begin{pgfscope}%
\pgfsys@transformshift{9.256923in}{0.715227in}%
\pgfsys@useobject{currentmarker}{}%
\end{pgfscope}%
\begin{pgfscope}%
\pgfsys@transformshift{9.436654in}{0.715160in}%
\pgfsys@useobject{currentmarker}{}%
\end{pgfscope}%
\begin{pgfscope}%
\pgfsys@transformshift{9.616385in}{0.714772in}%
\pgfsys@useobject{currentmarker}{}%
\end{pgfscope}%
\begin{pgfscope}%
\pgfsys@transformshift{9.796116in}{0.714525in}%
\pgfsys@useobject{currentmarker}{}%
\end{pgfscope}%
\begin{pgfscope}%
\pgfsys@transformshift{9.975847in}{0.714305in}%
\pgfsys@useobject{currentmarker}{}%
\end{pgfscope}%
\end{pgfscope}%
\begin{pgfscope}%
\pgfpathrectangle{\pgfqpoint{0.728688in}{0.521603in}}{\pgfqpoint{9.687500in}{4.235000in}}%
\pgfusepath{clip}%
\pgfsetbuttcap%
\pgfsetroundjoin%
\definecolor{currentfill}{rgb}{0.000000,0.000000,0.000000}%
\pgfsetfillcolor{currentfill}%
\pgfsetfillopacity{0.500000}%
\pgfsetlinewidth{1.003750pt}%
\definecolor{currentstroke}{rgb}{0.000000,0.000000,0.000000}%
\pgfsetstrokecolor{currentstroke}%
\pgfsetstrokeopacity{0.500000}%
\pgfsetdash{}{0pt}%
\pgfsys@defobject{currentmarker}{\pgfqpoint{-0.021960in}{-0.021960in}}{\pgfqpoint{0.021960in}{0.021960in}}{%
\pgfpathmoveto{\pgfqpoint{0.000000in}{-0.021960in}}%
\pgfpathcurveto{\pgfqpoint{0.005824in}{-0.021960in}}{\pgfqpoint{0.011410in}{-0.019646in}}{\pgfqpoint{0.015528in}{-0.015528in}}%
\pgfpathcurveto{\pgfqpoint{0.019646in}{-0.011410in}}{\pgfqpoint{0.021960in}{-0.005824in}}{\pgfqpoint{0.021960in}{0.000000in}}%
\pgfpathcurveto{\pgfqpoint{0.021960in}{0.005824in}}{\pgfqpoint{0.019646in}{0.011410in}}{\pgfqpoint{0.015528in}{0.015528in}}%
\pgfpathcurveto{\pgfqpoint{0.011410in}{0.019646in}}{\pgfqpoint{0.005824in}{0.021960in}}{\pgfqpoint{0.000000in}{0.021960in}}%
\pgfpathcurveto{\pgfqpoint{-0.005824in}{0.021960in}}{\pgfqpoint{-0.011410in}{0.019646in}}{\pgfqpoint{-0.015528in}{0.015528in}}%
\pgfpathcurveto{\pgfqpoint{-0.019646in}{0.011410in}}{\pgfqpoint{-0.021960in}{0.005824in}}{\pgfqpoint{-0.021960in}{0.000000in}}%
\pgfpathcurveto{\pgfqpoint{-0.021960in}{-0.005824in}}{\pgfqpoint{-0.019646in}{-0.011410in}}{\pgfqpoint{-0.015528in}{-0.015528in}}%
\pgfpathcurveto{\pgfqpoint{-0.011410in}{-0.019646in}}{\pgfqpoint{-0.005824in}{-0.021960in}}{\pgfqpoint{0.000000in}{-0.021960in}}%
\pgfpathlineto{\pgfqpoint{0.000000in}{-0.021960in}}%
\pgfpathclose%
\pgfusepath{stroke,fill}%
}%
\begin{pgfscope}%
\pgfsys@transformshift{1.169029in}{4.564103in}%
\pgfsys@useobject{currentmarker}{}%
\end{pgfscope}%
\begin{pgfscope}%
\pgfsys@transformshift{1.348760in}{4.375359in}%
\pgfsys@useobject{currentmarker}{}%
\end{pgfscope}%
\begin{pgfscope}%
\pgfsys@transformshift{1.528491in}{4.185422in}%
\pgfsys@useobject{currentmarker}{}%
\end{pgfscope}%
\begin{pgfscope}%
\pgfsys@transformshift{1.708222in}{3.983019in}%
\pgfsys@useobject{currentmarker}{}%
\end{pgfscope}%
\begin{pgfscope}%
\pgfsys@transformshift{1.887953in}{3.760243in}%
\pgfsys@useobject{currentmarker}{}%
\end{pgfscope}%
\begin{pgfscope}%
\pgfsys@transformshift{2.067684in}{3.524466in}%
\pgfsys@useobject{currentmarker}{}%
\end{pgfscope}%
\begin{pgfscope}%
\pgfsys@transformshift{2.247415in}{3.349319in}%
\pgfsys@useobject{currentmarker}{}%
\end{pgfscope}%
\begin{pgfscope}%
\pgfsys@transformshift{2.427146in}{3.014918in}%
\pgfsys@useobject{currentmarker}{}%
\end{pgfscope}%
\begin{pgfscope}%
\pgfsys@transformshift{2.606877in}{2.783607in}%
\pgfsys@useobject{currentmarker}{}%
\end{pgfscope}%
\begin{pgfscope}%
\pgfsys@transformshift{2.786608in}{2.489402in}%
\pgfsys@useobject{currentmarker}{}%
\end{pgfscope}%
\begin{pgfscope}%
\pgfsys@transformshift{2.966339in}{2.053102in}%
\pgfsys@useobject{currentmarker}{}%
\end{pgfscope}%
\begin{pgfscope}%
\pgfsys@transformshift{3.146070in}{1.735807in}%
\pgfsys@useobject{currentmarker}{}%
\end{pgfscope}%
\begin{pgfscope}%
\pgfsys@transformshift{3.325801in}{1.436998in}%
\pgfsys@useobject{currentmarker}{}%
\end{pgfscope}%
\begin{pgfscope}%
\pgfsys@transformshift{3.505532in}{1.112781in}%
\pgfsys@useobject{currentmarker}{}%
\end{pgfscope}%
\begin{pgfscope}%
\pgfsys@transformshift{3.685263in}{0.883268in}%
\pgfsys@useobject{currentmarker}{}%
\end{pgfscope}%
\begin{pgfscope}%
\pgfsys@transformshift{3.864994in}{0.821761in}%
\pgfsys@useobject{currentmarker}{}%
\end{pgfscope}%
\begin{pgfscope}%
\pgfsys@transformshift{4.044725in}{0.778708in}%
\pgfsys@useobject{currentmarker}{}%
\end{pgfscope}%
\begin{pgfscope}%
\pgfsys@transformshift{4.224456in}{0.766316in}%
\pgfsys@useobject{currentmarker}{}%
\end{pgfscope}%
\begin{pgfscope}%
\pgfsys@transformshift{4.404187in}{0.756322in}%
\pgfsys@useobject{currentmarker}{}%
\end{pgfscope}%
\begin{pgfscope}%
\pgfsys@transformshift{4.583918in}{0.752370in}%
\pgfsys@useobject{currentmarker}{}%
\end{pgfscope}%
\begin{pgfscope}%
\pgfsys@transformshift{4.763648in}{0.747130in}%
\pgfsys@useobject{currentmarker}{}%
\end{pgfscope}%
\begin{pgfscope}%
\pgfsys@transformshift{4.943379in}{0.741988in}%
\pgfsys@useobject{currentmarker}{}%
\end{pgfscope}%
\begin{pgfscope}%
\pgfsys@transformshift{5.123110in}{0.736495in}%
\pgfsys@useobject{currentmarker}{}%
\end{pgfscope}%
\begin{pgfscope}%
\pgfsys@transformshift{5.302841in}{0.737631in}%
\pgfsys@useobject{currentmarker}{}%
\end{pgfscope}%
\begin{pgfscope}%
\pgfsys@transformshift{5.482572in}{0.731761in}%
\pgfsys@useobject{currentmarker}{}%
\end{pgfscope}%
\begin{pgfscope}%
\pgfsys@transformshift{5.662303in}{0.732099in}%
\pgfsys@useobject{currentmarker}{}%
\end{pgfscope}%
\begin{pgfscope}%
\pgfsys@transformshift{5.842034in}{0.728507in}%
\pgfsys@useobject{currentmarker}{}%
\end{pgfscope}%
\begin{pgfscope}%
\pgfsys@transformshift{6.021765in}{0.728153in}%
\pgfsys@useobject{currentmarker}{}%
\end{pgfscope}%
\begin{pgfscope}%
\pgfsys@transformshift{6.201496in}{0.726730in}%
\pgfsys@useobject{currentmarker}{}%
\end{pgfscope}%
\begin{pgfscope}%
\pgfsys@transformshift{6.381227in}{0.724779in}%
\pgfsys@useobject{currentmarker}{}%
\end{pgfscope}%
\begin{pgfscope}%
\pgfsys@transformshift{6.560958in}{0.723371in}%
\pgfsys@useobject{currentmarker}{}%
\end{pgfscope}%
\begin{pgfscope}%
\pgfsys@transformshift{6.740689in}{0.722704in}%
\pgfsys@useobject{currentmarker}{}%
\end{pgfscope}%
\begin{pgfscope}%
\pgfsys@transformshift{6.920420in}{0.721829in}%
\pgfsys@useobject{currentmarker}{}%
\end{pgfscope}%
\begin{pgfscope}%
\pgfsys@transformshift{7.100151in}{0.721662in}%
\pgfsys@useobject{currentmarker}{}%
\end{pgfscope}%
\begin{pgfscope}%
\pgfsys@transformshift{7.279882in}{0.720446in}%
\pgfsys@useobject{currentmarker}{}%
\end{pgfscope}%
\begin{pgfscope}%
\pgfsys@transformshift{7.459613in}{0.719978in}%
\pgfsys@useobject{currentmarker}{}%
\end{pgfscope}%
\begin{pgfscope}%
\pgfsys@transformshift{7.639344in}{0.719070in}%
\pgfsys@useobject{currentmarker}{}%
\end{pgfscope}%
\begin{pgfscope}%
\pgfsys@transformshift{7.819075in}{0.719093in}%
\pgfsys@useobject{currentmarker}{}%
\end{pgfscope}%
\begin{pgfscope}%
\pgfsys@transformshift{7.998806in}{0.718690in}%
\pgfsys@useobject{currentmarker}{}%
\end{pgfscope}%
\begin{pgfscope}%
\pgfsys@transformshift{8.178537in}{0.717470in}%
\pgfsys@useobject{currentmarker}{}%
\end{pgfscope}%
\begin{pgfscope}%
\pgfsys@transformshift{8.358268in}{0.717466in}%
\pgfsys@useobject{currentmarker}{}%
\end{pgfscope}%
\begin{pgfscope}%
\pgfsys@transformshift{8.537999in}{0.717053in}%
\pgfsys@useobject{currentmarker}{}%
\end{pgfscope}%
\begin{pgfscope}%
\pgfsys@transformshift{8.717730in}{0.716508in}%
\pgfsys@useobject{currentmarker}{}%
\end{pgfscope}%
\begin{pgfscope}%
\pgfsys@transformshift{8.897461in}{0.715971in}%
\pgfsys@useobject{currentmarker}{}%
\end{pgfscope}%
\begin{pgfscope}%
\pgfsys@transformshift{9.077192in}{0.715778in}%
\pgfsys@useobject{currentmarker}{}%
\end{pgfscope}%
\begin{pgfscope}%
\pgfsys@transformshift{9.256923in}{0.715462in}%
\pgfsys@useobject{currentmarker}{}%
\end{pgfscope}%
\begin{pgfscope}%
\pgfsys@transformshift{9.436654in}{0.715133in}%
\pgfsys@useobject{currentmarker}{}%
\end{pgfscope}%
\begin{pgfscope}%
\pgfsys@transformshift{9.616385in}{0.714836in}%
\pgfsys@useobject{currentmarker}{}%
\end{pgfscope}%
\begin{pgfscope}%
\pgfsys@transformshift{9.796116in}{0.714465in}%
\pgfsys@useobject{currentmarker}{}%
\end{pgfscope}%
\begin{pgfscope}%
\pgfsys@transformshift{9.975847in}{0.714263in}%
\pgfsys@useobject{currentmarker}{}%
\end{pgfscope}%
\end{pgfscope}%
\begin{pgfscope}%
\pgfpathrectangle{\pgfqpoint{0.728688in}{0.521603in}}{\pgfqpoint{9.687500in}{4.235000in}}%
\pgfusepath{clip}%
\pgfsetrectcap%
\pgfsetroundjoin%
\pgfsetlinewidth{0.803000pt}%
\definecolor{currentstroke}{rgb}{0.690196,0.690196,0.690196}%
\pgfsetstrokecolor{currentstroke}%
\pgfsetdash{}{0pt}%
\pgfpathmoveto{\pgfqpoint{1.169029in}{0.521603in}}%
\pgfpathlineto{\pgfqpoint{1.169029in}{4.756603in}}%
\pgfusepath{stroke}%
\end{pgfscope}%
\begin{pgfscope}%
\pgfsetbuttcap%
\pgfsetroundjoin%
\definecolor{currentfill}{rgb}{0.000000,0.000000,0.000000}%
\pgfsetfillcolor{currentfill}%
\pgfsetlinewidth{0.803000pt}%
\definecolor{currentstroke}{rgb}{0.000000,0.000000,0.000000}%
\pgfsetstrokecolor{currentstroke}%
\pgfsetdash{}{0pt}%
\pgfsys@defobject{currentmarker}{\pgfqpoint{0.000000in}{-0.048611in}}{\pgfqpoint{0.000000in}{0.000000in}}{%
\pgfpathmoveto{\pgfqpoint{0.000000in}{0.000000in}}%
\pgfpathlineto{\pgfqpoint{0.000000in}{-0.048611in}}%
\pgfusepath{stroke,fill}%
}%
\begin{pgfscope}%
\pgfsys@transformshift{1.169029in}{0.521603in}%
\pgfsys@useobject{currentmarker}{}%
\end{pgfscope}%
\end{pgfscope}%
\begin{pgfscope}%
\definecolor{textcolor}{rgb}{0.000000,0.000000,0.000000}%
\pgfsetstrokecolor{textcolor}%
\pgfsetfillcolor{textcolor}%
\pgftext[x=1.169029in,y=0.424381in,,top]{\color{textcolor}\sffamily\fontsize{10.000000}{12.000000}\selectfont 0.0}%
\end{pgfscope}%
\begin{pgfscope}%
\pgfpathrectangle{\pgfqpoint{0.728688in}{0.521603in}}{\pgfqpoint{9.687500in}{4.235000in}}%
\pgfusepath{clip}%
\pgfsetrectcap%
\pgfsetroundjoin%
\pgfsetlinewidth{0.803000pt}%
\definecolor{currentstroke}{rgb}{0.690196,0.690196,0.690196}%
\pgfsetstrokecolor{currentstroke}%
\pgfsetdash{}{0pt}%
\pgfpathmoveto{\pgfqpoint{2.966339in}{0.521603in}}%
\pgfpathlineto{\pgfqpoint{2.966339in}{4.756603in}}%
\pgfusepath{stroke}%
\end{pgfscope}%
\begin{pgfscope}%
\pgfsetbuttcap%
\pgfsetroundjoin%
\definecolor{currentfill}{rgb}{0.000000,0.000000,0.000000}%
\pgfsetfillcolor{currentfill}%
\pgfsetlinewidth{0.803000pt}%
\definecolor{currentstroke}{rgb}{0.000000,0.000000,0.000000}%
\pgfsetstrokecolor{currentstroke}%
\pgfsetdash{}{0pt}%
\pgfsys@defobject{currentmarker}{\pgfqpoint{0.000000in}{-0.048611in}}{\pgfqpoint{0.000000in}{0.000000in}}{%
\pgfpathmoveto{\pgfqpoint{0.000000in}{0.000000in}}%
\pgfpathlineto{\pgfqpoint{0.000000in}{-0.048611in}}%
\pgfusepath{stroke,fill}%
}%
\begin{pgfscope}%
\pgfsys@transformshift{2.966339in}{0.521603in}%
\pgfsys@useobject{currentmarker}{}%
\end{pgfscope}%
\end{pgfscope}%
\begin{pgfscope}%
\definecolor{textcolor}{rgb}{0.000000,0.000000,0.000000}%
\pgfsetstrokecolor{textcolor}%
\pgfsetfillcolor{textcolor}%
\pgftext[x=2.966339in,y=0.424381in,,top]{\color{textcolor}\sffamily\fontsize{10.000000}{12.000000}\selectfont 0.2}%
\end{pgfscope}%
\begin{pgfscope}%
\pgfpathrectangle{\pgfqpoint{0.728688in}{0.521603in}}{\pgfqpoint{9.687500in}{4.235000in}}%
\pgfusepath{clip}%
\pgfsetrectcap%
\pgfsetroundjoin%
\pgfsetlinewidth{0.803000pt}%
\definecolor{currentstroke}{rgb}{0.690196,0.690196,0.690196}%
\pgfsetstrokecolor{currentstroke}%
\pgfsetdash{}{0pt}%
\pgfpathmoveto{\pgfqpoint{4.763648in}{0.521603in}}%
\pgfpathlineto{\pgfqpoint{4.763648in}{4.756603in}}%
\pgfusepath{stroke}%
\end{pgfscope}%
\begin{pgfscope}%
\pgfsetbuttcap%
\pgfsetroundjoin%
\definecolor{currentfill}{rgb}{0.000000,0.000000,0.000000}%
\pgfsetfillcolor{currentfill}%
\pgfsetlinewidth{0.803000pt}%
\definecolor{currentstroke}{rgb}{0.000000,0.000000,0.000000}%
\pgfsetstrokecolor{currentstroke}%
\pgfsetdash{}{0pt}%
\pgfsys@defobject{currentmarker}{\pgfqpoint{0.000000in}{-0.048611in}}{\pgfqpoint{0.000000in}{0.000000in}}{%
\pgfpathmoveto{\pgfqpoint{0.000000in}{0.000000in}}%
\pgfpathlineto{\pgfqpoint{0.000000in}{-0.048611in}}%
\pgfusepath{stroke,fill}%
}%
\begin{pgfscope}%
\pgfsys@transformshift{4.763648in}{0.521603in}%
\pgfsys@useobject{currentmarker}{}%
\end{pgfscope}%
\end{pgfscope}%
\begin{pgfscope}%
\definecolor{textcolor}{rgb}{0.000000,0.000000,0.000000}%
\pgfsetstrokecolor{textcolor}%
\pgfsetfillcolor{textcolor}%
\pgftext[x=4.763648in,y=0.424381in,,top]{\color{textcolor}\sffamily\fontsize{10.000000}{12.000000}\selectfont 0.4}%
\end{pgfscope}%
\begin{pgfscope}%
\pgfpathrectangle{\pgfqpoint{0.728688in}{0.521603in}}{\pgfqpoint{9.687500in}{4.235000in}}%
\pgfusepath{clip}%
\pgfsetrectcap%
\pgfsetroundjoin%
\pgfsetlinewidth{0.803000pt}%
\definecolor{currentstroke}{rgb}{0.690196,0.690196,0.690196}%
\pgfsetstrokecolor{currentstroke}%
\pgfsetdash{}{0pt}%
\pgfpathmoveto{\pgfqpoint{6.560958in}{0.521603in}}%
\pgfpathlineto{\pgfqpoint{6.560958in}{4.756603in}}%
\pgfusepath{stroke}%
\end{pgfscope}%
\begin{pgfscope}%
\pgfsetbuttcap%
\pgfsetroundjoin%
\definecolor{currentfill}{rgb}{0.000000,0.000000,0.000000}%
\pgfsetfillcolor{currentfill}%
\pgfsetlinewidth{0.803000pt}%
\definecolor{currentstroke}{rgb}{0.000000,0.000000,0.000000}%
\pgfsetstrokecolor{currentstroke}%
\pgfsetdash{}{0pt}%
\pgfsys@defobject{currentmarker}{\pgfqpoint{0.000000in}{-0.048611in}}{\pgfqpoint{0.000000in}{0.000000in}}{%
\pgfpathmoveto{\pgfqpoint{0.000000in}{0.000000in}}%
\pgfpathlineto{\pgfqpoint{0.000000in}{-0.048611in}}%
\pgfusepath{stroke,fill}%
}%
\begin{pgfscope}%
\pgfsys@transformshift{6.560958in}{0.521603in}%
\pgfsys@useobject{currentmarker}{}%
\end{pgfscope}%
\end{pgfscope}%
\begin{pgfscope}%
\definecolor{textcolor}{rgb}{0.000000,0.000000,0.000000}%
\pgfsetstrokecolor{textcolor}%
\pgfsetfillcolor{textcolor}%
\pgftext[x=6.560958in,y=0.424381in,,top]{\color{textcolor}\sffamily\fontsize{10.000000}{12.000000}\selectfont 0.6}%
\end{pgfscope}%
\begin{pgfscope}%
\pgfpathrectangle{\pgfqpoint{0.728688in}{0.521603in}}{\pgfqpoint{9.687500in}{4.235000in}}%
\pgfusepath{clip}%
\pgfsetrectcap%
\pgfsetroundjoin%
\pgfsetlinewidth{0.803000pt}%
\definecolor{currentstroke}{rgb}{0.690196,0.690196,0.690196}%
\pgfsetstrokecolor{currentstroke}%
\pgfsetdash{}{0pt}%
\pgfpathmoveto{\pgfqpoint{8.358268in}{0.521603in}}%
\pgfpathlineto{\pgfqpoint{8.358268in}{4.756603in}}%
\pgfusepath{stroke}%
\end{pgfscope}%
\begin{pgfscope}%
\pgfsetbuttcap%
\pgfsetroundjoin%
\definecolor{currentfill}{rgb}{0.000000,0.000000,0.000000}%
\pgfsetfillcolor{currentfill}%
\pgfsetlinewidth{0.803000pt}%
\definecolor{currentstroke}{rgb}{0.000000,0.000000,0.000000}%
\pgfsetstrokecolor{currentstroke}%
\pgfsetdash{}{0pt}%
\pgfsys@defobject{currentmarker}{\pgfqpoint{0.000000in}{-0.048611in}}{\pgfqpoint{0.000000in}{0.000000in}}{%
\pgfpathmoveto{\pgfqpoint{0.000000in}{0.000000in}}%
\pgfpathlineto{\pgfqpoint{0.000000in}{-0.048611in}}%
\pgfusepath{stroke,fill}%
}%
\begin{pgfscope}%
\pgfsys@transformshift{8.358268in}{0.521603in}%
\pgfsys@useobject{currentmarker}{}%
\end{pgfscope}%
\end{pgfscope}%
\begin{pgfscope}%
\definecolor{textcolor}{rgb}{0.000000,0.000000,0.000000}%
\pgfsetstrokecolor{textcolor}%
\pgfsetfillcolor{textcolor}%
\pgftext[x=8.358268in,y=0.424381in,,top]{\color{textcolor}\sffamily\fontsize{10.000000}{12.000000}\selectfont 0.8}%
\end{pgfscope}%
\begin{pgfscope}%
\pgfpathrectangle{\pgfqpoint{0.728688in}{0.521603in}}{\pgfqpoint{9.687500in}{4.235000in}}%
\pgfusepath{clip}%
\pgfsetrectcap%
\pgfsetroundjoin%
\pgfsetlinewidth{0.803000pt}%
\definecolor{currentstroke}{rgb}{0.690196,0.690196,0.690196}%
\pgfsetstrokecolor{currentstroke}%
\pgfsetdash{}{0pt}%
\pgfpathmoveto{\pgfqpoint{10.155578in}{0.521603in}}%
\pgfpathlineto{\pgfqpoint{10.155578in}{4.756603in}}%
\pgfusepath{stroke}%
\end{pgfscope}%
\begin{pgfscope}%
\pgfsetbuttcap%
\pgfsetroundjoin%
\definecolor{currentfill}{rgb}{0.000000,0.000000,0.000000}%
\pgfsetfillcolor{currentfill}%
\pgfsetlinewidth{0.803000pt}%
\definecolor{currentstroke}{rgb}{0.000000,0.000000,0.000000}%
\pgfsetstrokecolor{currentstroke}%
\pgfsetdash{}{0pt}%
\pgfsys@defobject{currentmarker}{\pgfqpoint{0.000000in}{-0.048611in}}{\pgfqpoint{0.000000in}{0.000000in}}{%
\pgfpathmoveto{\pgfqpoint{0.000000in}{0.000000in}}%
\pgfpathlineto{\pgfqpoint{0.000000in}{-0.048611in}}%
\pgfusepath{stroke,fill}%
}%
\begin{pgfscope}%
\pgfsys@transformshift{10.155578in}{0.521603in}%
\pgfsys@useobject{currentmarker}{}%
\end{pgfscope}%
\end{pgfscope}%
\begin{pgfscope}%
\definecolor{textcolor}{rgb}{0.000000,0.000000,0.000000}%
\pgfsetstrokecolor{textcolor}%
\pgfsetfillcolor{textcolor}%
\pgftext[x=10.155578in,y=0.424381in,,top]{\color{textcolor}\sffamily\fontsize{10.000000}{12.000000}\selectfont 1.0}%
\end{pgfscope}%
\begin{pgfscope}%
\pgfpathrectangle{\pgfqpoint{0.728688in}{0.521603in}}{\pgfqpoint{9.687500in}{4.235000in}}%
\pgfusepath{clip}%
\pgfsetrectcap%
\pgfsetroundjoin%
\pgfsetlinewidth{0.803000pt}%
\definecolor{currentstroke}{rgb}{0.600000,0.600000,0.600000}%
\pgfsetstrokecolor{currentstroke}%
\pgfsetstrokeopacity{0.200000}%
\pgfsetdash{}{0pt}%
\pgfpathmoveto{\pgfqpoint{1.618356in}{0.521603in}}%
\pgfpathlineto{\pgfqpoint{1.618356in}{4.756603in}}%
\pgfusepath{stroke}%
\end{pgfscope}%
\begin{pgfscope}%
\pgfsetbuttcap%
\pgfsetroundjoin%
\definecolor{currentfill}{rgb}{0.000000,0.000000,0.000000}%
\pgfsetfillcolor{currentfill}%
\pgfsetlinewidth{0.602250pt}%
\definecolor{currentstroke}{rgb}{0.000000,0.000000,0.000000}%
\pgfsetstrokecolor{currentstroke}%
\pgfsetdash{}{0pt}%
\pgfsys@defobject{currentmarker}{\pgfqpoint{0.000000in}{-0.027778in}}{\pgfqpoint{0.000000in}{0.000000in}}{%
\pgfpathmoveto{\pgfqpoint{0.000000in}{0.000000in}}%
\pgfpathlineto{\pgfqpoint{0.000000in}{-0.027778in}}%
\pgfusepath{stroke,fill}%
}%
\begin{pgfscope}%
\pgfsys@transformshift{1.618356in}{0.521603in}%
\pgfsys@useobject{currentmarker}{}%
\end{pgfscope}%
\end{pgfscope}%
\begin{pgfscope}%
\pgfpathrectangle{\pgfqpoint{0.728688in}{0.521603in}}{\pgfqpoint{9.687500in}{4.235000in}}%
\pgfusepath{clip}%
\pgfsetrectcap%
\pgfsetroundjoin%
\pgfsetlinewidth{0.803000pt}%
\definecolor{currentstroke}{rgb}{0.600000,0.600000,0.600000}%
\pgfsetstrokecolor{currentstroke}%
\pgfsetstrokeopacity{0.200000}%
\pgfsetdash{}{0pt}%
\pgfpathmoveto{\pgfqpoint{2.067684in}{0.521603in}}%
\pgfpathlineto{\pgfqpoint{2.067684in}{4.756603in}}%
\pgfusepath{stroke}%
\end{pgfscope}%
\begin{pgfscope}%
\pgfsetbuttcap%
\pgfsetroundjoin%
\definecolor{currentfill}{rgb}{0.000000,0.000000,0.000000}%
\pgfsetfillcolor{currentfill}%
\pgfsetlinewidth{0.602250pt}%
\definecolor{currentstroke}{rgb}{0.000000,0.000000,0.000000}%
\pgfsetstrokecolor{currentstroke}%
\pgfsetdash{}{0pt}%
\pgfsys@defobject{currentmarker}{\pgfqpoint{0.000000in}{-0.027778in}}{\pgfqpoint{0.000000in}{0.000000in}}{%
\pgfpathmoveto{\pgfqpoint{0.000000in}{0.000000in}}%
\pgfpathlineto{\pgfqpoint{0.000000in}{-0.027778in}}%
\pgfusepath{stroke,fill}%
}%
\begin{pgfscope}%
\pgfsys@transformshift{2.067684in}{0.521603in}%
\pgfsys@useobject{currentmarker}{}%
\end{pgfscope}%
\end{pgfscope}%
\begin{pgfscope}%
\pgfpathrectangle{\pgfqpoint{0.728688in}{0.521603in}}{\pgfqpoint{9.687500in}{4.235000in}}%
\pgfusepath{clip}%
\pgfsetrectcap%
\pgfsetroundjoin%
\pgfsetlinewidth{0.803000pt}%
\definecolor{currentstroke}{rgb}{0.600000,0.600000,0.600000}%
\pgfsetstrokecolor{currentstroke}%
\pgfsetstrokeopacity{0.200000}%
\pgfsetdash{}{0pt}%
\pgfpathmoveto{\pgfqpoint{2.517011in}{0.521603in}}%
\pgfpathlineto{\pgfqpoint{2.517011in}{4.756603in}}%
\pgfusepath{stroke}%
\end{pgfscope}%
\begin{pgfscope}%
\pgfsetbuttcap%
\pgfsetroundjoin%
\definecolor{currentfill}{rgb}{0.000000,0.000000,0.000000}%
\pgfsetfillcolor{currentfill}%
\pgfsetlinewidth{0.602250pt}%
\definecolor{currentstroke}{rgb}{0.000000,0.000000,0.000000}%
\pgfsetstrokecolor{currentstroke}%
\pgfsetdash{}{0pt}%
\pgfsys@defobject{currentmarker}{\pgfqpoint{0.000000in}{-0.027778in}}{\pgfqpoint{0.000000in}{0.000000in}}{%
\pgfpathmoveto{\pgfqpoint{0.000000in}{0.000000in}}%
\pgfpathlineto{\pgfqpoint{0.000000in}{-0.027778in}}%
\pgfusepath{stroke,fill}%
}%
\begin{pgfscope}%
\pgfsys@transformshift{2.517011in}{0.521603in}%
\pgfsys@useobject{currentmarker}{}%
\end{pgfscope}%
\end{pgfscope}%
\begin{pgfscope}%
\pgfpathrectangle{\pgfqpoint{0.728688in}{0.521603in}}{\pgfqpoint{9.687500in}{4.235000in}}%
\pgfusepath{clip}%
\pgfsetrectcap%
\pgfsetroundjoin%
\pgfsetlinewidth{0.803000pt}%
\definecolor{currentstroke}{rgb}{0.600000,0.600000,0.600000}%
\pgfsetstrokecolor{currentstroke}%
\pgfsetstrokeopacity{0.200000}%
\pgfsetdash{}{0pt}%
\pgfpathmoveto{\pgfqpoint{3.415666in}{0.521603in}}%
\pgfpathlineto{\pgfqpoint{3.415666in}{4.756603in}}%
\pgfusepath{stroke}%
\end{pgfscope}%
\begin{pgfscope}%
\pgfsetbuttcap%
\pgfsetroundjoin%
\definecolor{currentfill}{rgb}{0.000000,0.000000,0.000000}%
\pgfsetfillcolor{currentfill}%
\pgfsetlinewidth{0.602250pt}%
\definecolor{currentstroke}{rgb}{0.000000,0.000000,0.000000}%
\pgfsetstrokecolor{currentstroke}%
\pgfsetdash{}{0pt}%
\pgfsys@defobject{currentmarker}{\pgfqpoint{0.000000in}{-0.027778in}}{\pgfqpoint{0.000000in}{0.000000in}}{%
\pgfpathmoveto{\pgfqpoint{0.000000in}{0.000000in}}%
\pgfpathlineto{\pgfqpoint{0.000000in}{-0.027778in}}%
\pgfusepath{stroke,fill}%
}%
\begin{pgfscope}%
\pgfsys@transformshift{3.415666in}{0.521603in}%
\pgfsys@useobject{currentmarker}{}%
\end{pgfscope}%
\end{pgfscope}%
\begin{pgfscope}%
\pgfpathrectangle{\pgfqpoint{0.728688in}{0.521603in}}{\pgfqpoint{9.687500in}{4.235000in}}%
\pgfusepath{clip}%
\pgfsetrectcap%
\pgfsetroundjoin%
\pgfsetlinewidth{0.803000pt}%
\definecolor{currentstroke}{rgb}{0.600000,0.600000,0.600000}%
\pgfsetstrokecolor{currentstroke}%
\pgfsetstrokeopacity{0.200000}%
\pgfsetdash{}{0pt}%
\pgfpathmoveto{\pgfqpoint{3.864994in}{0.521603in}}%
\pgfpathlineto{\pgfqpoint{3.864994in}{4.756603in}}%
\pgfusepath{stroke}%
\end{pgfscope}%
\begin{pgfscope}%
\pgfsetbuttcap%
\pgfsetroundjoin%
\definecolor{currentfill}{rgb}{0.000000,0.000000,0.000000}%
\pgfsetfillcolor{currentfill}%
\pgfsetlinewidth{0.602250pt}%
\definecolor{currentstroke}{rgb}{0.000000,0.000000,0.000000}%
\pgfsetstrokecolor{currentstroke}%
\pgfsetdash{}{0pt}%
\pgfsys@defobject{currentmarker}{\pgfqpoint{0.000000in}{-0.027778in}}{\pgfqpoint{0.000000in}{0.000000in}}{%
\pgfpathmoveto{\pgfqpoint{0.000000in}{0.000000in}}%
\pgfpathlineto{\pgfqpoint{0.000000in}{-0.027778in}}%
\pgfusepath{stroke,fill}%
}%
\begin{pgfscope}%
\pgfsys@transformshift{3.864994in}{0.521603in}%
\pgfsys@useobject{currentmarker}{}%
\end{pgfscope}%
\end{pgfscope}%
\begin{pgfscope}%
\pgfpathrectangle{\pgfqpoint{0.728688in}{0.521603in}}{\pgfqpoint{9.687500in}{4.235000in}}%
\pgfusepath{clip}%
\pgfsetrectcap%
\pgfsetroundjoin%
\pgfsetlinewidth{0.803000pt}%
\definecolor{currentstroke}{rgb}{0.600000,0.600000,0.600000}%
\pgfsetstrokecolor{currentstroke}%
\pgfsetstrokeopacity{0.200000}%
\pgfsetdash{}{0pt}%
\pgfpathmoveto{\pgfqpoint{4.314321in}{0.521603in}}%
\pgfpathlineto{\pgfqpoint{4.314321in}{4.756603in}}%
\pgfusepath{stroke}%
\end{pgfscope}%
\begin{pgfscope}%
\pgfsetbuttcap%
\pgfsetroundjoin%
\definecolor{currentfill}{rgb}{0.000000,0.000000,0.000000}%
\pgfsetfillcolor{currentfill}%
\pgfsetlinewidth{0.602250pt}%
\definecolor{currentstroke}{rgb}{0.000000,0.000000,0.000000}%
\pgfsetstrokecolor{currentstroke}%
\pgfsetdash{}{0pt}%
\pgfsys@defobject{currentmarker}{\pgfqpoint{0.000000in}{-0.027778in}}{\pgfqpoint{0.000000in}{0.000000in}}{%
\pgfpathmoveto{\pgfqpoint{0.000000in}{0.000000in}}%
\pgfpathlineto{\pgfqpoint{0.000000in}{-0.027778in}}%
\pgfusepath{stroke,fill}%
}%
\begin{pgfscope}%
\pgfsys@transformshift{4.314321in}{0.521603in}%
\pgfsys@useobject{currentmarker}{}%
\end{pgfscope}%
\end{pgfscope}%
\begin{pgfscope}%
\pgfpathrectangle{\pgfqpoint{0.728688in}{0.521603in}}{\pgfqpoint{9.687500in}{4.235000in}}%
\pgfusepath{clip}%
\pgfsetrectcap%
\pgfsetroundjoin%
\pgfsetlinewidth{0.803000pt}%
\definecolor{currentstroke}{rgb}{0.600000,0.600000,0.600000}%
\pgfsetstrokecolor{currentstroke}%
\pgfsetstrokeopacity{0.200000}%
\pgfsetdash{}{0pt}%
\pgfpathmoveto{\pgfqpoint{5.212976in}{0.521603in}}%
\pgfpathlineto{\pgfqpoint{5.212976in}{4.756603in}}%
\pgfusepath{stroke}%
\end{pgfscope}%
\begin{pgfscope}%
\pgfsetbuttcap%
\pgfsetroundjoin%
\definecolor{currentfill}{rgb}{0.000000,0.000000,0.000000}%
\pgfsetfillcolor{currentfill}%
\pgfsetlinewidth{0.602250pt}%
\definecolor{currentstroke}{rgb}{0.000000,0.000000,0.000000}%
\pgfsetstrokecolor{currentstroke}%
\pgfsetdash{}{0pt}%
\pgfsys@defobject{currentmarker}{\pgfqpoint{0.000000in}{-0.027778in}}{\pgfqpoint{0.000000in}{0.000000in}}{%
\pgfpathmoveto{\pgfqpoint{0.000000in}{0.000000in}}%
\pgfpathlineto{\pgfqpoint{0.000000in}{-0.027778in}}%
\pgfusepath{stroke,fill}%
}%
\begin{pgfscope}%
\pgfsys@transformshift{5.212976in}{0.521603in}%
\pgfsys@useobject{currentmarker}{}%
\end{pgfscope}%
\end{pgfscope}%
\begin{pgfscope}%
\pgfpathrectangle{\pgfqpoint{0.728688in}{0.521603in}}{\pgfqpoint{9.687500in}{4.235000in}}%
\pgfusepath{clip}%
\pgfsetrectcap%
\pgfsetroundjoin%
\pgfsetlinewidth{0.803000pt}%
\definecolor{currentstroke}{rgb}{0.600000,0.600000,0.600000}%
\pgfsetstrokecolor{currentstroke}%
\pgfsetstrokeopacity{0.200000}%
\pgfsetdash{}{0pt}%
\pgfpathmoveto{\pgfqpoint{5.662303in}{0.521603in}}%
\pgfpathlineto{\pgfqpoint{5.662303in}{4.756603in}}%
\pgfusepath{stroke}%
\end{pgfscope}%
\begin{pgfscope}%
\pgfsetbuttcap%
\pgfsetroundjoin%
\definecolor{currentfill}{rgb}{0.000000,0.000000,0.000000}%
\pgfsetfillcolor{currentfill}%
\pgfsetlinewidth{0.602250pt}%
\definecolor{currentstroke}{rgb}{0.000000,0.000000,0.000000}%
\pgfsetstrokecolor{currentstroke}%
\pgfsetdash{}{0pt}%
\pgfsys@defobject{currentmarker}{\pgfqpoint{0.000000in}{-0.027778in}}{\pgfqpoint{0.000000in}{0.000000in}}{%
\pgfpathmoveto{\pgfqpoint{0.000000in}{0.000000in}}%
\pgfpathlineto{\pgfqpoint{0.000000in}{-0.027778in}}%
\pgfusepath{stroke,fill}%
}%
\begin{pgfscope}%
\pgfsys@transformshift{5.662303in}{0.521603in}%
\pgfsys@useobject{currentmarker}{}%
\end{pgfscope}%
\end{pgfscope}%
\begin{pgfscope}%
\pgfpathrectangle{\pgfqpoint{0.728688in}{0.521603in}}{\pgfqpoint{9.687500in}{4.235000in}}%
\pgfusepath{clip}%
\pgfsetrectcap%
\pgfsetroundjoin%
\pgfsetlinewidth{0.803000pt}%
\definecolor{currentstroke}{rgb}{0.600000,0.600000,0.600000}%
\pgfsetstrokecolor{currentstroke}%
\pgfsetstrokeopacity{0.200000}%
\pgfsetdash{}{0pt}%
\pgfpathmoveto{\pgfqpoint{6.111631in}{0.521603in}}%
\pgfpathlineto{\pgfqpoint{6.111631in}{4.756603in}}%
\pgfusepath{stroke}%
\end{pgfscope}%
\begin{pgfscope}%
\pgfsetbuttcap%
\pgfsetroundjoin%
\definecolor{currentfill}{rgb}{0.000000,0.000000,0.000000}%
\pgfsetfillcolor{currentfill}%
\pgfsetlinewidth{0.602250pt}%
\definecolor{currentstroke}{rgb}{0.000000,0.000000,0.000000}%
\pgfsetstrokecolor{currentstroke}%
\pgfsetdash{}{0pt}%
\pgfsys@defobject{currentmarker}{\pgfqpoint{0.000000in}{-0.027778in}}{\pgfqpoint{0.000000in}{0.000000in}}{%
\pgfpathmoveto{\pgfqpoint{0.000000in}{0.000000in}}%
\pgfpathlineto{\pgfqpoint{0.000000in}{-0.027778in}}%
\pgfusepath{stroke,fill}%
}%
\begin{pgfscope}%
\pgfsys@transformshift{6.111631in}{0.521603in}%
\pgfsys@useobject{currentmarker}{}%
\end{pgfscope}%
\end{pgfscope}%
\begin{pgfscope}%
\pgfpathrectangle{\pgfqpoint{0.728688in}{0.521603in}}{\pgfqpoint{9.687500in}{4.235000in}}%
\pgfusepath{clip}%
\pgfsetrectcap%
\pgfsetroundjoin%
\pgfsetlinewidth{0.803000pt}%
\definecolor{currentstroke}{rgb}{0.600000,0.600000,0.600000}%
\pgfsetstrokecolor{currentstroke}%
\pgfsetstrokeopacity{0.200000}%
\pgfsetdash{}{0pt}%
\pgfpathmoveto{\pgfqpoint{7.010286in}{0.521603in}}%
\pgfpathlineto{\pgfqpoint{7.010286in}{4.756603in}}%
\pgfusepath{stroke}%
\end{pgfscope}%
\begin{pgfscope}%
\pgfsetbuttcap%
\pgfsetroundjoin%
\definecolor{currentfill}{rgb}{0.000000,0.000000,0.000000}%
\pgfsetfillcolor{currentfill}%
\pgfsetlinewidth{0.602250pt}%
\definecolor{currentstroke}{rgb}{0.000000,0.000000,0.000000}%
\pgfsetstrokecolor{currentstroke}%
\pgfsetdash{}{0pt}%
\pgfsys@defobject{currentmarker}{\pgfqpoint{0.000000in}{-0.027778in}}{\pgfqpoint{0.000000in}{0.000000in}}{%
\pgfpathmoveto{\pgfqpoint{0.000000in}{0.000000in}}%
\pgfpathlineto{\pgfqpoint{0.000000in}{-0.027778in}}%
\pgfusepath{stroke,fill}%
}%
\begin{pgfscope}%
\pgfsys@transformshift{7.010286in}{0.521603in}%
\pgfsys@useobject{currentmarker}{}%
\end{pgfscope}%
\end{pgfscope}%
\begin{pgfscope}%
\pgfpathrectangle{\pgfqpoint{0.728688in}{0.521603in}}{\pgfqpoint{9.687500in}{4.235000in}}%
\pgfusepath{clip}%
\pgfsetrectcap%
\pgfsetroundjoin%
\pgfsetlinewidth{0.803000pt}%
\definecolor{currentstroke}{rgb}{0.600000,0.600000,0.600000}%
\pgfsetstrokecolor{currentstroke}%
\pgfsetstrokeopacity{0.200000}%
\pgfsetdash{}{0pt}%
\pgfpathmoveto{\pgfqpoint{7.459613in}{0.521603in}}%
\pgfpathlineto{\pgfqpoint{7.459613in}{4.756603in}}%
\pgfusepath{stroke}%
\end{pgfscope}%
\begin{pgfscope}%
\pgfsetbuttcap%
\pgfsetroundjoin%
\definecolor{currentfill}{rgb}{0.000000,0.000000,0.000000}%
\pgfsetfillcolor{currentfill}%
\pgfsetlinewidth{0.602250pt}%
\definecolor{currentstroke}{rgb}{0.000000,0.000000,0.000000}%
\pgfsetstrokecolor{currentstroke}%
\pgfsetdash{}{0pt}%
\pgfsys@defobject{currentmarker}{\pgfqpoint{0.000000in}{-0.027778in}}{\pgfqpoint{0.000000in}{0.000000in}}{%
\pgfpathmoveto{\pgfqpoint{0.000000in}{0.000000in}}%
\pgfpathlineto{\pgfqpoint{0.000000in}{-0.027778in}}%
\pgfusepath{stroke,fill}%
}%
\begin{pgfscope}%
\pgfsys@transformshift{7.459613in}{0.521603in}%
\pgfsys@useobject{currentmarker}{}%
\end{pgfscope}%
\end{pgfscope}%
\begin{pgfscope}%
\pgfpathrectangle{\pgfqpoint{0.728688in}{0.521603in}}{\pgfqpoint{9.687500in}{4.235000in}}%
\pgfusepath{clip}%
\pgfsetrectcap%
\pgfsetroundjoin%
\pgfsetlinewidth{0.803000pt}%
\definecolor{currentstroke}{rgb}{0.600000,0.600000,0.600000}%
\pgfsetstrokecolor{currentstroke}%
\pgfsetstrokeopacity{0.200000}%
\pgfsetdash{}{0pt}%
\pgfpathmoveto{\pgfqpoint{7.908941in}{0.521603in}}%
\pgfpathlineto{\pgfqpoint{7.908941in}{4.756603in}}%
\pgfusepath{stroke}%
\end{pgfscope}%
\begin{pgfscope}%
\pgfsetbuttcap%
\pgfsetroundjoin%
\definecolor{currentfill}{rgb}{0.000000,0.000000,0.000000}%
\pgfsetfillcolor{currentfill}%
\pgfsetlinewidth{0.602250pt}%
\definecolor{currentstroke}{rgb}{0.000000,0.000000,0.000000}%
\pgfsetstrokecolor{currentstroke}%
\pgfsetdash{}{0pt}%
\pgfsys@defobject{currentmarker}{\pgfqpoint{0.000000in}{-0.027778in}}{\pgfqpoint{0.000000in}{0.000000in}}{%
\pgfpathmoveto{\pgfqpoint{0.000000in}{0.000000in}}%
\pgfpathlineto{\pgfqpoint{0.000000in}{-0.027778in}}%
\pgfusepath{stroke,fill}%
}%
\begin{pgfscope}%
\pgfsys@transformshift{7.908941in}{0.521603in}%
\pgfsys@useobject{currentmarker}{}%
\end{pgfscope}%
\end{pgfscope}%
\begin{pgfscope}%
\pgfpathrectangle{\pgfqpoint{0.728688in}{0.521603in}}{\pgfqpoint{9.687500in}{4.235000in}}%
\pgfusepath{clip}%
\pgfsetrectcap%
\pgfsetroundjoin%
\pgfsetlinewidth{0.803000pt}%
\definecolor{currentstroke}{rgb}{0.600000,0.600000,0.600000}%
\pgfsetstrokecolor{currentstroke}%
\pgfsetstrokeopacity{0.200000}%
\pgfsetdash{}{0pt}%
\pgfpathmoveto{\pgfqpoint{8.807596in}{0.521603in}}%
\pgfpathlineto{\pgfqpoint{8.807596in}{4.756603in}}%
\pgfusepath{stroke}%
\end{pgfscope}%
\begin{pgfscope}%
\pgfsetbuttcap%
\pgfsetroundjoin%
\definecolor{currentfill}{rgb}{0.000000,0.000000,0.000000}%
\pgfsetfillcolor{currentfill}%
\pgfsetlinewidth{0.602250pt}%
\definecolor{currentstroke}{rgb}{0.000000,0.000000,0.000000}%
\pgfsetstrokecolor{currentstroke}%
\pgfsetdash{}{0pt}%
\pgfsys@defobject{currentmarker}{\pgfqpoint{0.000000in}{-0.027778in}}{\pgfqpoint{0.000000in}{0.000000in}}{%
\pgfpathmoveto{\pgfqpoint{0.000000in}{0.000000in}}%
\pgfpathlineto{\pgfqpoint{0.000000in}{-0.027778in}}%
\pgfusepath{stroke,fill}%
}%
\begin{pgfscope}%
\pgfsys@transformshift{8.807596in}{0.521603in}%
\pgfsys@useobject{currentmarker}{}%
\end{pgfscope}%
\end{pgfscope}%
\begin{pgfscope}%
\pgfpathrectangle{\pgfqpoint{0.728688in}{0.521603in}}{\pgfqpoint{9.687500in}{4.235000in}}%
\pgfusepath{clip}%
\pgfsetrectcap%
\pgfsetroundjoin%
\pgfsetlinewidth{0.803000pt}%
\definecolor{currentstroke}{rgb}{0.600000,0.600000,0.600000}%
\pgfsetstrokecolor{currentstroke}%
\pgfsetstrokeopacity{0.200000}%
\pgfsetdash{}{0pt}%
\pgfpathmoveto{\pgfqpoint{9.256923in}{0.521603in}}%
\pgfpathlineto{\pgfqpoint{9.256923in}{4.756603in}}%
\pgfusepath{stroke}%
\end{pgfscope}%
\begin{pgfscope}%
\pgfsetbuttcap%
\pgfsetroundjoin%
\definecolor{currentfill}{rgb}{0.000000,0.000000,0.000000}%
\pgfsetfillcolor{currentfill}%
\pgfsetlinewidth{0.602250pt}%
\definecolor{currentstroke}{rgb}{0.000000,0.000000,0.000000}%
\pgfsetstrokecolor{currentstroke}%
\pgfsetdash{}{0pt}%
\pgfsys@defobject{currentmarker}{\pgfqpoint{0.000000in}{-0.027778in}}{\pgfqpoint{0.000000in}{0.000000in}}{%
\pgfpathmoveto{\pgfqpoint{0.000000in}{0.000000in}}%
\pgfpathlineto{\pgfqpoint{0.000000in}{-0.027778in}}%
\pgfusepath{stroke,fill}%
}%
\begin{pgfscope}%
\pgfsys@transformshift{9.256923in}{0.521603in}%
\pgfsys@useobject{currentmarker}{}%
\end{pgfscope}%
\end{pgfscope}%
\begin{pgfscope}%
\pgfpathrectangle{\pgfqpoint{0.728688in}{0.521603in}}{\pgfqpoint{9.687500in}{4.235000in}}%
\pgfusepath{clip}%
\pgfsetrectcap%
\pgfsetroundjoin%
\pgfsetlinewidth{0.803000pt}%
\definecolor{currentstroke}{rgb}{0.600000,0.600000,0.600000}%
\pgfsetstrokecolor{currentstroke}%
\pgfsetstrokeopacity{0.200000}%
\pgfsetdash{}{0pt}%
\pgfpathmoveto{\pgfqpoint{9.706251in}{0.521603in}}%
\pgfpathlineto{\pgfqpoint{9.706251in}{4.756603in}}%
\pgfusepath{stroke}%
\end{pgfscope}%
\begin{pgfscope}%
\pgfsetbuttcap%
\pgfsetroundjoin%
\definecolor{currentfill}{rgb}{0.000000,0.000000,0.000000}%
\pgfsetfillcolor{currentfill}%
\pgfsetlinewidth{0.602250pt}%
\definecolor{currentstroke}{rgb}{0.000000,0.000000,0.000000}%
\pgfsetstrokecolor{currentstroke}%
\pgfsetdash{}{0pt}%
\pgfsys@defobject{currentmarker}{\pgfqpoint{0.000000in}{-0.027778in}}{\pgfqpoint{0.000000in}{0.000000in}}{%
\pgfpathmoveto{\pgfqpoint{0.000000in}{0.000000in}}%
\pgfpathlineto{\pgfqpoint{0.000000in}{-0.027778in}}%
\pgfusepath{stroke,fill}%
}%
\begin{pgfscope}%
\pgfsys@transformshift{9.706251in}{0.521603in}%
\pgfsys@useobject{currentmarker}{}%
\end{pgfscope}%
\end{pgfscope}%
\begin{pgfscope}%
\definecolor{textcolor}{rgb}{0.000000,0.000000,0.000000}%
\pgfsetstrokecolor{textcolor}%
\pgfsetfillcolor{textcolor}%
\pgftext[x=5.572438in,y=0.234413in,,top]{\color{textcolor}\sffamily\fontsize{10.000000}{12.000000}\selectfont probability \(\displaystyle p_4\) of a vaccinated person \(\displaystyle V\) occupying a grid node}%
\end{pgfscope}%
\begin{pgfscope}%
\pgfpathrectangle{\pgfqpoint{0.728688in}{0.521603in}}{\pgfqpoint{9.687500in}{4.235000in}}%
\pgfusepath{clip}%
\pgfsetrectcap%
\pgfsetroundjoin%
\pgfsetlinewidth{0.803000pt}%
\definecolor{currentstroke}{rgb}{0.690196,0.690196,0.690196}%
\pgfsetstrokecolor{currentstroke}%
\pgfsetdash{}{0pt}%
\pgfpathmoveto{\pgfqpoint{0.728688in}{0.713984in}}%
\pgfpathlineto{\pgfqpoint{10.416188in}{0.713984in}}%
\pgfusepath{stroke}%
\end{pgfscope}%
\begin{pgfscope}%
\pgfsetbuttcap%
\pgfsetroundjoin%
\definecolor{currentfill}{rgb}{0.000000,0.000000,0.000000}%
\pgfsetfillcolor{currentfill}%
\pgfsetlinewidth{0.803000pt}%
\definecolor{currentstroke}{rgb}{0.000000,0.000000,0.000000}%
\pgfsetstrokecolor{currentstroke}%
\pgfsetdash{}{0pt}%
\pgfsys@defobject{currentmarker}{\pgfqpoint{-0.048611in}{0.000000in}}{\pgfqpoint{-0.000000in}{0.000000in}}{%
\pgfpathmoveto{\pgfqpoint{-0.000000in}{0.000000in}}%
\pgfpathlineto{\pgfqpoint{-0.048611in}{0.000000in}}%
\pgfusepath{stroke,fill}%
}%
\begin{pgfscope}%
\pgfsys@transformshift{0.728688in}{0.713984in}%
\pgfsys@useobject{currentmarker}{}%
\end{pgfscope}%
\end{pgfscope}%
\begin{pgfscope}%
\definecolor{textcolor}{rgb}{0.000000,0.000000,0.000000}%
\pgfsetstrokecolor{textcolor}%
\pgfsetfillcolor{textcolor}%
\pgftext[x=0.322221in, y=0.661222in, left, base]{\color{textcolor}\sffamily\fontsize{10.000000}{12.000000}\selectfont 0.00}%
\end{pgfscope}%
\begin{pgfscope}%
\pgfpathrectangle{\pgfqpoint{0.728688in}{0.521603in}}{\pgfqpoint{9.687500in}{4.235000in}}%
\pgfusepath{clip}%
\pgfsetrectcap%
\pgfsetroundjoin%
\pgfsetlinewidth{0.803000pt}%
\definecolor{currentstroke}{rgb}{0.690196,0.690196,0.690196}%
\pgfsetstrokecolor{currentstroke}%
\pgfsetdash{}{0pt}%
\pgfpathmoveto{\pgfqpoint{0.728688in}{1.478749in}}%
\pgfpathlineto{\pgfqpoint{10.416188in}{1.478749in}}%
\pgfusepath{stroke}%
\end{pgfscope}%
\begin{pgfscope}%
\pgfsetbuttcap%
\pgfsetroundjoin%
\definecolor{currentfill}{rgb}{0.000000,0.000000,0.000000}%
\pgfsetfillcolor{currentfill}%
\pgfsetlinewidth{0.803000pt}%
\definecolor{currentstroke}{rgb}{0.000000,0.000000,0.000000}%
\pgfsetstrokecolor{currentstroke}%
\pgfsetdash{}{0pt}%
\pgfsys@defobject{currentmarker}{\pgfqpoint{-0.048611in}{0.000000in}}{\pgfqpoint{-0.000000in}{0.000000in}}{%
\pgfpathmoveto{\pgfqpoint{-0.000000in}{0.000000in}}%
\pgfpathlineto{\pgfqpoint{-0.048611in}{0.000000in}}%
\pgfusepath{stroke,fill}%
}%
\begin{pgfscope}%
\pgfsys@transformshift{0.728688in}{1.478749in}%
\pgfsys@useobject{currentmarker}{}%
\end{pgfscope}%
\end{pgfscope}%
\begin{pgfscope}%
\definecolor{textcolor}{rgb}{0.000000,0.000000,0.000000}%
\pgfsetstrokecolor{textcolor}%
\pgfsetfillcolor{textcolor}%
\pgftext[x=0.322221in, y=1.425988in, left, base]{\color{textcolor}\sffamily\fontsize{10.000000}{12.000000}\selectfont 0.05}%
\end{pgfscope}%
\begin{pgfscope}%
\pgfpathrectangle{\pgfqpoint{0.728688in}{0.521603in}}{\pgfqpoint{9.687500in}{4.235000in}}%
\pgfusepath{clip}%
\pgfsetrectcap%
\pgfsetroundjoin%
\pgfsetlinewidth{0.803000pt}%
\definecolor{currentstroke}{rgb}{0.690196,0.690196,0.690196}%
\pgfsetstrokecolor{currentstroke}%
\pgfsetdash{}{0pt}%
\pgfpathmoveto{\pgfqpoint{0.728688in}{2.243515in}}%
\pgfpathlineto{\pgfqpoint{10.416188in}{2.243515in}}%
\pgfusepath{stroke}%
\end{pgfscope}%
\begin{pgfscope}%
\pgfsetbuttcap%
\pgfsetroundjoin%
\definecolor{currentfill}{rgb}{0.000000,0.000000,0.000000}%
\pgfsetfillcolor{currentfill}%
\pgfsetlinewidth{0.803000pt}%
\definecolor{currentstroke}{rgb}{0.000000,0.000000,0.000000}%
\pgfsetstrokecolor{currentstroke}%
\pgfsetdash{}{0pt}%
\pgfsys@defobject{currentmarker}{\pgfqpoint{-0.048611in}{0.000000in}}{\pgfqpoint{-0.000000in}{0.000000in}}{%
\pgfpathmoveto{\pgfqpoint{-0.000000in}{0.000000in}}%
\pgfpathlineto{\pgfqpoint{-0.048611in}{0.000000in}}%
\pgfusepath{stroke,fill}%
}%
\begin{pgfscope}%
\pgfsys@transformshift{0.728688in}{2.243515in}%
\pgfsys@useobject{currentmarker}{}%
\end{pgfscope}%
\end{pgfscope}%
\begin{pgfscope}%
\definecolor{textcolor}{rgb}{0.000000,0.000000,0.000000}%
\pgfsetstrokecolor{textcolor}%
\pgfsetfillcolor{textcolor}%
\pgftext[x=0.322221in, y=2.190753in, left, base]{\color{textcolor}\sffamily\fontsize{10.000000}{12.000000}\selectfont 0.10}%
\end{pgfscope}%
\begin{pgfscope}%
\pgfpathrectangle{\pgfqpoint{0.728688in}{0.521603in}}{\pgfqpoint{9.687500in}{4.235000in}}%
\pgfusepath{clip}%
\pgfsetrectcap%
\pgfsetroundjoin%
\pgfsetlinewidth{0.803000pt}%
\definecolor{currentstroke}{rgb}{0.690196,0.690196,0.690196}%
\pgfsetstrokecolor{currentstroke}%
\pgfsetdash{}{0pt}%
\pgfpathmoveto{\pgfqpoint{0.728688in}{3.008280in}}%
\pgfpathlineto{\pgfqpoint{10.416188in}{3.008280in}}%
\pgfusepath{stroke}%
\end{pgfscope}%
\begin{pgfscope}%
\pgfsetbuttcap%
\pgfsetroundjoin%
\definecolor{currentfill}{rgb}{0.000000,0.000000,0.000000}%
\pgfsetfillcolor{currentfill}%
\pgfsetlinewidth{0.803000pt}%
\definecolor{currentstroke}{rgb}{0.000000,0.000000,0.000000}%
\pgfsetstrokecolor{currentstroke}%
\pgfsetdash{}{0pt}%
\pgfsys@defobject{currentmarker}{\pgfqpoint{-0.048611in}{0.000000in}}{\pgfqpoint{-0.000000in}{0.000000in}}{%
\pgfpathmoveto{\pgfqpoint{-0.000000in}{0.000000in}}%
\pgfpathlineto{\pgfqpoint{-0.048611in}{0.000000in}}%
\pgfusepath{stroke,fill}%
}%
\begin{pgfscope}%
\pgfsys@transformshift{0.728688in}{3.008280in}%
\pgfsys@useobject{currentmarker}{}%
\end{pgfscope}%
\end{pgfscope}%
\begin{pgfscope}%
\definecolor{textcolor}{rgb}{0.000000,0.000000,0.000000}%
\pgfsetstrokecolor{textcolor}%
\pgfsetfillcolor{textcolor}%
\pgftext[x=0.322221in, y=2.955518in, left, base]{\color{textcolor}\sffamily\fontsize{10.000000}{12.000000}\selectfont 0.15}%
\end{pgfscope}%
\begin{pgfscope}%
\pgfpathrectangle{\pgfqpoint{0.728688in}{0.521603in}}{\pgfqpoint{9.687500in}{4.235000in}}%
\pgfusepath{clip}%
\pgfsetrectcap%
\pgfsetroundjoin%
\pgfsetlinewidth{0.803000pt}%
\definecolor{currentstroke}{rgb}{0.690196,0.690196,0.690196}%
\pgfsetstrokecolor{currentstroke}%
\pgfsetdash{}{0pt}%
\pgfpathmoveto{\pgfqpoint{0.728688in}{3.773045in}}%
\pgfpathlineto{\pgfqpoint{10.416188in}{3.773045in}}%
\pgfusepath{stroke}%
\end{pgfscope}%
\begin{pgfscope}%
\pgfsetbuttcap%
\pgfsetroundjoin%
\definecolor{currentfill}{rgb}{0.000000,0.000000,0.000000}%
\pgfsetfillcolor{currentfill}%
\pgfsetlinewidth{0.803000pt}%
\definecolor{currentstroke}{rgb}{0.000000,0.000000,0.000000}%
\pgfsetstrokecolor{currentstroke}%
\pgfsetdash{}{0pt}%
\pgfsys@defobject{currentmarker}{\pgfqpoint{-0.048611in}{0.000000in}}{\pgfqpoint{-0.000000in}{0.000000in}}{%
\pgfpathmoveto{\pgfqpoint{-0.000000in}{0.000000in}}%
\pgfpathlineto{\pgfqpoint{-0.048611in}{0.000000in}}%
\pgfusepath{stroke,fill}%
}%
\begin{pgfscope}%
\pgfsys@transformshift{0.728688in}{3.773045in}%
\pgfsys@useobject{currentmarker}{}%
\end{pgfscope}%
\end{pgfscope}%
\begin{pgfscope}%
\definecolor{textcolor}{rgb}{0.000000,0.000000,0.000000}%
\pgfsetstrokecolor{textcolor}%
\pgfsetfillcolor{textcolor}%
\pgftext[x=0.322221in, y=3.720284in, left, base]{\color{textcolor}\sffamily\fontsize{10.000000}{12.000000}\selectfont 0.20}%
\end{pgfscope}%
\begin{pgfscope}%
\pgfpathrectangle{\pgfqpoint{0.728688in}{0.521603in}}{\pgfqpoint{9.687500in}{4.235000in}}%
\pgfusepath{clip}%
\pgfsetrectcap%
\pgfsetroundjoin%
\pgfsetlinewidth{0.803000pt}%
\definecolor{currentstroke}{rgb}{0.690196,0.690196,0.690196}%
\pgfsetstrokecolor{currentstroke}%
\pgfsetdash{}{0pt}%
\pgfpathmoveto{\pgfqpoint{0.728688in}{4.537811in}}%
\pgfpathlineto{\pgfqpoint{10.416188in}{4.537811in}}%
\pgfusepath{stroke}%
\end{pgfscope}%
\begin{pgfscope}%
\pgfsetbuttcap%
\pgfsetroundjoin%
\definecolor{currentfill}{rgb}{0.000000,0.000000,0.000000}%
\pgfsetfillcolor{currentfill}%
\pgfsetlinewidth{0.803000pt}%
\definecolor{currentstroke}{rgb}{0.000000,0.000000,0.000000}%
\pgfsetstrokecolor{currentstroke}%
\pgfsetdash{}{0pt}%
\pgfsys@defobject{currentmarker}{\pgfqpoint{-0.048611in}{0.000000in}}{\pgfqpoint{-0.000000in}{0.000000in}}{%
\pgfpathmoveto{\pgfqpoint{-0.000000in}{0.000000in}}%
\pgfpathlineto{\pgfqpoint{-0.048611in}{0.000000in}}%
\pgfusepath{stroke,fill}%
}%
\begin{pgfscope}%
\pgfsys@transformshift{0.728688in}{4.537811in}%
\pgfsys@useobject{currentmarker}{}%
\end{pgfscope}%
\end{pgfscope}%
\begin{pgfscope}%
\definecolor{textcolor}{rgb}{0.000000,0.000000,0.000000}%
\pgfsetstrokecolor{textcolor}%
\pgfsetfillcolor{textcolor}%
\pgftext[x=0.322221in, y=4.485049in, left, base]{\color{textcolor}\sffamily\fontsize{10.000000}{12.000000}\selectfont 0.25}%
\end{pgfscope}%
\begin{pgfscope}%
\pgfpathrectangle{\pgfqpoint{0.728688in}{0.521603in}}{\pgfqpoint{9.687500in}{4.235000in}}%
\pgfusepath{clip}%
\pgfsetrectcap%
\pgfsetroundjoin%
\pgfsetlinewidth{0.803000pt}%
\definecolor{currentstroke}{rgb}{0.600000,0.600000,0.600000}%
\pgfsetstrokecolor{currentstroke}%
\pgfsetstrokeopacity{0.200000}%
\pgfsetdash{}{0pt}%
\pgfpathmoveto{\pgfqpoint{0.728688in}{0.561031in}}%
\pgfpathlineto{\pgfqpoint{10.416188in}{0.561031in}}%
\pgfusepath{stroke}%
\end{pgfscope}%
\begin{pgfscope}%
\pgfsetbuttcap%
\pgfsetroundjoin%
\definecolor{currentfill}{rgb}{0.000000,0.000000,0.000000}%
\pgfsetfillcolor{currentfill}%
\pgfsetlinewidth{0.602250pt}%
\definecolor{currentstroke}{rgb}{0.000000,0.000000,0.000000}%
\pgfsetstrokecolor{currentstroke}%
\pgfsetdash{}{0pt}%
\pgfsys@defobject{currentmarker}{\pgfqpoint{-0.027778in}{0.000000in}}{\pgfqpoint{-0.000000in}{0.000000in}}{%
\pgfpathmoveto{\pgfqpoint{-0.000000in}{0.000000in}}%
\pgfpathlineto{\pgfqpoint{-0.027778in}{0.000000in}}%
\pgfusepath{stroke,fill}%
}%
\begin{pgfscope}%
\pgfsys@transformshift{0.728688in}{0.561031in}%
\pgfsys@useobject{currentmarker}{}%
\end{pgfscope}%
\end{pgfscope}%
\begin{pgfscope}%
\pgfpathrectangle{\pgfqpoint{0.728688in}{0.521603in}}{\pgfqpoint{9.687500in}{4.235000in}}%
\pgfusepath{clip}%
\pgfsetrectcap%
\pgfsetroundjoin%
\pgfsetlinewidth{0.803000pt}%
\definecolor{currentstroke}{rgb}{0.600000,0.600000,0.600000}%
\pgfsetstrokecolor{currentstroke}%
\pgfsetstrokeopacity{0.200000}%
\pgfsetdash{}{0pt}%
\pgfpathmoveto{\pgfqpoint{0.728688in}{0.866937in}}%
\pgfpathlineto{\pgfqpoint{10.416188in}{0.866937in}}%
\pgfusepath{stroke}%
\end{pgfscope}%
\begin{pgfscope}%
\pgfsetbuttcap%
\pgfsetroundjoin%
\definecolor{currentfill}{rgb}{0.000000,0.000000,0.000000}%
\pgfsetfillcolor{currentfill}%
\pgfsetlinewidth{0.602250pt}%
\definecolor{currentstroke}{rgb}{0.000000,0.000000,0.000000}%
\pgfsetstrokecolor{currentstroke}%
\pgfsetdash{}{0pt}%
\pgfsys@defobject{currentmarker}{\pgfqpoint{-0.027778in}{0.000000in}}{\pgfqpoint{-0.000000in}{0.000000in}}{%
\pgfpathmoveto{\pgfqpoint{-0.000000in}{0.000000in}}%
\pgfpathlineto{\pgfqpoint{-0.027778in}{0.000000in}}%
\pgfusepath{stroke,fill}%
}%
\begin{pgfscope}%
\pgfsys@transformshift{0.728688in}{0.866937in}%
\pgfsys@useobject{currentmarker}{}%
\end{pgfscope}%
\end{pgfscope}%
\begin{pgfscope}%
\pgfpathrectangle{\pgfqpoint{0.728688in}{0.521603in}}{\pgfqpoint{9.687500in}{4.235000in}}%
\pgfusepath{clip}%
\pgfsetrectcap%
\pgfsetroundjoin%
\pgfsetlinewidth{0.803000pt}%
\definecolor{currentstroke}{rgb}{0.600000,0.600000,0.600000}%
\pgfsetstrokecolor{currentstroke}%
\pgfsetstrokeopacity{0.200000}%
\pgfsetdash{}{0pt}%
\pgfpathmoveto{\pgfqpoint{0.728688in}{1.019890in}}%
\pgfpathlineto{\pgfqpoint{10.416188in}{1.019890in}}%
\pgfusepath{stroke}%
\end{pgfscope}%
\begin{pgfscope}%
\pgfsetbuttcap%
\pgfsetroundjoin%
\definecolor{currentfill}{rgb}{0.000000,0.000000,0.000000}%
\pgfsetfillcolor{currentfill}%
\pgfsetlinewidth{0.602250pt}%
\definecolor{currentstroke}{rgb}{0.000000,0.000000,0.000000}%
\pgfsetstrokecolor{currentstroke}%
\pgfsetdash{}{0pt}%
\pgfsys@defobject{currentmarker}{\pgfqpoint{-0.027778in}{0.000000in}}{\pgfqpoint{-0.000000in}{0.000000in}}{%
\pgfpathmoveto{\pgfqpoint{-0.000000in}{0.000000in}}%
\pgfpathlineto{\pgfqpoint{-0.027778in}{0.000000in}}%
\pgfusepath{stroke,fill}%
}%
\begin{pgfscope}%
\pgfsys@transformshift{0.728688in}{1.019890in}%
\pgfsys@useobject{currentmarker}{}%
\end{pgfscope}%
\end{pgfscope}%
\begin{pgfscope}%
\pgfpathrectangle{\pgfqpoint{0.728688in}{0.521603in}}{\pgfqpoint{9.687500in}{4.235000in}}%
\pgfusepath{clip}%
\pgfsetrectcap%
\pgfsetroundjoin%
\pgfsetlinewidth{0.803000pt}%
\definecolor{currentstroke}{rgb}{0.600000,0.600000,0.600000}%
\pgfsetstrokecolor{currentstroke}%
\pgfsetstrokeopacity{0.200000}%
\pgfsetdash{}{0pt}%
\pgfpathmoveto{\pgfqpoint{0.728688in}{1.172843in}}%
\pgfpathlineto{\pgfqpoint{10.416188in}{1.172843in}}%
\pgfusepath{stroke}%
\end{pgfscope}%
\begin{pgfscope}%
\pgfsetbuttcap%
\pgfsetroundjoin%
\definecolor{currentfill}{rgb}{0.000000,0.000000,0.000000}%
\pgfsetfillcolor{currentfill}%
\pgfsetlinewidth{0.602250pt}%
\definecolor{currentstroke}{rgb}{0.000000,0.000000,0.000000}%
\pgfsetstrokecolor{currentstroke}%
\pgfsetdash{}{0pt}%
\pgfsys@defobject{currentmarker}{\pgfqpoint{-0.027778in}{0.000000in}}{\pgfqpoint{-0.000000in}{0.000000in}}{%
\pgfpathmoveto{\pgfqpoint{-0.000000in}{0.000000in}}%
\pgfpathlineto{\pgfqpoint{-0.027778in}{0.000000in}}%
\pgfusepath{stroke,fill}%
}%
\begin{pgfscope}%
\pgfsys@transformshift{0.728688in}{1.172843in}%
\pgfsys@useobject{currentmarker}{}%
\end{pgfscope}%
\end{pgfscope}%
\begin{pgfscope}%
\pgfpathrectangle{\pgfqpoint{0.728688in}{0.521603in}}{\pgfqpoint{9.687500in}{4.235000in}}%
\pgfusepath{clip}%
\pgfsetrectcap%
\pgfsetroundjoin%
\pgfsetlinewidth{0.803000pt}%
\definecolor{currentstroke}{rgb}{0.600000,0.600000,0.600000}%
\pgfsetstrokecolor{currentstroke}%
\pgfsetstrokeopacity{0.200000}%
\pgfsetdash{}{0pt}%
\pgfpathmoveto{\pgfqpoint{0.728688in}{1.325796in}}%
\pgfpathlineto{\pgfqpoint{10.416188in}{1.325796in}}%
\pgfusepath{stroke}%
\end{pgfscope}%
\begin{pgfscope}%
\pgfsetbuttcap%
\pgfsetroundjoin%
\definecolor{currentfill}{rgb}{0.000000,0.000000,0.000000}%
\pgfsetfillcolor{currentfill}%
\pgfsetlinewidth{0.602250pt}%
\definecolor{currentstroke}{rgb}{0.000000,0.000000,0.000000}%
\pgfsetstrokecolor{currentstroke}%
\pgfsetdash{}{0pt}%
\pgfsys@defobject{currentmarker}{\pgfqpoint{-0.027778in}{0.000000in}}{\pgfqpoint{-0.000000in}{0.000000in}}{%
\pgfpathmoveto{\pgfqpoint{-0.000000in}{0.000000in}}%
\pgfpathlineto{\pgfqpoint{-0.027778in}{0.000000in}}%
\pgfusepath{stroke,fill}%
}%
\begin{pgfscope}%
\pgfsys@transformshift{0.728688in}{1.325796in}%
\pgfsys@useobject{currentmarker}{}%
\end{pgfscope}%
\end{pgfscope}%
\begin{pgfscope}%
\pgfpathrectangle{\pgfqpoint{0.728688in}{0.521603in}}{\pgfqpoint{9.687500in}{4.235000in}}%
\pgfusepath{clip}%
\pgfsetrectcap%
\pgfsetroundjoin%
\pgfsetlinewidth{0.803000pt}%
\definecolor{currentstroke}{rgb}{0.600000,0.600000,0.600000}%
\pgfsetstrokecolor{currentstroke}%
\pgfsetstrokeopacity{0.200000}%
\pgfsetdash{}{0pt}%
\pgfpathmoveto{\pgfqpoint{0.728688in}{1.631702in}}%
\pgfpathlineto{\pgfqpoint{10.416188in}{1.631702in}}%
\pgfusepath{stroke}%
\end{pgfscope}%
\begin{pgfscope}%
\pgfsetbuttcap%
\pgfsetroundjoin%
\definecolor{currentfill}{rgb}{0.000000,0.000000,0.000000}%
\pgfsetfillcolor{currentfill}%
\pgfsetlinewidth{0.602250pt}%
\definecolor{currentstroke}{rgb}{0.000000,0.000000,0.000000}%
\pgfsetstrokecolor{currentstroke}%
\pgfsetdash{}{0pt}%
\pgfsys@defobject{currentmarker}{\pgfqpoint{-0.027778in}{0.000000in}}{\pgfqpoint{-0.000000in}{0.000000in}}{%
\pgfpathmoveto{\pgfqpoint{-0.000000in}{0.000000in}}%
\pgfpathlineto{\pgfqpoint{-0.027778in}{0.000000in}}%
\pgfusepath{stroke,fill}%
}%
\begin{pgfscope}%
\pgfsys@transformshift{0.728688in}{1.631702in}%
\pgfsys@useobject{currentmarker}{}%
\end{pgfscope}%
\end{pgfscope}%
\begin{pgfscope}%
\pgfpathrectangle{\pgfqpoint{0.728688in}{0.521603in}}{\pgfqpoint{9.687500in}{4.235000in}}%
\pgfusepath{clip}%
\pgfsetrectcap%
\pgfsetroundjoin%
\pgfsetlinewidth{0.803000pt}%
\definecolor{currentstroke}{rgb}{0.600000,0.600000,0.600000}%
\pgfsetstrokecolor{currentstroke}%
\pgfsetstrokeopacity{0.200000}%
\pgfsetdash{}{0pt}%
\pgfpathmoveto{\pgfqpoint{0.728688in}{1.784655in}}%
\pgfpathlineto{\pgfqpoint{10.416188in}{1.784655in}}%
\pgfusepath{stroke}%
\end{pgfscope}%
\begin{pgfscope}%
\pgfsetbuttcap%
\pgfsetroundjoin%
\definecolor{currentfill}{rgb}{0.000000,0.000000,0.000000}%
\pgfsetfillcolor{currentfill}%
\pgfsetlinewidth{0.602250pt}%
\definecolor{currentstroke}{rgb}{0.000000,0.000000,0.000000}%
\pgfsetstrokecolor{currentstroke}%
\pgfsetdash{}{0pt}%
\pgfsys@defobject{currentmarker}{\pgfqpoint{-0.027778in}{0.000000in}}{\pgfqpoint{-0.000000in}{0.000000in}}{%
\pgfpathmoveto{\pgfqpoint{-0.000000in}{0.000000in}}%
\pgfpathlineto{\pgfqpoint{-0.027778in}{0.000000in}}%
\pgfusepath{stroke,fill}%
}%
\begin{pgfscope}%
\pgfsys@transformshift{0.728688in}{1.784655in}%
\pgfsys@useobject{currentmarker}{}%
\end{pgfscope}%
\end{pgfscope}%
\begin{pgfscope}%
\pgfpathrectangle{\pgfqpoint{0.728688in}{0.521603in}}{\pgfqpoint{9.687500in}{4.235000in}}%
\pgfusepath{clip}%
\pgfsetrectcap%
\pgfsetroundjoin%
\pgfsetlinewidth{0.803000pt}%
\definecolor{currentstroke}{rgb}{0.600000,0.600000,0.600000}%
\pgfsetstrokecolor{currentstroke}%
\pgfsetstrokeopacity{0.200000}%
\pgfsetdash{}{0pt}%
\pgfpathmoveto{\pgfqpoint{0.728688in}{1.937608in}}%
\pgfpathlineto{\pgfqpoint{10.416188in}{1.937608in}}%
\pgfusepath{stroke}%
\end{pgfscope}%
\begin{pgfscope}%
\pgfsetbuttcap%
\pgfsetroundjoin%
\definecolor{currentfill}{rgb}{0.000000,0.000000,0.000000}%
\pgfsetfillcolor{currentfill}%
\pgfsetlinewidth{0.602250pt}%
\definecolor{currentstroke}{rgb}{0.000000,0.000000,0.000000}%
\pgfsetstrokecolor{currentstroke}%
\pgfsetdash{}{0pt}%
\pgfsys@defobject{currentmarker}{\pgfqpoint{-0.027778in}{0.000000in}}{\pgfqpoint{-0.000000in}{0.000000in}}{%
\pgfpathmoveto{\pgfqpoint{-0.000000in}{0.000000in}}%
\pgfpathlineto{\pgfqpoint{-0.027778in}{0.000000in}}%
\pgfusepath{stroke,fill}%
}%
\begin{pgfscope}%
\pgfsys@transformshift{0.728688in}{1.937608in}%
\pgfsys@useobject{currentmarker}{}%
\end{pgfscope}%
\end{pgfscope}%
\begin{pgfscope}%
\pgfpathrectangle{\pgfqpoint{0.728688in}{0.521603in}}{\pgfqpoint{9.687500in}{4.235000in}}%
\pgfusepath{clip}%
\pgfsetrectcap%
\pgfsetroundjoin%
\pgfsetlinewidth{0.803000pt}%
\definecolor{currentstroke}{rgb}{0.600000,0.600000,0.600000}%
\pgfsetstrokecolor{currentstroke}%
\pgfsetstrokeopacity{0.200000}%
\pgfsetdash{}{0pt}%
\pgfpathmoveto{\pgfqpoint{0.728688in}{2.090562in}}%
\pgfpathlineto{\pgfqpoint{10.416188in}{2.090562in}}%
\pgfusepath{stroke}%
\end{pgfscope}%
\begin{pgfscope}%
\pgfsetbuttcap%
\pgfsetroundjoin%
\definecolor{currentfill}{rgb}{0.000000,0.000000,0.000000}%
\pgfsetfillcolor{currentfill}%
\pgfsetlinewidth{0.602250pt}%
\definecolor{currentstroke}{rgb}{0.000000,0.000000,0.000000}%
\pgfsetstrokecolor{currentstroke}%
\pgfsetdash{}{0pt}%
\pgfsys@defobject{currentmarker}{\pgfqpoint{-0.027778in}{0.000000in}}{\pgfqpoint{-0.000000in}{0.000000in}}{%
\pgfpathmoveto{\pgfqpoint{-0.000000in}{0.000000in}}%
\pgfpathlineto{\pgfqpoint{-0.027778in}{0.000000in}}%
\pgfusepath{stroke,fill}%
}%
\begin{pgfscope}%
\pgfsys@transformshift{0.728688in}{2.090562in}%
\pgfsys@useobject{currentmarker}{}%
\end{pgfscope}%
\end{pgfscope}%
\begin{pgfscope}%
\pgfpathrectangle{\pgfqpoint{0.728688in}{0.521603in}}{\pgfqpoint{9.687500in}{4.235000in}}%
\pgfusepath{clip}%
\pgfsetrectcap%
\pgfsetroundjoin%
\pgfsetlinewidth{0.803000pt}%
\definecolor{currentstroke}{rgb}{0.600000,0.600000,0.600000}%
\pgfsetstrokecolor{currentstroke}%
\pgfsetstrokeopacity{0.200000}%
\pgfsetdash{}{0pt}%
\pgfpathmoveto{\pgfqpoint{0.728688in}{2.396468in}}%
\pgfpathlineto{\pgfqpoint{10.416188in}{2.396468in}}%
\pgfusepath{stroke}%
\end{pgfscope}%
\begin{pgfscope}%
\pgfsetbuttcap%
\pgfsetroundjoin%
\definecolor{currentfill}{rgb}{0.000000,0.000000,0.000000}%
\pgfsetfillcolor{currentfill}%
\pgfsetlinewidth{0.602250pt}%
\definecolor{currentstroke}{rgb}{0.000000,0.000000,0.000000}%
\pgfsetstrokecolor{currentstroke}%
\pgfsetdash{}{0pt}%
\pgfsys@defobject{currentmarker}{\pgfqpoint{-0.027778in}{0.000000in}}{\pgfqpoint{-0.000000in}{0.000000in}}{%
\pgfpathmoveto{\pgfqpoint{-0.000000in}{0.000000in}}%
\pgfpathlineto{\pgfqpoint{-0.027778in}{0.000000in}}%
\pgfusepath{stroke,fill}%
}%
\begin{pgfscope}%
\pgfsys@transformshift{0.728688in}{2.396468in}%
\pgfsys@useobject{currentmarker}{}%
\end{pgfscope}%
\end{pgfscope}%
\begin{pgfscope}%
\pgfpathrectangle{\pgfqpoint{0.728688in}{0.521603in}}{\pgfqpoint{9.687500in}{4.235000in}}%
\pgfusepath{clip}%
\pgfsetrectcap%
\pgfsetroundjoin%
\pgfsetlinewidth{0.803000pt}%
\definecolor{currentstroke}{rgb}{0.600000,0.600000,0.600000}%
\pgfsetstrokecolor{currentstroke}%
\pgfsetstrokeopacity{0.200000}%
\pgfsetdash{}{0pt}%
\pgfpathmoveto{\pgfqpoint{0.728688in}{2.549421in}}%
\pgfpathlineto{\pgfqpoint{10.416188in}{2.549421in}}%
\pgfusepath{stroke}%
\end{pgfscope}%
\begin{pgfscope}%
\pgfsetbuttcap%
\pgfsetroundjoin%
\definecolor{currentfill}{rgb}{0.000000,0.000000,0.000000}%
\pgfsetfillcolor{currentfill}%
\pgfsetlinewidth{0.602250pt}%
\definecolor{currentstroke}{rgb}{0.000000,0.000000,0.000000}%
\pgfsetstrokecolor{currentstroke}%
\pgfsetdash{}{0pt}%
\pgfsys@defobject{currentmarker}{\pgfqpoint{-0.027778in}{0.000000in}}{\pgfqpoint{-0.000000in}{0.000000in}}{%
\pgfpathmoveto{\pgfqpoint{-0.000000in}{0.000000in}}%
\pgfpathlineto{\pgfqpoint{-0.027778in}{0.000000in}}%
\pgfusepath{stroke,fill}%
}%
\begin{pgfscope}%
\pgfsys@transformshift{0.728688in}{2.549421in}%
\pgfsys@useobject{currentmarker}{}%
\end{pgfscope}%
\end{pgfscope}%
\begin{pgfscope}%
\pgfpathrectangle{\pgfqpoint{0.728688in}{0.521603in}}{\pgfqpoint{9.687500in}{4.235000in}}%
\pgfusepath{clip}%
\pgfsetrectcap%
\pgfsetroundjoin%
\pgfsetlinewidth{0.803000pt}%
\definecolor{currentstroke}{rgb}{0.600000,0.600000,0.600000}%
\pgfsetstrokecolor{currentstroke}%
\pgfsetstrokeopacity{0.200000}%
\pgfsetdash{}{0pt}%
\pgfpathmoveto{\pgfqpoint{0.728688in}{2.702374in}}%
\pgfpathlineto{\pgfqpoint{10.416188in}{2.702374in}}%
\pgfusepath{stroke}%
\end{pgfscope}%
\begin{pgfscope}%
\pgfsetbuttcap%
\pgfsetroundjoin%
\definecolor{currentfill}{rgb}{0.000000,0.000000,0.000000}%
\pgfsetfillcolor{currentfill}%
\pgfsetlinewidth{0.602250pt}%
\definecolor{currentstroke}{rgb}{0.000000,0.000000,0.000000}%
\pgfsetstrokecolor{currentstroke}%
\pgfsetdash{}{0pt}%
\pgfsys@defobject{currentmarker}{\pgfqpoint{-0.027778in}{0.000000in}}{\pgfqpoint{-0.000000in}{0.000000in}}{%
\pgfpathmoveto{\pgfqpoint{-0.000000in}{0.000000in}}%
\pgfpathlineto{\pgfqpoint{-0.027778in}{0.000000in}}%
\pgfusepath{stroke,fill}%
}%
\begin{pgfscope}%
\pgfsys@transformshift{0.728688in}{2.702374in}%
\pgfsys@useobject{currentmarker}{}%
\end{pgfscope}%
\end{pgfscope}%
\begin{pgfscope}%
\pgfpathrectangle{\pgfqpoint{0.728688in}{0.521603in}}{\pgfqpoint{9.687500in}{4.235000in}}%
\pgfusepath{clip}%
\pgfsetrectcap%
\pgfsetroundjoin%
\pgfsetlinewidth{0.803000pt}%
\definecolor{currentstroke}{rgb}{0.600000,0.600000,0.600000}%
\pgfsetstrokecolor{currentstroke}%
\pgfsetstrokeopacity{0.200000}%
\pgfsetdash{}{0pt}%
\pgfpathmoveto{\pgfqpoint{0.728688in}{2.855327in}}%
\pgfpathlineto{\pgfqpoint{10.416188in}{2.855327in}}%
\pgfusepath{stroke}%
\end{pgfscope}%
\begin{pgfscope}%
\pgfsetbuttcap%
\pgfsetroundjoin%
\definecolor{currentfill}{rgb}{0.000000,0.000000,0.000000}%
\pgfsetfillcolor{currentfill}%
\pgfsetlinewidth{0.602250pt}%
\definecolor{currentstroke}{rgb}{0.000000,0.000000,0.000000}%
\pgfsetstrokecolor{currentstroke}%
\pgfsetdash{}{0pt}%
\pgfsys@defobject{currentmarker}{\pgfqpoint{-0.027778in}{0.000000in}}{\pgfqpoint{-0.000000in}{0.000000in}}{%
\pgfpathmoveto{\pgfqpoint{-0.000000in}{0.000000in}}%
\pgfpathlineto{\pgfqpoint{-0.027778in}{0.000000in}}%
\pgfusepath{stroke,fill}%
}%
\begin{pgfscope}%
\pgfsys@transformshift{0.728688in}{2.855327in}%
\pgfsys@useobject{currentmarker}{}%
\end{pgfscope}%
\end{pgfscope}%
\begin{pgfscope}%
\pgfpathrectangle{\pgfqpoint{0.728688in}{0.521603in}}{\pgfqpoint{9.687500in}{4.235000in}}%
\pgfusepath{clip}%
\pgfsetrectcap%
\pgfsetroundjoin%
\pgfsetlinewidth{0.803000pt}%
\definecolor{currentstroke}{rgb}{0.600000,0.600000,0.600000}%
\pgfsetstrokecolor{currentstroke}%
\pgfsetstrokeopacity{0.200000}%
\pgfsetdash{}{0pt}%
\pgfpathmoveto{\pgfqpoint{0.728688in}{3.161233in}}%
\pgfpathlineto{\pgfqpoint{10.416188in}{3.161233in}}%
\pgfusepath{stroke}%
\end{pgfscope}%
\begin{pgfscope}%
\pgfsetbuttcap%
\pgfsetroundjoin%
\definecolor{currentfill}{rgb}{0.000000,0.000000,0.000000}%
\pgfsetfillcolor{currentfill}%
\pgfsetlinewidth{0.602250pt}%
\definecolor{currentstroke}{rgb}{0.000000,0.000000,0.000000}%
\pgfsetstrokecolor{currentstroke}%
\pgfsetdash{}{0pt}%
\pgfsys@defobject{currentmarker}{\pgfqpoint{-0.027778in}{0.000000in}}{\pgfqpoint{-0.000000in}{0.000000in}}{%
\pgfpathmoveto{\pgfqpoint{-0.000000in}{0.000000in}}%
\pgfpathlineto{\pgfqpoint{-0.027778in}{0.000000in}}%
\pgfusepath{stroke,fill}%
}%
\begin{pgfscope}%
\pgfsys@transformshift{0.728688in}{3.161233in}%
\pgfsys@useobject{currentmarker}{}%
\end{pgfscope}%
\end{pgfscope}%
\begin{pgfscope}%
\pgfpathrectangle{\pgfqpoint{0.728688in}{0.521603in}}{\pgfqpoint{9.687500in}{4.235000in}}%
\pgfusepath{clip}%
\pgfsetrectcap%
\pgfsetroundjoin%
\pgfsetlinewidth{0.803000pt}%
\definecolor{currentstroke}{rgb}{0.600000,0.600000,0.600000}%
\pgfsetstrokecolor{currentstroke}%
\pgfsetstrokeopacity{0.200000}%
\pgfsetdash{}{0pt}%
\pgfpathmoveto{\pgfqpoint{0.728688in}{3.314186in}}%
\pgfpathlineto{\pgfqpoint{10.416188in}{3.314186in}}%
\pgfusepath{stroke}%
\end{pgfscope}%
\begin{pgfscope}%
\pgfsetbuttcap%
\pgfsetroundjoin%
\definecolor{currentfill}{rgb}{0.000000,0.000000,0.000000}%
\pgfsetfillcolor{currentfill}%
\pgfsetlinewidth{0.602250pt}%
\definecolor{currentstroke}{rgb}{0.000000,0.000000,0.000000}%
\pgfsetstrokecolor{currentstroke}%
\pgfsetdash{}{0pt}%
\pgfsys@defobject{currentmarker}{\pgfqpoint{-0.027778in}{0.000000in}}{\pgfqpoint{-0.000000in}{0.000000in}}{%
\pgfpathmoveto{\pgfqpoint{-0.000000in}{0.000000in}}%
\pgfpathlineto{\pgfqpoint{-0.027778in}{0.000000in}}%
\pgfusepath{stroke,fill}%
}%
\begin{pgfscope}%
\pgfsys@transformshift{0.728688in}{3.314186in}%
\pgfsys@useobject{currentmarker}{}%
\end{pgfscope}%
\end{pgfscope}%
\begin{pgfscope}%
\pgfpathrectangle{\pgfqpoint{0.728688in}{0.521603in}}{\pgfqpoint{9.687500in}{4.235000in}}%
\pgfusepath{clip}%
\pgfsetrectcap%
\pgfsetroundjoin%
\pgfsetlinewidth{0.803000pt}%
\definecolor{currentstroke}{rgb}{0.600000,0.600000,0.600000}%
\pgfsetstrokecolor{currentstroke}%
\pgfsetstrokeopacity{0.200000}%
\pgfsetdash{}{0pt}%
\pgfpathmoveto{\pgfqpoint{0.728688in}{3.467139in}}%
\pgfpathlineto{\pgfqpoint{10.416188in}{3.467139in}}%
\pgfusepath{stroke}%
\end{pgfscope}%
\begin{pgfscope}%
\pgfsetbuttcap%
\pgfsetroundjoin%
\definecolor{currentfill}{rgb}{0.000000,0.000000,0.000000}%
\pgfsetfillcolor{currentfill}%
\pgfsetlinewidth{0.602250pt}%
\definecolor{currentstroke}{rgb}{0.000000,0.000000,0.000000}%
\pgfsetstrokecolor{currentstroke}%
\pgfsetdash{}{0pt}%
\pgfsys@defobject{currentmarker}{\pgfqpoint{-0.027778in}{0.000000in}}{\pgfqpoint{-0.000000in}{0.000000in}}{%
\pgfpathmoveto{\pgfqpoint{-0.000000in}{0.000000in}}%
\pgfpathlineto{\pgfqpoint{-0.027778in}{0.000000in}}%
\pgfusepath{stroke,fill}%
}%
\begin{pgfscope}%
\pgfsys@transformshift{0.728688in}{3.467139in}%
\pgfsys@useobject{currentmarker}{}%
\end{pgfscope}%
\end{pgfscope}%
\begin{pgfscope}%
\pgfpathrectangle{\pgfqpoint{0.728688in}{0.521603in}}{\pgfqpoint{9.687500in}{4.235000in}}%
\pgfusepath{clip}%
\pgfsetrectcap%
\pgfsetroundjoin%
\pgfsetlinewidth{0.803000pt}%
\definecolor{currentstroke}{rgb}{0.600000,0.600000,0.600000}%
\pgfsetstrokecolor{currentstroke}%
\pgfsetstrokeopacity{0.200000}%
\pgfsetdash{}{0pt}%
\pgfpathmoveto{\pgfqpoint{0.728688in}{3.620092in}}%
\pgfpathlineto{\pgfqpoint{10.416188in}{3.620092in}}%
\pgfusepath{stroke}%
\end{pgfscope}%
\begin{pgfscope}%
\pgfsetbuttcap%
\pgfsetroundjoin%
\definecolor{currentfill}{rgb}{0.000000,0.000000,0.000000}%
\pgfsetfillcolor{currentfill}%
\pgfsetlinewidth{0.602250pt}%
\definecolor{currentstroke}{rgb}{0.000000,0.000000,0.000000}%
\pgfsetstrokecolor{currentstroke}%
\pgfsetdash{}{0pt}%
\pgfsys@defobject{currentmarker}{\pgfqpoint{-0.027778in}{0.000000in}}{\pgfqpoint{-0.000000in}{0.000000in}}{%
\pgfpathmoveto{\pgfqpoint{-0.000000in}{0.000000in}}%
\pgfpathlineto{\pgfqpoint{-0.027778in}{0.000000in}}%
\pgfusepath{stroke,fill}%
}%
\begin{pgfscope}%
\pgfsys@transformshift{0.728688in}{3.620092in}%
\pgfsys@useobject{currentmarker}{}%
\end{pgfscope}%
\end{pgfscope}%
\begin{pgfscope}%
\pgfpathrectangle{\pgfqpoint{0.728688in}{0.521603in}}{\pgfqpoint{9.687500in}{4.235000in}}%
\pgfusepath{clip}%
\pgfsetrectcap%
\pgfsetroundjoin%
\pgfsetlinewidth{0.803000pt}%
\definecolor{currentstroke}{rgb}{0.600000,0.600000,0.600000}%
\pgfsetstrokecolor{currentstroke}%
\pgfsetstrokeopacity{0.200000}%
\pgfsetdash{}{0pt}%
\pgfpathmoveto{\pgfqpoint{0.728688in}{3.925998in}}%
\pgfpathlineto{\pgfqpoint{10.416188in}{3.925998in}}%
\pgfusepath{stroke}%
\end{pgfscope}%
\begin{pgfscope}%
\pgfsetbuttcap%
\pgfsetroundjoin%
\definecolor{currentfill}{rgb}{0.000000,0.000000,0.000000}%
\pgfsetfillcolor{currentfill}%
\pgfsetlinewidth{0.602250pt}%
\definecolor{currentstroke}{rgb}{0.000000,0.000000,0.000000}%
\pgfsetstrokecolor{currentstroke}%
\pgfsetdash{}{0pt}%
\pgfsys@defobject{currentmarker}{\pgfqpoint{-0.027778in}{0.000000in}}{\pgfqpoint{-0.000000in}{0.000000in}}{%
\pgfpathmoveto{\pgfqpoint{-0.000000in}{0.000000in}}%
\pgfpathlineto{\pgfqpoint{-0.027778in}{0.000000in}}%
\pgfusepath{stroke,fill}%
}%
\begin{pgfscope}%
\pgfsys@transformshift{0.728688in}{3.925998in}%
\pgfsys@useobject{currentmarker}{}%
\end{pgfscope}%
\end{pgfscope}%
\begin{pgfscope}%
\pgfpathrectangle{\pgfqpoint{0.728688in}{0.521603in}}{\pgfqpoint{9.687500in}{4.235000in}}%
\pgfusepath{clip}%
\pgfsetrectcap%
\pgfsetroundjoin%
\pgfsetlinewidth{0.803000pt}%
\definecolor{currentstroke}{rgb}{0.600000,0.600000,0.600000}%
\pgfsetstrokecolor{currentstroke}%
\pgfsetstrokeopacity{0.200000}%
\pgfsetdash{}{0pt}%
\pgfpathmoveto{\pgfqpoint{0.728688in}{4.078951in}}%
\pgfpathlineto{\pgfqpoint{10.416188in}{4.078951in}}%
\pgfusepath{stroke}%
\end{pgfscope}%
\begin{pgfscope}%
\pgfsetbuttcap%
\pgfsetroundjoin%
\definecolor{currentfill}{rgb}{0.000000,0.000000,0.000000}%
\pgfsetfillcolor{currentfill}%
\pgfsetlinewidth{0.602250pt}%
\definecolor{currentstroke}{rgb}{0.000000,0.000000,0.000000}%
\pgfsetstrokecolor{currentstroke}%
\pgfsetdash{}{0pt}%
\pgfsys@defobject{currentmarker}{\pgfqpoint{-0.027778in}{0.000000in}}{\pgfqpoint{-0.000000in}{0.000000in}}{%
\pgfpathmoveto{\pgfqpoint{-0.000000in}{0.000000in}}%
\pgfpathlineto{\pgfqpoint{-0.027778in}{0.000000in}}%
\pgfusepath{stroke,fill}%
}%
\begin{pgfscope}%
\pgfsys@transformshift{0.728688in}{4.078951in}%
\pgfsys@useobject{currentmarker}{}%
\end{pgfscope}%
\end{pgfscope}%
\begin{pgfscope}%
\pgfpathrectangle{\pgfqpoint{0.728688in}{0.521603in}}{\pgfqpoint{9.687500in}{4.235000in}}%
\pgfusepath{clip}%
\pgfsetrectcap%
\pgfsetroundjoin%
\pgfsetlinewidth{0.803000pt}%
\definecolor{currentstroke}{rgb}{0.600000,0.600000,0.600000}%
\pgfsetstrokecolor{currentstroke}%
\pgfsetstrokeopacity{0.200000}%
\pgfsetdash{}{0pt}%
\pgfpathmoveto{\pgfqpoint{0.728688in}{4.231905in}}%
\pgfpathlineto{\pgfqpoint{10.416188in}{4.231905in}}%
\pgfusepath{stroke}%
\end{pgfscope}%
\begin{pgfscope}%
\pgfsetbuttcap%
\pgfsetroundjoin%
\definecolor{currentfill}{rgb}{0.000000,0.000000,0.000000}%
\pgfsetfillcolor{currentfill}%
\pgfsetlinewidth{0.602250pt}%
\definecolor{currentstroke}{rgb}{0.000000,0.000000,0.000000}%
\pgfsetstrokecolor{currentstroke}%
\pgfsetdash{}{0pt}%
\pgfsys@defobject{currentmarker}{\pgfqpoint{-0.027778in}{0.000000in}}{\pgfqpoint{-0.000000in}{0.000000in}}{%
\pgfpathmoveto{\pgfqpoint{-0.000000in}{0.000000in}}%
\pgfpathlineto{\pgfqpoint{-0.027778in}{0.000000in}}%
\pgfusepath{stroke,fill}%
}%
\begin{pgfscope}%
\pgfsys@transformshift{0.728688in}{4.231905in}%
\pgfsys@useobject{currentmarker}{}%
\end{pgfscope}%
\end{pgfscope}%
\begin{pgfscope}%
\pgfpathrectangle{\pgfqpoint{0.728688in}{0.521603in}}{\pgfqpoint{9.687500in}{4.235000in}}%
\pgfusepath{clip}%
\pgfsetrectcap%
\pgfsetroundjoin%
\pgfsetlinewidth{0.803000pt}%
\definecolor{currentstroke}{rgb}{0.600000,0.600000,0.600000}%
\pgfsetstrokecolor{currentstroke}%
\pgfsetstrokeopacity{0.200000}%
\pgfsetdash{}{0pt}%
\pgfpathmoveto{\pgfqpoint{0.728688in}{4.384858in}}%
\pgfpathlineto{\pgfqpoint{10.416188in}{4.384858in}}%
\pgfusepath{stroke}%
\end{pgfscope}%
\begin{pgfscope}%
\pgfsetbuttcap%
\pgfsetroundjoin%
\definecolor{currentfill}{rgb}{0.000000,0.000000,0.000000}%
\pgfsetfillcolor{currentfill}%
\pgfsetlinewidth{0.602250pt}%
\definecolor{currentstroke}{rgb}{0.000000,0.000000,0.000000}%
\pgfsetstrokecolor{currentstroke}%
\pgfsetdash{}{0pt}%
\pgfsys@defobject{currentmarker}{\pgfqpoint{-0.027778in}{0.000000in}}{\pgfqpoint{-0.000000in}{0.000000in}}{%
\pgfpathmoveto{\pgfqpoint{-0.000000in}{0.000000in}}%
\pgfpathlineto{\pgfqpoint{-0.027778in}{0.000000in}}%
\pgfusepath{stroke,fill}%
}%
\begin{pgfscope}%
\pgfsys@transformshift{0.728688in}{4.384858in}%
\pgfsys@useobject{currentmarker}{}%
\end{pgfscope}%
\end{pgfscope}%
\begin{pgfscope}%
\pgfpathrectangle{\pgfqpoint{0.728688in}{0.521603in}}{\pgfqpoint{9.687500in}{4.235000in}}%
\pgfusepath{clip}%
\pgfsetrectcap%
\pgfsetroundjoin%
\pgfsetlinewidth{0.803000pt}%
\definecolor{currentstroke}{rgb}{0.600000,0.600000,0.600000}%
\pgfsetstrokecolor{currentstroke}%
\pgfsetstrokeopacity{0.200000}%
\pgfsetdash{}{0pt}%
\pgfpathmoveto{\pgfqpoint{0.728688in}{4.690764in}}%
\pgfpathlineto{\pgfqpoint{10.416188in}{4.690764in}}%
\pgfusepath{stroke}%
\end{pgfscope}%
\begin{pgfscope}%
\pgfsetbuttcap%
\pgfsetroundjoin%
\definecolor{currentfill}{rgb}{0.000000,0.000000,0.000000}%
\pgfsetfillcolor{currentfill}%
\pgfsetlinewidth{0.602250pt}%
\definecolor{currentstroke}{rgb}{0.000000,0.000000,0.000000}%
\pgfsetstrokecolor{currentstroke}%
\pgfsetdash{}{0pt}%
\pgfsys@defobject{currentmarker}{\pgfqpoint{-0.027778in}{0.000000in}}{\pgfqpoint{-0.000000in}{0.000000in}}{%
\pgfpathmoveto{\pgfqpoint{-0.000000in}{0.000000in}}%
\pgfpathlineto{\pgfqpoint{-0.027778in}{0.000000in}}%
\pgfusepath{stroke,fill}%
}%
\begin{pgfscope}%
\pgfsys@transformshift{0.728688in}{4.690764in}%
\pgfsys@useobject{currentmarker}{}%
\end{pgfscope}%
\end{pgfscope}%
\begin{pgfscope}%
\definecolor{textcolor}{rgb}{0.000000,0.000000,0.000000}%
\pgfsetstrokecolor{textcolor}%
\pgfsetfillcolor{textcolor}%
\pgftext[x=0.266665in,y=2.639103in,,bottom,rotate=90.000000]{\color{textcolor}\sffamily\fontsize{10.000000}{12.000000}\selectfont avg. infection rate \(\displaystyle \overline{\langle I\rangle}\)}%
\end{pgfscope}%
\begin{pgfscope}%
\pgfpathrectangle{\pgfqpoint{0.728688in}{0.521603in}}{\pgfqpoint{9.687500in}{4.235000in}}%
\pgfusepath{clip}%
\pgfsetbuttcap%
\pgfsetroundjoin%
\pgfsetlinewidth{1.003750pt}%
\definecolor{currentstroke}{rgb}{0.000000,0.000000,1.000000}%
\pgfsetstrokecolor{currentstroke}%
\pgfsetstrokeopacity{0.500000}%
\pgfsetdash{{3.700000pt}{1.600000pt}}{0.000000pt}%
\pgfpathmoveto{\pgfqpoint{1.169029in}{4.046571in}}%
\pgfpathlineto{\pgfqpoint{1.348760in}{4.207524in}}%
\pgfpathlineto{\pgfqpoint{1.528491in}{3.612032in}}%
\pgfpathlineto{\pgfqpoint{1.708222in}{3.675538in}}%
\pgfpathlineto{\pgfqpoint{1.887953in}{3.140982in}}%
\pgfpathlineto{\pgfqpoint{2.067684in}{3.043000in}}%
\pgfpathlineto{\pgfqpoint{2.247415in}{0.862755in}}%
\pgfpathlineto{\pgfqpoint{2.427146in}{1.019771in}}%
\pgfpathlineto{\pgfqpoint{2.606877in}{0.809341in}}%
\pgfpathlineto{\pgfqpoint{2.786608in}{1.230678in}}%
\pgfpathlineto{\pgfqpoint{2.966339in}{0.982488in}}%
\pgfpathlineto{\pgfqpoint{3.146070in}{0.878766in}}%
\pgfpathlineto{\pgfqpoint{3.325801in}{0.780542in}}%
\pgfpathlineto{\pgfqpoint{3.505532in}{0.772835in}}%
\pgfpathlineto{\pgfqpoint{3.685263in}{0.745829in}}%
\pgfpathlineto{\pgfqpoint{3.864994in}{0.778630in}}%
\pgfpathlineto{\pgfqpoint{4.044725in}{0.752521in}}%
\pgfpathlineto{\pgfqpoint{4.224456in}{0.752342in}}%
\pgfpathlineto{\pgfqpoint{4.404187in}{0.755628in}}%
\pgfpathlineto{\pgfqpoint{4.583918in}{0.731131in}}%
\pgfpathlineto{\pgfqpoint{4.763648in}{0.732625in}}%
\pgfpathlineto{\pgfqpoint{4.943379in}{0.731609in}}%
\pgfpathlineto{\pgfqpoint{5.123110in}{0.734955in}}%
\pgfpathlineto{\pgfqpoint{5.302841in}{0.739974in}}%
\pgfpathlineto{\pgfqpoint{5.482572in}{0.741348in}}%
\pgfpathlineto{\pgfqpoint{5.662303in}{0.727128in}}%
\pgfpathlineto{\pgfqpoint{5.842034in}{0.722886in}}%
\pgfpathlineto{\pgfqpoint{6.021765in}{0.720078in}}%
\pgfpathlineto{\pgfqpoint{6.201496in}{0.726053in}}%
\pgfpathlineto{\pgfqpoint{6.381227in}{0.719301in}}%
\pgfpathlineto{\pgfqpoint{6.560958in}{0.718883in}}%
\pgfpathlineto{\pgfqpoint{6.740689in}{0.728502in}}%
\pgfpathlineto{\pgfqpoint{6.920420in}{0.721811in}}%
\pgfpathlineto{\pgfqpoint{7.100151in}{0.719720in}}%
\pgfpathlineto{\pgfqpoint{7.279882in}{0.719540in}}%
\pgfpathlineto{\pgfqpoint{7.459613in}{0.722647in}}%
\pgfpathlineto{\pgfqpoint{7.639344in}{0.718166in}}%
\pgfpathlineto{\pgfqpoint{7.819075in}{0.719182in}}%
\pgfpathlineto{\pgfqpoint{7.998806in}{0.715717in}}%
\pgfpathlineto{\pgfqpoint{8.178537in}{0.718106in}}%
\pgfpathlineto{\pgfqpoint{8.358268in}{0.718106in}}%
\pgfpathlineto{\pgfqpoint{8.537999in}{0.715955in}}%
\pgfpathlineto{\pgfqpoint{8.717730in}{0.715298in}}%
\pgfpathlineto{\pgfqpoint{8.897461in}{0.715657in}}%
\pgfpathlineto{\pgfqpoint{9.077192in}{0.717808in}}%
\pgfpathlineto{\pgfqpoint{9.256923in}{0.715478in}}%
\pgfpathlineto{\pgfqpoint{9.436654in}{0.714103in}}%
\pgfpathlineto{\pgfqpoint{9.616385in}{0.716135in}}%
\pgfpathlineto{\pgfqpoint{9.796116in}{0.714283in}}%
\pgfpathlineto{\pgfqpoint{9.975847in}{0.714342in}}%
\pgfusepath{stroke}%
\end{pgfscope}%
\begin{pgfscope}%
\pgfpathrectangle{\pgfqpoint{0.728688in}{0.521603in}}{\pgfqpoint{9.687500in}{4.235000in}}%
\pgfusepath{clip}%
\pgfsetbuttcap%
\pgfsetroundjoin%
\pgfsetlinewidth{1.003750pt}%
\definecolor{currentstroke}{rgb}{0.980392,0.164706,0.333333}%
\pgfsetstrokecolor{currentstroke}%
\pgfsetstrokeopacity{0.500000}%
\pgfsetdash{{3.700000pt}{1.600000pt}}{0.000000pt}%
\pgfpathmoveto{\pgfqpoint{1.169029in}{4.391817in}}%
\pgfpathlineto{\pgfqpoint{1.348760in}{4.322162in}}%
\pgfpathlineto{\pgfqpoint{1.528491in}{4.026152in}}%
\pgfpathlineto{\pgfqpoint{1.708222in}{3.852214in}}%
\pgfpathlineto{\pgfqpoint{1.887953in}{3.563958in}}%
\pgfpathlineto{\pgfqpoint{2.067684in}{3.194653in}}%
\pgfpathlineto{\pgfqpoint{2.247415in}{2.992969in}}%
\pgfpathlineto{\pgfqpoint{2.427146in}{2.672533in}}%
\pgfpathlineto{\pgfqpoint{2.606877in}{2.406884in}}%
\pgfpathlineto{\pgfqpoint{2.786608in}{2.273233in}}%
\pgfpathlineto{\pgfqpoint{2.966339in}{1.935697in}}%
\pgfpathlineto{\pgfqpoint{3.146070in}{1.472835in}}%
\pgfpathlineto{\pgfqpoint{3.325801in}{0.914585in}}%
\pgfpathlineto{\pgfqpoint{3.505532in}{0.808922in}}%
\pgfpathlineto{\pgfqpoint{3.685263in}{0.784725in}}%
\pgfpathlineto{\pgfqpoint{3.864994in}{0.777092in}}%
\pgfpathlineto{\pgfqpoint{4.044725in}{0.782215in}}%
\pgfpathlineto{\pgfqpoint{4.224456in}{0.762887in}}%
\pgfpathlineto{\pgfqpoint{4.404187in}{0.735836in}}%
\pgfpathlineto{\pgfqpoint{4.583918in}{0.743051in}}%
\pgfpathlineto{\pgfqpoint{4.763648in}{0.742334in}}%
\pgfpathlineto{\pgfqpoint{4.943379in}{0.731923in}}%
\pgfpathlineto{\pgfqpoint{5.123110in}{0.737465in}}%
\pgfpathlineto{\pgfqpoint{5.302841in}{0.734134in}}%
\pgfpathlineto{\pgfqpoint{5.482572in}{0.732909in}}%
\pgfpathlineto{\pgfqpoint{5.662303in}{0.729443in}}%
\pgfpathlineto{\pgfqpoint{5.842034in}{0.725948in}}%
\pgfpathlineto{\pgfqpoint{6.021765in}{0.726098in}}%
\pgfpathlineto{\pgfqpoint{6.201496in}{0.724440in}}%
\pgfpathlineto{\pgfqpoint{6.381227in}{0.726830in}}%
\pgfpathlineto{\pgfqpoint{6.560958in}{0.723230in}}%
\pgfpathlineto{\pgfqpoint{6.740689in}{0.724664in}}%
\pgfpathlineto{\pgfqpoint{6.920420in}{0.720870in}}%
\pgfpathlineto{\pgfqpoint{7.100151in}{0.720586in}}%
\pgfpathlineto{\pgfqpoint{7.279882in}{0.720750in}}%
\pgfpathlineto{\pgfqpoint{7.459613in}{0.721139in}}%
\pgfpathlineto{\pgfqpoint{7.639344in}{0.719003in}}%
\pgfpathlineto{\pgfqpoint{7.819075in}{0.719122in}}%
\pgfpathlineto{\pgfqpoint{7.998806in}{0.717852in}}%
\pgfpathlineto{\pgfqpoint{8.178537in}{0.719167in}}%
\pgfpathlineto{\pgfqpoint{8.358268in}{0.717494in}}%
\pgfpathlineto{\pgfqpoint{8.537999in}{0.716643in}}%
\pgfpathlineto{\pgfqpoint{8.717730in}{0.716404in}}%
\pgfpathlineto{\pgfqpoint{8.897461in}{0.716090in}}%
\pgfpathlineto{\pgfqpoint{9.077192in}{0.716344in}}%
\pgfpathlineto{\pgfqpoint{9.256923in}{0.715089in}}%
\pgfpathlineto{\pgfqpoint{9.436654in}{0.715283in}}%
\pgfpathlineto{\pgfqpoint{9.616385in}{0.714835in}}%
\pgfpathlineto{\pgfqpoint{9.796116in}{0.714566in}}%
\pgfpathlineto{\pgfqpoint{9.975847in}{0.714327in}}%
\pgfusepath{stroke}%
\end{pgfscope}%
\begin{pgfscope}%
\pgfpathrectangle{\pgfqpoint{0.728688in}{0.521603in}}{\pgfqpoint{9.687500in}{4.235000in}}%
\pgfusepath{clip}%
\pgfsetbuttcap%
\pgfsetroundjoin%
\pgfsetlinewidth{1.003750pt}%
\definecolor{currentstroke}{rgb}{0.239216,0.478431,0.992157}%
\pgfsetstrokecolor{currentstroke}%
\pgfsetstrokeopacity{0.500000}%
\pgfsetdash{{3.700000pt}{1.600000pt}}{0.000000pt}%
\pgfpathmoveto{\pgfqpoint{1.169029in}{4.508214in}}%
\pgfpathlineto{\pgfqpoint{1.348760in}{4.350030in}}%
\pgfpathlineto{\pgfqpoint{1.528491in}{4.111408in}}%
\pgfpathlineto{\pgfqpoint{1.708222in}{3.898727in}}%
\pgfpathlineto{\pgfqpoint{1.887953in}{3.717998in}}%
\pgfpathlineto{\pgfqpoint{2.067684in}{3.402746in}}%
\pgfpathlineto{\pgfqpoint{2.247415in}{3.179113in}}%
\pgfpathlineto{\pgfqpoint{2.427146in}{2.843213in}}%
\pgfpathlineto{\pgfqpoint{2.606877in}{2.673695in}}%
\pgfpathlineto{\pgfqpoint{2.786608in}{2.320542in}}%
\pgfpathlineto{\pgfqpoint{2.966339in}{1.944972in}}%
\pgfpathlineto{\pgfqpoint{3.146070in}{1.702614in}}%
\pgfpathlineto{\pgfqpoint{3.325801in}{1.103289in}}%
\pgfpathlineto{\pgfqpoint{3.505532in}{1.007478in}}%
\pgfpathlineto{\pgfqpoint{3.685263in}{0.803504in}}%
\pgfpathlineto{\pgfqpoint{3.864994in}{0.846261in}}%
\pgfpathlineto{\pgfqpoint{4.044725in}{0.763596in}}%
\pgfpathlineto{\pgfqpoint{4.224456in}{0.764336in}}%
\pgfpathlineto{\pgfqpoint{4.404187in}{0.759059in}}%
\pgfpathlineto{\pgfqpoint{4.583918in}{0.750605in}}%
\pgfpathlineto{\pgfqpoint{4.763648in}{0.748484in}}%
\pgfpathlineto{\pgfqpoint{4.943379in}{0.741053in}}%
\pgfpathlineto{\pgfqpoint{5.123110in}{0.734007in}}%
\pgfpathlineto{\pgfqpoint{5.302841in}{0.734862in}}%
\pgfpathlineto{\pgfqpoint{5.482572in}{0.729369in}}%
\pgfpathlineto{\pgfqpoint{5.662303in}{0.731867in}}%
\pgfpathlineto{\pgfqpoint{5.842034in}{0.729111in}}%
\pgfpathlineto{\pgfqpoint{6.021765in}{0.727804in}}%
\pgfpathlineto{\pgfqpoint{6.201496in}{0.726949in}}%
\pgfpathlineto{\pgfqpoint{6.381227in}{0.725582in}}%
\pgfpathlineto{\pgfqpoint{6.560958in}{0.724895in}}%
\pgfpathlineto{\pgfqpoint{6.740689in}{0.721590in}}%
\pgfpathlineto{\pgfqpoint{6.920420in}{0.722128in}}%
\pgfpathlineto{\pgfqpoint{7.100151in}{0.722307in}}%
\pgfpathlineto{\pgfqpoint{7.279882in}{0.720369in}}%
\pgfpathlineto{\pgfqpoint{7.459613in}{0.719708in}}%
\pgfpathlineto{\pgfqpoint{7.639344in}{0.719215in}}%
\pgfpathlineto{\pgfqpoint{7.819075in}{0.718719in}}%
\pgfpathlineto{\pgfqpoint{7.998806in}{0.718327in}}%
\pgfpathlineto{\pgfqpoint{8.178537in}{0.717572in}}%
\pgfpathlineto{\pgfqpoint{8.358268in}{0.717240in}}%
\pgfpathlineto{\pgfqpoint{8.537999in}{0.717341in}}%
\pgfpathlineto{\pgfqpoint{8.717730in}{0.716687in}}%
\pgfpathlineto{\pgfqpoint{8.897461in}{0.716187in}}%
\pgfpathlineto{\pgfqpoint{9.077192in}{0.715814in}}%
\pgfpathlineto{\pgfqpoint{9.256923in}{0.715227in}}%
\pgfpathlineto{\pgfqpoint{9.436654in}{0.715160in}}%
\pgfpathlineto{\pgfqpoint{9.616385in}{0.714772in}}%
\pgfpathlineto{\pgfqpoint{9.796116in}{0.714525in}}%
\pgfpathlineto{\pgfqpoint{9.975847in}{0.714305in}}%
\pgfusepath{stroke}%
\end{pgfscope}%
\begin{pgfscope}%
\pgfpathrectangle{\pgfqpoint{0.728688in}{0.521603in}}{\pgfqpoint{9.687500in}{4.235000in}}%
\pgfusepath{clip}%
\pgfsetbuttcap%
\pgfsetroundjoin%
\pgfsetlinewidth{1.003750pt}%
\definecolor{currentstroke}{rgb}{0.000000,0.000000,0.000000}%
\pgfsetstrokecolor{currentstroke}%
\pgfsetstrokeopacity{0.500000}%
\pgfsetdash{{3.700000pt}{1.600000pt}}{0.000000pt}%
\pgfpathmoveto{\pgfqpoint{1.169029in}{4.564103in}}%
\pgfpathlineto{\pgfqpoint{1.348760in}{4.375359in}}%
\pgfpathlineto{\pgfqpoint{1.528491in}{4.185422in}}%
\pgfpathlineto{\pgfqpoint{1.708222in}{3.983019in}}%
\pgfpathlineto{\pgfqpoint{1.887953in}{3.760243in}}%
\pgfpathlineto{\pgfqpoint{2.067684in}{3.524466in}}%
\pgfpathlineto{\pgfqpoint{2.247415in}{3.349319in}}%
\pgfpathlineto{\pgfqpoint{2.427146in}{3.014918in}}%
\pgfpathlineto{\pgfqpoint{2.606877in}{2.783607in}}%
\pgfpathlineto{\pgfqpoint{2.786608in}{2.489402in}}%
\pgfpathlineto{\pgfqpoint{2.966339in}{2.053102in}}%
\pgfpathlineto{\pgfqpoint{3.146070in}{1.735807in}}%
\pgfpathlineto{\pgfqpoint{3.325801in}{1.436998in}}%
\pgfpathlineto{\pgfqpoint{3.505532in}{1.112781in}}%
\pgfpathlineto{\pgfqpoint{3.685263in}{0.883268in}}%
\pgfpathlineto{\pgfqpoint{3.864994in}{0.821761in}}%
\pgfpathlineto{\pgfqpoint{4.044725in}{0.778708in}}%
\pgfpathlineto{\pgfqpoint{4.224456in}{0.766316in}}%
\pgfpathlineto{\pgfqpoint{4.404187in}{0.756322in}}%
\pgfpathlineto{\pgfqpoint{4.583918in}{0.752370in}}%
\pgfpathlineto{\pgfqpoint{4.763648in}{0.747130in}}%
\pgfpathlineto{\pgfqpoint{4.943379in}{0.741988in}}%
\pgfpathlineto{\pgfqpoint{5.123110in}{0.736495in}}%
\pgfpathlineto{\pgfqpoint{5.302841in}{0.737631in}}%
\pgfpathlineto{\pgfqpoint{5.482572in}{0.731761in}}%
\pgfpathlineto{\pgfqpoint{5.662303in}{0.732099in}}%
\pgfpathlineto{\pgfqpoint{5.842034in}{0.728507in}}%
\pgfpathlineto{\pgfqpoint{6.021765in}{0.728153in}}%
\pgfpathlineto{\pgfqpoint{6.201496in}{0.726730in}}%
\pgfpathlineto{\pgfqpoint{6.381227in}{0.724779in}}%
\pgfpathlineto{\pgfqpoint{6.560958in}{0.723371in}}%
\pgfpathlineto{\pgfqpoint{6.740689in}{0.722704in}}%
\pgfpathlineto{\pgfqpoint{6.920420in}{0.721829in}}%
\pgfpathlineto{\pgfqpoint{7.100151in}{0.721662in}}%
\pgfpathlineto{\pgfqpoint{7.279882in}{0.720446in}}%
\pgfpathlineto{\pgfqpoint{7.459613in}{0.719978in}}%
\pgfpathlineto{\pgfqpoint{7.639344in}{0.719070in}}%
\pgfpathlineto{\pgfqpoint{7.819075in}{0.719093in}}%
\pgfpathlineto{\pgfqpoint{7.998806in}{0.718690in}}%
\pgfpathlineto{\pgfqpoint{8.178537in}{0.717470in}}%
\pgfpathlineto{\pgfqpoint{8.358268in}{0.717466in}}%
\pgfpathlineto{\pgfqpoint{8.537999in}{0.717053in}}%
\pgfpathlineto{\pgfqpoint{8.717730in}{0.716508in}}%
\pgfpathlineto{\pgfqpoint{8.897461in}{0.715971in}}%
\pgfpathlineto{\pgfqpoint{9.077192in}{0.715778in}}%
\pgfpathlineto{\pgfqpoint{9.256923in}{0.715462in}}%
\pgfpathlineto{\pgfqpoint{9.436654in}{0.715133in}}%
\pgfpathlineto{\pgfqpoint{9.616385in}{0.714836in}}%
\pgfpathlineto{\pgfqpoint{9.796116in}{0.714465in}}%
\pgfpathlineto{\pgfqpoint{9.975847in}{0.714263in}}%
\pgfusepath{stroke}%
\end{pgfscope}%
\begin{pgfscope}%
\pgfsetrectcap%
\pgfsetmiterjoin%
\pgfsetlinewidth{0.803000pt}%
\definecolor{currentstroke}{rgb}{0.000000,0.000000,0.000000}%
\pgfsetstrokecolor{currentstroke}%
\pgfsetdash{}{0pt}%
\pgfpathmoveto{\pgfqpoint{0.728688in}{0.521603in}}%
\pgfpathlineto{\pgfqpoint{0.728688in}{4.756603in}}%
\pgfusepath{stroke}%
\end{pgfscope}%
\begin{pgfscope}%
\pgfsetrectcap%
\pgfsetmiterjoin%
\pgfsetlinewidth{0.803000pt}%
\definecolor{currentstroke}{rgb}{0.000000,0.000000,0.000000}%
\pgfsetstrokecolor{currentstroke}%
\pgfsetdash{}{0pt}%
\pgfpathmoveto{\pgfqpoint{10.416188in}{0.521603in}}%
\pgfpathlineto{\pgfqpoint{10.416188in}{4.756603in}}%
\pgfusepath{stroke}%
\end{pgfscope}%
\begin{pgfscope}%
\pgfsetrectcap%
\pgfsetmiterjoin%
\pgfsetlinewidth{0.803000pt}%
\definecolor{currentstroke}{rgb}{0.000000,0.000000,0.000000}%
\pgfsetstrokecolor{currentstroke}%
\pgfsetdash{}{0pt}%
\pgfpathmoveto{\pgfqpoint{0.728688in}{0.521603in}}%
\pgfpathlineto{\pgfqpoint{10.416188in}{0.521603in}}%
\pgfusepath{stroke}%
\end{pgfscope}%
\begin{pgfscope}%
\pgfsetrectcap%
\pgfsetmiterjoin%
\pgfsetlinewidth{0.803000pt}%
\definecolor{currentstroke}{rgb}{0.000000,0.000000,0.000000}%
\pgfsetstrokecolor{currentstroke}%
\pgfsetdash{}{0pt}%
\pgfpathmoveto{\pgfqpoint{0.728688in}{4.756603in}}%
\pgfpathlineto{\pgfqpoint{10.416188in}{4.756603in}}%
\pgfusepath{stroke}%
\end{pgfscope}%
\begin{pgfscope}%
\definecolor{textcolor}{rgb}{0.000000,0.000000,0.000000}%
\pgfsetstrokecolor{textcolor}%
\pgfsetfillcolor{textcolor}%
\pgftext[x=5.572438in,y=4.839937in,,base]{\color{textcolor}\sffamily\fontsize{12.000000}{14.400000}\selectfont \(\displaystyle \overline{\langle I\rangle}\) over \(\displaystyle p_4\) for \(\displaystyle T=1000\) with \(\displaystyle p_1=p_2=p_3=0.5\)}%
\end{pgfscope}%
\begin{pgfscope}%
\pgfsetbuttcap%
\pgfsetmiterjoin%
\definecolor{currentfill}{rgb}{1.000000,1.000000,1.000000}%
\pgfsetfillcolor{currentfill}%
\pgfsetfillopacity{0.800000}%
\pgfsetlinewidth{1.003750pt}%
\definecolor{currentstroke}{rgb}{0.800000,0.800000,0.800000}%
\pgfsetstrokecolor{currentstroke}%
\pgfsetstrokeopacity{0.800000}%
\pgfsetdash{}{0pt}%
\pgfpathmoveto{\pgfqpoint{9.386482in}{3.830063in}}%
\pgfpathlineto{\pgfqpoint{10.318966in}{3.830063in}}%
\pgfpathquadraticcurveto{\pgfqpoint{10.346743in}{3.830063in}}{\pgfqpoint{10.346743in}{3.857841in}}%
\pgfpathlineto{\pgfqpoint{10.346743in}{4.659381in}}%
\pgfpathquadraticcurveto{\pgfqpoint{10.346743in}{4.687159in}}{\pgfqpoint{10.318966in}{4.687159in}}%
\pgfpathlineto{\pgfqpoint{9.386482in}{4.687159in}}%
\pgfpathquadraticcurveto{\pgfqpoint{9.358704in}{4.687159in}}{\pgfqpoint{9.358704in}{4.659381in}}%
\pgfpathlineto{\pgfqpoint{9.358704in}{3.857841in}}%
\pgfpathquadraticcurveto{\pgfqpoint{9.358704in}{3.830063in}}{\pgfqpoint{9.386482in}{3.830063in}}%
\pgfpathlineto{\pgfqpoint{9.386482in}{3.830063in}}%
\pgfpathclose%
\pgfusepath{stroke,fill}%
\end{pgfscope}%
\begin{pgfscope}%
\pgfsetbuttcap%
\pgfsetroundjoin%
\definecolor{currentfill}{rgb}{0.000000,0.000000,1.000000}%
\pgfsetfillcolor{currentfill}%
\pgfsetfillopacity{0.500000}%
\pgfsetlinewidth{1.003750pt}%
\definecolor{currentstroke}{rgb}{0.000000,0.000000,1.000000}%
\pgfsetstrokecolor{currentstroke}%
\pgfsetstrokeopacity{0.500000}%
\pgfsetdash{}{0pt}%
\pgfsys@defobject{currentmarker}{\pgfqpoint{-0.021960in}{-0.021960in}}{\pgfqpoint{0.021960in}{0.021960in}}{%
\pgfpathmoveto{\pgfqpoint{0.000000in}{-0.021960in}}%
\pgfpathcurveto{\pgfqpoint{0.005824in}{-0.021960in}}{\pgfqpoint{0.011410in}{-0.019646in}}{\pgfqpoint{0.015528in}{-0.015528in}}%
\pgfpathcurveto{\pgfqpoint{0.019646in}{-0.011410in}}{\pgfqpoint{0.021960in}{-0.005824in}}{\pgfqpoint{0.021960in}{0.000000in}}%
\pgfpathcurveto{\pgfqpoint{0.021960in}{0.005824in}}{\pgfqpoint{0.019646in}{0.011410in}}{\pgfqpoint{0.015528in}{0.015528in}}%
\pgfpathcurveto{\pgfqpoint{0.011410in}{0.019646in}}{\pgfqpoint{0.005824in}{0.021960in}}{\pgfqpoint{0.000000in}{0.021960in}}%
\pgfpathcurveto{\pgfqpoint{-0.005824in}{0.021960in}}{\pgfqpoint{-0.011410in}{0.019646in}}{\pgfqpoint{-0.015528in}{0.015528in}}%
\pgfpathcurveto{\pgfqpoint{-0.019646in}{0.011410in}}{\pgfqpoint{-0.021960in}{0.005824in}}{\pgfqpoint{-0.021960in}{0.000000in}}%
\pgfpathcurveto{\pgfqpoint{-0.021960in}{-0.005824in}}{\pgfqpoint{-0.019646in}{-0.011410in}}{\pgfqpoint{-0.015528in}{-0.015528in}}%
\pgfpathcurveto{\pgfqpoint{-0.011410in}{-0.019646in}}{\pgfqpoint{-0.005824in}{-0.021960in}}{\pgfqpoint{0.000000in}{-0.021960in}}%
\pgfpathlineto{\pgfqpoint{0.000000in}{-0.021960in}}%
\pgfpathclose%
\pgfusepath{stroke,fill}%
}%
\begin{pgfscope}%
\pgfsys@transformshift{9.553148in}{4.562539in}%
\pgfsys@useobject{currentmarker}{}%
\end{pgfscope}%
\end{pgfscope}%
\begin{pgfscope}%
\definecolor{textcolor}{rgb}{0.000000,0.000000,0.000000}%
\pgfsetstrokecolor{textcolor}%
\pgfsetfillcolor{textcolor}%
\pgftext[x=9.803148in,y=4.526080in,left,base]{\color{textcolor}\sffamily\fontsize{10.000000}{12.000000}\selectfont \(\displaystyle L=16\)}%
\end{pgfscope}%
\begin{pgfscope}%
\pgfsetbuttcap%
\pgfsetroundjoin%
\definecolor{currentfill}{rgb}{0.980392,0.164706,0.333333}%
\pgfsetfillcolor{currentfill}%
\pgfsetfillopacity{0.500000}%
\pgfsetlinewidth{1.003750pt}%
\definecolor{currentstroke}{rgb}{0.980392,0.164706,0.333333}%
\pgfsetstrokecolor{currentstroke}%
\pgfsetstrokeopacity{0.500000}%
\pgfsetdash{}{0pt}%
\pgfsys@defobject{currentmarker}{\pgfqpoint{-0.021960in}{-0.021960in}}{\pgfqpoint{0.021960in}{0.021960in}}{%
\pgfpathmoveto{\pgfqpoint{0.000000in}{-0.021960in}}%
\pgfpathcurveto{\pgfqpoint{0.005824in}{-0.021960in}}{\pgfqpoint{0.011410in}{-0.019646in}}{\pgfqpoint{0.015528in}{-0.015528in}}%
\pgfpathcurveto{\pgfqpoint{0.019646in}{-0.011410in}}{\pgfqpoint{0.021960in}{-0.005824in}}{\pgfqpoint{0.021960in}{0.000000in}}%
\pgfpathcurveto{\pgfqpoint{0.021960in}{0.005824in}}{\pgfqpoint{0.019646in}{0.011410in}}{\pgfqpoint{0.015528in}{0.015528in}}%
\pgfpathcurveto{\pgfqpoint{0.011410in}{0.019646in}}{\pgfqpoint{0.005824in}{0.021960in}}{\pgfqpoint{0.000000in}{0.021960in}}%
\pgfpathcurveto{\pgfqpoint{-0.005824in}{0.021960in}}{\pgfqpoint{-0.011410in}{0.019646in}}{\pgfqpoint{-0.015528in}{0.015528in}}%
\pgfpathcurveto{\pgfqpoint{-0.019646in}{0.011410in}}{\pgfqpoint{-0.021960in}{0.005824in}}{\pgfqpoint{-0.021960in}{0.000000in}}%
\pgfpathcurveto{\pgfqpoint{-0.021960in}{-0.005824in}}{\pgfqpoint{-0.019646in}{-0.011410in}}{\pgfqpoint{-0.015528in}{-0.015528in}}%
\pgfpathcurveto{\pgfqpoint{-0.011410in}{-0.019646in}}{\pgfqpoint{-0.005824in}{-0.021960in}}{\pgfqpoint{0.000000in}{-0.021960in}}%
\pgfpathlineto{\pgfqpoint{0.000000in}{-0.021960in}}%
\pgfpathclose%
\pgfusepath{stroke,fill}%
}%
\begin{pgfscope}%
\pgfsys@transformshift{9.553148in}{4.358681in}%
\pgfsys@useobject{currentmarker}{}%
\end{pgfscope}%
\end{pgfscope}%
\begin{pgfscope}%
\definecolor{textcolor}{rgb}{0.000000,0.000000,0.000000}%
\pgfsetstrokecolor{textcolor}%
\pgfsetfillcolor{textcolor}%
\pgftext[x=9.803148in,y=4.322223in,left,base]{\color{textcolor}\sffamily\fontsize{10.000000}{12.000000}\selectfont \(\displaystyle L=32\)}%
\end{pgfscope}%
\begin{pgfscope}%
\pgfsetbuttcap%
\pgfsetroundjoin%
\definecolor{currentfill}{rgb}{0.239216,0.478431,0.992157}%
\pgfsetfillcolor{currentfill}%
\pgfsetfillopacity{0.500000}%
\pgfsetlinewidth{1.003750pt}%
\definecolor{currentstroke}{rgb}{0.239216,0.478431,0.992157}%
\pgfsetstrokecolor{currentstroke}%
\pgfsetstrokeopacity{0.500000}%
\pgfsetdash{}{0pt}%
\pgfsys@defobject{currentmarker}{\pgfqpoint{-0.021960in}{-0.021960in}}{\pgfqpoint{0.021960in}{0.021960in}}{%
\pgfpathmoveto{\pgfqpoint{0.000000in}{-0.021960in}}%
\pgfpathcurveto{\pgfqpoint{0.005824in}{-0.021960in}}{\pgfqpoint{0.011410in}{-0.019646in}}{\pgfqpoint{0.015528in}{-0.015528in}}%
\pgfpathcurveto{\pgfqpoint{0.019646in}{-0.011410in}}{\pgfqpoint{0.021960in}{-0.005824in}}{\pgfqpoint{0.021960in}{0.000000in}}%
\pgfpathcurveto{\pgfqpoint{0.021960in}{0.005824in}}{\pgfqpoint{0.019646in}{0.011410in}}{\pgfqpoint{0.015528in}{0.015528in}}%
\pgfpathcurveto{\pgfqpoint{0.011410in}{0.019646in}}{\pgfqpoint{0.005824in}{0.021960in}}{\pgfqpoint{0.000000in}{0.021960in}}%
\pgfpathcurveto{\pgfqpoint{-0.005824in}{0.021960in}}{\pgfqpoint{-0.011410in}{0.019646in}}{\pgfqpoint{-0.015528in}{0.015528in}}%
\pgfpathcurveto{\pgfqpoint{-0.019646in}{0.011410in}}{\pgfqpoint{-0.021960in}{0.005824in}}{\pgfqpoint{-0.021960in}{0.000000in}}%
\pgfpathcurveto{\pgfqpoint{-0.021960in}{-0.005824in}}{\pgfqpoint{-0.019646in}{-0.011410in}}{\pgfqpoint{-0.015528in}{-0.015528in}}%
\pgfpathcurveto{\pgfqpoint{-0.011410in}{-0.019646in}}{\pgfqpoint{-0.005824in}{-0.021960in}}{\pgfqpoint{0.000000in}{-0.021960in}}%
\pgfpathlineto{\pgfqpoint{0.000000in}{-0.021960in}}%
\pgfpathclose%
\pgfusepath{stroke,fill}%
}%
\begin{pgfscope}%
\pgfsys@transformshift{9.553148in}{4.154824in}%
\pgfsys@useobject{currentmarker}{}%
\end{pgfscope}%
\end{pgfscope}%
\begin{pgfscope}%
\definecolor{textcolor}{rgb}{0.000000,0.000000,0.000000}%
\pgfsetstrokecolor{textcolor}%
\pgfsetfillcolor{textcolor}%
\pgftext[x=9.803148in,y=4.118366in,left,base]{\color{textcolor}\sffamily\fontsize{10.000000}{12.000000}\selectfont \(\displaystyle L=64\)}%
\end{pgfscope}%
\begin{pgfscope}%
\pgfsetbuttcap%
\pgfsetroundjoin%
\definecolor{currentfill}{rgb}{0.000000,0.000000,0.000000}%
\pgfsetfillcolor{currentfill}%
\pgfsetfillopacity{0.500000}%
\pgfsetlinewidth{1.003750pt}%
\definecolor{currentstroke}{rgb}{0.000000,0.000000,0.000000}%
\pgfsetstrokecolor{currentstroke}%
\pgfsetstrokeopacity{0.500000}%
\pgfsetdash{}{0pt}%
\pgfsys@defobject{currentmarker}{\pgfqpoint{-0.021960in}{-0.021960in}}{\pgfqpoint{0.021960in}{0.021960in}}{%
\pgfpathmoveto{\pgfqpoint{0.000000in}{-0.021960in}}%
\pgfpathcurveto{\pgfqpoint{0.005824in}{-0.021960in}}{\pgfqpoint{0.011410in}{-0.019646in}}{\pgfqpoint{0.015528in}{-0.015528in}}%
\pgfpathcurveto{\pgfqpoint{0.019646in}{-0.011410in}}{\pgfqpoint{0.021960in}{-0.005824in}}{\pgfqpoint{0.021960in}{0.000000in}}%
\pgfpathcurveto{\pgfqpoint{0.021960in}{0.005824in}}{\pgfqpoint{0.019646in}{0.011410in}}{\pgfqpoint{0.015528in}{0.015528in}}%
\pgfpathcurveto{\pgfqpoint{0.011410in}{0.019646in}}{\pgfqpoint{0.005824in}{0.021960in}}{\pgfqpoint{0.000000in}{0.021960in}}%
\pgfpathcurveto{\pgfqpoint{-0.005824in}{0.021960in}}{\pgfqpoint{-0.011410in}{0.019646in}}{\pgfqpoint{-0.015528in}{0.015528in}}%
\pgfpathcurveto{\pgfqpoint{-0.019646in}{0.011410in}}{\pgfqpoint{-0.021960in}{0.005824in}}{\pgfqpoint{-0.021960in}{0.000000in}}%
\pgfpathcurveto{\pgfqpoint{-0.021960in}{-0.005824in}}{\pgfqpoint{-0.019646in}{-0.011410in}}{\pgfqpoint{-0.015528in}{-0.015528in}}%
\pgfpathcurveto{\pgfqpoint{-0.011410in}{-0.019646in}}{\pgfqpoint{-0.005824in}{-0.021960in}}{\pgfqpoint{0.000000in}{-0.021960in}}%
\pgfpathlineto{\pgfqpoint{0.000000in}{-0.021960in}}%
\pgfpathclose%
\pgfusepath{stroke,fill}%
}%
\begin{pgfscope}%
\pgfsys@transformshift{9.553148in}{3.950967in}%
\pgfsys@useobject{currentmarker}{}%
\end{pgfscope}%
\end{pgfscope}%
\begin{pgfscope}%
\definecolor{textcolor}{rgb}{0.000000,0.000000,0.000000}%
\pgfsetstrokecolor{textcolor}%
\pgfsetfillcolor{textcolor}%
\pgftext[x=9.803148in,y=3.914509in,left,base]{\color{textcolor}\sffamily\fontsize{10.000000}{12.000000}\selectfont \(\displaystyle L=128\)}%
\end{pgfscope}%
\end{pgfpicture}%
\makeatother%
\endgroup%
}
    \caption{Graphic}\label{fig:Res_Dis_Avg_Inf_over_p4}
\end{figure}

\subsection{Time Development of the Expected Ratio of Infected People}

While previously the average of the ratio of infected individuals has been taken over time, the focus should now be layed on the time development of th einfection rate $\langle I\rangle_t$ for $N=20$ samples.
As grid size $L=64$ was chosen for appropriate balance between running time and accuracy, the number of simulation steps again was set to $T=1000$.